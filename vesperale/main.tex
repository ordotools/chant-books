\documentclass[]{book}

\usepackage{longtable}
\usepackage{booktabs}

\title{VESPERALE}
\date{}

% \usepackage{chantbook}


\begin{document}

\maketitle

\chapter{Kalendarium perpetuum.}

\begin{longtable}{l|l|r|p{.8\linewidth}}

	\hline
	\multicolumn{4}{c}{\textbf{Januarius}} \\
	\hline
	\endhead

	A & Kal.  & 1  & Circumcisio Domini. \textit{Duplex II. classis.} \\
	b & iv    & 2  & Octava S. Stephani. \textit{Duplex. Commemoratio} Octavarum. \\
	c & iii   & 3  & Octava S. Joannis Apost. et Evang. \textit{Duplex. Comm. }Octav\ae\ SS. Innocentium. \\
	d & Prid. & 4  & Octava SS. Innocentium Martyrum. \textit{Duplex.} \\
	e & Non.  & 5  & Vigilia Epiphani\ae. \textit{Semiduplex. Comm.} S. Telesphori Pap\ae, Martyris. \\
	f & viii  & 6  & Epiphania Domini. \textit{Duplex I. classis cum Octava.} \\
	g & vii   & 7  & De Octava Epiphani\ae. \textit{Semiduplex.} \\
	A & vi    & 8  & De Octava. \textit{Semiduplex.} \\
	b & v     & 9  & De Octava. \\
	c & iv    & 10 & De Octava. \\
	d & iii   & 11 & De Octava. \textit{Comm.} S. Hygini Pap\ae, Martyris. \\

	
\end{longtable}

	
\end{document}

