\section*{In 3\textsuperscript{rd} Nocturn}

    \buildpsalm{an--dixi_iniquis--solesmes_1961}{74}{7c}

    \buildpsalm{an--terra_tremuit--solesmes_1961}{75}{8c}
    
    \buildpsalm{an--in_die_tribulationis--solesmes_1961.gabc}{76}{7a}

    \versicle{
	    Ex(h)súr(i)ge,(h) Dó(g)mi(g)ne.(g.) (::)
    }{
	    Et(h) jú(h)di(h)ca(h) cáu(i)sam(h) mé(g)am.(g.) (::)
    }

    \rubric{\black{Pater noster} in silence.}

    \redtitle{Lesson 7.}

    De Epístola prima beáti Pauli Apóstoli ad Corínthios

    \rubric{1 Cor 11:17-22}

    Hoc autem præcípio: non laudans quod non in mélius, sed in detérius
    convenítis. Primum quidem conveniéntibus vobis in Ecclésiam, áudio
    scissúras esse inter vos, et ex parte credo. Nam opórtet et hǽreses esse,
    ut et qui probáti sunt, manifésti fiant in vobis. Conveniéntibus ergo vobis
    in unum, jam non est Domínicam cenam manducáre. Unusquísque enim suam cenam
    præsúmit ad manducándum. Et álius quidem ésurit, álius autem ébrius est.
    Numquid domos non habétis ad manducándum et bibéndum? aut Ecclésiam Dei
    contémnitis, et confúnditis eos, qui non habent? Quid dicam vobis? Laudo
    vos? In hoc non laudo.

    \responsory{re--in_monte_oliveti--solesmes_1961}{7}

    \redtitle{Lesson 8.}

    \rubric{1 Cor 11:23-26}

    Ego enim accépi a Dómino quod et trádidi vobis, quóniam Dóminus Jesus, in qua
    nocte tradebátur, accépit panem, Et grátias agens fregit, et dixit: Accípite,
    et manducáte: hoc est corpus meum, quod pro vobis tradétur: hoc fácite in meam
    commemoratiónem. Simíliter et cálicem, postquam cœnávit, dicens: Hic calix
    novum testaméntum est in meo sánguine: hoc fácite, quotiescúmque bibétis, in
    meam commemoratiónem. Quotiescúmque enim manducábitis panem hunc, et cálicem
    bibétis, mortem Dómini annuntiábitis donec véniat.

    \responsory{re--una_hora--solesmes_1961.1}{7}

    \redtitle{Lesson 9.}

    \rubric{1 Cor 11:27-34}

    Itaque quicúmque manducáverit panem hunc, vel bíberit cálicem Dómini indígne,
    reus erit córporis et sánguinis Dómini. Probet autem seípsum homo: et sic de
    pane illo edat, et de cálice bibat. Qui enim mandúcat et bibit indígne,
    judícium sibi mandúcat et bibit, non dijúdicans corpus Dómini. Ideo inter vos
    multi infírmi et imbecílles, et dórmiunt multi. Quod, si nosmetípsos
    dijudicarémus, non útique judicarémur. Dum judicámur autem, a Dómino
    corrípimur, ut non cum hoc mundo damnémur. Itaque, fratres mei, cum convenítis
    ad manducándum, ínvicem exspectáte. Si quis ésurit, domi mandúcet: ut non in
    judícium conveniátis. Cétera autem, cum vénero, dispónam.

    \responsory{re--seniores--solesmes_1961}{1}
