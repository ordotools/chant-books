\documentclass[statementpaper,10pt,openright,openbib]{memoir}

\usepackage{liturgicalbooklet}

\begin{document}

\chapter*{Ordinarium Divini Officii ad Primam}

Pater. Ave. Credo.

\greannotation{\color{red}\vbar}
\gabcsnippet{(c3) De(h)us(h) <c><v>\grecross</v></c>(,) in(h) ad(h)ju(h)tó(i)ri(h)um(h) me(h)um(h) in(h)tén(g)de.(h.) (::)
<c><sp>R/</sp>.</c> Dó(h)mi(h)ne,(h) ad(h) ad(h)ju(h)ván(h)dum(h) me(h) fe(h)stí(g)na.(h) (:)
Gló(h)ri(h)a(h) Pa(h)tri,(h) et(h) Fí(h)li(h)o,(h) (,) et(h) Spi(h)rí(h)tu(h)i(h) San(g)cto.(h) (:)
Si(h)cut(h) er(h)at(h) in(h) prin(h)cí(h)pi(h)o,(h) et(h) nunc,(h) et(h) sem(h)per,(h) (,) et(h) in(h) sǽ(h)cu(h)la(h) sæ(h)cu(h)ló(h)rum.(h) A(g)men.(h)
(::) Al(h)le(i)lú(hg~)ja.(g) (::) <c><i>Vel:</i></c>() Laus(h) ti(h)bi,(h) Dó(h)mi(h)ne,(h) Rex(h) æ(h)tér(h)næ(i) gló(h)ri(h)æ.(g) (::)}

\section*{Hymnus}

\smallskip
1. Tonus in Dominicis per Annum et minoribus Festis.
\nopagebreak
\gregorioscore{jamLucis_dominicis.gabc}

\medskip
2. Tonus in Feriis per Annum et Festis Simplicibus.
\nopagebreak
\gregorioscore{hy--jam_lucis_orto_sidere_(feriis_et_officiis_simplicibus_per_annum)--solesmes_1934.gabc}

\medskip
3. Tonus in majoribus Festis.
\nopagebreak
\gregorioscore{hy--jam_lucis_orto_sidere_(solemnitatibus)--solesmes_1934}

\medskip

\rubric{Alii toni pro variis Temporibus et Festis habentur propriis locis.\\

Conclusio communis in Humno pr\ae cedenti, et in aliis idem metrum
habentibus, semper omittitur, quando specialis in omnibus Horis adhibenda pr\ae
scribitur; et, si plures Conclusiones propri\ae\ occurrant, sumitur Conclusio
Officii currentis, aut secus Officii ipsa die et primo quedem loco inter cetera
propriam conclusionem habentia commemorandi, aut demum de occurenti Octava
communi.\\

Expleto Hymo, conveniens dicitur Antiphona usque ad Asteriscum (*), prouti occurrent Officium requirit.\\

}

\section*{Per Annum}

In Officio Dominicali:\\
\rubric{In Dominics per Annum minoribus, Antiphona ut in Psalterio.}

In Officio feriali:\\
\rubric{In omnibus per Annum Feriis, Antiphona de Feria currenti, ut in Psalterio.}

\medskip
\hrule
\medskip

\rubric{
	Post enuntiatam Antiphonam dicuntur tres Psalmi Officio diei congruentes.\\
	Quando autem adhibetur posterius Laudum Schema, Psalmus, in priori Ludeum ferialium schemate omissus, ad Primam Feri\ae\ currentis post alios Psalmos resumitur.\\
	Repetita post ultimum Psalmum integre Antiphona, conveniens dicitur Capitulum; nimirum:
}

\smallskip

\rubric{
	In omnibus Dominicis, etiam antidipatis, in Officio cujuslibet Festi vel Octav\ae, ac sanct\ae\ Mari\ae\ in Sabbato:
}

\capitulum{1. Tim. 1:17.}{Regi s\ae cul\'orum imort\'ali et invis\'ibili, \dagger\ soli Deo ho\textit{nor et} \textbf{gl\'o}ria * in s\ae cula s\ae cul\'orum \textbf{A}men.}\nopagebreak

\rbar Deo \textbf{gr\'a}tias.

\smallskip

\rubric{In omnibus autem Feriis et Vigiliis communibus:}

\capitulum{Zach. 8:19}{Pacem et verit\'a\textit{tem di}\textbf{l\'i}gite, * ait D\'ominus om\textbf{n\'i}potens.}\nopagebreak

\rbar Deo \textbf{gr\'a}tias.

\medskip

Resp. breve per Annum.\nopagebreak

\gregorioscore{rb--christe_fili_dei_vivi--vatican.gabc}





\end{document}
