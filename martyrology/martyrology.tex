
\documentclass[11pt,twoside, openright]{memoir}

% ------------ Running titles control ------------
% Defaults (change here or with the command below):
\newcommand{\latinrunning}{Latin Title}
\newcommand{\englishrunning}{English Title}

% \setrunningtitles{<Latin>}{<English>} — call this anywhere to switch
\newcommand{\setrunningtitles}[2]{%
  \gdef\englishrunning{#1}% % TODO: see if this is the way that we need to go.
  \gdef\latinrunning{#1}%
}


\usepackage{fancyhdr}
\fancyhf{}
\pagestyle{fancy}

% % ------------ Page style: numbers outside, titles centered ------------
% \makeevenfoot{dualheads}{}{}{} % empty footers
%
% % Optional spacing tweaks
% \setlength{\headsep}{18pt}
\checkandfixthelayout % memoir convenience


% --- Engines & fonts (compile with XeLaTeX or LuaLaTeX) ---
\usepackage{fontspec}
\usepackage{polyglossia}
\setmainlanguage{english}
\setotherlanguage{latin}

\newif\ifminion
\IfFontExistsTF{Minion 3}{\miniontrue}{\minionfalse}
\ifminion
  \setmainfont{Minion 3}
  \newfontfamily\latinfont{Minion 3}
\else
  \setmainfont{Libertinus Serif}
  \newfontfamily\latinfont{Libertinus Serif}
\fi

% --- Gregorio & colors ---
\usepackage{gregoriotex}
\usepackage{xcolor}
\definecolor{gregoriocolor}{HTML}{C6171C} % for both text and (if used) gregorio

\newcommand{\rubric}[1]{%
	{\color{gregoriocolor}\small #1}%
}

% Response symbol convenience
\providecommand{\rbar}{\Rbar}

\fancyhead[LE]{\thepage \hfill \latinrunning}
\fancyhead[RO]{\englishrunning \hfill \thepage}

% Colored rule under the header
\renewcommand{\headrulewidth}{0.5pt}
\renewcommand{\headrule}{%
  {\color{gregoriocolor}\hrule height 0.5pt width\headwidth \vskip-\headrulewidth}
}

% --- Layout & tables ---
\usepackage{paracol}
\usepackage{geometry}
\usepackage{tabularx}
\usepackage{array}
\usepackage{booktabs}
\usepackage{longtable}

% --- Typography / hyphenation ---
\usepackage{microtype}
\usepackage{needspace}
\geometry{margin=1in}
\setlength{\parskip}{0.6em}
\setlength{\parindent}{0pt}
\emergencystretch=2em

% --- Drop caps ---
\usepackage{lettrine}
\renewcommand{\LettrineFontHook}{\color{gregoriocolor}}

% % --- Memoir pagestyle: page number at the OUTER edge; months in headers ---
% \makepagestyle{marty}
% % Even (left) pages: outer = left. Show page no. at left, Latin month after it.
% \makeevenhead{marty}{\thepage\qquad\textsc{\leftmark}}{}{}
% % Odd (right) pages: outer = right. Show English month before page no. at right.
% \makeoddhead{marty}{}{}{\textsc{\rightmark}\qquad\thepage}
% \makeevenfoot{marty}{}{}{}
% \makeoddfoot{marty}{}{}{}
% \pagestyle{marty}

% \title{Roman Martyrology}
% \author{}
% \date{}

\fancyfoot{}

\begin{document}

% ------------ Title page (no numbering), then blank page ------------
\pagenumbering{gobble}          % suppress numbers completely
\thispagestyle{empty}
\begin{center}
  \vspace*{0.18\textheight}
  {\Huge\bfseries Roman Martyrology\par}
  \vspace{1ex}
  {\Large\itshape In Latin and English.\par}
  \vspace{2em}
  \rule{0.5\textwidth}{0.4pt}\par
  \vspace{1em}
  % {\large Your Name\par}
  \vfill
  % {\small Roman Catholic Institute \the\year\par}
\end{center}
\clearpage
\thispagestyle{empty}\null\clearpage % blank verso, also unnumbered


% ------------ Start main matter & headers ------------
\pagenumbering{arabic} % this sets the numbering for the bottom.
% \pagestyle{dualheads}

% Set initial running titles for left/right pages:
\setrunningtitles{Januarius}{January}


% ---- martyrology/mart01/mart0101.htm
\needspace{10\baselineskip}
\begin{paracol}{2}
\selectlanguage{latin}
\begin{center}{\color{gregoriocolor} Kaléndis Januárii. Luna\dots\ }\end{center}
\switchcolumn
\selectlanguage{english}
\begin{center}{\color{gregoriocolor} The   First Day of 
 January. The\dots\ Day of the Moon.}\end{center}
\end{paracol}

% TODO: find out a way we can distribute the letter-number pairs evenly.
\noindent\begin{tabularx}{\linewidth}{*{19}{>{\centering\arraybackslash}X}}
 \textcolor{gregoriocolor}{a} & \textcolor{gregoriocolor}{b} & \textcolor{gregoriocolor}{c} & \textcolor{gregoriocolor}{d} & \textcolor{gregoriocolor}{e} & \textcolor{gregoriocolor}{f} & \textcolor{gregoriocolor}{g} & \textcolor{gregoriocolor}{h} & \textcolor{gregoriocolor}{i} & \textcolor{gregoriocolor}{k} & \textcolor{gregoriocolor}{l} & \textcolor{gregoriocolor}{m} & \textcolor{gregoriocolor}{n} & \textcolor{gregoriocolor}{p} & \textcolor{gregoriocolor}{q} & \textcolor{gregoriocolor}{r} & \textcolor{gregoriocolor}{s} & \textcolor{gregoriocolor}{t} & \textcolor{gregoriocolor}{u} \\
 2 & 3 & 4 & 5 & 6 & 7 & 8 & 9 & 10 & 11 & 12 & 13 & 14 & 15 & 16 & 17 & 18 & 19 & 20 \\
\end{tabularx}
\vspace{0.5\baselineskip}
\noindent\begin{tabularx}{\linewidth}{*{12}{>{\centering\arraybackslash}X}}
 \textcolor{gregoriocolor}{A} & \textcolor{gregoriocolor}{B} & \textcolor{gregoriocolor}{C} & \textcolor{gregoriocolor}{D} & \textcolor{gregoriocolor}{E} & F & \textcolor{gregoriocolor}{F} & \textcolor{gregoriocolor}{G} & \textcolor{gregoriocolor}{H} & \textcolor{gregoriocolor}{M} & \textcolor{gregoriocolor}{N} & \textcolor{gregoriocolor}{P} \\
 21 & 22 & 23 & 24 & 25 & 26 & 26 & 27 & 28 & 29 & 30 & 1 \\
\end{tabularx}

\begin{paracol}{2}
\selectlanguage{latin}
\lettrine[lines=2]{C}{ircumcísio} Dómini nostri Jesu Christi, et Octáva Nativitátis ejúsdem.
\switchcolumn
\selectlanguage{english}
\lettrine[lines=2]{T}{he} Circumcision of our Lord Jesus Christ, and the octave of his Nativity.
\switchcolumn*
\selectlanguage{latin}
Romæ pássio sanctæ Martínæ, Vírginis et 
 Mártyris; quæ, sub Alexándro Imperatóre, divérsis tormentórum genéribus 
 cruciáta, tandem, gládio percússa, martyrii palmam adépta est. Ipsíus 
 vero festum tértio Kaléndas Februárii recólitur.
\switchcolumn
\selectlanguage{english}
At Rome, under Emperor Alexander, St. Martina, virgin, who endured various 
 kinds of torments, and being beheaded, received the palm of martyrdom. 
 Her feast is kept on the 30th of this month.
\switchcolumn*
\selectlanguage{latin}
Cæsaréæ, in Cappadócia, deposítio sancti Basilíi, cognoménto Magni, Epíscopi, Confessóris et Ecclésiæ Doctóris; qui, témpore Valéntis Imperatóris, doctrína et sapiéntia insignítus omnibúsque virtútibus exornátus, mirabíliter effúlsit, et Ecclésiam advérsus Ariános et Macedoniános inexpugnábili constántia deféndit. Ejus autem festívitas potíssimum ágitur décimo octávo Kaléndas Júlii, quo die Epíscopus ordinátus est.
\switchcolumn
\selectlanguage{english}
At Caesarea in Cappadocia, the death of St. Basil the Great, bishop, 
 confessor, and doctor of the Church, renowned for his learning and wisdom 
 and gifted with every virtue, who during the reign of Emperor Valens 
 wonderfully displayed his talents as he defended the Church with great 
 constancy against the Arians and Macedonians. His feast, however, is 
 appropriately kept on the 14th of June, the day on which he was consecrated 
 bishop.
\switchcolumn*
\selectlanguage{latin}
Apud montem Senárium, in Etrúria, natális sancti Bonfílii Confessóris, e 
 septem Fundatóribus Ordinis Servórum beátæ 
 Maríæ Vírginis, quam cum diem impénse coluísset, ab ipsa in cælum repénte 
 evocátus est. Illíus porro ac Sociórum festum prídie Idus Februárii 
 celebrátur.
\switchcolumn
\selectlanguage{english}
In Tuscany, on Mount Senario, St. Bonfilius, confessor, one of the seven 
 founders of the Order of the Servites of the Blessed Virgin Mary, who, 
 having honoured her devoutly, was suddenly called to heaven by her. 
 His feast, with that of his companions, is kept on February 12th.
\switchcolumn*
\selectlanguage{latin}
Romæ sancti Almáchii Mártyris, qui, cum díceret: 
 « Hódie Octávæ Domínici diéi sunt, cessáte a superstitiónibus idolórum et a 
 sacrifíciis pollútis », proptérea, jubénte Præfécto Urbis Alípio, a 
 gladiatóribus occísus est.
\switchcolumn
\selectlanguage{english}
At Rome, St. Almachius, martyr, who, by the command of Alipius, governor of 
 the city, was killed by the gladiators for saying, ``Today is the Octave of 
 our Lord's birth; put an end to the worship of idols, and abstain from 
 unclean sacrifices.''
\switchcolumn*
\selectlanguage{latin}
Item Romæ, via Appia, corónæ sanctórum mílitum 
 trigínta Mártyrum, sub Diocletiáno Imperatóre.
\switchcolumn
\selectlanguage{english}
In the same city, on the Appian Way, the crowning with martyrdom of thirty 
 holy soldiers under Emperor Diocletian.
\switchcolumn*
\selectlanguage{latin}
Apud Spolétum sancti Concórdii, Presbyteri et Mártyris; qui, tempóribus 
 Antoníni Imperatóris, primo cæsus fústibus, 
 dehinc equúleo suspénsus, ac póstea macerátus in cárcere, ibíque Angélica 
 visitatióne confortátus, demum gládio vitam finívit.
\switchcolumn
\selectlanguage{english}
At Spoleto, in the time of Emperor Antoninus, St. Concordius, priest and 
 martyr, who was beaten with clubs, then stretched on the rack, and after a 
 long confinement in prison, where he was visited by an angel, lost his life 
 by the sword.
\switchcolumn*
\selectlanguage{latin}
Eódem die sancti Magni Mártyris.
\switchcolumn
\selectlanguage{english}
The same day, St. Magnus, martyr.
\switchcolumn*
\selectlanguage{latin}
In Africa beáti Fulgéntii, Ruspénsis Ecclésiæ 
 Epíscopi, qui, témpore Wandálicæ persecutiónis, ob cathólicam fidem 
 eximiámque doctrínam, ab Ariánis multa perpéssus et in Sardíniam relegátus 
 est; atque tandem, ad própriam Ecclésiam redíre permíssus, vita et verbo 
 clarus, sancto fine quiévit.
\switchcolumn
\selectlanguage{english}
In Africa, St. Fulgentius, bishop of Rusp, who suffered much from the 
 Arians, during the persecution of the Vandals, for holding the Catholic 
 faith and teaching an excellent doctrine. After being banished to 
 Sardinia, he was permitted to return to his diocese, where he ended his life 
 by a holy death, leaving a reputation for sanctity and eloquence.
\switchcolumn*
\selectlanguage{latin}
Teáte, in Aprútio citerióre, natális sancti Justíni, ejúsdem civitátis 
 Epíscopi, sanctitáte vitæ ac miráculis clari.
\switchcolumn
\selectlanguage{english}
At Chieti in Abruzzo, the birthday of St. Justin, bishop of that city, 
 illustrious for holiness of life and for his miracles.
\switchcolumn*
\selectlanguage{latin}
In território Lugdunénsi, monastério Jurénsium, sancti Eugéndi Abbátis, 
 cujus vita virtútibus et miráculis plena refúlsit.
\switchcolumn
\selectlanguage{english}
In the diocese of Lyons, in the monastery of St. Claude, St. Eugendus, 
 abbot, whose life was eminent for virtues and miracles.
\switchcolumn*
\selectlanguage{latin}
Apud Silviníacum, in Gállia, sancti Odilónis, 
 Abbátis Cluniacénsis, qui primus Commemoratiónem ómnium Fidélium Defunctórum, 
 prima die post festum ómnium Sanctórum, in suis monastériis fíeri præcépit; 
 quem ritum póstea universális Ecclésia recípiens comprobávit.
\switchcolumn
\selectlanguage{english}
At Souvigny in France, St. Odilo, abbot of Cluny, who was the first to 
 prescribe that the commemoration of all the faithful departed should be made 
 in his monasteries the day after the feast of All Saints. This 
 practice was afterwards received and approved by the universal Church.
\switchcolumn*
\selectlanguage{latin}
Romæ natális sancti Vincéntii Maríæ Strambi, 
 Epíscopi Maceraténsis et Tolentíni, Congregatiónis a Cruce et Passióne Jesu 
 Sodális, pastoráli zelo præclári, quem Pius Papa Duodécimus inter Sanctos 
 rétulit.
\switchcolumn
\selectlanguage{english}
At Rome, the birthday of St. Vincent Maria Strambi, Bishop of Macerata and 
 Tolentino, of the Order of Passionists, renowned for his pastoral zeal, whom 
 Pope Pius XII numbered among the saints.
\switchcolumn*
\selectlanguage{latin}
Alexandríæ deposítio sanctæ Euphrósynæ Vírginis, 
 quæ in monastério virtúte abstinéntiæ ac miráculis cláruit.
\switchcolumn
\selectlanguage{english}
At Alexandria, the departure from this world of St. Euphrosyna, virgin, who 
 was renowned in her monastery for the virtue of abstinence, and for the gift 
 of miracles.
\switchcolumn*
\selectlanguage{latin}
Et álibi aliórum 
 plurimórum sanctórum Mártyrum et Confessórum, atque sanctárum Vírginum.

\textcolor{gregoriocolor}{\Rbar.}~Deo grátias.
\switchcolumn
\selectlanguage{english}
And elsewhere in divers 
 places, many other holy martyrs, confessors, and holy virgins.

\textcolor{gregoriocolor}{\Rbar.}~Thanks be to God.
\switchcolumn*
\selectlanguage{latin}
\end{paracol}


% ---- martyrology/mart01/mart0102.htm
\needspace{10\baselineskip}
\begin{paracol}{2}
\selectlanguage{latin}
\begin{center}{\color{gregoriocolor} Quarto Nonas Januárii. 
 Luna\dots\ }\end{center}
\switchcolumn
\selectlanguage{english}
\begin{center}{\color{gregoriocolor} The   Second Day of 
 January. The\dots\ Day of the Moon.}\end{center}
\end{paracol}

\noindent\begin{tabularx}{\linewidth}{*{19}{>{\centering\arraybackslash}X}}
 \textcolor{gregoriocolor}{a} & \textcolor{gregoriocolor}{b} & \textcolor{gregoriocolor}{c} & \textcolor{gregoriocolor}{d} & \textcolor{gregoriocolor}{e} & \textcolor{gregoriocolor}{f} & \textcolor{gregoriocolor}{g} & \textcolor{gregoriocolor}{h} & \textcolor{gregoriocolor}{i} & \textcolor{gregoriocolor}{k} & \textcolor{gregoriocolor}{l} & \textcolor{gregoriocolor}{m} & \textcolor{gregoriocolor}{n} & \textcolor{gregoriocolor}{p} & \textcolor{gregoriocolor}{q} & \textcolor{gregoriocolor}{r} & \textcolor{gregoriocolor}{s} & \textcolor{gregoriocolor}{t} & \textcolor{gregoriocolor}{u} \\
 3 & 4 & 5 & 6 & 7 & 8 & 9 & 10 & 11 & 12 & 13 & 14 & 15 & 16 & 17 & 18 & 19 & 20 & 21 \\
\end{tabularx}
\vspace{0.5\baselineskip}
\noindent\begin{tabularx}{\linewidth}{*{12}{>{\centering\arraybackslash}X}}
 \textcolor{gregoriocolor}{A} & \textcolor{gregoriocolor}{B} & \textcolor{gregoriocolor}{C} & \textcolor{gregoriocolor}{D} & \textcolor{gregoriocolor}{E} & F & \textcolor{gregoriocolor}{F} & \textcolor{gregoriocolor}{G} & \textcolor{gregoriocolor}{H} & \textcolor{gregoriocolor}{M} & \textcolor{gregoriocolor}{N} & \textcolor{gregoriocolor}{P} \\
 22 & 23 & 24 & 25 & 26 & 27 & 27 & 28 & 29 & 30 & 1 & 2 \\
\end{tabularx}

\begin{paracol}{2}
\selectlanguage{latin}
\lettrine[lines=1]{O}{ctáva} sancti Stéphani Protomártyris.
\switchcolumn
\selectlanguage{english}
\lettrine[lines=1]{T}{he} Octave of St. Stephen, the first martyr.
\switchcolumn*
\selectlanguage{latin}
Romæ commemorátio plurimórum sanctórum Mártyrum, 
 qui, spreto Diocletiáni Imperatóris edícto quo tradi sacri Códices 
 jubebántur, pótius córpora carnifícibus quam sancta dare cánibus maluérunt.
\switchcolumn
\selectlanguage{english}
At Rome, the commemoration of many holy martyrs, who, despising the edict of 
 Emperor Diocletian, which ordered that the sacred books should be delivered 
 up, preferred to offer their bodies to the executioners rather than to give 
 holy things to dogs.
\switchcolumn*
\selectlanguage{latin}
Antiochíæ pássio beáti Isidóri Epíscopi.
\switchcolumn
\selectlanguage{english}
At Antioch, the passion of blessed Isidore, bishop.
\switchcolumn*
\selectlanguage{latin}
Tomis, in Ponto, sanctórum fratrum Argéi, 
 Narcíssi et Marcellíni púeri. Hic, sub Licínio Príncipe, cum inter 
 tirónes esset comprehénsus et nollet militáre, hinc, cæsus ad mortem ac diu 
 macerátus in cárcere, demum, in mare demérsus, martyrium consummávit; ejus 
 autem fratres gládio perémpti sunt.
\switchcolumn
\selectlanguage{english}
At Tomis in Pontus, in the time of Emperor Licinius, three holy brothers, 
 Argeus, Narcissus, and the young man Marcellinus. This last, being 
 enrolled among the new soldiers, and refusing to serve, was beaten almost to 
 death, and for a long while kept in prison. Being finally cast into 
 the sea, he finished his martyrdom, and his brothers were beheaded.
\switchcolumn*
\selectlanguage{latin}
Medioláni sancti Martiniáni Epíscopi.
\switchcolumn
\selectlanguage{english}
At Milan, St. Martinian, bishop.
\switchcolumn*
\selectlanguage{latin}
Nítriæ, in Ægypto, beáti Isidóri, Epíscopi et 
 Confessóris.
\switchcolumn
\selectlanguage{english}
In Nitria in Egypt, blessed Isidore, bishop and confessor.
\switchcolumn*
\selectlanguage{latin}
Ipso die sancti Siridiónis Epíscopi.
\switchcolumn
\selectlanguage{english}
The same day, St. Siridion, bishop.
\switchcolumn*
\selectlanguage{latin}
In Thebáide sancti Macárii Alexandríni, Presbyteri et Abbátis.
\switchcolumn
\selectlanguage{english}
In Thebais, St. Macarius of Alexandria, abbot.
\switchcolumn*
\selectlanguage{latin}
\end{paracol}


% ---- martyrology/mart01/mart0103.htm
\needspace{10\baselineskip}
\begin{paracol}{2}
\selectlanguage{latin}
\begin{center}{\color{gregoriocolor} Tértio Nonas Januárii. 
 Luna\dots\ }\end{center}
\switchcolumn
\selectlanguage{english}
\begin{center}{\color{gregoriocolor} The   Third Day of 
 January. The\dots\ Day of the Moon.}\end{center}
\end{paracol}

\noindent\begin{tabularx}{\linewidth}{*{19}{>{\centering\arraybackslash}X}}
 \textcolor{gregoriocolor}{a} & \textcolor{gregoriocolor}{b} & \textcolor{gregoriocolor}{c} & \textcolor{gregoriocolor}{d} & \textcolor{gregoriocolor}{e} & \textcolor{gregoriocolor}{f} & \textcolor{gregoriocolor}{g} & \textcolor{gregoriocolor}{h} & \textcolor{gregoriocolor}{i} & \textcolor{gregoriocolor}{k} & \textcolor{gregoriocolor}{l} & \textcolor{gregoriocolor}{m} & \textcolor{gregoriocolor}{n} & \textcolor{gregoriocolor}{p} & \textcolor{gregoriocolor}{q} & \textcolor{gregoriocolor}{r} & \textcolor{gregoriocolor}{s} & \textcolor{gregoriocolor}{t} & \textcolor{gregoriocolor}{u} \\
 4 & 5 & 6 & 7 & 8 & 9 & 10 & 11 & 12 & 13 & 14 & 15 & 16 & 17 & 18 & 19 & 20 & 21 & 22 \\
\end{tabularx}
\vspace{0.5\baselineskip}
\noindent\begin{tabularx}{\linewidth}{*{12}{>{\centering\arraybackslash}X}}
 \textcolor{gregoriocolor}{A} & \textcolor{gregoriocolor}{B} & \textcolor{gregoriocolor}{C} & \textcolor{gregoriocolor}{D} & \textcolor{gregoriocolor}{E} & F & \textcolor{gregoriocolor}{F} & \textcolor{gregoriocolor}{G} & \textcolor{gregoriocolor}{H} & \textcolor{gregoriocolor}{M} & \textcolor{gregoriocolor}{N} & \textcolor{gregoriocolor}{P} \\
 23 & 24 & 25 & 26 & 27 & 28 & 28 & 29 & 30 & 1 & 2 & 3 \\
\end{tabularx}

\begin{paracol}{2}
\selectlanguage{latin}
\lettrine[lines=1]{O}{ctava} sancti Joánnis, Apóstoli et Evangelístæ.
\switchcolumn
\selectlanguage{english}
\lettrine[lines=1]{T}{he} Octave of St. John, apostle and evangelist.
\switchcolumn*
\selectlanguage{latin}
Romæ, via Appia, natális sancti Anthéri, Papæ 
 et Mártyris; qui sub Júlio Maximíno passus est, et in cœmetério Callísti 
 sepúltus.
\switchcolumn
\selectlanguage{english}
At Rome, on the Appian Way, the birthday of Pope St. Anterus, who suffered 
 under Julius Maximinus, and was buried in the cemetery of Callistus.
\switchcolumn*
\selectlanguage{latin}
Viénnæ, in Gállia, sancti Floréntii Epíscopi, 
 qui, témpore Galliéni Imperatóris, in exsílium relegátus, illic martyrium 
 consummávit.
\switchcolumn
\selectlanguage{english}
At Vienne in France, St. Florentius, bishop, who was sent into exile and who 
 was martyred in the time of Emperor Gallienus.
\switchcolumn*
\selectlanguage{latin}
Apud civitátem Aulánam, in Palæstína, pássio 
 sancti Petri, qui crucis supplício interémptus est.
\switchcolumn
\selectlanguage{english}
In the city of Aulona in Palestine, the crucifixion of St. Peter.
\switchcolumn*
\selectlanguage{latin}
In Hellespónto sanctórum Mártyrum Cyríni, Primi et Theógenis.
\switchcolumn
\selectlanguage{english}
In the Hellespont, the holy martyrs Cyrinus, Primus, and Theogenes.
\switchcolumn*
\selectlanguage{latin}
Cæsaréæ, in Cappadócia, sancti Górdii 
 Centuriónis, Mártyris; de cujus láudibus exstat præclára Basilíi Magni 
 orátio, in ejus die festo hábita.
\switchcolumn
\selectlanguage{english}
At Caesarea in Cappadocia, St. Gordius, centurion, in whose praise is extant 
 a celebrated discourse delivered by St. Basil the Great on the day of his 
 festival.
\switchcolumn*
\selectlanguage{latin}
In Cilícia sanctórum Mártyrum Zósimi, et 
 Athanásii Commentariénsis.
\switchcolumn
\selectlanguage{english}
In Cilicia, the holy martyrs Zosimus and the notary Athanasius.
\switchcolumn*
\selectlanguage{latin}
Item sanctórum Theopémpti et Theónæ, qui, in 
 persecutióne Diocletiáni, illústre martyrium obiérunt.
\switchcolumn
\selectlanguage{english}
Also, the Saints Theopemptus and Theonas, who suffered a glorious martyrdom 
 in the persecution of Diocletian.
\switchcolumn*
\selectlanguage{latin}
Patávii sancti Daniélis Mártyris.
\switchcolumn
\selectlanguage{english}
At Padua, St. Daniel, martyr.
\switchcolumn*
\selectlanguage{latin}
Lutétiæ Parisiórum sanctæ Genovéfæ Vírginis, 
 quæ, a beáto Germáno, Antisiodorénsi Epíscopo, Christo dicáta, admirándis 
 virtútibus et miráculis cláruit.
\switchcolumn
\selectlanguage{english}
At Paris, St. Genevieve, virgin, who was consecrated to Christ by St. 
 Germanus, bishop of Auxerre, and who became famous for her admirable virtues 
 and miracles.
\switchcolumn*
\selectlanguage{latin}
\end{paracol}


% ---- martyrology/mart01/mart0104.htm
\needspace{10\baselineskip}
\begin{paracol}{2}
\selectlanguage{latin}
\begin{center}{\color{gregoriocolor} Prídie Nonas Januárii. 
 Luna\dots\ }\end{center}
\switchcolumn
\selectlanguage{english}
\begin{center}{\color{gregoriocolor} The   Fourth Day of 
 January. The\dots\ Day of the Moon.}\end{center}
\end{paracol}

\noindent\begin{tabularx}{\linewidth}{*{19}{>{\centering\arraybackslash}X}}
 \textcolor{gregoriocolor}{a} & \textcolor{gregoriocolor}{b} & \textcolor{gregoriocolor}{c} & \textcolor{gregoriocolor}{d} & \textcolor{gregoriocolor}{e} & \textcolor{gregoriocolor}{f} & \textcolor{gregoriocolor}{g} & \textcolor{gregoriocolor}{h} & \textcolor{gregoriocolor}{i} & \textcolor{gregoriocolor}{k} & \textcolor{gregoriocolor}{l} & \textcolor{gregoriocolor}{m} & \textcolor{gregoriocolor}{n} & \textcolor{gregoriocolor}{p} & \textcolor{gregoriocolor}{q} & \textcolor{gregoriocolor}{r} & \textcolor{gregoriocolor}{s} & \textcolor{gregoriocolor}{t} & \textcolor{gregoriocolor}{u} \\
 5 & 6 & 7 & 8 & 9 & 10 & 11 & 12 & 13 & 14 & 15 & 16 & 17 & 18 & 19 & 20 & 21 & 22 & 23 \\
\end{tabularx}
\vspace{0.5\baselineskip}
\noindent\begin{tabularx}{\linewidth}{*{12}{>{\centering\arraybackslash}X}}
 \textcolor{gregoriocolor}{A} & \textcolor{gregoriocolor}{B} & \textcolor{gregoriocolor}{C} & \textcolor{gregoriocolor}{D} & \textcolor{gregoriocolor}{E} & F & \textcolor{gregoriocolor}{F} & \textcolor{gregoriocolor}{G} & \textcolor{gregoriocolor}{H} & \textcolor{gregoriocolor}{M} & \textcolor{gregoriocolor}{N} & \textcolor{gregoriocolor}{P} \\
 24 & 25 & 26 & 27 & 28 & 29 & 29 & 30 & 1 & 2 & 3 & 4 \\
\end{tabularx}

\begin{paracol}{2}
\selectlanguage{latin}
\lettrine[lines=1]{O}{ctáva} sanctórum Innocéntium Mártyrum.
\switchcolumn
\selectlanguage{english}
\lettrine[lines=1]{T}{he} Octave of the Holy Innocents.
\switchcolumn*
\selectlanguage{latin}
In Creta natális sancti Titi, qui, ab Apóstolo Paulo Epíscopus Creténsium 
 ordinátus, et, post prædicatiónis offícium 
 fidelíssime consummátum, finem beátum adéptus, in ea sepúltus est Ecclésia, 
 ubi a beáto Apóstolo dignus miníster fúerat constitútus. Ipsíus tamen 
 festívitas octávo Idus Februárii celebrátur.
\switchcolumn
\selectlanguage{english}
In Crete, the birthday of St. Titus, who was consecrated bishop of that 
 island by the apostle St. Paul. After having faithfully performed the 
 duty of preaching the Gospel, he reached the end of his saintly life, and 
 was buried in the church of which he had been made a worthy minister by the 
 holy apostle.
\switchcolumn*
\selectlanguage{latin}
Romæ sanctórum Mártyrum Prisci Presbyteri, et 
 Priscilliáni Clérici, ac Benedíctæ, religiósæ féminæ; qui, témpore 
 impiíssimi Juliáni, gládio martyrium complevérunt.
\switchcolumn
\selectlanguage{english}
At Rome, in the reign of the impious Julian, the holy martyrs Priscus, a 
 priest, Priscillian, a cleric; and Benedicta, a religious woman, whose 
 martyrdom was ended by the sword.
\switchcolumn*
\selectlanguage{latin}
Item Romæ beátæ Dafrósæ, uxóris sancti Flaviáni 
 Mártyris, ac matris sanctárum Bibiánæ et Demétriæ, Vírginum et Mártyrum; quæ, 
 post interfectiónem viri sui, primum exsílio relegáta, deínde, sub præfáto 
 Príncipe, cápite plexa est.
\switchcolumn
\selectlanguage{english}
Also at Rome, under the same emperor, blessed Dafrosa, wife of the martyr 
 St. Flavian, and mother of Saints Bibiana and Demetria, virgin martyrs. 
 After her husband had been killed, she was first banished and then beheaded.
\switchcolumn*
\selectlanguage{latin}
Bonóniæ sanctórum Hermétis, Aggǽi et Caji 
 Mártyrum, qui sub Maximiáno Imperatóre passi sunt.
\switchcolumn
\selectlanguage{english}
At Bologna, the Saints Hermes, Aggaeus, and Caius, martyrs, who suffered 
 under Emperor Maximian.
\switchcolumn*
\selectlanguage{latin}
Adruméti, in Africa, commemorátio sancti Mávili 
 Mártyris, qui in persecutióne Sevéri Imperatóris, a sævíssimo Præside 
 Scápula damnátus ad béstias, martyrii corónam accépit.
\switchcolumn
\selectlanguage{english}
At Adrumetum in Africa, in the persecution of Severus, the commemoration of 
 St. Mavilus, martyr, who, being condemned by the very cruel governor Scapula 
 to be devoured by wild beasts, received the crown of martyrdom.
\switchcolumn*
\selectlanguage{latin}
Item in Africa præclarissimórum Mártyrum 
 Aquilíni, Gémini, Eugénii, Marciáni, Quincti, Theódoti et Tryphónis.
\switchcolumn
\selectlanguage{english}
Also in Africa, the most renowned martyrs Aquilinus, Geminus, Eugenius, 
 Marcian, Quinctus, Theodotus, and Tryphon.
\switchcolumn*
\selectlanguage{latin}
Apud Língonas, in Gállia, sancti Gregórii 
 Epíscopi, miráculis clari.
\switchcolumn
\selectlanguage{english}
At Langres in France, St. Gregory, a bishop renowned for miracles.
\switchcolumn*
\selectlanguage{latin}
Rhemis, in Gállia, sancti Rigobérti, Epíscopi et Confessóris.
\switchcolumn
\selectlanguage{english}
At Rheims in France, St. Rigobertus, bishop and confessor.
\switchcolumn*
\selectlanguage{latin}
\end{paracol}


% ---- martyrology/mart01/mart0105.htm
\needspace{10\baselineskip}
\begin{paracol}{2}
\selectlanguage{latin}
\begin{center}{\color{gregoriocolor} Nonis Januárii. 
 Luna\dots\ }\end{center}
\switchcolumn
\selectlanguage{english}
\begin{center}{\color{gregoriocolor} The   Fifth Day of 
 January. The\dots\ Day of the Moon.}\end{center}
\end{paracol}

\noindent\begin{tabularx}{\linewidth}{*{19}{>{\centering\arraybackslash}X}}
 \textcolor{gregoriocolor}{a} & \textcolor{gregoriocolor}{b} & \textcolor{gregoriocolor}{c} & \textcolor{gregoriocolor}{d} & \textcolor{gregoriocolor}{e} & \textcolor{gregoriocolor}{f} & \textcolor{gregoriocolor}{g} & \textcolor{gregoriocolor}{h} & \textcolor{gregoriocolor}{i} & \textcolor{gregoriocolor}{k} & \textcolor{gregoriocolor}{l} & \textcolor{gregoriocolor}{m} & \textcolor{gregoriocolor}{n} & \textcolor{gregoriocolor}{p} & \textcolor{gregoriocolor}{q} & \textcolor{gregoriocolor}{r} & \textcolor{gregoriocolor}{s} & \textcolor{gregoriocolor}{t} & \textcolor{gregoriocolor}{u} \\
 6 & 7 & 8 & 9 & 10 & 11 & 12 & 13 & 14 & 15 & 16 & 17 & 18 & 19 & 20 & 21 & 22 & 23 & 24 \\
\end{tabularx}
\vspace{0.5\baselineskip}
\noindent\begin{tabularx}{\linewidth}{*{12}{>{\centering\arraybackslash}X}}
 \textcolor{gregoriocolor}{A} & \textcolor{gregoriocolor}{B} & \textcolor{gregoriocolor}{C} & \textcolor{gregoriocolor}{D} & \textcolor{gregoriocolor}{E} & F & \textcolor{gregoriocolor}{F} & \textcolor{gregoriocolor}{G} & \textcolor{gregoriocolor}{H} & \textcolor{gregoriocolor}{M} & \textcolor{gregoriocolor}{N} & \textcolor{gregoriocolor}{P} \\
 25 & 26 & 27 & 28 & 29 & 30 & 30 & 1 & 2 & 3 & 4 & 5 \\
\end{tabularx}

\begin{paracol}{2}
\selectlanguage{latin}
\lettrine[lines=1]{V}{igília} Epiphaníæ Dómini.
\switchcolumn
\selectlanguage{english}
\lettrine[lines=1]{T}{he} Vigil of the Epiphany of our Lord.
\switchcolumn*
\selectlanguage{latin}
Romæ sancti Telésphori, Papæ et Mártyris; qui, 
 sub Antoníno Pio, post multos labóres, pro Christi confessióne, illústre 
 martyrium duxit.
\switchcolumn
\selectlanguage{english}
At Rome, in the time of Antoninus Pius, St. Telesphorus, pope, who, after 
 many sufferings for the confession of Christ, underwent a glorious 
 martyrdom.
\switchcolumn*
\selectlanguage{latin}
In Anglia natális sancti Eduárdi, Regis Anglórum et Confessóris; qui virtúte 
 castitátis et grátia miraculórum fuit insígnis. Ejus autem festívitas, 
 ex decréto Innocéntii Papæ Undécimi, tértio 
 Idus Octóbris, quo die sacrum ejus corpus translátum fuit, potíssimum 
 celebrátur.
\switchcolumn
\selectlanguage{english}
In England, St. Edward, king and confessor, illustrious by the virtue of 
 chastity and the gift of miracles. His feast, by order of Pope 
 Innocent XI, is celebrated on the 13th of October, on which day his holy 
 body was transferred.
\switchcolumn*
\selectlanguage{latin}
In Ægypto commemorátio plurimórum sanctórum 
 Mártyrum, qui in Thebáide, sub persecutióne Diocletiáni, divérso tormentórum 
 génere cæsi sunt.
\switchcolumn
\selectlanguage{english}
In Egypt, during the persecution of Diocletian, the commemoration of many 
 holy martyrs who were put to death in Thebais by various kinds of torments.
\switchcolumn*
\selectlanguage{latin}
Antiochíæ sancti Simeónis Mónachi, qui, multos 
 annos in colúmna stans vixit, unde et Stylítæ cognómen accépit; cujus vita 
 et conversátio éxstitit admirábilis.
\switchcolumn
\selectlanguage{english}
At Antioch, St. Simeon, monk, admirable both for his life and for his 
 conversation. He lived for many years standing on a pillar, and was 
 for that reason called Stylites.
\switchcolumn*
\selectlanguage{latin}
Romæ sanctæ Æmiliánæ Vírginis, ámitæ sancti 
 Gregórii Papæ; quæ, vocánte Tharsílla soróre, quæ ad Deum præcésserat, hac 
 ipsa die migrávit ad Dóminum.
\switchcolumn
\selectlanguage{english}
At Rome, the holy virgin Emiliana, aunt of Pope St. Gregory. Being 
 called to God by her sister Tharsilla, who had preceded her, she departed to 
 heaven on this day.
\switchcolumn*
\selectlanguage{latin}
Alexandríæ sanctæ Syncléticæ Vírginis, cujus 
 res præcláre gestas sanctus Athanásius monuméntis litterárum commendávit.
\switchcolumn
\selectlanguage{english}
At Alexandria, St. Syncletica, whose noble deeds have been recorded by St. 
 Athanasius.
\switchcolumn*
\selectlanguage{latin}
In Ægypto sanctæ Apollináris Vírginis.
\switchcolumn
\selectlanguage{english}
In Egypt, St. Apollinaris, virgin.
\switchcolumn*
\selectlanguage{latin}
\end{paracol}


% ---- martyrology/mart01/mart0106.htm
\needspace{10\baselineskip}
\begin{paracol}{2}
\selectlanguage{latin}
\begin{center}{\color{gregoriocolor} Octávo Idus Januárii. 
 Luna\dots\ }\end{center}
\switchcolumn
\selectlanguage{english}
\begin{center}{\color{gregoriocolor} The Sixth Day of 
 January. The\dots\ Day of the Moon.}\end{center}
\end{paracol}

\noindent\begin{tabularx}{\linewidth}{*{19}{>{\centering\arraybackslash}X}}
 \textcolor{gregoriocolor}{a} & \textcolor{gregoriocolor}{b} & \textcolor{gregoriocolor}{c} & \textcolor{gregoriocolor}{d} & \textcolor{gregoriocolor}{e} & \textcolor{gregoriocolor}{f} & \textcolor{gregoriocolor}{g} & \textcolor{gregoriocolor}{h} & \textcolor{gregoriocolor}{i} & \textcolor{gregoriocolor}{k} & \textcolor{gregoriocolor}{l} & \textcolor{gregoriocolor}{m} & \textcolor{gregoriocolor}{n} & \textcolor{gregoriocolor}{p} & \textcolor{gregoriocolor}{q} & \textcolor{gregoriocolor}{r} & \textcolor{gregoriocolor}{s} & \textcolor{gregoriocolor}{t} & \textcolor{gregoriocolor}{u} \\
 7 & 8 & 9 & 10 & 11 & 12 & 13 & 14 & 15 & 16 & 17 & 18 & 19 & 20 & 21 & 22 & 23 & 24 & 25 \\
\end{tabularx}
\vspace{0.5\baselineskip}
\noindent\begin{tabularx}{\linewidth}{*{12}{>{\centering\arraybackslash}X}}
 \textcolor{gregoriocolor}{A} & \textcolor{gregoriocolor}{B} & \textcolor{gregoriocolor}{C} & \textcolor{gregoriocolor}{D} & \textcolor{gregoriocolor}{E} & F & \textcolor{gregoriocolor}{F} & \textcolor{gregoriocolor}{G} & \textcolor{gregoriocolor}{H} & \textcolor{gregoriocolor}{M} & \textcolor{gregoriocolor}{N} & \textcolor{gregoriocolor}{P} \\
 26 & 27 & 28 & 29 & 30 & 1 & 1 & 2 & 3 & 4 & 5 & 6 \\
\end{tabularx}

\begin{paracol}{2}
\selectlanguage{latin}
\lettrine[lines=1]{E}{piphanía} Domini.
\switchcolumn
\selectlanguage{english}
\lettrine[lines=1]{T}{he} Epiphany of our Lord.
\switchcolumn*
\selectlanguage{latin}
Floréntiæ natális sancti Andréæ Corsíni, civis 
 Florentíni, ex Ordine Carmelitárum, Epíscopi Fæsuláni et Confessóris; quem, 
 miráculis clarum, Urbánus Papa Octávus in Sanctórum númerum rétulit. Ejus autem festívitas recólitur prídie nonas Februárii.
\switchcolumn
\selectlanguage{english}
At Florence, St. Andrew Corsini, a Florentine Carmelite and bishop of 
 Fiesole. Being celebrated for miracles, he was ranked among the saints 
 by Urban VIII. His festival is kept on the 4th of February.
\switchcolumn*
\selectlanguage{latin}
Barcinóne, in Hispánia, item natális sancti Raymúndi de Pénafort, ex Ordine 
 Prædicatórum, Confessóris, doctrína et 
 sanctitáte célebris. Ipsíus vero festum décimo Kaléndas Februárii 
 celebrátur.
\switchcolumn
\selectlanguage{english}
At Barcelona in Spain, St. Raymond of Pennafort, of the Order of Preachers, 
 celebrated for sanctity and learning. His festival is kept on the 23rd 
 of this month.
\switchcolumn*
\selectlanguage{latin}
In Africa commemorátio plurimórum sanctórum Mártyrum, qui, in persecutióne 
 Sevéri, ad palum ligáti sunt et igne consúmpti.
\switchcolumn
\selectlanguage{english}
In Africa, the commemoration of many holy martyrs who were burned at the 
 stake in the persecution of Severus.
\switchcolumn*
\selectlanguage{latin}
In território Rheménsi pássio sanctæ Macræ 
 Vírginis, quæ, in persecutióne Diocletiáni, jubénte Rictiováro Præside, cum 
 in ignem esset præcipitáta et permansísset illæsa, dehinc, mamíllis 
 abscíssis et squalóre cárceris afflícta, super testas étiam acutíssimas ac 
 prunas volutáta, tandem orans migrávit ad Dóminum.
\switchcolumn
\selectlanguage{english}
In the diocese of Rheims, the martyrdom of St. Macra, virgin, who, in the 
 persecution of Diocletian, was cast into the fire by order of the governor 
 Rictiovarus. As she remained uninjured, she had her breasts cut away, 
 was imprisoned in a foul dungeon, rolled upon broken earthenware and burning 
 coals, and finally she gave up her soul while engaged in prayer.
\switchcolumn*
\selectlanguage{latin}
Rhédonis, in Gállia, sancti Melánii, Epíscopi 
 et Confessóris; qui, post innumerabílium signa virtútum, júgiter cælo 
 inténtus, gloriósus migrávit a sæculo.
\switchcolumn
\selectlanguage{english}
At Rennes in France, St. Melanius, bishop and confessor, who, after a life 
 remarkable for innumerable virtues, with his thoughts constantly fixed on 
 heaven, gloriously departed from this world.
\switchcolumn*
\selectlanguage{latin}
Geris, in Ægypto, sancti Nilammónis reclúsi, 
 qui, dum ad Episcopátum traherétur invítus, in oratióne spíritum Deo 
 réddidit.
\switchcolumn
\selectlanguage{english}
At Geris in Egypt, St. Nilammon, anchoret, who, while he was carried to a 
 bishopric against his will, gave up his soul to God in prayer.
\switchcolumn*
\selectlanguage{latin}
\end{paracol}


% ---- martyrology/mart01/mart0107.htm
\needspace{10\baselineskip}
\begin{paracol}{2}
\selectlanguage{latin}
\begin{center}{\color{gregoriocolor} Séptimo Idus Januárii. 
 Luna\dots\ }\end{center}
\switchcolumn
\selectlanguage{english}
\begin{center}{\color{gregoriocolor} The Seventh Day of 
 January. The\dots\ Day of the Moon.}\end{center}
\end{paracol}

\noindent\begin{tabularx}{\linewidth}{*{19}{>{\centering\arraybackslash}X}}
 \textcolor{gregoriocolor}{a} & \textcolor{gregoriocolor}{b} & \textcolor{gregoriocolor}{c} & \textcolor{gregoriocolor}{d} & \textcolor{gregoriocolor}{e} & \textcolor{gregoriocolor}{f} & \textcolor{gregoriocolor}{g} & \textcolor{gregoriocolor}{h} & \textcolor{gregoriocolor}{i} & \textcolor{gregoriocolor}{k} & \textcolor{gregoriocolor}{l} & \textcolor{gregoriocolor}{m} & \textcolor{gregoriocolor}{n} & \textcolor{gregoriocolor}{p} & \textcolor{gregoriocolor}{q} & \textcolor{gregoriocolor}{r} & \textcolor{gregoriocolor}{s} & \textcolor{gregoriocolor}{t} & \textcolor{gregoriocolor}{u} \\
 8 & 9 & 10 & 11 & 12 & 13 & 14 & 15 & 16 & 17 & 18 & 19 & 20 & 21 & 22 & 23 & 24 & 25 & 26 \\
\end{tabularx}
\vspace{0.5\baselineskip}
\noindent\begin{tabularx}{\linewidth}{*{12}{>{\centering\arraybackslash}X}}
 \textcolor{gregoriocolor}{A} & \textcolor{gregoriocolor}{B} & \textcolor{gregoriocolor}{C} & \textcolor{gregoriocolor}{D} & \textcolor{gregoriocolor}{E} & F & \textcolor{gregoriocolor}{F} & \textcolor{gregoriocolor}{G} & \textcolor{gregoriocolor}{H} & \textcolor{gregoriocolor}{M} & \textcolor{gregoriocolor}{N} & \textcolor{gregoriocolor}{P} \\
 27 & 28 & 29 & 30 & 1 & 2 & 2 & 3 & 4 & 5 & 6 & 7 \\
\end{tabularx}

\begin{paracol}{2}
\selectlanguage{latin}
\lettrine[lines=1]{R}{elátio} púeri Jesu de Ægypto.
\switchcolumn
\selectlanguage{english}
\lettrine[lines=1]{T}{he} return of the Child Jesus from Egypt.
\switchcolumn*
\selectlanguage{latin}
Nicomedíæ natális beáti Luciáni, Ecclésiæ 
 Antiochénæ Presbyteri et Mártyris; qui, satis clarus doctrína et eloquéntia, 
 passus est, ob Christi confessiónem, in persecutióne Galérii Maximiáni, 
 sepultúsque est Helenópoli, in Bithynia. Ipsíus autem laudes sanctus 
 Joánnes Chrysóstomus celebrávit.
\switchcolumn
\selectlanguage{english}
The birthday of blessed Lucian, a priest of the Church of Antioch and 
 martyr, who was distinguished for his learning and eloquence. He 
 suffered at Nicomedia for the confession of Christ, in the persecution of 
 Galerius Maximian, and was buried at Helenopolis, in Bithynia. His 
 praises have been proclaimed by St. John Chrysostom.
\switchcolumn*
\selectlanguage{latin}
Antiochíæ sancti Cleri Diáconi, qui pro 
 confessiónis glória, sépties tortus ac diu macerátus in cárcere, ad últimum, 
 gládio decollátus, martyrium consummávit.
\switchcolumn
\selectlanguage{english}
At Antioch, St. Clerus, deacon, who, for having professed faith in Christ, 
 was seven times tortured, kept in prison a long while, and at length his 
 martyrdom was ended by decapitation.
\switchcolumn*
\selectlanguage{latin}
In civitáte Heracléa sanctórum Mártyrum Felícis et Januárii.
\switchcolumn
\selectlanguage{english}
In the city of Heraclea, the holy martyrs Felix and Januarius.
\switchcolumn*
\selectlanguage{latin}
Eódem die sancti Juliáni Mártyris.
\switchcolumn
\selectlanguage{english}
The same day, St. Julian, martyr.
\switchcolumn*
\selectlanguage{latin}
In Dánia sancti Canúti, Regis et Mártyris.
\switchcolumn
\selectlanguage{english}
In Denmark, St. Canute, king and martyr.
\switchcolumn*
\selectlanguage{latin}
Papíæ sancti Crispíni, Epíscopi et Confessóris.
\switchcolumn
\selectlanguage{english}
At Pavia, St. Crispin, bishop and confessor.
\switchcolumn*
\selectlanguage{latin}
In Dácia sancti Nicétæ Epíscopi, qui feras et 
 bárbaras gentes, Evangélii prædicatióne, mites réddidit ac mansuétas.
\switchcolumn
\selectlanguage{english}
In Dacia, St. Nicetas, bishop, who made fierce and barbarous nations humane 
 and meek by preaching the Gospel to them.
\switchcolumn*
\selectlanguage{latin}
In Ægypto beáti Theodóri Mónachi, qui, témpore 
 Constantíni Magni, flóruit sanctitáte; cujus méminit sanctus Athanásius in 
 vita sancti Antónii.
\switchcolumn
\selectlanguage{english}
In Egypt, St. Theodore, a saintly monk, who flourished in the time of 
 Constantine the Great. He is mentioned by St. Athanasius in his Life 
 of St. Anthony.
\switchcolumn*
\selectlanguage{latin}
\end{paracol}


% ---- martyrology/mart01/mart0108.htm
\needspace{10\baselineskip}
\begin{paracol}{2}
\selectlanguage{latin}
\begin{center}{\color{gregoriocolor} Sexto Idus Januárii. 
 Luna\dots\ }\end{center}
\switchcolumn
\selectlanguage{english}
\begin{center}{\color{gregoriocolor} The Eighth Day of 
 January. The\dots\ Day of the Moon.}\end{center}
\end{paracol}

\noindent\begin{tabularx}{\linewidth}{*{19}{>{\centering\arraybackslash}X}}
 \textcolor{gregoriocolor}{a} & \textcolor{gregoriocolor}{b} & \textcolor{gregoriocolor}{c} & \textcolor{gregoriocolor}{d} & \textcolor{gregoriocolor}{e} & \textcolor{gregoriocolor}{f} & \textcolor{gregoriocolor}{g} & \textcolor{gregoriocolor}{h} & \textcolor{gregoriocolor}{i} & \textcolor{gregoriocolor}{k} & \textcolor{gregoriocolor}{l} & \textcolor{gregoriocolor}{m} & \textcolor{gregoriocolor}{n} & \textcolor{gregoriocolor}{p} & \textcolor{gregoriocolor}{q} & \textcolor{gregoriocolor}{r} & \textcolor{gregoriocolor}{s} & \textcolor{gregoriocolor}{t} & \textcolor{gregoriocolor}{u} \\
 9 & 10 & 11 & 12 & 13 & 14 & 15 & 16 & 17 & 18 & 19 & 20 & 21 & 22 & 23 & 24 & 25 & 26 & 27 \\
\end{tabularx}
\vspace{0.5\baselineskip}
\noindent\begin{tabularx}{\linewidth}{*{12}{>{\centering\arraybackslash}X}}
 \textcolor{gregoriocolor}{A} & \textcolor{gregoriocolor}{B} & \textcolor{gregoriocolor}{C} & \textcolor{gregoriocolor}{D} & \textcolor{gregoriocolor}{E} & F & \textcolor{gregoriocolor}{F} & \textcolor{gregoriocolor}{G} & \textcolor{gregoriocolor}{H} & \textcolor{gregoriocolor}{M} & \textcolor{gregoriocolor}{N} & \textcolor{gregoriocolor}{P} \\
 28 & 29 & 30 & 1 & 2 & 3 & 3 & 4 & 5 & 6 & 7 & 8 \\
\end{tabularx}

\begin{paracol}{2}
\selectlanguage{latin}
\lettrine[lines=2]{V}{enétiis} deposítio sancti Lauréntii Justiniáni, primi Patriárchæ 
 urbis ejúsdem et Confessóris; quem, doctrína et supérnis divínæ sapiéntiæ 
 charismátibus copiosíssime replétum, Alexánder Octávus, Póntifex Máximus, in 
 Sanctórum númerum rétulit. Ipsíus autem festívitas Nonis Septémbris, 
 quo die Cáthedram pontificálem ascéndit, potíssimum celebrátur.
\switchcolumn
\selectlanguage{english}
\lettrine[lines=2]{A}{t} Venice, the death of St. Lawrence Justinian, confessor, first patriarch 
 of that city. Eminent for learning, and abundantly filled with the 
 heavenly gifts of divine wisdom, he was ranked among the saints by Alexander 
 VIII. He is again mentioned on the 5th of September, on which day he 
 ascended the pontifical throne.
\switchcolumn*
\selectlanguage{latin}
Bellóvaci, in Gálliis, sanctórum Mártyrum 
 Luciáni Presbyteri, Maximiáni et Juliáni. Horum duo últimi a persecutóribus 
 gládio perémpti sunt; beátum autem Luciánus, qui, una cum sancto Dionysio, 
 in Gálliam vénerat, et ipse, post nímiam cædem, cum Christi nomen viva voce 
 confitéri non timuísset, priórum senténtiam excépit.
\switchcolumn
\selectlanguage{english}
At Beauvais in France, the holy martyrs Lucian, priest, Maximian and Julian. 
 The last two were killed with the sword by the persecutors; but blessed 
 Lucian, who had come to France with St. Denis, after the slaughter of his 
 companions, not fearing to confess the Name of Christ openly, received the 
 same sentence of death.
\switchcolumn*
\selectlanguage{latin}
In Libya sanctórum Mártyrum Theóphili Diáconi, et Helládii, qui, primo 
 laniáti ac téstulis peracútis compúncti, demum, in ignem conjécti, ánimas 
 Deo reddidérunt.
\switchcolumn
\selectlanguage{english}
In Libya, the holy martyrs Theophilus, deacon, and Helladius, who, after 
 having their bodies lacerated and cut with sharp pieces of earthenware, were 
 cast into the fire, and rendered their souls unto God.
\switchcolumn*
\selectlanguage{latin}
Augustodúni sancti Eugeniáni Mártyris.
\switchcolumn
\selectlanguage{english}
At Autun, St. Eugenian, martyr.
\switchcolumn*
\selectlanguage{latin}
Hierápoli, in Asia, sancti Apollináris Epíscopi, qui, sub Marco Antoníno 
 Vero, sanctitáte atque doctrína flóruit.
\switchcolumn
\selectlanguage{english}
At Hierapolis in Asia, St. Apollinaris, bishop, who was conspicuous for 
 sanctity and learning under Marucs Antoninus Verus.
\switchcolumn*
\selectlanguage{latin}
Neápoli, in Campánia, natális sancti Severíni Epíscopi, qui fuit frater 
 beáti Victoríni Mártyris; et, post multárum virtútum perpetratiónem, plenus 
 sanctitáte quiévit.
\switchcolumn
\selectlanguage{english}
At Naples in Campania, the birthday of the bishop St. Severin, brother to 
 the blessed martyr Victorinus, who, after working many miracles, died, 
 replenished with virtues and merits.
\switchcolumn*
\selectlanguage{latin}
Metis, in Gállia, sancti Patiéntis Epíscopi.
\switchcolumn
\selectlanguage{english}
At Metz in France, St. Patiens, bishop.
\switchcolumn*
\selectlanguage{latin}
Papíæ sancti Máximi, Epíscopi et Confessóris.
\switchcolumn
\selectlanguage{english}
At Pavia, St. Maximus, bishop and confessor.
\switchcolumn*
\selectlanguage{latin}
Ratisbónæ, in Bavária, sancti Erhárdi Epíscopi.
\switchcolumn
\selectlanguage{english}
At Ratisbon in Bavaria, St. Erhard, bishop.
\switchcolumn*
\selectlanguage{latin}
Apud Nóricos sancti Severíni Abbátis, qui apud 
 eam gentem Evangélium propagávit, et Noricórum dictus est Apóstolus. 
 Ejus corpus ad Lucullánum prope Neápolim, in Campánia, divínitus delátum, 
 inde póstea ad monastérium sancti Severíni translátum est.
\switchcolumn
\selectlanguage{english}
Among the inhabitants of Noricum (now Austria), the abbot St. Severin, who 
 propagated the Gospel in that country, and is called its apostle. By 
 divine power his body was carried to Lucullano, near Naples, and thence 
 transferred to the monastery of St. Severin.
\switchcolumn*
\selectlanguage{latin}
\end{paracol}


% ---- martyrology/mart01/mart0109.htm
\needspace{10\baselineskip}
\begin{paracol}{2}
\selectlanguage{latin}
\begin{center}{\color{gregoriocolor} Quinto Idus Januárii. 
 Luna\dots\ }\end{center}
\switchcolumn
\selectlanguage{english}
\begin{center}{\color{gregoriocolor} The Ninth Day of 
 January. The\dots\ Day of the Moon.}\end{center}
\end{paracol}

\noindent\begin{tabularx}{\linewidth}{*{19}{>{\centering\arraybackslash}X}}
 \textcolor{gregoriocolor}{a} & \textcolor{gregoriocolor}{b} & \textcolor{gregoriocolor}{c} & \textcolor{gregoriocolor}{d} & \textcolor{gregoriocolor}{e} & \textcolor{gregoriocolor}{f} & \textcolor{gregoriocolor}{g} & \textcolor{gregoriocolor}{h} & \textcolor{gregoriocolor}{i} & \textcolor{gregoriocolor}{k} & \textcolor{gregoriocolor}{l} & \textcolor{gregoriocolor}{m} & \textcolor{gregoriocolor}{n} & \textcolor{gregoriocolor}{p} & \textcolor{gregoriocolor}{q} & \textcolor{gregoriocolor}{r} & \textcolor{gregoriocolor}{s} & \textcolor{gregoriocolor}{t} & \textcolor{gregoriocolor}{u} \\
 10 & 11 & 12 & 13 & 14 & 15 & 16 & 17 & 18 & 19 & 20 & 21 & 22 & 23 & 24 & 25 & 26 & 27 & 28 \\
\end{tabularx}
\vspace{0.5\baselineskip}
\noindent\begin{tabularx}{\linewidth}{*{12}{>{\centering\arraybackslash}X}}
 \textcolor{gregoriocolor}{A} & \textcolor{gregoriocolor}{B} & \textcolor{gregoriocolor}{C} & \textcolor{gregoriocolor}{D} & \textcolor{gregoriocolor}{E} & F & \textcolor{gregoriocolor}{F} & \textcolor{gregoriocolor}{G} & \textcolor{gregoriocolor}{H} & \textcolor{gregoriocolor}{M} & \textcolor{gregoriocolor}{N} & \textcolor{gregoriocolor}{P} \\
 29 & 30 & 1 & 2 & 3 & 4 & 4 & 5 & 6 & 7 & 8 & 9 \\
\end{tabularx}

\begin{paracol}{2}
\selectlanguage{latin}
\lettrine[lines=2]{A}{ntiochíæ,} sub Diocletiáno et Maximiáno, 
 natális sanctórum Juliáni Mártyris, et Basilíssæ Vírginis, ipsíus Juliáni 
 uxóris. Hæc, virginitáte cum viro suo serváta, in pace vitam finívit; 
 Juliánus vero (postquam multitúdo Sacerdótum et Ministrórum Ecclésiæ 
 Christi, quæ, propter immanitátem persecutiónis, ad eos confúgerat, igne 
 cremáta est), Marciáni Præsidis jussu, plúrimis torméntis cruciátus, 
 capitálem senténtiam accépit. Cum ipso étiam Antónius Présbyter, et Anastásius, quem idem Juliánus, a morte suscitátum, grátiæ Christi partícipem fécerat, et Celsus puer cum hujus matre Marcionílla, ac septem 
 fratres, aliíque plúrimi passi sunt.
\switchcolumn
\selectlanguage{english}
\lettrine[lines=2]{A}{t} Antioch, in the reign of Diocletian and Maximian, the birthday of the 
 Saints Julian, martyr, and Basilissa, his virgin wife. She, having 
 lived in a state of virginity with her husband, reached the end of her days 
 in peace. But Julian, after the death by fire of a multitude of 
 priests and ministers of the Church of Christ, who had taken refuge in his 
 house from the severity of the persecution, was ordered by the governor 
 Marcian to be tormented in many ways and executed. With him there 
 suffered Anthony, a priest, and Anastasius, whom Julian raised from the 
 dead, and made partaker of the grace of Christ; also Celsus, a boy, with his 
 mother Marcionilla, seven brothers, and many others.
\switchcolumn*
\selectlanguage{latin}
Smyrnæ sanctórum Mártyrum Vitális, Revocáti et 
 Fortunáti.
\switchcolumn
\selectlanguage{english}
At Smyrna, the holy martyrs Vitalis, Revocatus, and Fortunatus.
\switchcolumn*
\selectlanguage{latin}
In Africa sanctórum Mártyrum Epictéti, Jucúndi, 
 Secúndi, Vitális, Felícis et aliórum septem.
\switchcolumn
\selectlanguage{english}
In Africa, the holy martyrs Epictetus, Jucundus, Secundus, Vitalis, Felix, 
 and seven others.
\switchcolumn*
\selectlanguage{latin}
In Mauritánia Cæsariénsi sanctæ Marciánæ 
 Vírginis, quæ, béstiis trádita, martyrium consummávit.
\switchcolumn
\selectlanguage{english}
In Algeria, St. Marciana, virgin, who received her martyrdom after being 
 condemned to the beasts.
\switchcolumn*
\selectlanguage{latin}
Sebáste, in Arménia, sancti Petri Epíscopi, fílii sanctórum Basilíi 
 et Emméliæ, atque fratris item sanctórum Basilíi Magni et Gregórii Nysséni 
 Episcopórum, ac Macrínæ Vírginis.
\switchcolumn
\selectlanguage{english}
At Sebaste in Armenia, St. Peter, bishop, the son of Saints Basil and 
 Emmelia, and also the brother of Saints Basil the Great, Gregory of Nyssa, 
 bishops, and Macrina, virgin.
\switchcolumn*
\selectlanguage{latin}
Ancónæ sancti Marcellíni Epíscopi, qui urbem 
 illam (ut sanctus Gregórius Papa scribit) divína virtúte ab incéndio 
 liberávit.
\switchcolumn
\selectlanguage{english}
At Ancona, St. Marcellinus, bishop, who, according to St. Gregory, 
 miraculously delivered that city from destruction by fire.
\switchcolumn*
\selectlanguage{latin}
\end{paracol}


% ---- martyrology/mart01/mart0110.htm
\needspace{10\baselineskip}
\begin{paracol}{2}
\selectlanguage{latin}
\begin{center}{\color{gregoriocolor} Quarto Idus Januárii. 
 Luna\dots\ }\end{center}
\switchcolumn
\selectlanguage{english}
\begin{center}{\color{gregoriocolor} The Tenth Day of 
 January. The\dots\ Day of the Moon.}\end{center}
\end{paracol}

\noindent\begin{tabularx}{\linewidth}{*{19}{>{\centering\arraybackslash}X}}
 \textcolor{gregoriocolor}{a} & \textcolor{gregoriocolor}{b} & \textcolor{gregoriocolor}{c} & \textcolor{gregoriocolor}{d} & \textcolor{gregoriocolor}{e} & \textcolor{gregoriocolor}{f} & \textcolor{gregoriocolor}{g} & \textcolor{gregoriocolor}{h} & \textcolor{gregoriocolor}{i} & \textcolor{gregoriocolor}{k} & \textcolor{gregoriocolor}{l} & \textcolor{gregoriocolor}{m} & \textcolor{gregoriocolor}{n} & \textcolor{gregoriocolor}{p} & \textcolor{gregoriocolor}{q} & \textcolor{gregoriocolor}{r} & \textcolor{gregoriocolor}{s} & \textcolor{gregoriocolor}{t} & \textcolor{gregoriocolor}{u} \\
 11 & 12 & 13 & 14 & 15 & 16 & 17 & 18 & 19 & 20 & 21 & 22 & 23 & 24 & 25 & 26 & 27 & 28 & 29 \\
\end{tabularx}
\vspace{0.5\baselineskip}
\noindent\begin{tabularx}{\linewidth}{*{12}{>{\centering\arraybackslash}X}}
 \textcolor{gregoriocolor}{A} & \textcolor{gregoriocolor}{B} & \textcolor{gregoriocolor}{C} & \textcolor{gregoriocolor}{D} & \textcolor{gregoriocolor}{E} & F & \textcolor{gregoriocolor}{F} & \textcolor{gregoriocolor}{G} & \textcolor{gregoriocolor}{H} & \textcolor{gregoriocolor}{M} & \textcolor{gregoriocolor}{N} & \textcolor{gregoriocolor}{P} \\
 30 & 1 & 2 & 3 & 4 & 5 & 5 & 6 & 7 & 8 & 9 & 10 \\
\end{tabularx}

\begin{paracol}{2}
\selectlanguage{latin}
\lettrine[lines=2]{I}{n} Thebáide natális beáti Pauli, primi Eremítæ, 
 Confessóris, qui, a sextodécimo ætátis suæ anno usque ad centésimum décimum 
 tértium, solus in erémo permánsit; cujus ánimam, inter Apostolórum et 
 Prophetárum choros, ad cælum ferri ab Angelis sanctus Antónius vidit. 
 Ejus autem festívitas décimo octávo Kaléndas Februárii celebrátur.
\switchcolumn
\selectlanguage{english}
\lettrine[lines=2]{I}{n} Thebais, the birthday of St. Paul, the first hermit who lived alone in 
 the desert from the sixteenth to the one hundred and thirteenth year of his 
 age. His soul was seen by St. Anthony carried by angels among the 
 choirs of apostles and prophets. His feast is kept on the 15th of this 
 month.
\switchcolumn*
\selectlanguage{latin}
In Cypro beáti Nicánoris, qui fuit unus de 
 septem primis Diáconis; atque, grátia fídei et virtúte admirándus, 
 gloriosíssime coronátus est.
\switchcolumn
\selectlanguage{english}
In Cyprus, blessed Nicanor, one of the first seven deacons, a man of 
 admirable faith and virtue, who received the crown of glory.
\switchcolumn*
\selectlanguage{latin}
Romæ sancti Agathónis Papæ, qui, sanctitáte et 
 doctrína conspícuus, quiévit in pace.
\switchcolumn
\selectlanguage{english}
At Rome, Pope St. Agatho, who, by a holy death, concluded a life remarkable 
 for sanctity and learning.
\switchcolumn*
\selectlanguage{latin}
Bitúricis, in Aquitánia, sancti Willhélmi, 
 Epíscopi et Confessóris, signis et virtútibus clari; quem Honórius Papa 
 Tértius in Sanctórum cánonem adscrípsit.
\switchcolumn
\selectlanguage{english}
At Bourges in Aquitaine, St. William, archbishop and confessor, renowned for 
 miracles and virtues. He was canonized by Pope Honorius III.
\switchcolumn*
\selectlanguage{latin}
Medioláni sancti Joánnis Boni, Epíscopi et Confessóris.
\switchcolumn
\selectlanguage{english}
At Milan, St. John the Good, bishop and confessor.
\switchcolumn*
\selectlanguage{latin}
Constantinópoli sancti Marciáni Presbyteri.
\switchcolumn
\selectlanguage{english}
At Constantinople, St. Marcian, priest.
\switchcolumn*
\selectlanguage{latin}
In monastério Cuxanénsi, in Gállia, natális sancti Petri Urséoli 
 Confessóris, qui, antea Venetiárum Dux et deínde Mónachus ex Ordine sancti 
 Benedícti, pietáte et virtútibus cláruit.
\switchcolumn
\selectlanguage{english}
In the monastery of Cusani in France, the birthday of St. Peter Orsini, 
 confessor, previously the Doge of Venice and afterwards monk of the Order of 
 St. Benedict, renowned for piety and miracles.
\switchcolumn*
\selectlanguage{latin}
Arétii, in Túscia, beáti Gregórii Décimi, civis Placentíni, qui, ex 
 Archidiácono Leodiénsi Summus Póntifex renuntiátus, Concílium Lugdunénse 
 secúndum celebrávit, Græcísque ad unitátem 
 fídei recéptis, compósitis Christianórum dissídiis, Terræ Sanctæ 
 recuperatióne institúta, de universáli Ecclésia, quam sanctíssime gubernávit, 
 óptime méritus est.
\switchcolumn
\selectlanguage{english}
At Arezzo in Tuscany, blessed Gregory X, a native of Piacenza, who was 
 elected Sovereign Pontiff while he was archdeacon of Liege. He held 
 the second Council of Lyons, received the Greeks into the unity of the 
 Church, appeased discords among the Christians, made generous efforts for 
 the recovery of the Holy Land, and governed the Church in a most holy 
 manner.
\switchcolumn*
\selectlanguage{latin}
\end{paracol}


% ---- martyrology/mart01/mart0111.htm
\needspace{10\baselineskip}
\begin{paracol}{2}
\selectlanguage{latin}
\begin{center}{\color{gregoriocolor} Tértio Idus Januárii. 
 Luna\dots\ }\end{center}
\switchcolumn
\selectlanguage{english}
\begin{center}{\color{gregoriocolor} The Eleventh Day of 
 January. The\dots\ Day of the Moon.}\end{center}
\end{paracol}

\noindent\begin{tabularx}{\linewidth}{*{19}{>{\centering\arraybackslash}X}}
 \textcolor{gregoriocolor}{a} & \textcolor{gregoriocolor}{b} & \textcolor{gregoriocolor}{c} & \textcolor{gregoriocolor}{d} & \textcolor{gregoriocolor}{e} & \textcolor{gregoriocolor}{f} & \textcolor{gregoriocolor}{g} & \textcolor{gregoriocolor}{h} & \textcolor{gregoriocolor}{i} & \textcolor{gregoriocolor}{k} & \textcolor{gregoriocolor}{l} & \textcolor{gregoriocolor}{m} & \textcolor{gregoriocolor}{n} & \textcolor{gregoriocolor}{p} & \textcolor{gregoriocolor}{q} & \textcolor{gregoriocolor}{r} & \textcolor{gregoriocolor}{s} & \textcolor{gregoriocolor}{t} & \textcolor{gregoriocolor}{u} \\
 12 & 13 & 14 & 15 & 16 & 17 & 18 & 19 & 20 & 21 & 22 & 23 & 24 & 25 & 26 & 27 & 28 & 29 & 30 \\
\end{tabularx}
\vspace{0.5\baselineskip}
\noindent\begin{tabularx}{\linewidth}{*{12}{>{\centering\arraybackslash}X}}
 \textcolor{gregoriocolor}{A} & \textcolor{gregoriocolor}{B} & \textcolor{gregoriocolor}{C} & \textcolor{gregoriocolor}{D} & \textcolor{gregoriocolor}{E} & F & \textcolor{gregoriocolor}{F} & \textcolor{gregoriocolor}{G} & \textcolor{gregoriocolor}{H} & \textcolor{gregoriocolor}{M} & \textcolor{gregoriocolor}{N} & \textcolor{gregoriocolor}{P} \\
 1 & 2 & 3 & 4 & 5 & 6 & 6 & 7 & 8 & 9 & 10 & 11 \\
\end{tabularx}

\begin{paracol}{2}
\selectlanguage{latin}
\lettrine[lines=2]{R}{omæ} sancti Hygíni, Papæ et Mártyris; qui, in 
 persecutióne Antoníni, glorióse martyrium consummávit.
\switchcolumn
\selectlanguage{english}
\lettrine[lines=2]{A}{t} Rome, St. Hyginus, pope, who suffered a glorious martyrdom in the 
 persecution of Antoninus.
\switchcolumn*
\selectlanguage{latin}
Item Romæ natális sancti Melchíadis, Papæ et 
 Mártyris; qui multa, in persecutióne Maximiáni, passus est, atque, réddita 
 Ecclésiæ pace, quiévit in Dómino. Ipsíus autem festívitas quarto Idus 
 Decémbris celebrátur.
\switchcolumn
\selectlanguage{english}
Also at Rome, the birthday of St. Melchiades, who, having suffered much in 
 the persecution of Maximian, went to his rest in the Lord after peace 
 returned to the Church. His feast day is on the 10th of December.
\switchcolumn*
\selectlanguage{latin}
Firmi, in Picéno, sancti Alexándri, Epíscopi et Mártyris.
\switchcolumn
\selectlanguage{english}
At Fermo in Piceno, St. Alexander, bishop and martyr.
\switchcolumn*
\selectlanguage{latin}
Ambiáni, in Gállia, sancti Sálvii, Epíscopi et Mártyris.
\switchcolumn
\selectlanguage{english}
At Amiens in France, St. Salvius, bishop and martyr.
\switchcolumn*
\selectlanguage{latin}
In Africa beáti Sálvii Mártyris, in cujus natáli sanctus Augustínus sermónem 
 hábuit ad pópulum Carthaginénsem.
\switchcolumn
\selectlanguage{english}
In Africa, blessed Salvius, martyr, on whose birthday St. Augustine preached 
 to the people of Carthage.
\switchcolumn*
\selectlanguage{latin}
Alexandríæ sanctórum Mártyrum Petri, Sevéri et 
 Léucii.
\switchcolumn
\selectlanguage{english}
At Alexandria, the holy martyrs Peter, Severus and Leucius.
\switchcolumn*
\selectlanguage{latin}
Brundúsii sancti Léucii, Epíscopi et 
 Confessóris.
\switchcolumn
\selectlanguage{english}
At Brindisi, St. Leucius, bishop and confessor.
\switchcolumn*
\selectlanguage{latin}
In Judæa sancti Theodósii Cœnobiárchæ, in vico 
Cappadóciæ Magariásso nati; qui, multa passus pro fide cathólica, in pace 
tandem quiévit in eo monastério, quod ille super solitárium Hierosolymitánæ 
diœcésis montem exstrúxerat.
\switchcolumn
\selectlanguage{english}
In Judea, St. Theodosius, abbot, born in Cappadocia in the village of 
 Magarisso, who, after having endured great sufferings for the Catholic 
 faith, took his rest in peace at the monastery which he had erected on a 
 lonely hill in the diocese of Jerusalem.
\switchcolumn*
\selectlanguage{latin}
In Thebáide sancti Palǽmonis Abbátis, qui fuit 
 magíster sancti Pachómii.
\switchcolumn
\selectlanguage{english}
In Thebais, St. Palaemon, abbot, who was the teacher of St. Pachomius.
\switchcolumn*
\selectlanguage{latin}
Suppentóniæ, apud montem Soráctem, sancti 
 Anastásii Mónachi, et Sociórum; qui, divínitus vocáti, felíciter migravérunt 
 ad Dóminum.
\switchcolumn
\selectlanguage{english}
At Suppentonia, near Mount Soracte, St. Athanasius, monk, and his 
 companions, who were called by a voice from heaven to enter the kingdom of 
 God.
\switchcolumn*
\selectlanguage{latin}
Papíæ sanctæ Honorátæ Vírginis.
\switchcolumn
\selectlanguage{english}
At Pavia, St. Honorata, virgin.
\switchcolumn*
\selectlanguage{latin}
\end{paracol}


% ---- martyrology/mart01/mart0112.htm
\needspace{10\baselineskip}
\begin{paracol}{2}
\selectlanguage{latin}
\begin{center}{\color{gregoriocolor} Prídie Idus Januárii. 
 Luna\dots\ }\end{center}
\switchcolumn
\selectlanguage{english}
\begin{center}{\color{gregoriocolor} The Twelfth Day of 
 January. The\dots\ Day of the Moon.}\end{center}
\end{paracol}

\noindent\begin{tabularx}{\linewidth}{*{19}{>{\centering\arraybackslash}X}}
 \textcolor{gregoriocolor}{a} & \textcolor{gregoriocolor}{b} & \textcolor{gregoriocolor}{c} & \textcolor{gregoriocolor}{d} & \textcolor{gregoriocolor}{e} & \textcolor{gregoriocolor}{f} & \textcolor{gregoriocolor}{g} & \textcolor{gregoriocolor}{h} & \textcolor{gregoriocolor}{i} & \textcolor{gregoriocolor}{k} & \textcolor{gregoriocolor}{l} & \textcolor{gregoriocolor}{m} & \textcolor{gregoriocolor}{n} & \textcolor{gregoriocolor}{p} & \textcolor{gregoriocolor}{q} & \textcolor{gregoriocolor}{r} & \textcolor{gregoriocolor}{s} & \textcolor{gregoriocolor}{t} & \textcolor{gregoriocolor}{u} \\
 13 & 14 & 15 & 16 & 17 & 18 & 19 & 20 & 21 & 22 & 23 & 24 & 25 & 26 & 27 & 28 & 29 & 30 & 1 \\
\end{tabularx}
\vspace{0.5\baselineskip}
\noindent\begin{tabularx}{\linewidth}{*{12}{>{\centering\arraybackslash}X}}
 \textcolor{gregoriocolor}{A} & \textcolor{gregoriocolor}{B} & \textcolor{gregoriocolor}{C} & \textcolor{gregoriocolor}{D} & \textcolor{gregoriocolor}{E} & F & \textcolor{gregoriocolor}{F} & \textcolor{gregoriocolor}{G} & \textcolor{gregoriocolor}{H} & \textcolor{gregoriocolor}{M} & \textcolor{gregoriocolor}{N} & \textcolor{gregoriocolor}{P} \\
 2 & 3 & 4 & 5 & 6 & 7 & 7 & 8 & 9 & 10 & 11 & 12 \\
\end{tabularx}

\begin{paracol}{2}
\selectlanguage{latin}
\lettrine[lines=2]{R}{omæ} sanctæ Tatiánæ Mártyris, quæ, sub 
 Alexándro Imperatóre, uncis atque pectínibus laniáta, béstiis expósita et in 
 ignem missa, sed nil læsa, demum, gládio percússa, migrávit in cælum.
\switchcolumn
\selectlanguage{english}
\lettrine[lines=2]{A}{t} Rome, in the time of Emperor Alexander, St. Tatiana, martyr, who had her 
 flesh torn with iron hooks and combs, was thrown to the beasts and cast into 
 the fire, but, having received no injury, was beheaded, and thus went to 
 heaven.
\switchcolumn*
\selectlanguage{latin}
Constantinópoli sanctórum Tígrii Presbyteri, et Eutrópii Lectóris; qui, 
 Arcádii Imperatóris témpore, cum de incéndio quo Ecclésia princeps et 
 Senátus cúria conflagráverant, tamquam 
 per eos ad exsílium sancti Joánnis Chrysóstomi ulciscéndum excitáto, per 
 calúmniam accusáti essent, sub Præfécto urbis 
 Optáto, inánium deórum superstitióne implícito et Christiánæ religiónis 
 osóre, passi sunt.
\switchcolumn
\selectlanguage{english}
At Constantinople, the Saints Tigrius, priest, and Eutropius, lector, who, 
 in the time of Emperor Arcadius, were falsely accused of the fire which 
 destroyed the principal church and the senate building in order to avenge 
 the exile of St. John Chrysostom. They suffered under Optatus, prefect 
 of the city, who was given to the worship of false gods and a hatred for the 
 Christian religion.
\switchcolumn*
\selectlanguage{latin}
In Achája sancti Sátyri Mártyris, qui, cum ante 
 quoddam idólum transíret, in illud exsufflávit, signans sibi frontem, atque 
 statim idólum córruit; ob quam causam decollátus est.
\switchcolumn
\selectlanguage{english}
In Achaia, St. Satyrus, martyr. As he passed before an idol and 
 breathed upon it, making the sign of the cross upon his forehead, the idol 
 immediately fell to the ground; for this reason he was beheaded.
\switchcolumn*
\selectlanguage{latin}
Eódem die sancti Arcádii Mártyris, génere et miráculis clari.
\switchcolumn
\selectlanguage{english}
On the same day, St. Arcadius, martyr, illustrious for his noble extraction 
 and miracles.
\switchcolumn*
\selectlanguage{latin}
In Africa sanctórum Mártyrum Zótici, Rogáti, 
 Modésti, Cástuli, et corónæ mílitum quadragínta.
\switchcolumn
\selectlanguage{english}
In Africa, the holy martyrs Zoticus, Rogatus, Modestus, Castulus, and forty 
 soldiers gloriously crowned.
\switchcolumn*
\selectlanguage{latin}
Tíbure sancti Zótici Mártyris.
\switchcolumn
\selectlanguage{english}
At Tivoli, St. Zoticus, martyr.
\switchcolumn*
\selectlanguage{latin}
Ephesi pássio sanctórum quadragínta duórum Monachórum, qui ob cultum 
 sanctárum Imáginum, sub Constantíno Coprónymo, 
 sævíssime cruciáti, martyrium complevérunt.
\switchcolumn
\selectlanguage{english}
At Ephesus, under Constantine Copronymus, the passion of forty-two holy 
 monks, who endured martyrdom after being most cruelly tortured for the 
 defence of sacred images.
\switchcolumn*
\selectlanguage{latin}
Ravénnæ sancti Joánnis, Epíscopi et Confessóris.
\switchcolumn
\selectlanguage{english}
At Ravenna, St. John, bishop and confessor.
\switchcolumn*
\selectlanguage{latin}
Verónæ sancti Probi Epíscopi.
\switchcolumn
\selectlanguage{english}
At Verona, St. Probus, bishop.
\switchcolumn*
\selectlanguage{latin}
In Anglia sancti Benedícti, Abbátis et Confessóris.
\switchcolumn
\selectlanguage{english}
In England, St. Benedict Biscop, abbot and confessor.
\switchcolumn*
\selectlanguage{latin}
\end{paracol}


% ---- martyrology/mart01/mart0113.htm
\needspace{10\baselineskip}
\begin{paracol}{2}
\selectlanguage{latin}
\begin{center}{\color{gregoriocolor} Idibus Januárii. 
 Luna\dots\ }\end{center}
\switchcolumn
\selectlanguage{english}
\begin{center}{\color{gregoriocolor} The Thirteenth Day of 
 January. The\dots\ Day of the Moon.}\end{center}
\end{paracol}

\noindent\begin{tabularx}{\linewidth}{*{19}{>{\centering\arraybackslash}X}}
 \textcolor{gregoriocolor}{a} & \textcolor{gregoriocolor}{b} & \textcolor{gregoriocolor}{c} & \textcolor{gregoriocolor}{d} & \textcolor{gregoriocolor}{e} & \textcolor{gregoriocolor}{f} & \textcolor{gregoriocolor}{g} & \textcolor{gregoriocolor}{h} & \textcolor{gregoriocolor}{i} & \textcolor{gregoriocolor}{k} & \textcolor{gregoriocolor}{l} & \textcolor{gregoriocolor}{m} & \textcolor{gregoriocolor}{n} & \textcolor{gregoriocolor}{p} & \textcolor{gregoriocolor}{q} & \textcolor{gregoriocolor}{r} & \textcolor{gregoriocolor}{s} & \textcolor{gregoriocolor}{t} & \textcolor{gregoriocolor}{u} \\
 14 & 15 & 16 & 17 & 18 & 19 & 20 & 21 & 22 & 23 & 24 & 25 & 26 & 27 & 28 & 29 & 30 & 1 & 2 \\
\end{tabularx}
\vspace{0.5\baselineskip}
\noindent\begin{tabularx}{\linewidth}{*{12}{>{\centering\arraybackslash}X}}
 \textcolor{gregoriocolor}{A} & \textcolor{gregoriocolor}{B} & \textcolor{gregoriocolor}{C} & \textcolor{gregoriocolor}{D} & \textcolor{gregoriocolor}{E} & F & \textcolor{gregoriocolor}{F} & \textcolor{gregoriocolor}{G} & \textcolor{gregoriocolor}{H} & \textcolor{gregoriocolor}{M} & \textcolor{gregoriocolor}{N} & \textcolor{gregoriocolor}{P} \\
 3 & 4 & 5 & 6 & 7 & 8 & 8 & 9 & 10 & 11 & 12 & 13 \\
\end{tabularx}

\begin{paracol}{2}
\selectlanguage{latin}
\lettrine[lines=1]{O}{ctáva} Epiphaníæ Domini.
\switchcolumn
\selectlanguage{english}
\lettrine[lines=1]{T}{he} Octave of the Epiphany of our Lord.
\switchcolumn*
\selectlanguage{latin}
Pictávis, in Gállia, natális sancti Hilárii, Epíscopi et Confessóris; qui, 
 ob cathólicam fidem, quam strénue propugnávit, quadriénnio apud Phrygiam 
 relegátus, ibi, inter ália mirácula, mórtuum suscitávit. Eum Pius 
 Nonus, Póntifex Máximus, universális Ecclésiæ 
 Doctórem declarávit et confirmávit. Ipsíus autem festum sequénti die 
 celebrátur.
\switchcolumn
\selectlanguage{english}
At Poitiers in France, the birthday of St. Hilary, bishop and confessor of 
 the Catholic faith which he courageously defended, and for which he was 
 banished for four years to Phrygia, where, among other miracles, he raised a 
 man from the dead. Pius IX declared him a doctor of the Church. 
 His festival is celebrated tomorrow.
\switchcolumn*
\selectlanguage{latin}
Rhemis, in Gállia, item natális sancti Remígii, Epíscopi et Confessóris. 
 Hic gentem Francórum convértit ad Christum, Clodovéo, 
 ipsórum Rege, sacris baptísmatis undis et fídei sacraméntis initiáto; et, 
 cum annos plúrimos in Episcopátu explésset, sanctitáte et miraculórum glória 
 conspícuus, decéssit e vita. Ejus vero festívitas Kaléndis Octóbris 
 potíssimum recólitur, quo die sacrum ipsíus corpus translátum fuit.
\switchcolumn
\selectlanguage{english}
At Rheims in France, St. Remigius, bishop and and confessor, who converted 
 the Franks to Christ, and brought Clovis, their king, to the holy font of 
 baptism and instructed him in the mysteries of faith. After he had 
 been bishop for many years, and had distinguished himself by his sanctity 
 and the power of working miracles, he departed this life. His feast is 
 kept on the 1st of October, on which day his holy body was transferred.
\switchcolumn*
\selectlanguage{latin}
Romæ, via Lavicána, corónæ sanctórum mílitum 
 quadragínta, quas ipsi, sub Galliéno Imperatóre, pro veræ fídei confessióne 
 percípere meruérunt.
\switchcolumn
\selectlanguage{english}
At Rome, on the Via Lavicana, the crowning of forty holy soldiers, a reward 
 they merited by confessing the true faith under Emperor Gallienus.
\switchcolumn*
\selectlanguage{latin}
Córdubæ, in Hispánia, sanctórum Mártyrum 
 Gumesíndi Presbyteri, et Servidéi Mónachi.
\switchcolumn
\selectlanguage{english}
At Cordova, the holy martyrs Gumesind, priest, and Servideus, monk.
\switchcolumn*
\selectlanguage{latin}
In Sardínia sancti Potíti Mártyris, qui, sub Antoníno Imperatóre et Gelásio 
 Præside, multa passus, demum gládio martyrium 
 consecútus est.
\switchcolumn
\selectlanguage{english}
In Sardinia, St. Potitus, martyr, who, having suffered much under Emperor 
 Antoninus and the governor Gelasius, was at last put to death by the sword.
\switchcolumn*
\selectlanguage{latin}
Singidóni, in Mysia superióre, sanctórum 
 Mártyrum Hérmyli et Stratoníci, qui, post sæva torménta, sub Licínio 
 Imperatóre, in Istrum flumen demérsi sunt.
\switchcolumn
\selectlanguage{english}
At Belgrade in Serbia, the holy martyrs Hermylus and Stratonicus, who were 
 severely tormented under Emperor Licinius, and then drowned in the river 
 Danube.
\switchcolumn*
\selectlanguage{latin}
Cæsaréæ, in Cappadócia, sancti Leóntii Epíscopi, 
 qui, sub Licínio, advérsus Gentíles, et, sub Constantíno, advérsus Ariános 
 plúrimum decertávit.
\switchcolumn
\selectlanguage{english}
At Caesarea in Cappadocia, St. Leontius, bishop, who fought strongly against 
 the heathens in the reign of Licinius, and against the Arians in the reign 
 of Constantine.
\switchcolumn*
\selectlanguage{latin}
Tréviris sancti Agrítii Epíscopi.
\switchcolumn
\selectlanguage{english}
At Treves, St. Agritius, bishop.
\switchcolumn*
\selectlanguage{latin}
In Versíaco monastério, in Gállia, sancti 
 Vivéntii Confessóris.
\switchcolumn
\selectlanguage{english}
In the monastery of Verzy in France, St. Viventius, confessor.
\switchcolumn*
\selectlanguage{latin}
Amaséæ, in Ponto, sanctæ Gláphyræ Vírginis.
\switchcolumn
\selectlanguage{english}
At Amasea in Pontus, St. Glaphyra, virgin.
\switchcolumn*
\selectlanguage{latin}
Medioláni, in cœnóbio sanctæ Marthæ, Beátæ 
 Verónicæ de Binásco Vírginis, ex Ordine sancti Augustíni.
\switchcolumn
\selectlanguage{english}
At Milan, in the monastery of St. Martha, blessed Veronica of Binasco, 
 virgin, of the Order of St. Augustine.
\switchcolumn*
\selectlanguage{latin}
\end{paracol}


% ---- martyrology/mart01/mart0114.htm
\needspace{10\baselineskip}
\begin{paracol}{2}
\selectlanguage{latin}
\begin{center}{\color{gregoriocolor} Décimo nono Kaléndas Februárii. 
 Luna\dots\ }\end{center}
\switchcolumn
\selectlanguage{english}
\begin{center}{\color{gregoriocolor} The Fourteenth Day of 
 January. The\dots\ Day of the Moon.}\end{center}
\end{paracol}

\noindent\begin{tabularx}{\linewidth}{*{19}{>{\centering\arraybackslash}X}}
 \textcolor{gregoriocolor}{a} & \textcolor{gregoriocolor}{b} & \textcolor{gregoriocolor}{c} & \textcolor{gregoriocolor}{d} & \textcolor{gregoriocolor}{e} & \textcolor{gregoriocolor}{f} & \textcolor{gregoriocolor}{g} & \textcolor{gregoriocolor}{h} & \textcolor{gregoriocolor}{i} & \textcolor{gregoriocolor}{k} & \textcolor{gregoriocolor}{l} & \textcolor{gregoriocolor}{m} & \textcolor{gregoriocolor}{n} & \textcolor{gregoriocolor}{p} & \textcolor{gregoriocolor}{q} & \textcolor{gregoriocolor}{r} & \textcolor{gregoriocolor}{s} & \textcolor{gregoriocolor}{t} & \textcolor{gregoriocolor}{u} \\
 15 & 16 & 17 & 18 & 19 & 20 & 21 & 22 & 23 & 24 & 25 & 26 & 27 & 28 & 29 & 30 & 1 & 2 & 3 \\
\end{tabularx}
\vspace{0.5\baselineskip}
\noindent\begin{tabularx}{\linewidth}{*{12}{>{\centering\arraybackslash}X}}
 \textcolor{gregoriocolor}{A} & \textcolor{gregoriocolor}{B} & \textcolor{gregoriocolor}{C} & \textcolor{gregoriocolor}{D} & \textcolor{gregoriocolor}{E} & F & \textcolor{gregoriocolor}{F} & \textcolor{gregoriocolor}{G} & \textcolor{gregoriocolor}{H} & \textcolor{gregoriocolor}{M} & \textcolor{gregoriocolor}{N} & \textcolor{gregoriocolor}{P} \\
 4 & 5 & 6 & 7 & 8 & 9 & 9 & 10 & 11 & 12 & 13 & 14 \\
\end{tabularx}

\begin{paracol}{2}
\selectlanguage{latin}
\lettrine[lines=2]{S}{ancti} Hilárii, Epíscopi Pictaviénsis, Confessóris et Ecclésiæ 
 Doctóris; qui prídie hujus diéi evolávit in cælum.
\switchcolumn
\selectlanguage{english}
\lettrine[lines=2]{S}{t.} Hilary, bishop of Poitiers, confessor and doctor of the Church, who 
 entered heaven on the thirteenth day of this month.
\switchcolumn*
\selectlanguage{latin}
Nolæ, in Campánia, natális sancti Felícis 
 Presbyteri, qui (ut sanctus Paulínus Epíscopus scribit), cum a 
 persecutóribus post torménta in cárcerem missus esset, et cóchleis ac 
 téstulis vinctus superpósitus jacéret, nocte ab Angelo solútus atque edúctus 
 fuit; póstmodum vero, cessánte persecutióne, ibídem, cum multos ad Christi 
 fidem exémplo vitæ ac doctrína convertísset, clarus miráculis quiévit in 
 pace.
\switchcolumn
\selectlanguage{english}
At Nola in Campania, the birthday of St. Felix, priest, who (as is related 
 by bishop St. Paulinus), after being subjected to torments by the 
 persecutors, was cast into prison, bound hand and foot, and extended on 
 shells and broken earthenware. In the night, however, his bonds were 
 loosened and he was delivered by an angel. The persecution over, he 
 brought many to the faith of Christ by his exemplary life and teaching, and, 
 renowned for miracles, rested in peace.
\switchcolumn*
\selectlanguage{latin}
In Judæa sancti Malachíæ Prophétæ.
\switchcolumn
\selectlanguage{english}
In Judea, St. Malachy, prophet.
\switchcolumn*
\selectlanguage{latin}
In monte Sina sanctórum trigínta octo Monachórum, a Saracénis ob Christi 
 fidem interfectórum.
\switchcolumn
\selectlanguage{english}
On Mount Sinai, thirty-eight holy monks killed by the Saracens for the faith 
 of Christ.
\switchcolumn*
\selectlanguage{latin}
In Rhaíthi regióne, in Ægypto, sanctórum 
 quadragínta trium Monachórum, qui, pro Christiána religióne, a Blémmiis 
 occísi sunt.
\switchcolumn
\selectlanguage{english}
In Egypt, in the district of Raithy, forty-three holy monks, who were put to 
 death by the Blemmians for the Christian religion.
\switchcolumn*
\selectlanguage{latin}
Medioláni sancti Dátii, Epíscopi et Confessóris; cujus méminit beátus 
 Gregórius Papa.
\switchcolumn
\selectlanguage{english}
At Milan, St. Datius, bishop and confessor, mentioned by pope St. Gregory.
\switchcolumn*
\selectlanguage{latin}
In Africa sancti Euphrásii Epíscopi.
\switchcolumn
\selectlanguage{english}
In Africa, St. Euphrasius, bishop.
\switchcolumn*
\selectlanguage{latin}
Neocæsaréæ, in Ponto, sanctæ Macrínæ, discípulæ 
 beáti Gregórii Thaumatúrgi, et áviæ sancti Basilíi, quæ eúndem Basilíum 
 educávit in fide.
\switchcolumn
\selectlanguage{english}
At Neocaesarea in Pontus, St. Macrina, disciple of St. Gregory the 
 Wonder-Worker, and grandmother of St. Basil, whom she educated in the 
 Christian faith.
\switchcolumn*
\selectlanguage{latin}
\end{paracol}


% ---- martyrology/mart01/mart0115.htm
\needspace{10\baselineskip}
\begin{paracol}{2}
\selectlanguage{latin}
\begin{center}{\color{gregoriocolor} Décimo octávo Kaléndas Februárii. 
 Luna\dots\ }\end{center}
\switchcolumn
\selectlanguage{english}
\begin{center}{\color{gregoriocolor} The Fifteenth Day of 
 January. The\dots\ Day of the Moon.}\end{center}
\end{paracol}

\noindent\begin{tabularx}{\linewidth}{*{19}{>{\centering\arraybackslash}X}}
 \textcolor{gregoriocolor}{a} & \textcolor{gregoriocolor}{b} & \textcolor{gregoriocolor}{c} & \textcolor{gregoriocolor}{d} & \textcolor{gregoriocolor}{e} & \textcolor{gregoriocolor}{f} & \textcolor{gregoriocolor}{g} & \textcolor{gregoriocolor}{h} & \textcolor{gregoriocolor}{i} & \textcolor{gregoriocolor}{k} & \textcolor{gregoriocolor}{l} & \textcolor{gregoriocolor}{m} & \textcolor{gregoriocolor}{n} & \textcolor{gregoriocolor}{p} & \textcolor{gregoriocolor}{q} & \textcolor{gregoriocolor}{r} & \textcolor{gregoriocolor}{s} & \textcolor{gregoriocolor}{t} & \textcolor{gregoriocolor}{u} \\
 16 & 17 & 18 & 19 & 20 & 21 & 22 & 23 & 24 & 25 & 26 & 27 & 28 & 29 & 30 & 1 & 2 & 3 & 4 \\
\end{tabularx}
\vspace{0.5\baselineskip}
\noindent\begin{tabularx}{\linewidth}{*{12}{>{\centering\arraybackslash}X}}
 \textcolor{gregoriocolor}{A} & \textcolor{gregoriocolor}{B} & \textcolor{gregoriocolor}{C} & \textcolor{gregoriocolor}{D} & \textcolor{gregoriocolor}{E} & F & \textcolor{gregoriocolor}{F} & \textcolor{gregoriocolor}{G} & \textcolor{gregoriocolor}{H} & \textcolor{gregoriocolor}{M} & \textcolor{gregoriocolor}{N} & \textcolor{gregoriocolor}{P} \\
 5 & 6 & 7 & 8 & 9 & 10 & 10 & 11 & 12 & 13 & 14 & 15 \\
\end{tabularx}

\begin{paracol}{2}
\selectlanguage{latin}
\lettrine[lines=2]{S}{ancti} Pauli, primi Eremítæ, Confessóris; qui 
 quarto Idus Januárii inter beatórum ágmina translátus fuit.
\switchcolumn
\selectlanguage{english}
\lettrine[lines=2]{S}{t.} Paul, the first hermit, who was carried to the home of the blessed on 
 the tenth of this month.
\switchcolumn*
\selectlanguage{latin}
In território Andegavénsi beáti Mauri Abbátis, qui fuit discípulus sancti 
 Benedícti; et, hujus disciplínis usque ab infántia erudítus, quantum in eis 
 profécerit, inter ália quæ apud eum pósitus 
 gessit (res nova et post Petrum fere inusitáti), pédibus super aquas 
 incédens patefécit. In Gállias inde ab ipso Benedícto diréctus, ibi, 
 constrúcto célebri monastério, cui quadragínta annis præfuit, miraculórum 
 glória clarus, in pace quiévit.
\switchcolumn
\selectlanguage{english}
In the diocese of Angers, blessed Maurus, abbot and disciple of St. Benedict. 
 Beginning his discipline in infancy, he made great progress with so able a 
 master, for while he was still under the saint's instruction he miraculously 
 walked upon the water, a prodigy unheard of since the days of St. Peter. 
 Sent later to France by St. Benedict, he built a famous monastery, which he 
 governed for forty years, and after performing striking miracles, he rested 
 in peace.
\switchcolumn*
\selectlanguage{latin}
In Judæa sanctórum Hábacuc et Michǽæ 
 Prophetárum, quorum corpora, sub Theodósio senióre, divína revelatióne sunt 
 repérta.
\switchcolumn
\selectlanguage{english}
In Judea, the holy prophets Habakkuk and Micah, whose bodies were found by 
 divine revelation in the days of Theodosius the Elder.
\switchcolumn*
\selectlanguage{latin}
Cárali, in Sardínia, sancti Ephísii Mártyris, 
 qui, in persecutióne Diocletiáni, sub Flaviáno Júdice, plúrimis torméntis 
 divína virtúte superátis, demum, abscíssis cervícibus, victor migrávit in 
 cælum.
\switchcolumn
\selectlanguage{english}
At Cagliari in Sardinia, St. Ephisius, martyr, who, in the persecution of 
 Diocletian and under the judge Flavian, having, by the assistance of God, 
 overcome many torments, was beheaded and ascended to heaven.
\switchcolumn*
\selectlanguage{latin}
Anágniæ 
 sanctæ Secundínæ, Vírginis et Mártyris; quæ sub Décio Imperatóre passa est.
\switchcolumn
\selectlanguage{english}
At Anagni, St. Secundina, virgin and martyr, who suffered under Emperor 
 Decius.
\switchcolumn*
\selectlanguage{latin}
Nolæ, in Campánia, sancti Máximi Epíscopi.
\switchcolumn
\selectlanguage{english}
At Nola in Campania, St. Maximus, bishop.
\switchcolumn*
\selectlanguage{latin}
Arvérnis, in Gállia, sancti Boníti, Epíscopi et Confessóris.
\switchcolumn
\selectlanguage{english}
In Auvergne in France, St. Bonitus, bishop and confessor.
\switchcolumn*
\selectlanguage{latin}
In Ægypto sancti Macárii Abbátis, qui fuit 
 discípulus beáti Antónii, ac vita et miráculis celebérrimus éxstitit.
\switchcolumn
\selectlanguage{english}
In Egypt, St. Macarius, abbot, disciple of St. Anthony, very celebrated for 
 his life and miracles.
\switchcolumn*
\selectlanguage{latin}
Alexandríæ beáti Isidóri, sanctitáte vitæ, fide 
 et miráculis clari.
\switchcolumn
\selectlanguage{english}
At Alexandria, blessed Isidore, renowned for holiness of life, faith, and 
 miracles.
\switchcolumn*
\selectlanguage{latin}
Constantinópoli sancti Joánnis Calybítæ, qui 
 aliquándiu in ángulo domus patérnæ, deínde in tugúrio, ignótus paréntibus, 
 habitávit; a quibus in morte ágnitus, miráculis cláruit. Ipsíus corpus 
 póstea Romam translátum, et in Insulæ Tiberínæ Ecclésia, in ejus honórem 
 erécta, collocátum est.
\switchcolumn
\selectlanguage{english}
At Constantinople, St. John Calybita. For some time living unknown to 
 his parents in a corner of their house, and later in a hut on an island in 
 the Tiber, he was recognized by them only at his death. Being renowned 
 for miracles, his body was afterwards taken to Rome and buried on the Island 
 in the Tiber, where a church was subsequently erected in his honour.
\switchcolumn*
\selectlanguage{latin}
\end{paracol}


% ---- martyrology/mart01/mart0116.htm
\needspace{10\baselineskip}
\begin{paracol}{2}
\selectlanguage{latin}
\begin{center}{\color{gregoriocolor} Décimo séptimo Kaléndas Februárii. 
 Luna\dots\ }\end{center}
\switchcolumn
\selectlanguage{english}
\begin{center}{\color{gregoriocolor} The Sixteenth Day of 
 January. The\dots\ Day of the Moon.}\end{center}
\end{paracol}

\noindent\begin{tabularx}{\linewidth}{*{19}{>{\centering\arraybackslash}X}}
 \textcolor{gregoriocolor}{a} & \textcolor{gregoriocolor}{b} & \textcolor{gregoriocolor}{c} & \textcolor{gregoriocolor}{d} & \textcolor{gregoriocolor}{e} & \textcolor{gregoriocolor}{f} & \textcolor{gregoriocolor}{g} & \textcolor{gregoriocolor}{h} & \textcolor{gregoriocolor}{i} & \textcolor{gregoriocolor}{k} & \textcolor{gregoriocolor}{l} & \textcolor{gregoriocolor}{m} & \textcolor{gregoriocolor}{n} & \textcolor{gregoriocolor}{p} & \textcolor{gregoriocolor}{q} & \textcolor{gregoriocolor}{r} & \textcolor{gregoriocolor}{s} & \textcolor{gregoriocolor}{t} & \textcolor{gregoriocolor}{u} \\
 17 & 18 & 19 & 20 & 21 & 22 & 23 & 24 & 25 & 26 & 27 & 28 & 29 & 30 & 1 & 2 & 3 & 4 & 5 \\
\end{tabularx}
\vspace{0.5\baselineskip}
\noindent\begin{tabularx}{\linewidth}{*{12}{>{\centering\arraybackslash}X}}
 \textcolor{gregoriocolor}{A} & \textcolor{gregoriocolor}{B} & \textcolor{gregoriocolor}{C} & \textcolor{gregoriocolor}{D} & \textcolor{gregoriocolor}{E} & F & \textcolor{gregoriocolor}{F} & \textcolor{gregoriocolor}{G} & \textcolor{gregoriocolor}{H} & \textcolor{gregoriocolor}{M} & \textcolor{gregoriocolor}{N} & \textcolor{gregoriocolor}{P} \\
 6 & 7 & 8 & 9 & 10 & 11 & 11 & 12 & 13 & 14 & 15 & 16 \\
\end{tabularx}

\begin{paracol}{2}
\selectlanguage{latin}
\lettrine[lines=2]{R}{omæ,} via Salária, natális sancti Marcélli 
 Primi, Papæ et Mártyris; qui, ob cathólicæ fídei confessiónem, jubénte 
 Maxéntio tyránno, primo cæsus est fústibus, deínde ad servítium animálium 
 cum custódia pública deputátus, et ibídem, serviéndo indútus amíctu cilícino, 
 defúnctus est.
\switchcolumn
\selectlanguage{english}
\lettrine[lines=2]{A}{t} Rome, on the Salarian Way, the birthday of Pope St. Marcellus I, a martyr 
 for the confession of the Catholic faith. By command of the tyrant 
 Maxentius he was beaten with clubs, then sent to take care of animals, with 
 a guard to watch him. In this servile office, dressed in haircloth, he 
 departed this life.
\switchcolumn*
\selectlanguage{latin}
Marróchii, in Africa, pássio sanctórum quinque Protomártyrum Ordinis Minórum, 
 scílicet Berárdi, Petri atque Othónis Sacerdótum, Accúrsii et Adjúti 
 Laicórum; qui, ob Christiánæ fídei 
 prædicatiónem ac Mahuméticæ reprobatiónem legis, post vária torménta et 
 ludíbria, a Saracenórum Rege, scissis gládio capítibus, enecáti sunt.
\switchcolumn
\selectlanguage{english}
In Morocco in Africa, the martyrdom of the five Protomartyrs of the Order of 
 Friars Minor, Berard, Peter, and Otto who were priests, and Accursius and 
 Adjutus who were lay brothers. For preaching the Catholic faith, and 
 because of their hatred of the Mohammedan Law, after various torments and 
 mockeries by the Saracen king, they were beheaded.
\switchcolumn*
\selectlanguage{latin}
Rhinocolúræ, in Ægypto, sancti Melæ Epíscopi, 
 qui, sub Valénte exsílium et ália grávia pro fide cathólica passus, in pace 
 quiévit.
\switchcolumn
\selectlanguage{english}
At Rhinocolura in Egypt, the holy bishop St. Melas, who rested in peace 
 after suffering exile and other painful trials for the Catholic faith during 
 the reign of Emperor Valens.
\switchcolumn*
\selectlanguage{latin}
Areláte, in Gállia, sancti Honoráti, Epíscopi et Confessóris; cujus vita tam 
 doctrína quam miráculis fuit illústris.
\switchcolumn
\selectlanguage{english}
At Arles in France, St. Honoratus, bishop and confessor, whose life was 
 renowned for learning and for miracles.
\switchcolumn*
\selectlanguage{latin}
Opitérgii, in Venetórum fínibus, sancti Titiáni, Epíscopi et Confessóris.
\switchcolumn
\selectlanguage{english}
At Oderzo near Venice, St. Titian, bishop and confessor.
\switchcolumn*
\selectlanguage{latin}
Fundis, in Látio, sancti Honoráti Abbátis, cujus méminit beátus Gregórius 
 Papa.
\switchcolumn
\selectlanguage{english}
At Fondi in Lazio, St. Honoratus, abbot, mentioned by Pope St. Gregory.
\switchcolumn*
\selectlanguage{latin}
In castro cui nomen Macériæ, ad Altéjam flúvium, 
 in Gállia, sancti Furséi Confessóris; cujus corpus ad monastérium Perónæ 
 póstmodum translátum est.
\switchcolumn
\selectlanguage{english}
At Froheins, in the diocese of Amiens in France, St. Fursey, confessor, 
 whose body was afterwards transferred to the monastery of Peronne.
\switchcolumn*
\selectlanguage{latin}
Romæ sanctæ Priscíllæ, quæ se súaque pio 
 Mártyrum obséquio mancipávit.
\switchcolumn
\selectlanguage{english}
At Rome, St. Priscilla, who devoted herself and her goods to the service of 
 the martyrs.
\switchcolumn*
\selectlanguage{latin}
\end{paracol}


% ---- martyrology/mart01/mart0117.htm
\needspace{10\baselineskip}
\begin{paracol}{2}
\selectlanguage{latin}
\begin{center}{\color{gregoriocolor} Décimo sexto Kaléndas Februárii. 
 Luna\dots\ }\end{center}
\switchcolumn
\selectlanguage{english}
\begin{center}{\color{gregoriocolor} The Seventeenth Day of 
 January. The\dots\ Day of the Moon.}\end{center}
\end{paracol}

\noindent\begin{tabularx}{\linewidth}{*{19}{>{\centering\arraybackslash}X}}
 \textcolor{gregoriocolor}{a} & \textcolor{gregoriocolor}{b} & \textcolor{gregoriocolor}{c} & \textcolor{gregoriocolor}{d} & \textcolor{gregoriocolor}{e} & \textcolor{gregoriocolor}{f} & \textcolor{gregoriocolor}{g} & \textcolor{gregoriocolor}{h} & \textcolor{gregoriocolor}{i} & \textcolor{gregoriocolor}{k} & \textcolor{gregoriocolor}{l} & \textcolor{gregoriocolor}{m} & \textcolor{gregoriocolor}{n} & \textcolor{gregoriocolor}{p} & \textcolor{gregoriocolor}{q} & \textcolor{gregoriocolor}{r} & \textcolor{gregoriocolor}{s} & \textcolor{gregoriocolor}{t} & \textcolor{gregoriocolor}{u} \\
 18 & 19 & 20 & 21 & 22 & 23 & 24 & 25 & 26 & 27 & 28 & 29 & 30 & 1 & 2 & 3 & 4 & 5 & 6 \\
\end{tabularx}
\vspace{0.5\baselineskip}
\noindent\begin{tabularx}{\linewidth}{*{12}{>{\centering\arraybackslash}X}}
 \textcolor{gregoriocolor}{A} & \textcolor{gregoriocolor}{B} & \textcolor{gregoriocolor}{C} & \textcolor{gregoriocolor}{D} & \textcolor{gregoriocolor}{E} & F & \textcolor{gregoriocolor}{F} & \textcolor{gregoriocolor}{G} & \textcolor{gregoriocolor}{H} & \textcolor{gregoriocolor}{M} & \textcolor{gregoriocolor}{N} & \textcolor{gregoriocolor}{P} \\
 7 & 8 & 9 & 10 & 11 & 12 & 12 & 13 & 14 & 15 & 16 & 17 \\
\end{tabularx}

\begin{paracol}{2}
\selectlanguage{latin}
\lettrine[lines=2]{I}{n} Thebáide sancti Antónii Abbátis, qui, multórum Monachórum Pater, vita et 
 miráculis præclaríssimus vixit; cujus gesta 
 sanctus Athanásius insígni volúmine prosecútus est. Ejus autem sacrum 
 corpus, sub Justiniáno Imperatóre, divína revelatióne repértum et 
 Alexandríam delátum, in Ecclésia sancti Joánnis Baptístæ humátum fuit.
\switchcolumn
\selectlanguage{english}
\lettrine[lines=2]{I}{n} Thebais, St. Anthony, abbot and spiritual guide of many monks, who was 
 most celebrated for his life and miracles of which St. Athanasius has 
 written a detailed account. His holy body was found by a divine 
 revelation during the reign of Emperor Justinian and brought to Alexandria, 
 where it was buried in the church of St. John Baptist.
\switchcolumn*
\selectlanguage{latin}
Apud Língonas, in Gállia, sanctórum 
 tergeminórum Speusíppi, Eleusíppi et Meleusíppi; qui, cum ávia sua Leonílla, 
 martyrio coronáti sunt, témpore Marci Aurélii Imperatóris.
\switchcolumn
\selectlanguage{english}
At Langres in France, in the time of Marcus Aurelius, the Saints Speusippus, 
 Eleusippus, and Meleusippus, born at one birth, were crowned with martyrdom 
 together with their grandmother Leonilla.
\switchcolumn*
\selectlanguage{latin}
Apud Bitúricas, in Aquitánia, deposítio sancti 
 Sulpícii Epíscopi, cognoménto Pii, cujus vita et mors pretiósa gloriósis 
 miráculis commendátur.
\switchcolumn
\selectlanguage{english}
At Bourges in Aquitaine, the death of the holy Bishop Sulpice, surnamed Pius, whose life 
 and precious death were approved by glorious miracles.
\switchcolumn*
\selectlanguage{latin}
Romæ, in monastério sancti Andréæ, beatórum 
 Monachórum Antónii, Méruli et Joánnis, de quibus scribit sanctus Gregórius 
 Papa.
\switchcolumn
\selectlanguage{english}
At Rome, in the monastery of St. Andrew, the blessed monks Anthony, Merulus, 
 and John, of whom Pope St. Gregory speaks in his writings.
\switchcolumn*
\selectlanguage{latin}
In fínibus Edessénæ regiónis, in Mesopotámia, sancti Juliáni Eremítæ, cognoménto Sabæ, qui, Valéntis Imperatóris témpore, 
 fidem cathólicam, Antiochíæ ferme collápsam, virtúte miraculórum eréxit.
\switchcolumn
\selectlanguage{english}
At Edessa in Mesopotamia, in the time of Emperor Valens, St. Julian Sabas 
 the Elder, who miraculously restored the Catholic faith at Antioch, although 
 it was almost destroyed in that city.
\switchcolumn*
\selectlanguage{latin}
Romæ Invéntio sanctórum Mártyrum Diodóri 
 Presbyteri, Mariáni Diáconi, et Sociórum; qui, sancto Stéphano Papa 
 Ecclésiam Dei regénte, martyrium Kaléndis Decémbris sunt assecúti.
\switchcolumn
\selectlanguage{english}
At Rome, the finding of the holy martyrs Diodorus, priest, and Marian, 
 deacon, and their companions. They suffered martyrdom on the 1st of 
 December during the pontificate of Pope St. Stephen.
\switchcolumn*
\selectlanguage{latin}
\end{paracol}


% ---- martyrology/mart01/mart0118.htm
\needspace{10\baselineskip}
\begin{paracol}{2}
\selectlanguage{latin}
\begin{center}{\color{gregoriocolor} Quintodécimo Kaléndas Februárii. 
 Luna\dots\ }\end{center}
\switchcolumn
\selectlanguage{english}
\begin{center}{\color{gregoriocolor} The Eighteenth Day of 
 January. The\dots\ Day of the Moon.}\end{center}
\end{paracol}

\noindent\begin{tabularx}{\linewidth}{*{19}{>{\centering\arraybackslash}X}}
 \textcolor{gregoriocolor}{a} & \textcolor{gregoriocolor}{b} & \textcolor{gregoriocolor}{c} & \textcolor{gregoriocolor}{d} & \textcolor{gregoriocolor}{e} & \textcolor{gregoriocolor}{f} & \textcolor{gregoriocolor}{g} & \textcolor{gregoriocolor}{h} & \textcolor{gregoriocolor}{i} & \textcolor{gregoriocolor}{k} & \textcolor{gregoriocolor}{l} & \textcolor{gregoriocolor}{m} & \textcolor{gregoriocolor}{n} & \textcolor{gregoriocolor}{p} & \textcolor{gregoriocolor}{q} & \textcolor{gregoriocolor}{r} & \textcolor{gregoriocolor}{s} & \textcolor{gregoriocolor}{t} & \textcolor{gregoriocolor}{u} \\
 19 & 20 & 21 & 22 & 23 & 24 & 25 & 26 & 27 & 28 & 29 & 30 & 1 & 2 & 3 & 4 & 5 & 6 & 7 \\
\end{tabularx}
\vspace{0.5\baselineskip}
\noindent\begin{tabularx}{\linewidth}{*{12}{>{\centering\arraybackslash}X}}
 \textcolor{gregoriocolor}{A} & \textcolor{gregoriocolor}{B} & \textcolor{gregoriocolor}{C} & \textcolor{gregoriocolor}{D} & \textcolor{gregoriocolor}{E} & F & \textcolor{gregoriocolor}{F} & \textcolor{gregoriocolor}{G} & \textcolor{gregoriocolor}{H} & \textcolor{gregoriocolor}{M} & \textcolor{gregoriocolor}{N} & \textcolor{gregoriocolor}{P} \\
 8 & 9 & 10 & 11 & 12 & 13 & 13 & 14 & 15 & 16 & 17 & 18 \\
\end{tabularx}

\begin{paracol}{2}
\selectlanguage{latin}
\lettrine[lines=2]{C}{áthedra} sancti Petri Apóstoli, qua primum Romæ 
 sedit.
\switchcolumn
\selectlanguage{english}
\lettrine[lines=2]{T}{he} Chair of St. Peter the Apostle, who established the Holy See at Rome.
\switchcolumn*
\selectlanguage{latin}
Ibídem pássio sanctæ Priscæ, Vírginis et 
 Mártyris; quæ sub Cláudio Imperatóre, post multa torménta, martyrio coronáta 
 est.
\switchcolumn
\selectlanguage{english}
In the same place, under Emperor Claudius, the passion of St. Prisca, virgin 
 and martyr, who, after undergoing many torments, was crowned with martyrdom.
\switchcolumn*
\selectlanguage{latin}
In Ponto natális sanctórum Mártyrum Moséi et 
 Ammónii, qui, cum essent mílites, primo ad metálla damnáti sunt, ac 
 novíssime igni tráditi.
\switchcolumn
\selectlanguage{english}
In Pontus, the birthday of the holy martyrs Mosseus and Ammonius, soldiers, 
 who were first condemned to work in the metal mines, then cast into the 
 fire.
\switchcolumn*
\selectlanguage{latin}
Ibídem sancti Athenógenis, antíqui Theólogi, 
 qui, per ignem consummatúrus martyrium, hymnum lætus cécinit, quem et 
 discípulis scriptum relíquit.
\switchcolumn
\selectlanguage{english}
In the same country, St. Athenogenes, an aged divine, who, on the point of 
 being martyred by fire, joyfully sang a hymn, which he left in writing to 
 his disciples.
\switchcolumn*
\selectlanguage{latin}
Turónis, in Gállia, sancti Volusiáni Epíscopi, qui, a Gothis captus, in 
 exsílio spíritum Deo réddidit.
\switchcolumn
\selectlanguage{english}
At Tours in France, St. Volusian, bishop, who was made captive by the Goths, 
 and in exile gave up his soul unto God.
\switchcolumn*
\selectlanguage{latin}
In monastério Lutrénsi, in Burgúndia, sancti Deícolæ 
 Abbátis, qui, natióne Hibérnus, discípulus fuit beáti Columbáni.
\switchcolumn
\selectlanguage{english}
In the monastery of Lure in Burgundy, St. Deicola, abbot, a native of 
 Ireland and a disciple of St. Columban.
\switchcolumn*
\selectlanguage{latin}
Turónis, in Gállia, sancti Leobárdi reclúsi, qui mira abstinéntia et 
 humilitáte refúlsit.
\switchcolumn
\selectlanguage{english}
At Tours in France, St. Leobard, anchoret, a man of wonderful abstinence and 
 humility.
\switchcolumn*
\selectlanguage{latin}
Novocómi sanctæ Liberátæ Vírginis.
\switchcolumn
\selectlanguage{english}
At Como, St. Liberata, virgin.
\switchcolumn*
\selectlanguage{latin}
Budæ, in Hungária, sanctæ Margarítæ Vírginis, e 
 régia Arpadénsium família, Ordinis sancti Domínici Moniális, virtúte 
 castitátis et arctíssima pæniténtia insígnis, quam Pius Duodécimus, Póntifex 
 Máximus, sanctárum Vírginum catálogo adscrípsit.
\switchcolumn
\selectlanguage{english}
At Buda in Hungary, St. Margaret, virgin, from the royal family of Arpad, 
 and a nun of the Order of St. Dominic, endued with the virtues of chastity 
 and a burning penitence. The Supreme Pontiff, Pius XII, added her to 
 the list of holy virgins.
\switchcolumn*
\selectlanguage{latin}
\end{paracol}


% ---- martyrology/mart01/mart0119.htm
\needspace{10\baselineskip}
\begin{paracol}{2}
\selectlanguage{latin}
\begin{center}{\color{gregoriocolor} Quartodécimo Kaléndas Februárii. 
 Luna\dots\ }\end{center}
\switchcolumn
\selectlanguage{english}
\begin{center}{\color{gregoriocolor} The Nineteenth Day of 
 January. The\dots\ Day of the Moon.}\end{center}
\end{paracol}

\noindent\begin{tabularx}{\linewidth}{*{19}{>{\centering\arraybackslash}X}}
 \textcolor{gregoriocolor}{a} & \textcolor{gregoriocolor}{b} & \textcolor{gregoriocolor}{c} & \textcolor{gregoriocolor}{d} & \textcolor{gregoriocolor}{e} & \textcolor{gregoriocolor}{f} & \textcolor{gregoriocolor}{g} & \textcolor{gregoriocolor}{h} & \textcolor{gregoriocolor}{i} & \textcolor{gregoriocolor}{k} & \textcolor{gregoriocolor}{l} & \textcolor{gregoriocolor}{m} & \textcolor{gregoriocolor}{n} & \textcolor{gregoriocolor}{p} & \textcolor{gregoriocolor}{q} & \textcolor{gregoriocolor}{r} & \textcolor{gregoriocolor}{s} & \textcolor{gregoriocolor}{t} & \textcolor{gregoriocolor}{u} \\
 20 & 21 & 22 & 23 & 24 & 25 & 26 & 27 & 28 & 29 & 30 & 1 & 2 & 3 & 4 & 5 & 6 & 7 & 8 \\
\end{tabularx}
\vspace{0.5\baselineskip}
\noindent\begin{tabularx}{\linewidth}{*{12}{>{\centering\arraybackslash}X}}
 \textcolor{gregoriocolor}{A} & \textcolor{gregoriocolor}{B} & \textcolor{gregoriocolor}{C} & \textcolor{gregoriocolor}{D} & \textcolor{gregoriocolor}{E} & F & \textcolor{gregoriocolor}{F} & \textcolor{gregoriocolor}{G} & \textcolor{gregoriocolor}{H} & \textcolor{gregoriocolor}{M} & \textcolor{gregoriocolor}{N} & \textcolor{gregoriocolor}{P} \\
 9 & 10 & 11 & 12 & 13 & 14 & 14 & 15 & 16 & 17 & 18 & 19 \\
\end{tabularx}

\begin{paracol}{2}
\selectlanguage{latin}
\lettrine[lines=2]{R}{omæ,} via Cornélia, sanctórum Mártyrum Márii et 
 Marthæ cónjugum, et filiórum Audífacis et Abachum, nobílium Persárum; qui 
 Romam, tempóribus Cláudii Príncipis, ad oratiónem vénerant. Ex eis 
 vero, post tolerátos fustes, equúleum, ignes, ungues férreos manuúmque 
 præcisiónem, Martha in Nympha necáta est; céteri sunt decolláti, et córpora 
 eórum incénsa.
\switchcolumn
\selectlanguage{english}
\lettrine[lines=2]{A}{t} Rome, on the Cornelian Road, the holy martyrs Marius and his wife Martha, 
 with their sons Audifax and Habbakuk, noble Persians, who came to Rome 
 through devotion in the time of Emperor Claudius. After they had been 
 beaten with rods, tormented on the rack and with fire, lacerated with iron 
 hooks, and had endured the cutting off of their hands, Martha was put to 
 death in the place called Nympha; the others were beheaded and cast into the 
 fire.
\switchcolumn*
\selectlanguage{latin}
Item sancti Canúti, Regis et Mártyris.
\switchcolumn
\selectlanguage{english}
Also St. Canute, king and martyr.
\switchcolumn*
\selectlanguage{latin}
Smyrnæ natális beáti Germánici Mártyris, qui, 
 sub Marco Antoníno et Lúcio Aurélio, cum primævæ ætátis venustáte floréret, 
 damnátus a Júdice, et, per grátiam virtútis Dei, metum corpóreæ fragilitátis 
 exclúdens, præparátum sibi béstiam sponte provocávit; cujus déntibus 
 comminútus, vero pani Dómino Jesu Christo, pro ipso móriens, méruit incorporári.
\switchcolumn
\selectlanguage{english}
At Smyrna, under Marcus Antoninus and Lucius Aurelius, the birthday of 
 blessed Germanicus, martyr, who, in the bloom of youth, being strengthened 
 by the grace of God, and banishing all fear, provoked the beast which, by 
 order of the judge, was to devour him. Being ground by its teeth, he 
 deserved to be incorporated into the true Bread of Life, Christ Jesus, for 
 whom he died.
\switchcolumn*
\selectlanguage{latin}
In Africa sanctórum Mártyrum Pauli, Geróntii, Januárii, Saturníni, Succéssi, 
 Júlii, Cati, Piæ et Germánæ.
\switchcolumn
\selectlanguage{english}
In Africa., the holy martyrs Paul, Gerontius, Januarius, Saturninus, 
 Successus, Julius, Catus, Pia, and Germana.
\switchcolumn*
\selectlanguage{latin}
Apud Spolétum pássio sancti Pontiáni Mártyris, qui, témpore Antoníni 
 Imperatóris, a Fabiáno Júdice, pro Christo vehementíssime virgis cæsus, 
 jussus est super carbónes nudis pédibus ambuláre, sed a carbónibus nil læsus, 
 equúleo et uncínis férreis jussus est suspéndi, et sic in cárcerem trudi, 
 ubi Angélica visitatióne méruit confortári; postque leónibus expósitus et 
 plumbo fervénti perfúsus, tandem gládio percússus est.
\switchcolumn
\selectlanguage{english}
At Spoleto, in the days of Emperor Antoninus, the passion of St. Pontian, 
 martyr, who was barbarously scourged for Christ by the command of the judge 
 Fabian, and then compelled to walk barefoot on burning coals. As he 
 was uninjured by the fire, he was put on the rack, was torn with iron hooks, 
 then thrown into a dungeon, where he was comforted by the visit of an angel. 
 He was afterwards exposed to the lions, had melted lead poured over him, and 
 finally died by the sword.
\switchcolumn*
\selectlanguage{latin}
Laudæ, in Insúbria, sancti Bassiáni, Epíscopi 
 et Confessóris; qui advérsus hæréticos, una cum sancto Ambrósio, strénue 
 decertávit.
\switchcolumn
\selectlanguage{english}
At Lodi in Lombardy, St. Bassian, bishop and confessor, who, in conjunction 
 with St. Ambrose, courageously combatted the heretics.
\switchcolumn*
\selectlanguage{latin}
Wigórniæ, in Anglia, sancti Wulstáni, Epíscopi 
 et Confessóris, méritis et miráculis conspícui; qui ab Innocéntio Papa 
 Tértio inter Sanctos relátus est.
\switchcolumn
\selectlanguage{english}
At Worcester, England, St. Wulfstan, bishop and confessor, conspicuous for 
 merits and miracles. He was ranked among the saints by Innocent III.
\switchcolumn*
\selectlanguage{latin}
\end{paracol}


% ---- martyrology/mart01/mart0120.htm
\needspace{10\baselineskip}
\begin{paracol}{2}
\selectlanguage{latin}
\begin{center}{\color{gregoriocolor} Tertiodécimo Kaléndas Februárii. 
 Luna\dots\ }\end{center}
\switchcolumn
\selectlanguage{english}
\begin{center}{\color{gregoriocolor} The Twentieth Day of 
 January. The\dots\ Day of the Moon.}\end{center}
\end{paracol}

\noindent\begin{tabularx}{\linewidth}{*{19}{>{\centering\arraybackslash}X}}
 \textcolor{gregoriocolor}{a} & \textcolor{gregoriocolor}{b} & \textcolor{gregoriocolor}{c} & \textcolor{gregoriocolor}{d} & \textcolor{gregoriocolor}{e} & \textcolor{gregoriocolor}{f} & \textcolor{gregoriocolor}{g} & \textcolor{gregoriocolor}{h} & \textcolor{gregoriocolor}{i} & \textcolor{gregoriocolor}{k} & \textcolor{gregoriocolor}{l} & \textcolor{gregoriocolor}{m} & \textcolor{gregoriocolor}{n} & \textcolor{gregoriocolor}{p} & \textcolor{gregoriocolor}{q} & \textcolor{gregoriocolor}{r} & \textcolor{gregoriocolor}{s} & \textcolor{gregoriocolor}{t} & \textcolor{gregoriocolor}{u} \\
 21 & 22 & 23 & 24 & 25 & 26 & 27 & 28 & 29 & 30 & 1 & 2 & 3 & 4 & 5 & 6 & 7 & 8 & 9 \\
\end{tabularx}
\vspace{0.5\baselineskip}
\noindent\begin{tabularx}{\linewidth}{*{12}{>{\centering\arraybackslash}X}}
 \textcolor{gregoriocolor}{A} & \textcolor{gregoriocolor}{B} & \textcolor{gregoriocolor}{C} & \textcolor{gregoriocolor}{D} & \textcolor{gregoriocolor}{E} & F & \textcolor{gregoriocolor}{F} & \textcolor{gregoriocolor}{G} & \textcolor{gregoriocolor}{H} & \textcolor{gregoriocolor}{M} & \textcolor{gregoriocolor}{N} & \textcolor{gregoriocolor}{P} \\
 10 & 11 & 12 & 13 & 14 & 15 & 15 & 16 & 17 & 18 & 19 & 20 \\
\end{tabularx}

\begin{paracol}{2}
\selectlanguage{latin}
\lettrine[lines=2]{R}{omæ} natális sancti Fabiáni, Papæ et Mártyris; 
 qui, Décii témpore, martyrium passus est, atque in cœmetério Callísti 
 sepúltus.
\switchcolumn
\selectlanguage{english}
\lettrine[lines=2]{A}{t} Rome, the birthday of St. Fabian, pope, who suffered martyrdom in the 
 time of Decius, and was buried in the cemetery of Callistus.
\switchcolumn*
\selectlanguage{latin}
Item Romæ, ad Catacúmbas, sancti Sebastiáni 
 Mártyris, qui, Diocletiáno Imperatóre, cum habéret principátum primæ 
 cohórtis, jussus est, sub título christianitátis, ligári in médio campo, et 
 sagittári a milítibus, atque ad últimum fústibus cædi, donec defíceret.
\switchcolumn
\selectlanguage{english}
Also at Rome, in the catacombs, the martyr St. Sebastian. He was 
 commander of the first cohort under Emperor Diocletian, and for professing 
 Christianity he was bound to a tree in the centre of a vast field, shot with 
 arrows by the soldiers, and beaten with clubs until he expired.
\switchcolumn*
\selectlanguage{latin}
Nicǽæ, in Bithynia, sancti Neóphyti Mártyris, 
 qui, quintumdécimum annum ætátis agens, flagris cæsus, in fornácem immíssus, 
 feris objéctus, et, cum illæsus permanéret et Christi fidem constánter 
 profiterétur, gládio tandem occísus est.
\switchcolumn
\selectlanguage{english}
At Nicea in Bithynia, St. Neophytus, martyr, who in the fifteenth year of 
 his age, was scourged, cast into a furnace, and exposed to wild beasts. 
 As he remained uninjured, and constantly confessed the faith of Christ, he 
 was at last killed with the sword.
\switchcolumn*
\selectlanguage{latin}
Cæsénæ sancti Mauri Epíscopi, virtútibus et 
 miráculis clari.
\switchcolumn
\selectlanguage{english}
At Cesena, St. Maur, bishop, renowned for virtues and miracles.
\switchcolumn*
\selectlanguage{latin}
In Palæstína natális sancti Euthymii Abbátis, 
 qui zelo cathólicæ discíplinæ et virtúte miraculórum, témpore Marciáni 
 Imperatóris, in Ecclésia flóruit.
\switchcolumn
\selectlanguage{english}
In Palestine, in the time of Emperor Marcian, the birthday of St. Euthymius, 
 abbot, who flourished in the Church, full of zeal for Catholic discipline, 
 and gifted with miracles.
\switchcolumn*
\selectlanguage{latin}
\end{paracol}


% ---- martyrology/mart01/mart0121.htm
\needspace{10\baselineskip}
\begin{paracol}{2}
\selectlanguage{latin}
\begin{center}{\color{gregoriocolor} Duodécimo Kaléndas Februárii. 
 Luna\dots\ }\end{center}
\switchcolumn
\selectlanguage{english}
\begin{center}{\color{gregoriocolor} The Twenty-First Day of 
 January. The\dots\ Day of the Moon.}\end{center}
\end{paracol}

\noindent\begin{tabularx}{\linewidth}{*{19}{>{\centering\arraybackslash}X}}
 \textcolor{gregoriocolor}{a} & \textcolor{gregoriocolor}{b} & \textcolor{gregoriocolor}{c} & \textcolor{gregoriocolor}{d} & \textcolor{gregoriocolor}{e} & \textcolor{gregoriocolor}{f} & \textcolor{gregoriocolor}{g} & \textcolor{gregoriocolor}{h} & \textcolor{gregoriocolor}{i} & \textcolor{gregoriocolor}{k} & \textcolor{gregoriocolor}{l} & \textcolor{gregoriocolor}{m} & \textcolor{gregoriocolor}{n} & \textcolor{gregoriocolor}{p} & \textcolor{gregoriocolor}{q} & \textcolor{gregoriocolor}{r} & \textcolor{gregoriocolor}{s} & \textcolor{gregoriocolor}{t} & \textcolor{gregoriocolor}{u} \\
 22 & 23 & 24 & 25 & 26 & 27 & 28 & 29 & 30 & 1 & 2 & 3 & 4 & 5 & 6 & 7 & 8 & 9 & 10 \\
\end{tabularx}
\vspace{0.5\baselineskip}
\noindent\begin{tabularx}{\linewidth}{*{12}{>{\centering\arraybackslash}X}}
 \textcolor{gregoriocolor}{A} & \textcolor{gregoriocolor}{B} & \textcolor{gregoriocolor}{C} & \textcolor{gregoriocolor}{D} & \textcolor{gregoriocolor}{E} & F & \textcolor{gregoriocolor}{F} & \textcolor{gregoriocolor}{G} & \textcolor{gregoriocolor}{H} & \textcolor{gregoriocolor}{M} & \textcolor{gregoriocolor}{N} & \textcolor{gregoriocolor}{P} \\
 11 & 12 & 13 & 14 & 15 & 16 & 16 & 17 & 18 & 19 & 20 & 21 \\
\end{tabularx}

\begin{paracol}{2}
\selectlanguage{latin}
\lettrine[lines=2]{R}{omæ} pássio sanctæ Agnétis, Vírginis et 
 Mártyris; quæ, sub Præfécto Urbis Symphrónio, ígnibus injécta, sed iis per 
 oratiónem ejus exstínctis, gládio percússa est. De ea beátus 
 Hierónymus hæc scribit: « Omnium géntium lítteris atque linguis, præcípue in Ecclésiis, Agnétis vita laudáta est; quæ et ætátem vicit et tyránnum, et títulum castitátis martyrio consecrávit ».
\switchcolumn
\selectlanguage{english}
\lettrine[lines=2]{A}{t} Rome, the passion of St. Agnes, virgin, who under Symphronius, governor 
 of the city, was thrown into the fire, but after it was extinguished by her 
 prayers, she was slain with the sword. Of her, St. Jerome writes: 
 ``Agnes is praised in the writings and by the tongues of all nations, 
 especially in the churches. She overcame the weakness of her age, 
 conquered the cruelty of the tyrant, and consecrated her chastity by 
 martyrdom.''
\switchcolumn*
\selectlanguage{latin}
Athénis natális sancti Públii Epíscopi, qui Atheniénsium Ecclésiam, post 
 sanctum Dionysium Areopagítam, nobíliter rexit; et, præclárus 
 virtútibus ac doctrínæ laude præfúlgens, ob Christi martyrium glorióse 
 coronátur.
\switchcolumn
\selectlanguage{english}
At Athens, the birthday of St. Publius, bishop, who, as successor of St. 
 Denis the Areopagite, nobly governed the Church of Athens. No less 
 celebrated for the lustre of his virtues than for the brilliancy of his 
 learning, he was gloriously crowned for having borne testimony to Christ.
\switchcolumn*
\selectlanguage{latin}
Tarracóne, in Hispánia, sanctórum Mártyrum Fructuósi Epíscopi, Augúrii et 
 Eulógii Diaconórum. Hi, témpore Galliéni, primo in cárcerem trusi, 
 deínde flammis injécti, et, exústis vínculis, mánibus in modum crucis 
 expánsis orántes, martyrium complevérunt; in quorum die natáli sanctus 
 Augustínus sermónem ad pópulum hábuit.
\switchcolumn
\selectlanguage{english}
At Terragona in Spain, during the reign of Gallienus, the holy martyrs 
 Fructuosus, a bishop, Augurius and Eulogius, deacons. They were taken 
 from prison, cast into the fire, where, their bonds being burnt, they 
 extended their arms in the form of a cross, and thus in prayer they died. 
 On their anniversary, St. Augustine preached a sermon to his people.
\switchcolumn*
\selectlanguage{latin}
In monastério Einsidlénsi, apud Helvétios, sancti Meinrádi, Presbyteri et 
 Mónachi; qui eódem in loco, ubi póstea monastérium ipsum excrévit, eremíticæ 
 inténtus vitæ, a latrónibus interféctus est. Ipsíus vero beáti viri 
 corpus, olim in Augiénsi Germániæ monastério sepúltum, ad Einsidlénse 
 monastérium deínde relátum fuit.
\switchcolumn
\selectlanguage{english}
In the monastery of Einsiedeln in Switzerland, St. Meinrad, priest and monk, 
 who was slain by robbers after having lived as a hermit in this place where 
 the monastery was later built. The body of this holy man was first 
 buried in the monastery of Reichenau in Germany, and from there it was 
 transferred to the monastery of Einsiedeln.
\switchcolumn*
\selectlanguage{latin}
Trecis, in Gállia, sancti Pátrocli Mártyris, qui martyrii corónam sub 
 Aureliáno Imperatóre proméruit.
\switchcolumn
\selectlanguage{english}
At Troyes in France, St. Patroclus, martyr, who won the crown of martyrdom 
 under Emperor Aurelian.
\switchcolumn*
\selectlanguage{latin}
Papíæ sancti Epiphánii, Epíscopi et Confessóris.
\switchcolumn
\selectlanguage{english}
At Pavia, St. Epiphanius, bishop and confessor.
\switchcolumn*
\selectlanguage{latin}
\end{paracol}


% ---- martyrology/mart01/mart0122.htm
\needspace{10\baselineskip}
\begin{paracol}{2}
\selectlanguage{latin}
\begin{center}{\color{gregoriocolor} Undécimo Kaléndas Februárii. 
 Luna\dots\ }\end{center}
\switchcolumn
\selectlanguage{english}
\begin{center}{\color{gregoriocolor} The Twenty-Second Day of 
 January. The\dots\ Day of the Moon.}\end{center}
\end{paracol}

\noindent\begin{tabularx}{\linewidth}{*{19}{>{\centering\arraybackslash}X}}
 \textcolor{gregoriocolor}{a} & \textcolor{gregoriocolor}{b} & \textcolor{gregoriocolor}{c} & \textcolor{gregoriocolor}{d} & \textcolor{gregoriocolor}{e} & \textcolor{gregoriocolor}{f} & \textcolor{gregoriocolor}{g} & \textcolor{gregoriocolor}{h} & \textcolor{gregoriocolor}{i} & \textcolor{gregoriocolor}{k} & \textcolor{gregoriocolor}{l} & \textcolor{gregoriocolor}{m} & \textcolor{gregoriocolor}{n} & \textcolor{gregoriocolor}{p} & \textcolor{gregoriocolor}{q} & \textcolor{gregoriocolor}{r} & \textcolor{gregoriocolor}{s} & \textcolor{gregoriocolor}{t} & \textcolor{gregoriocolor}{u} \\
 23 & 24 & 25 & 26 & 27 & 28 & 29 & 30 & 1 & 2 & 3 & 4 & 5 & 6 & 7 & 8 & 9 & 10 & 11 \\
\end{tabularx}
\vspace{0.5\baselineskip}
\noindent\begin{tabularx}{\linewidth}{*{12}{>{\centering\arraybackslash}X}}
 \textcolor{gregoriocolor}{A} & \textcolor{gregoriocolor}{B} & \textcolor{gregoriocolor}{C} & \textcolor{gregoriocolor}{D} & \textcolor{gregoriocolor}{E} & F & \textcolor{gregoriocolor}{F} & \textcolor{gregoriocolor}{G} & \textcolor{gregoriocolor}{H} & \textcolor{gregoriocolor}{M} & \textcolor{gregoriocolor}{N} & \textcolor{gregoriocolor}{P} \\
 12 & 13 & 14 & 15 & 16 & 17 & 17 & 18 & 19 & 20 & 21 & 22 \\
\end{tabularx}

\begin{paracol}{2}
\selectlanguage{latin}
\lettrine[lines=2]{V}{aléntiæ,} in Hispánia Tarraconénsi, sancti 
 Vincéntii, Levítæ et Mártyris; qui, sub impiíssimo Præside Daciáno, cárceres, 
 famem, equúleum, distorsiónes membrórum, láminas candéntes, férream cratem 
 ignítam áliaque tormentórum génera perpéssus, ad martyrii præmium evolávit 
 in cælum; cujus passiónis nóbilem triúmphum Prudéntius luculénter vérsibus 
 exséquitur, et beátus Augustínus ac sanctus Leo Papa summis láudibus 
 comméndant.
\switchcolumn
\selectlanguage{english}
\lettrine[lines=2]{A}{t} Valencia in Spain, while the wicked Dacian was governor, St. Vincent, 
 deacon and martyr, who, after suffering imprisonment, hunger, the rack, and 
 the disjointing of his limbs, was burned with plates of heated metal and on 
 the gridiron, and tormented in other ways, then took his flight to heaven, 
 there to receive the reward of martyrdom. His noble triumph over his 
 sufferings has been skillfully set forth in verse by Prudentius, and also 
 was eulogized by St. Augustine and Pope St. Leo.
\switchcolumn*
\selectlanguage{latin}
Apud Bethsáloen, in Assyria, sancti Anastásii 
 Persæ Mónachi, qui, post plúrima torménta cárceris, vérberum et vinculórum, 
 quæ in Cæsaréa Palæstínæ perpéssus fúerat, a Persárum Rege Chósroa multis 
 pœnis afféctus, ad últimum decollátus est, cum prius septuagínta Sócios, qui 
 fúerant in fluénta demérsi, ad martyrium præmisísset. Ejus caput Romam, 
 ad Aquas Sálvias, delátum est, una cum veneránda ejus imágine, cujus aspéctu 
 fugári dæmones morbósque curári, Acta secúndi Concílii Nicǽni testántur.
\switchcolumn
\selectlanguage{english}
At Bethsaloen in Assyria, St. Anastasius, a Persian monk, who after 
 suffering much at Caesarea in Palestine from imprisonment, stripes, and 
 fetters, had to bear many afflictions from Chosroes, king of Persia, who 
 caused him to be beheaded. He had sent before him to martyrdom seventy 
 of his companions, who were drowned in a river. His head was brought 
 to Rome, at Aquae Salviae, together with his revered image, by the sight of 
 which demons are expelled, and diseases cured, as is attested by the Acts of 
 the second Council of Nicea.
\switchcolumn*
\selectlanguage{latin}
Ebredúni, in Gálliis, sanctórum Mártyrum Vincéntii, Oróntii et Victóris; qui 
 martyrio in Diocletiáni persecutióne coronáti sunt.
\switchcolumn
\selectlanguage{english}
At Embrun in France, the holy martyrs Vincent, Orontius, and Victor who were 
 crowned with martyrdom in the persecution of Diocletian.
\switchcolumn*
\selectlanguage{latin}
Nováriæ sancti Gaudéntii, Epíscopi et 
 Confessóris.
\switchcolumn
\selectlanguage{english}
At Novara, St. Gaudentius, bishop and confessor.
\switchcolumn*
\selectlanguage{latin}
Soræ sancti Domínici Abbátis, miráculis clari.
\switchcolumn
\selectlanguage{english}
At Sora, the abbot St. Dominic, renowned for miracles.
\switchcolumn*
\selectlanguage{latin}
\end{paracol}


% ---- martyrology/mart01/mart0123.htm
\needspace{10\baselineskip}
\begin{paracol}{2}
\selectlanguage{latin}
\begin{center}{\color{gregoriocolor} Décimo Kaléndas Februárii. 
 Luna\dots\ }\end{center}
\switchcolumn
\selectlanguage{english}
\begin{center}{\color{gregoriocolor} The Twenty-Third Day of 
 January. The\dots\ Day of the Moon.}\end{center}
\end{paracol}

\noindent\begin{tabularx}{\linewidth}{*{19}{>{\centering\arraybackslash}X}}
 \textcolor{gregoriocolor}{a} & \textcolor{gregoriocolor}{b} & \textcolor{gregoriocolor}{c} & \textcolor{gregoriocolor}{d} & \textcolor{gregoriocolor}{e} & \textcolor{gregoriocolor}{f} & \textcolor{gregoriocolor}{g} & \textcolor{gregoriocolor}{h} & \textcolor{gregoriocolor}{i} & \textcolor{gregoriocolor}{k} & \textcolor{gregoriocolor}{l} & \textcolor{gregoriocolor}{m} & \textcolor{gregoriocolor}{n} & \textcolor{gregoriocolor}{p} & \textcolor{gregoriocolor}{q} & \textcolor{gregoriocolor}{r} & \textcolor{gregoriocolor}{s} & \textcolor{gregoriocolor}{t} & \textcolor{gregoriocolor}{u} \\
 24 & 25 & 26 & 27 & 28 & 29 & 30 & 1 & 2 & 3 & 4 & 5 & 6 & 7 & 8 & 9 & 10 & 11 & 12 \\
\end{tabularx}
\vspace{0.5\baselineskip}
\noindent\begin{tabularx}{\linewidth}{*{12}{>{\centering\arraybackslash}X}}
 \textcolor{gregoriocolor}{A} & \textcolor{gregoriocolor}{B} & \textcolor{gregoriocolor}{C} & \textcolor{gregoriocolor}{D} & \textcolor{gregoriocolor}{E} & F & \textcolor{gregoriocolor}{F} & \textcolor{gregoriocolor}{G} & \textcolor{gregoriocolor}{H} & \textcolor{gregoriocolor}{M} & \textcolor{gregoriocolor}{N} & \textcolor{gregoriocolor}{P} \\
 13 & 14 & 15 & 16 & 17 & 18 & 18 & 19 & 20 & 21 & 22 & 23 \\
\end{tabularx}

\begin{paracol}{2}
\selectlanguage{latin}
\lettrine[lines=2]{S}{ancti} Raymúndi de Pénafort, ex Ordine Prædicatórum, 
 Confessóris; cujus dies natális octávo Idus Januárii recólitur.
\switchcolumn
\selectlanguage{english}
\lettrine[lines=2]{S}{t.} Raymond of Pennafort, of the Order of Preachers, whose birthday is the 
 sixth of this month.
\switchcolumn*
\selectlanguage{latin}
Romæ sanctæ Emerentiánæ, Vírginis et Mártyris; 
 quæ adhuc catechúmena, dum oráret ad sepúlcrum sanctæ Agnétis, cujus fúerat 
 collactánea, a Gentílibus lapidáta est.
\switchcolumn
\selectlanguage{english}
At Rome, the holy virgin and martyr, St. Emerentiana. Being yet a 
 catechumen, she was stoned to death by the heathens while praying at the 
 tomb of St. Agnes, her foster sister.
\switchcolumn*
\selectlanguage{latin}
Philíppis, in Macedónia, sancti Pármenæ, qui 
 fuit unus de septem primis Diáconis. Hic, tráditus grátiæ Dei, injúnctum sibi a frátribus offícium prædicatiónis plena fide consúmmans, 
 martyrii glóriam, sub Trajáno, est adéptus.
\switchcolumn
\selectlanguage{english}
At Philippi in Macedonia, St. Parmenas, one of the first seven deacons, who 
 by the grace of God faithfully discharged the office of preaching committed 
 to him, and obtained the glory of martyrdom in the time of Trajan.
\switchcolumn*
\selectlanguage{latin}
Ancyræ, in Galátia, sancti Cleméntis Epíscopi, 
 qui, sæpius cruciátus, tandem, sub Diocletiáno Imperatóre, martyrium 
 consummávit.
\switchcolumn
\selectlanguage{english}
At Ancyra in Galatia, St. Clement, bishop. After enduring frequent 
 torments, he finally completed his martyrdom under Diocletian.
\switchcolumn*
\selectlanguage{latin}
Ibídem sancti Agathángeli, qui eódem die, sub 
 Lúcio Præside, passus est.
\switchcolumn
\selectlanguage{english}
In the same place, and on the same day, St. Agathangelus who suffered under 
 the governor Lucius.
\switchcolumn*
\selectlanguage{latin}
Cæsaréæ, in Mauritánia, sanctórum Mártyrum 
 Severiáni et Aquilæ uxóris, ígnibus combustórum.
\switchcolumn
\selectlanguage{english}
At Caesarea in Morocco, the holy martyrs Severian and his wife Aquila, who 
 were consumed by fire.
\switchcolumn*
\selectlanguage{latin}
Apud Antínoum, Ægypti urbem, sancti Asclæ 
 Mártyris, qui, post divérsa torménta, pretiósam Deo ánimam, in flumen 
 præcipitátus, réddidit.
\switchcolumn
\selectlanguage{english}
At Antinoum, a city of Egypt, St. Ascla, martyr, who, after various 
 torments, was thrown into a river and gave up his precious soul unto God.
\switchcolumn*
\selectlanguage{latin}
Alexandríæ sancti Joánnis Eleemosynárii, 
 ejúsdem urbis Epíscopi, misericórdia in páuperes celebérrimi.
\switchcolumn
\selectlanguage{english}
At Alexandria, St. John the Almoner, bishop of that city, celebrated for his 
 charity towards the poor.
\switchcolumn*
\selectlanguage{latin}
Toléti, in Hispánia, sancti Ildefónsi Epíscopi, qui, ob singulárem vitæ 
 integritátem, susceptámque fídei defensiónem advérsus hæréticos, sanctíssimæ 
 Dei Genitrícis virginitátem impugnántes, ab eádem Vírgine María donátus est 
 candidíssima veste, ac demum, sanctitáte célebris, in cælum vocátus.
\switchcolumn
\selectlanguage{english}
At Toledo, St. Ildefonse, bishop, renowned for sanctity. On account of 
 his great purity of life, and his defence of the virginity of the Mother of 
 God, against the heretics who denied it, he received from her a brilliant 
 white vestment, and was called to heaven.
\switchcolumn*
\selectlanguage{latin}
In Província Valériæ sancti Martyrii Mónachi, 
 cujus méminit beátus Gregórius Papa.
\switchcolumn
\selectlanguage{english}
In the province of Valeria, St. Martyrius, monk, mentioned by Pope St. 
 Gregory.
\switchcolumn*
\selectlanguage{latin}
\end{paracol}


% ---- martyrology/mart01/mart0124.htm
\needspace{10\baselineskip}
\begin{paracol}{2}
\selectlanguage{latin}
\begin{center}{\color{gregoriocolor} Nono Kaléndas Februárii. 
 Luna\dots\ }\end{center}
\switchcolumn
\selectlanguage{english}
\begin{center}{\color{gregoriocolor} The Twenty-Fourth Day of 
 January. The\dots\ Day of the Moon.}\end{center}
\end{paracol}

\noindent\begin{tabularx}{\linewidth}{*{19}{>{\centering\arraybackslash}X}}
 \textcolor{gregoriocolor}{a} & \textcolor{gregoriocolor}{b} & \textcolor{gregoriocolor}{c} & \textcolor{gregoriocolor}{d} & \textcolor{gregoriocolor}{e} & \textcolor{gregoriocolor}{f} & \textcolor{gregoriocolor}{g} & \textcolor{gregoriocolor}{h} & \textcolor{gregoriocolor}{i} & \textcolor{gregoriocolor}{k} & \textcolor{gregoriocolor}{l} & \textcolor{gregoriocolor}{m} & \textcolor{gregoriocolor}{n} & \textcolor{gregoriocolor}{p} & \textcolor{gregoriocolor}{q} & \textcolor{gregoriocolor}{r} & \textcolor{gregoriocolor}{s} & \textcolor{gregoriocolor}{t} & \textcolor{gregoriocolor}{u} \\
 25 & 26 & 27 & 28 & 29 & 30 & 1 & 2 & 3 & 4 & 5 & 6 & 7 & 8 & 9 & 10 & 11 & 12 & 13 \\
\end{tabularx}
\vspace{0.5\baselineskip}
\noindent\begin{tabularx}{\linewidth}{*{12}{>{\centering\arraybackslash}X}}
 \textcolor{gregoriocolor}{A} & \textcolor{gregoriocolor}{B} & \textcolor{gregoriocolor}{C} & \textcolor{gregoriocolor}{D} & \textcolor{gregoriocolor}{E} & F & \textcolor{gregoriocolor}{F} & \textcolor{gregoriocolor}{G} & \textcolor{gregoriocolor}{H} & \textcolor{gregoriocolor}{M} & \textcolor{gregoriocolor}{N} & \textcolor{gregoriocolor}{P} \\
 14 & 15 & 16 & 17 & 18 & 19 & 19 & 20 & 21 & 22 & 23 & 24 \\
\end{tabularx}

\begin{paracol}{2}
\selectlanguage{latin}
\lettrine[lines=2]{A}{pud} Ephesum sancti Timóthei, qui fuit 
 discípulus beáti Pauli Apóstoli; atque, ab eódem Ephesi ordinátus Epíscopus, 
 ibi, post multos pro Christo agónes, cum Diánæ immolántes argúeret, 
 lapídibus óbrutus est, ac paulo post obdormívit in Dómino.
\switchcolumn
\selectlanguage{english}
\lettrine[lines=2]{A}{t} Ephesus, St. Timothy, disciple of the apostle St. Paul, who ordained him 
 bishop of that city. After many labours for Christ, he was stoned for 
 rebuking those who offered sacrifices to Diana, and shortly after went 
 peacefully to his rest in the Lord.
\switchcolumn*
\selectlanguage{latin}
Antiochíæ sancti Bábilæ Epíscopi, qui, in 
 persecutióne Décii, póstea quam frequénter passiónibus suis ac cruciátibus 
 glorificáverat Deum, gloriósæ vitæ finem sortítus est in vínculis férreis, 
 cum quibus et suum corpus sepelíri mandávit. Referúntur étiam passi 
 cum eo tres púeri, scílicet Urbánus, Prilidiánus et Epolónius, quos ille in 
 Christi fide instrúxerat.
\switchcolumn
\selectlanguage{english}
At Antioch, in the persecution of Decius, Bishop St. Babylas, who frequently 
 glorified God by his sufferings and torments, ended his life in chains, with 
 which he ordered his body to be buried. Three boys, whom he had 
 instructed in the faith of Christ, Urbanus, Prilidian, and Epolonius, are 
 said to have suffered with him.
\switchcolumn*
\selectlanguage{latin}
Fulgínei, in Umbria, sancti Feliciáni, qui, a sancto Victóre Papa Primo 
 Epíscopus ejúsdem civitátis ordinátus, illic, post multos labóres, in 
 última senectúte, sub Décio Imperatóre, martyrio coronátus est.
\switchcolumn
\selectlanguage{english}
At Foligno in Umbria, St. Felician, consecrated bishop of that city by Pope 
 St. Victor I. After many labours, in extreme old age, he was crowned 
 with martyrdom in the time of Decius.
\switchcolumn*
\selectlanguage{latin}
Neocæsaréæ, in Mauritánia, sanctórum Mártyrum 
 Mardónii, Musónii, Eugénii et Metélli; qui omnes igni tráditi sunt, et eórum 
 relíquiæ in flumen dispérsæ.
\switchcolumn
\selectlanguage{english}
At Neocaesarea, the holy martyrs Mardonius, Musonius, Eugenius, and Metellus, 
 who were all burned to death, and their remains thrown into the river.
\switchcolumn*
\selectlanguage{latin}
Item sanctórum Mártyrum Thyrsi et Projécti.
\switchcolumn
\selectlanguage{english}
Also, the holy martyrs Thyrsus and Projectus.
\switchcolumn*
\selectlanguage{latin}
Cínguli, in Picéno, sancti Exsuperántii 
 Confessóris, ejúsdem civitátis Epíscopi, ob miraculórum famam illústris.
\switchcolumn
\selectlanguage{english}
At Cingoli in Piceno, St. Exuperantius, confessor and bishop of that city, 
 who attained great fame by his miracles.
\switchcolumn*
\selectlanguage{latin}
Bonóniæ sancti Zamæ, qui, a sancto Dionysio, 
 Románo Pontífice, primus ejúsdem civitátis Epíscopus ordinátus, illic 
 Christiánam fidem mirífice propagávit.
\switchcolumn
\selectlanguage{english}
At Bologna, St. Zamas, the first bishop of that city, who was consecrated by 
 Pope St. Denis, and there did wonders in spreading the Christian faith.
\switchcolumn*
\selectlanguage{latin}
Item beáti Suráni Abbátis, qui, témpore Longobardórum, sanctitáte flóruit.
\switchcolumn
\selectlanguage{english}
Also, blessed Suranus, abbot, who lived in the time of the Lombards.
\switchcolumn*
\selectlanguage{latin}
\end{paracol}


% ---- martyrology/mart01/mart0125.htm
\needspace{10\baselineskip}
\begin{paracol}{2}
\selectlanguage{latin}
\begin{center}{\color{gregoriocolor} Octávo Kaléndas Februárii. 
 Luna\dots\ }\end{center}
\switchcolumn
\selectlanguage{english}
\begin{center}{\color{gregoriocolor} The Twenty-Fifth Day of 
 January. The\dots\ Day of the Moon.}\end{center}
\end{paracol}

\noindent\begin{tabularx}{\linewidth}{*{19}{>{\centering\arraybackslash}X}}
 \textcolor{gregoriocolor}{a} & \textcolor{gregoriocolor}{b} & \textcolor{gregoriocolor}{c} & \textcolor{gregoriocolor}{d} & \textcolor{gregoriocolor}{e} & \textcolor{gregoriocolor}{f} & \textcolor{gregoriocolor}{g} & \textcolor{gregoriocolor}{h} & \textcolor{gregoriocolor}{i} & \textcolor{gregoriocolor}{k} & \textcolor{gregoriocolor}{l} & \textcolor{gregoriocolor}{m} & \textcolor{gregoriocolor}{n} & \textcolor{gregoriocolor}{p} & \textcolor{gregoriocolor}{q} & \textcolor{gregoriocolor}{r} & \textcolor{gregoriocolor}{s} & \textcolor{gregoriocolor}{t} & \textcolor{gregoriocolor}{u} \\
 26 & 27 & 28 & 29 & 30 & 1 & 2 & 3 & 4 & 5 & 6 & 7 & 8 & 9 & 10 & 11 & 12 & 13 & 14 \\
\end{tabularx}
\vspace{0.5\baselineskip}
\noindent\begin{tabularx}{\linewidth}{*{12}{>{\centering\arraybackslash}X}}
 \textcolor{gregoriocolor}{A} & \textcolor{gregoriocolor}{B} & \textcolor{gregoriocolor}{C} & \textcolor{gregoriocolor}{D} & \textcolor{gregoriocolor}{E} & F & \textcolor{gregoriocolor}{F} & \textcolor{gregoriocolor}{G} & \textcolor{gregoriocolor}{H} & \textcolor{gregoriocolor}{M} & \textcolor{gregoriocolor}{N} & \textcolor{gregoriocolor}{P} \\
 15 & 16 & 17 & 18 & 19 & 20 & 20 & 21 & 22 & 23 & 24 & 25 \\
\end{tabularx}

\begin{paracol}{2}
\selectlanguage{latin}
\lettrine[lines=2]{C}{onvérsio} sancti Pauli Apóstoli, quæ fuit anno 
 secúndo ab Ascensióne Domini.
\switchcolumn
\selectlanguage{english}
\lettrine[lines=2]{T}{he} conversion of St. Paul the Apostle, which occurred in the second year 
 after the Ascension of our Lord.
\switchcolumn*
\selectlanguage{latin}
Apud Damáscum natális sancti Ananíæ, qui fuit 
 discípulus Dómini, et eúndem Paulum Apóstolum baptizávit. Ipse autem, 
 cum Damásci, et Eleutherópoli, alibíque Evangélium prædicásset, tandem, sub 
 Licínio Júdice, nervis cæsus et laniátus, ac lapídibus oppréssus, martyrium 
 consummávit.
\switchcolumn
\selectlanguage{english}
At Damascus, the birthday of St. Ananias, who was a disciple of our Lord, 
 and baptized the apostle Paul. After he had preached the Gospel at 
 Damascus, Eleutheropolis, and elsewhere, he was scourged under the judge 
 Licinius, had his flesh torn, and lastly being overwhelmed with stones, 
 ended his martyrdom.
\switchcolumn*
\selectlanguage{latin}
Arvérnis, in Gállia, sanctórum Præjécti 
 Epíscopi, et Amaríni, Abbátis Cloroangiénsis; qui ambo a procéribus ejúsdem 
 urbis passi sunt.
\switchcolumn
\selectlanguage{english}
In the Auvergne in France, the Saints Praejectus, bishop, and Amarinus, 
 abbot of Doroang, who were murdered by the leading men of that city.
\switchcolumn*
\selectlanguage{latin}
Antiochíæ sanctórum Mártyrum Juventíni et 
 Máximi, qui, sub Juliáno Apóstata, martyrio coronáti sunt; in quorum die 
 natáli sanctus Joánnes Chrysóstomus sermónem ad pópulum hábuit.
\switchcolumn
\selectlanguage{english}
At Antioch, in the time of Julian the Apostate, the holy martyrs Juvenius 
 and Maximus, who were crowned with martyrdom. On their birthday, St. 
 John Chrysostom preached a sermon to his people.
\switchcolumn*
\selectlanguage{latin}
Item sanctórum Mártyrum Donáti, Sabíni et Agapis.
\switchcolumn
\selectlanguage{english}
Also, the holy martyrs Donatus, Sabinus, and Agape.
\switchcolumn*
\selectlanguage{latin}
Tomis, in Scythia, sancti Bretanniónis Epíscopi, qui mira sanctitáte et 
 cathólicæ fídei zelo, sub Ariáno Imperatóre 
 Valénte, cui fórtiter réstitit, in Ecclésia flóruit.
\switchcolumn
\selectlanguage{english}
At Tomis in Scythia, St. Bretannio, bishop, who worked in the Church shewing 
 great sanctity and zeal for the Catholic faith, and was at the same time 
 bravely opposed to the Arian emperor Valens.
\switchcolumn*
\selectlanguage{latin}
Marciánis, in Gállia, sancti Poppónis, Presbyteri et Abbátis, miráculis 
 clari.
\switchcolumn
\selectlanguage{english}
At Marchiennes in France, St. Poppo, priest and abbot, renowned for his 
 miracles.
\switchcolumn*
\selectlanguage{latin}
\end{paracol}


% ---- martyrology/mart01/mart0126.htm
\needspace{10\baselineskip}
\begin{paracol}{2}
\selectlanguage{latin}
\begin{center}{\color{gregoriocolor} Séptimo Kaléndas Februárii. 
 Luna\dots\ }\end{center}
\switchcolumn
\selectlanguage{english}
\begin{center}{\color{gregoriocolor} The Twenty-Sixth Day of 
 January. The\dots\ Day of the Moon.}\end{center}
\end{paracol}

\noindent\begin{tabularx}{\linewidth}{*{19}{>{\centering\arraybackslash}X}}
 \textcolor{gregoriocolor}{a} & \textcolor{gregoriocolor}{b} & \textcolor{gregoriocolor}{c} & \textcolor{gregoriocolor}{d} & \textcolor{gregoriocolor}{e} & \textcolor{gregoriocolor}{f} & \textcolor{gregoriocolor}{g} & \textcolor{gregoriocolor}{h} & \textcolor{gregoriocolor}{i} & \textcolor{gregoriocolor}{k} & \textcolor{gregoriocolor}{l} & \textcolor{gregoriocolor}{m} & \textcolor{gregoriocolor}{n} & \textcolor{gregoriocolor}{p} & \textcolor{gregoriocolor}{q} & \textcolor{gregoriocolor}{r} & \textcolor{gregoriocolor}{s} & \textcolor{gregoriocolor}{t} & \textcolor{gregoriocolor}{u} \\
 27 & 28 & 29 & 30 & 1 & 2 & 3 & 4 & 5 & 6 & 7 & 8 & 9 & 10 & 11 & 12 & 13 & 14 & 15 \\
\end{tabularx}
\vspace{0.5\baselineskip}
\noindent\begin{tabularx}{\linewidth}{*{12}{>{\centering\arraybackslash}X}}
 \textcolor{gregoriocolor}{A} & \textcolor{gregoriocolor}{B} & \textcolor{gregoriocolor}{C} & \textcolor{gregoriocolor}{D} & \textcolor{gregoriocolor}{E} & F & \textcolor{gregoriocolor}{F} & \textcolor{gregoriocolor}{G} & \textcolor{gregoriocolor}{H} & \textcolor{gregoriocolor}{M} & \textcolor{gregoriocolor}{N} & \textcolor{gregoriocolor}{P} \\
 16 & 17 & 18 & 19 & 20 & 21 & 21 & 22 & 23 & 24 & 25 & 26 \\
\end{tabularx}

\begin{paracol}{2}
\selectlanguage{latin}
\lettrine[lines=2]{S}{ancti} Polycárpi, Epíscopi Smyrnénsis et Mártyris; qui martyrii corónam 
 séptimo Kaléndas Mártii consecútus est.
\switchcolumn
\selectlanguage{english}
\lettrine[lines=2]{S}{t.} Polycarp, Bishop of Smyrna and martyr, who gained the crown of martyrdom 
 on the 23rd of February.
\switchcolumn*
\selectlanguage{latin}
Hippóne Régio, in Africa, sanctórum Theógenis 
 Epíscopi, et aliórum trigínta sex; qui, in persecutióne Valeriáni, 
 contemnéntes temporálem mortem, corónam ætérnæ vitæ adépti sunt.
\switchcolumn
\selectlanguage{english}
At Hippo in Africa, the holy bishop Theogenes and thirty-six others, who, 
 despising temporal death, obtained the crown of eternal life in the 
 persecution of Valerian.
\switchcolumn*
\selectlanguage{latin}
Apud Béthlehem Judæ dormítio sanctæ Paulæ Víduæ, 
 quæ, cum esset e nobilíssimo Senatórum génere, cum beáta Vírgine Christi 
 Eustóchio, fília sua, renúntians sæculo, facultátes suas paupéribus 
 distríbuit, et ad Præsépe Dómini se recépit; ibíque, multis virtútibus 
 prǽdita et longo coronáta martyrio, ad cæléstia regna transívit. 
 Ipsíus autem vitam, virtútibus admirándum, sanctus Hierónymus scripsit.
\switchcolumn
\selectlanguage{english}
At Bethlehem of Judea, the death of St. Paula, widow, mother of St. 
 Eustochium, a virgin of Christ, who abandoned her worldly prospects, though 
 she was descended from a noble line of senators, distributed her goods to 
 the poor, and retired to our Lord's manger, where, endowed with many 
 virtues, and crowned with a long martyrdom, she departed for the kingdom of 
 heaven. Her admirable life was written by St. Jerome.
\switchcolumn*
\selectlanguage{latin}
\end{paracol}


% ---- martyrology/mart01/mart0127.htm
\needspace{10\baselineskip}
\begin{paracol}{2}
\selectlanguage{latin}
\begin{center}{\color{gregoriocolor} Sexto Kaléndas Februárii. 
 Luna\dots\ }\end{center}
\switchcolumn
\selectlanguage{english}
\begin{center}{\color{gregoriocolor} The Twenty-Seventh Day of 
 January. The\dots\ Day of the Moon.}\end{center}
\end{paracol}

\noindent\begin{tabularx}{\linewidth}{*{19}{>{\centering\arraybackslash}X}}
 \textcolor{gregoriocolor}{a} & \textcolor{gregoriocolor}{b} & \textcolor{gregoriocolor}{c} & \textcolor{gregoriocolor}{d} & \textcolor{gregoriocolor}{e} & \textcolor{gregoriocolor}{f} & \textcolor{gregoriocolor}{g} & \textcolor{gregoriocolor}{h} & \textcolor{gregoriocolor}{i} & \textcolor{gregoriocolor}{k} & \textcolor{gregoriocolor}{l} & \textcolor{gregoriocolor}{m} & \textcolor{gregoriocolor}{n} & \textcolor{gregoriocolor}{p} & \textcolor{gregoriocolor}{q} & \textcolor{gregoriocolor}{r} & \textcolor{gregoriocolor}{s} & \textcolor{gregoriocolor}{t} & \textcolor{gregoriocolor}{u} \\
 28 & 29 & 30 & 1 & 2 & 3 & 4 & 5 & 6 & 7 & 8 & 9 & 10 & 11 & 12 & 13 & 14 & 15 & 16 \\
\end{tabularx}
\vspace{0.5\baselineskip}
\noindent\begin{tabularx}{\linewidth}{*{12}{>{\centering\arraybackslash}X}}
 \textcolor{gregoriocolor}{A} & \textcolor{gregoriocolor}{B} & \textcolor{gregoriocolor}{C} & \textcolor{gregoriocolor}{D} & \textcolor{gregoriocolor}{E} & F & \textcolor{gregoriocolor}{F} & \textcolor{gregoriocolor}{G} & \textcolor{gregoriocolor}{H} & \textcolor{gregoriocolor}{M} & \textcolor{gregoriocolor}{N} & \textcolor{gregoriocolor}{P} \\
 17 & 18 & 19 & 20 & 21 & 22 & 22 & 23 & 24 & 25 & 26 & 27 \\
\end{tabularx}

\begin{paracol}{2}
\selectlanguage{latin}
\lettrine[lines=2]{S}{ancti} Joánnis Chrysóstomi, Epíscopi Constantinopolitáni, Confessóris et 
 Ecclésiæ Doctóris, cæléstis Oratórum sacrórum 
 Patróni; qui décimo octávo Kaléndas Octóbris obdormívit in Dómino. 
 Ejus sacrum corpus, sub Theodósio junióre, hac die Constantinópolim, inde 
 póstea Romam translátum fuit, et in Basílica Príncipis Apostolórum cónditum.
\switchcolumn
\selectlanguage{english}
\lettrine[lines=2]{S}{t.} John Chrysostom, Bishop of Constantinople, confessor and doctor of the 
 Church, and the heavenly patron of preachers, who fell asleep in the Lord on 
 the 14th of September. His holy body was brought to Constantinople on 
 this day in the reign of Theodosius the younger; it was afterwards taken to 
 Rome and placed in the basilica of the Prince of the Apostles.
\switchcolumn*
\selectlanguage{latin}
Bríxiæ natális sanctæ Angelæ Meríci Vírginis, 
 ex tértio Ordine sancti Francísci, quæ Societátem Vírginum sanctæ Ursulæ 
 instítuit, quarum præcípuum munus esset dirígere adolescéntulas in vias 
 Dómini. Ejus tamen festívitas 
 Kaléndis Júnii celebrátur.
\switchcolumn
\selectlanguage{english}
At Brescia, the birthday of St. Angela Merici, virgin, who belonged to the 
 Third Order of St. Francis, and who founded the Order of the Nuns of St. 
 Ursula, whose principal aim is to direct young girls in the ways of the 
 Lord. Her feast is celebrated on the 1st 
 of June.
\switchcolumn*
\selectlanguage{latin}
Apud Cenómanos, in Gállia, deposítio sancti 
 Juliáni, ejúsdem urbis primi Epíscopi, quem sanctus Petrus illuc ad 
 prædicándum Evangélium misit.
\switchcolumn
\selectlanguage{english}
At Le Mans in France, the death of St. Julian, the first bishop of that 
 city, who was sent there by St. Peter to preach the Gospel.
\switchcolumn*
\selectlanguage{latin}
Soræ sancti Juliáni Mártyris, qui, in 
 persecutióne Antoníni, sub Flaviáno Præside, comprehénsus est, et, cum 
 idolórum templum, dum ipse torquerétur, corruísset, martyrii corónam, 
 truncáto cápite, accépit.
\switchcolumn
\selectlanguage{english}
At Sora, St. Julian, martyr, who, being arrested in the persecution of 
 Antoninus, was beheaded because a pagan temple had fallen to the ground 
 while he was being tortured.
\switchcolumn*
\selectlanguage{latin}
In Africa sancti Avíti Mártyris.
\switchcolumn
\selectlanguage{english}
In Africa, St. Avitus, martyr.
\switchcolumn*
\selectlanguage{latin}
Ibídem sanctórum Mártyrum Dátii, Reátri et Sociórum, qui in persecutióne 
 Wandálica passi sunt.
\switchcolumn
\selectlanguage{english}
In the same country, the holy martyrs Datius, Reatrus, and their companions, 
 who suffered in the persecution of the Vandals.
\switchcolumn*
\selectlanguage{latin}
Item sanctórum Datívi, Juliáni, Vincéntii atque aliórum vigínti septem 
 Mártyrum.
\switchcolumn
\selectlanguage{english}
Also, the holy martyrs Dativus, Julian, Vincent, and twenty-seven others.
\switchcolumn*
\selectlanguage{latin}
Romæ sancti Vitaliáni Papæ.
\switchcolumn
\selectlanguage{english}
At Rome, St. Vitalian, pope.
\switchcolumn*
\selectlanguage{latin}
In monastério Bobacénsi, in Gállia, sancti Mauri Abbátis.
\switchcolumn
\selectlanguage{english}
In the monastery of Bobbio in France, St. Maur, abbot.
\switchcolumn*
\selectlanguage{latin}
\end{paracol}


% ---- martyrology/mart01/mart0128.htm
\needspace{10\baselineskip}
\begin{paracol}{2}
\selectlanguage{latin}
\begin{center}{\color{gregoriocolor} Quinto Kaléndas Februárii. 
 Luna\dots\ }\end{center}
\switchcolumn
\selectlanguage{english}
\begin{center}{\color{gregoriocolor} The Twenty-Eighth Day of 
 January. The\dots\ Day of the Moon.}\end{center}
\end{paracol}

\noindent\begin{tabularx}{\linewidth}{*{19}{>{\centering\arraybackslash}X}}
 \textcolor{gregoriocolor}{a} & \textcolor{gregoriocolor}{b} & \textcolor{gregoriocolor}{c} & \textcolor{gregoriocolor}{d} & \textcolor{gregoriocolor}{e} & \textcolor{gregoriocolor}{f} & \textcolor{gregoriocolor}{g} & \textcolor{gregoriocolor}{h} & \textcolor{gregoriocolor}{i} & \textcolor{gregoriocolor}{k} & \textcolor{gregoriocolor}{l} & \textcolor{gregoriocolor}{m} & \textcolor{gregoriocolor}{n} & \textcolor{gregoriocolor}{p} & \textcolor{gregoriocolor}{q} & \textcolor{gregoriocolor}{r} & \textcolor{gregoriocolor}{s} & \textcolor{gregoriocolor}{t} & \textcolor{gregoriocolor}{u} \\
 29 & 30 & 1 & 2 & 3 & 4 & 5 & 6 & 7 & 8 & 9 & 10 & 11 & 12 & 13 & 14 & 15 & 16 & 17 \\
\end{tabularx}
\vspace{0.5\baselineskip}
\noindent\begin{tabularx}{\linewidth}{*{12}{>{\centering\arraybackslash}X}}
 \textcolor{gregoriocolor}{A} & \textcolor{gregoriocolor}{B} & \textcolor{gregoriocolor}{C} & \textcolor{gregoriocolor}{D} & \textcolor{gregoriocolor}{E} & F & \textcolor{gregoriocolor}{F} & \textcolor{gregoriocolor}{G} & \textcolor{gregoriocolor}{H} & \textcolor{gregoriocolor}{M} & \textcolor{gregoriocolor}{N} & \textcolor{gregoriocolor}{P} \\
 18 & 19 & 20 & 21 & 22 & 23 & 23 & 24 & 25 & 26 & 27 & 28 \\
\end{tabularx}

\begin{paracol}{2}
\selectlanguage{latin}
\lettrine[lines=2]{S}{ancti} Petri Nolásci Confessóris, qui Ordinis beátæ 
 Maríæ de Mercéde redemptiónis captivórum éxstitit Fundátor, et octávo Kaléndas Januárii obdormívit in Dómino.
\switchcolumn
\selectlanguage{english}
\lettrine[lines=2]{S}{t.} Peter Nolasco, confessor, who founded the Order of Our Lady of Ransom 
 for the redemption of captives, and who fell asleep in the Lord on the 25th 
 of December.
\switchcolumn*
\selectlanguage{latin}
Romæ sanctæ Agnétis, Vírginis et Mártyris, 
 secúndo.
\switchcolumn
\selectlanguage{english}
At Rome, the second feast of St. Agnes, virgin and martyr.
\switchcolumn*
\selectlanguage{latin}
Alexandríæ natális sancti Cyrílli, ejúsdem 
 urbis Epíscopi, Confessóris et Ecclésiæ Doctóris; qui, cathólicæ fídei 
 præclaríssimus propugnátor, doctrína et sanctitáte illústris quiévit in 
 pace. Ejus tamen festívitas quinto Idus Februárii celebrátur.
\switchcolumn
\selectlanguage{english}
At Alexandria, the birthday of St. Cyril, bishop of that city, a most 
 celebrated defender of the Catholic faith, who died in peace, with a great 
 reputation for learning and sanctity. His feast, however, is kept on 
 the ninth of February.
\switchcolumn*
\selectlanguage{latin}
Romæ sancti Flaviáni Mártyris, qui sub 
 Diocletiáno passus est.
\switchcolumn
\selectlanguage{english}
At Rome, St. Flavian, martyr, who suffered under Diocletian.
\switchcolumn*
\selectlanguage{latin}
Alexandríæ pássio plurimórum sanctórum Mártyrum, 
 qui, hac ipsa die, a factióne Syriáni, Ducis Ariáni, dum in Ecclésia synáxim 
 ágerent, divérso mortis génere sunt interémpti.
\switchcolumn
\selectlanguage{english}
At Alexandria, the commemoration of many holy martyrs, who, while they were 
 at Mass in the church on this day, were put to death in different ways by 
 the followers of Syrianus, an Arian general.
\switchcolumn*
\selectlanguage{latin}
Apollóniæ sanctórum Mártyrum, Léucii, Thyrsi et 
 Calliníci; qui, témpore Décii Imperatóris, váriis tormentórum genéribus 
 cruciáti, ac primus et últimus abscissióne cápitis, médius cælésti voce 
 evocátus spíritum reddens, martyrium consummárunt.
\switchcolumn
\selectlanguage{english}
At Appollonia, the holy martyrs Thrysus, Leucius, and Callinicus, who were 
 made to undergo various torments in the time of Emperor Decius. 
 Thyrsus and Callinicus were beheaded; Leucius, called by a heavenly voice, 
 yielded his soul unto God.
\switchcolumn*
\selectlanguage{latin}
In Thebáide sanctórum Mártyrum Leónidæ et 
 Sociórum, qui, témpore Diocletiáni, palmam martyrii sunt assecúti.
\switchcolumn
\selectlanguage{english}
In Thebais, the holy martyrs Leonides and his companions, who obtained the 
 palm of martyrdom in the time of Diocletian.
\switchcolumn*
\selectlanguage{latin}
Cæsaraugústæ, in Hispánia, sancti Valérii 
 Epíscopi.
\switchcolumn
\selectlanguage{english}
At Saragossa in Spain, St. Valerius, bishop.
\switchcolumn*
\selectlanguage{latin}
Conchæ, in Hispánia, natális sancti Juliáni 
 Epíscopi, qui, érogans in páuperes bona Ecclésiæ, ópera mánuum sibi more 
 Apostólico victum quærens, clarus miráculis quiévit in pace.
\switchcolumn
\selectlanguage{english}
At Cuenca in Spain, the birthday of St. Julian, bishop, who, after bestowing 
 the goods of the Church on the poor, like the apostles, supported himself by the work of 
 his hands, and went to his God famous for his miracles.
\switchcolumn*
\selectlanguage{latin}
In monastério Reomaénsi, in Gállia, deposítio sancti Joánnis Presbyteri, 
 viri Deo devóti.
\switchcolumn
\selectlanguage{english}
In the monastery of Rheims in France, the death of the holy priest John, a 
 devout man of God.
\switchcolumn*
\selectlanguage{latin}
In Palæstína sancti Jacóbi Eremítæ, qui, post 
 lapsum, diu, pæniténtiæ causa, in sepúlcro látuit, et clarus miráculis 
 migrávit ad Dóminum.
\switchcolumn
\selectlanguage{english}
In Palestine, St. James, hermit, who hid himself a long time in a sepulchre 
 in order to do penance for a fault he had committed, and, being celebrated 
 for miracles, departed for heaven.
\switchcolumn*
\selectlanguage{latin}
\end{paracol}


% ---- martyrology/mart01/mart0129.htm
\needspace{10\baselineskip}
\begin{paracol}{2}
\selectlanguage{latin}
\begin{center}{\color{gregoriocolor} Quarto Kaléndas Februárii. 
 Luna\dots\ }\end{center}
\switchcolumn
\selectlanguage{english}
\begin{center}{\color{gregoriocolor} The Twenty-Ninth Day of 
 January. The\dots\ Day of the Moon.}\end{center}
\end{paracol}

\noindent\begin{tabularx}{\linewidth}{*{19}{>{\centering\arraybackslash}X}}
 \textcolor{gregoriocolor}{a} & \textcolor{gregoriocolor}{b} & \textcolor{gregoriocolor}{c} & \textcolor{gregoriocolor}{d} & \textcolor{gregoriocolor}{e} & \textcolor{gregoriocolor}{f} & \textcolor{gregoriocolor}{g} & \textcolor{gregoriocolor}{h} & \textcolor{gregoriocolor}{i} & \textcolor{gregoriocolor}{k} & \textcolor{gregoriocolor}{l} & \textcolor{gregoriocolor}{m} & \textcolor{gregoriocolor}{n} & \textcolor{gregoriocolor}{p} & \textcolor{gregoriocolor}{q} & \textcolor{gregoriocolor}{r} & \textcolor{gregoriocolor}{s} & \textcolor{gregoriocolor}{t} & \textcolor{gregoriocolor}{u} \\
 30 & 1 & 2 & 3 & 4 & 5 & 6 & 7 & 8 & 9 & 10 & 11 & 12 & 13 & 14 & 15 & 16 & 17 & 18 \\
\end{tabularx}
\vspace{0.5\baselineskip}
\noindent\begin{tabularx}{\linewidth}{*{12}{>{\centering\arraybackslash}X}}
 \textcolor{gregoriocolor}{A} & \textcolor{gregoriocolor}{B} & \textcolor{gregoriocolor}{C} & \textcolor{gregoriocolor}{D} & \textcolor{gregoriocolor}{E} & F & \textcolor{gregoriocolor}{F} & \textcolor{gregoriocolor}{G} & \textcolor{gregoriocolor}{H} & \textcolor{gregoriocolor}{M} & \textcolor{gregoriocolor}{N} & \textcolor{gregoriocolor}{P} \\
 19 & 20 & 21 & 22 & 23 & 24 & 24 & 25 & 26 & 27 & 28 & 29 \\
\end{tabularx}

\begin{paracol}{2}
\selectlanguage{latin}
\lettrine[lines=2]{S}{ancti} Francísci Salésii, Epíscopi Gebennénsis, Confessóris et Ecclésiæ 
 Doctóris, ómnium Scriptórum catholicórum, diáriis aliísve scriptis in vulgus 
 edéndis sapiéntiam Christiánam illustrántium ac provehéntium et tutántium, 
 peculiáris apud Deum Patróni; qui migrávit in cælum quinto Kaléndas Januárii, 
 sed hac die, ob Translatiónem córporis ejus, potíssimum cólitur.
\switchcolumn
\selectlanguage{english}
\lettrine[lines=2]{S}{t.} Francis de Sales, bishop of Geneva, confessor and doctor of the Church, 
 special patron before God of all Catholic writers in explaining, promoting, 
 or defending Christian doctrine either by publishing journals or other 
 writings in the vernacular. He departed to heaven on the 28th of 
 December, but because of the transfer of his body on this day, his feast is 
 now celebrated.
\switchcolumn*
\selectlanguage{latin}
Tréviris deposítio beáti Valérii Epíscopi, qui fuit discípulus sancti Petri 
 Apóstoli.
\switchcolumn
\selectlanguage{english}
At Treves, the death of the blessed bishop Valerius, disciple of the apostle 
 St. Peter.
\switchcolumn*
\selectlanguage{latin}
Romæ, via Nomentána, natális sanctórum Mártyrum 
 Pápiæ et Mauri mílitum, témpore Diocletiáni Imperatóris; quorum ora jussit 
 Laodícius, Urbis Præféctus, ad primam Christi confessiónem lapídibus 
 contúndi, et sic eos in cárcerem trahi, ac póstea fústibus cædi, atque ad 
 últimum plumbátis pércuti, donec exspirárent.
\switchcolumn
\selectlanguage{english}
At Rome, on the Via Nomentana, the birthday of the holy martyrs Papias and 
 Maur, soldiers under Emperor Diocletian. At their first confession of 
 Christ they had their mouths bruised with stones and were thrown into prison 
 by order of Laodicius, prefect of the city. Afterwards they were 
 beaten with rods and with leaded whips until they expired.
\switchcolumn*
\selectlanguage{latin}
Perúsiæ sancti Constántii, Epíscopi et Mártyris; 
 qui, una cum Sóciis, sub Marco Aurélio Imperatóre, ob fídei defensiónem, 
 martyrii corónam accépit.
\switchcolumn
\selectlanguage{english}
At Perugia, in the time of Marcus Aurelius, St. Constantius, bishop and 
 martyr, who, together with his companions, received the crown of martyrdom 
 for the defence of the faith.
\switchcolumn*
\selectlanguage{latin}
Medioláni sancti Aquilíni Presbyteri, qui, ab Ariánis gládio in gútture 
 transfíxus, martyrio coronátur.
\switchcolumn
\selectlanguage{english}
At Milan, St. Aquilinus, priest, who was crowned with martyrdom by having 
 his throat pierced with a sword by the Arians.
\switchcolumn*
\selectlanguage{latin}
Edéssæ, in Syria, sanctórum Mártyrum Sarbélii 
 et Bárbeæ soróris, qui, a beáto Barsimǽo Epíscopo baptizáti, ambo, in 
 persecutióne Trajáni, sub Lysia Præside, martyrio coronáti sunt.
\switchcolumn
\selectlanguage{english}
At Edessa in Syria, the holy martyrs Sabellus and his sister Barbea, who 
 were baptized by the blessed bishop Barsimaeus, and crowned with martyrdom 
 in the persecution of Trajan, under the governor Lysias.
\switchcolumn*
\selectlanguage{latin}
In território Tricassíno sancti Sabiniáni Mártyris, qui, jubénte Aureliáno 
 Imperatóre, pro fide Christi decollátus est.
\switchcolumn
\selectlanguage{english}
In the territory of Troyes, St. Sabinian, martyr, who was beheaded for the 
 faith of Christ by command of the emperor Aurelian.
\switchcolumn*
\selectlanguage{latin}
Apud Bitúricas, in Aquitánia, sancti Sulpícii Sevéri 
 Epíscopi, virtútibus et eruditióne conspícui.
\switchcolumn
\selectlanguage{english}
At Bourges, St. Sulpice Severus, bishop, distinguished by his virtues and 
 learning.
\switchcolumn*
\selectlanguage{latin}
\end{paracol}


% ---- martyrology/mart01/mart0130.htm
\needspace{10\baselineskip}
\begin{paracol}{2}
\selectlanguage{latin}
\begin{center}{\color{gregoriocolor} Tértio Kaléndas Februárii. 
 Luna\dots\ }\end{center}
\switchcolumn
\selectlanguage{english}
\begin{center}{\color{gregoriocolor} The Thirtieth Day of 
 January. The\dots\ Day of the Moon.}\end{center}
\end{paracol}

\noindent\begin{tabularx}{\linewidth}{*{19}{>{\centering\arraybackslash}X}}
 \textcolor{gregoriocolor}{a} & \textcolor{gregoriocolor}{b} & \textcolor{gregoriocolor}{c} & \textcolor{gregoriocolor}{d} & \textcolor{gregoriocolor}{e} & \textcolor{gregoriocolor}{f} & \textcolor{gregoriocolor}{g} & \textcolor{gregoriocolor}{h} & \textcolor{gregoriocolor}{i} & \textcolor{gregoriocolor}{k} & \textcolor{gregoriocolor}{l} & \textcolor{gregoriocolor}{m} & \textcolor{gregoriocolor}{n} & \textcolor{gregoriocolor}{p} & \textcolor{gregoriocolor}{q} & \textcolor{gregoriocolor}{r} & \textcolor{gregoriocolor}{s} & \textcolor{gregoriocolor}{t} & \textcolor{gregoriocolor}{u} \\
 1 & 2 & 3 & 4 & 5 & 6 & 7 & 8 & 9 & 10 & 11 & 12 & 13 & 14 & 15 & 16 & 17 & 18 & 19 \\
\end{tabularx}
\vspace{0.5\baselineskip}
\noindent\begin{tabularx}{\linewidth}{*{12}{>{\centering\arraybackslash}X}}
 \textcolor{gregoriocolor}{A} & \textcolor{gregoriocolor}{B} & \textcolor{gregoriocolor}{C} & \textcolor{gregoriocolor}{D} & \textcolor{gregoriocolor}{E} & F & \textcolor{gregoriocolor}{F} & \textcolor{gregoriocolor}{G} & \textcolor{gregoriocolor}{H} & \textcolor{gregoriocolor}{M} & \textcolor{gregoriocolor}{N} & \textcolor{gregoriocolor}{P} \\
 20 & 21 & 22 & 23 & 24 & 25 & 25 & 26 & 27 & 28 & 29 & 30 \\
\end{tabularx}

\begin{paracol}{2}
\selectlanguage{latin}
\lettrine[lines=2]{S}{anctæ} Martínæ, Vírginis et Mártyris, cujus 
 dies natális Kaléndis Januárii recólitur.
\switchcolumn
\selectlanguage{english}
\lettrine[lines=2]{S}{t.} Martina, virgin and martyr, who is commemorated on her birthday, the 
 first day of this month.
\switchcolumn*
\selectlanguage{latin}
Edéssæ, in Syria, sancti Barsimǽi Epíscopi, 
 qui, cum Gentíles plúrimos convertísset ad fidem et præmisísset ad corónam, 
 eos secútus est, sub Trajáno, cum palma martyrii.
\switchcolumn
\selectlanguage{english}
At Edessa in Syria, in the reign of Trajan, St. Barsimaeus, bishop, who 
 converted many Gentiles to the faith, sent them before him to gain their 
 crown, and then followed them with the palm of martyrdom.
\switchcolumn*
\selectlanguage{latin}
Antiochíæ pássio beáti Hippólyti Presbyteri, 
 qui, decéptus aliquándiu schísmate Nováti, sed, operánte grátia Christi, 
 corréctus, ad unitátem Ecclésiæ rédiit, pro qua et in qua póstea illústre 
 martyrium consummávit. Hic, rogátus a suis quænam secta vérior esse 
 servándam quam Petri Cáthedra custodíret, júgulum præbuit.
\switchcolumn
\selectlanguage{english}
At Antioch, the passion of the blessed Hippolytus, priest, who for a short 
 time deceived by the Novatian schism, was converted by the grace of Christ, 
 and returned to the unity of the Church, for which and in which he 
 afterwards underwent a glorious martyrdom. Being asked by the 
 schismatics, which was the better side, he said that he detested the 
 doctrine of Novatus, and that the faith which the Chair of Peter taught 
 ought to be professed, after which he was beheaded.
\switchcolumn*
\selectlanguage{latin}
In Africa pássio sanctórum Mártyrum Feliciáni, Philappiáni et aliórum centum 
 vigínti quátuor.
\switchcolumn
\selectlanguage{english}
In Africa, the passion of the holy martyrs Felician, Philappian, and one 
 hundred and twenty-four others.
\switchcolumn*
\selectlanguage{latin}
Item beáti Alexándri, qui, in persecutióne Décii, comprehénsus est, ac, longævæ 
 ætátis veneránda canítie et confessióne iteráta respléndens, inter 
 carníficum torménta réddidit spíritum.
\switchcolumn
\selectlanguage{english}
Blessed Alexander, a man of venerable aspect and advanced age, who was 
 apprehended in the persecution of Decius. After gloriously and 
 repeatedly confessing the faith, in the midst of torments he gave up his 
 soul unto God.
\switchcolumn*
\selectlanguage{latin}
Edéssæ, in Syria, sancti Barsis Epíscopi, dono 
 curatiónum illústris; qui, a Valénte, Imperatóre Ariáno, in díssitas 
 regiónes ob fidem cathólicam relegátus, ac tríplici mutatióne fatigátus 
 exsílii, vitam finívit.
\switchcolumn
\selectlanguage{english}
At Edessa in Syria, St Barses, bishop, renowned for the gift of healing 
 diseases. For holding to the Catholic faith he was banished by the 
 Arian emperor Valens into the most remote corner of that country, and he 
 there ended his days.
\switchcolumn*
\selectlanguage{latin}
Hierosólymis natális sancti Matthíæ Epíscopi, 
 de quo mira et plena fídei gesta narrántur; qui, sub Hadriáno, multa pro 
 Christo perpéssus est, ac demum in pace quiévit.
\switchcolumn
\selectlanguage{english}
At Jerusalem, the birthday of St. Matthias, bishop, of whom wonderful deeds 
 are related which were inspired by faith. After having endured many 
 trials for Christ under Adrian, he passed away in peace.
\switchcolumn*
\selectlanguage{latin}
Papíæ sancti Armentárii, Epíscopi et 
 Confessóris.
\switchcolumn
\selectlanguage{english}
At Pavia, St. Armentarius, bishop and confessor.
\switchcolumn*
\selectlanguage{latin}
In Malbódio, Hannóniæ monastério, sanctæ 
 Aldegúndis Vírginis, témpore Dagobérti Regis.
\switchcolumn
\selectlanguage{english}
In Hainaut, in the monastery of Maubeuge, St. Aldegund, virgin, who lived in 
 the time of King Dagobert.
\switchcolumn*
\selectlanguage{latin}
Vitérbii sanctæ Hyacínthæ de Mariscóttis 
 Vírginis, ex tértio sancti Francísci Ordine Sanctimoniális, pæniténtia et 
 caritáte insígnis; quam Pius Papa Séptimus Sanctis adscrípsit.
\switchcolumn
\selectlanguage{english}
At Viterbo, the holy virgin Hyacinth Mariscotti, a nun of the Third Order of 
 St. Francis, distinguished for the virtues of penance and charity. She 
 was inscribed among the saints by Pope Pius VII.
\switchcolumn*
\selectlanguage{latin}
Medioláni sanctæ Savínæ, féminæ religiosíssimæ, 
 quæ, ad sepúlcra sanctórum Náboris et Felícis Mártyrum orans, obdormívit in 
 Dómino.
\switchcolumn
\selectlanguage{english}
At Milan, St. Savina, a most religious woman, who went to rest in the Lord 
 while praying at the tomb of the holy martyrs Nabor and Felix.
\switchcolumn*
\selectlanguage{latin}
In território Parisiénsi sanctæ Bathíldis 
 Regínæ, sanctitáte et miraculórum glória præcláræ.
\switchcolumn
\selectlanguage{english}
In the district of Paris, St. Bathilde, queen, renowned for the worthiness 
 of her miracles and her sanctity.
\switchcolumn*
\selectlanguage{latin}
\end{paracol}


% ---- martyrology/mart01/mart0131.htm
\needspace{10\baselineskip}
\begin{paracol}{2}
\selectlanguage{latin}
\begin{center}{\color{gregoriocolor} Prídie Kaléndas Februárii. 
 Luna\dots\ }\end{center}
\switchcolumn
\selectlanguage{english}
\begin{center}{\color{gregoriocolor} The Thirty-First Day of 
 January. The\dots\ Day of the Moon.}\end{center}
\end{paracol}

\noindent\begin{tabularx}{\linewidth}{*{19}{>{\centering\arraybackslash}X}}
 \textcolor{gregoriocolor}{a} & \textcolor{gregoriocolor}{b} & \textcolor{gregoriocolor}{c} & \textcolor{gregoriocolor}{d} & \textcolor{gregoriocolor}{e} & \textcolor{gregoriocolor}{f} & \textcolor{gregoriocolor}{g} & \textcolor{gregoriocolor}{h} & \textcolor{gregoriocolor}{i} & \textcolor{gregoriocolor}{k} & \textcolor{gregoriocolor}{l} & \textcolor{gregoriocolor}{m} & \textcolor{gregoriocolor}{n} & \textcolor{gregoriocolor}{p} & \textcolor{gregoriocolor}{q} & \textcolor{gregoriocolor}{r} & \textcolor{gregoriocolor}{s} & \textcolor{gregoriocolor}{t} & \textcolor{gregoriocolor}{u} \\
 2 & 3 & 4 & 5 & 6 & 7 & 8 & 9 & 10 & 11 & 12 & 13 & 14 & 15 & 16 & 17 & 18 & 19 & 20 \\
\end{tabularx}
\vspace{0.5\baselineskip}
\noindent\begin{tabularx}{\linewidth}{*{12}{>{\centering\arraybackslash}X}}
 \textcolor{gregoriocolor}{A} & \textcolor{gregoriocolor}{B} & \textcolor{gregoriocolor}{C} & \textcolor{gregoriocolor}{D} & \textcolor{gregoriocolor}{E} & F & \textcolor{gregoriocolor}{F} & \textcolor{gregoriocolor}{G} & \textcolor{gregoriocolor}{H} & \textcolor{gregoriocolor}{M} & \textcolor{gregoriocolor}{N} & \textcolor{gregoriocolor}{P} \\
 21 & 22 & 23 & 24 & 25 & 26 & 26 & 27 & 28 & 29 & 30 & 1 \\
\end{tabularx}

\begin{paracol}{2}
\selectlanguage{latin}
\lettrine[lines=2]{A}{ugústæ} Taurinórum sancti Joánnis Bosco, 
 Confessóris, Societátis Salesiánæ ac Institúti Filiárum Maríæ Auxiliatrícis 
 Fundatóris, animárum zelo et fídei propagándæ conspícui, quem Pius Papa 
 Undécimus Sanctórum fastis adscrípsit.
\switchcolumn
\selectlanguage{english}
\lettrine[lines=2]{A}{t} Turin,the birthday of St. John Bosco, confessor, founder of the Salesian 
 Congregation and of the Institute of the Daughters of Mary, Help of 
 Christians. Conspicuous for his zeal for souls and for the propagation 
 of the faith, he was canonized by Pope Pius XI.
\switchcolumn*
\selectlanguage{latin}
Romæ, via Portuénsi, sanctórum Mártyrum Cyri et 
 Joánnis, qui pro confessióne Christi, post multa torménta, cápite truncáti 
 sunt.
\switchcolumn
\selectlanguage{english}
At Rome, on the road to Ostia, the holy martyrs Cyrus and John, who were 
 beheaded after suffering many torments for the name of Christ.
\switchcolumn*
\selectlanguage{latin}
Alexandríæ natális sancti Metráni Mártyris, 
 qui, sub Décio Imperatóre, cum ad jussiónem Paganórum nollet ímpia verba 
 proférre, hi totum ejus corpus fústibus collisérunt, vultúmque et óculos 
 præacútis cálamis terebrántes, cum cruciátibus expulérunt ipsum extra urbem, 
 ibíque lapídibus oppréssum interemérunt.
\switchcolumn
\selectlanguage{english}
At Alexandria, in the time of Emperor Decius, the birthday of St. Metran, 
 martyr, who, because he refused to utter blasphemous words at the bidding of 
 the pagans, had his body all bruised with blows, and his face and eyes 
 pierced with sharp pointed reeds. He was then driven out of the city 
 and stoned to death.
\switchcolumn*
\selectlanguage{latin}
Ibídem sanctórum Mártyrum Saturníni, Thyrsi et Victóris.
\switchcolumn
\selectlanguage{english}
In the same place, the holy martyrs Saturninus, Thyrsus, and Victor.
\switchcolumn*
\selectlanguage{latin}
Item Alexandríæ sanctórum Mártyrum Tharsícii, 
 Zótici, Cyríaci et Sociórum.
\switchcolumn
\selectlanguage{english}
Also at Alexandria, the holy martyrs Tharsicius, Zoticus, Cyriacus, and 
 their companions.
\switchcolumn*
\selectlanguage{latin}
Cyzici, in Hellespónto, sanctæ Tryphǽnæ 
 Mártyris, quæ, plúrimis torméntis superátis, a tauro demum necáta, martyrii 
 palmam proméruit.
\switchcolumn
\selectlanguage{english}
At Cyzicum in the Hellespont, St. Triphenes, martyr, who overcame various 
 torments, but was finally killed by a bull, and thus merited the palm of 
 martyrdom.
\switchcolumn*
\selectlanguage{latin}
Mútinæ sancti Geminiáni Epíscopi, miraculórum 
 glória conspícui.
\switchcolumn
\selectlanguage{english}
At Modena, St. Geminian, bishop, made illustrious by his miracles.
\switchcolumn*
\selectlanguage{latin}
In Província Mediolanénsi sancti Júlii, Presbyteri et Confessóris, témpore 
 Imperatóris Theodósii.
\switchcolumn
\selectlanguage{english}
In the province of Milan, St. Julius, priest and confessor, in the reign of 
 the emperor Theodosius.
\switchcolumn*
\selectlanguage{latin}
Neápoli sancti Francísci Xavérii-Maríæ Biánchi, 
 Confessóris, Clérici reguláris sancti Pauli, signis, donis cæléstibus et 
 admirábili patiéntia illústris, quem Pius Papa Duodécimus ad suprémos 
 honóres Sanctórum éxtulit.
\switchcolumn
\selectlanguage{english}
At Naples, St. Francis Xavier-Maria Bianchi, confessor, cleric regular of 
 St. Paul, renowned for miracles, heavenly gifts and an admirable patience, 
 whom Pope Pius XII raised to the supreme honour of sainthood.
\switchcolumn*
\selectlanguage{latin}
Romæ sanctæ Marcéllæ Víduæ, cujus præcláras 
 laudes beátus Hierónymus scripsit.
\switchcolumn
\selectlanguage{english}
At Rome, St. Marcella, widow, whose meritorious deeds are related by St. 
 Jerome.
\switchcolumn*
\selectlanguage{latin}
Item Romæ beátæ Ludovícæ Albertóniæ, Víduæ 
 Románæ, ex tértio Ordine sancti Francísci, virtútibus claræ.
\switchcolumn
\selectlanguage{english}
Also at Rome, blessed Louise Albertonia, a Roman widow, member of the Third 
 Order of St. Francis, distinguished for her virtues.
\switchcolumn*
\selectlanguage{latin}
Eódem die Translátio sancti Marci Evangelístæ, 
 cum sacrum ejus corpus ex Alexandría, a bárbaris tunc occupáta, Venétias 
 allátum, ibídem in majóri Ecclésia, ejus nómine consecráta, 
 honorificentíssime cónditum fuit.
\switchcolumn
\selectlanguage{english}
The same day, the transfer of the revered body of the Evangelist St. Mark 
 from the city of Alexandria in Egypt, then occupied by barbarians, to 
 Venice, and with the greatest honours placed in the large church dedicated 
 to his name.
\switchcolumn*
\selectlanguage{latin}
\end{paracol}


\setrunningtitles{Februarius}{February}
% ---- martyrology/mart02/mart0201.htm
\needspace{10\baselineskip}
\begin{paracol}{2}
\selectlanguage{latin}
\begin{center}{\color{gregoriocolor} Kaléndis Februárii. 
 Luna\dots\ }\end{center}
\switchcolumn
\selectlanguage{english}
\begin{center}{\color{gregoriocolor} The First Day of 
 February. The\dots\ Day of the Moon.}\end{center}
\end{paracol}

\noindent\begin{tabularx}{\linewidth}{*{19}{>{\centering\arraybackslash}X}}
 \textcolor{gregoriocolor}{a} & \textcolor{gregoriocolor}{b} & \textcolor{gregoriocolor}{c} & \textcolor{gregoriocolor}{d} & \textcolor{gregoriocolor}{e} & \textcolor{gregoriocolor}{f} & \textcolor{gregoriocolor}{g} & \textcolor{gregoriocolor}{h} & \textcolor{gregoriocolor}{i} & \textcolor{gregoriocolor}{k} & \textcolor{gregoriocolor}{l} & \textcolor{gregoriocolor}{m} & \textcolor{gregoriocolor}{n} & \textcolor{gregoriocolor}{p} & \textcolor{gregoriocolor}{q} & \textcolor{gregoriocolor}{r} & \textcolor{gregoriocolor}{s} & \textcolor{gregoriocolor}{t} & \textcolor{gregoriocolor}{u} \\
 3 & 4 & 5 & 6 & 7 & 8 & 9 & 10 & 11 & 12 & 13 & 14 & 15 & 16 & 17 & 18 & 19 & 20 & 21 \\
\end{tabularx}
\vspace{0.5\baselineskip}
\noindent\begin{tabularx}{\linewidth}{*{12}{>{\centering\arraybackslash}X}}
 \textcolor{gregoriocolor}{A} & \textcolor{gregoriocolor}{B} & \textcolor{gregoriocolor}{C} & \textcolor{gregoriocolor}{D} & \textcolor{gregoriocolor}{E} & F & \textcolor{gregoriocolor}{F} & \textcolor{gregoriocolor}{G} & \textcolor{gregoriocolor}{H} & \textcolor{gregoriocolor}{M} & \textcolor{gregoriocolor}{N} & \textcolor{gregoriocolor}{P} \\
 22 & 23 & 24 & 25 & 26 & 27 & 27 & 28 & 29 & 30 & 1 & 2 \\
\end{tabularx}

\begin{paracol}{2}
\selectlanguage{latin}
\lettrine[lines=2]{S}{ancti} Ignátii, Epíscopi Antiochéni et Mártyris, qui glorióse martyrium 
 consummávit tertiodécimo Kaléndas Januárii.
\switchcolumn
\selectlanguage{english}
\lettrine[lines=2]{S}{t.} Ignatius, bishop of Antioch and martyr, who gloriously suffered 
 martyrdom on the 20th of December.
\switchcolumn*
\selectlanguage{latin}
Smyrnæ sancti Piónii, Presbyteri et Mártyris; 
 qui, post apologías pro fide Christiána conscríptas, post squalórem cárceris, 
 ubi multos fratrum ad martyrii tolerántiam suis exhortatiónibus roborávit, 
 tandem, cruciátibus multis vexátus, clavis confíxus et ardénti rogo 
 superpósitus, beátum pro Christo finem sortítus est. Cum ipso autem at 
 álii quíndecim passi sunt.
\switchcolumn
\selectlanguage{english}
At Smyrna, St. Pionius, priest and martyr, who, after writing apologies for 
 the Catholic faith, and after suffering imprisonment in a loathsome dungeon, 
 where by his exhortations he encouraged many of his brethren even to 
 martyrdom, and after enduring excruciating pains from being pierced with 
 nails and laid on a hot fire, ended happily his life for Christ. With 
 him suffered fifteen others.
\switchcolumn*
\selectlanguage{latin}
Ravénnæ sancti Sevéri Epíscopi, qui, ob 
 præclára mérita, signo colúmbæ fuit eléctus.
\switchcolumn
\selectlanguage{english}
At Ravenna, the holy bishop Severus, whose great virtues deserved that he 
 should be raised to the episcopate, which action was confirmed with the sign 
 of a dove.
\switchcolumn*
\selectlanguage{latin}
In civitáte Tricastína, in Gállia, sancti Pauli Epíscopi, cujus vita 
 virtútibus cláruit, et mors pretiósa miráculis commendátur.
\switchcolumn
\selectlanguage{english}
At Trois-Châteaux in France, St. Paul, bishop, whose life was eminent for 
 virtues, and whose death was made precious by miracles.
\switchcolumn*
\selectlanguage{latin}
Apud Kildáriam, in Hibérnia, sanctæ Brígidæ 
 Vírginis, quæ, cum lignum altáris tetigísset in testimónium virginitátis suæ, 
 lignum ipsum statim víride factum est.
\switchcolumn
\selectlanguage{english}
At Kildare in Ireland, St. Bridget, virgin. Once, when she touched the 
 wood of an altar, it immediately sprouted into life, in testimony of her 
 virginity.
\switchcolumn*
\selectlanguage{latin}
Apud Castrum Florentínum, in Etrúria, beátæ 
 Viridiánæ, Vírginis reclúsæ, ex Ordine Vallis Umbrósæ.
\switchcolumn
\selectlanguage{english}
At Castel-Fiorentino in Tuscany, the blessed virgin Veridiana, a recluse of 
 the Order Vallombrosa.
\switchcolumn*
\selectlanguage{latin}
\end{paracol}


% ---- martyrology/mart02/mart0202.htm
\needspace{10\baselineskip}
\begin{paracol}{2}
\selectlanguage{latin}
\begin{center}{\color{gregoriocolor} Quarto Nonas Februárii. 
 Luna\dots\ }\end{center}
\switchcolumn
\selectlanguage{english}
\begin{center}{\color{gregoriocolor} The Second Day of 
 February. The\dots\ Day of the Moon.}\end{center}
\end{paracol}

\noindent\begin{tabularx}{\linewidth}{*{19}{>{\centering\arraybackslash}X}}
 \textcolor{gregoriocolor}{a} & \textcolor{gregoriocolor}{b} & \textcolor{gregoriocolor}{c} & \textcolor{gregoriocolor}{d} & \textcolor{gregoriocolor}{e} & \textcolor{gregoriocolor}{f} & \textcolor{gregoriocolor}{g} & \textcolor{gregoriocolor}{h} & \textcolor{gregoriocolor}{i} & \textcolor{gregoriocolor}{k} & \textcolor{gregoriocolor}{l} & \textcolor{gregoriocolor}{m} & \textcolor{gregoriocolor}{n} & \textcolor{gregoriocolor}{p} & \textcolor{gregoriocolor}{q} & \textcolor{gregoriocolor}{r} & \textcolor{gregoriocolor}{s} & \textcolor{gregoriocolor}{t} & \textcolor{gregoriocolor}{u} \\
 4 & 5 & 6 & 7 & 8 & 9 & 10 & 11 & 12 & 13 & 14 & 15 & 16 & 17 & 18 & 19 & 20 & 21 & 22 \\
\end{tabularx}
\vspace{0.5\baselineskip}
\noindent\begin{tabularx}{\linewidth}{*{12}{>{\centering\arraybackslash}X}}
 \textcolor{gregoriocolor}{A} & \textcolor{gregoriocolor}{B} & \textcolor{gregoriocolor}{C} & \textcolor{gregoriocolor}{D} & \textcolor{gregoriocolor}{E} & F & \textcolor{gregoriocolor}{F} & \textcolor{gregoriocolor}{G} & \textcolor{gregoriocolor}{H} & \textcolor{gregoriocolor}{M} & \textcolor{gregoriocolor}{N} & \textcolor{gregoriocolor}{P} \\
 23 & 24 & 25 & 26 & 27 & 28 & 28 & 29 & 30 & 1 & 2 & 3 \\
\end{tabularx}

\begin{paracol}{2}
\selectlanguage{latin}
\lettrine[lines=2]{P}{urificátio} beátæ Maríæ Vírginis, quæ a Græcis 
 Hypapánte Dómini appellátur.
\switchcolumn
\selectlanguage{english}
\lettrine[lines=2]{T}{he} Purification of the Blessed Virgin Mary, called by the Greeks the 
 Hypapante (meeting) of the Lord.
\switchcolumn*
\selectlanguage{latin}
Cæsaréæ, in Palæstína, sancti Cornélii 
 Centuriónis, quem beátus Petrus Apóstolus baptizávit, et apud præfátam urbem 
 Episcopáli sublimávit honóre.
\switchcolumn
\selectlanguage{english}
At Caesarea in Palestine, St. Cornelius, a centurion, whom the blessed 
 apostle Peter baptized, and raised to the episcopal dignity in that city.
\switchcolumn*
\selectlanguage{latin}
Romæ, via Salária, pássio sancti Aproniáni 
 Commentariénsis, qui, adhuc Gentílis, cum sanctum Sisínium e cárcere 
 edúceret ut Laodício Præfécto præsentáret, vocémque de cælo factam audíret: 
 « Veníte, benedícti Patris mei, percípite regnum, quod vobis parátum est a 
 constitutióne mundi », credens baptizátur, et póstea, in confessióne Dómini, 
 vitæ finem senténtia capitáli accépit.
\switchcolumn
\selectlanguage{english}
At Rome, on the Salarian Way, the passion of St. Apronian, a notary. 
 While he was yet a heathen, and was leading St. Sisinius out of prison to 
 present him before the governor Laodicius, he head a voice from heaven 
 saying: Come, ye blessed of my Father, possess the kingdom which I have 
 prepared for you from the beginning of the world.`` At once he 
 believed, was baptized, and after confessing our Lord, received sentence of 
 death.
\switchcolumn*
\selectlanguage{latin}
Item Romæ sanctórum Mártyrum Fortunáti, 
 Feliciáni, Firmi et Cándidi.
\switchcolumn
\selectlanguage{english}
Also at Rome, the holy martyrs Fortunatus, Felician, Firmus and Candidus.
\switchcolumn*
\selectlanguage{latin}
Aureliánis, in Gállia, sancti Flósculi Epíscopi.
\switchcolumn
\selectlanguage{english}
At Orleans in France, the holy bishop Flosculus.
\switchcolumn*
\selectlanguage{latin}
Cantuáriæ, in Anglia, natális sancti Lauréntii 
 Epíscopi, qui, post sanctum Augustínum, eam Ecclésiam gubernávit, et Regem 
 ipsum ad fidem convértit.
\switchcolumn
\selectlanguage{english}
At Canterbury in England, the birthday of St. Lawrence, bishop, who 
 succeeded St. Augustine in the government of that church, and converted the 
 king himself to the faith.
\switchcolumn*
\selectlanguage{latin}
Prati, in Etrúria, sanctæ Catharínæ de Rícciis, 
 Vírginis Florentínæ, ex Ordine Prædicatórum, ob cæléstium donórum copiam 
 insígnis; quam Benedíctus Décimus quartus, Póntifex Máximus, sanctárum 
 Vírginum fastis adscrípsit.
\switchcolumn
\selectlanguage{english}
At Prati in Tuscany, St. Catherine de Ricci, a virgin of Florence, member of 
 the Order of Preachers, famous for a plenitude of heavenly gifts. Pope 
 Benedict XIV placed her name on the roll of holy virgins.
\switchcolumn*
\selectlanguage{latin}
Burdígalæ, in Gállia, sanctæ Joánnæ de 
 Léstonnac, Víduæ, Institúti Filiárum beátæ Vírginis Maríæ Fundatrícis, 
 caritátis stúdio ac puellárum instituendárum cura insígnis; quam Pius Papa 
 Duodécimus Sanctárum número accénsuit.
\switchcolumn
\selectlanguage{english}
At Bordeaux in France, St. Joan de Lestonnac, widow, foundress of the 
 Daughters of the blessed Virgin Mary, renowned for the practice of charity 
 and the care of her girl pupils, and whom Pope Pius XII raised to the number 
 of the saints.
\switchcolumn*
\selectlanguage{latin}
\end{paracol}


% ---- martyrology/mart02/mart0203.htm
\needspace{10\baselineskip}
\begin{paracol}{2}
\selectlanguage{latin}
\begin{center}{\color{gregoriocolor} Tértio Nonas Februárii. 
 Luna\dots\ }\end{center}
\switchcolumn
\selectlanguage{english}
\begin{center}{\color{gregoriocolor} The Third Day of 
 February. The\dots\ Day of the Moon.}\end{center}
\end{paracol}

\noindent\begin{tabularx}{\linewidth}{*{19}{>{\centering\arraybackslash}X}}
 \textcolor{gregoriocolor}{a} & \textcolor{gregoriocolor}{b} & \textcolor{gregoriocolor}{c} & \textcolor{gregoriocolor}{d} & \textcolor{gregoriocolor}{e} & \textcolor{gregoriocolor}{f} & \textcolor{gregoriocolor}{g} & \textcolor{gregoriocolor}{h} & \textcolor{gregoriocolor}{i} & \textcolor{gregoriocolor}{k} & \textcolor{gregoriocolor}{l} & \textcolor{gregoriocolor}{m} & \textcolor{gregoriocolor}{n} & \textcolor{gregoriocolor}{p} & \textcolor{gregoriocolor}{q} & \textcolor{gregoriocolor}{r} & \textcolor{gregoriocolor}{s} & \textcolor{gregoriocolor}{t} & \textcolor{gregoriocolor}{u} \\
 5 & 6 & 7 & 8 & 9 & 10 & 11 & 12 & 13 & 14 & 15 & 16 & 17 & 18 & 19 & 20 & 21 & 22 & 23 \\
\end{tabularx}
\vspace{0.5\baselineskip}
\noindent\begin{tabularx}{\linewidth}{*{12}{>{\centering\arraybackslash}X}}
 \textcolor{gregoriocolor}{A} & \textcolor{gregoriocolor}{B} & \textcolor{gregoriocolor}{C} & \textcolor{gregoriocolor}{D} & \textcolor{gregoriocolor}{E} & F & \textcolor{gregoriocolor}{F} & \textcolor{gregoriocolor}{G} & \textcolor{gregoriocolor}{H} & \textcolor{gregoriocolor}{M} & \textcolor{gregoriocolor}{N} & \textcolor{gregoriocolor}{P} \\
 24 & 25 & 26 & 27 & 28 & 29 & 29 & 30 & 1 & 2 & 3 & 4 \\
\end{tabularx}

\begin{paracol}{2}
\selectlanguage{latin}
\lettrine[lines=2]{S}{ebáste,} in Arménia, pássio sancti Blásii, Epíscopi et Mártyris; qui, 
 multórum patrátor miraculórum, sub Agricoláo Præside, 
 post diútinam cæsiónem, atque in ligno suspensiónem, ubi férreis pectínibus 
 carnes ejus dirúptæ sunt, post tetérrimum cárcerem et in lacum demersiónem, 
 unde salvus exívit, tandem, jubénte eódem Júdice, una cum duóbus púeris, 
 cápite truncátur. Ante ipsum vero septem mulíeres, quæ guttas 
 sánguinis ex ejúsdem Mártyris córpore defluéntes, dum torquerétur, 
 colligébant, proptérea, deprehénsæ quod essent Christiánæ, omnes, post dira 
 torménta, gládio percússæ sunt.
\switchcolumn
\selectlanguage{english}
\lettrine[lines=2]{A}{t} Sebaste in Armenia, in the time of the governor Agricolaus, the passion 
 of St. Blase, bishop and martyr, who, after working many miracles, was 
 scourged for a long time, suspended from a tree where his flesh was 
 lacerated with iron combs. He was then imprisoned in a dark dungeon, 
 thrown into a lake from which he came out safe, and finally, by order of the 
 judge, he and two boys were beheaded. Before him, seven women who were 
 gathering the drops of his blood during his torture, were recognized as 
 Christians, and after undergoing severe torments, were put to death by the 
 sword.
\switchcolumn*
\selectlanguage{latin}
In Africa sancti Celeríni Diáconi, qui, decem et novem dies custódia 
 cárceris septus, in nervo et ferro variísque pœnis 
 gloriósus fuit Christi Conféssor; et, dum inexpugnábili firmitáte certáminis 
 sui vicit adversárium, vincéndi céteris viam fecit.
\switchcolumn
\selectlanguage{english}
In Africa, St. Celerinus, deacon, who was kept nineteen days in prison 
 burdened with fetters, and who gloriously confessed Christ in the midst of 
 afflictions. By overcoming the enemy with invincible constancy, he 
 shewed to others the road to victory.
\switchcolumn*
\selectlanguage{latin}
Ibídem sanctórum trium Mártyrum, ipsíus Celeríni Diáconi consanguineórum, 
 scílicet Laurentíni pátrui, Ignátii avúnculi, et Celerínæ 
 áviæ, qui ántea martyrio coronáti fúerant; de quorum ómnium gloriósis 
 láudibus exstat beáti Cypriáni epístola.
\switchcolumn
\selectlanguage{english}
In the same place, three holy martyrs who were relatives of the same deacon 
 Celerinus; his father's brother Laurentinus, his mother's brother Ignatius 
 and his grandmother Celerina. They were crowned with martyrdom 
 earlier, and were praised highly in an epistle by blessed Cyprian.
\switchcolumn*
\selectlanguage{latin}
Item in Africa sanctórum Mártyrum Felícis, Symphrónii, Hippólyti et Sociórum.
\switchcolumn
\selectlanguage{english}
Likewise in Africa, the holy martyrs Felix, Symphronius, Hippolytus, and 
 their companions.
\switchcolumn*
\selectlanguage{latin}
In óppido Vapíngo, in Gállia, sanctórum Tigídis 
 et Remédii Episcopórum.
\switchcolumn
\selectlanguage{english}
In the town of Gap in France, the holy bishops Tigides and Remedius.
\switchcolumn*
\selectlanguage{latin}
Lugdúni, in Gállia, sanctórum Lupicíni et Felícis,
 ítidem Episcopórum.
\switchcolumn
\selectlanguage{english}
At Lyons in France, Saints Lupicinus and Felix, also bishops.
\switchcolumn*
\selectlanguage{latin}
Bremæ sancti Anschárii, Hamburgénsis ac póstea 
 Breménsis simul Epíscopi, qui Suécos et Danos ad Christi fidem convértit, et 
 a Gregório Papa Quarto Legátus Apostólicus totíus Septentriónis fuit 
 institútus.
\switchcolumn
\selectlanguage{english}
At Bremen, St. Ansgar, bishop of Hamburg and later of Bremen, who converted 
 the Swedes and the Danes to the faith of Christ. He was appointed 
 Apostolic Delegate of all the North by Pope Gregory IV.
\switchcolumn*
\selectlanguage{latin}
\end{paracol}


% ---- martyrology/mart02/mart0204.htm
\needspace{10\baselineskip}
\begin{paracol}{2}
\selectlanguage{latin}
\begin{center}{\color{gregoriocolor} Prídie Nonas Februárii. 
 Luna\dots\ }\end{center}
\switchcolumn
\selectlanguage{english}
\begin{center}{\color{gregoriocolor} The Fourth Day of 
 February. The\dots\ Day of the Moon.}\end{center}
\end{paracol}

\noindent\begin{tabularx}{\linewidth}{*{19}{>{\centering\arraybackslash}X}}
 \textcolor{gregoriocolor}{a} & \textcolor{gregoriocolor}{b} & \textcolor{gregoriocolor}{c} & \textcolor{gregoriocolor}{d} & \textcolor{gregoriocolor}{e} & \textcolor{gregoriocolor}{f} & \textcolor{gregoriocolor}{g} & \textcolor{gregoriocolor}{h} & \textcolor{gregoriocolor}{i} & \textcolor{gregoriocolor}{k} & \textcolor{gregoriocolor}{l} & \textcolor{gregoriocolor}{m} & \textcolor{gregoriocolor}{n} & \textcolor{gregoriocolor}{p} & \textcolor{gregoriocolor}{q} & \textcolor{gregoriocolor}{r} & \textcolor{gregoriocolor}{s} & \textcolor{gregoriocolor}{t} & \textcolor{gregoriocolor}{u} \\
 6 & 7 & 8 & 9 & 10 & 11 & 12 & 13 & 14 & 15 & 16 & 17 & 18 & 19 & 20 & 21 & 22 & 23 & 24 \\
\end{tabularx}
\vspace{0.5\baselineskip}
\noindent\begin{tabularx}{\linewidth}{*{12}{>{\centering\arraybackslash}X}}
 \textcolor{gregoriocolor}{A} & \textcolor{gregoriocolor}{B} & \textcolor{gregoriocolor}{C} & \textcolor{gregoriocolor}{D} & \textcolor{gregoriocolor}{E} & F & \textcolor{gregoriocolor}{F} & \textcolor{gregoriocolor}{G} & \textcolor{gregoriocolor}{H} & \textcolor{gregoriocolor}{M} & \textcolor{gregoriocolor}{N} & \textcolor{gregoriocolor}{P} \\
 25 & 26 & 27 & 28 & 29 & 1 & 30 & 1 & 2 & 3 & 4 & 5 \\
\end{tabularx}

\begin{paracol}{2}
\selectlanguage{latin}
\lettrine[lines=2]{S}{ancti} Andréæ Corsíni, ex Ordine Carmelitárum, 
 Epíscopi Fæsuláni et Confessóris; cujus dies natális ágitur octávo Idus 
 Januárii.
\switchcolumn
\selectlanguage{english}
\lettrine[lines=2]{S}{t.} Andrew Corsini, Carmelite bishop of Fiesole, confessor, whose birthday 
 is the 6th of January.
\switchcolumn*
\selectlanguage{latin}
Romæ sancti Eutychii Mártyris, qui illústre 
 martyrium consummávit, ac sepúltus est in cœmetério Callísti; ejúsque 
 sepúlcrum póstea sanctus Dámasus Papa vérsibus exornávit.
\switchcolumn
\selectlanguage{english}
At Rome, St. Eutychius, who endured a glorious martyrdom and was buried in 
 the cemetery of Callistus. Pope St. Damasus wrote an epitaph in verse 
 for his tomb.
\switchcolumn*
\selectlanguage{latin}
Thumi, in Ægypto, pássio beáti Philéæ, ejúsdem 
 civitátis Epíscopi, et Philorómi, Tribúni mílitum; qui, in persecutióne 
 Diocletiáni, cum a cognátis et amícis suadéri non possunt ut sibi párcerent, 
 ambo, datis cervícibus, palmas a Dómino meruérunt. Cum ipsis innúmera 
 étiam multitúdo fidélium ex eádem urbe, pastóris sui vestígia sequens, 
 martyrio coronáta est.
\switchcolumn
\selectlanguage{english}
At Thumis in Egypt, in the persecution of Diocletian, the passion of blessed 
 Philaeus, bishop of that city, and of Philoromus, military tribune, who 
 rejected the exhortations of their relatives and friends to save themselves, 
 offered themselves to death, and so merited immortal palms from God. 
 With them was crowned with martyrdom a numberless multitude of the faithful 
 of the same place, who followed the example of their pastor.
\switchcolumn*
\selectlanguage{latin}
Foro Semprónii sanctórum Mártyrum Aquilíni, Gémini, Gelásii, Magni et Donáti.
\switchcolumn
\selectlanguage{english}
At Fossombrone, the holy martyrs Aquilinus, Geminus, Gelasius, Magnus, and 
 Donatus.
\switchcolumn*
\selectlanguage{latin}
In regno Maravénsi apud Indos Orientáles, sancti Joánnis de Britto, 
 Sacerdótis e Societáte Jesu, qui cum multos infidéles ad fidem convertísset, 
 glorióso martyrio coronátus est.
\switchcolumn
\selectlanguage{english}
In Marava Kingdom in India, St. John de Britto, priest of the Society of 
 Jesus, who having converted many infidels to the faith, was gloriously 
 crowned with martyrdom.
\switchcolumn*
\selectlanguage{latin}
Trecis, in Gállia, sancti Aventíni, Presbyteri et Confessóris.
\switchcolumn
\selectlanguage{english}
At Troyes in France, St. Aventin, priest and confessor.
\switchcolumn*
\selectlanguage{latin}
Pelúsii, in Ægypto, sancti Isidóri, Presbyteri 
 et Mónachi, méritis et doctrína conspícui.
\switchcolumn
\selectlanguage{english}
At Pelusium in Egypt, St. Isidore, a monk renowned for merit and learning.
\switchcolumn*
\selectlanguage{latin}
Sempringhámiæ, in Anglia, sancti Gilbérti, 
 Presbyteri et Confessóris; qui Ordinis Sempringhamiénsis fuit Institútor.
\switchcolumn
\selectlanguage{english}
At Sempringham in England, St. Gilbert, priest and confessor, who founded a 
 religious order at Sempringham.
\switchcolumn*
\selectlanguage{latin}
In oppido Amatrícis, in Aprútio, deposítio sancti Joséphi a Leoníssa, 
 Sacerdótis ex Ordine Minórum Capuccinórum et Confessóris; quem, ob fídei prædicatiónem 
 a Mahumetánis dira perpéssum, labóribus apostólicis et miráculis clarum, 
 Benedíctus Décimus quartus, Póntifex Máximus, in Sanctórum cánonem rétulit.
\switchcolumn
\selectlanguage{english}
In the town of Amatrice, in the diocese of Rieti, the death of St. Joseph of 
 Leonissa, a Capuchin priest who suffered greatly from the Mohammedans. 
 As he was celebrated for his apostolic labours and miracles, he was placed 
 on the list of holy confessors by the Sovereign Pontiff, Benedict XIV.
\switchcolumn*
\selectlanguage{latin}
Bremæ Commemorátio sancti Rembérti, qui, sancti 
 Anschárii discípulus, in ipsíus locum, hac die, óbitum magístri sui próxime 
 subsequénti, olim Hamburgénsis simul ac Breménsis Epíscopus eléctus est.
\switchcolumn
\selectlanguage{english}
At Bremen, the commemoration of St. Rembert, who was a disciple of St. 
 Ansgar, and on this day took his place as bishop of Hamburg and Bremen, the 
 day after the death of his master.
\switchcolumn*
\selectlanguage{latin}
Bitúrcis, in Aquitánia, sanctæ Joánnæ de Valois, 
 Gálliæ Regínæ, Ordinis sanctíssimæ Annuntiatiónis beátæ Maríæ Vírginis 
 Fundatrícis, pietáte et singulári Crucis participatióne illústris, a Pio 
 Papa Duodécimo Sanctárum fastis adscríptæ.
\switchcolumn
\selectlanguage{english}
At Bourges in Aquitaine, St. Jane de Valois, Queen of France, foundress of 
 the Order of Sisters of the Annunciation of the blessed Virgin Mary, 
 renowned for her piety and singular devotion to the Cross, whom Pope Pius 
 XII added to the catalogue of saints.
\switchcolumn*
\selectlanguage{latin}
\end{paracol}


% ---- martyrology/mart02/mart0205.htm
\needspace{10\baselineskip}
\begin{paracol}{2}
\selectlanguage{latin}
\begin{center}{\color{gregoriocolor} Nonis Februárii. 
 Luna\dots\ }\end{center}
\switchcolumn
\selectlanguage{english}
\begin{center}{\color{gregoriocolor} The Fifth Day of 
 February. The\dots\ Day of the Moon.}\end{center}
\end{paracol}

\noindent\begin{tabularx}{\linewidth}{*{19}{>{\centering\arraybackslash}X}}
 \textcolor{gregoriocolor}{a} & \textcolor{gregoriocolor}{b} & \textcolor{gregoriocolor}{c} & \textcolor{gregoriocolor}{d} & \textcolor{gregoriocolor}{e} & \textcolor{gregoriocolor}{f} & \textcolor{gregoriocolor}{g} & \textcolor{gregoriocolor}{h} & \textcolor{gregoriocolor}{i} & \textcolor{gregoriocolor}{k} & \textcolor{gregoriocolor}{l} & \textcolor{gregoriocolor}{m} & \textcolor{gregoriocolor}{n} & \textcolor{gregoriocolor}{p} & \textcolor{gregoriocolor}{q} & \textcolor{gregoriocolor}{r} & \textcolor{gregoriocolor}{s} & \textcolor{gregoriocolor}{t} & \textcolor{gregoriocolor}{u} \\
 7 & 8 & 9 & 10 & 11 & 12 & 13 & 14 & 15 & 16 & 17 & 18 & 19 & 20 & 21 & 22 & 23 & 24 & 25 \\
\end{tabularx}
\vspace{0.5\baselineskip}
\noindent\begin{tabularx}{\linewidth}{*{12}{>{\centering\arraybackslash}X}}
 \textcolor{gregoriocolor}{A} & \textcolor{gregoriocolor}{B} & \textcolor{gregoriocolor}{C} & \textcolor{gregoriocolor}{D} & \textcolor{gregoriocolor}{E} & F & \textcolor{gregoriocolor}{F} & \textcolor{gregoriocolor}{G} & \textcolor{gregoriocolor}{H} & \textcolor{gregoriocolor}{M} & \textcolor{gregoriocolor}{N} & \textcolor{gregoriocolor}{P} \\
 26 & 27 & 28 & 29 & 1 & 2 & 1 & 2 & 3 & 4 & 5 & 6 \\
\end{tabularx}

\begin{paracol}{2}
\selectlanguage{latin}
\lettrine[lines=2]{C}{átanæ,} in Sicília, natális sanctæ Agathæ, 
 Vírginis et Mártyris; quæ, tempóribus Décii Imperatóris, sub Quinctiáno 
 Júdice, post álapas et cárcerem, post equúleum et torsiónes, post mamillárum 
 abscissiónem, post volutatiónem in téstulis et carbónibus, tandem in cárcere, 
 Deum precans, consummáta est.
\switchcolumn
\selectlanguage{english}
\lettrine[lines=2]{A}{t} Catana in Sicily, in the time of Emperor Decius and the judge Quinctian, 
 the birthday of St. Agatha, virgin and martyr. After being buffeted, 
 imprisoned, tortured, racked, dragged over pieces of earthenware and burning 
 coals, and having her breasts cut away, she completed her sacrifice in 
 prison while engaged in prayer.
\switchcolumn*
\selectlanguage{latin}
Nangasáchii, in Japónia, pássio vigínti sex Mártyrum, e quibus tres 
 Sacerdótes atque unus Cléricus et duo Láici ad Ordinem Minórum, tres, et in 
 eis unus quidem Cléricus, ad Societátem Jesu, ac septémdecim ad tértium 
 sancti Francísci Ordinem revocántur. Hi omnes pro cathólica fide, in 
 crucem acti et lanceárum íctibus perfóssi, inter divínas laudes ejusdémque 
 fídei prædicatiónem, glorióse occubuérunt; et a 
 Pio Nono, Pontífice Máximo, Sanctórum fastis adscrípti sunt.
\switchcolumn
\selectlanguage{english}
At Nagasaki in Japan, the passion of twenty-six martyrs. Three 
 priests, one cleric, and two lay brothers were members of the Order of 
 Friars Minor; one cleric was of the Society of Jesus, and seventeen belonged 
 to the Third Order of St. Francis. All of them, placed upon crosses 
 for the Catholic faith, and pierced with lances, gloriously died in praising 
 God and preaching that same faith. Their names were added to the roll 
 of saints by Pope Pius IX.
\switchcolumn*
\selectlanguage{latin}
In Ponto commemorátio plurimórum sanctórum Mártyrum, in persecutióne 
 Maximiáni; quorum álii plumbo liquénti perfúsi, álii acútis arundínibus in 
 únguibus cruciáti, ac multis horréndis vexáti torméntis, iisdémque sæpius 
 iterátis, palmas a Dómino et corónas illústri passióne meruérunt.
\switchcolumn
\selectlanguage{english}
In Pontus, during the persecution of Maximian, the commemoration of many 
 holy martyrs, some of whom had molten lead poured on them, others had sharp 
 reeds thrust under their nails, and were often horribly tormented in many 
 other ways. Thus, by their glorious suffering, they deserved to 
 receive at the hands of God palms of victory and their crowns.
\switchcolumn*
\selectlanguage{latin}
Alexandríæ sancti Isidóri, mílitis et Mártyris; 
 qui, in persecutióne Décii, a Numeriáno, exércitus Duce, ob Christi fidem, 
 cápite cæsus est.
\switchcolumn
\selectlanguage{english}
At Alexandria, during the persecution of Decius, St. Isidore, martyr, who 
 was beheaded for the faith of Christ by Numerian, general of the army.
\switchcolumn*
\selectlanguage{latin}
Viénnæ beáti Avíti, Epíscopi et Confessóris; 
 cujus fide, indústria atque admirábili doctrína ab Ariánæ hæresis 
 infestatióne sunt Gálliæ defénsæ.
\switchcolumn
\selectlanguage{english}
At Vienne, blessed Avitus, bishop and confessor, whose faith, labours, and 
 admirable learning protected France against the ravages of the Arian heresy.
\switchcolumn*
\selectlanguage{latin}
Sabióne, in Rhǽtia secúnda, sancti Ingenuíni 
 Epíscopi, cujus vita miráculis éxstitit gloriósa. Sacrum vero ipsíus 
 corpus Brixinónem póstea translátum fuit, ibíque honorífice asservátum.
\switchcolumn
\selectlanguage{english}
At Sabion in the Tyrol, St. Genuinus, bishop, whose illustrious life 
 abounded in miracles. His revered body was afterwards taken to Brixen 
 where a shrine was erected in his honour.
\switchcolumn*
\selectlanguage{latin}
Brixinóne sancti Albuíni Epíscopi, qui eam in civitátem e Sabióne Cáthedram 
 Episcopálem tránstulit, et ibídem, virtútum signis
 émicans, migrávit ad Dóminum.
\switchcolumn
\selectlanguage{english}
At Brixen, St. Albinus, bishop, who moved the Episcopal See from Sabion to 
 that city, and there, eminent by virtue of his miracles, passed to the Lord.
\switchcolumn*
\selectlanguage{latin}
\end{paracol}


% ---- martyrology/mart02/mart0206.htm
\needspace{10\baselineskip}
\begin{paracol}{2}
\selectlanguage{latin}
\begin{center}{\color{gregoriocolor} Octávo Idus Februárii. 
 Luna\dots\ }\end{center}
\switchcolumn
\selectlanguage{english}
\begin{center}{\color{gregoriocolor} The Sixth Day of 
 February. The\dots\ Day of the Moon.}\end{center}
\end{paracol}

\noindent\begin{tabularx}{\linewidth}{*{19}{>{\centering\arraybackslash}X}}
 \textcolor{gregoriocolor}{a} & \textcolor{gregoriocolor}{b} & \textcolor{gregoriocolor}{c} & \textcolor{gregoriocolor}{d} & \textcolor{gregoriocolor}{e} & \textcolor{gregoriocolor}{f} & \textcolor{gregoriocolor}{g} & \textcolor{gregoriocolor}{h} & \textcolor{gregoriocolor}{i} & \textcolor{gregoriocolor}{k} & \textcolor{gregoriocolor}{l} & \textcolor{gregoriocolor}{m} & \textcolor{gregoriocolor}{n} & \textcolor{gregoriocolor}{p} & \textcolor{gregoriocolor}{q} & \textcolor{gregoriocolor}{r} & \textcolor{gregoriocolor}{s} & \textcolor{gregoriocolor}{t} & \textcolor{gregoriocolor}{u} \\
 8 & 9 & 10 & 11 & 12 & 13 & 14 & 15 & 16 & 17 & 18 & 19 & 20 & 21 & 22 & 23 & 24 & 25 & 26 \\
\end{tabularx}
\vspace{0.5\baselineskip}
\noindent\begin{tabularx}{\linewidth}{*{12}{>{\centering\arraybackslash}X}}
 \textcolor{gregoriocolor}{A} & \textcolor{gregoriocolor}{B} & \textcolor{gregoriocolor}{C} & \textcolor{gregoriocolor}{D} & \textcolor{gregoriocolor}{E} & F & \textcolor{gregoriocolor}{F} & \textcolor{gregoriocolor}{G} & \textcolor{gregoriocolor}{H} & \textcolor{gregoriocolor}{M} & \textcolor{gregoriocolor}{N} & \textcolor{gregoriocolor}{P} \\
 27 & 28 & 29 & 1 & 2 & 3 & 2 & 3 & 4 & 5 & 6 & 7 \\
\end{tabularx}

\begin{paracol}{2}
\selectlanguage{latin}
\lettrine[lines=2]{S}{ancti} Titi, Epíscopi Creténsium et Confessóris, cujus dies natális occúrrit 
 prídie Nonas Januárii.
\switchcolumn
\selectlanguage{english}
\lettrine[lines=2]{S}{t.} Titus, confessor and bishop of Crete, whose birthday is on the fourth of 
 January.
\switchcolumn*
\selectlanguage{latin}
Cæsaréæ, in Cappadócia, natális sanctæ Dorothéæ, 
 Vírginis et Mártyris; quæ, sub Saprício, illíus Provínciæ Præside, primum 
 per equúlei extensiónem vexáta, dehinc palmis diutíssime cæsa, capitáli 
 senténtia ad últimum puníta est. In ejus confessióne Theóphilus quidam 
 scholásticus ad Christi fidem convérsus, et mox equúleo acérimme tortus, 
 novíssime gládio cæsus est.
\switchcolumn
\selectlanguage{english}
At Caesarea in Cappadocia, the birthday of St. Dorothy, virgin and martyr, 
 who was stretched on the rack, then scourged for a long time with the boughs 
 of a palm tree, and finally condemned to capital punishment by Sapricius, 
 governor of the province. Her noble confession of Christ converted a 
 lawyer named Theophilus, who also was tortured in a barbarous manner, and 
 finally put to death by the sword.
\switchcolumn*
\selectlanguage{latin}
Eméssæ, in Phœnícia, sancti Silváni Epíscopi, 
 qui cum eídem Ecclésiæ annis quadragínta præfuísset, tandem, sub Maximiáno 
 Imperatóre, una cum duóbus áliis, objéctus feris membratímque discérptus, 
 martyrii palmam accépit.
\switchcolumn
\selectlanguage{english}
At Emessa in Phoenicia, in the time of Emperor Maximian, St. Silvanus, 
 bishop, who, after having governed that church for forty years, was 
 delivered to the beasts with two other Christians, and having his limbs all 
 mangled, received the crown of martyrdom.
\switchcolumn*
\selectlanguage{latin}
Eódem die sanctórum Mártyrum Saturníni, Theóphili et Revocátæ.
\switchcolumn
\selectlanguage{english}
The same day, the holy martyrs Caturninus, Theophilus, and Revocata.
\switchcolumn*
\selectlanguage{latin}
Arvérnis, in Gállia, sancti Antholiáni Mártyris.
\switchcolumn
\selectlanguage{english}
In Auvergne in France, St. Atholian, martyr.
\switchcolumn*
\selectlanguage{latin}
Atrébati, in Gálliis, sancti Vedásti, ejúsdem 
 civitátis Epíscopi, cujus vita et mors plúrimis miráculis éxstitit gloriósa.
\switchcolumn
\selectlanguage{english}
At Arras in France, St. Vedast, bishop of that city. The glory of his 
 life and death is attested by many miracles.
\switchcolumn*
\selectlanguage{latin}
Elnóne, in Gállia, sancti Amándi, Epíscopi Trajecténsis, qui miráculis, cum 
 vivus tum mórtuus, glorióse refúlsit; cujus nómine póstmodum insignítum est 
 óppidum, in quo ille monastérium exstrúxerat et mortálem vitam absólverat.
\switchcolumn
\selectlanguage{english}
At Elnon in France, St. Amand, bishop of Maestricht, who was renowned for 
 his miracles during his life and in death. In the town which was named 
 after him, he lived and died in a monastery that he had built.
\switchcolumn*
\selectlanguage{latin}
Bonóniæ sancti Guaríni Cardinális et Epíscopi 
 Prænestíni, vitæ sanctitáte conspícui.
\switchcolumn
\selectlanguage{english}
At Bologna, St. Guarinus, bishop of Palestrina and cardinal, conspicuous for 
 his holiness of life.
\switchcolumn*
\selectlanguage{latin}
\end{paracol}


% ---- martyrology/mart02/mart0207.htm
\needspace{10\baselineskip}
\begin{paracol}{2}
\selectlanguage{latin}
\begin{center}{\color{gregoriocolor} Séptimo Idus Februárii. 
 Luna\dots\ }\end{center}
\switchcolumn
\selectlanguage{english}
\begin{center}{\color{gregoriocolor} The Seventh Day of 
 February. The\dots\ Day of the Moon.}\end{center}
\end{paracol}

\noindent\begin{tabularx}{\linewidth}{*{19}{>{\centering\arraybackslash}X}}
 \textcolor{gregoriocolor}{a} & \textcolor{gregoriocolor}{b} & \textcolor{gregoriocolor}{c} & \textcolor{gregoriocolor}{d} & \textcolor{gregoriocolor}{e} & \textcolor{gregoriocolor}{f} & \textcolor{gregoriocolor}{g} & \textcolor{gregoriocolor}{h} & \textcolor{gregoriocolor}{i} & \textcolor{gregoriocolor}{k} & \textcolor{gregoriocolor}{l} & \textcolor{gregoriocolor}{m} & \textcolor{gregoriocolor}{n} & \textcolor{gregoriocolor}{p} & \textcolor{gregoriocolor}{q} & \textcolor{gregoriocolor}{r} & \textcolor{gregoriocolor}{s} & \textcolor{gregoriocolor}{t} & \textcolor{gregoriocolor}{u} \\
 9 & 10 & 11 & 12 & 13 & 14 & 15 & 16 & 17 & 18 & 19 & 20 & 21 & 22 & 23 & 24 & 25 & 26 & 27 \\
\end{tabularx}
\vspace{0.5\baselineskip}
\noindent\begin{tabularx}{\linewidth}{*{12}{>{\centering\arraybackslash}X}}
 \textcolor{gregoriocolor}{A} & \textcolor{gregoriocolor}{B} & \textcolor{gregoriocolor}{C} & \textcolor{gregoriocolor}{D} & \textcolor{gregoriocolor}{E} & F & \textcolor{gregoriocolor}{F} & \textcolor{gregoriocolor}{G} & \textcolor{gregoriocolor}{H} & \textcolor{gregoriocolor}{M} & \textcolor{gregoriocolor}{N} & \textcolor{gregoriocolor}{P} \\
 28 & 29 & 1 & 2 & 3 & 4 & 3 & 4 & 5 & 6 & 7 & 8 \\
\end{tabularx}

\begin{paracol}{2}
\selectlanguage{latin}
\lettrine[lines=2]{S}{ancti} Romuáldi Abbátis, Monachórum Camaldulénsium Patris, cujus dies 
 natális tertiodécimo Kaléndas Júlii recensétur, sed festívitas hac die, ob 
 Translatiónem córporis ejus, potíssimum celebrátur.
\switchcolumn
\selectlanguage{english}
\lettrine[lines=2]{S}{t.} Romuald, founder of the Camaldolese monks, whose birthday is the 19th of 
 June, but celebrated today because of the transference of his body.
\switchcolumn*
\selectlanguage{latin}
Augústæ, cui nunc Londíni nomen, in Británnia, natális beáti Auguli Epíscopi, qui, ætátis cursu per martyrium expléto, 
 ætérna præmia suscípere méruit.
\switchcolumn
\selectlanguage{english}
At London, England, the birthday of blessed Augulus, bishop, who ended the 
 course of his life by martyrdom, and deserved to receive an eternal 
 recompense.
\switchcolumn*
\selectlanguage{latin}
In Phrygia sancti Adáuci Mártyris, qui, ex Itálico génere clarus, et omni 
 fere dignitátum gradu ab Imperatóribus insignítus, tandem, cum adhuc Quæstóris 
 offício fungerétur, martyrii coróna pro fídei defensióne dignátus est.
\switchcolumn
\selectlanguage{english}
In Phrygia, St. Adaucus, martyr, an Italian of noble birth, who was honoured 
 by the emperors with almost every dignity. While he was still 
 discharging the office of quæstor, he was judged worthy of the crown of 
 martyrdom for his defence of the faith.
\switchcolumn*
\selectlanguage{latin}
Ibídem plurimórum sanctórum Mártyrum, urbis uníus cívium, quorum dux erat 
 idem Adáucus; qui, cum omnes Christiáni essent, et constánter in fídei 
 confessióne persísterent, a Galério Maximiáno Imperatóre sunt igne consúmpti.
\switchcolumn
\selectlanguage{english}
Also, many holy martyrs, citizens of this same city of which Adaucus was 
 mayor. As they were all Christians, and persisted in the confession of 
 the faith, they were burned to death by Emperor Galerius Maximian.
\switchcolumn*
\selectlanguage{latin}
Heracléæ, in Ponto, sancti Theodóri, ductóris 
 mílitum, qui, Licínio imperánte, post multa torménta, truncátus cápite, 
 victor migrávit in cælum.
\switchcolumn
\selectlanguage{english}
At Heraclea, in the reign of Licinius, St. Theodore, a military officer, who 
 was beheaded after undergoing many torments, and went victoriously to 
 heaven.
\switchcolumn*
\selectlanguage{latin}
In Ægypto sancti Móysis, Epíscopi venerábilis, 
 qui primum in erémo vitam solitáriam duxit; deínde, peténte Regína 
 Saracenórum Máuvia, Epíscopus factus, gentem illam ferocíssimam magna ex 
 parte ad fidem convértit, et gloriósus méritis quiévit in pace.
\switchcolumn
\selectlanguage{english}
In Egypt, St. Moses, a venerable bishop, who first led a solitary life in 
 the desert, and afterwards, at the request of Mauvia, queen of the Saracens, 
 converted to the faith the greater part of that barbarous people. 
 Being made a bishop, and rich in merits, he peacefully went to his reward.
\switchcolumn*
\selectlanguage{latin}
Lucæ, in Túscia, deposítio sancti Richárdi, 
 Regis Anglórum, qui pater éxstitit sancti Willebáldi, Eistetténsis Epíscopi, 
 ac sanctæ Walbúrgæ Vírginis.
\switchcolumn
\selectlanguage{english}
At Lucca in Tuscany, the death of St. Richard, king of England. He was 
 the father of St. Willebald, bishop of Eichstadt, and of St. Walburga, 
 virgin.
\switchcolumn*
\selectlanguage{latin}
Bonóniæ sanctæ Juliánæ Víduæ.
\switchcolumn
\selectlanguage{english}
At Bologna, St. Juliana, widow.
\switchcolumn*
\selectlanguage{latin}
\end{paracol}


% ---- martyrology/mart02/mart0208.htm
\needspace{10\baselineskip}
\begin{paracol}{2}
\selectlanguage{latin}
\begin{center}{\color{gregoriocolor} Sexto Idus Februárii. 
 Luna\dots\ }\end{center}
\switchcolumn
\selectlanguage{english}
\begin{center}{\color{gregoriocolor} The Eighth Day of 
 February. The\dots\ Day of the Moon.}\end{center}
\end{paracol}

\noindent\begin{tabularx}{\linewidth}{*{19}{>{\centering\arraybackslash}X}}
 \textcolor{gregoriocolor}{a} & \textcolor{gregoriocolor}{b} & \textcolor{gregoriocolor}{c} & \textcolor{gregoriocolor}{d} & \textcolor{gregoriocolor}{e} & \textcolor{gregoriocolor}{f} & \textcolor{gregoriocolor}{g} & \textcolor{gregoriocolor}{h} & \textcolor{gregoriocolor}{i} & \textcolor{gregoriocolor}{k} & \textcolor{gregoriocolor}{l} & \textcolor{gregoriocolor}{m} & \textcolor{gregoriocolor}{n} & \textcolor{gregoriocolor}{p} & \textcolor{gregoriocolor}{q} & \textcolor{gregoriocolor}{r} & \textcolor{gregoriocolor}{s} & \textcolor{gregoriocolor}{t} & \textcolor{gregoriocolor}{u} \\
 10 & 11 & 12 & 13 & 14 & 15 & 16 & 17 & 18 & 19 & 20 & 21 & 22 & 23 & 24 & 25 & 26 & 27 & 28 \\
\end{tabularx}
\vspace{0.5\baselineskip}
\noindent\begin{tabularx}{\linewidth}{*{12}{>{\centering\arraybackslash}X}}
 \textcolor{gregoriocolor}{A} & \textcolor{gregoriocolor}{B} & \textcolor{gregoriocolor}{C} & \textcolor{gregoriocolor}{D} & \textcolor{gregoriocolor}{E} & F & \textcolor{gregoriocolor}{F} & \textcolor{gregoriocolor}{G} & \textcolor{gregoriocolor}{H} & \textcolor{gregoriocolor}{M} & \textcolor{gregoriocolor}{N} & \textcolor{gregoriocolor}{P} \\
 29 & 1 & 2 & 3 & 4 & 5 & 4 & 5 & 6 & 7 & 8 & 9 \\
\end{tabularx}

\begin{paracol}{2}
\selectlanguage{latin}
\lettrine[lines=2]{S}{ancti} Joánnis de Matha, Presbyteri et Confessóris, qui Ordinis sanctíssimæ 
 Trinitátis redemptiónis captivórum fuit Institútor, et sextodécimo Kaléndas 
 Januárii obdormívit in Dómino.
\switchcolumn
\selectlanguage{english}
\lettrine[lines=2]{S}{t.} John of Matha, priest and confessor, founder of the Order of the Most 
 Holy Trinity for the redemption of captives, who went to repose in the Lord 
 on the 17th of December.
\switchcolumn*
\selectlanguage{latin}
Somáschæ, in território Bergoménsi, natális 
 sancti Hierónymi Æmiliáni Confessóris, qui Congregatiónis Somáschæ Fundátor 
 éxstitit; atque, plúribus in vita et post mortem miráculis illústris, a 
 Cleménte Décimo tértio, Pontífice Máximo, Sanctórum fastis adscríptus est, 
 et a Pio Papa Undécimo universális orphanórum ac derelíctæ juventútis 
 Patrónus apud Deum eléctus et declarátus. Ejus tamen festívitas 
 tertiodécimo Kaléndas Augústi recólitur.
\switchcolumn
\selectlanguage{english}
At Somascha, in the district of Bergamo, the birthday of St. Jerome Emilian, 
 confessor, who was the founder of the Congregation of Somascha. 
 Illustrious both during his life and after death for many miracles, he was 
 inscribed in the roll of the saints by Pope Clement XIII. Pope Pius XI 
 chose and declared him to be the heavenly patron of orphans and abandoned 
 children. His feast is celebrated on the 20th of July.
\switchcolumn*
\selectlanguage{latin}
Romæ sanctórum Mártyrum Pauli, Lúcii et Cyríaci.
\switchcolumn
\selectlanguage{english}
At Rome, the holy martyrs Paul, Lucius, and Cyriacus.
\switchcolumn*
\selectlanguage{latin}
In Arménia minóre pássio sanctórum Mártyrum Dionysii,
 Æmiliáni et Sebastiáni.
\switchcolumn
\selectlanguage{english}
In Lesser Armenia, the birthday of the holy martyrs Denis, Aemilian, and 
 Sebastian.
\switchcolumn*
\selectlanguage{latin}
Constantinópoli natális sanctórum Mártyrum Monachórum monastérii Dii, 
 qui, ob defensiónem fídei cathólicæ, cum 
 tulíssent lítteras sancti Felícis Papæ Tértii advérsus Acácium, diríssime 
 cæsi sunt.
\switchcolumn
\selectlanguage{english}
At Constantinople, the birthday of the holy martyrs, monks of the monastery 
 of Dius. While bringing the letter of Pope St. Felix against Acacius, 
 they were barbarously killed for their defence of the Catholic faith.
\switchcolumn*
\selectlanguage{latin}
In Pérside commemorátio sanctórum Mártyrum, qui, sub Rege Persárum Cábade, 
 ob Christiánam fidem, divérsis suplíciis necáti sunt.
\switchcolumn
\selectlanguage{english}
In Persia, in the time of King Cabades, the commemoration of the holy 
 martyrs, who were put to death by various kinds of torments on account of 
 their Christian faith.
\switchcolumn*
\selectlanguage{latin}
Alexandríæ pássio sanctæ Coínthæ Mártyris, quam 
 Pagáni, sub Décio Imperatóre, corréptam et ad idóla perdúctam, hæc adoráre 
 cogébant; quod cum illa éxsecrans recusáret, ipsíus pedes vínculis 
 innexuérunt, eámque, trahéntes sic vinctam per civitátis platéas, horréndo 
 supplício discerpsérunt.
\switchcolumn
\selectlanguage{english}
At Alexandria, under Emperor Decius, the martyr St. Cointha, whom the pagans 
 seized, led to the idols, and urged to adore them. As she refused with 
 horror, they put her feet in chains, and dragged her through the streets of 
 the city, mangling her body in a most barbarous manner.
\switchcolumn*
\selectlanguage{latin}
Papíæ sancti Juvéntii Epíscopi, qui strénue in 
 Evangélio laborávit.
\switchcolumn
\selectlanguage{english}
At Pavia, St. Juventius, bishop, who laboured with zeal in preaching the 
 Gospel.
\switchcolumn*
\selectlanguage{latin}
Medioláni deposítio sancti Honoráti, Epíscopi 
 et Confessóris.
\switchcolumn
\selectlanguage{english}
At Milan, the death of St. Honoratus, bishop and confessor.
\switchcolumn*
\selectlanguage{latin}
Virodúni, in Gállia, sancti Pauli Epíscopi, miraculórum dono illústris.
\switchcolumn
\selectlanguage{english}
At Verdun in France, St. Paul, a bishop renowned for his miracles.
\switchcolumn*
\selectlanguage{latin}
Apud Murétum, in agro Lemovicénsi, natális sancti Stéphani Abbátis, qui 
 Grandimonténsis Ordinis Institútor fuit, ac virtútibus et miráculis cláruit.
\switchcolumn
\selectlanguage{english}
At Muret, near Limoges, the birthday of the abbot St. Stephen, founder of 
 the order of Grandmont, celebrated for his virtues and miracles.
\switchcolumn*
\selectlanguage{latin}
In monastério Vallis Umbrósæ Beáti Petri, 
 Cardinális et Epíscopi Albanénsis, ex Ordine Vallis Umbrósæ, cognoménto 
 Ignei, quia per ignem illæsus transívit.
\switchcolumn
\selectlanguage{english}
In the monastery of Vallombrosa, blessed Peter, cardinal and bishop of 
 Albano, a member of the Congregation of Vallombrosa of the Order of St. 
 Benedict. He was surnamed Igneus because he passed through fire 
 unharmed.
\switchcolumn*
\selectlanguage{latin}
\end{paracol}


% ---- martyrology/mart02/mart0209.htm
\needspace{10\baselineskip}
\begin{paracol}{2}
\selectlanguage{latin}
\begin{center}{\color{gregoriocolor} Quinto Idus Februárii. 
 Luna\dots\ }\end{center}
\switchcolumn
\selectlanguage{english}
\begin{center}{\color{gregoriocolor} The Ninth Day of 
 February. The\dots\ Day of the Moon.}\end{center}
\end{paracol}

\noindent\begin{tabularx}{\linewidth}{*{19}{>{\centering\arraybackslash}X}}
 \textcolor{gregoriocolor}{a} & \textcolor{gregoriocolor}{b} & \textcolor{gregoriocolor}{c} & \textcolor{gregoriocolor}{d} & \textcolor{gregoriocolor}{e} & \textcolor{gregoriocolor}{f} & \textcolor{gregoriocolor}{g} & \textcolor{gregoriocolor}{h} & \textcolor{gregoriocolor}{i} & \textcolor{gregoriocolor}{k} & \textcolor{gregoriocolor}{l} & \textcolor{gregoriocolor}{m} & \textcolor{gregoriocolor}{n} & \textcolor{gregoriocolor}{p} & \textcolor{gregoriocolor}{q} & \textcolor{gregoriocolor}{r} & \textcolor{gregoriocolor}{s} & \textcolor{gregoriocolor}{t} & \textcolor{gregoriocolor}{u} \\
 11 & 12 & 13 & 14 & 15 & 16 & 17 & 18 & 19 & 20 & 21 & 22 & 23 & 24 & 25 & 26 & 27 & 28 & 29 \\
\end{tabularx}
\vspace{0.5\baselineskip}
\noindent\begin{tabularx}{\linewidth}{*{12}{>{\centering\arraybackslash}X}}
 \textcolor{gregoriocolor}{A} & \textcolor{gregoriocolor}{B} & \textcolor{gregoriocolor}{C} & \textcolor{gregoriocolor}{D} & \textcolor{gregoriocolor}{E} & F & \textcolor{gregoriocolor}{F} & \textcolor{gregoriocolor}{G} & \textcolor{gregoriocolor}{H} & \textcolor{gregoriocolor}{M} & \textcolor{gregoriocolor}{N} & \textcolor{gregoriocolor}{P} \\
 1 & 2 & 3 & 4 & 5 & 6 & 5 & 6 & 7 & 8 & 9 & 10 \\
\end{tabularx}

\begin{paracol}{2}
\selectlanguage{latin}
\lettrine[lines=2]{S}{ancti} Cyrílli, Epíscopi Alexandríni, Confessóris et Ecclésiæ 
 Doctóris; cujus dies natális quinto Kaléndas Februárii recensétur.
\switchcolumn
\selectlanguage{english}
\lettrine[lines=2]{S}{t.} Cyril, bishop of Alexandria, confessor and doctor of the Church. 
 His birthday was mentioned on the 28th of January.
\switchcolumn*
\selectlanguage{latin}
Alexandríæ natális sanctæ Apollóniæ, Vírginis 
 et Mártyris, cui persecutóres, sub Décio, dentes omnes primum excussérunt. 
 Deínde, constrúcto ac succénso rogo, iídem commináti sunt, nisi cum eis ímpia verba proférret, vivam se eam incensúros; at illa, cum páululum intra 
 semetípsam deliberásset, repénte se de mánibus impiórum prorípuit, et in 
 ignem, quem paráverant, majóre Sancti Spíritus flamma intus æstuans, sponte 
 ita prosilívit, ut perterreréntur étiam ipsi crudelitátis auctóres, quod 
 prómptior invénta esset ad mortem fémina quam persecútor ad pœnam.
\switchcolumn
\selectlanguage{english}
At Alexandria, in the reign of Decius, the birthday of St. Apollonia, 
 virgin, who had all her teeth broken out by the persecutors; then, having 
 constructed and lighted a pyre, they threatened to burn her alive unless she 
 uttered with them certain impious words. Deliberating a while within 
 herself, she suddenly slipped from their grasp, and prompted by the greater 
 fire of the Holy Ghost within her, she rushed voluntarily into the fire 
 which they had prepared. Those responsible for her death were struck 
 with terror at the sight of a woman who was more willing to die than they to 
 kill her.
\switchcolumn*
\selectlanguage{latin}
Romæ pássio sanctórum Mártyrum Alexándri et 
 aliórum trigínta octo coronatórum.
\switchcolumn
\selectlanguage{english}
At Rome, the passion of the holy martyrs Alexander and thirty-eight others 
 crowned with him.
\switchcolumn*
\selectlanguage{latin}
In castéllo Lemelénsi, in Africa, sanctórum Mártyrum Primi et Donáti 
 Diaconórum, qui, cum altáre in Ecclésia tutaréntur, a Donatístis occísi sunt.
\switchcolumn
\selectlanguage{english}
In the village of Lamelum in Africa, the holy martyrs Primus and Donatus, 
 deacons, who were killed by the Donatists as they guarded the altar in the 
 church.
\switchcolumn*
\selectlanguage{latin}
Solis, in Cypro, sanctórum Mártyrum Ammónii et Alexándri.
\switchcolumn
\selectlanguage{english}
At Solum in Cyprus, the holy martyrs Ammonius and Alexander.
\switchcolumn*
\selectlanguage{latin}
Antiochíæ sancti Nicéphori Mártyris, qui sub 
 Valeriáno Imperatóre, cápite cæsus, martyrii corónam accépit.
\switchcolumn
\selectlanguage{english}
At Antioch, under Emperor Valerian, St. Nicephorus, martyr, who was beheaded 
 and thus received the crown of martyrdom.
\switchcolumn*
\selectlanguage{latin}
In monastério Fontanéllæ, in Gállia, sancti 
 Ansbérti, Rotomagénsis Epíscopi.
\switchcolumn
\selectlanguage{english}
In the monastery of Fontanelle in France, St. Ansbert, bishop of Rouen.
\switchcolumn*
\selectlanguage{latin}
Canúsii, in Apúlia, sancti Sabíni, Epíscopi et Confessóris; qui (ut beátus 
 Gregórius Papa refert), prophetíæ spíritu ac 
 miraculórum dono præditus, sibi jam cæco exhíbitum a fámulo, præmiis 
 corrúpto, venéni póculum divino agnóvit instínctu, sed, prænuntiáta mox a 
 Deo suménda de corruptóre vindícta signóque Crucis facto, venénum secúrus 
 ebíbit ac nullum ex eo nocuméntum accépit.
\switchcolumn
\selectlanguage{english}
At Canossa in Apulia, St. Sabinus, bishop and confessor. Blessed Pope 
 Gregory tells that he was endowed with the spirit of prophecy and the power 
 of miracles. After he had become blind, when a cup of poison was 
 offered to him by a servant who was bribed, he knew it by divine instinct. 
 He, however, declared that God would punish the one who had bribed the 
 servant, and, making the sign of the cross, he drank the poison without 
 anxiety and without harmful effect.
\switchcolumn*
\selectlanguage{latin}
\end{paracol}


% ---- martyrology/mart02/mart0210.htm
\needspace{10\baselineskip}
\begin{paracol}{2}
\selectlanguage{latin}
\begin{center}{\color{gregoriocolor} Quarto Idus Februárii. 
 Luna\dots\ }\end{center}
\switchcolumn
\selectlanguage{english}
\begin{center}{\color{gregoriocolor} The Tenth Day of 
 February. The\dots\ Day of the Moon.}\end{center}
\end{paracol}

\noindent\begin{tabularx}{\linewidth}{*{19}{>{\centering\arraybackslash}X}}
 \textcolor{gregoriocolor}{a} & \textcolor{gregoriocolor}{b} & \textcolor{gregoriocolor}{c} & \textcolor{gregoriocolor}{d} & \textcolor{gregoriocolor}{e} & \textcolor{gregoriocolor}{f} & \textcolor{gregoriocolor}{g} & \textcolor{gregoriocolor}{h} & \textcolor{gregoriocolor}{i} & \textcolor{gregoriocolor}{k} & \textcolor{gregoriocolor}{l} & \textcolor{gregoriocolor}{m} & \textcolor{gregoriocolor}{n} & \textcolor{gregoriocolor}{p} & \textcolor{gregoriocolor}{q} & \textcolor{gregoriocolor}{r} & \textcolor{gregoriocolor}{s} & \textcolor{gregoriocolor}{t} & \textcolor{gregoriocolor}{u} \\
 12 & 13 & 14 & 15 & 16 & 17 & 18 & 19 & 20 & 21 & 22 & 23 & 24 & 25 & 26 & 27 & 28 & 29 & 1 \\
\end{tabularx}
\vspace{0.5\baselineskip}
\noindent\begin{tabularx}{\linewidth}{*{12}{>{\centering\arraybackslash}X}}
 \textcolor{gregoriocolor}{A} & \textcolor{gregoriocolor}{B} & \textcolor{gregoriocolor}{C} & \textcolor{gregoriocolor}{D} & \textcolor{gregoriocolor}{E} & F & \textcolor{gregoriocolor}{F} & \textcolor{gregoriocolor}{G} & \textcolor{gregoriocolor}{H} & \textcolor{gregoriocolor}{M} & \textcolor{gregoriocolor}{N} & \textcolor{gregoriocolor}{P} \\
 2 & 3 & 4 & 5 & 6 & 7 & 6 & 7 & 8 & 9 & 10 & 11 \\
\end{tabularx}

\begin{paracol}{2}
\selectlanguage{latin}
\lettrine[lines=2]{A}{pud} montem Cassínum sanctæ Scholásticæ 
 Vírginis, soróris sancti Benedícti Abbátis; qui ejus ánimam, instar colúmbæ, 
 migrántem e córpore in cælum ascéndere vidit.
\switchcolumn
\selectlanguage{english}
\lettrine[lines=2]{O}{n} Monte Cassino, St. Scholastica, virgin, whose soul was seen by her 
 brother, St. Benedict, abbot, leaving her body in the form of a dove, and 
 ascending into heaven.
\switchcolumn*
\selectlanguage{latin}
Romæ sanctórum Mártyrum Zótici, Irenǽi, 
 Hyacínthi et Amántii.
\switchcolumn
\selectlanguage{english}
At Rome, the holy martyrs Zoticus, Irenaeus, Hyacinth, and Amantius.
\switchcolumn*
\selectlanguage{latin}
Ibídem, via Lavicána, sanctórum decem mílitum Mártyrum.
\switchcolumn
\selectlanguage{english}
In the same place, on the Via Lavicana, ten holy soldiers, martyrs.
\switchcolumn*
\selectlanguage{latin}
Item Romæ, via Appia, sanctæ Sotéris, Vírginis 
 et Mártyris; quæ (ut scribit sanctus Ambrósius), nóbili génere nata, paréntum 
 Consulátus et Præfectúras ob Christum contémpsit. Hæc, jussa idolis 
 immoláre, et non acquiéscens, gráviter et diutíssime álapis cæsa est; et, 
 cum cétera quoque pœnárum génera vicísset, demum, percússa gládio, læta 
 migrávit ad Sponsum.
\switchcolumn
\selectlanguage{english}
Also at Rome, on the Appian Way, St. Soter, virgin and martyr, descended of 
 a noble family, but as St. Ambrose mentions, for the love of Christ she set 
 at naught the consular and other dignitaries of her people. Upon her 
 refusal to sacrifice to the gods, she was for a long time cruelly scourged. 
 She overcame these and various other torments, then was struck with the 
 sword; and joyfully went to her heavenly spouse.
\switchcolumn*
\selectlanguage{latin}
In Campánia sancti Silváni, Epíscopi et Confessóris.
\switchcolumn
\selectlanguage{english}
In Campania, St. Silvanus, bishop and confessor.
\switchcolumn*
\selectlanguage{latin}
In Stábulo Rhodis, in territorio Senénsi, sancti Guiliélmi Eremítæ.
\switchcolumn
\selectlanguage{english}
At Malavalle, near Siena, St. William, hermit.
\switchcolumn*
\selectlanguage{latin}
In pago Rotomagénsi sanctæ Austrebértæ Vírginis, 
 miráculis célebris.
\switchcolumn
\selectlanguage{english}
In the diocese of Rouen, St. Austreberta, virgin, renowned for miracles.
\switchcolumn*
\selectlanguage{latin}
\end{paracol}


% ---- martyrology/mart02/mart0211.htm
\needspace{10\baselineskip}
\begin{paracol}{2}
\selectlanguage{latin}
\begin{center}{\color{gregoriocolor} Tértio Idus Februárii. 
 Luna\dots\ }\end{center}
\switchcolumn
\selectlanguage{english}
\begin{center}{\color{gregoriocolor} The Eleventh Day of 
 February. The\dots\ Day of the Moon.}\end{center}
\end{paracol}

\noindent\begin{tabularx}{\linewidth}{*{19}{>{\centering\arraybackslash}X}}
 \textcolor{gregoriocolor}{a} & \textcolor{gregoriocolor}{b} & \textcolor{gregoriocolor}{c} & \textcolor{gregoriocolor}{d} & \textcolor{gregoriocolor}{e} & \textcolor{gregoriocolor}{f} & \textcolor{gregoriocolor}{g} & \textcolor{gregoriocolor}{h} & \textcolor{gregoriocolor}{i} & \textcolor{gregoriocolor}{k} & \textcolor{gregoriocolor}{l} & \textcolor{gregoriocolor}{m} & \textcolor{gregoriocolor}{n} & \textcolor{gregoriocolor}{p} & \textcolor{gregoriocolor}{q} & \textcolor{gregoriocolor}{r} & \textcolor{gregoriocolor}{s} & \textcolor{gregoriocolor}{t} & \textcolor{gregoriocolor}{u} \\
 13 & 14 & 15 & 16 & 17 & 18 & 19 & 20 & 21 & 22 & 23 & 24 & 25 & 26 & 27 & 28 & 29 & 1 & 2 \\
\end{tabularx}
\vspace{0.5\baselineskip}
\noindent\begin{tabularx}{\linewidth}{*{12}{>{\centering\arraybackslash}X}}
 \textcolor{gregoriocolor}{A} & \textcolor{gregoriocolor}{B} & \textcolor{gregoriocolor}{C} & \textcolor{gregoriocolor}{D} & \textcolor{gregoriocolor}{E} & F & \textcolor{gregoriocolor}{F} & \textcolor{gregoriocolor}{G} & \textcolor{gregoriocolor}{H} & \textcolor{gregoriocolor}{M} & \textcolor{gregoriocolor}{N} & \textcolor{gregoriocolor}{P} \\
 3 & 4 & 5 & 6 & 7 & 8 & 7 & 8 & 9 & 10 & 11 & 12 \\
\end{tabularx}

\begin{paracol}{2}
\selectlanguage{latin}
\lettrine[lines=2]{L}{apúrdi,} in Gállia, Apparítio beátæ Maríæ 
 Vírginis Immaculátæ.
\switchcolumn
\selectlanguage{english}
\lettrine[lines=2]{A}{t} Lourdes in France, the apparition of Blessed Mary, Virgin Immaculate.
\switchcolumn*
\selectlanguage{latin}
Hadrianópoli, in Thrácia, sanctórum Mártyrum Lúcii Epíscopi, et Sociórum 
 ejus, sub Constántio. Ipse Lúcius, ab Ariánis multa perpéssus, in 
 vínculis martyrium consummávit; céteri vero nobilióres cívium, cum Ariános, 
 in Sardicénsi Concílio tunc damnátos, recípere noluíssent, a Philágrio 
 Cómite capitálem senténtiam excepérunt.
\switchcolumn
\selectlanguage{english}
At Adrianople, the holy martyrs Lucius, bishop, and his companions. 
 Lucius suffered much from the Arians under Constantius, and completed his 
 martyrdom in prison. The others, among the foremost citizens, refusing 
 to communicate with the Arians, who were just condemned in the Council of 
 Sardica, were sentenced to capital punishment by the count Philagrius.
\switchcolumn*
\selectlanguage{latin}
In Africa natális sanctórum Mártyrum Saturníni Presbyteri, Datívi, Felícis, 
 Ampélii et Sociórum, qui, in persecutióne Diocletiáni, cum ad Domínicum ex 
 more celebrándum conveníssent, idcírco, a milítibus comprehénsi, sub Anolíno 
 Procónsule passi sunt.
\switchcolumn
\selectlanguage{english}
In Africa, during the persecution of Diocletian, the birthday of the holy 
 martyrs Saturninus, a priest, Davitus, Felix, Ampelius, and their 
 companions. They had, as was their custom, assembled for Mass when 
 they were seized by the soldiers and put to death, under the proconsul 
 Anolinus.
\switchcolumn*
\selectlanguage{latin}
In Numídia commemorátio plurimórum sanctórum Mártyrum, qui, in 
 eádem 
 persecutióne comprehénsi sunt, et, cum juxta Imperatóris edíctum divínas 
 Scriptúras trádere noluíssent, gravíssimis excruciáti sunt supplíciis, ac 
 tandem occísi.
\switchcolumn
\selectlanguage{english}
In Numidia, in the same persecution, the commemoration of many holy martyrs, 
 who, refusing after their apprehension to deliver the holy Scriptures in 
 conformity with an imperial edict, were given over to most painful torments 
 and slain.
\switchcolumn*
\selectlanguage{latin}
Romæ sancti Gregórii Papæ Secúndi, qui Leónis 
 Isáurici impietáti acérrime réstitit, et sanctum Bonifátium ad prædicándum 
 Evangélium in Germániam misit.
\switchcolumn
\selectlanguage{english}
At Rome, Pope St. Gregory II, who courageously withstood the impiety of Leo 
 the Isaurian, and sent St. Boniface to preach the Gospel in Germany.
\switchcolumn*
\selectlanguage{latin}
Item Romæ sancti Paschális Papæ Primi, qui 
 plúrima sanctórum Mártyrum córpora levávit e cryptis, éaque in divérsis 
 Urbis Ecclésiis honorífice collocávit.
\switchcolumn
\selectlanguage{english}
Also at Rome, Pope St. Paschal I, who raised many bodies of the holy martyrs 
 from their crypts, and buried them with honour in various churches in the 
 city.
\switchcolumn*
\selectlanguage{latin}
Ravénnæ sancti Calóceri, Epíscopi et 
 Confessóris.
\switchcolumn
\selectlanguage{english}
At Ravenna, St. Calocerus, bishop and confessor.
\switchcolumn*
\selectlanguage{latin}
Medioláni sancti Lázari Epíscopi.
\switchcolumn
\selectlanguage{english}
At Milan, St. Lazarus, bishop.
\switchcolumn*
\selectlanguage{latin}
Cápuæ sancti Castrénsis Epíscopi.
\switchcolumn
\selectlanguage{english}
At Capua, St. Castrensis, bishop.
\switchcolumn*
\selectlanguage{latin}
In Castro Nantoniénsi, in Gállia, sancti Severíni, qui fuit Abbas monastérii 
 Agaunénsis, suísque précibus cultórem Dei, Regem Clodovéum, 
 a diútina infirmitáte liberávit.
\switchcolumn
\selectlanguage{english}
At Chateau Landon in France, St. Severin, abbot of the monastery of Agaune, 
 by whose prayers the Christian king Clovis was delivered from a long 
 sickness.
\switchcolumn*
\selectlanguage{latin}
In Ægypto sancti Jonæ Mónachi, virtútibus clari.
\switchcolumn
\selectlanguage{english}
In Egypt, St. Jonas, a monk, eminent for his virtues.
\switchcolumn*
\selectlanguage{latin}
Viénnæ, in Gállia, Translátio córporis sancti 
 Desidérii, Epíscopi et Mártyris, ex território Lugdunénsi, in quo ipse olim 
 passus fúerat décimo Kaléndas Júnii.
\switchcolumn
\selectlanguage{english}
At Vienne in France, the translation of the body of St. Desiderius, bishop 
 and martyr, from the district of Lyons where he had died on the 23rd of May.
\switchcolumn*
\selectlanguage{latin}
\end{paracol}


% ---- martyrology/mart02/mart0212.htm
\needspace{10\baselineskip}
\begin{paracol}{2}
\selectlanguage{latin}
\begin{center}{\color{gregoriocolor} Prídie Idus Februárii. 
 Luna\dots\ }\end{center}
\switchcolumn
\selectlanguage{english}
\begin{center}{\color{gregoriocolor} The Twelfth Day of 
 February. The\dots\ Day of the Moon.}\end{center}
\end{paracol}

\noindent\begin{tabularx}{\linewidth}{*{19}{>{\centering\arraybackslash}X}}
 \textcolor{gregoriocolor}{a} & \textcolor{gregoriocolor}{b} & \textcolor{gregoriocolor}{c} & \textcolor{gregoriocolor}{d} & \textcolor{gregoriocolor}{e} & \textcolor{gregoriocolor}{f} & \textcolor{gregoriocolor}{g} & \textcolor{gregoriocolor}{h} & \textcolor{gregoriocolor}{i} & \textcolor{gregoriocolor}{k} & \textcolor{gregoriocolor}{l} & \textcolor{gregoriocolor}{m} & \textcolor{gregoriocolor}{n} & \textcolor{gregoriocolor}{p} & \textcolor{gregoriocolor}{q} & \textcolor{gregoriocolor}{r} & \textcolor{gregoriocolor}{s} & \textcolor{gregoriocolor}{t} & \textcolor{gregoriocolor}{u} \\
 14 & 15 & 16 & 17 & 18 & 19 & 20 & 21 & 22 & 23 & 24 & 25 & 26 & 27 & 28 & 29 & 1 & 2 & 3 \\
\end{tabularx}
\vspace{0.5\baselineskip}
\noindent\begin{tabularx}{\linewidth}{*{12}{>{\centering\arraybackslash}X}}
 \textcolor{gregoriocolor}{A} & \textcolor{gregoriocolor}{B} & \textcolor{gregoriocolor}{C} & \textcolor{gregoriocolor}{D} & \textcolor{gregoriocolor}{E} & F & \textcolor{gregoriocolor}{F} & \textcolor{gregoriocolor}{G} & \textcolor{gregoriocolor}{H} & \textcolor{gregoriocolor}{M} & \textcolor{gregoriocolor}{N} & \textcolor{gregoriocolor}{P} \\
 4 & 5 & 6 & 7 & 8 & 9 & 8 & 9 & 10 & 11 & 12 & 13 \\
\end{tabularx}

\begin{paracol}{2}
\selectlanguage{latin}
\lettrine[lines=2]{S}{anctórum} septem Fundatórum Ordinis Servórum beátæ 
 Maríæ Vírginis, Confessórum, quorum deposítio respectívis diébus recólitur. 
 Quos autem in vita unus veræ fraternitátis spíritus sociávit, et indivísa 
 post óbitum venerátio pópuli prosecúta est, eos Leo Décimus tértius, 
 Póntifex Máximus, una páriter Sanctórum fastis accénsuit.
\switchcolumn
\selectlanguage{english}
\lettrine[lines=2]{T}{he} seven Holy Founders of the Order of Servites of the Blessed Virgin Mary, 
 whose deaths are noted on their respective days. As one spirit of true 
 fraternal love united them in life, and as the people joined them together 
 in the same veneration after death, Pope Leo XIII placed them together in 
 the catalogue of the saints.
\switchcolumn*
\selectlanguage{latin}
In Africa sancti Damiáni, mílitis et Mártyris.
\switchcolumn
\selectlanguage{english}
In Africa, St. Damian, soldier and martyr.
\switchcolumn*
\selectlanguage{latin}
Carthágine sanctórum Mártyrum Modésti et Juliáni.
\switchcolumn
\selectlanguage{english}
At Carthage, the holy martyrs Modestus and Julian.
\switchcolumn*
\selectlanguage{latin}
Alexandríæ sanctórum Mártyrum Modésti et 
 Ammónii infántum.
\switchcolumn
\selectlanguage{english}
At Alexandria, the holy children Modestus and Ammonius, martyrs.
\switchcolumn*
\selectlanguage{latin}
Barcinóne, in Hispánia, sanctæ Euláliæ Vírginis, 
 quæ, témpore Diocletiáni Imperatóris, equúleum, úngulas flammásque perpéssa, 
 demum, cruci affíxa, gloriósam martyrii corónam accépit.
\switchcolumn
\selectlanguage{english}
At Barcelona in Spain, in the time of Emperor Diocletian, St. Eulalia, 
 virgin, who, being racked, torn with iron hooks, cast into the fire, and 
 crucified, received the glorious crown of martyrdom.
\switchcolumn*
\selectlanguage{latin}
Constantinópoli sancti Melétii, Epíscopi Antiochéni, qui, pro fide cathólica 
 sæpe exsílium passus, demum in eádem urbe 
 migrávit ad Dóminum. Ejus virtútes sanctus Joánnes Chrysóstomus et 
 sanctus Gregórius Nyssénus summis láudibus celebrárunt.
\switchcolumn
\selectlanguage{english}
At Constantinople, St. Meletius, bishop of Antioch, who often suffered exile 
 for the Catholic faith, and finally died at Constantinople and went to his 
 reward. His virtues have been extolled by St. John Chrysostom and St. 
 Gregory of Nyssa.
\switchcolumn*
\selectlanguage{latin}
Item Constantinópoli sancti Antónii Epíscopi, témpore Leónis sexti 
 Imperatóris.
\switchcolumn
\selectlanguage{english}
Also at Constantinople, St. Anthony, a bishop in the time of Emperor Leo VI.
\switchcolumn*
\selectlanguage{latin}
Verónæ sancti Gaudéntii, Epíscopi et 
 Confessóris.
\switchcolumn
\selectlanguage{english}
At Verona, St. Gaudentius, bishop and confessor.
\switchcolumn*
\selectlanguage{latin}
\end{paracol}


% ---- martyrology/mart02/mart0213.htm
\needspace{10\baselineskip}
\begin{paracol}{2}
\selectlanguage{latin}
\begin{center}{\color{gregoriocolor} Idibus Februárii. 
 Luna\dots\ }\end{center}
\switchcolumn
\selectlanguage{english}
\begin{center}{\color{gregoriocolor} The Thirteenth Day of 
 February. The\dots\ Day of the Moon.}\end{center}
\end{paracol}

\noindent\begin{tabularx}{\linewidth}{*{19}{>{\centering\arraybackslash}X}}
 \textcolor{gregoriocolor}{a} & \textcolor{gregoriocolor}{b} & \textcolor{gregoriocolor}{c} & \textcolor{gregoriocolor}{d} & \textcolor{gregoriocolor}{e} & \textcolor{gregoriocolor}{f} & \textcolor{gregoriocolor}{g} & \textcolor{gregoriocolor}{h} & \textcolor{gregoriocolor}{i} & \textcolor{gregoriocolor}{k} & \textcolor{gregoriocolor}{l} & \textcolor{gregoriocolor}{m} & \textcolor{gregoriocolor}{n} & \textcolor{gregoriocolor}{p} & \textcolor{gregoriocolor}{q} & \textcolor{gregoriocolor}{r} & \textcolor{gregoriocolor}{s} & \textcolor{gregoriocolor}{t} & \textcolor{gregoriocolor}{u} \\
 15 & 16 & 17 & 18 & 19 & 20 & 21 & 22 & 23 & 24 & 25 & 26 & 27 & 28 & 29 & 1 & 2 & 3 & 4 \\
\end{tabularx}
\vspace{0.5\baselineskip}
\noindent\begin{tabularx}{\linewidth}{*{12}{>{\centering\arraybackslash}X}}
 \textcolor{gregoriocolor}{A} & \textcolor{gregoriocolor}{B} & \textcolor{gregoriocolor}{C} & \textcolor{gregoriocolor}{D} & \textcolor{gregoriocolor}{E} & F & \textcolor{gregoriocolor}{F} & \textcolor{gregoriocolor}{G} & \textcolor{gregoriocolor}{H} & \textcolor{gregoriocolor}{M} & \textcolor{gregoriocolor}{N} & \textcolor{gregoriocolor}{P} \\
 5 & 6 & 7 & 8 & 9 & 10 & 9 & 10 & 11 & 12 & 13 & 14 \\
\end{tabularx}

\begin{paracol}{2}
\selectlanguage{latin}
\lettrine[lines=2]{A}{ntiochíæ} natális sancti Agábi Prophétæ, de quo 
 beátus Lucas in Actibus Apostólicis scribit.
\switchcolumn
\selectlanguage{english}
\lettrine[lines=2]{A}{t} Antioch, the birthday of St. Agabus, prophet, of whom mention is made by 
 St. Luke in the Acts of the Apostles.
\switchcolumn*
\selectlanguage{latin}
Tudérti, in Umbria, sancti Benígni, Presbyteri et Mártyris; qui, Diocletiáni 
 et Maximiáni Imperatórum témpore, cum fidem Christiánam verbo et exémplo 
 propagáre non desísteret, ab idolórum cultóribus captus est, ac, váriis 
 afféctus supplíciis, sacerdotále munus honóre martyrii cumulávit.
\switchcolumn
\selectlanguage{english}
At Todi in Umbria, St. Benignus, priest and martyr, who would not cease 
 spreading the Christian faith. In the reign of Emperors Diocletian and 
 Maximian he was taken by the pagans, suffered various tortures, and finally 
 reached the perfection of his priestly office with the honour of martyrdom.
\switchcolumn*
\selectlanguage{latin}
Melitínæ, in Arménia, sancti Polyéucti Mártyris, 
 qui in persecutióne Décii multa passus, martyrii corónam adéptus est.
\switchcolumn
\selectlanguage{english}
At Meletine in Armenia, in the persecution of Decius, St. Polyeuctus, who, 
 after many sufferings, obtained the crown of martyrdom.
\switchcolumn*
\selectlanguage{latin}
Lugdúni, in Gállia, sancti Juliáni Mártyris.
\switchcolumn
\selectlanguage{english}
At Lyons in France, St. Julian, martyr.
\switchcolumn*
\selectlanguage{latin}
Ravénnæ sanctárum Fuscæ Vírginis, ejúsque 
 nutrícis Mauræ; quæ, Décio imperánte, multa sub Quinctiáno Præside perpéssæ, 
 demum, gládio transfíxæ, martyrium consummárunt.
\switchcolumn
\selectlanguage{english}
At Ravenna, in the time of Emperor Decius and the governor Quinctian, the 
 Saints Fusca, virgin, and Maura, her nurse. They endured many 
 afflictions, but were finally transfixed with a sword, and thus ended their 
 martyrdom.
\switchcolumn*
\selectlanguage{latin}
Lugdúni, in Gállia, sancti Stéphani, Epíscopi et Confessóris.
\switchcolumn
\selectlanguage{english}
At Lyons in France, St. Stephen, bishop and confessor.
\switchcolumn*
\selectlanguage{latin}
Reáte sancti Stéphani Abbátis, miræ patiéntiæ 
 viri; in cujus tránsitu (ut refert beátus Gregórius Papa) sancti Angeli, 
 céteris étiam vidéntibus, adfuérunt.
\switchcolumn
\selectlanguage{english}
At Rieti, the abbot St. Stephen, a man of wonderful patience, at whose 
 death, as is related by blessed Pope Gregory, the holy angels were present 
 and visible to all.
\switchcolumn*
\selectlanguage{latin}
\end{paracol}


% ---- martyrology/mart02/mart0214.htm
\needspace{10\baselineskip}
\begin{paracol}{2}
\selectlanguage{latin}
\begin{center}{\color{gregoriocolor} Sextodécimo Kaléndas Mártii. 
 Luna\dots\ }\end{center}
\switchcolumn
\selectlanguage{english}
\begin{center}{\color{gregoriocolor} The Fourteenth Day of 
 February. The\dots\ Day of the Moon.}\end{center}
\end{paracol}

\noindent\begin{tabularx}{\linewidth}{*{19}{>{\centering\arraybackslash}X}}
 \textcolor{gregoriocolor}{a} & \textcolor{gregoriocolor}{b} & \textcolor{gregoriocolor}{c} & \textcolor{gregoriocolor}{d} & \textcolor{gregoriocolor}{e} & \textcolor{gregoriocolor}{f} & \textcolor{gregoriocolor}{g} & \textcolor{gregoriocolor}{h} & \textcolor{gregoriocolor}{i} & \textcolor{gregoriocolor}{k} & \textcolor{gregoriocolor}{l} & \textcolor{gregoriocolor}{m} & \textcolor{gregoriocolor}{n} & \textcolor{gregoriocolor}{p} & \textcolor{gregoriocolor}{q} & \textcolor{gregoriocolor}{r} & \textcolor{gregoriocolor}{s} & \textcolor{gregoriocolor}{t} & \textcolor{gregoriocolor}{u} \\
 16 & 17 & 18 & 19 & 20 & 21 & 22 & 23 & 24 & 25 & 26 & 27 & 28 & 29 & 1 & 2 & 3 & 4 & 5 \\
\end{tabularx}
\vspace{0.5\baselineskip}
\noindent\begin{tabularx}{\linewidth}{*{12}{>{\centering\arraybackslash}X}}
 \textcolor{gregoriocolor}{A} & \textcolor{gregoriocolor}{B} & \textcolor{gregoriocolor}{C} & \textcolor{gregoriocolor}{D} & \textcolor{gregoriocolor}{E} & F & \textcolor{gregoriocolor}{F} & \textcolor{gregoriocolor}{G} & \textcolor{gregoriocolor}{H} & \textcolor{gregoriocolor}{M} & \textcolor{gregoriocolor}{N} & \textcolor{gregoriocolor}{P} \\
 6 & 7 & 8 & 9 & 10 & 11 & 10 & 11 & 12 & 13 & 14 & 15 \\
\end{tabularx}

\begin{paracol}{2}
\selectlanguage{latin}
% NOTE: this is lines=1 because of a bad page break on p. 42... Can we fix this?
\lettrine[lines=1]{R}{omæ,} via Flamínia, natális sancti Valentíni, 
 Presbyteri et Mártyris, qui, post multa sanitátum et doctrínæ insígnia, 
 fústibus cæsus et decollátus est, sub Cláudio Cǽsare.
\switchcolumn
\selectlanguage{english}
\lettrine[lines=1]{A}{t} Rome, on the Flaminian Way, in the time of Emperor Claudius, the birthday 
 of St. Valentine, priest and martyr, who after having cured and instructed 
 many persons, was beaten with clubs and beheaded.
\switchcolumn*
\selectlanguage{latin}
Ibídem deposítio sancti Cyrílli, Epíscopi et Confessóris; qui, una cum 
 sancto Methódio, simíliter Epíscopo et fratre suo, cujus dies natális octávo 
 Idus Aprílis recensétur, multas Slávicas gentes 
 earúmque Reges ad fidem Christi perdúxit. Horum tamen Sanctórum 
 festívitas Nonis Júlii celebrátur.
\switchcolumn
\selectlanguage{english}
In the same place, St. Cyril, bishop, who together with his brother 
 Methodius, also a bishop, whose birthday is the 6th of April, brought many 
 people and the rulers of Moravia to the faith of Christ. Their feast 
 is celebrated on the 7th of July.
\switchcolumn*
\selectlanguage{latin}
Item Romæ sanctórum Mártyrum Vitális, Felículæ 
 et Zenónis.
\switchcolumn
\selectlanguage{english}
Also at Rome, the holy martyrs Vitalis, Felicula and Zeno.
\switchcolumn*
\selectlanguage{latin}
Iterámnæ sancti Valentíni, Epíscopi et Mártyris, 
 qui, post diútinam cædem mancipátus custódiæ, et, cum superári non posset, 
 tandem, médiæ noctis siléntio ejéctus de cárcere, decollátus est, jussu 
 Præfécti urbis Plácidi.
\switchcolumn
\selectlanguage{english}
At Teramo, St. Valentine, bishop and martyr, who was scourged, committed to 
 prison, and, because he remained unshaken in his faith, was taken out of his 
 dungeon in the dead of night and beheaded by order of Placidus, prefect of 
 the city.
\switchcolumn*
\selectlanguage{latin}
Alexandríæ sanctórum Mártyrum Cyriónis, 
 Presbyteri, Bassiáni Lectóris, Agathónis Exorcístæ, et Móysis; qui omnes, 
 igne combústi, evolavérunt ad cælum.
\switchcolumn
\selectlanguage{english}
At Alexandria, the holy martyrs Cyrion, priest; Bassian, lector; Agatho, 
 exorcist; and Moses, who perished in the flames and took their flight to 
 heaven.
\switchcolumn*
\selectlanguage{latin}
Interámnæ sanctórum Próculi, Ephébi et 
 Apollónii Mártyrum, qui, cum ad sancti Valentíni corpus vigílias ágerent, 
 Leóntii Consuláris jussu comprehénsi sunt, et gládio cæsi.
\switchcolumn
\selectlanguage{english}
At Teramo, the holy martyrs Proculus, Ephebus, and Apollonius, who, while 
 keeping watch at the body of St. Valentine, were arrested and put to the 
 sword by command of the consular officer, Leontius.
\switchcolumn*
\selectlanguage{latin}
Alexandríæ sanctórum Mártyrum Bassi, Antónii et 
 Protólici, qui demérsi sunt in mare.
\switchcolumn
\selectlanguage{english}
At Alexandria, the holy martyrs Bassus, Anthony, and Protolicus, who were 
 drowned in the sea.
\switchcolumn*
\selectlanguage{latin}
Item Alexandríæ sanctórum Dionysii et Ammónii 
 decollatórum.
\switchcolumn
\selectlanguage{english}
Also at Alexandria, the Saints Denis and Ammonius, who were beheaded.
\switchcolumn*
\selectlanguage{latin}
Neápoli, in Campánia, sancti Nostriáni Epíscopi, qui in cathólica fide 
 contra hæréticam pravitátem tuénda éxstitit 
 insígnis.
\switchcolumn
\selectlanguage{english}
At Naples, in Campania, St. Nostrian, bishop, who was outstanding for his 
 defence of the Catholic faith against heretical errors.
\switchcolumn*
\selectlanguage{latin}
Ravénnæ sancti Eleuchádii, Epíscopi et 
 Confessóris.
\switchcolumn
\selectlanguage{english}
At Ravenna, St. Eleuchadius, bishop and confessor.
\switchcolumn*
\selectlanguage{latin}
In Bithynia sancti Auxéntii Abbátis.
\switchcolumn
\selectlanguage{english}
In Bithynia, St. Auxentius, abbot.
\switchcolumn*
\selectlanguage{latin}
Apud Surréntum sancti Antoníni Abbátis, qui e monastério Cassinénsi, a 
 Longobárdis devastáto, in solitúdinem ejúsdem urbis secéssit; ibíque, 
 sanctitáte célebris, obdormívit in Dómino. Ipsíus corpus multis 
 quotídie miráculis, et præsértim in energúmenis 
 liberándis effúlget.
\switchcolumn
\selectlanguage{english}
At Sorrento, St. Anthony, abbot, who, when the monastery of Monte Cassino 
 was devastated by the Lombards, withdrew into a solitude of the 
 neighbourhood, where, celebrated for his holiness, he went calmly to his 
 repose in God. His body is daily glorified by many miracles, 
 especially by the deliverance of possessed persons.
\switchcolumn*
\selectlanguage{latin}
\end{paracol}


% ---- martyrology/mart02/mart0215.htm
\needspace{10\baselineskip}
\begin{paracol}{2}
\selectlanguage{latin}
\begin{center}{\color{gregoriocolor} Quintodécimo Kaléndas Mártii. 
 Luna\dots\ }\end{center}
\switchcolumn
\selectlanguage{english}
\begin{center}{\color{gregoriocolor} The Fifteenth Day of 
 February. The\dots\ Day of the Moon.}\end{center}
\end{paracol}

\noindent\begin{tabularx}{\linewidth}{*{19}{>{\centering\arraybackslash}X}}
 \textcolor{gregoriocolor}{a} & \textcolor{gregoriocolor}{b} & \textcolor{gregoriocolor}{c} & \textcolor{gregoriocolor}{d} & \textcolor{gregoriocolor}{e} & \textcolor{gregoriocolor}{f} & \textcolor{gregoriocolor}{g} & \textcolor{gregoriocolor}{h} & \textcolor{gregoriocolor}{i} & \textcolor{gregoriocolor}{k} & \textcolor{gregoriocolor}{l} & \textcolor{gregoriocolor}{m} & \textcolor{gregoriocolor}{n} & \textcolor{gregoriocolor}{p} & \textcolor{gregoriocolor}{q} & \textcolor{gregoriocolor}{r} & \textcolor{gregoriocolor}{s} & \textcolor{gregoriocolor}{t} & \textcolor{gregoriocolor}{u} \\
 17 & 18 & 19 & 20 & 21 & 22 & 23 & 24 & 25 & 26 & 27 & 28 & 29 & 1 & 2 & 3 & 4 & 5 & 6 \\
\end{tabularx}
\vspace{0.5\baselineskip}
\noindent\begin{tabularx}{\linewidth}{*{12}{>{\centering\arraybackslash}X}}
 \textcolor{gregoriocolor}{A} & \textcolor{gregoriocolor}{B} & \textcolor{gregoriocolor}{C} & \textcolor{gregoriocolor}{D} & \textcolor{gregoriocolor}{E} & F & \textcolor{gregoriocolor}{F} & \textcolor{gregoriocolor}{G} & \textcolor{gregoriocolor}{H} & \textcolor{gregoriocolor}{M} & \textcolor{gregoriocolor}{N} & \textcolor{gregoriocolor}{P} \\
 7 & 8 & 9 & 10 & 11 & 12 & 11 & 12 & 13 & 14 & 15 & 16 \\
\end{tabularx}

\begin{paracol}{2}
\selectlanguage{latin}
\lettrine[lines=2]{B}{ríxiæ} natális sanctórum Mártyrum Faustíni et 
 Jovítæ fratrum, qui sub Hadriáno Imperatóre, post multa præclára ob Christi 
 fidem suscépta certámina, victrícem martyrii corónam accepérunt.
\switchcolumn
\selectlanguage{english}
\lettrine[lines=2]{A}{t} Brescia, in the time of Emperor Adrian, the birthday of the holy martyrs 
 Faustinus and Jovita, who received the triumphant crown of martyrdom after 
 many glorious combats for the faith of Christ.
\switchcolumn*
\selectlanguage{latin}
Romæ sancti Cratónis Mártyris, qui, cum uxóre 
 sua et univérsa domo a beáto Valentíno Epíscopo baptizátus, non multo post, 
 una cum illis, martyrio consummátus est.
\switchcolumn
\selectlanguage{english}
At Rome, St. Craton, martyr. A short time after being baptized with 
 his wife and all his household by the holy bishop Valentine, he was put to 
 death with them.
\switchcolumn*
\selectlanguage{latin}
Interámnæ natális sanctórum Mártyrum Saturníni, 
 Cástuli, Magni et Lúcii.
\switchcolumn
\selectlanguage{english}
At Teramo, the birthday of the holy martyrs Saturninus, Castulus, Magnus, 
 and Lucius.
\switchcolumn*
\selectlanguage{latin}
Ibídem sanctæ Agapis, Vírginis et Mártyris.
\switchcolumn
\selectlanguage{english}
In the same place, St. Agape, virgin and martyr.
\switchcolumn*
\selectlanguage{latin}
Vasióne, in Gálliis, sancti Quinídii Epíscopi, cujus mortem in conspéctu 
 Dómini pretiósam mirácula crebra testántur.
\switchcolumn
\selectlanguage{english}
At Vaison in France, St. Quinidius, bishop, whose death was precious in the 
 sight of God, as is shewn by frequent miracles.
\switchcolumn*
\selectlanguage{latin}
Cápuæ sancti Decorósi, Epíscopi et Confessóris.
\switchcolumn
\selectlanguage{english}
At Capua, St. Decorosus, bishop and confessor.
\switchcolumn*
\selectlanguage{latin}
In Província Valériæ sancti Sevéri Presbyteri, 
 qui (ut beátus Gregórius Papa scribit), fusis lácrimis, defúnctum revocávit 
 ad vitam.
\switchcolumn
\selectlanguage{english}
In the province of Valeria, St. Severus, priest, of whom St. Gregory says 
 that by his tears he recalled a dead man to life.
\switchcolumn*
\selectlanguage{latin}
Antiochíæ sancti Joséphi Diáconi.
\switchcolumn
\selectlanguage{english}
At Antioch, St. Joseph, deacon.
\switchcolumn*
\selectlanguage{latin}
Arvérnis, in Gállia, sanctæ Geórgiæ Vírginis.
\switchcolumn
\selectlanguage{english}
In Auvergne in France, St. Georgia, virgin.
\switchcolumn*
\selectlanguage{latin}
\end{paracol}


% ---- martyrology/mart02/mart0216.htm
\needspace{10\baselineskip}
\begin{paracol}{2}
\selectlanguage{latin}
\begin{center}{\color{gregoriocolor} Quartodécimo Kaléndas Mártii. 
 Luna\dots\ }\end{center}
\switchcolumn
\selectlanguage{english}
\begin{center}{\color{gregoriocolor} The Sixteenth Day of 
 February. The\dots\ Day of the Moon.}\end{center}
\end{paracol}

\noindent\begin{tabularx}{\linewidth}{*{19}{>{\centering\arraybackslash}X}}
 \textcolor{gregoriocolor}{a} & \textcolor{gregoriocolor}{b} & \textcolor{gregoriocolor}{c} & \textcolor{gregoriocolor}{d} & \textcolor{gregoriocolor}{e} & \textcolor{gregoriocolor}{f} & \textcolor{gregoriocolor}{g} & \textcolor{gregoriocolor}{h} & \textcolor{gregoriocolor}{i} & \textcolor{gregoriocolor}{k} & \textcolor{gregoriocolor}{l} & \textcolor{gregoriocolor}{m} & \textcolor{gregoriocolor}{n} & \textcolor{gregoriocolor}{p} & \textcolor{gregoriocolor}{q} & \textcolor{gregoriocolor}{r} & \textcolor{gregoriocolor}{s} & \textcolor{gregoriocolor}{t} & \textcolor{gregoriocolor}{u} \\
 18 & 19 & 20 & 21 & 22 & 23 & 24 & 25 & 26 & 27 & 28 & 29 & 1 & 2 & 3 & 4 & 5 & 6 & 7 \\
\end{tabularx}
\vspace{0.5\baselineskip}
\noindent\begin{tabularx}{\linewidth}{*{12}{>{\centering\arraybackslash}X}}
 \textcolor{gregoriocolor}{A} & \textcolor{gregoriocolor}{B} & \textcolor{gregoriocolor}{C} & \textcolor{gregoriocolor}{D} & \textcolor{gregoriocolor}{E} & F & \textcolor{gregoriocolor}{F} & \textcolor{gregoriocolor}{G} & \textcolor{gregoriocolor}{H} & \textcolor{gregoriocolor}{M} & \textcolor{gregoriocolor}{N} & \textcolor{gregoriocolor}{P} \\
 8 & 9 & 10 & 11 & 12 & 13 & 12 & 13 & 14 & 15 & 16 & 17 \\
\end{tabularx}

\begin{paracol}{2}
\selectlanguage{latin}
\lettrine[lines=2]{R}{omæ} beáti Onésimi, de quo sanctus Paulus 
 Apóstolus ad Philémonem scribit; quem étiam, post sanctum Timótheum, 
 Ephesiórum Epíscopum ordinávit, prædicationísque verbum illi commisit. 
 Ipse autem Onésimus, vinctus Romam perdúctus ac pro fide Christi lapidátus, 
 primo ibídem sepúltus fuit; inde ad locum ubi Epíscopus fúerat ordinátus, 
 corpus ejus delátum est.
\switchcolumn
\selectlanguage{english}
\lettrine[lines=2]{A}{t} Rome, blessed Onesimus, concerning whom the apostle St. Paul wrote to 
 Philemon. He made him bishop of Ephesus after St. Timothy, and 
 committed to him the office of preaching. Being led a prisoner to 
 Rome, and stoned to death for the faith of Christ, he was first buried 
 there, but his body was afterwards taken to the place where he had been 
 bishop.
\switchcolumn*
\selectlanguage{latin}
In Ægypto sancti Juliáni Mártyris, cum áliis 
 quinque míllibus.
\switchcolumn
\selectlanguage{english}
In Egypt, St. Julian, martyr, with five thousand other Christians.
\switchcolumn*
\selectlanguage{latin}
Cæsaréæ, in Palæstína, sanctórum Mártyrum 
 Ægyptiórum Elíæ, Jeremíæ, Isaíæ, Samuélis et Daniélis; qui, cum spontánee 
 ministrássent Confessóribus in Cilícia ad metálla damnátis, et inde 
 reverteréntur, sunt comprehénsi, et a Firmiliáno Præside, sub Galério 
 Maximiáno Imperatóre, sævíssime torti, gládio demum percússi sunt. 
 Post eos sanctus Porphyrius, Pámphili Mártyris fámulus, et sanctus Seléucus 
 Cáppadox, qui iterátis certamínibus sæpe vícerant, rursus cruciáti sunt, 
 atque alter incéndio, gládio alter corónam martyrii accepérunt.
\switchcolumn
\selectlanguage{english}
At Caesarea, in Palestine, the holy martyrs Elias, Jeremias, Isaias, Samuel, 
 and Daniel. These Egyptians of their own accord ministered to the 
 confessors condemned to labour in the mines of Cilicia, but were arrested 
 upon their return, and after being cruelly tortured by the governor 
 Firmilian, under Emperor Galerius Maximian, were put to the sword. 
 After them, St. Porphyry, servant of the martyr Pamphilus, and St. Seleucus 
 the Cappadocian, who had been triumphant in several previous tests, being 
 again tortured, now won the crown of martyrdom, the one by fire, the other 
 by the sword.
\switchcolumn*
\selectlanguage{latin}
Nicomedíæ sanctæ Juliánæ, Vírginis et Mártyris; 
 quæ, sub Maximiáno Imperatóre, primum a patre suo Africáno gráviter cæsa, 
 deínde ab Evilásio Præfécto, cui núbere recusáverat, várie cruciáta, et 
 póstmodum in cárcerem detrúsa, ubi palam cum diábolo conflíxit, demum, cum 
 flammas ígnium et ollam fervéntem superásset, cápitis decollatióne martyrium 
 consummávit. Ipsíus autem corpus póstea Cumas, in Campánia translátum 
 est.
\switchcolumn
\selectlanguage{english}
At Nicomedia, St. Juliana, virgin and martyr. Under Emperor Maximian, 
 she was first severely scourged by her own father, Africanus, and then made 
 to suffer many torments by the prefect Evilasius, whom she had refused to 
 marry. Later thrown into prison, she encountered the evil spirit in a 
 visible manner. Finally, because the fiery furnace and a caldron of 
 boiling oil could do her no injury, her martyrdom was fulfilled by 
 beheading. Her body was later transferred to Cumi in Campania.
\switchcolumn*
\selectlanguage{latin}
Bríxiæ sancti Faustíni, Epíscopi et Confessóris.
\switchcolumn
\selectlanguage{english}
At Brescia, St. Faustinus, bishop and confessor.
\switchcolumn*
\selectlanguage{latin}
\end{paracol}


% ---- martyrology/mart02/mart0217.htm
\needspace{10\baselineskip}
\begin{paracol}{2}
\selectlanguage{latin}
\begin{center}{\color{gregoriocolor} Tertiodécimo Kaléndas Mártii. 
 Luna\dots\ }\end{center}
\switchcolumn
\selectlanguage{english}
\begin{center}{\color{gregoriocolor} The Seventeenth Day of 
 February. The\dots\ Day of the Moon.}\end{center}
\end{paracol}

\noindent\begin{tabularx}{\linewidth}{*{19}{>{\centering\arraybackslash}X}}
 \textcolor{gregoriocolor}{a} & \textcolor{gregoriocolor}{b} & \textcolor{gregoriocolor}{c} & \textcolor{gregoriocolor}{d} & \textcolor{gregoriocolor}{e} & \textcolor{gregoriocolor}{f} & \textcolor{gregoriocolor}{g} & \textcolor{gregoriocolor}{h} & \textcolor{gregoriocolor}{i} & \textcolor{gregoriocolor}{k} & \textcolor{gregoriocolor}{l} & \textcolor{gregoriocolor}{m} & \textcolor{gregoriocolor}{n} & \textcolor{gregoriocolor}{p} & \textcolor{gregoriocolor}{q} & \textcolor{gregoriocolor}{r} & \textcolor{gregoriocolor}{s} & \textcolor{gregoriocolor}{t} & \textcolor{gregoriocolor}{u} \\
 19 & 20 & 21 & 22 & 23 & 24 & 25 & 26 & 27 & 28 & 29 & 1 & 2 & 3 & 4 & 5 & 6 & 7 & 8 \\
\end{tabularx}
\vspace{0.5\baselineskip}
\noindent\begin{tabularx}{\linewidth}{*{12}{>{\centering\arraybackslash}X}}
 \textcolor{gregoriocolor}{A} & \textcolor{gregoriocolor}{B} & \textcolor{gregoriocolor}{C} & \textcolor{gregoriocolor}{D} & \textcolor{gregoriocolor}{E} & F & \textcolor{gregoriocolor}{F} & \textcolor{gregoriocolor}{G} & \textcolor{gregoriocolor}{H} & \textcolor{gregoriocolor}{M} & \textcolor{gregoriocolor}{N} & \textcolor{gregoriocolor}{P} \\
 9 & 10 & 11 & 12 & 13 & 14 & 13 & 14 & 15 & 16 & 17 & 18 \\
\end{tabularx}

\begin{paracol}{2}
\selectlanguage{latin}
\lettrine[lines=2]{F}{loréntiæ} natális sancti Aléxii Falconérii 
 Confessóris, e septem Fundatóribus Ordinis Servórum beátæ Maríæ Vírginis; 
 qui, décimo supra centésimum vitæ suæ anno, Christi Jesu et Angelórum 
 præséntia recreátus, beáto fine quiévit. Ipsíus tamen ac Sociórum 
 festum prídie Idus Februárii celebrátur.
\switchcolumn
\selectlanguage{english}
\lettrine[lines=2]{I}{n} Florence, the birthday of St. Alexis Falconieri, confessor, one of the 
 seven founders of the Order of the Servites of the Blessed Virgin Mary. 
 In the one hundred and tenth year of his age, he ended his blessed career in 
 the consoling presence of Christ Jesus and the angels. His feast, with 
 that of his companions, is kept on the 12th of February.
\switchcolumn*
\selectlanguage{latin}
Romæ pássio sancti Faustíni, quem álii 
 quadragínta quátuor secúti sunt ad corónam.
\switchcolumn
\selectlanguage{english}
At Rome, the passion of St. Faustinus, whom forty-four others followed to 
 receive the crown of martyrdom.
\switchcolumn*
\selectlanguage{latin}
In Pérside natális beáti Polychrónii, Epíscopi Babylónis, qui, in 
 persecutióne Décii, ore lapídibus cæso, mánibus 
 exténsis, ad cælum óculos élevans, emísit spíritum.
\switchcolumn
\selectlanguage{english}
In Persia, during the persecution of Decius, the birthday of blessed 
 Polychronius, bishop of Babylon, who, being struck in the mouth with stones, 
 died with hands outstretched and eyes lifted towards heaven.
\switchcolumn*
\selectlanguage{latin}
Concórdiæ, in Venetórum fínibus, sanctórum 
 Mártyrum Donáti, Secundiáni et Rómuli, cum áliis octogínta sex, ejúsdem 
 corónæ consórtibus.
\switchcolumn
\selectlanguage{english}
At Concordia, the holy martyrs Donatus, Secundian, and Romulus, with 
 eighty-six others, partakers of the same crown.
\switchcolumn*
\selectlanguage{latin}
Cæsaréæ, in Palæstína, sancti Theodúli senis, 
 qui, cum esset ex família Prǽsidis Firmiliáni, et, Mártyrum excitátus 
 exémplo, Christum constánter confiterétur, martyrii palmam, cruci affíxus, 
 nóbili triúmpho proméruit.
\switchcolumn
\selectlanguage{english}
At Caesarea in Palestine, the death of St. Theodulus, in the service of the 
 governor Firmilian, at a great age. Prompted by the example of the 
 martyrs, he confessed Christ with constancy, and was nailed to a cross. 
 By this noble victory he merited the palm of martyrdom.
\switchcolumn*
\selectlanguage{latin}
Ibidem sancti Juliáni Cappádocis, qui, 
 exosculátus necatórum Mártyrum córpora, et ídeo ut Christiánus delátus et ad 
 Prǽsidem ductus, lento igne jussus est combúri.
\switchcolumn
\selectlanguage{english}
In the same place, St. Julian the Cappadocian, who, because he had kissed 
 the relics of martyrs, was denounced as a Christian. Being taken to 
 the governor, he was ordered to be burned to death over a slow fire.
\switchcolumn*
\selectlanguage{latin}
In pago Tarvanénsi, in Gállia, sancti Silvíni, Epíscopi Tolosáni.
\switchcolumn
\selectlanguage{english}
In the territory of Terouanne in France, St. Silvinus, bishop of Toulouse.
\switchcolumn*
\selectlanguage{latin}
In monastério Cluain-ednechénsi, in Hibérnia, sancti Fintáni, Presbyteri et 
 Abbátis.
\switchcolumn
\selectlanguage{english}
In the monastery of Cluainedhech in Ireland, St. Fintan, abbot.
\switchcolumn*
\selectlanguage{latin}
\end{paracol}


% ---- martyrology/mart02/mart0218.htm
\needspace{10\baselineskip}
\begin{paracol}{2}
\selectlanguage{latin}
\begin{center}{\color{gregoriocolor} Duodécimo Kaléndas Mártii. 
 Luna\dots\ }\end{center}
\switchcolumn
\selectlanguage{english}
\begin{center}{\color{gregoriocolor} The Eighteenth Day of 
 February. The\dots\ Day of the Moon.}\end{center}
\end{paracol}

\noindent\begin{tabularx}{\linewidth}{*{19}{>{\centering\arraybackslash}X}}
 \textcolor{gregoriocolor}{a} & \textcolor{gregoriocolor}{b} & \textcolor{gregoriocolor}{c} & \textcolor{gregoriocolor}{d} & \textcolor{gregoriocolor}{e} & \textcolor{gregoriocolor}{f} & \textcolor{gregoriocolor}{g} & \textcolor{gregoriocolor}{h} & \textcolor{gregoriocolor}{i} & \textcolor{gregoriocolor}{k} & \textcolor{gregoriocolor}{l} & \textcolor{gregoriocolor}{m} & \textcolor{gregoriocolor}{n} & \textcolor{gregoriocolor}{p} & \textcolor{gregoriocolor}{q} & \textcolor{gregoriocolor}{r} & \textcolor{gregoriocolor}{s} & \textcolor{gregoriocolor}{t} & \textcolor{gregoriocolor}{u} \\
 20 & 21 & 22 & 23 & 24 & 25 & 26 & 27 & 28 & 29 & 1 & 2 & 3 & 4 & 5 & 6 & 7 & 8 & 9 \\
\end{tabularx}
\vspace{0.5\baselineskip}
\noindent\begin{tabularx}{\linewidth}{*{12}{>{\centering\arraybackslash}X}}
 \textcolor{gregoriocolor}{A} & \textcolor{gregoriocolor}{B} & \textcolor{gregoriocolor}{C} & \textcolor{gregoriocolor}{D} & \textcolor{gregoriocolor}{E} & F & \textcolor{gregoriocolor}{F} & \textcolor{gregoriocolor}{G} & \textcolor{gregoriocolor}{H} & \textcolor{gregoriocolor}{M} & \textcolor{gregoriocolor}{N} & \textcolor{gregoriocolor}{P} \\
 10 & 11 & 12 & 13 & 14 & 15 & 14 & 15 & 16 & 17 & 18 & 19 \\
\end{tabularx}

\begin{paracol}{2}
\selectlanguage{latin}
\lettrine[lines=2]{H}{ierosólymis} natális sancti Simeónis, Epíscopi et Mártyris; qui fílius 
 Cléophæ et propínquus Salvatóris secúndum 
 carnem fuísse tráditur. Hic, Hierosolymórum Epíscopus post Jacóbum, 
 fratrem Dómini, ordinátus, et, in Trajáni persecutióne, multis supplíciis 
 afféctus, martyrio consummátus est, ómnibus qui áderant et Júdice ipso 
 mirántibus ut centum vigínti annórum senex fórtiter constantérque supplícium 
 crucis pertulísset.
\switchcolumn
\selectlanguage{english}
\lettrine[lines=2]{A}{t} Jerusalem, the birthday of St. Simeon, bishop and martyr, who is said to 
 have been the son of Cleophas, and a relative of the Saviour according to 
 the flesh. He was consecrated bishop of Jerusalem after St. James, the 
 cousin of our Lord. In the persecution of Trajan, after having endured 
 many torments, his martyrdom was completed. All who were present, even 
 the judge himself, were astonished that a man one hundred and twenty years 
 of age could bear the torment of crucifixion with such fortitude and 
 constancy.
\switchcolumn*
\selectlanguage{latin}
Apud Ostia Tiberína sanctórum Mártyrum Máximi et Cláudii fratrum, et Præpedígnæ, 
 uxóris Cláudii, cum duóbus fíliis Alexándro et Cútia; qui, cum essent 
 præclaríssimi géneris, omnes, jubénte Diocletiáno, tenti atque in exsílium 
 deportáti sunt, ac deínde, incéndio concremáti, Deo ipsi odoríferum martyrii 
 sacrifícium obtulérunt. Eórum relíquiæ, in flumen projéctæ et a 
 Christiánis perquisítæ, juxta eándem civitátem sepúltæ sunt.
\switchcolumn
\selectlanguage{english}
At Ostia, the holy martyrs Maximus and his brother Claudius, and Praepedigna, 
 the wife of Claudius, with her two sons Alexander and Cutias. Although 
 all of a noble birth, by the order of Diocletian, they were apprehended and 
 sent into exile. Afterwards being burned alive, they offered to God 
 the sweet sacrifice of martyrdom. Their remains were cast into the 
 river, but the Christians found them and buried them near the city.
\switchcolumn*
\selectlanguage{latin}
In Africa sanctórum Mártyrum Lúcii, Silváni, Rútuli, 
 Clássici, Secundíni, Frúctuli et Máximi.
\switchcolumn
\selectlanguage{english}
In Africa, the holy martyrs Lucius, Sylvanus, Rutulus, Classicus, Secundinus, 
 Fructulus, and Maximus.
\switchcolumn*
\selectlanguage{latin}
Constantinópoli sancti Flaviáni Epíscopi, qui, cum fidem cathólicam Ephesi 
 propugnáret, ab ímpii Dióscori factióne pugnis 
 et cálcibus percússus est, et, in exsílium actus, ibídem post tríduum vitam 
 finívit.
\switchcolumn
\selectlanguage{english}
At Constantinople, St. Flavian, bishop, who, for having defended the 
 Catholic faith at Ephesus, was attacked with slaps and kicks by the faction 
 of the impious Dioscorus, and then driven into exile where he died within 
 three days.
\switchcolumn*
\selectlanguage{latin}
Toléti, in Hispánia, sancti Helládii, Epíscopi et Confessóris, qui a sancto 
 Ildefónso, Toletáno Episcopo, multis láudibus celebrátur.
\switchcolumn
\selectlanguage{english}
At Toledo, Spain, St. Helladius, bishop and confessor, who received much 
 praise from St. Ildefonse, Bishop of Toledo.
\switchcolumn*
\selectlanguage{latin}
\end{paracol}


% ---- martyrology/mart02/mart0219.htm
\needspace{10\baselineskip}
\begin{paracol}{2}
\selectlanguage{latin}
\begin{center}{\color{gregoriocolor} Undécimo Kaléndas Mártii. 
 Luna\dots\ }\end{center}
\switchcolumn
\selectlanguage{english}
\begin{center}{\color{gregoriocolor} The Nineteenth Day of 
 February. The\dots\ Day of the Moon.}\end{center}
\end{paracol}

\noindent\begin{tabularx}{\linewidth}{*{19}{>{\centering\arraybackslash}X}}
 \textcolor{gregoriocolor}{a} & \textcolor{gregoriocolor}{b} & \textcolor{gregoriocolor}{c} & \textcolor{gregoriocolor}{d} & \textcolor{gregoriocolor}{e} & \textcolor{gregoriocolor}{f} & \textcolor{gregoriocolor}{g} & \textcolor{gregoriocolor}{h} & \textcolor{gregoriocolor}{i} & \textcolor{gregoriocolor}{k} & \textcolor{gregoriocolor}{l} & \textcolor{gregoriocolor}{m} & \textcolor{gregoriocolor}{n} & \textcolor{gregoriocolor}{p} & \textcolor{gregoriocolor}{q} & \textcolor{gregoriocolor}{r} & \textcolor{gregoriocolor}{s} & \textcolor{gregoriocolor}{t} & \textcolor{gregoriocolor}{u} \\
 21 & 22 & 23 & 24 & 25 & 26 & 27 & 28 & 29 & 1 & 2 & 3 & 4 & 5 & 6 & 7 & 8 & 9 & 10 \\
\end{tabularx}
\vspace{0.5\baselineskip}
\noindent\begin{tabularx}{\linewidth}{*{12}{>{\centering\arraybackslash}X}}
 \textcolor{gregoriocolor}{A} & \textcolor{gregoriocolor}{B} & \textcolor{gregoriocolor}{C} & \textcolor{gregoriocolor}{D} & \textcolor{gregoriocolor}{E} & F & \textcolor{gregoriocolor}{F} & \textcolor{gregoriocolor}{G} & \textcolor{gregoriocolor}{H} & \textcolor{gregoriocolor}{M} & \textcolor{gregoriocolor}{N} & \textcolor{gregoriocolor}{P} \\
 11 & 12 & 13 & 14 & 15 & 16 & 15 & 16 & 17 & 18 & 19 & 20 \\
\end{tabularx}

\begin{paracol}{2}
\selectlanguage{latin}
\lettrine[lines=2]{R}{omæ} natális sancti Gabíni, Presbyteri et 
 Mártyris, qui fuit frater beáti Caji Papæ, atque, a Diocletiáno diu in 
 custódia vínculis afflíctus, pretiósa morte sibi cæli gáudia comparávit.
\switchcolumn
\selectlanguage{english}
\lettrine[lines=2]{A}{t} Rome, the birthday of St. Gavinus, priest and martyr, brother of blessed 
 Pope Caius. After being chained in prison for a long time by 
 Diocletian, he obtained the joys of heaven by his esteemed death.
\switchcolumn*
\selectlanguage{latin}
In Africa sanctórum Mártyrum Públii, Juliáni, Marcélli et aliórum.
\switchcolumn
\selectlanguage{english}
In Africa, the holy martyrs Publius, Julian, Marcellus, and others.
\switchcolumn*
\selectlanguage{latin}
In Palæstína commemorátio sanctórum Monachórum, 
 et aliórum Mártyrum, qui a Saracénis, sub Duce Alamúndaro, ob Christi fidem, 
 sævíssime cæsi sunt.
\switchcolumn
\selectlanguage{english}
In Palestine, the commemoration of the holy monks and other martyrs who were 
 barbarously massacred for the faith of Christ by the Saracens, under their 
 leader Almondhar.
\switchcolumn*
\selectlanguage{latin}
Neápoli, in Campánia, sancti Quod-vult-Deus, Carthaginénsis Epíscopi, qui, 
 una cum Clero, a Rege Ariáno Genseríco in 
 fractas et absque remígiis ac velis naves impósitus, præter spem Neápolim áppulit, ibíque, in exsílio pósitus, Conféssor occúbuit.
\switchcolumn
\selectlanguage{english}
At Naples in Campania, St. Quodvultdeus, bishop of Carthage. The Arian 
 king Genseric placed him together with his clergy into boats which were 
 broken and without oars and sails, but they unexpectedly reached Naples. 
 He died in exile as a confessor.
\switchcolumn*
\selectlanguage{latin}
Hierosólymis sancti Zambdæ Epíscopi.
\switchcolumn
\selectlanguage{english}
At Jerusalem, St. Zambdas, bishop.
\switchcolumn*
\selectlanguage{latin}
Solis, in Cypro, sancti Auxíbii Epíscopi.
\switchcolumn
\selectlanguage{english}
At Soli in Cyprus, St. Auxibius, bishop.
\switchcolumn*
\selectlanguage{latin}
Apud Benevéntum sancti Barbáti Epíscopi, qui, sanctitáte célebris, 
 Longobárdos et eórum Ducem convértit ad Christum.
\switchcolumn
\selectlanguage{english}
At Benevento, St. Barbatus, a bishop illustrious for sanctity, who converted 
 the Lombards and their chief to the faith of Christ.
\switchcolumn*
\selectlanguage{latin}
Medioláni sancti Mansuéti, Epíscopi et Confessóris.
\switchcolumn
\selectlanguage{english}
At Milan, St. Mansuetus, bishop and confessor.
\switchcolumn*
\selectlanguage{latin}
\end{paracol}


% ---- martyrology/mart02/mart0220.htm
\needspace{10\baselineskip}
\begin{paracol}{2}
\selectlanguage{latin}
\begin{center}{\color{gregoriocolor} Décimo Kaléndas Mártii. 
 Luna\dots\ }\end{center}
\switchcolumn
\selectlanguage{english}
\begin{center}{\color{gregoriocolor} The Twentieth Day of 
 February. The\dots\ Day of the Moon.}\end{center}
\end{paracol}

\noindent\begin{tabularx}{\linewidth}{*{19}{>{\centering\arraybackslash}X}}
 \textcolor{gregoriocolor}{a} & \textcolor{gregoriocolor}{b} & \textcolor{gregoriocolor}{c} & \textcolor{gregoriocolor}{d} & \textcolor{gregoriocolor}{e} & \textcolor{gregoriocolor}{f} & \textcolor{gregoriocolor}{g} & \textcolor{gregoriocolor}{h} & \textcolor{gregoriocolor}{i} & \textcolor{gregoriocolor}{k} & \textcolor{gregoriocolor}{l} & \textcolor{gregoriocolor}{m} & \textcolor{gregoriocolor}{n} & \textcolor{gregoriocolor}{p} & \textcolor{gregoriocolor}{q} & \textcolor{gregoriocolor}{r} & \textcolor{gregoriocolor}{s} & \textcolor{gregoriocolor}{t} & \textcolor{gregoriocolor}{u} \\
 22 & 23 & 24 & 25 & 26 & 27 & 28 & 29 & 1 & 2 & 3 & 4 & 5 & 6 & 7 & 8 & 9 & 10 & 11 \\
\end{tabularx}
\vspace{0.5\baselineskip}
\noindent\begin{tabularx}{\linewidth}{*{12}{>{\centering\arraybackslash}X}}
 \textcolor{gregoriocolor}{A} & \textcolor{gregoriocolor}{B} & \textcolor{gregoriocolor}{C} & \textcolor{gregoriocolor}{D} & \textcolor{gregoriocolor}{E} & F & \textcolor{gregoriocolor}{F} & \textcolor{gregoriocolor}{G} & \textcolor{gregoriocolor}{H} & \textcolor{gregoriocolor}{M} & \textcolor{gregoriocolor}{N} & \textcolor{gregoriocolor}{P} \\
 12 & 13 & 14 & 15 & 16 & 17 & 16 & 17 & 18 & 19 & 20 & 21 \\
\end{tabularx}

\begin{paracol}{2}
\selectlanguage{latin}
\lettrine[lines=2]{T}{yri,} in Phœnícia, commemorátio beatórum 
 Mártyrum, quorum númerum solíus sciéntia Dei cólligit. Hi omnes, sub 
 Diocletiáno Imperatóre, a Vetúrio, mílitum magístro, multis tormentórum 
 genéribus, sibi ínvicem succedéntibus, occísi sunt; nam, primo quidem 
 flagris toto córpore dilaniáti, inde divérsis bestiárum genéribus tráditi, 
 sed ab illis divína virtúte nil læsi, post, áddita feritáte ignis ac ferri, 
 martyrium consummárunt. Eórum vero gloriósam multitúdinem ad victóriam 
 incitábant Epíscopi Tyránnio, Silvánus, Péleus et Nilus, ac Presbyter 
 Zenóbius, qui, felíci agóne, una cum illis, martyrii palmam adépti sunt.
\switchcolumn
\selectlanguage{english}
\lettrine[lines=2]{A}{t} Tyre in Phoenicia, the commemoration of many blessed martyrs, whose 
 number is known to God alone. Under Emperor Diocletian, they were put 
 to death after a long and varied series of torments by the military 
 commander Veturius. They first had their bodies torn with scourges, 
 then delivered to several different kinds of beasts. Providence 
 prevented their injury throughout all this, but their martyrdom was granted 
 by means of fire and the sword. Tyrannio, Sylvanus, Peleus, and Nilus, 
 all bishops, and Zenobius, a priest, urged the gloriously assembled 
 multitude to victory, and they all endured the test successfully to win the 
 palm of martyrdom.
\switchcolumn*
\selectlanguage{latin}
Constantinópoli sancti Eleuthérii, Epíscopi et Mártyris.
\switchcolumn
\selectlanguage{english}
At Constantinople, St. Eleutherius, bishop and martyr.
\switchcolumn*
\selectlanguage{latin}
In Pérside natális sancti Sadoth Epíscopi, et aliórum centum vigínti octo; 
 qui, sub Rege Persárum Sápore, cum Solem 
 adoráre renuíssent, crudéli nece præcláras sibi corónas comparárunt.
\switchcolumn
\selectlanguage{english}
In Persia, in the time of King Sapor, the birthday of St. Sadoth, bishop, 
 and one hundred and twenty-eight others who refused to adore the sun, but 
 who by a cruel death purchased shining crowns.
\switchcolumn*
\selectlanguage{latin}
In Cypro sanctórum Mártyrum Potámii et Nemésii.
\switchcolumn
\selectlanguage{english}
In the island of Cyprus, the holy martyrs Pothamius and Nemesius.
\switchcolumn*
\selectlanguage{latin}
Cátanæ, in Sicília, sancti Leónis Epíscopi, qui 
 virtútibus atque miráculis coruscávit.
\switchcolumn
\selectlanguage{english}
At Catania in Sicily, St. Leo, bishop, illustrious for virtues and miracles.
\switchcolumn*
\selectlanguage{latin}
Eódem die sancti Euchérii, Aurelianénsis Epíscopi, qui eo magis miráculis 
 cláruit, pro plúribus invidórum calúmniis fuit oppréssus.
\switchcolumn
\selectlanguage{english}
The same day, St. Eucherius, bishop of Orleans, who, the more he was 
 oppressed by the calumnies of the envious, the more he impressed them with 
 his miracles.
\switchcolumn*
\selectlanguage{latin}
Tornáci, in Gálliis, sancti Eleuthérii, Epíscopi et Confessóris.
\switchcolumn
\selectlanguage{english}
At Tournai in Belgium, St. Eleutherius, bishop and confessor.
\switchcolumn*
\selectlanguage{latin}
\end{paracol}


% ---- martyrology/mart02/mart0221.htm
\needspace{10\baselineskip}
\begin{paracol}{2}
\selectlanguage{latin}
\begin{center}{\color{gregoriocolor} Nono Kaléndas Mártii. 
 Luna\dots\ }\end{center}
\switchcolumn
\selectlanguage{english}
\begin{center}{\color{gregoriocolor} The Twenty-First Day of 
 February. The\dots\ Day of the Moon.}\end{center}
\end{paracol}

\noindent\begin{tabularx}{\linewidth}{*{19}{>{\centering\arraybackslash}X}}
 \textcolor{gregoriocolor}{a} & \textcolor{gregoriocolor}{b} & \textcolor{gregoriocolor}{c} & \textcolor{gregoriocolor}{d} & \textcolor{gregoriocolor}{e} & \textcolor{gregoriocolor}{f} & \textcolor{gregoriocolor}{g} & \textcolor{gregoriocolor}{h} & \textcolor{gregoriocolor}{i} & \textcolor{gregoriocolor}{k} & \textcolor{gregoriocolor}{l} & \textcolor{gregoriocolor}{m} & \textcolor{gregoriocolor}{n} & \textcolor{gregoriocolor}{p} & \textcolor{gregoriocolor}{q} & \textcolor{gregoriocolor}{r} & \textcolor{gregoriocolor}{s} & \textcolor{gregoriocolor}{t} & \textcolor{gregoriocolor}{u} \\
 23 & 24 & 25 & 26 & 27 & 28 & 29 & 1 & 2 & 3 & 4 & 5 & 6 & 7 & 8 & 9 & 10 & 11 & 12 \\
\end{tabularx}
\vspace{0.5\baselineskip}
\noindent\begin{tabularx}{\linewidth}{*{12}{>{\centering\arraybackslash}X}}
 \textcolor{gregoriocolor}{A} & \textcolor{gregoriocolor}{B} & \textcolor{gregoriocolor}{C} & \textcolor{gregoriocolor}{D} & \textcolor{gregoriocolor}{E} & F & \textcolor{gregoriocolor}{F} & \textcolor{gregoriocolor}{G} & \textcolor{gregoriocolor}{H} & \textcolor{gregoriocolor}{M} & \textcolor{gregoriocolor}{N} & \textcolor{gregoriocolor}{P} \\
 13 & 14 & 15 & 16 & 17 & 18 & 17 & 18 & 19 & 20 & 21 & 22 \\
\end{tabularx}

\begin{paracol}{2}
\selectlanguage{latin}
\lettrine[lines=2]{S}{cythópoli,} in Palæstína, sancti Severiáni, 
 Epíscopi et Mártyris, qui, Eutychiánis acérrime se oppónens, gládio 
 perémptus est.
\switchcolumn
\selectlanguage{english}
\lettrine[lines=2]{A}{t} Scythopolis in Palestine, St. Severian, bishop and martyr, who was 
 beheaded by the Eutychians because he opposed them so zealously.
\switchcolumn*
\selectlanguage{latin}
In Sicília natális sanctórum Mártyrum septuagínta novem, qui sub Diocletiáno, 
 per divérsa torménta, confessiónis suæ corónam 
 percípere meruérunt.
\switchcolumn
\selectlanguage{english}
In Sicily, in the reign of Diocletian, the birthday of seventy-nine holy 
 martyrs, who, by reason of various tortures for their confession of faith, 
 deserved to receive an immortal crown.
\switchcolumn*
\selectlanguage{latin}
Adruméti, in Africa, sanctórum Mártyrum Véruli, 
 Secundíni, Sirícii, Felícis, Sérvuli, Saturníni, Fortunáti et aliórum 
 séxdecim, qui in persecutióne Wandálica, ob cathólicæ fídei confessiónem, 
 martyrio coronáti sunt.
\switchcolumn
\selectlanguage{english}
At Adrumetum in Africa, during the persecution of the Vandals, the holy 
 martyrs, Verulus, Secundinus, Siricius, Felix, Servulus, Saturninus, 
 Fortunatus, and sixteen others, who were crowned with martyrdom for 
 professing the Catholic faith.
\switchcolumn*
\selectlanguage{latin}
Damásci sancti Petri Maviméni, qui, cum díceret Arábibus quibúsdam, ad se
 ægrótum veniéntibus: « Omnis qui fidem 
 Christiánam cathólicam non ampléctitur, damnátus est, sicut et Máhumet, 
 pseudoprophéta vester », ab illis est necatus.
\switchcolumn
\selectlanguage{english}
At Damascus, St. Peter Mavimenus, who was killed by some Arabs who visited 
 him in his sickness, because he said to them: ``Whoever does not embrace the 
 Christian and Catholic faith is lost, like your false prophet Mohammed.''
\switchcolumn*
\selectlanguage{latin}
Metis, in Gállia, sancti Felícis Epíscopi.
\switchcolumn
\selectlanguage{english}
At Metz in France, St. Felix, bishop.
\switchcolumn*
\selectlanguage{latin}
Bríxiæ sancti Patérii Epíscopi.
\switchcolumn
\selectlanguage{english}
At Brescia, St. Paterius, bishop.
\switchcolumn*
\selectlanguage{latin}
\end{paracol}


% ---- martyrology/mart02/mart0222.htm
\needspace{10\baselineskip}
\begin{paracol}{2}
\selectlanguage{latin}
\begin{center}{\color{gregoriocolor} Octávo Kaléndas Mártii. 
 Luna\dots\ }\end{center}
\switchcolumn
\selectlanguage{english}
\begin{center}{\color{gregoriocolor} The Twenty-Second Day of 
 February. The\dots\ Day of the Moon.}\end{center}
\end{paracol}

\noindent\begin{tabularx}{\linewidth}{*{19}{>{\centering\arraybackslash}X}}
 \textcolor{gregoriocolor}{a} & \textcolor{gregoriocolor}{b} & \textcolor{gregoriocolor}{c} & \textcolor{gregoriocolor}{d} & \textcolor{gregoriocolor}{e} & \textcolor{gregoriocolor}{f} & \textcolor{gregoriocolor}{g} & \textcolor{gregoriocolor}{h} & \textcolor{gregoriocolor}{i} & \textcolor{gregoriocolor}{k} & \textcolor{gregoriocolor}{l} & \textcolor{gregoriocolor}{m} & \textcolor{gregoriocolor}{n} & \textcolor{gregoriocolor}{p} & \textcolor{gregoriocolor}{q} & \textcolor{gregoriocolor}{r} & \textcolor{gregoriocolor}{s} & \textcolor{gregoriocolor}{t} & \textcolor{gregoriocolor}{u} \\
 24 & 25 & 26 & 27 & 28 & 29 & 1 & 2 & 3 & 4 & 5 & 6 & 7 & 8 & 9 & 10 & 11 & 12 & 13 \\
\end{tabularx}
\vspace{0.5\baselineskip}
\noindent\begin{tabularx}{\linewidth}{*{12}{>{\centering\arraybackslash}X}}
 \textcolor{gregoriocolor}{A} & \textcolor{gregoriocolor}{B} & \textcolor{gregoriocolor}{C} & \textcolor{gregoriocolor}{D} & \textcolor{gregoriocolor}{E} & F & \textcolor{gregoriocolor}{F} & \textcolor{gregoriocolor}{G} & \textcolor{gregoriocolor}{H} & \textcolor{gregoriocolor}{M} & \textcolor{gregoriocolor}{N} & \textcolor{gregoriocolor}{P} \\
 14 & 15 & 16 & 17 & 18 & 19 & 18 & 19 & 20 & 21 & 22 & 23 \\
\end{tabularx}

\begin{paracol}{2}
\selectlanguage{latin}
\lettrine[lines=2]{A}{ntiochíæ} Cáthedra sancti Petri Apóstoli, ubi 
 primum discípuli cognomináti sunt Christiáni.
\switchcolumn
\selectlanguage{english}
\lettrine[lines=2]{T}{he} Chair of St. Peter at Antioch, where the disciples were first called 
 Christians.
\switchcolumn*
\selectlanguage{latin}
Favéntiæ, in Æmília, natális sancti Petri 
 Damiáni, Cardinális atque Epíscopi Ostiénsis et Confessóris, ex Ordine 
 Camaldulénsi, doctrína et sanctitáte célebris, quem Leo Papa Duodécimus 
 Doctórem universális Ecclésiæ declarávit. Ipsíus autem festum sequénti 
 die celebrátur.
\switchcolumn
\selectlanguage{english}
At Faenza in Emilia, the birthday of St. Peter Damian, cardinal bishop of 
 Ostia and confessor. He was a Camaldolese monk, famous for his 
 learning and sanctity, whom Pope Leo XII declared a doctor of the universal 
 Church. His feast is celebrated tomorrow.
\switchcolumn*
\selectlanguage{latin}
Salamínæ, in Cypro, sancti Aristiónis, qui (ut 
 mox memorándus Pápias testátur) fuit unus de septuagínta duóbus Christi 
 discípulis.
\switchcolumn
\selectlanguage{english}
At Salamis in Cyprus, St. Aristio, who (says Papias, the next to be 
 mentioned) was one of the seventy-two disciples of Christ.
\switchcolumn*
\selectlanguage{latin}
Hierápoli, in Phrygia, beáti Pápiæ, ejúsdem 
 civitátis Epíscopi, qui sancti Joánnis Senióris audítor, Polycárpi autem 
 sodális fuit.
\switchcolumn
\selectlanguage{english}
At Hierapolis in Phrygia, blessed Papias, bishop of that city, who was a 
 companion of Polycarp and a disciple of St. John.
\switchcolumn*
\selectlanguage{latin}
In Arábia commemorátio plurimórum sanctórum Mártyrum, qui, sub Galério 
 Maximiáno Imperatóre, sævíssime cæsi sunt.
\switchcolumn
\selectlanguage{english}
In Arabia, the commemoration of many holy martyrs who were barbarously put 
 to death under Emperor Galerius Maximian.
\switchcolumn*
\selectlanguage{latin}
Alexandríæ sancti Abílii Epíscopi, qui, 
 secúndus post beátum Marcum, factus ejúsdem civitátis Epíscopus, sacerdótium 
 virtúte conspícuus ministrávit.
\switchcolumn
\selectlanguage{english}
At Alexandria, St. Abilias, bishop, who was the second shepherd of that city 
 after St. Mark, and who administered his charge with eminent piety.
\switchcolumn*
\selectlanguage{latin}
Viénnæ, in Gállia, sancti Paschásii Epíscopi, 
 eruditióne et morum sanctitáte præclári.
\switchcolumn
\selectlanguage{english}
At Vienne in France, St. Paschasius, bishop, celebrated for his learning and 
 holy life.
\switchcolumn*
\selectlanguage{latin}
Ravénnæ sancti Maximiáni, Epíscopi et 
 Confessóris.
\switchcolumn
\selectlanguage{english}
At Ravenna, St. Maximian, bishop and confessor.
\switchcolumn*
\selectlanguage{latin}
Cortónæ, in Túscia, sanctæ Margarítæ, ex tértio 
 Ordine sancti Francísci; quæ admirábili pæniténtia et ubérrimis lácrimis 
 máculas anteáctæ vitæ indesinénter abstérsit. Ipsíus corpus, 
 mirabíliter incorrúptum, suávem spirans odórem et crebris miráculis clarum, 
 ibídem magno cum honóre cólitur.
\switchcolumn
\selectlanguage{english}
At Cortona in Tuscany, St. Margaret of the Third Order of St. Francis. 
 By means of commendable penance and fruitful tears, she wiped away the 
 stains of her previous life. Her body miraculously remained incorrupt 
 for more than four centuries, giving forth a sweet odour, and producing 
 frequent miracles. It is honoured in that place with great devotion.
\switchcolumn*
\selectlanguage{latin}
\end{paracol}


% ---- martyrology/mart02/mart0223.htm
\needspace{10\baselineskip}
\begin{paracol}{2}
\selectlanguage{latin}
\begin{center}{\color{gregoriocolor} Séptimo Kaléndas Mártii. 
 Luna\dots\ }\end{center}
\switchcolumn
\selectlanguage{english}
\begin{center}{\color{gregoriocolor} The Twenty-Third Day of 
 February. The\dots\ Day of the Moon.}\end{center}
\end{paracol}

\noindent\begin{tabularx}{\linewidth}{*{19}{>{\centering\arraybackslash}X}}
 \textcolor{gregoriocolor}{a} & \textcolor{gregoriocolor}{b} & \textcolor{gregoriocolor}{c} & \textcolor{gregoriocolor}{d} & \textcolor{gregoriocolor}{e} & \textcolor{gregoriocolor}{f} & \textcolor{gregoriocolor}{g} & \textcolor{gregoriocolor}{h} & \textcolor{gregoriocolor}{i} & \textcolor{gregoriocolor}{k} & \textcolor{gregoriocolor}{l} & \textcolor{gregoriocolor}{m} & \textcolor{gregoriocolor}{n} & \textcolor{gregoriocolor}{p} & \textcolor{gregoriocolor}{q} & \textcolor{gregoriocolor}{r} & \textcolor{gregoriocolor}{s} & \textcolor{gregoriocolor}{t} & \textcolor{gregoriocolor}{u} \\
 25 & 26 & 27 & 28 & 29 & 1 & 2 & 3 & 4 & 5 & 6 & 7 & 8 & 9 & 10 & 11 & 12 & 13 & 14 \\
\end{tabularx}
\vspace{0.5\baselineskip}
\noindent\begin{tabularx}{\linewidth}{*{12}{>{\centering\arraybackslash}X}}
 \textcolor{gregoriocolor}{A} & \textcolor{gregoriocolor}{B} & \textcolor{gregoriocolor}{C} & \textcolor{gregoriocolor}{D} & \textcolor{gregoriocolor}{E} & F & \textcolor{gregoriocolor}{F} & \textcolor{gregoriocolor}{G} & \textcolor{gregoriocolor}{H} & \textcolor{gregoriocolor}{M} & \textcolor{gregoriocolor}{N} & \textcolor{gregoriocolor}{P} \\
 15 & 16 & 17 & 18 & 19 & 20 & 19 & 20 & 21 & 22 & 23 & 24 \\
\end{tabularx}

\begin{paracol}{2}
\selectlanguage{latin}
\lettrine[lines=1]{V}{igília} sancti Matthíæ Apóstoli.
\switchcolumn
\selectlanguage{english}
\lettrine[lines=1]{T}{he} Vigil of St. Matthias the Apostle.
\switchcolumn*
\selectlanguage{latin}
Sancti Petri Damiáni, ex Ordine Camaldulénsi, Cardinális et Epíscopi 
 Ostiénsis, Confessóris et Ecclésiæ Doctóris, 
 qui evolávit in cælum prídie hujus diéi.
\switchcolumn
\selectlanguage{english}
St. Peter Damian, a Camaldolese monk, cardinal bishop of Ostia, confessor 
 and doctor of the Church, who died on the 22nd of February.
\switchcolumn*
\selectlanguage{latin}
Smyrnæ natális sancti Polycárpi, qui, beáti 
 Joánnis Apóstoli discípulus, et ab eo ejúsdem civitátis Epíscopus ordinátus, 
 totíus Asiæ Princeps fuit. Póstea, sub Marco Antoníno et Lúcio Aurélio 
 Cómmodo, sedénte Procónsule et univérso pópulo in theátro advérsus eum 
 personánte, igni tráditus est; et, cum ab igne mínime læderétur, martyrii 
 corónam, gládio confóssus, accépit. Cum illo étiam álii duódecim, qui 
 ex Philadelphía vénerant, in eádem Smyrnénsi urbe, martyrio consummáti sunt. 
 Ipsíus tamen Polycárpi festum séptimo Kaléndas Februárii celebrátur.
\switchcolumn
\selectlanguage{english}
At Smyrna, the birthday of St. Polycarp, a disciple of St. John the Apostle, 
 by whom he was consecrated bishop of that city, and appointed primate of all 
 Asia. Under Marcus Antonius and Lucius Aurelius Commodus, when the 
 proconsul and all those assembled in the amphitheatre cried out against him, 
 he was delivered to the fire, but since it did not harm him, he received the 
 crown of martyrdom by the sword. With him, twelve others who came from 
 Philadelphia met their death by martyrdom in the same city. The feast 
 of St. Polycarp is kept on the 26th of January.
\switchcolumn*
\selectlanguage{latin}
Apud Sírmium beáti Siréni, Mónachi et Mártyris, qui, jubénte Maximiáno 
 Imperatóre, reténtus est, et, cum se Christiánum esse confiterétur, cápite 
 obtruncátus.
\switchcolumn
\selectlanguage{english}
At Sirmio, blessed Sirenus, monk and martyr. He was arrested by order 
 of Emperor Maximian and beheaded for confessing that he was a Christian.
\switchcolumn*
\selectlanguage{latin}
Ibídem natális sanctórum septuagínta duórum Mártyrum, qui, martyrii certámen 
 in præfáta urbe consummántes, mansúra 
 percepérunt regna.
\switchcolumn
\selectlanguage{english}
In the same place, the birthday of seventy-two holy martyrs, who suffered 
 martyrdom in the same city and who took possession of the everlasting 
 kingdom.
\switchcolumn*
\selectlanguage{latin}
In civitáte Asturicénsi, in Hispánia, sanctæ 
 Marthæ, Vírginis et Mártyris, quæ, sub Décio Imperatóre et Patérno 
 Procónsule, dire ob Christi fidem est cruciáta et gládio tandem occísa.
\switchcolumn
\selectlanguage{english}
In the city of Astorga in Spain, St. Martha, virgin and martyr, under 
 Emperor Decius and the proconsul Paternus. She was cruelly tortured 
 for the faith of Christ and was finally slain by the sword.
\switchcolumn*
\selectlanguage{latin}
Constantinópoli sancti Lázari Mónachi, qui, cum 
 sacras Imágines píngeret, idcírco, Imperatóris Iconoclástæ Theóphili jussu, 
 diris suppliciis excruciátur, et ei manus candénti ferro combúritur; sed, 
 Dei virtúte sanátus, abrásas Imágines sanctas pingéndo restítuit, ac demum 
 in pace quiévit.
\switchcolumn
\selectlanguage{english}
At Constantinople, St. Lazarus, monk. The Iconoclast emperor 
 Theophilus commanded him to be tortured with severe punishments because he 
 had painted some sacred pictures. His hand was burned with a hot iron, 
 but it was healed by the power of God, after which he repainted the holy 
 pictures that had been destroyed. He ended his life in peace.
\switchcolumn*
\selectlanguage{latin}
Bríxiæ sancti Felícis Epíscopi.
\switchcolumn
\selectlanguage{english}
At Brescia, St. Felix, bishop.
\switchcolumn*
\selectlanguage{latin}
Romæ sancti Polycárpi Presbyteri, qui, cum 
 beáto Sebastiáno, plúrimos ad Christi fidem convértit, atque ad martyrii 
 glóriam exhortándo perdúxit.
\switchcolumn
\selectlanguage{english}
At Rome, St. Polycarp, priest, who with blessed Sebastian converted many to 
 the faith of Christ, and by his exhortation led them to the glory of 
 martyrdom.
\switchcolumn*
\selectlanguage{latin}
Híspali, in Hispánia, sancti Floréntii 
 Confessóris.
\switchcolumn
\selectlanguage{english}
At Seville in Spain, St. Florentius, confessor.
\switchcolumn*
\selectlanguage{latin}
Tudérti, in Umbria, sanctæ Románæ Vírginis, quæ, 
 a sancto Silvéstro Papa baptizáta, in antris et spelúncis cæléstem vitam 
 duxit, et miraculórum glória cláruit.
\switchcolumn
\selectlanguage{english}
At Todi in Umbria, St. Romana, virgin, who was baptized by Pope St. 
 Sylvester, led a life of holiness in dens and caves, and wrought glorious 
 miracles.
\switchcolumn*
\selectlanguage{latin}
In Anglia sanctæ Milbúrgis Vírginis, fíliæ 
 Regis Merciórum.
\switchcolumn
\selectlanguage{english}
In England, St. Milburga, virgin, the daughter of the king of Mercia.
\switchcolumn*
\selectlanguage{latin}
\end{paracol}


% ---- martyrology/mart02/mart0223leap.htm
\needspace{10\baselineskip}
\begin{paracol}{2}
\selectlanguage{latin}
\begin{center}{\color{gregoriocolor} Séptimo Kaléndas Mártii. 
 Luna\dots\ }\end{center}
\switchcolumn
\selectlanguage{english}
\begin{center}{\color{gregoriocolor} The Twenty-Third Day of 
 February. The\dots\ Day of the Moon.}\end{center}
\end{paracol}

\noindent\begin{tabularx}{\linewidth}{*{19}{>{\centering\arraybackslash}X}}
 \textcolor{gregoriocolor}{a} & \textcolor{gregoriocolor}{b} & \textcolor{gregoriocolor}{c} & \textcolor{gregoriocolor}{d} & \textcolor{gregoriocolor}{e} & \textcolor{gregoriocolor}{f} & \textcolor{gregoriocolor}{g} & \textcolor{gregoriocolor}{h} & \textcolor{gregoriocolor}{i} & \textcolor{gregoriocolor}{k} & \textcolor{gregoriocolor}{l} & \textcolor{gregoriocolor}{m} & \textcolor{gregoriocolor}{n} & \textcolor{gregoriocolor}{p} & \textcolor{gregoriocolor}{q} & \textcolor{gregoriocolor}{r} & \textcolor{gregoriocolor}{s} & \textcolor{gregoriocolor}{t} & \textcolor{gregoriocolor}{u} \\
 25 & 26 & 27 & 28 & 29 & 1 & 2 & 3 & 4 & 5 & 6 & 7 & 8 & 9 & 10 & 11 & 12 & 13 & 14 \\
\end{tabularx}
\vspace{0.5\baselineskip}
\noindent\begin{tabularx}{\linewidth}{*{12}{>{\centering\arraybackslash}X}}
 \textcolor{gregoriocolor}{A} & \textcolor{gregoriocolor}{B} & \textcolor{gregoriocolor}{C} & \textcolor{gregoriocolor}{D} & \textcolor{gregoriocolor}{E} & F & \textcolor{gregoriocolor}{F} & \textcolor{gregoriocolor}{G} & \textcolor{gregoriocolor}{H} & \textcolor{gregoriocolor}{M} & \textcolor{gregoriocolor}{N} & \textcolor{gregoriocolor}{P} \\
 15 & 16 & 17 & 18 & 19 & 20 & 19 & 20 & 21 & 22 & 23 & 24 \\
\end{tabularx}

\begin{paracol}{2}
\selectlanguage{latin}
\lettrine[lines=2]{S}{ancti} Petri Damiáni, ex Ordine Camaldulénsi, Cardinális et Epíscopi 
 Ostiénsis, Confessóris et Ecclésiæ Doctoris, 
 qui evolávit in cælum prídie hujus diéi.
\switchcolumn
\selectlanguage{english}
\lettrine[lines=2]{S}{t.} Peter Damian, a Camaldolese monk, cardinal bishop of Ostia, confessor 
 and doctor of the Church, who died on the 22nd of February.
\switchcolumn*
\selectlanguage{latin}
Smyrnæ natális sancti Polycárpi, qui, beáti 
 Joánnis Apóstoli discípulus, et ab eo ejúsdem civitátis Epíscopus ordinátus, 
 totíus Asiæ Princeps fuit. Póstea, sub Marco Antoníno et Lúcio Aurélio 
 Cómmodo, sedénte Procónsule et univérso pópulo in theátro advérsus eum 
 personánte, igni tráditus est; et, cum ab igne mínime læderétur, martyrii 
 corónam, gládio confóssus, accépit. Cum illo étiam álii duódecim, qui 
 ex Philadelphía vénerant, in eádem Smyrnénsi urbe, martyrio consummáti sunt. 
 Ipsíus tamen Polycárpi festum séptimo Kaléndas Februárii celebrátur.
\switchcolumn
\selectlanguage{english}
At Smyrna, the birthday of St. Polycarp, a disciple of St. John the Apostle, 
 by whom he was consecrated bishop of that city, and appointed primate of all 
 Asia. Under Marcus Antonius and Lucius Aurelius Commodus, when the 
 proconsul and all those assembled in the amphitheatre cried out against him, 
 he was delivered to the fire, but since it did not harm him, he received the 
 crown of martyrdom by the sword. With him, twelve others who came from 
 Philadelphia met their death by martyrdom in the same city. The feast 
 of St. Polycarp is kept on the 26th of January.
\switchcolumn*
\selectlanguage{latin}
Apud Sírmium beáti Siréni, Mónachi et Mártyris, qui, jubénte Maximiáno 
 Imperatóre, reténtus est, et, cum se Christiánum esse confiterétur, cápite 
 obtruncátus.
\switchcolumn
\selectlanguage{english}
At Sirmio, blessed Sirenus, monk and martyr. He was arrested by order 
 of Emperor Maximian and beheaded for confessing that he was a Christian.
\switchcolumn*
\selectlanguage{latin}
Ibídem natális sanctórum septuagínta duórum Mártyrum, qui, martyrii certámen 
 in præfáta urbe consummántes, mansúra 
 percepérunt regna.
\switchcolumn
\selectlanguage{english}
In the same place, the birthday of seventy-two holy martyrs, who suffered 
 martyrdom in the same city and who took possession of the everlasting 
 kingdom.
\switchcolumn*
\selectlanguage{latin}
In civitáte Asturicénsi, in Hispánia, sanctæ 
 Marthæ, Vírginis et Mártyris, quæ, sub Décio Imperatóre et Patérno 
 Procónsule, dire ob Christi fidem est cruciáta et gládio tandem occísa.
\switchcolumn
\selectlanguage{english}
In the city of Astorga in Spain, St. Martha, virgin and martyr, under 
 Emperor Decius and the proconsul Paternus. She was cruelly tortured 
 for the faith of Christ and was finally slain by the sword.
\switchcolumn*
\selectlanguage{latin}
Constantinópoli sancti Lázari Mónachi, qui, cum 
 sacras Imágines píngeret, idcírco, Imperatóris Iconoclástæ Theóphili jussu, 
 diris supplíciis excruciátur, et ei manus candénti ferro combúritur; sed, 
 Dei virtúte sanátus, abrásas Imágines sanctas pingéndo restítuit, ac demum 
 in pace quiévit.
\switchcolumn
\selectlanguage{english}
At Constantinople, St. Lazarus, monk. The Iconoclast emperor 
 Theophilus commanded him to be tortured with severe punishments because he 
 had painted some sacred pictures. His hand was burned with a hot iron, 
 but it was healed by the power of God, after which he repainted the holy 
 pictures that had been destroyed. He ended his life in peace.
\switchcolumn*
\selectlanguage{latin}
Bríxiæ sancti Felícis Epíscopi.
\switchcolumn
\selectlanguage{english}
At Brescia, St. Felix, bishop.
\switchcolumn*
\selectlanguage{latin}
Romæ sancti Polycárpi Presbyteri, qui, cum 
 beáto Sebastiáno, plúrimos ad Christi fidem convértit, atque ad martyrii 
 glóriam exhortándo perdúxit.
\switchcolumn
\selectlanguage{english}
At Rome, St. Polycarp, priest, who with blessed Sebastian converted many to 
 the faith of Christ, and by his exhortation led them to the glory of 
 martyrdom.
\switchcolumn*
\selectlanguage{latin}
Híspali, in Hispánia, sancti Floréntii 
 Confessóris.
\switchcolumn
\selectlanguage{english}
At Seville in Spain, St. Florentius, confessor.
\switchcolumn*
\selectlanguage{latin}
Tudérti, in Umbria, sanctæ Románæ Vírginis, quæ, 
 a sancto Silvéstro Papa baptizáta, in antris et spelúncis cæléstem vitam 
 duxit, et miraculórum glória cláruit.
\switchcolumn
\selectlanguage{english}
At Todi in Umbria, St. Romana, virgin, who was baptized by Pope St. 
 Sylvester, led a life of holiness in dens and caves, and wrought glorious 
 miracles.
\switchcolumn*
\selectlanguage{latin}
In Anglia sanctæ Milbúrgis Vírginis, fíliæ 
 Regis Merciórum.
\switchcolumn
\selectlanguage{english}
In England, St. Milburga, virgin, the daughter of the king of Mercia.
\switchcolumn*
\selectlanguage{latin}
\end{paracol}


% ---- martyrology/mart02/mart0224.htm
\needspace{10\baselineskip}
\begin{paracol}{2}
\selectlanguage{latin}
\begin{center}{\color{gregoriocolor} Sexto Kaléndas Mártii. 
 Luna\dots\ }\end{center}
\switchcolumn
\selectlanguage{english}
\begin{center}{\color{gregoriocolor} The Twenty-Fourth Day of February. The\dots\ Day of the Moon.}\end{center}
\end{paracol}

\noindent\begin{tabularx}{\linewidth}{*{19}{>{\centering\arraybackslash}X}}
 \textcolor{gregoriocolor}{a} & \textcolor{gregoriocolor}{b} & \textcolor{gregoriocolor}{c} & \textcolor{gregoriocolor}{d} & \textcolor{gregoriocolor}{e} & \textcolor{gregoriocolor}{f} & \textcolor{gregoriocolor}{g} & \textcolor{gregoriocolor}{h} & \textcolor{gregoriocolor}{i} & \textcolor{gregoriocolor}{k} & \textcolor{gregoriocolor}{l} & \textcolor{gregoriocolor}{m} & \textcolor{gregoriocolor}{n} & \textcolor{gregoriocolor}{p} & \textcolor{gregoriocolor}{q} & \textcolor{gregoriocolor}{r} & \textcolor{gregoriocolor}{s} & \textcolor{gregoriocolor}{t} & \textcolor{gregoriocolor}{u} \\
 26 & 27 & 28 & 29 & 1 & 2 & 3 & 4 & 5 & 6 & 7 & 8 & 9 & 10 & 11 & 12 & 13 & 14 & 15 \\
\end{tabularx}
\vspace{0.5\baselineskip}
\noindent\begin{tabularx}{\linewidth}{*{12}{>{\centering\arraybackslash}X}}
 \textcolor{gregoriocolor}{A} & \textcolor{gregoriocolor}{B} & \textcolor{gregoriocolor}{C} & \textcolor{gregoriocolor}{D} & \textcolor{gregoriocolor}{E} & F & \textcolor{gregoriocolor}{F} & \textcolor{gregoriocolor}{G} & \textcolor{gregoriocolor}{H} & \textcolor{gregoriocolor}{M} & \textcolor{gregoriocolor}{N} & \textcolor{gregoriocolor}{P} \\
 16 & 17 & 18 & 19 & 20 & 21 & 20 & 21 & 22 & 23 & 24 & 25 \\
\end{tabularx}

\begin{paracol}{2}
\selectlanguage{latin}
\lettrine[lines=2]{I}{n} Judæa natális sancti Matthíæ Apóstoli, qui, 
 post Ascensiónem Dómini ab Apóstolis in Judæ proditóris locum sorte eléctus, 
 pro Evangélii prædicatióne martyrium passus est.
\switchcolumn
\selectlanguage{english}
\lettrine[lines=2]{I}{n} Judea, the birthday of St. Matthias the Apostle. After the 
 Ascension of our Lord, the Apostles chose him, by lot, to fill the place of 
 Judas the traitor, and he suffered martyrdom for the preaching of the 
 Gospel.
\switchcolumn*
\selectlanguage{latin}
Romæ sanctæ Primitívæ Mártyris.
\switchcolumn
\selectlanguage{english}
At Rome, St. Primitiva, martyr.
\switchcolumn*
\selectlanguage{latin}
Rotómagi pássio sancti Prætextáti, Epíscopi et 
 Mártyris.
\switchcolumn
\selectlanguage{english}
At Rouen, the passion of St. Praetextatus, bishop and martyr.
\switchcolumn*
\selectlanguage{latin}
Cæsaréæ, in Cappadócia, sancti Sérgii Mártyris, 
 cujus gesta præclára habéntur.
\switchcolumn
\selectlanguage{english}
At Caesarea in Cappadocia, St. Sergius, martyr, of whose life a beautiful 
 account still exists.
\switchcolumn*
\selectlanguage{latin}
In Africa sanctórum Mártyrum Montáni, Lúcii, Juliáni, Victórici, 
 Flaviáni et Sociórum, qui discípuli fuérunt sancti Cypriáni, et, sub 
 Valeriáno Imperatóre, martyrium consummárunt.
\switchcolumn
\selectlanguage{english}
In Africa, the holy martyrs Montanus, Lucius, Julian, Victoricus, Flavian, 
 and their companions. They were disciples of St. Cyprian and suffered 
 martyrdom under Emperor Valerian.
\switchcolumn*
\selectlanguage{latin}
Tréviris sancti Modésti, Epíscopi et 
 Confessóris.
\switchcolumn
\selectlanguage{english}
At Treves, St. Modestus, bishop and confessor.
\switchcolumn*
\selectlanguage{latin}
Apud Stylum, in Calábria, sancti Joánnis, cognoménto Therísti, monásticæ 
 vitæ laude, et sanctitáte insígnis.
\switchcolumn
\selectlanguage{english}
At Stylo in Calabria, St. John Therestus, noted for his sanctity, and his 
 high regard for the monastic life.
\switchcolumn*
\selectlanguage{latin}
In Anglia sancti Edilbérti, Regis Cantiórum, quem sanctus Augustínus, 
 Anglórum Epíscopus, ad Christi fidem convértit.
\switchcolumn
\selectlanguage{english}
In England, St. Ethelbert, ruler of Kent, converted to the faith of Christ 
 by the English bishop, St. Augustine.
\switchcolumn*
\selectlanguage{latin}
Hierosólymis prima Invéntio cápitis sancti Joánnis Baptístæ, 
 Præcursóris Domini.
\switchcolumn
\selectlanguage{english}
At Jerusalem, the finding for the first time of the head of St. John the 
 Baptist, Precursor of the Lord.
\switchcolumn*
\selectlanguage{latin}
\end{paracol}


% ---- martyrology/mart02/mart0224leap.htm
\needspace{10\baselineskip}
\begin{paracol}{2}
\selectlanguage{latin}
\begin{center}{\color{gregoriocolor} Sexto Kaléndas Mártii. 
 Luna\dots\ }\end{center}
\switchcolumn
\selectlanguage{english}
\begin{center}{\color{gregoriocolor} The Twenty-Fourth Day of February. The\dots\ Day of the Moon.}\end{center}
\end{paracol}

\noindent\begin{tabularx}{\linewidth}{*{19}{>{\centering\arraybackslash}X}}
 \textcolor{gregoriocolor}{a} & \textcolor{gregoriocolor}{b} & \textcolor{gregoriocolor}{c} & \textcolor{gregoriocolor}{d} & \textcolor{gregoriocolor}{e} & \textcolor{gregoriocolor}{f} & \textcolor{gregoriocolor}{g} & \textcolor{gregoriocolor}{h} & \textcolor{gregoriocolor}{i} & \textcolor{gregoriocolor}{k} & \textcolor{gregoriocolor}{l} & \textcolor{gregoriocolor}{m} & \textcolor{gregoriocolor}{n} & \textcolor{gregoriocolor}{p} & \textcolor{gregoriocolor}{q} & \textcolor{gregoriocolor}{r} & \textcolor{gregoriocolor}{s} & \textcolor{gregoriocolor}{t} & \textcolor{gregoriocolor}{u} \\
 26 & 27 & 28 & 29 & 1 & 2 & 3 & 4 & 5 & 6 & 7 & 8 & 9 & 10 & 11 & 12 & 13 & 14 & 15 \\
\end{tabularx}
\vspace{0.5\baselineskip}
\noindent\begin{tabularx}{\linewidth}{*{12}{>{\centering\arraybackslash}X}}
 \textcolor{gregoriocolor}{A} & \textcolor{gregoriocolor}{B} & \textcolor{gregoriocolor}{C} & \textcolor{gregoriocolor}{D} & \textcolor{gregoriocolor}{E} & F & \textcolor{gregoriocolor}{F} & \textcolor{gregoriocolor}{G} & \textcolor{gregoriocolor}{H} & \textcolor{gregoriocolor}{M} & \textcolor{gregoriocolor}{N} & \textcolor{gregoriocolor}{P} \\
 16 & 17 & 18 & 19 & 20 & 21 & 20 & 21 & 22 & 23 & 24 & 25 \\
\end{tabularx}

% NOTE: this is a special case, so we keep the conclusion and response
\begin{paracol}{2}
\selectlanguage{latin}
\lettrine[lines=2]{V}{igília} sancti Matthíæ 
Apóstoli.
\switchcolumn
\selectlanguage{english}
\lettrine[lines=2]{T}{he} Vigil of St. Matthias the Apostle.
\switchcolumn*
\selectlanguage{latin}
Item commemorátio plurimórum sanctórum Mártyrum et Confessórum, atque sanctárum Vírginum. \textcolor{gregoriocolor}{\Rbar.}~Deo grátias.
\switchcolumn
\selectlanguage{english}
Also the commemoration of many other holy martyrs, confessors, and holy virgins. \textcolor{gregoriocolor}{\Rbar.}~Thanks be to God.
\switchcolumn*
\selectlanguage{latin}
\end{paracol}


% ---- martyrology/mart02/mart0225.htm
\needspace{10\baselineskip}
\begin{paracol}{2}
\selectlanguage{latin}
\begin{center}{\color{gregoriocolor} Quinto Kaléndas Mártii. 
 Luna\dots\ }\end{center}
\switchcolumn
\selectlanguage{english}
\begin{center}{\color{gregoriocolor} The Twenty-Fifth Day of February. The\dots\ Day of the Moon.}\end{center}
\end{paracol}

\noindent\begin{tabularx}{\linewidth}{*{19}{>{\centering\arraybackslash}X}}
 \textcolor{gregoriocolor}{a} & \textcolor{gregoriocolor}{b} & \textcolor{gregoriocolor}{c} & \textcolor{gregoriocolor}{d} & \textcolor{gregoriocolor}{e} & \textcolor{gregoriocolor}{f} & \textcolor{gregoriocolor}{g} & \textcolor{gregoriocolor}{h} & \textcolor{gregoriocolor}{i} & \textcolor{gregoriocolor}{k} & \textcolor{gregoriocolor}{l} & \textcolor{gregoriocolor}{m} & \textcolor{gregoriocolor}{n} & \textcolor{gregoriocolor}{p} & \textcolor{gregoriocolor}{q} & \textcolor{gregoriocolor}{r} & \textcolor{gregoriocolor}{s} & \textcolor{gregoriocolor}{t} & \textcolor{gregoriocolor}{u} \\
 27 & 28 & 29 & 1 & 2 & 3 & 4 & 5 & 6 & 7 & 8 & 9 & 10 & 11 & 12 & 13 & 14 & 15 & 16 \\
\end{tabularx}
\vspace{0.5\baselineskip}
\noindent\begin{tabularx}{\linewidth}{*{12}{>{\centering\arraybackslash}X}}
 \textcolor{gregoriocolor}{A} & \textcolor{gregoriocolor}{B} & \textcolor{gregoriocolor}{C} & \textcolor{gregoriocolor}{D} & \textcolor{gregoriocolor}{E} & F & \textcolor{gregoriocolor}{F} & \textcolor{gregoriocolor}{G} & \textcolor{gregoriocolor}{H} & \textcolor{gregoriocolor}{M} & \textcolor{gregoriocolor}{N} & \textcolor{gregoriocolor}{P} \\
 17 & 18 & 19 & 20 & 21 & 22 & 21 & 22 & 23 & 24 & 25 & 26 \\
\end{tabularx}

\begin{paracol}{2}
\selectlanguage{latin}
\lettrine[lines=2]{I}{n} Ægypto natális sanctórum Mártyrum Victoríni, 
 Victóris, Nicéphori, Claudiáni, Dióscori, Serapiónis et Pápiæ, sub Numeriáno 
 Imperatóre. Horum duo primi, pro confessióne fídei, exquisíta 
 suppliciórum génera constánter passi, cápite plectúntur; Nicéphorus, post 
 cratículas candéntes ignésque superátos, minutátim concísus est; Claudiánus 
 et Dióscorus flammis incénsi; Serápion vero et Pápias gládio cæsi sunt.
\switchcolumn
\selectlanguage{english}
\lettrine[lines=2]{I}{n} Egypt, under Emperor Numerian, the birthday of the holy martyrs 
 Victorinus, Victor, Nicephorus, Claudian, Dioscorus, Serapion, and Papias. 
 After patiently enduring extreme tortures, the first two were beheaded for 
 the confession of the faith, Nicephorus was laid on a heated gridiron, 
 placed over the fire, then thoroughly hacked with a knife; Claudian and 
 Dioscorus were burned at the stake; Serapion and Papias were slain with the 
 sword.
\switchcolumn*
\selectlanguage{latin}
In Africa sanctórum Mártyrum Donáti, Justi, Herénæ 
 et Sociórum.
\switchcolumn
\selectlanguage{english}
In Africa, the holy martyrs Donatus, Justus, Herenas, and their companions.
\switchcolumn*
\selectlanguage{latin}
Constantinópoli sancti Tharásii Epíscopi, eruditióne et pietáte insígnis; ad 
 quem exstat Hadriáni Papæ Primi epístola pro 
 defensióne sanctárum Imáginum.
\switchcolumn
\selectlanguage{english}
At Constantinople, St. Tharasius, bishop, a man of great learning and piety. 
 There exists a letter defending sacred images, written to him by Pope 
 Hadrian I.
\switchcolumn*
\selectlanguage{latin}
Naziánzi, in Cappadócia, sancti Cæsárii, qui 
 beátæ Nonnæ fílius ac beatórum Gregórii Theólogi et Gorgóniæ fuit frater, et 
 quem idem Gregórius inter ágmina beatórum se vidísse testátur.
\switchcolumn
\selectlanguage{english}
At Nazianzus, St. Caesarius, who was the son of blessed Nonna, and whom his 
 brother, blessed Gregory the Theologian, says he saw among the hosts of the 
 blessed.
\switchcolumn*
\selectlanguage{latin}
In monastério Heidenhémii, diœcésis 
 Eistetténsis, in Germánia, sanctæ Walbúrgæ Vírginis, quæ fuit fília sancti 
 Richárdi, Anglórum Regis, et soror sancti Willebáldi, Eistetténsis Epíscopi.
\switchcolumn
\selectlanguage{english}
In the monastery of Heidenheim, in the Eichstadt diocese in Germany, St. 
 Walburga, virgin. She was the daughter of St. Richard, king of 
 England, and sister of St. Willebald, bishop of Eichstadt.
\switchcolumn*
\selectlanguage{latin}
\end{paracol}


% ---- martyrology/mart02/mart0225leap.htm
\needspace{10\baselineskip}
\begin{paracol}{2}
\selectlanguage{latin}
\begin{center}{\color{gregoriocolor} Sexto Kaléndas Mártii. 
 Luna\dots\ }\end{center}
\switchcolumn
\selectlanguage{english}
\begin{center}{\color{gregoriocolor} The Twenty-Fifth Day of February. The\dots\ Day of the Moon.}\end{center}
\end{paracol}

\noindent\begin{tabularx}{\linewidth}{*{19}{>{\centering\arraybackslash}X}}
 \textcolor{gregoriocolor}{a} & \textcolor{gregoriocolor}{b} & \textcolor{gregoriocolor}{c} & \textcolor{gregoriocolor}{d} & \textcolor{gregoriocolor}{e} & \textcolor{gregoriocolor}{f} & \textcolor{gregoriocolor}{g} & \textcolor{gregoriocolor}{h} & \textcolor{gregoriocolor}{i} & \textcolor{gregoriocolor}{k} & \textcolor{gregoriocolor}{l} & \textcolor{gregoriocolor}{m} & \textcolor{gregoriocolor}{n} & \textcolor{gregoriocolor}{p} & \textcolor{gregoriocolor}{q} & \textcolor{gregoriocolor}{r} & \textcolor{gregoriocolor}{s} & \textcolor{gregoriocolor}{t} & \textcolor{gregoriocolor}{u} \\
 26 & 27 & 28 & 29 & 1 & 2 & 3 & 4 & 5 & 6 & 7 & 8 & 9 & 10 & 11 & 12 & 13 & 14 & 15 \\
\end{tabularx}
\vspace{0.5\baselineskip}
\noindent\begin{tabularx}{\linewidth}{*{12}{>{\centering\arraybackslash}X}}
 \textcolor{gregoriocolor}{A} & \textcolor{gregoriocolor}{B} & \textcolor{gregoriocolor}{C} & \textcolor{gregoriocolor}{D} & \textcolor{gregoriocolor}{E} & F & \textcolor{gregoriocolor}{F} & \textcolor{gregoriocolor}{G} & \textcolor{gregoriocolor}{H} & \textcolor{gregoriocolor}{M} & \textcolor{gregoriocolor}{N} & \textcolor{gregoriocolor}{P} \\
 16 & 17 & 18 & 19 & 20 & 21 & 20 & 21 & 22 & 23 & 24 & 25 \\
\end{tabularx}

\begin{paracol}{2}
\selectlanguage{latin}
\lettrine[lines=2]{I}{n} Judæa natális sancti Matthíæ Apóstoli, qui, 
 post Ascensiónem Dómini ab Apóstolis in Judæ proditóris locum sorte eléctus, 
 pro Evangélii prædicatióne martyrium passus est.
\switchcolumn
\selectlanguage{english}
\lettrine[lines=2]{I}{n} Judea, the birthday of St. Matthias the Apostle. After the 
 Ascension of our Lord, the Apostles chose him, by lot, to fill the place of 
 Judas the traitor, and he suffered martyrdom for the preaching of the 
 Gospel.
\switchcolumn*
\selectlanguage{latin}
Romæ sanctæ Primitívæ Mártyris.
\switchcolumn
\selectlanguage{english}
At Rome, St. Primitiva, martyr.
\switchcolumn*
\selectlanguage{latin}
Rotómagi pássio sancti Prætextáti, Epíscopi et 
 Mártyris.
\switchcolumn
\selectlanguage{english}
At Rouen, the passion of St. Praetextatus, bishop and martyr.
\switchcolumn*
\selectlanguage{latin}
Cæsaréæ, in Cappadócia, sancti Sérgii Mártyris, 
 cujus gesta præclára habéntur.
\switchcolumn
\selectlanguage{english}
At Caesarea in Cappadocia, St. Sergius, martyr, of whose life a beautiful 
 account still exists.
\switchcolumn*
\selectlanguage{latin}
In Africa sanctórum Mártyrum Montáni, Lúcii, Juliáni, Victórici, 
 Flaviáni et Sociórum, qui discípuli fuérunt sancti Cypriáni, et, sub 
 Valeriáno Imperatóre, martyrium consummárunt.
\switchcolumn
\selectlanguage{english}
In Africa, the holy martyrs Montanus, Lucius, Julian, Victoricus, Flavian, 
 and their companions. They were disciples of St. Cyprian and suffered 
 martyrdom under Emperor Valerian.
\switchcolumn*
\selectlanguage{latin}
Tréviris sancti Modésti, Epíscopi et 
 Confessóris.
\switchcolumn
\selectlanguage{english}
At Treves, St. Modestus, bishop and confessor.
\switchcolumn*
\selectlanguage{latin}
Apud Stylum, in Calábria, sancti Joánnis, cognoménto Therísti, monásticæ 
 vitæ laude, et sanctitáte insígnis.
\switchcolumn
\selectlanguage{english}
At Stylo in Calabria, St. John Therestus, noted for his sanctity, and his 
 high regard for the monastic life.
\switchcolumn*
\selectlanguage{latin}
In Anglia sancti Edilbérti, Regis Cantiórum, quem sanctus Augustínus, 
 Anglórum Epíscopus, ad Christi fidem convértit.
\switchcolumn
\selectlanguage{english}
In England, St. Ethelbert, ruler of Kent, converted to the faith of Christ 
 by the English bishop, St. Augustine.
\switchcolumn*
\selectlanguage{latin}
Hierosólymis prima Invéntio cápitis sancti Joánnis Baptístæ, 
 Præcursóris Domini.
\switchcolumn
\selectlanguage{english}
At Jerusalem, the finding for the first time of the head of St. John the 
 Baptist, Precursor of the Lord.
\switchcolumn*
\selectlanguage{latin}
\end{paracol}


% ---- martyrology/mart02/mart0226.htm
\needspace{10\baselineskip}
\begin{paracol}{2}
\selectlanguage{latin}
\begin{center}{\color{gregoriocolor} Quarto Kaléndas Mártii. 
 Luna\dots\ }\end{center}
\switchcolumn
\selectlanguage{english}
\begin{center}{\color{gregoriocolor} The Twenty-Sixth Day of February. The\dots\ Day of the Moon.}\end{center}
\end{paracol}

\noindent\begin{tabularx}{\linewidth}{*{19}{>{\centering\arraybackslash}X}}
 \textcolor{gregoriocolor}{a} & \textcolor{gregoriocolor}{b} & \textcolor{gregoriocolor}{c} & \textcolor{gregoriocolor}{d} & \textcolor{gregoriocolor}{e} & \textcolor{gregoriocolor}{f} & \textcolor{gregoriocolor}{g} & \textcolor{gregoriocolor}{h} & \textcolor{gregoriocolor}{i} & \textcolor{gregoriocolor}{k} & \textcolor{gregoriocolor}{l} & \textcolor{gregoriocolor}{m} & \textcolor{gregoriocolor}{n} & \textcolor{gregoriocolor}{p} & \textcolor{gregoriocolor}{q} & \textcolor{gregoriocolor}{r} & \textcolor{gregoriocolor}{s} & \textcolor{gregoriocolor}{t} & \textcolor{gregoriocolor}{u} \\
 28 & 29 & 1 & 2 & 3 & 4 & 5 & 6 & 7 & 8 & 9 & 10 & 11 & 12 & 13 & 14 & 15 & 16 & 17 \\
\end{tabularx}
\vspace{0.5\baselineskip}
\noindent\begin{tabularx}{\linewidth}{*{12}{>{\centering\arraybackslash}X}}
 \textcolor{gregoriocolor}{A} & \textcolor{gregoriocolor}{B} & \textcolor{gregoriocolor}{C} & \textcolor{gregoriocolor}{D} & \textcolor{gregoriocolor}{E} & F & \textcolor{gregoriocolor}{F} & \textcolor{gregoriocolor}{G} & \textcolor{gregoriocolor}{H} & \textcolor{gregoriocolor}{M} & \textcolor{gregoriocolor}{N} & \textcolor{gregoriocolor}{P} \\
 18 & 19 & 20 & 21 & 22 & 23 & 22 & 23 & 24 & 25 & 26 & 27 \\
\end{tabularx}

\begin{paracol}{2}
\selectlanguage{latin}
\lettrine[lines=2]{P}{erge,} in Pamphylia, natális beáti Néstoris 
 Epíscopi, qui, in persecutióne Décii, cum diu noctúque oratióni insísteret 
 póstulans ut grex Christi custodirétur, comprehénsus est, ac, nomen Dómini 
 mira liberáte et alacritáte conféssus, Præsidis Polliónis jussu equúleo 
 sævíssime est cruciátus; ac demum, cum se Christo semper adhæsúrum 
 constánter profiterétur, crucis suspéndio victor in cælum migrávit.
\switchcolumn
\selectlanguage{english}
\lettrine[lines=2]{A}{t} Pergen in Pamphylia, during the persecution of Decius, the birthday of 
 blessed Nestor, bishop, who praying night and day for the safety of the 
 flock of Christ, was put under arrest. Because he confessed the Name 
 of the Lord with great zeal and freedom, he was cruelly tortured on the rack 
 by order of Pollio the governor. When he still courageously proclaimed 
 that he would remain ever faithful to Christ, he was crucified, and thus 
 triumphantly went to heaven.
\switchcolumn*
\selectlanguage{latin}
Ibídem pássio sanctórum Pápiæ, Diodóri, Conónis 
 et Claudiáni, qui sanctum Néstorem martyrio præcessérunt.
\switchcolumn
\selectlanguage{english}
In the same place, the passion of Saints Papias, Diodorus, Conon, and 
 Claudian, who preceded St. Nestor to martyrdom.
\switchcolumn*
\selectlanguage{latin}
Item sanctórum Mártyrum Fortunáti, Felícis et aliórum vigínti septem.
\switchcolumn
\selectlanguage{english}
Also, the holy martyrs Fortunatus, Felix, and twenty-seven others.
\switchcolumn*
\selectlanguage{latin}
Alexandríæ sancti Alexándri Epíscopi, gloriósi 
 senis, qui, post beátum Petrum, ejúsdem civitátis Epíscopum, zelo fídei 
 succénsus, Aríum, Presbyterum suum, hærética impietáte depravátum et divina 
 veritáte convíctum, de Ecclésia ejécit; ac póstea inter trecéntos decem et 
 octo Patres, in Nicǽno Concílio eúndem damnávit.
\switchcolumn
\selectlanguage{english}
At Alexandria, Bishop St. Alexander, an aged man held in great honour, who 
 succeeded blessed Peter as bishop of that city. He expelled Arius, one 
 of his priests, from the Church because he was tainted with heretical 
 ímpiety and convicted in the face of divine truth. Later on he was one 
 of the three hundred and eighteen Fathers who condemned him in the Council 
 of Nicaea.
\switchcolumn*
\selectlanguage{latin}
Bonóniæ sancti Faustiniáni Epíscopi, qui eam 
 Ecclésiam, Diocletiáni persecutióne vexátam, verbo prædicatiónis firmávit et 
 auxit.
\switchcolumn
\selectlanguage{english}
At Bologna, the bishop St. Faustinian. His preaching strengthened and 
 multiplied the faithful of that church when it was so much afflicted during 
 the persecution of Diocletian.
\switchcolumn*
\selectlanguage{latin}
Gazæ, in Palæstína, sancti Porphyrii Epíscopi, 
 qui, témpore Arcádii Imperatóris, Marnam idólum ejúsque templum evértit, ac, 
 multa passus, quiévit in Dómino.
\switchcolumn
\selectlanguage{english}
At Gaza in Palestine, St. Porphyry, bishop, in the time of Emperor Arcadius. 
 He overthrew the idol Marna and its temple, and after many sufferings, went 
 to his rest in the Lord.
\switchcolumn*
\selectlanguage{latin}
Floréntiæ sancti Andréæ, Epíscopi et 
 Confessóris.
\switchcolumn
\selectlanguage{english}
At Florence, St. Andrew, bishop and confessor.
\switchcolumn*
\selectlanguage{latin}
In território Archiacénsi, in Gállia, sancti Victóris Confessóris, cujus 
 laudes sanctus Bernárdus conscrípsit.
\switchcolumn
\selectlanguage{english}
In the province of Champagne in France, St. Victor, confessor, about whom 
 eulogies have been written by St. Bernard.
\switchcolumn*
\selectlanguage{latin}
\end{paracol}


% ---- martyrology/mart02/mart0226leap.htm
\needspace{10\baselineskip}
\begin{paracol}{2}
\selectlanguage{latin}
\begin{center}{\color{gregoriocolor} Quinto Kaléndas Mártii. 
 Luna\dots\ }\end{center}
\switchcolumn
\selectlanguage{english}
\begin{center}{\color{gregoriocolor} The Twenty-Sixth Day of February. The\dots\ Day of the Moon.}\end{center}
\end{paracol}

\noindent\begin{tabularx}{\linewidth}{*{19}{>{\centering\arraybackslash}X}}
 \textcolor{gregoriocolor}{a} & \textcolor{gregoriocolor}{b} & \textcolor{gregoriocolor}{c} & \textcolor{gregoriocolor}{d} & \textcolor{gregoriocolor}{e} & \textcolor{gregoriocolor}{f} & \textcolor{gregoriocolor}{g} & \textcolor{gregoriocolor}{h} & \textcolor{gregoriocolor}{i} & \textcolor{gregoriocolor}{k} & \textcolor{gregoriocolor}{l} & \textcolor{gregoriocolor}{m} & \textcolor{gregoriocolor}{n} & \textcolor{gregoriocolor}{p} & \textcolor{gregoriocolor}{q} & \textcolor{gregoriocolor}{r} & \textcolor{gregoriocolor}{s} & \textcolor{gregoriocolor}{t} & \textcolor{gregoriocolor}{u} \\
 27 & 28 & 29 & 1 & 2 & 3 & 4 & 5 & 6 & 7 & 8 & 9 & 10 & 11 & 12 & 13 & 14 & 15 & 16 \\
\end{tabularx}
\vspace{0.5\baselineskip}
\noindent\begin{tabularx}{\linewidth}{*{12}{>{\centering\arraybackslash}X}}
 \textcolor{gregoriocolor}{A} & \textcolor{gregoriocolor}{B} & \textcolor{gregoriocolor}{C} & \textcolor{gregoriocolor}{D} & \textcolor{gregoriocolor}{E} & F & \textcolor{gregoriocolor}{F} & \textcolor{gregoriocolor}{G} & \textcolor{gregoriocolor}{H} & \textcolor{gregoriocolor}{M} & \textcolor{gregoriocolor}{N} & \textcolor{gregoriocolor}{P} \\
 17 & 18 & 19 & 20 & 21 & 22 & 21 & 22 & 23 & 24 & 25 & 26 \\
\end{tabularx}

\begin{paracol}{2}
\selectlanguage{latin}
\lettrine[lines=2]{I}{n} Ægypto natális sanctórum Mártyrum Victoríni, 
 Victóris, Nicéphori, Claudiáni, Dióscori, Serapiónis et Pápiæ, sub Numeriáno 
 Imperatóre. Horum duo primi, pro confessióne fídei, exquisíta 
 suppliciórum génera constánter passi, cápite plectúntur; Nicéphorus, post 
 cratículas candéntes ignésque superátos, minutátim concísus est; Claudiánus 
 et Dióscorus flammis incénsi; Serápion vero et Pápias gládio cæsi sunt.
\switchcolumn
\selectlanguage{english}
\lettrine[lines=2]{I}{n} Egypt, under Emperor Numerian, the birthday of the holy martyrs 
 Victorinus, Victor, Nicephorus, Claudian, Dioscorus, Serapion, and Papias. 
 After patiently enduring extreme tortures, the first two were beheaded for 
 the confession of the faith, Nicephorus was laid on a heated gridiron, 
 placed over the fire, then thoroughly hacked with a knife; Claudian and 
 Dioscorus were burned at the stake; Serapion and Papias were slain with the 
 sword.
\switchcolumn*
\selectlanguage{latin}
In Africa sanctórum Mártyrum Donáti, Justi, Herénæ 
 et Sociórum.
\switchcolumn
\selectlanguage{english}
In Africa, the holy martyrs Donatus, Justus, Herenas, and their companions.
\switchcolumn*
\selectlanguage{latin}
Constantinópoli sancti Tharásii Epíscopi, eruditióne et pietáte insígnis; ad 
 quem exstat Hadriáni Papæ Primi epístola pro 
 defensióne sanctárum Imáginum.
\switchcolumn
\selectlanguage{english}
At Constantinople, St. Tharasius, bishop, a man of great learning and piety. 
 There exists a letter defending sacred images, written to him by Pope 
 Hadrian I.
\switchcolumn*
\selectlanguage{latin}
Naziánzi, in Cappadócia, sancti Cæsárii, qui 
 beátæ Nonnæ fílius ac beatórum Gregórii Theólogi et Gorgóniæ fuit frater, et 
 quem idem Gregórius inter ágmina beatórum se vidísse testátur.
\switchcolumn
\selectlanguage{english}
At Nazianzus, St. Caesarius, who was the son of blessed Nonna, and whom his 
 brother, blessed Gregory the Theologian, says he saw among the hosts of the 
 blessed.
\switchcolumn*
\selectlanguage{latin}
In monastério Heidenhémii, diœcésis 
 Eistetténsis, in Germánia, sanctæ Walbúrgæ Vírginis, quæ fuit fília sancti 
 Richárdi, Anglórum Regis, et soror sancti Willebáldi, Eistetténsis Epíscopi.
\switchcolumn
\selectlanguage{english}
In the monastery of Heidenheim, in the Eichstadt diocese in Germany, St. 
 Walburga, virgin. She was the daughter of St. Richard, king of 
 England, and sister of St. Willebald, bishop of Eichstadt.
\switchcolumn*
\selectlanguage{latin}
\end{paracol}


% ---- martyrology/mart02/mart0227.htm
\needspace{10\baselineskip}
\begin{paracol}{2}
\selectlanguage{latin}
\begin{center}{\color{gregoriocolor} Tértio Kaléndas Mártii. 
 Luna\dots\ }\end{center}
\switchcolumn
\selectlanguage{english}
\begin{center}{\color{gregoriocolor} The Twenty-Seventh Day of February. The\dots\ Day of the Moon.}\end{center}
\end{paracol}

\noindent\begin{tabularx}{\linewidth}{*{19}{>{\centering\arraybackslash}X}}
 \textcolor{gregoriocolor}{a} & \textcolor{gregoriocolor}{b} & \textcolor{gregoriocolor}{c} & \textcolor{gregoriocolor}{d} & \textcolor{gregoriocolor}{e} & \textcolor{gregoriocolor}{f} & \textcolor{gregoriocolor}{g} & \textcolor{gregoriocolor}{h} & \textcolor{gregoriocolor}{i} & \textcolor{gregoriocolor}{k} & \textcolor{gregoriocolor}{l} & \textcolor{gregoriocolor}{m} & \textcolor{gregoriocolor}{n} & \textcolor{gregoriocolor}{p} & \textcolor{gregoriocolor}{q} & \textcolor{gregoriocolor}{r} & \textcolor{gregoriocolor}{s} & \textcolor{gregoriocolor}{t} & \textcolor{gregoriocolor}{u} \\
 29 & 1 & 2 & 3 & 4 & 5 & 6 & 7 & 8 & 9 & 10 & 11 & 12 & 13 & 14 & 15 & 16 & 17 & 18 \\
\end{tabularx}
\vspace{0.5\baselineskip}
\noindent\begin{tabularx}{\linewidth}{*{12}{>{\centering\arraybackslash}X}}
 \textcolor{gregoriocolor}{A} & \textcolor{gregoriocolor}{B} & \textcolor{gregoriocolor}{C} & \textcolor{gregoriocolor}{D} & \textcolor{gregoriocolor}{E} & F & \textcolor{gregoriocolor}{F} & \textcolor{gregoriocolor}{G} & \textcolor{gregoriocolor}{H} & \textcolor{gregoriocolor}{M} & \textcolor{gregoriocolor}{N} & \textcolor{gregoriocolor}{P} \\
 19 & 20 & 21 & 22 & 23 & 24 & 23 & 24 & 25 & 26 & 27 & 28 \\
\end{tabularx}

\begin{paracol}{2}
\selectlanguage{latin}
\lettrine[lines=2]{I}{nsulæ,} in Aprútio, sancti Gabriélis a Vírgine 
 Perdolénte, Clérici Congregatiónis a Cruce et Passióne Dómini nuncupátæ, et 
 Confessóris; qui, magnis intra breve vitæ spátium méritis et post mortem 
 miráculis clarus, a Benedícto Papa Décimo quinto in Sanctórum cánonem 
 relátus est.
\switchcolumn
\selectlanguage{english}
\lettrine[lines=2]{A}{t} Isola, in the province of Abruzzi, St. Gabriel of the Sorrowful Virgin, 
 confessor and cleric of the Passionist Congregation. Having been known 
 for his merits during his short life, and after death renowned for miracles, 
 Pope Benedict XV enrolled him in the canon of the saints.
\switchcolumn*
\selectlanguage{latin}
Romæ natális sanctórum Mártyrum Alexándri, 
 Abúndii, Antígoni et Fortunáti.
\switchcolumn
\selectlanguage{english}
At Rome, the birthday of the holy martyrs, Alexander, Abundius, Antigonus, 
 and Fortunatus.
\switchcolumn*
\selectlanguage{latin}
Alexandríæ pássio sancti Juliáni Mártyris, qui, 
 cum ita pódagra constríctus esset, ut neque incédere neque stare posset, una 
 cum duóbus fámulis, qui eum in sella gestábant, Júdici offértur; quorum 
 alter fidem negávit, alter, nómine Eunus, cum dómino suo perdurávit in 
 confessióne Christi. Ipse porro Juliánus et Eunus, camélis impósiti, 
 per totam urbem circumdúci jubéntur, et flagris laniári, ac tandem, incénso 
 rogo, hinc inde spectánte pópulo, combúri.
\switchcolumn
\selectlanguage{english}
At Alexandria, the passion of St. Julian, martyr. Although he was so 
 afflicted with gout that he could neither walk nor stand, he was taken 
 before the judge with two servants, who carried him in a chair. One of 
 these denied his faith, but the other, named Eunus, persevered with Julian 
 in confessing Christ. Both were set on camels, led through the whole 
 city, scourged, and then burned alive in the presence of all the people.
\switchcolumn*
\selectlanguage{latin}
Ibídem sancti Besæ mílitis, qui, cum 
 insultántes in prædíctos Mártyres cohibéret, delátus est ad Júdicem, et, pro 
 fide constánter agens, cápite truncátus.
\switchcolumn
\selectlanguage{english}
In the same city, St. Besas, a soldier. He had rebuked those who 
 insulted the martyrs just mentioned, and so was denounced before the judge. 
 Because he continued to proclaim his attachment to the faith he was 
 beheaded.
\switchcolumn*
\selectlanguage{latin}
Híspali, in Hispánia, natális sancti Leándri, 
 ejúsdem civitátis Epíscopi, qui, sanctórum Isidóri Epíscopi ac Florentínæ 
 Vírginis frater, sua prædicatióne et indústria gentem Visigothórum, 
 adjuvánte Reccarédo, eórum Rege, ab Ariána impietáte ad cathólicam fidem 
 convértit.
\switchcolumn
\selectlanguage{english}
At Seville in Spain, the birthday of St. Leander, bishop of that city, and 
 of St. Florentina, virgin. By his preaching and zeal the Visigoths, 
 with the help of King Recared, were converted from the Arian heresy to the 
 Catholic faith.
\switchcolumn*
\selectlanguage{latin}
Constantinópoli sanctórum Confessórum Basilíi 
 et Procópii, qui, témpore Leónis Imperatóris, pro cultu sanctárum Imáginum 
 strénue decertárunt.
\switchcolumn
\selectlanguage{english}
At Constantinople, in the time of Emperor Leo, the holy confessors Basil and 
 Procopius, who fought courageously in behalf of the veneration of sacred 
 images.
\switchcolumn*
\selectlanguage{latin}
Lugdúni, in Gállia, sancti Baldoméri Subdiáconi, viri Deo devóti, cujus 
 sepúlcrum crebris miráculis illustrátur.
\switchcolumn
\selectlanguage{english}
At Lyons, St. Baldomer, subdeacon and man of God, whose tomb is graced by 
 many miracles.
\switchcolumn*
\selectlanguage{latin}
\end{paracol}


% ---- martyrology/mart02/mart0227leap.htm
\needspace{10\baselineskip}
\begin{paracol}{2}
\selectlanguage{latin}
\begin{center}{\color{gregoriocolor} Quarto Kaléndas Mártii. 
 Luna\dots\ }\end{center}
\switchcolumn
\selectlanguage{english}
\begin{center}{\color{gregoriocolor} The Twenty-Seventh Day of February. The\dots\ Day of the Moon.}\end{center}
\end{paracol}

\noindent\begin{tabularx}{\linewidth}{*{19}{>{\centering\arraybackslash}X}}
 \textcolor{gregoriocolor}{a} & \textcolor{gregoriocolor}{b} & \textcolor{gregoriocolor}{c} & \textcolor{gregoriocolor}{d} & \textcolor{gregoriocolor}{e} & \textcolor{gregoriocolor}{f} & \textcolor{gregoriocolor}{g} & \textcolor{gregoriocolor}{h} & \textcolor{gregoriocolor}{i} & \textcolor{gregoriocolor}{k} & \textcolor{gregoriocolor}{l} & \textcolor{gregoriocolor}{m} & \textcolor{gregoriocolor}{n} & \textcolor{gregoriocolor}{p} & \textcolor{gregoriocolor}{q} & \textcolor{gregoriocolor}{r} & \textcolor{gregoriocolor}{s} & \textcolor{gregoriocolor}{t} & \textcolor{gregoriocolor}{u} \\
 28 & 29 & 1 & 2 & 3 & 4 & 5 & 6 & 7 & 8 & 9 & 10 & 11 & 12 & 13 & 14 & 15 & 16 & 17 \\
\end{tabularx}
\vspace{0.5\baselineskip}
\noindent\begin{tabularx}{\linewidth}{*{12}{>{\centering\arraybackslash}X}}
 \textcolor{gregoriocolor}{A} & \textcolor{gregoriocolor}{B} & \textcolor{gregoriocolor}{C} & \textcolor{gregoriocolor}{D} & \textcolor{gregoriocolor}{E} & F & \textcolor{gregoriocolor}{F} & \textcolor{gregoriocolor}{G} & \textcolor{gregoriocolor}{H} & \textcolor{gregoriocolor}{M} & \textcolor{gregoriocolor}{N} & \textcolor{gregoriocolor}{P} \\
 18 & 19 & 20 & 21 & 22 & 23 & 22 & 23 & 24 & 25 & 26 & 27 \\
\end{tabularx}

\begin{paracol}{2}
\selectlanguage{latin}
\lettrine[lines=2]{P}{erge,} in Pamphylia, natális beáti Néstoris 
 Epíscopi, qui, in persecutióne Décii, cum diu noctúque oratióni insísteret 
 póstulans ut grex Christi custodirétur, comprehénsus est, ac, nomen Dómini 
 mira liberáte et alacritáte conféssus, Prǽsidis Polliónis jussu equúleo 
 sævíssime est cruciátus; ac demum, cum se Christo semper adhæsúrum 
 constánter profiterétur, crucis suspéndio victor in cælum migrávit.
\switchcolumn
\selectlanguage{english}
\lettrine[lines=2]{A}{t} Pergen in Pamphylia, during the persecution of Decius, the birthday of 
 blessed Nestor, bishop, who praying night and day for the safety of the 
 flock of Christ, was put under arrest. Because he confessed the Name 
 of the Lord with great zeal and freedom, he was cruelly tortured on the rack 
 by order of Pollio the governor. When he still courageously proclaimed 
 that he would remain ever faithful to Christ, he was crucified, and thus 
 triumphantly went to heaven.
\switchcolumn*
\selectlanguage{latin}
Ibídem pássio sanctórum Pápiæ, Diodóri, Conónis 
 et Claudiáni, qui sanctum Néstorem martyrio præcessérunt.
\switchcolumn
\selectlanguage{english}
In the same place, the passion of Saints Papias, Diodorus, Conon, and 
 Claudian, who preceded St. Nestor to martyrdom.
\switchcolumn*
\selectlanguage{latin}
Item sanctórum Mártyrum Fortunáti, Felícis et aliórum vigínti septem.
\switchcolumn
\selectlanguage{english}
Also, the holy martyrs Fortunatus, Felix, and twenty-seven others.
\switchcolumn*
\selectlanguage{latin}
Alexandríæ sancti Alexándri Epíscopi, gloriósi 
 senis, qui, post beátum Petrum, ejúsdem civitátis Episcopum, zelo fídei 
 succénsus, Aríum, Presbyterum suum, hærética impietáte depravátum et divína 
 veritáte convíctum, de Ecclésia ejécit; ac póstea inter trecéntos decem et 
 octo Patres, in Nicǽno Concílio eúndem damnávit.
\switchcolumn
\selectlanguage{english}
At Alexandria, Bishop St. Alexander, an aged man held in great honour, who 
 succeeded blessed Peter as bishop of that city. He expelled Arius, one 
 of his priests, from the Church because he was tainted with heretical 
 ímpiety and convicted in the face of divine truth. Later on he was one 
 of the three hundred and eighteen Fathers who condemned him in the Council 
 of Nicaea.
\switchcolumn*
\selectlanguage{latin}
Bonóniæ sancti Faustiniáni Epíscopi, qui eam 
 Ecclésiam, Diocletiáni persecutióne vexátam, verbo prædicatiónis firmávit et 
 auxit.
\switchcolumn
\selectlanguage{english}
At Bologna, the bishop St. Faustinian. His preaching strengthened and 
 multiplied the faithful of that church when it was so much afflicted during 
 the persecution of Diocletian.
\switchcolumn*
\selectlanguage{latin}
Gazæ, in Palæstína, sancti Porphyrii Epíscopi, 
 qui, témpore Arcádii Imperatóris, Marnam idólum ejúsque templum evértit, ac, 
 multa passus, quiévit in Dómino.
\switchcolumn
\selectlanguage{english}
At Gaza in Palestine, St. Porphyry, bishop, in the time of Emperor Arcadius. 
 He overthrew the idol Marna and its temple, and after many sufferings, went 
 to his rest in the Lord.
\switchcolumn*
\selectlanguage{latin}
Floréntiæ sancti Andréæ, Epíscopi et 
 Confessóris.
\switchcolumn
\selectlanguage{english}
At Florence, St. Andrew, bishop and confessor.
\switchcolumn*
\selectlanguage{latin}
In território Archiacénsi, in Gállia, sancti Victóris Confessóris, cujus 
 laudes sanctus Bernárdus conscrípsit.
\switchcolumn
\selectlanguage{english}
In the province of Champagne in France, St. Victor, confessor, about whom 
 eulogies have been written by St. Bernard.
\switchcolumn*
\selectlanguage{latin}
\end{paracol}


% ---- martyrology/mart02/mart0228.htm
\needspace{10\baselineskip}
\begin{paracol}{2}
\selectlanguage{latin}
\begin{center}{\color{gregoriocolor} Prídie Kaléndas Mártii. 
 Luna\dots\ }\end{center}
\switchcolumn
\selectlanguage{english}
\begin{center}{\color{gregoriocolor} The Twenty-Eighth Day of February. The\dots\ Day of the Moon.}\end{center}
\end{paracol}

\noindent\begin{tabularx}{\linewidth}{*{19}{>{\centering\arraybackslash}X}}
 \textcolor{gregoriocolor}{a} & \textcolor{gregoriocolor}{b} & \textcolor{gregoriocolor}{c} & \textcolor{gregoriocolor}{d} & \textcolor{gregoriocolor}{e} & \textcolor{gregoriocolor}{f} & \textcolor{gregoriocolor}{g} & \textcolor{gregoriocolor}{h} & \textcolor{gregoriocolor}{i} & \textcolor{gregoriocolor}{k} & \textcolor{gregoriocolor}{l} & \textcolor{gregoriocolor}{m} & \textcolor{gregoriocolor}{n} & \textcolor{gregoriocolor}{p} & \textcolor{gregoriocolor}{q} & \textcolor{gregoriocolor}{r} & \textcolor{gregoriocolor}{s} & \textcolor{gregoriocolor}{t} & \textcolor{gregoriocolor}{u} \\
 1 & 2 & 3 & 4 & 5 & 6 & 7 & 8 & 9 & 10 & 11 & 12 & 13 & 14 & 15 & 16 & 17 & 18 & 19 \\
\end{tabularx}
\vspace{0.5\baselineskip}
\noindent\begin{tabularx}{\linewidth}{*{12}{>{\centering\arraybackslash}X}}
 \textcolor{gregoriocolor}{A} & \textcolor{gregoriocolor}{B} & \textcolor{gregoriocolor}{C} & \textcolor{gregoriocolor}{D} & \textcolor{gregoriocolor}{E} & F & \textcolor{gregoriocolor}{F} & \textcolor{gregoriocolor}{G} & \textcolor{gregoriocolor}{H} & \textcolor{gregoriocolor}{M} & \textcolor{gregoriocolor}{N} & \textcolor{gregoriocolor}{P} \\
 20 & 21 & 22 & 23 & 24 & 25 & 24 & 25 & 26 & 27 & 28 & 29 \\
\end{tabularx}

\begin{paracol}{2}
\selectlanguage{latin}
\lettrine[lines=2]{R}{omæ} natális sanctórum Mártyrum Macárii, Rufíni, 
 Justi et Theóphili.
\switchcolumn
\selectlanguage{english}
\lettrine[lines=2]{A}{t} Rome, the birthday of the holy martyrs Macarius, Rufinus, Justus, and 
 Theophilus.
\switchcolumn*
\selectlanguage{latin}
Alexandríæ pássio sanctórum Cæreális, Púpuli, 
 Caji et Serapiónis.
\switchcolumn
\selectlanguage{english}
At Alexandria, the passion of the Saints Caerealis, Pupulus, Caius, and 
 Serapion.
\switchcolumn*
\selectlanguage{latin}
Ibídem commemorátio sanctórum Presbyterórum, Diaconórum et aliórum 
 plurimórum; qui, témpore Valeriáni Imperatóris, cum pestis sævíssima 
 grassarétur, morbo laborántibus ministrántes, libentíssime mortem oppetiére, 
 et quos velut Mártyres religiósa piórum fides venerári consuévit.
\switchcolumn
\selectlanguage{english}
In the same city, in the reign of Emperor Valerian, the commemoration of the 
 holy priests, deacons, and many others. When a most deadly epidemic 
 was raging, they willingly met their death by ministering to the sick. 
 The religious sentiment of the pious faithful has generally venerated them 
 as martyrs.
\switchcolumn*
\selectlanguage{latin}
Romæ sancti Hílari, Papæ et Confessóris.
\switchcolumn
\selectlanguage{english}
At Rome, St. Hilary, pope and confessor.
\switchcolumn*
\selectlanguage{latin}
In território Lugdunénsi, locis Jurénsibus, deposítio sancti Románi Abbátis, 
 qui primum illic eremíticam vitam duxit, et, multis virtútibus ac miráculis 
 clarus, plurimórum póstea Pater éxstitit Monachórum.
\switchcolumn
\selectlanguage{english}
In the territory of Lyons, in the Jura Mountains, the death of St. Romanus, 
 abbot, who first had led the life of a hermit there. His reputation 
 for virtues and miracles brought under his guidance many monks.
\switchcolumn*
\selectlanguage{latin}
Papíæ Translátio córporis sancti Augustíni 
 Epíscopi, Confessóris et Ecclésiæ Doctóris, ex Sardínia ínsula, ópera 
 Luitprándi, Regis Longobardórum.
\switchcolumn
\selectlanguage{english}
At Papia, the transfer, ordered by the Lombard King Luitprand, of the body 
 of St. Augustine, bishop, away from the island of Sardinia.
\switchcolumn*
\selectlanguage{latin}
\end{paracol}


% ---- martyrology/mart02/mart0228leap.htm
\needspace{10\baselineskip}
\begin{paracol}{2}
\selectlanguage{latin}
\begin{center}{\color{gregoriocolor} Tértio Kaléndas Mártii. 
 Luna\dots\ }\end{center}
\switchcolumn
\selectlanguage{english}
\begin{center}{\color{gregoriocolor} The Twenty-Eighth Day of February. The\dots\ Day of the Moon.}\end{center}
\end{paracol}

\noindent\begin{tabularx}{\linewidth}{*{19}{>{\centering\arraybackslash}X}}
 \textcolor{gregoriocolor}{a} & \textcolor{gregoriocolor}{b} & \textcolor{gregoriocolor}{c} & \textcolor{gregoriocolor}{d} & \textcolor{gregoriocolor}{e} & \textcolor{gregoriocolor}{f} & \textcolor{gregoriocolor}{g} & \textcolor{gregoriocolor}{h} & \textcolor{gregoriocolor}{i} & \textcolor{gregoriocolor}{k} & \textcolor{gregoriocolor}{l} & \textcolor{gregoriocolor}{m} & \textcolor{gregoriocolor}{n} & \textcolor{gregoriocolor}{p} & \textcolor{gregoriocolor}{q} & \textcolor{gregoriocolor}{r} & \textcolor{gregoriocolor}{s} & \textcolor{gregoriocolor}{t} & \textcolor{gregoriocolor}{u} \\
 29 & 1 & 2 & 3 & 4 & 5 & 6 & 7 & 8 & 9 & 10 & 11 & 12 & 13 & 14 & 15 & 16 & 17 & 18 \\
\end{tabularx}
\vspace{0.5\baselineskip}
\noindent\begin{tabularx}{\linewidth}{*{12}{>{\centering\arraybackslash}X}}
 \textcolor{gregoriocolor}{A} & \textcolor{gregoriocolor}{B} & \textcolor{gregoriocolor}{C} & \textcolor{gregoriocolor}{D} & \textcolor{gregoriocolor}{E} & F & \textcolor{gregoriocolor}{F} & \textcolor{gregoriocolor}{G} & \textcolor{gregoriocolor}{H} & \textcolor{gregoriocolor}{M} & \textcolor{gregoriocolor}{N} & \textcolor{gregoriocolor}{P} \\
 19 & 20 & 21 & 22 & 23 & 24 & 23 & 24 & 25 & 26 & 27 & 28 \\
\end{tabularx}

\begin{paracol}{2}
\selectlanguage{latin}
\lettrine[lines=2]{I}{nsulæ,} in Aprútio, sancti Gabriélis a Vírgine 
 Perdolénte, Clérici Congregatiónis a Cruce et Passióne Dómini nuncupátæ, et 
 Confessóris; qui, magnis intra breve vitæ spátium méritis et post mortem 
 miráculis clarus, a Benedícto Papa Décimo quinto in Sanctórum cánonem 
 relátus est.
\switchcolumn
\selectlanguage{english}
\lettrine[lines=2]{A}{t} Isola, in the province of Abruzzi, St. Gabriel of the Sorrowful Virgin, 
 confessor and cleric of the Passionist Congregation. Having been known 
 for his merits during his short life, and after death renowned for miracles, 
 Pope Benedict XV enrolled him in the canon of the saints.
\switchcolumn*
\selectlanguage{latin}
Romæ natális sanctórum Mártyrum Alexándri, 
 Abúndii, Antígoni et Fortunáti.
\switchcolumn
\selectlanguage{english}
At Rome, the birthday of the holy martyrs, Alexander, Abundius, Antigonus, 
 and Fortunatus.
\switchcolumn*
\selectlanguage{latin}
Alexandríæ pássio sancti Juliáni Mártyris, qui, 
 cum ita pódagra constríctus esset, ut neque incédere neque stare posset, una 
 cum duóbus fámulis, qui eum in sella gestábant, Júdici offértur; quorum 
 alter fidem negávit, alter, nómine Eunus, cum dómino suo perdurávit in 
 confessióne Christi. Ipse porro Juliánus et Eunus, camélis impósiti, 
 per totam urbem circumdúci jubéntur, et flagris laniári, ac tandem, incénso 
 rogo, hinc inde spectánte pópulo, combúri.
\switchcolumn
\selectlanguage{english}
At Alexandria, the passion of St. Julian, martyr. Although he was so 
 afflicted with gout that he could neither walk nor stand, he was taken 
 before the judge with two servants, who carried him in a chair. One of 
 these denied his faith, but the other, named Eunus, persevered with Julian 
 in confessing Christ. Both were set on camels, led through the whole 
 city, scourged, and then burned alive in the presence of all the people.
\switchcolumn*
\selectlanguage{latin}
Ibídem sancti Besæ mílitis, qui, cum 
 insultántes in prædictos Mártyres cohibéret, delátus est ad Júdicem, et, pro 
 fide constánter agens, cápite truncátus.
\switchcolumn
\selectlanguage{english}
In the same city, St. Besas, a soldier. He had rebuked those who 
 insulted the martyrs just mentioned, and so was denounced before the judge. 
 Because he continued to proclaim his attachment to the faith he was 
 beheaded.
\switchcolumn*
\selectlanguage{latin}
Híspali, in Hispánia, natális sancti Leándri, 
 ejúsdem civitátis Epíscopi, qui, sanctórum Isidóri Epíscopi ac Florentínæ 
 Vírginis frater, sua prædicatióne et indústria gentem Visigothórum, 
 adjuvánte Reccarédo, eórum Rege, ab Ariána impietáte ad cathólicam fidem 
 convértit.
\switchcolumn
\selectlanguage{english}
At Seville in Spain, the birthday of St. Leander, bishop of that city, and 
 of St. Florentina, virgin. By his preaching and zeal the Visigoths, 
 with the help of King Recared, were converted from the Arian heresy to the 
 Catholic faith.
\switchcolumn*
\selectlanguage{latin}
Constantinópoli sanctórum Confessórum Basilíi 
 et Procópii, qui, témpore Leónis Imperatóris, pro cultu sanctárum Imáginum 
 strénue decertárunt.
\switchcolumn
\selectlanguage{english}
At Constantinople, in the time of Emperor Leo, the holy confessors Basil and 
 Procopius, who fought courageously in behalf of the veneration of sacred 
 images.
\switchcolumn*
\selectlanguage{latin}
Lugdúni, in Gállia, sancti Baldoméri Subdiáconi, viri Deo devóti, cujus 
 sepúlcrum crebris miráculis illustrátur.
\switchcolumn
\selectlanguage{english}
At Lyons, St. Baldomer, subdeacon and man of God, whose tomb is graced by 
 many miracles.
\switchcolumn*
\selectlanguage{latin}
\end{paracol}


% ---- martyrology/mart02/mart0229.leap.htm
\needspace{10\baselineskip}
\begin{paracol}{2}
\selectlanguage{latin}
\begin{center}{\color{gregoriocolor} Prídie Kaléndas Mártii. 
 Luna\dots\ }\end{center}
\switchcolumn
\selectlanguage{english}
\begin{center}{\color{gregoriocolor} The Twenty-Ninth Day of February. The\dots\ Day of the Moon.}\end{center}
\end{paracol}

\noindent\begin{tabularx}{\linewidth}{*{19}{>{\centering\arraybackslash}X}}
 \textcolor{gregoriocolor}{a} & \textcolor{gregoriocolor}{b} & \textcolor{gregoriocolor}{c} & \textcolor{gregoriocolor}{d} & \textcolor{gregoriocolor}{e} & \textcolor{gregoriocolor}{f} & \textcolor{gregoriocolor}{g} & \textcolor{gregoriocolor}{h} & \textcolor{gregoriocolor}{i} & \textcolor{gregoriocolor}{k} & \textcolor{gregoriocolor}{l} & \textcolor{gregoriocolor}{m} & \textcolor{gregoriocolor}{n} & \textcolor{gregoriocolor}{p} & \textcolor{gregoriocolor}{q} & \textcolor{gregoriocolor}{r} & \textcolor{gregoriocolor}{s} & \textcolor{gregoriocolor}{t} & \textcolor{gregoriocolor}{u} \\
 1 & 2 & 3 & 4 & 5 & 6 & 7 & 8 & 9 & 10 & 11 & 12 & 13 & 14 & 15 & 16 & 17 & 18 & 19 \\
\end{tabularx}
\vspace{0.5\baselineskip}
\noindent\begin{tabularx}{\linewidth}{*{12}{>{\centering\arraybackslash}X}}
 \textcolor{gregoriocolor}{A} & \textcolor{gregoriocolor}{B} & \textcolor{gregoriocolor}{C} & \textcolor{gregoriocolor}{D} & \textcolor{gregoriocolor}{E} & F & \textcolor{gregoriocolor}{F} & \textcolor{gregoriocolor}{G} & \textcolor{gregoriocolor}{H} & \textcolor{gregoriocolor}{M} & \textcolor{gregoriocolor}{N} & \textcolor{gregoriocolor}{P} \\
 20 & 21 & 22 & 23 & 24 & 25 & 24 & 25 & 26 & 27 & 28 & 29 \\
\end{tabularx}

\begin{paracol}{2}
\selectlanguage{latin}
\lettrine[lines=2]{R}{omæ} natális sanctórum Mártyrum Macárii, Rufíni, 
 Justi et Theóphili.
\switchcolumn
\selectlanguage{english}
\lettrine[lines=2]{A}{t} Rome, the birthday of the holy martyrs Macarius, Rufinus, Justus, and 
 Theophilus.
\switchcolumn*
\selectlanguage{latin}
Alexandríæ pássio sanctórum Cæreális, Púpuli, 
 Caji et Serapiónis.
\switchcolumn
\selectlanguage{english}
At Alexandria, the passion of the Saints Caerealis, Pupulus, Caius, and 
 Serapion.
\switchcolumn*
\selectlanguage{latin}
Ibídem commemorátio sanctórum Presbyterórum, Diaconórum et aliórum 
 plurimórum; qui, témpore Valeriáni Imperatóris, cum pestis sævíssima 
 grassarétur, morbo laborántibus ministrántes, libentíssime mortem oppetiére, 
 et quos velut Mártyres religiósa piórum fides venerári consuévit.
\switchcolumn
\selectlanguage{english}
In the same city, in the reign of Emperor Valerian, the commemoration of the 
 holy priests, deacons, and many others. When a most deadly epidemic 
 was raging, they willingly met their death by ministering to the sick. 
 The religious sentiment of the pious faithful has generally venerated them 
 as martyrs.
\switchcolumn*
\selectlanguage{latin}
Romæ sancti Hílari, Papæ et Confessóris.
\switchcolumn
\selectlanguage{english}
At Rome, St. Hilary, pope and confessor.
\switchcolumn*
\selectlanguage{latin}
In território Lugdunénsi, locis Jurénsibus, deposítio sancti Románi Abbátis, 
 qui primum illic eremíticam vitam duxit, et, multis virtútibus ac miráculis 
 clarus, plurimórum póstea Pater éxstitit Monachórum.
\switchcolumn
\selectlanguage{english}
In the territory of Lyons, in the Jura Mountains, the death of St. Romanus, 
 abbot, who first had led the life of a hermit there. His reputation 
 for virtues and miracles brought under his guidance many monks.
\switchcolumn*
\selectlanguage{latin}
Papíæ Translátio córporis sancti Augustíni 
 Epíscopi, Confessóris et Ecclésiæ Doctóris, ex Sardínia ínsula, ópera 
 Luitprándi, Regis Longobardórum.
\switchcolumn
\selectlanguage{english}
At Papia, the transfer, ordered by the Lombard King Luitprand, of the body 
 of St. Augustine, bishop, away from the island of Sardinia.
\switchcolumn*
\selectlanguage{latin}
\end{paracol}


\setrunningtitles{March}{Martius}

% ---- martyrology/mart03/mart0301.htm
\needspace{10\baselineskip}
\begin{paracol}{2}
\selectlanguage{latin}
\begin{center}{\color{gregoriocolor} Kaléndis Mártii. 
 Luna\dots\ }\end{center}
\switchcolumn
\selectlanguage{english}
\begin{center}{\color{gregoriocolor} The First Day of March. 
 The\dots\ Day of the Moon.}\end{center}
\end{paracol}

\noindent\begin{tabularx}{\linewidth}{*{19}{>{\centering\arraybackslash}X}}
 \textcolor{gregoriocolor}{a} & \textcolor{gregoriocolor}{b} & \textcolor{gregoriocolor}{c} & \textcolor{gregoriocolor}{d} & \textcolor{gregoriocolor}{e} & \textcolor{gregoriocolor}{f} & \textcolor{gregoriocolor}{g} & \textcolor{gregoriocolor}{h} & \textcolor{gregoriocolor}{i} & \textcolor{gregoriocolor}{k} & \textcolor{gregoriocolor}{l} & \textcolor{gregoriocolor}{m} & \textcolor{gregoriocolor}{n} & \textcolor{gregoriocolor}{p} & \textcolor{gregoriocolor}{q} & \textcolor{gregoriocolor}{r} & \textcolor{gregoriocolor}{s} & \textcolor{gregoriocolor}{t} & \textcolor{gregoriocolor}{u} \\
 2 & 3 & 4 & 5 & 6 & 7 & 8 & 9 & 10 & 11 & 12 & 13 & 14 & 15 & 16 & 17 & 18 & 19 & 20 \\
\end{tabularx}
\vspace{0.5\baselineskip}
\noindent\begin{tabularx}{\linewidth}{*{12}{>{\centering\arraybackslash}X}}
 \textcolor{gregoriocolor}{A} & \textcolor{gregoriocolor}{B} & \textcolor{gregoriocolor}{C} & \textcolor{gregoriocolor}{D} & \textcolor{gregoriocolor}{E} & F & \textcolor{gregoriocolor}{F} & \textcolor{gregoriocolor}{G} & \textcolor{gregoriocolor}{H} & \textcolor{gregoriocolor}{M} & \textcolor{gregoriocolor}{N} & \textcolor{gregoriocolor}{P} \\
 21 & 22 & 23 & 24 & 25 & 26 & 25 & 26 & 27 & 28 & 29 & 1 \\
\end{tabularx}

\begin{paracol}{2}
\selectlanguage{latin}
\lettrine[lines=2]{R}{omæ} sanctórum Mártyrum ducentórum sexagínta, 
 quos jussit primo Cláudius, pro Christi nómine damnátos, extra portam 
 Saláriam arénam fódere, deínde in amphitheátro sagíttis mílitum intérfici.
\switchcolumn
\selectlanguage{english}
\lettrine[lines=2]{A}{t} Rome, two hundred and sixty holy martyrs condemned for the name of 
 Christ. Claudius ordered them to dig sand beyond the Salarian Gate, 
 then to have soldiers in the amphitheatre shoot them with arrows.
\switchcolumn*
\selectlanguage{latin}
Item natális sanctórum Mártyrum Leónis, Donáti, Abundántii, Nicéphori 
 et aliórum novem.
\switchcolumn
\selectlanguage{english}
Also, the birthday of the holy martyrs Leo, Donatus, Abundantius, Nicephorus, 
 and nine others.
\switchcolumn*
\selectlanguage{latin}
Massíliæ, in Gállia, sanctórum Mártyrum 
 Hermétis et Hadriáni.
\switchcolumn
\selectlanguage{english}
At Marseilles in France, the holy martyrs Hermes and Adrian.
\switchcolumn*
\selectlanguage{latin}
Heliópoli, apud Líbanum, sanctæ Eudóciæ 
 Mártyris, quæ, in persecutióne Trajáni, a Theódoto Epíscopo baptizáta et ad 
 certámen muníta, ibídem, Vincéntii Præsidis jussu percússa gládio, martyrii 
 corónam accépit.
\switchcolumn
\selectlanguage{english}
At Heliopolis, St. Eudocia, martyr in the persecution of Trajan. She 
 was baptized by Bishop Theodotus, and being fortified for the combat, was 
 put to the sword at the command of Vincent the governor, and thus received 
 the crown of martyrdom.
\switchcolumn*
\selectlanguage{latin}
Eódem die sanctæ Antonínæ Mártyris, quæ in 
 persecutióne Diocletiáni, cum Gentílium deos irrisísset, ídeo, post vários 
 cruciátus, in vase quodam inclúsa, in palúdem urbis Ceæ demérsa est.
\switchcolumn
\selectlanguage{english}
On the same day, St. Antonina, martyr. For deriding the gods of the 
 heathen, in the persecution of Diocletian, she was, after various torments, 
 shut up in a cask and drowned in a marsh near the city of Cea.
\switchcolumn*
\selectlanguage{latin}
Romæ natális sancti Felícis Papæ Tértii, qui 
 sancti Gregórii Magni átavus fuit; qui étiam (ut ipse Gregórius refert), 
 sanctæ Tharsíllæ nepti appárens, illam ad cæléstia regna vocávit.
\switchcolumn
\selectlanguage{english}
At Rome, the birthday of Pope St. Felix III, ancestor of St. Gregory the 
 Great, who relates of him that he appeared to St. Tharsilla, his niece, and 
 called her to the kingdom of heaven.
\switchcolumn*
\selectlanguage{latin}
Apud civitátem Werdam sancti Suitbérti Epíscopi, qui, témpore sancti Sérgii 
 Papæ Primi, apud Frísones, Bátavos et álios 
 Germániæ pópulos Evangélium prædicávit.
\switchcolumn
\selectlanguage{english}
At Kaiserswerdt, Bishop St. Swidbert, who, in the time of Pope Sergius, 
 preached the Gospel among the Frisians, Batavians, and other Germanic 
 peoples.
\switchcolumn*
\selectlanguage{latin}
Andégavi, in Gállia, sancti Albíni, Epíscopi et 
 Confessóris, viri præclaríssimæ virtútis et sanctitátis.
\switchcolumn
\selectlanguage{english}
At Angers in France, St. Albinus, bishop and confessor, a man of most 
 eminent virtue and piety.
\switchcolumn*
\selectlanguage{latin}
Apud Cenómanos, in Gállia, sancti Siviárdi 
 Abbátis.
\switchcolumn
\selectlanguage{english}
At Le Mans in France, St. Siviard, abbot.
\switchcolumn*
\selectlanguage{latin}
Perúsiæ Translátio sancti Herculáni, Epíscopi 
 et Mártyris, qui jussu Tótilæ, Gothórum Regis, decollátus est. Ipsíus 
 autem corpus ita cápiti unítum atque sanum, quadragésimo post abscissiónem 
 die (ut scribit sanctus Gregórius Papa), repértum est, ac si nulla ferri 
 incísio illud tetigísset.
\switchcolumn
\selectlanguage{english}
At Perugia, the transferral of the body of St. Herculanus, bishop and 
 martyr, who was beheaded by order of Totila, king of the Goths. Forty 
 days after the decapitation, Pope St. Gregory relates that the head had been 
 rejoined to the body as if it had never been touched by the sword.
\switchcolumn*
\selectlanguage{latin}
\end{paracol}


% ---- martyrology/mart03/mart0302.htm
\needspace{10\baselineskip}
\begin{paracol}{2}
\selectlanguage{latin}
\begin{center}{\color{gregoriocolor} Sexto Nonas Mártii. 
 Luna\dots\ }\end{center}
\switchcolumn
\selectlanguage{english}
\begin{center}{\color{gregoriocolor} The Second Day of March. 
 The\dots\ Day of the Moon.}\end{center}
\end{paracol}

\noindent\begin{tabularx}{\linewidth}{*{19}{>{\centering\arraybackslash}X}}
 \textcolor{gregoriocolor}{a} & \textcolor{gregoriocolor}{b} & \textcolor{gregoriocolor}{c} & \textcolor{gregoriocolor}{d} & \textcolor{gregoriocolor}{e} & \textcolor{gregoriocolor}{f} & \textcolor{gregoriocolor}{g} & \textcolor{gregoriocolor}{h} & \textcolor{gregoriocolor}{i} & \textcolor{gregoriocolor}{k} & \textcolor{gregoriocolor}{l} & \textcolor{gregoriocolor}{m} & \textcolor{gregoriocolor}{n} & \textcolor{gregoriocolor}{p} & \textcolor{gregoriocolor}{q} & \textcolor{gregoriocolor}{r} & \textcolor{gregoriocolor}{s} & \textcolor{gregoriocolor}{t} & \textcolor{gregoriocolor}{u} \\
 3 & 4 & 5 & 6 & 7 & 8 & 9 & 10 & 11 & 12 & 13 & 14 & 15 & 16 & 17 & 18 & 19 & 20 & 21 \\
\end{tabularx}
\vspace{0.5\baselineskip}
\noindent\begin{tabularx}{\linewidth}{*{12}{>{\centering\arraybackslash}X}}
 \textcolor{gregoriocolor}{A} & \textcolor{gregoriocolor}{B} & \textcolor{gregoriocolor}{C} & \textcolor{gregoriocolor}{D} & \textcolor{gregoriocolor}{E} & F & \textcolor{gregoriocolor}{F} & \textcolor{gregoriocolor}{G} & \textcolor{gregoriocolor}{H} & \textcolor{gregoriocolor}{M} & \textcolor{gregoriocolor}{N} & \textcolor{gregoriocolor}{P} \\
 22 & 23 & 24 & 25 & 26 & 27 & 26 & 27 & 28 & 29 & 1 & 2 \\
\end{tabularx}

\begin{paracol}{2}
\selectlanguage{latin}
\lettrine[lines=2]{R}{omæ,} via Latína, sanctórum Mártyrum Jovíni et 
 Basiléi, qui passi sunt sub Valeriáno et Galliéno Imperatóribus.
\switchcolumn
\selectlanguage{english}
\lettrine[lines=2]{A}{t} Rome, on the Latin Way, the holy martyrs Jovinus and Basileus, who 
 suffered under Emperors Valerian and Gallienus.
\switchcolumn*
\selectlanguage{latin}
Item Romæ plurimórum sanctórum Mártyrum, qui 
 sub Alexándro Imperatóre et Ulpiáno Præfécto, diu cruciáti, ad extrémum 
 capitáli senténtia damnáti sunt.
\switchcolumn
\selectlanguage{english}
Also at Rome, under Emperor Alexander and the prefect Ulpian, many holy 
 martyrs who were a long time tortured and condemned to capital punishment.
\switchcolumn*
\selectlanguage{latin}
Cæsaréæ, in Cappadócia, sanctórum Mártyrum 
 Lúcii Epíscopi, Absalónis et Lórgii.
\switchcolumn
\selectlanguage{english}
At Caesarea, in Cappadocia, the holy martyrs Lucius, bishop, Absalon, and 
 Lorgius.
\switchcolumn*
\selectlanguage{latin}
In Portu Románo sanctórum Mártyrum Pauli, Heráclíi, Secundíllæ 
 et Januáriæ.
\switchcolumn
\selectlanguage{english}
At Porto, near Rome, the holy martyrs Paul, Heraclius, Secundilla, and 
 Januaria.
\switchcolumn*
\selectlanguage{latin}
In Campánia commemorátio sanctórum octogínta Mártyrum, qui, cum nollent 
 carnes immolátas comédere nec caput capræ 
 adoráre, a Longobárdis sævíssime cæsi sunt.
\switchcolumn
\selectlanguage{english}
In Campania, the commemoration of eighty holy martyrs, who were barbarously 
 killed by the Lombards because they would not eat flesh that had been 
 offered to the idols, nor would they adore the head of a goat.
\switchcolumn*
\selectlanguage{latin}
Lichféldiæ, in Anglia, sancti Ceáddæ, Epíscopi 
 Merciórum et Lindisfarnórum, cujus præcláras virtútes sanctus Beda 
 Venerábilis commémorat.
\switchcolumn
\selectlanguage{english}
At Lichfield in England, St. Chad, bishop of Mercia and Lindisfarne, whose 
 excellent virtues are mentioned by St. Venerable Bede.
\switchcolumn*
\selectlanguage{latin}
\end{paracol}


% ---- martyrology/mart03/mart0303.htm
\needspace{10\baselineskip}
\begin{paracol}{2}
\selectlanguage{latin}
\begin{center}{\color{gregoriocolor} Quinto Nonas Mártii. 
 Luna\dots\ }\end{center}
\switchcolumn
\selectlanguage{english}
\begin{center}{\color{gregoriocolor} The Third Day of March. 
 The\dots\ Day of the Moon.}\end{center}
\end{paracol}

\noindent\begin{tabularx}{\linewidth}{*{19}{>{\centering\arraybackslash}X}}
 \textcolor{gregoriocolor}{a} & \textcolor{gregoriocolor}{b} & \textcolor{gregoriocolor}{c} & \textcolor{gregoriocolor}{d} & \textcolor{gregoriocolor}{e} & \textcolor{gregoriocolor}{f} & \textcolor{gregoriocolor}{g} & \textcolor{gregoriocolor}{h} & \textcolor{gregoriocolor}{i} & \textcolor{gregoriocolor}{k} & \textcolor{gregoriocolor}{l} & \textcolor{gregoriocolor}{m} & \textcolor{gregoriocolor}{n} & \textcolor{gregoriocolor}{p} & \textcolor{gregoriocolor}{q} & \textcolor{gregoriocolor}{r} & \textcolor{gregoriocolor}{s} & \textcolor{gregoriocolor}{t} & \textcolor{gregoriocolor}{u} \\
 4 & 5 & 6 & 7 & 8 & 9 & 10 & 11 & 12 & 13 & 14 & 15 & 16 & 17 & 18 & 19 & 20 & 21 & 22 \\
\end{tabularx}
\vspace{0.5\baselineskip}
\noindent\begin{tabularx}{\linewidth}{*{12}{>{\centering\arraybackslash}X}}
 \textcolor{gregoriocolor}{A} & \textcolor{gregoriocolor}{B} & \textcolor{gregoriocolor}{C} & \textcolor{gregoriocolor}{D} & \textcolor{gregoriocolor}{E} & F & \textcolor{gregoriocolor}{F} & \textcolor{gregoriocolor}{G} & \textcolor{gregoriocolor}{H} & \textcolor{gregoriocolor}{M} & \textcolor{gregoriocolor}{N} & \textcolor{gregoriocolor}{P} \\
 23 & 24 & 25 & 26 & 27 & 28 & 27 & 28 & 29 & 1 & 2 & 3 \\
\end{tabularx}

\begin{paracol}{2}
\selectlanguage{latin}
\lettrine[lines=2]{C}{æsaréæ,} in Palæstína, sanctórum Mártyrum 
 Maríni mílitis, et Astérii Senatóris, in persecutióne Valeriáni. Horum 
 prior, cum accusátus esset a commilitiónibus ut Christiánus, et, 
 interrogátus a Júdice, se Christiánum esse voce claríssima testarétur, 
 martyrii corónam abscissióne cápitis accépit; cumque Astérius corpus 
 Mártyris, cápite truncátum, subjéctis húmeris et substráta veste, qua 
 induebátur, excíperet, honórem quem Martyrii détulit, contínuo et ipse 
 Martyr accépit.
\switchcolumn
\selectlanguage{english}
\lettrine[lines=2]{A}{t} Caesarea in Palestine, during the persecution of Valerian, the holy 
 martyrs Marinus, soldier, and Asterius, senator. The former was 
 examined by the judge on the charge laid against him by his fellow soldiers 
 of being a Christian, and as he admitted the accusation in a firm tone of 
 voice, he was beheaded, and thus received the crown of martyrdom. His 
 mutilated body was taken by Asterius on his own shoulders, and wrapped in 
 the garment which he himself wore. This service at once gained for 
 Asterius the palm of martyrdom as a reward for the honour which he had given 
 to a martyr.
\switchcolumn*
\selectlanguage{latin}
Calagúrri, in Hispánia, natális sanctórum Mártyrum Hemitérii et Cheledónii 
 fratrum, qui, cum apud Legiónem, Gallǽciæ 
 urbem, in castris militárent, ambo, exsurgénte persecutiónis procélla, pro 
 confessióne nóminis Christi, Calagúrrim usque profécti, ibi, plúribus 
 torméntis afflícti, martyrio coronáti sunt.
\switchcolumn
\selectlanguage{english}
At Calahorra in Spain, the birthday of the holy martyrs Hermiterius and 
 Cheledonius, soldiers in the army at Leon, a city of Galicia. Upon the 
 approach of persecution they went to Calahorra in order to confess the name 
 of Christ, and after enduring many torments there, they were crowned with 
 martyrdom.
\switchcolumn*
\selectlanguage{latin}
Eódem die pássio sanctórum Felícis, Lucióli, Fortunáti, Márciæ 
 et Sociórum.
\switchcolumn
\selectlanguage{english}
The same day, the passion of the Saints Felix, Luciolus, Fortunatus, Marcia, 
 and their companions.
\switchcolumn*
\selectlanguage{latin}
Item sanctórum mílitum Cleoníci, Eutrópii et 
 Basilísci, qui, in persecutióne Maximiáni, sub Asclepíade Præside, crucis 
 supplício felíciter triumphárunt.
\switchcolumn
\selectlanguage{english}
Also, the sainted soldiers Cleonicus, Eutropius, and Basiliscus, who 
 gloriously triumphed by death on the cross under the governor Asclepias 
 during the persecution of Maximian.
\switchcolumn*
\selectlanguage{latin}
Bríxiæ sancti Titiáni, Epíscopi et Confessóris.
\switchcolumn
\selectlanguage{english}
At Brescia, St. Titian, bishop and confessor.
\switchcolumn*
\selectlanguage{latin}
Bambérgæ sanctæ Cunegúndis Augústæ, quæ, sancto 
 Henríco Primo, Romanórum Imperatóri, nupta, perpétuam virginitátem, ipso 
 annuénte, servávit; ac, bonórum óperum méritis cumuláta, sancto fine quiévit, 
 et post óbitum miráculis cláruit.
\switchcolumn
\selectlanguage{english}
At Bamberg, Empress St. Cunegunda, who preserved her virginity with the 
 consent of her husband, Emperor Henry I. She completed a life rich in 
 meritorious good works with a holy death, and afterward worked many 
 miracles.
\switchcolumn*
\selectlanguage{latin}
\end{paracol}


% ---- martyrology/mart03/mart0304.htm
\needspace{10\baselineskip}
\begin{paracol}{2}
\selectlanguage{latin}
\begin{center}{\color{gregoriocolor} Quarto Nonas Mártii. 
 Luna\dots\ }\end{center}
\switchcolumn
\selectlanguage{english}
\begin{center}{\color{gregoriocolor} The Fourth Day of March. 
 The\dots\ Day of the Moon.}\end{center}
\end{paracol}

\noindent\begin{tabularx}{\linewidth}{*{19}{>{\centering\arraybackslash}X}}
 \textcolor{gregoriocolor}{a} & \textcolor{gregoriocolor}{b} & \textcolor{gregoriocolor}{c} & \textcolor{gregoriocolor}{d} & \textcolor{gregoriocolor}{e} & \textcolor{gregoriocolor}{f} & \textcolor{gregoriocolor}{g} & \textcolor{gregoriocolor}{h} & \textcolor{gregoriocolor}{i} & \textcolor{gregoriocolor}{k} & \textcolor{gregoriocolor}{l} & \textcolor{gregoriocolor}{m} & \textcolor{gregoriocolor}{n} & \textcolor{gregoriocolor}{p} & \textcolor{gregoriocolor}{q} & \textcolor{gregoriocolor}{r} & \textcolor{gregoriocolor}{s} & \textcolor{gregoriocolor}{t} & \textcolor{gregoriocolor}{u} \\
 5 & 6 & 7 & 8 & 9 & 10 & 11 & 12 & 13 & 14 & 15 & 16 & 17 & 18 & 19 & 20 & 21 & 22 & 23 \\
\end{tabularx}
\vspace{0.5\baselineskip}
\noindent\begin{tabularx}{\linewidth}{*{12}{>{\centering\arraybackslash}X}}
 \textcolor{gregoriocolor}{A} & \textcolor{gregoriocolor}{B} & \textcolor{gregoriocolor}{C} & \textcolor{gregoriocolor}{D} & \textcolor{gregoriocolor}{E} & F & \textcolor{gregoriocolor}{F} & \textcolor{gregoriocolor}{G} & \textcolor{gregoriocolor}{H} & \textcolor{gregoriocolor}{M} & \textcolor{gregoriocolor}{N} & \textcolor{gregoriocolor}{P} \\
 24 & 25 & 26 & 27 & 28 & 29 & 28 & 29 & 1 & 2 & 3 & 4 \\
\end{tabularx}

\begin{paracol}{2}
\selectlanguage{latin}
\lettrine[lines=2]{V}{ilnæ,} in Lithuánia, beáti Casimíri Confessóris, 
 e Casimíro Rege progéniti; quem Leo Décimus, Románus Póntifex, in Sanctórum 
 númerum rétulit.
\switchcolumn
\selectlanguage{english}
\lettrine[lines=2]{A}{t} Vilnius in Lithuania, blessed Casimir, confessor, the son of King Casimir, 
 whom Pope Leo X inscribed in the roll of the saints.
\switchcolumn*
\selectlanguage{latin}
Romæ, via Appia, natális sancti Lúcii Primi, 
 Papæ et Mártyris; qui, primo in persecutióne Valeriáni ob Christi fidem 
 exsílio relegátus, et póstmodum divíno nutu ad Ecclésiam suam redíre 
 permíssus, tandem, cum plúrimum advérsus Novatiános laborásset, cápitis 
 obtruncatióne martyrium complévit. Eum vero sanctus Cypriánus summis 
 láudibus celebrávit.
\switchcolumn
\selectlanguage{english}
At Rome, on the Appian Way, during the persecution of Valerian, the birthday 
 of St. Lucius, pope and martyr, who was first exiled for the faith of 
 Christ, but being permitted by divine Providence to return to his church, 
 after labouring long against the Novatians, he suffered martyrdom by 
 beheading. His praises have been published by St. Cyprian.
\switchcolumn*
\selectlanguage{latin}
Nicomedíæ sancti Hadriáni Mártyris, cum áliis 
 vigínti tribus, qui omnes, sub Diocletiáno Imperatóre, martyrium crurifrágio 
 consummárunt. Eórum relíquiæ, a Christiánis Byzántium delátæ, 
 reverénti honóre sepúltæ fuérunt; inde póstea sancti Hadriáni corpus Romam 
 translátum fuit sexto Idus Septémbris, quo die festívitas ejus potíssimum 
 celebrátur.
\switchcolumn
\selectlanguage{english}
At Nicomedia, in the reign of Emperor Diocletian, the martyr St. Adrian and 
 twenty-three others, who endured martyrdom by having their limbs crushed. 
 Their remains were taken to Byzantium by the Christians, and buried there 
 with reverence and honour. Afterwards the body of St. Adrian was 
 transferred to Rome on the 8th of September, on which day his feast is 
 observed.
\switchcolumn*
\selectlanguage{latin}
Romæ, via Appia, sanctórum Mártyrum nongentórum, 
 qui pósiti sunt in cœmetério ad sanctam Cæcíliam.
\switchcolumn
\selectlanguage{english}
At Rome, on the Appian Way, nine hundred holy martyrs who were buried in the 
 cemetery of St. Cecilia.
\switchcolumn*
\selectlanguage{latin}
Apud Chersonésum pássio sanctórum Episcopórum Basilíi, 
 Eugénii, Agathodóri, Elpídii, Æthérii, Capitónis, Ephræm, Néstoris et 
 Arcádii.
\switchcolumn
\selectlanguage{english}
In Chersonesus, the passion of the saintly bishops, Basil, Eugene, 
 Agathodorus, Elpidius, Aetherius, Capito, Ephrem, Nestor, and Arcadius.
\switchcolumn*
\selectlanguage{latin}
Eódem die sancti Caji Palatíni, in mare demérsi, et aliórum vigínti septem.
\switchcolumn
\selectlanguage{english}
On the same day, St. Caius Palatinus and twenty-seven others who were cast 
 into the sea.
\switchcolumn*
\selectlanguage{latin}
Item pássio sanctórum Archelái, Cyrílli et Phótii.
\switchcolumn
\selectlanguage{english}
Also, the passion of Saints Archelaus, Cyril and Photius.
\switchcolumn*
\selectlanguage{latin}
\end{paracol}


% ---- martyrology/mart03/mart0305.htm
\needspace{10\baselineskip}
\begin{paracol}{2}
\selectlanguage{latin}
\begin{center}{\color{gregoriocolor} Tértio Nonas Mártii. 
 Luna\dots\ }\end{center}
\switchcolumn
\selectlanguage{english}
\begin{center}{\color{gregoriocolor} The Fifth Day of March. 
 The\dots\ Day of the Moon.}\end{center}
\end{paracol}

\noindent\begin{tabularx}{\linewidth}{*{19}{>{\centering\arraybackslash}X}}
 \textcolor{gregoriocolor}{a} & \textcolor{gregoriocolor}{b} & \textcolor{gregoriocolor}{c} & \textcolor{gregoriocolor}{d} & \textcolor{gregoriocolor}{e} & \textcolor{gregoriocolor}{f} & \textcolor{gregoriocolor}{g} & \textcolor{gregoriocolor}{h} & \textcolor{gregoriocolor}{i} & \textcolor{gregoriocolor}{k} & \textcolor{gregoriocolor}{l} & \textcolor{gregoriocolor}{m} & \textcolor{gregoriocolor}{n} & \textcolor{gregoriocolor}{p} & \textcolor{gregoriocolor}{q} & \textcolor{gregoriocolor}{r} & \textcolor{gregoriocolor}{s} & \textcolor{gregoriocolor}{t} & \textcolor{gregoriocolor}{u} \\
 6 & 7 & 8 & 9 & 10 & 11 & 12 & 13 & 14 & 15 & 16 & 17 & 18 & 19 & 20 & 21 & 22 & 23 & 24 \\
\end{tabularx}
\vspace{0.5\baselineskip}
\noindent\begin{tabularx}{\linewidth}{*{12}{>{\centering\arraybackslash}X}}
 \textcolor{gregoriocolor}{A} & \textcolor{gregoriocolor}{B} & \textcolor{gregoriocolor}{C} & \textcolor{gregoriocolor}{D} & \textcolor{gregoriocolor}{E} & F & \textcolor{gregoriocolor}{F} & \textcolor{gregoriocolor}{G} & \textcolor{gregoriocolor}{H} & \textcolor{gregoriocolor}{M} & \textcolor{gregoriocolor}{N} & \textcolor{gregoriocolor}{P} \\
 25 & 26 & 27 & 28 & 29 & 30 & 29 & 1 & 2 & 3 & 4 & 5 \\
\end{tabularx}

\begin{paracol}{2}
\selectlanguage{latin}
\lettrine[lines=2]{A}{ntiochíæ} natális sancti Phocæ Mártyris, qui, 
 post multas, quas pro nómine Redemptoris passus est, injúrias, quáliter de 
 antíquo illo serpénte triumpháverit, hódie quoque pópulis eo miráculo 
 declarátur, quod, si quíspiam a serpénte morsus fúerit, hic, ut jánuam 
 Basílicæ Mártyris credens attígerit, conféstim, evacuáta venéni virtúte, 
 sanátur.
\switchcolumn
\selectlanguage{english}
\lettrine[lines=2]{A}{t} Antioch, the birthday of the martyr St. Phocas, who triumphed over the 
 ageless Serpent after many injuries which he suffered for the Name of the 
 Redeemer. That triumph is still manifested to the people in our day, 
 for if any one stung by a snake touches with faith the door of the martyr's 
 basilica, the power of the venom disappears, and he is immediately cured.
\switchcolumn*
\selectlanguage{latin}
Cæsaréæ, in Palæstína, sancti Hadriáni Mártyris, 
 qui in persecutióne Diocletiáni Imperatóris, jussu Firmiliáni Præsidis, 
 prius ob Christi fidem leóni objéctus, deínde gládio jugulátus, martyrii 
 corónam accépit.
\switchcolumn
\selectlanguage{english}
At Caesarea in Palestine, in the persecution of Diocletian, the martyr St. 
 Adrian. He was first exposed to a lion for the faith of Christ, and 
 then slain with the sword by order of the governor Firmilian, and thus 
 received the crown of martyrdom.
\switchcolumn*
\selectlanguage{latin}
Eódem die pássio sanctórum Eusébii Palatini, et aliórum novem Mártyrum.
\switchcolumn
\selectlanguage{english}
The same day, the passion of the holy martyrs Eusebius Palatinus and nine 
 others.
\switchcolumn*
\selectlanguage{latin}
Cæsearéæ, in Palæstína, sancti Theóphili 
 Epíscopi, qui sub Sevéro Príncipe, sapiéntia et vitæ integritáte conspícuus, 
 emícuit.
\switchcolumn
\selectlanguage{english}
At Caesarea in Palestine, in the time of Emperor Severus, St. Theophilus, 
 bishop, who was conspicuous for his wisdom and the purity of his life.
\switchcolumn*
\selectlanguage{latin}
Ad ripam Jordánis, item in Palæstína, sancti 
 Gerásimi, Anachorétæ et Abbátis; qui témpore Zenónis Imperatóris flóruit.
\switchcolumn
\selectlanguage{english}
Also in Palestine, on the banks of the Jordan, the anchoret St. Gerasimus, 
 who lived in the time of Emperor Zeno.
\switchcolumn*
\selectlanguage{latin}
Neápoli, in Campánia, deposítio sancti Joánnis-Joséphi a Cruce, Sacerdótis 
 ex Ordine Minórum et Confessóris, qui, sanctórum Ordini Seráphico insígne 
 decus áddidit, atque a Gregório Papa Décimo sexto in Sanctórum cánonem 
 est relátus.
\switchcolumn
\selectlanguage{english}
At Naples, in Campania, the death of St. John Joseph of the Cross, priest of 
 the Order of Friars Minor, and confessor. By emulating the virtues of 
 St. Francis of Assisi and of St. Peter Alcantara, he added great glory to 
 the Seraphic Order. He was canonized by Pope Gregory XVI.
\switchcolumn*
\selectlanguage{latin}
\end{paracol}


% ---- martyrology/mart03/mart0306.htm
\needspace{10\baselineskip}
\begin{paracol}{2}
\selectlanguage{latin}
\begin{center}{\color{gregoriocolor} Prídie Nonas Mártii. 
 Luna\dots\ }\end{center}
\switchcolumn
\selectlanguage{english}
\begin{center}{\color{gregoriocolor} The Sixth Day of March. 
 The\dots\ Day of the Moon.}\end{center}
\end{paracol}

\noindent\begin{tabularx}{\linewidth}{*{19}{>{\centering\arraybackslash}X}}
 \textcolor{gregoriocolor}{a} & \textcolor{gregoriocolor}{b} & \textcolor{gregoriocolor}{c} & \textcolor{gregoriocolor}{d} & \textcolor{gregoriocolor}{e} & \textcolor{gregoriocolor}{f} & \textcolor{gregoriocolor}{g} & \textcolor{gregoriocolor}{h} & \textcolor{gregoriocolor}{i} & \textcolor{gregoriocolor}{k} & \textcolor{gregoriocolor}{l} & \textcolor{gregoriocolor}{m} & \textcolor{gregoriocolor}{n} & \textcolor{gregoriocolor}{p} & \textcolor{gregoriocolor}{q} & \textcolor{gregoriocolor}{r} & \textcolor{gregoriocolor}{s} & \textcolor{gregoriocolor}{t} & \textcolor{gregoriocolor}{u} \\
 7 & 8 & 9 & 10 & 11 & 12 & 13 & 14 & 15 & 16 & 17 & 18 & 19 & 20 & 21 & 22 & 23 & 24 & 25 \\
\end{tabularx}
\vspace{0.5\baselineskip}
\noindent\begin{tabularx}{\linewidth}{*{12}{>{\centering\arraybackslash}X}}
 \textcolor{gregoriocolor}{A} & \textcolor{gregoriocolor}{B} & \textcolor{gregoriocolor}{C} & \textcolor{gregoriocolor}{D} & \textcolor{gregoriocolor}{E} & F & \textcolor{gregoriocolor}{F} & \textcolor{gregoriocolor}{G} & \textcolor{gregoriocolor}{H} & \textcolor{gregoriocolor}{M} & \textcolor{gregoriocolor}{N} & \textcolor{gregoriocolor}{P} \\
 26 & 27 & 28 & 29 & 30 & 1 & 1 & 2 & 3 & 4 & 5 & 6 \\
\end{tabularx}

\begin{paracol}{2}
\selectlanguage{latin}
\lettrine[lines=2]{S}{anctárum} Perpétuæ et Felicitátis Mártyrum, quæ 
 sequénti die gloriósam martyrii corónam a Dómino recepérunt.
\switchcolumn
\selectlanguage{english}
\lettrine[lines=2]{S}{aints} Perpetua and Felicity, who, on the day following this, received from 
 the Lord the glorious crown of martyrdom.
\switchcolumn*
\selectlanguage{latin}
Dertónæ sancti Marciáni, Epíscopi et Mártyris, 
 qui sub Trajáno, pro Christi glória occísus, coronátur.
\switchcolumn
\selectlanguage{english}
At Tortona, St. Marcian, bishop and martyr, who was put to death for the 
 sake of Christ by Trajan, and thereby received the crown of immortality.
\switchcolumn*
\selectlanguage{latin}
Nicomedíæ natális sanctórum Mártyrum Victóris 
 et Victoríni, qui, per triénnium, cum Claudiáno et uxóre ejus Bassa, 
 torméntis multis afflícti et retrúsi in cárcerem, ibídem vitæ suæ cursum 
 implevérunt.
\switchcolumn
\selectlanguage{english}
At Nicomedia, the birthday of the holy martyrs Victor and Victorinus, who 
 were, with Claudian and his wife Bassa, subjected to many torments for a 
 period of three years, after which they were cast into prison, where they 
 ended their pilgrimage of life.
\switchcolumn*
\selectlanguage{latin}
In Cypro sancti Conónis Mártyris, qui sub Décio Imperatóre, clavis confíxus 
 pedes et ante currum jussus cúrrere, in génua procúbuit, atque in oratióne 
 réddidit spíritum.
\switchcolumn
\selectlanguage{english}
In Cyprus, in the time of Emperor Decius, St. Conon, martyr. He was 
 compelled to run before a chariot, with his feet pierced with nails, and 
 falling to his knees, he died in prayer.
\switchcolumn*
\selectlanguage{latin}
In Syria pássio sanctórum quadragínta duórum Mártyrum, qui, in Amório 
 comprehénsi et illuc perdúcti, ibi, egrégio perácto certámine, victóres 
 palmam martyrii percepérunt.
\switchcolumn
\selectlanguage{english}
In Syria, the passion of forty-two holy martyrs, who were arrested in 
 Amorium and taken to Syria, where they valiantly endured the test and 
 received the crown of martyrdom.
\switchcolumn*
\selectlanguage{latin}
Constantinópoli sancti Evágrii, qui, témpore Valéntis a Cathólicis eléctus 
 Epíscopus, et ab eódem Imperatóre in exsílium missus, Conféssor migrávit ad 
 Dóminum.v
\switchcolumn
\selectlanguage{english}
At Constantinople, St. Evagrius, who was elected Catholic bishop in the 
 reign of Valens, and being exiled by that emperor, later departed for 
 heaven.
\switchcolumn*
\selectlanguage{latin}
Bonóniæ sancti Basilíi Epíscopi, qui, a sancto 
 Silvéstro Papa ordinátus, verbo et exémplo créditam sibi Ecclésiam 
 sanctíssime gubernávit.
\switchcolumn
\selectlanguage{english}
At Bologna, St. Basil, bishop, who was ordained by Pope St. Sylvester, and 
 who governed the church entrusted to his care with great holiness, both by 
 word and example.
\switchcolumn*
\selectlanguage{latin}
Barcinóne, in Hispánia, beáti Ollegárii, primum Canónici, et póstea Epíscopi 
 Barcinonénsis, et Archiepíscopi Tarraconénsis.
\switchcolumn
\selectlanguage{english}
At Barcelona in Spain, blessed Ollegar, who was first a canon and afterwards 
 bishop of Barcelona and archbishop of Tarragona.
\switchcolumn*
\selectlanguage{latin}
Vitérbii beátæ Rosæ Vírginis, ex tértio Ordine 
 sancti Francísci.
\switchcolumn
\selectlanguage{english}
At Viterbo, blessed Rose, a virgin of the Third Order of St. Francis.
\switchcolumn*
\selectlanguage{latin}
Apud Gandávum, in Flándria, sanctæ Colétæ 
 Vírginis, quæ, primum tértii Ordinis Franciscális régulam proféssa, deínde, 
 divíno Spíritu affláta, quamplúra Moniálium secúndi ejúsdem Ordinis 
 monastéria primævæ restítuit disciplínæ; atque, divínis exornáta virtútibus 
 et innúmeris clara miráculis, a Pio Séptimo, Pontífice Máximo, in albo 
 Sanctórum adscrípta est.
\switchcolumn
\selectlanguage{english}
At Ghent in Flanders, St. Colette, virgin, who at first professed the rule 
 of the Third Order of St. Francis, and afterwards, by the inspiration of the 
 Holy Spirit, restored the pristine discipline to a great number of 
 monasteries of Nuns of the Second Order. Because she was graced 
 with heavenly virtues, and performed innumerable miracles, she was inscribed 
 on the roll of saints by Pope Pius VII.
\switchcolumn*
\selectlanguage{latin}
\end{paracol}


% ---- martyrology/mart03/mart0307.htm
\needspace{10\baselineskip}
\begin{paracol}{2}
\selectlanguage{latin}
\begin{center}{\color{gregoriocolor} Nonis Mártii. 
 Luna\dots\ }\end{center}
\switchcolumn
\selectlanguage{english}
\begin{center}{\color{gregoriocolor} The Seventh Day of March. 
 The\dots\ Day of the Moon.}\end{center}
\end{paracol}

\noindent\begin{tabularx}{\linewidth}{*{19}{>{\centering\arraybackslash}X}}
 \textcolor{gregoriocolor}{a} & \textcolor{gregoriocolor}{b} & \textcolor{gregoriocolor}{c} & \textcolor{gregoriocolor}{d} & \textcolor{gregoriocolor}{e} & \textcolor{gregoriocolor}{f} & \textcolor{gregoriocolor}{g} & \textcolor{gregoriocolor}{h} & \textcolor{gregoriocolor}{i} & \textcolor{gregoriocolor}{k} & \textcolor{gregoriocolor}{l} & \textcolor{gregoriocolor}{m} & \textcolor{gregoriocolor}{n} & \textcolor{gregoriocolor}{p} & \textcolor{gregoriocolor}{q} & \textcolor{gregoriocolor}{r} & \textcolor{gregoriocolor}{s} & \textcolor{gregoriocolor}{t} & \textcolor{gregoriocolor}{u} \\
 8 & 9 & 10 & 11 & 12 & 13 & 14 & 15 & 16 & 17 & 18 & 19 & 20 & 21 & 22 & 23 & 24 & 25 & 26 \\
\end{tabularx}
\vspace{0.5\baselineskip}
\noindent\begin{tabularx}{\linewidth}{*{12}{>{\centering\arraybackslash}X}}
 \textcolor{gregoriocolor}{A} & \textcolor{gregoriocolor}{B} & \textcolor{gregoriocolor}{C} & \textcolor{gregoriocolor}{D} & \textcolor{gregoriocolor}{E} & F & \textcolor{gregoriocolor}{F} & \textcolor{gregoriocolor}{G} & \textcolor{gregoriocolor}{H} & \textcolor{gregoriocolor}{M} & \textcolor{gregoriocolor}{N} & \textcolor{gregoriocolor}{P} \\
 27 & 28 & 29 & 30 & 1 & 2 & 2 & 3 & 4 & 5 & 6 & 7 \\
\end{tabularx}

\begin{paracol}{2}
\selectlanguage{latin}
\lettrine[lines=2]{I}{n} monastério Fossæ Novæ, prope Tarracínam, in 
 Campánia, sancti Thomæ Aquinátis, Confessóris et Ecclésiæ Doctóris, ex 
 Ordine Prædicatórum, nobilitáte géneris, vitæ sanctitáte et Theologíæ 
 sciéntia illustríssimi; quem Leo Papa Décimus tértius cæléstem Scholárum 
 ómnium catholicárum Patrónum declarávit.
\switchcolumn
\selectlanguage{english}
\lettrine[lines=2]{I}{n} the monastery of Fossanova, near Terracina in Campania, St. Thomas 
 Aquinas, confessor and doctor of the Church, a member of the Order of 
 Preachers, famous for his noble family, for the sanctity of his life, and 
 for his knowledge of theology. Pope Leo XIII declared him the heavenly 
 patron of all Catholic schools.
\switchcolumn*
\selectlanguage{latin}
Carthágine natális sanctárum Perpétuæ et 
 Felicitátis Mártyrum; e quibus Felícitas, cum esset prægnans (ut sanctus 
 Augustínus ait), juxta leges exspectáta ut páreret, dum parturiébat, dolébat, 
 objécta feris gaudébat. Passi quoque sunt cum eis Sátyrus, Saturnínus, 
 Revocátus et Secúndulus; quorum últimus quiévit in cárcere, réliqui omnes a 
 váriis béstiis sunt vexáti, ac demum gladiórum íctibus confécti, sub Sevéro 
 Príncipe, Sanctárum vero Perpétuæ et Felicitátis festum prídie hujus diéi 
 recólitur.
\switchcolumn
\selectlanguage{english}
At Carthage, the birthday of Saints Perpetua and Felicity, martyrs. 
 St. Augustine relates that Felicity being with child, her execution was 
 deferred, according to the law, until after her delivery. He states 
 that while she was in labour, she mourned, and when cast to the beasts, she 
 rejoiced. With them suffered Satyrus, Saturninus, Revocatus, and 
 Secundulus, the last of whom died in prison; the others were delivered to 
 the beasts, all during the reign of Severus. The feast of Saints 
 Perpetua and Felicity was celebrated yesterday.
\switchcolumn*
\selectlanguage{latin}
Cæsaréæ, in Palæstína, pássio sancti Eubúli, 
 qui fuit sócius sancti Hadriáni, atque, bíduo post illum, laniátus a 
 leónibus et gládio trucidátus, martyrii corónam, últimus ómnium in ea 
 civitáte, accépit.
\switchcolumn
\selectlanguage{english}
At Caesarea in Palestine, the passion of St. Eubulus, who was a companion of 
 St. Adrian. Two days after the latter's death, he was mangled by the 
 lions and put to death by the sword. He was the last of all those who 
 received the crown of martyrdom in that city.
\switchcolumn*
\selectlanguage{latin}
Nicomedíæ sancti Theóphili Epíscopi, qui, ob 
 cultum sanctárum Imáginum in exsílium pulsus, illic defúnctus est.
\switchcolumn
\selectlanguage{english}
At Nicomedia, St. Theophilus, bishop, who was driven into exile for the 
 veneration of sacred images, and died there.
\switchcolumn*
\selectlanguage{latin}
Pelúsii, in Ægypto, sancti Pauli Epíscopi, qui, 
 eándem ob causam, exsul occúbuit.
\switchcolumn
\selectlanguage{english}
At Pelusium in Egypt, St. Paul, bishop, who died in exile for the same 
 cause.
\switchcolumn*
\selectlanguage{latin}
Bríxiæ sancti Gaudiósi, Epíscopi et Confessóris.
\switchcolumn
\selectlanguage{english}
At Brescia, St. Gaudiosus, bishop and confessor.
\switchcolumn*
\selectlanguage{latin}
In Thebáide sancti Pauli, cognoménto Símplicis.
\switchcolumn
\selectlanguage{english}
In Thebais, St. Paul, surnamed the Simple.
\switchcolumn*
\selectlanguage{latin}
Floréntiæ, in Etrúria, sanctæ Terésiæ Margarítæ 
 Redi, Vírginis, Ordinis Carmelitárum Excalceatárum, vitæ puritáte ac 
 simplicitáte admirábilis, quam Pius Papa Undécimus sanctárum Vírginum albo 
 adscrípsit.
\switchcolumn
\selectlanguage{english}
At Florence in Etruria, St. Teresa Margaret Redi, virgin, a member of the 
 Order of Discalced Carmelites, of such admirable purity and simplicity that 
 Pope Pius XI solemnly enrolled her on the scroll of holy virgins.
\switchcolumn*
\selectlanguage{latin}
\end{paracol}


% ---- martyrology/mart03/mart0308.htm
\needspace{10\baselineskip}
\begin{paracol}{2}
\selectlanguage{latin}
\begin{center}{\color{gregoriocolor} Octávo Idus Mártii. 
 Luna\dots\ }\end{center}
\switchcolumn
\selectlanguage{english}
\begin{center}{\color{gregoriocolor} The Eighth Day of March. 
 The\dots\ Day of the Moon.}\end{center}
\end{paracol}

\noindent\begin{tabularx}{\linewidth}{*{19}{>{\centering\arraybackslash}X}}
 \textcolor{gregoriocolor}{a} & \textcolor{gregoriocolor}{b} & \textcolor{gregoriocolor}{c} & \textcolor{gregoriocolor}{d} & \textcolor{gregoriocolor}{e} & \textcolor{gregoriocolor}{f} & \textcolor{gregoriocolor}{g} & \textcolor{gregoriocolor}{h} & \textcolor{gregoriocolor}{i} & \textcolor{gregoriocolor}{k} & \textcolor{gregoriocolor}{l} & \textcolor{gregoriocolor}{m} & \textcolor{gregoriocolor}{n} & \textcolor{gregoriocolor}{p} & \textcolor{gregoriocolor}{q} & \textcolor{gregoriocolor}{r} & \textcolor{gregoriocolor}{s} & \textcolor{gregoriocolor}{t} & \textcolor{gregoriocolor}{u} \\
 9 & 10 & 11 & 12 & 13 & 14 & 15 & 16 & 17 & 18 & 19 & 20 & 21 & 22 & 23 & 24 & 25 & 26 & 27 \\
\end{tabularx}
\vspace{0.5\baselineskip}
\noindent\begin{tabularx}{\linewidth}{*{12}{>{\centering\arraybackslash}X}}
 \textcolor{gregoriocolor}{A} & \textcolor{gregoriocolor}{B} & \textcolor{gregoriocolor}{C} & \textcolor{gregoriocolor}{D} & \textcolor{gregoriocolor}{E} & F & \textcolor{gregoriocolor}{F} & \textcolor{gregoriocolor}{G} & \textcolor{gregoriocolor}{H} & \textcolor{gregoriocolor}{M} & \textcolor{gregoriocolor}{N} & \textcolor{gregoriocolor}{P} \\
 28 & 29 & 30 & 1 & 2 & 3 & 3 & 4 & 5 & 6 & 7 & 8 \\
\end{tabularx}

\begin{paracol}{2}
\selectlanguage{latin}
\lettrine[lines=2]{G}{ranátæ,} in Hispánia, sancti Joánnis de Deo 
 Confessóris, qui Ordinis Fratrum Hospitalitátis infirmórum fuit Institútor, 
 ac misericórdia in páuperes et sui despéctu éxstitit insígnis; quem Leo 
 Décimus tértius, Póntifex Máximus, cæléstem ómnium hospitálium et infirmórum 
 Patrónum renuntiávit.
\switchcolumn
\selectlanguage{english}
\lettrine[lines=2]{A}{t} Granada in Spain, St. John of God, founder of the Order of Brothers 
 Hospitallers, famed for his mercy to the poor, and his contempt of self. 
 Pope Leo XIII appointed him as heavenly patron of the sick and of all 
 hospitals.
\switchcolumn*
\selectlanguage{latin}
Nicomedíæ sancti Quinctílis, Epíscopi et 
 Mártyris.
\switchcolumn
\selectlanguage{english}
At Nicomedia, St. Quinctilis, bishop and martyr.
\switchcolumn*
\selectlanguage{latin}
In Africa sanctórum Mártyrum Cyrílli Epíscopi, Rogáti, Felícis, item Rogáti, 
 Beátæ, Heréniæ, Felicitátis, Urbáni, Silváni et 
 Mamílli.
\switchcolumn
\selectlanguage{english}
In Africa, the martyred Saints Cyril, bishop, Rogatus, Felix, another 
 Rogatus, Beata, Herenia, Felicitas, Urban, Silvanus, and Mamillus.
\switchcolumn*
\selectlanguage{latin}
Apud Antínoum, Ægypti urbem, natális sanctórum 
 Mártyrum Apollóni Diáconi, et Philémonis; qui, tenti et ad Júdicem addúcti, 
 et, cum sacrificáre idólis constánter renuíssent, ambo, perforátis calcáneis, 
 per civitátem horribíliter tracti, ac novíssime, gládio cæsi, martyrium 
 complevérunt.
\switchcolumn
\selectlanguage{english}
At Antinous, a city of Egypt, the birthday of the holy martyrs Apollonius, 
 deacon, and Philemon. They firmly refused to sacrifice to the idols, 
 and when arrested and brought to the judge they had their heels pierced, 
 were barbarously dragged through the city, at last completing their 
 martyrdom by being slain by the sword.
\switchcolumn*
\selectlanguage{latin}
Ibídem pássio sanctórum Ariáni Præsidis, 
 Theótici et aliórum trium; quos Judex submérsos in mare necávit, sed 
 delphinórum obséquio córpora eórum ad littus deláta sunt.
\switchcolumn
\selectlanguage{english}
In the same place, the passion of Saints Arian, governor, Theoticus, and 
 three others, whom the judge put to death by drowning in the sea. 
 Their bodies, however, were brought back by some dolphins.
\switchcolumn*
\selectlanguage{latin}
Carthágine sancti Póntii, qui fuit Diáconus beáti Cypriáni Epíscopi, et, 
 usque ad diem mortis illíus sústinens cum ipso exsílium, vitæ 
 et passiónis ejus egrégium volúmen relíquit, atque, in suis passiónibus 
 Dóminum semper gloríficans, corónam vitæ proméruit.
\switchcolumn
\selectlanguage{english}
At Carthage, St. Pontius, deacon of the blessed Cyprian, bishop, who 
 remained until death in exile with him, and composed an excellent history of 
 his life and martyrdom. By ever glorifying God in his own sufferings, 
 he merited the crown of life.
\switchcolumn*
\selectlanguage{latin}
Toléti, in Hispánia, deposítio beáti Juliáni, Epíscopi et Confessóris, 
 sanctitáte et doctrína celebérrimi.
\switchcolumn
\selectlanguage{english}
At Toledo in Spain, the death of blessed Julian, bishop and confessor, most 
 celebrated for his sanctity and learning.
\switchcolumn*
\selectlanguage{latin}
In Anglia sancti Felícis Epíscopi, qui orientáles Anglos ad fidem convértit.
\switchcolumn
\selectlanguage{english}
In England, St. Felix, bishop, who converted the East Angles to the faith.
\switchcolumn*
\selectlanguage{latin}
\end{paracol}


% ---- martyrology/mart03/mart0309.htm
\needspace{10\baselineskip}
\begin{paracol}{2}
\selectlanguage{latin}
\begin{center}{\color{gregoriocolor} Séptimo Idus Mártii. 
 Luna\dots\ }\end{center}
\switchcolumn
\selectlanguage{english}
\begin{center}{\color{gregoriocolor} The Ninth Day of March. 
 The\dots\ Day of the Moon.}\end{center}
\end{paracol}

\noindent\begin{tabularx}{\linewidth}{*{19}{>{\centering\arraybackslash}X}}
 \textcolor{gregoriocolor}{a} & \textcolor{gregoriocolor}{b} & \textcolor{gregoriocolor}{c} & \textcolor{gregoriocolor}{d} & \textcolor{gregoriocolor}{e} & \textcolor{gregoriocolor}{f} & \textcolor{gregoriocolor}{g} & \textcolor{gregoriocolor}{h} & \textcolor{gregoriocolor}{i} & \textcolor{gregoriocolor}{k} & \textcolor{gregoriocolor}{l} & \textcolor{gregoriocolor}{m} & \textcolor{gregoriocolor}{n} & \textcolor{gregoriocolor}{p} & \textcolor{gregoriocolor}{q} & \textcolor{gregoriocolor}{r} & \textcolor{gregoriocolor}{s} & \textcolor{gregoriocolor}{t} & \textcolor{gregoriocolor}{u} \\
 10 & 11 & 12 & 13 & 14 & 15 & 16 & 17 & 18 & 19 & 20 & 21 & 22 & 23 & 24 & 25 & 26 & 27 & 28 \\
\end{tabularx}
\vspace{0.5\baselineskip}
\noindent\begin{tabularx}{\linewidth}{*{12}{>{\centering\arraybackslash}X}}
 \textcolor{gregoriocolor}{A} & \textcolor{gregoriocolor}{B} & \textcolor{gregoriocolor}{C} & \textcolor{gregoriocolor}{D} & \textcolor{gregoriocolor}{E} & F & \textcolor{gregoriocolor}{F} & \textcolor{gregoriocolor}{G} & \textcolor{gregoriocolor}{H} & \textcolor{gregoriocolor}{M} & \textcolor{gregoriocolor}{N} & \textcolor{gregoriocolor}{P} \\
 29 & 30 & 1 & 2 & 3 & 4 & 4 & 5 & 6 & 7 & 8 & 9 \\
\end{tabularx}

\begin{paracol}{2}
\selectlanguage{latin}
\lettrine[lines=2]{R}{omæ} sanctæ Francíscæ Víduæ, nobilitáte géneris, 
 vitæ sanctitáte et miraculórum dono célebris.
\switchcolumn
\selectlanguage{english}
\lettrine[lines=2]{A}{t} Rome, St. Frances, widow, renowned for her noble family, holy life, and 
 the gift of miracles.
\switchcolumn*
\selectlanguage{latin}
Apud Sebásten, in Arménia, natális sanctórum Quadragínta mílitum Cappadócum, 
 qui, témpore Licínii Imperatóris, sub Præside 
 Agricoláo, post víncula et cárceres tetérrimos, post cæsas lapídibus fácies, 
 nudi sub dio, frigidíssimo híemis témpore, supra stagnum rigens pernoctáre 
 jussi sunt, ubi gelu constrícta eórum córpora disrumpebántur, ac demum 
 crurifrágio martyrium consummárunt. Erant autem inter eos nobilióres 
 Cyrion et Cándidus; eorúmque ómnium præcláras glórias sanctus Basilíus 
 aliíque Patres scriptis suis celebrárunt. Ipsórum porro Mártyrum 
 festívitas sequénti die recólitur.
\switchcolumn
\selectlanguage{english}
At Sebaste in Armenia, under the governor Agricolaus, in the time of Emperor 
 Licinius, the birthday of forty holy soldiers of Cappadocia. After 
 being chained down in foul dungeons, after having their faces bruised with 
 stones, and being condemned to spend the night naked, in the open during the 
 coldest part of winter, on a frozen lake where their bodies were benumbed 
 and covered with ice, they completed their martyrdom by having their limbs 
 crushed. The most noteworthy among them were Cyrion and Candidus. 
 Their glorious triumph has been celebrated by St. Basil and other Fathers in 
 their writings. Their feast is kept tomorrow.
\switchcolumn*
\selectlanguage{latin}
Nyssæ deposítio sancti Gregórii Epíscopi, qui 
 sanctórum Basilíi et Emméliæ fílius, et sanctórum item Basilíi Magni ac 
 Petri Sebasténsis Episcopórum et Macrínæ Vírginis frater éxstitit; atque, 
 vita et eruditióne claríssimus, ob fídei cathólicæ defensiónem, sub Ariáno 
 Imperatóre Valénte, civitáte sua pulsus est.
\switchcolumn
\selectlanguage{english}
At Nyssa, the death of St. Gregory, the son of Saints Basil and Emmelia, and 
 the brother of Saints Basil the Great, bishop, and Peter, bishop of Sebaste, 
 and Macrina, virgin. His life and his great learning brought him fame. 
 He was driven from his own city for having defended the Catholic faith 
 during the reign of the Arian emperor Valens.
\switchcolumn*
\selectlanguage{latin}
Barcinóne, in Hispánia, sancti Paciáni Epíscopi, tam vita quam sermóne 
 conspícui; qui, témpore Theodósii Príncipis, in 
 última senectúte finem vitæ 
 sortítus est.
\switchcolumn
\selectlanguage{english}
At Barcelona in Spain, Bishop St. Pacian, distinguished by his life and 
 preaching. He ended his career in extreme old age, in the time of 
 Emperor Theodosius.
\switchcolumn*
\selectlanguage{latin}
Bonóniæ sanctæ Catharínæ Vírginis, e secúndo 
 Ordine sancti Francísci, quæ vitæ sanctitáte fuit illústris. Ipsíus 
 autem corpus magno cum honóre ibídem cólitur.
\switchcolumn
\selectlanguage{english}
At Bologna, St. Catherine, virgin, of the Second Order of St. Francis, 
 illustrious for the holiness of her life. Her body is greatly honoured 
 in that city.
\switchcolumn*
\selectlanguage{latin}
\end{paracol}


% ---- martyrology/mart03/mart0310.htm
\needspace{10\baselineskip}
\begin{paracol}{2}
\selectlanguage{latin}
\begin{center}{\color{gregoriocolor} Sexto Idus Mártii. 
 Luna\dots\ }\end{center}
\switchcolumn
\selectlanguage{english}
\begin{center}{\color{gregoriocolor} The Tenth Day of March. 
 The\dots\ Day of the Moon.}\end{center}
\end{paracol}

\noindent\begin{tabularx}{\linewidth}{*{19}{>{\centering\arraybackslash}X}}
 \textcolor{gregoriocolor}{a} & \textcolor{gregoriocolor}{b} & \textcolor{gregoriocolor}{c} & \textcolor{gregoriocolor}{d} & \textcolor{gregoriocolor}{e} & \textcolor{gregoriocolor}{f} & \textcolor{gregoriocolor}{g} & \textcolor{gregoriocolor}{h} & \textcolor{gregoriocolor}{i} & \textcolor{gregoriocolor}{k} & \textcolor{gregoriocolor}{l} & \textcolor{gregoriocolor}{m} & \textcolor{gregoriocolor}{n} & \textcolor{gregoriocolor}{p} & \textcolor{gregoriocolor}{q} & \textcolor{gregoriocolor}{r} & \textcolor{gregoriocolor}{s} & \textcolor{gregoriocolor}{t} & \textcolor{gregoriocolor}{u} \\
 11 & 12 & 13 & 14 & 15 & 16 & 17 & 18 & 19 & 20 & 21 & 22 & 23 & 24 & 25 & 26 & 27 & 28 & 29 \\
\end{tabularx}
\vspace{0.5\baselineskip}
\noindent\begin{tabularx}{\linewidth}{*{12}{>{\centering\arraybackslash}X}}
 \textcolor{gregoriocolor}{A} & \textcolor{gregoriocolor}{B} & \textcolor{gregoriocolor}{C} & \textcolor{gregoriocolor}{D} & \textcolor{gregoriocolor}{E} & F & \textcolor{gregoriocolor}{F} & \textcolor{gregoriocolor}{G} & \textcolor{gregoriocolor}{H} & \textcolor{gregoriocolor}{M} & \textcolor{gregoriocolor}{N} & \textcolor{gregoriocolor}{P} \\
 30 & 1 & 2 & 3 & 4 & 5 & 5 & 6 & 7 & 8 & 9 & 10 \\
\end{tabularx}

\begin{paracol}{2}
\selectlanguage{latin}
\lettrine[lines=2]{S}{anctórum} Quadragínta Mártyrum, quorum natális prídie hujus diéi recensétur.
\switchcolumn
\selectlanguage{english}
\lettrine[lines=2]{T}{he} forty holy martyrs whose birthday was commemorated yesterday.
\switchcolumn*
\selectlanguage{latin}
Apaméæ, in Phrygia, natális sanctórum Mártyrum 
 Caji et Alexándri, qui (ut scribit Apollináris, Hierapolitánus Epíscopus, in 
 libro advérsus Cataphrygas hæréticos), in persecutióne Marci Antoníni et 
 Lúcii Veri, glorióso martyrio coronáti sunt.
\switchcolumn
\selectlanguage{english}
At Apamea in Phrygia, during the persecution of Marcus Antoninus and Lucius 
 Verus, the birthday of the holy martyrs Caius and Alexander. They were 
 crowned with a glorious martyrdom, as is related by Apollinaris, bishop of 
 Hierapolis, in his book against the Cataphrygian heretics.
\switchcolumn*
\selectlanguage{latin}
In Pérside pássio sanctórum quadragínta duórum Mártyrum.
\switchcolumn
\selectlanguage{english}
In Persia, the passion of forty-two holy martyrs.
\switchcolumn*
\selectlanguage{latin}
Corínthi sanctórum Mártyrum Codráti, Dionysii, Cypriáni, Anécti, Pauli et 
 Crescéntis, qui, in persecutióne Décii et Valeriáni, sub Jásone 
 Prǽside, gládio cæsi sunt.
\switchcolumn
\selectlanguage{english}
At Corinth, the holy martyrs Codratus, Denis, Cyprian, Anectus, Paul, and 
 Crescens, who were slain with the sword in the persecution of Decius and 
 Valerian, under Jason, the governor.
\switchcolumn*
\selectlanguage{latin}
In Africa sancti Victóris Mártyris, in cujus solemnitáte sanctus Augustínus 
 ad pópulum de ipso tractátum hábuit.
\switchcolumn
\selectlanguage{english}
In Africa, St. Victor, martyr, on whose feast day St. Augustine delivered a 
 sermon to his people.
\switchcolumn*
\selectlanguage{latin}
Romæ sancti Simplícii, Papæ et Confessóris.
\switchcolumn
\selectlanguage{english}
At Rome, St. Simplicius, pope and confessor.
\switchcolumn*
\selectlanguage{latin}
Hierosólymis sancti Macárii, Epíscopi et Confessóris; cujus hortátu loca 
 sancta a Constantíno Magno et beáta Hélena, 
 ejus matre, expurgáta sunt et sacris Basílicis illustráta.
\switchcolumn
\selectlanguage{english}
At Jerusalem, St. Macarius, bishop and confessor, at whose exhortation the 
 holy places were purged by Constantine the Great and St. Helen, his mother, 
 and beautified by sacred basilicas.
\switchcolumn*
\selectlanguage{latin}
Lutétiæ Parisiórum deposítio sancti Droctovéi 
 Abbátis, qui fuit discípulus beáti Germáni Epíscopi.
\switchcolumn
\selectlanguage{english}
At Paris, the death of Abbot St. Droctoveus, who was a disciple of the 
 saintly Bishop Germanus.
\switchcolumn*
\selectlanguage{latin}
In monastério Bobiénsi sancti Attalæ Abbátis, 
 miráculis clari.
\switchcolumn
\selectlanguage{english}
In the monastery of Bobbio, St. Attala, abbot, renowned for his miracles.
\switchcolumn*
\selectlanguage{latin}
\end{paracol}


% ---- martyrology/mart03/mart0311.htm
\needspace{10\baselineskip}
\begin{paracol}{2}
\selectlanguage{latin}
\begin{center}{\color{gregoriocolor} Quinto Idus Mártii. 
 Luna\dots\ }\end{center}
\switchcolumn
\selectlanguage{english}
\begin{center}{\color{gregoriocolor} The Eleventh Day of March. 
 The\dots\ Day of the Moon.}\end{center}
\end{paracol}

\noindent\begin{tabularx}{\linewidth}{*{19}{>{\centering\arraybackslash}X}}
 \textcolor{gregoriocolor}{a} & \textcolor{gregoriocolor}{b} & \textcolor{gregoriocolor}{c} & \textcolor{gregoriocolor}{d} & \textcolor{gregoriocolor}{e} & \textcolor{gregoriocolor}{f} & \textcolor{gregoriocolor}{g} & \textcolor{gregoriocolor}{h} & \textcolor{gregoriocolor}{i} & \textcolor{gregoriocolor}{k} & \textcolor{gregoriocolor}{l} & \textcolor{gregoriocolor}{m} & \textcolor{gregoriocolor}{n} & \textcolor{gregoriocolor}{p} & \textcolor{gregoriocolor}{q} & \textcolor{gregoriocolor}{r} & \textcolor{gregoriocolor}{s} & \textcolor{gregoriocolor}{t} & \textcolor{gregoriocolor}{u} \\
 12 & 13 & 14 & 15 & 16 & 17 & 18 & 19 & 20 & 21 & 22 & 23 & 24 & 25 & 26 & 27 & 28 & 29 & 30 \\
\end{tabularx}
\vspace{0.5\baselineskip}
\noindent\begin{tabularx}{\linewidth}{*{12}{>{\centering\arraybackslash}X}}
 \textcolor{gregoriocolor}{A} & \textcolor{gregoriocolor}{B} & \textcolor{gregoriocolor}{C} & \textcolor{gregoriocolor}{D} & \textcolor{gregoriocolor}{E} & F & \textcolor{gregoriocolor}{F} & \textcolor{gregoriocolor}{G} & \textcolor{gregoriocolor}{H} & \textcolor{gregoriocolor}{M} & \textcolor{gregoriocolor}{N} & \textcolor{gregoriocolor}{P} \\
 1 & 2 & 3 & 4 & 5 & 6 & 6 & 7 & 8 & 9 & 10 & 11 \\
\end{tabularx}

\begin{paracol}{2}
\selectlanguage{latin}
\lettrine[lines=2]{S}{ardis} sancti Euthymii Epíscopi, qui, ob cultum sanctárum Imáginum, a 
 Michaéle, Imperatóre Iconoclásta, in exsílium missus est; ac demum, 
 Theóphilo imperánte, búbulis nervis, 
 inhumániter cæsus, martyrium consummávit.
\switchcolumn
\selectlanguage{english}
\lettrine[lines=2]{A}{t} Sardis, St. Euthymius, bishop, who was sent into exile by the Iconoclast 
 emperor Michael for the veneration of sacred images. Afterwards, in 
 the reign of Theophilus, he was barbarously beaten with knotted clubs, which 
 completed his martyrdom.
\switchcolumn*
\selectlanguage{latin}
Córdubæ, in Hispánia, sancti Eulógii, 
 Presbyteri et Mártyris; qui, in persecutióne Saracenórum, ob præcláram et 
 intrépidam Christi confessiónem, verbéribus et álapis cæsus ac decollátus 
 gládio, adjúngi ejúsdem urbis Martyribus méruit, quorum pro fide certámina 
 scribéndo fúerat æmulátus.
\switchcolumn
\selectlanguage{english}
At Cordova in Spain, St. Eulogius, priest, who deserved to be associated 
 with the martyrs of that city because, in writing of their trials for the 
 faith, he had envied their happiness. On account of his own fearless 
 and intrepid confession of Christ, he was scourged and beaten with rods, and 
 finally beheaded during the Saracen persecution.
\switchcolumn*
\selectlanguage{latin}
Carthágine sanctórum Mártyrum Heráclíi et 
 Zósimi.
\switchcolumn
\selectlanguage{english}
At Carthage, the holy martyrs Heraclius and Zosimus.
\switchcolumn*
\selectlanguage{latin}
Alexandríæ pássio sanctórum Cándidi, Piperiónis 
 et aliórum vigínti.
\switchcolumn
\selectlanguage{english}
At Alexandria, the passion of Saints Candidus, Piperion, and twenty others.
\switchcolumn*
\selectlanguage{latin}
Laodicéæ, in Syria, sanctórum Mártyrum Tróphimi 
 et Thali, qui in persecutióne Diocletiáni, post multa sǽvaque torménta, 
 corónas glóriæ sunt assecúti.
\switchcolumn
\selectlanguage{english}
At Laodicea in Syria, during the persecution of Diocletian, the holy martyrs 
 Trophimus and Thalus, who obtained their crowns of glory after many severe 
 torments.
\switchcolumn*
\selectlanguage{latin}
Antiochíæ commemorátio plurimórum sanctórum 
 Mártyrum, quorum álii, Maximiáni Imperatóris mandáto, candéntibus cratículis 
 superpósiti, non ad mortem sed a diutúrnum cruciátum assáti, álii áliis 
 sævíssimis affécti supplíciis, ad palmam martyrii pervenérunt.
\switchcolumn
\selectlanguage{english}
At Antioch, the Commemoration of many holy martyrs, some of whom by order of 
 Emperor Maximian were laid on red hot gridirons, not to be burned to death, 
 but to continue their suffering a longer time; others were subjected to 
 different horrible torments, and won the palm of martyrdom.
\switchcolumn*
\selectlanguage{latin}
Item sanctórum Mártyrum Gorgónii et Firmi.
\switchcolumn
\selectlanguage{english}
Also, Saints Gorgonius and Firmus.
\switchcolumn*
\selectlanguage{latin}
Hierosólymis sancti Sophrónii Epíscopi.
\switchcolumn
\selectlanguage{english}
At Jerusalem, Bishop St. Sophronius.
\switchcolumn*
\selectlanguage{latin}
Medioláni sancti Benedícti Epíscopi.
\switchcolumn
\selectlanguage{english}
At Milan, St. Benedict, bishop.
\switchcolumn*
\selectlanguage{latin}
In fínibus Ambianénsium sancti Firmíni Abbátis.
\switchcolumn
\selectlanguage{english}
In the diocese of Amiens, St. Firmin, abbot.
\switchcolumn*
\selectlanguage{latin}
Carthágine sancti Constantíni Confessóris.
\switchcolumn
\selectlanguage{english}
At Carthage, St. Constantine, confessor.
\switchcolumn*
\selectlanguage{latin}
Babúci, in Hérnicis, sancti Petri Confessóris, 
 miraculórum glória conspícui.
\switchcolumn
\selectlanguage{english}
At Babucum in Campania, St. Peter, confessor, who was renowned for his 
 miracles.
\switchcolumn*
\selectlanguage{latin}
\end{paracol}


% ---- martyrology/mart03/mart0312.htm
\needspace{10\baselineskip}
\begin{paracol}{2}
\selectlanguage{latin}
\begin{center}{\color{gregoriocolor} Quarto Idus Mártii. 
 Luna\dots\ }\end{center}
\switchcolumn
\selectlanguage{english}
\begin{center}{\color{gregoriocolor} The Twelfth Day of March. 
 The\dots\ Day of the Moon.}\end{center}
\end{paracol}

\noindent\begin{tabularx}{\linewidth}{*{19}{>{\centering\arraybackslash}X}}
 \textcolor{gregoriocolor}{a} & \textcolor{gregoriocolor}{b} & \textcolor{gregoriocolor}{c} & \textcolor{gregoriocolor}{d} & \textcolor{gregoriocolor}{e} & \textcolor{gregoriocolor}{f} & \textcolor{gregoriocolor}{g} & \textcolor{gregoriocolor}{h} & \textcolor{gregoriocolor}{i} & \textcolor{gregoriocolor}{k} & \textcolor{gregoriocolor}{l} & \textcolor{gregoriocolor}{m} & \textcolor{gregoriocolor}{n} & \textcolor{gregoriocolor}{p} & \textcolor{gregoriocolor}{q} & \textcolor{gregoriocolor}{r} & \textcolor{gregoriocolor}{s} & \textcolor{gregoriocolor}{t} & \textcolor{gregoriocolor}{u} \\
 13 & 14 & 15 & 16 & 17 & 18 & 19 & 20 & 21 & 22 & 23 & 24 & 25 & 26 & 27 & 28 & 29 & 30 & 1 \\
\end{tabularx}
\vspace{0.5\baselineskip}
\noindent\begin{tabularx}{\linewidth}{*{12}{>{\centering\arraybackslash}X}}
 \textcolor{gregoriocolor}{A} & \textcolor{gregoriocolor}{B} & \textcolor{gregoriocolor}{C} & \textcolor{gregoriocolor}{D} & \textcolor{gregoriocolor}{E} & F & \textcolor{gregoriocolor}{F} & \textcolor{gregoriocolor}{G} & \textcolor{gregoriocolor}{H} & \textcolor{gregoriocolor}{M} & \textcolor{gregoriocolor}{N} & \textcolor{gregoriocolor}{P} \\
 2 & 3 & 4 & 5 & 6 & 7 & 7 & 8 & 9 & 10 & 11 & 12 \\
\end{tabularx}

\begin{paracol}{2}
\selectlanguage{latin}
\lettrine[lines=2]{R}{omæ} sancti Gregórii Primi, Papæ, Confessóris 
 et Ecclésiæ Doctóris exímii; qui, ob res præcláræ gestas atque Anglos ad 
 Christi fidem convérsos, Magnus est dictus et Anglórum Apóstolus appellátus.
\switchcolumn
\selectlanguage{english}
\lettrine[lines=2]{A}{t} Rome, St. Gregory, pope and eminent doctor of the Church, who on account 
 of his illustrious deeds and the conversion of the English to the faith of 
 Christ, was surnamed the Great, and called the Apostle of England.
\switchcolumn*
\selectlanguage{latin}
Ibídem deposítio sancti Innocéntii Primi, Papæ 
 et Confessóris. Ipsíus autem festum quinto Kaléndas Augústi celebrátur.
\switchcolumn
\selectlanguage{english}
In the same place, the death of St. Innocent I, pope and confessor. 
 His feast is celebrated on the 28th of July.
\switchcolumn*
\selectlanguage{latin}
Item Romæ sancti Mamiliáni Mártyris.
\switchcolumn
\selectlanguage{english}
Also at Rome, St. Mamilian, martyr.
\switchcolumn*
\selectlanguage{latin}
Nicomedíæ sanctórum Egdúni Presbyteri, et 
 aliórum septem; qui singuli diébus suffocáti sunt, ut céteris metus 
 incuterétur.
\switchcolumn
\selectlanguage{english}
At Nicomedia, St. Egdunus, priest, and seven others, who, one by one, on 
 successive days, were strangled in order to terrify those who remained.
\switchcolumn*
\selectlanguage{latin}
Ibídem pássio sancti Petri Mártyris, qui, cum esset cubiculárius Diocletiáni 
 Imperatóris, et libérius de imménsis Mártyrum supplíciis quererétur, 
 proptérea, jubénte eódem, in médium addúcitur, ac primo suspénsus, 
 diutíssime flagris torquétur, deínde acéto ac sale perfúsus, ad últimum in 
 cratícula lento igne assátur, sicque vere Petri éxstitit et fídei heres et 
 nóminis.
\switchcolumn
\selectlanguage{english}
In the same city, the passion of the martyr St. Peter, chamberlain to 
 Emperor Diocletian. For openly complaining of the atrocious torments 
 inflicted upon the martyrs, he was, by order of the emperor, first suspended 
 and for a long time scourged, then had salt and vinegar poured on his 
 wounds, and finally was burned on a grate over a slow fire. Thus did 
 he become a true heir of St. Peter's name and faith.
\switchcolumn*
\selectlanguage{latin}
Constantinópoli sancti Theóphanis, qui, ex ditíssimo pauper Mónachus 
 efféctus, ab ímpio Leóne Arméno, pro cultu sacrárum Imáginum, biénnio 
 deténtus est in cárcere, et inde in Samothráciam deportátus, ibídem,
 ærúmnis conféctus, réddidit spíritum, multísque 
 miráculis cláruit.
\switchcolumn
\selectlanguage{english}
At Constantinople, St. Theophanes, who gave up great riches to embrace the 
 poverty of the monastic state. The impious Leo the Armenian kept him 
 in prison for two years because of his veneration of sacred images, and 
 later sent him into Thrace in exile. There, overwhelmed with 
 afflictions, but famous for miracles, death came upon him.
\switchcolumn*
\selectlanguage{latin}
Cápuæ sancti Bernárdi, Epíscopi et Confessóris.
\switchcolumn
\selectlanguage{english}
At Capua, St. Bernard, bishop and confessor.
\switchcolumn*
\selectlanguage{latin}
\end{paracol}


% ---- martyrology/mart03/mart0313.htm
\needspace{10\baselineskip}
\begin{paracol}{2}
\selectlanguage{latin}
\begin{center}{\color{gregoriocolor} Tértio Idus Mártii. 
 Luna\dots\ }\end{center}
\switchcolumn
\selectlanguage{english}
\begin{center}{\color{gregoriocolor} The Thirteenth Day of March. 
 The\dots\ Day of the Moon.}\end{center}
\end{paracol}

\noindent\begin{tabularx}{\linewidth}{*{19}{>{\centering\arraybackslash}X}}
 \textcolor{gregoriocolor}{a} & \textcolor{gregoriocolor}{b} & \textcolor{gregoriocolor}{c} & \textcolor{gregoriocolor}{d} & \textcolor{gregoriocolor}{e} & \textcolor{gregoriocolor}{f} & \textcolor{gregoriocolor}{g} & \textcolor{gregoriocolor}{h} & \textcolor{gregoriocolor}{i} & \textcolor{gregoriocolor}{k} & \textcolor{gregoriocolor}{l} & \textcolor{gregoriocolor}{m} & \textcolor{gregoriocolor}{n} & \textcolor{gregoriocolor}{p} & \textcolor{gregoriocolor}{q} & \textcolor{gregoriocolor}{r} & \textcolor{gregoriocolor}{s} & \textcolor{gregoriocolor}{t} & \textcolor{gregoriocolor}{u} \\
 14 & 15 & 16 & 17 & 18 & 19 & 20 & 21 & 22 & 23 & 24 & 25 & 26 & 27 & 28 & 29 & 30 & 1 & 2 \\
\end{tabularx}
\vspace{0.5\baselineskip}
\noindent\begin{tabularx}{\linewidth}{*{12}{>{\centering\arraybackslash}X}}
 \textcolor{gregoriocolor}{A} & \textcolor{gregoriocolor}{B} & \textcolor{gregoriocolor}{C} & \textcolor{gregoriocolor}{D} & \textcolor{gregoriocolor}{E} & F & \textcolor{gregoriocolor}{F} & \textcolor{gregoriocolor}{G} & \textcolor{gregoriocolor}{H} & \textcolor{gregoriocolor}{M} & \textcolor{gregoriocolor}{N} & \textcolor{gregoriocolor}{P} \\
 3 & 4 & 5 & 6 & 7 & 8 & 8 & 9 & 10 & 11 & 12 & 13 \\
\end{tabularx}

\begin{paracol}{2}
\selectlanguage{latin}
\lettrine[lines=2]{C}{órdubæ} in Hispánia, sanctórum Mártyrum 
 Ruderíci Presbyteri, et Salomónis.
\switchcolumn
\selectlanguage{english}
\lettrine[lines=2]{A}{t} Cordova in Spain, the holy martyrs Roderick, priest, and Solomon.
\switchcolumn*
\selectlanguage{latin}
Nicomediæ natális sanctórum Mártyrum Macedónii, 
 Patríciæ uxóris, et Modéstæ fíliæ.
\switchcolumn
\selectlanguage{english}
At Nicomedia, the birthday of the holy martyrs Macedonius, Patricia, his 
 wife, and his daughter Modesta.
\switchcolumn*
\selectlanguage{latin}
Nicǽæ, in Bithynia, sanctórum Mártyrum Theusétæ, 
 ejúsque fílii Horris, Theodóræ, Nymphodóræ, Marci et Arábiæ; qui omnes pro 
 Christo igni tráditi sunt.
\switchcolumn
\selectlanguage{english}
At Nicaea in Bithynia, the holy martyrs Theusetas and Horres, his son; 
 Theodore, Nymphodora, Mark, and Arabia, who were all burned to death for 
 Christ.
\switchcolumn*
\selectlanguage{latin}
Hermópoli, in Ægypto, sancti Sabíni Mártyris, 
 qui multa passus, tandem, projéctus in flumen, martyrium consummávit.
\switchcolumn
\selectlanguage{english}
At Hermopolis in Egypt, the martyr St. Sabinus, who suffered many torments, 
 and at last completed his martyrdom by being cast into a river.
\switchcolumn*
\selectlanguage{latin}
In Pérside sanctæ Christínæ, Vírginis et 
 Mártyris.
\switchcolumn
\selectlanguage{english}
In Persia, St. Christina, virgin and martyr.
\switchcolumn*
\selectlanguage{latin}
Apud Camerínum sancti Ansovíni, Epíscopi et Confessóris.
\switchcolumn
\selectlanguage{english}
At Camerino, St. Ansovinus, bishop and confessor.
\switchcolumn*
\selectlanguage{latin}
In Thebáide deposítio sanctæ Euphrásiæ Vírginis.
\switchcolumn
\selectlanguage{english}
In Thebais, the death of St. Euphrasia, virgin.
\switchcolumn*
\selectlanguage{latin}
Constantinópoli Translátio sancti Nicéphori, 
 Epíscopi ejúsdem urbis et Confessóris; cujus corpus e Proconnéso, 
 Propóntidis ínsula, ubi ipse quarto Nonas Júnii ob sanctárum Imáginum cultum 
 exsul obíerat, Constantinópolim relátum est, atque a sancto illíus civitátis 
 Epíscopo Methódio honorífice in templo sanctórum Apostolórum sepúltum, hac 
 ipsa recurrénte die, in qua olim idem Nicéphorus in exsílium fúerat 
 deportátus.
\switchcolumn
\selectlanguage{english}
At Constantinople, the transferral of the body of St. Nicephorus, bishop of 
 that city, and confessor. The body was returned from the island of 
 Propontis in the Proconnesus, where his death occurred on the 5th of June 
 while in exile for his reverence of sacred images. He was buried with 
 honour by Bishop Methodius in the Church of the Holy Apostles on this the 
 anniversary day of his exile.
\switchcolumn*
\selectlanguage{latin}
\end{paracol}


% ---- martyrology/mart03/mart0314.htm
\needspace{10\baselineskip}
\begin{paracol}{2}
\selectlanguage{latin}
\begin{center}{\color{gregoriocolor} Prídie Idus Mártii. 
 Luna\dots\ }\end{center}
\switchcolumn
\selectlanguage{english}
\begin{center}{\color{gregoriocolor} The Fourteenth Day of March. 
 The\dots\ Day of the Moon.}\end{center}
\end{paracol}

\noindent\begin{tabularx}{\linewidth}{*{19}{>{\centering\arraybackslash}X}}
 \textcolor{gregoriocolor}{a} & \textcolor{gregoriocolor}{b} & \textcolor{gregoriocolor}{c} & \textcolor{gregoriocolor}{d} & \textcolor{gregoriocolor}{e} & \textcolor{gregoriocolor}{f} & \textcolor{gregoriocolor}{g} & \textcolor{gregoriocolor}{h} & \textcolor{gregoriocolor}{i} & \textcolor{gregoriocolor}{k} & \textcolor{gregoriocolor}{l} & \textcolor{gregoriocolor}{m} & \textcolor{gregoriocolor}{n} & \textcolor{gregoriocolor}{p} & \textcolor{gregoriocolor}{q} & \textcolor{gregoriocolor}{r} & \textcolor{gregoriocolor}{s} & \textcolor{gregoriocolor}{t} & \textcolor{gregoriocolor}{u} \\
 15 & 16 & 17 & 18 & 19 & 20 & 21 & 22 & 23 & 24 & 25 & 26 & 27 & 28 & 29 & 30 & 1 & 2 & 3 \\
\end{tabularx}
\vspace{0.5\baselineskip}
\noindent\begin{tabularx}{\linewidth}{*{12}{>{\centering\arraybackslash}X}}
 \textcolor{gregoriocolor}{A} & \textcolor{gregoriocolor}{B} & \textcolor{gregoriocolor}{C} & \textcolor{gregoriocolor}{D} & \textcolor{gregoriocolor}{E} & F & \textcolor{gregoriocolor}{F} & \textcolor{gregoriocolor}{G} & \textcolor{gregoriocolor}{H} & \textcolor{gregoriocolor}{M} & \textcolor{gregoriocolor}{N} & \textcolor{gregoriocolor}{P} \\
 4 & 5 & 6 & 7 & 8 & 9 & 9 & 10 & 11 & 12 & 13 & 14 \\
\end{tabularx}

\begin{paracol}{2}
\selectlanguage{latin}
\lettrine[lines=2]{R}{omæ,} in agro Veráno, sancti Leónis, Epíscopi 
 et Mártyris.
\switchcolumn
\selectlanguage{english}
\lettrine[lines=2]{A}{t} Rome, in the Veranian Field, St. Leo, bishop and martyr.
\switchcolumn*
\selectlanguage{latin}
Item Romæ natális sanctórum quadragínta septem 
 Mártyrum, qui baptizáti sunt a beáto Apóstolo Petro, cum tenerétur in 
 custódia Mamertíni cum Coapóstolo suo Paulo, ubi novem menses deténti sunt; 
 qui omnes, sub devotíssima fídei confessióne, Neroniáno gládio consúmpti 
 sunt.
\switchcolumn
\selectlanguage{english}
Also at Rome, the birthday of forty-seven holy martyrs who were baptized by 
 the apostle St. Peter while in the Mamertine Prison with St. Paul his fellow 
 apostle. After an imprisonment of nine months, they all fell by the 
 sword of Nero for their generous confession of faith.
\switchcolumn*
\selectlanguage{latin}
In Província Valériæ sanctórum duórum 
 Monachórum, quos Longobárdi suspéndio necavérunt in árbore; in qua Mártyres, 
 licet defúncti, ab hóstibus ipsis audíti sunt psállere.
\switchcolumn
\selectlanguage{english}
In the province of Valeria, two saintly monks, who were hanged on a tree by 
 the Lombards, and although dead, were heard singing psalms even by their 
 enemies.
\switchcolumn*
\selectlanguage{latin}
In ea étiam persecutióne Diáconus Ecclésiæ 
 Marsicánæ, in confessióne fídei, cápite truncátus est.
\switchcolumn
\selectlanguage{english}
In the same persecution, a deacon of the church of Marsico who was beheaded 
 for the confession of faith.
\switchcolumn*
\selectlanguage{latin}
In Africa sanctórum Mártyrum Petri et Aphrodísii, qui, in persecutióne 
 Wandálica, martyrii corónam percepérunt.
\switchcolumn
\selectlanguage{english}
In Africa, the holy martyrs Peter and Aphrodisius, who received the crown of 
 martyrdom in the Vandal persecution.
\switchcolumn*
\selectlanguage{latin}
Carrhis, in Mesopotámia, sancti Eutychii patrícii, et Sociórum, qui ab 
 Evelid, Arabum Rege, ob fídei confessiónem, interémpti sunt.
\switchcolumn
\selectlanguage{english}
At Carrhae in Mesopotamia, the patrician St. Eutychius and his companions, 
 who were killed by Evelid, king of Arabia, for the confession of the faith.
\switchcolumn*
\selectlanguage{latin}
Halberstátti, in Germánia, dormítio beátæ 
 Mathíldis Regínæ, matris Othónis Primi, Romanórum Imperatóris, humilitáte et 
 patiéntia conspícuæ.
\switchcolumn
\selectlanguage{english}
At Halberstadt in Germany, the death of blessed Queen Matilda, mother of 
 Emperor Otto I, renowned for humility and patience.
\switchcolumn*
\selectlanguage{latin}
\end{paracol}


% ---- martyrology/mart03/mart0315.htm
\needspace{10\baselineskip}
\begin{paracol}{2}
\selectlanguage{latin}
\begin{center}{\color{gregoriocolor} Idibus Mártii. 
 Luna\dots\ }\end{center}
\switchcolumn
\selectlanguage{english}
\begin{center}{\color{gregoriocolor} The Fifteenth Day of March. 
 The\dots\ Day of the Moon.}\end{center}
\end{paracol}

\noindent\begin{tabularx}{\linewidth}{*{19}{>{\centering\arraybackslash}X}}
 \textcolor{gregoriocolor}{a} & \textcolor{gregoriocolor}{b} & \textcolor{gregoriocolor}{c} & \textcolor{gregoriocolor}{d} & \textcolor{gregoriocolor}{e} & \textcolor{gregoriocolor}{f} & \textcolor{gregoriocolor}{g} & \textcolor{gregoriocolor}{h} & \textcolor{gregoriocolor}{i} & \textcolor{gregoriocolor}{k} & \textcolor{gregoriocolor}{l} & \textcolor{gregoriocolor}{m} & \textcolor{gregoriocolor}{n} & \textcolor{gregoriocolor}{p} & \textcolor{gregoriocolor}{q} & \textcolor{gregoriocolor}{r} & \textcolor{gregoriocolor}{s} & \textcolor{gregoriocolor}{t} & \textcolor{gregoriocolor}{u} \\
 16 & 17 & 18 & 19 & 20 & 21 & 22 & 23 & 24 & 25 & 26 & 27 & 28 & 29 & 30 & 1 & 2 & 3 & 4 \\
\end{tabularx}
\vspace{0.5\baselineskip}
\noindent\begin{tabularx}{\linewidth}{*{12}{>{\centering\arraybackslash}X}}
 \textcolor{gregoriocolor}{A} & \textcolor{gregoriocolor}{B} & \textcolor{gregoriocolor}{C} & \textcolor{gregoriocolor}{D} & \textcolor{gregoriocolor}{E} & F & \textcolor{gregoriocolor}{F} & \textcolor{gregoriocolor}{G} & \textcolor{gregoriocolor}{H} & \textcolor{gregoriocolor}{M} & \textcolor{gregoriocolor}{N} & \textcolor{gregoriocolor}{P} \\
 5 & 6 & 7 & 8 & 9 & 10 & 10 & 11 & 12 & 13 & 14 & 15 \\
\end{tabularx}

\begin{paracol}{2}
\selectlanguage{latin}
\lettrine[lines=2]{C}{æsaréæ,} in Cappadócia, pássio sancti Longíni 
 mílitis, qui Dómini latus láncea perforásse perhibétur.
\switchcolumn
\selectlanguage{english}
\lettrine[lines=2]{A}{t} Caesarea in Cappadocia, the martyrdom of St. Longinus, the soldier who is 
 said to have pierced our Lord's side with a lance.
\switchcolumn*
\selectlanguage{latin}
Eódem die natális sancti Aristobúli, 
 Apostolórum discípuli, qui, cursu prædicatiónis perácto, martyrium 
 consummávit.
\switchcolumn
\selectlanguage{english}
The same day, the birthday of St. Aristobulus, a disciple of the apostles, 
 who completed by martyrdom a life spent in preaching the Gospel.
\switchcolumn*
\selectlanguage{latin}
In Hellespónto sancti Menígni fullónis, qui sub Décio Imperatóre passus est.
\switchcolumn
\selectlanguage{english}
In the Hellespont, St. Menignus, a dyer, who suffered under Decius.
\switchcolumn*
\selectlanguage{latin}
In Ægypto sancti Nicándri Mártyris, qui, cum 
 sanctórum Mártyrum relíquias studióse perquíreret, et ipse méruit éffici 
 Martyr, sub Diocletiáno Imperatóre.
\switchcolumn
\selectlanguage{english}
In Egypt, St. Nicander, martyr, who sought diligently for the remains of the 
 holy martyrs, and thus merited to be made a martyr himself, under Emperor 
 Diocletian.
\switchcolumn*
\selectlanguage{latin}
Córdubæ, in Hispánia, sanctæ Leocrítiæ, 
 Vírginis et Mártyris; quæ ob Christi fidem, in persecutióne Arábica, 
 divérsis cruciátibus afflícta et gládio decolláta est.
\switchcolumn
\selectlanguage{english}
At Cordova in Spain, St. Leocritia, virgin and martyr. She suffered 
 various cruel tortures and was beheaded for the faith of Christ during the 
 Arabian persecution.
\switchcolumn*
\selectlanguage{latin}
Thessalonícæ sanctæ Matrónæ, quæ, cum esset 
 ancílla cujúsdam Judǽæ, et occúlte Christum cóleret, ac furtívis oratiónibus 
 quotídie Ecclésiam frequentáret, a dómina sua est deprehénsa et 
 multiplíciter afflícta, atque novíssime, robústis fústibus usque ad mortem 
 cæsa, in confessióne Christi, incorrúptum Deo spíritum réddidit.
\switchcolumn
\selectlanguage{english}
At Thessalonica, St. Matrona, servant of a Jewess, who, worshipping Christ 
 secretly, and stealing away daily to pray in the church, was detected by her 
 mistress and subjected to many trials. Being at last beaten to death 
 with large clubs, she gave up her pure soul to God in confessing Christ.
\switchcolumn*
\selectlanguage{latin}
Reáte sancti Probi Epíscopi, cui moriénti Juvenális et Eleuthérius Mártyres 
 adfuérunt.
\switchcolumn
\selectlanguage{english}
At Rieti, the bishop St. Probus, at whose death the martyrs Juvenal and 
 Eleutherius were present.
\switchcolumn*
\selectlanguage{latin}
Vindobónæ, in Austria, sancti Cleméntis-Maríæ 
 Hofbauer, Sacerdótis proféssi Congregatiónis a sanctíssimo Redemptóre 
 nuncupátæ, plúrimis in Dei glória et animárum salúte promovénda ac dilatánda 
 ipsa Congregatióne exantlátis labóribus insígnis; quem, virtútibus et 
 miráculis clarum, Pius Décimus, Póntifex Máximus, in Sanctórum cánonem 
 rétulit.
\switchcolumn
\selectlanguage{english}
At Vienna in Austria, St. Clement Mary Hofbauer, a priest of the 
 Congregation of the Most Holy Redeemer, renowned for his great devotion in 
 promoting the glory of God and the salvation of souls, and in extending that 
 order. He was canonized by Pope Pius X.
\switchcolumn*
\selectlanguage{latin}
Apud Cápuam sancti Speciósi Mónachi, cujus ánimam (ut scribit beátus 
 Gregórius Papa) germánus ejus deférri vidit in cælum.
\switchcolumn
\selectlanguage{english}
At Capua, the monk St. Speciosus, whose soul his brother saw being 
 taken into heaven, as is recorded by Pope St. Gregory.
\switchcolumn*
\selectlanguage{latin}
Lutétiæ Parisiórum sanctæ Ludovícæ de Marillac, 
 víduæ Le Gras, Societátis Puellárum a Caritáte una cum sancto Vincéntio a 
 Paulo Fundatrícis, egénis sublevándis addictíssimæ, quam Pius Papa Undécimus 
 Sanctárum fastis accénsuit.
\switchcolumn
\selectlanguage{english}
At Paris, the birthday of St. Louise de Marillac, a widow of Le Gras, 
 co-founder with St. Vincent de Paul of the Society of the Daughters of 
 Charity. Outstanding for her virtues and miracles, her name was 
 inscribed on the roll of the saints by Pope Pius XI.
\switchcolumn*
\selectlanguage{latin}
\end{paracol}


% ---- martyrology/mart03/mart0316.htm
\needspace{10\baselineskip}
\begin{paracol}{2}
\selectlanguage{latin}
\begin{center}{\color{gregoriocolor} Décimo séptimo Kaléndas Aprilis. 
 Luna\dots\ }\end{center}
\switchcolumn
\selectlanguage{english}
\begin{center}{\color{gregoriocolor} The Sixteenth Day of March. 
 The\dots\ Day of the Moon.}\end{center}
\end{paracol}

\noindent\begin{tabularx}{\linewidth}{*{19}{>{\centering\arraybackslash}X}}
 \textcolor{gregoriocolor}{a} & \textcolor{gregoriocolor}{b} & \textcolor{gregoriocolor}{c} & \textcolor{gregoriocolor}{d} & \textcolor{gregoriocolor}{e} & \textcolor{gregoriocolor}{f} & \textcolor{gregoriocolor}{g} & \textcolor{gregoriocolor}{h} & \textcolor{gregoriocolor}{i} & \textcolor{gregoriocolor}{k} & \textcolor{gregoriocolor}{l} & \textcolor{gregoriocolor}{m} & \textcolor{gregoriocolor}{n} & \textcolor{gregoriocolor}{p} & \textcolor{gregoriocolor}{q} & \textcolor{gregoriocolor}{r} & \textcolor{gregoriocolor}{s} & \textcolor{gregoriocolor}{t} & \textcolor{gregoriocolor}{u} \\
 17 & 18 & 19 & 20 & 21 & 22 & 23 & 24 & 25 & 26 & 27 & 28 & 29 & 30 & 1 & 2 & 3 & 4 & 5 \\
\end{tabularx}
\vspace{0.5\baselineskip}
\noindent\begin{tabularx}{\linewidth}{*{12}{>{\centering\arraybackslash}X}}
 \textcolor{gregoriocolor}{A} & \textcolor{gregoriocolor}{B} & \textcolor{gregoriocolor}{C} & \textcolor{gregoriocolor}{D} & \textcolor{gregoriocolor}{E} & F & \textcolor{gregoriocolor}{F} & \textcolor{gregoriocolor}{G} & \textcolor{gregoriocolor}{H} & \textcolor{gregoriocolor}{M} & \textcolor{gregoriocolor}{N} & \textcolor{gregoriocolor}{P} \\
 6 & 7 & 8 & 9 & 10 & 11 & 11 & 12 & 13 & 14 & 15 & 16 \\
\end{tabularx}

\begin{paracol}{2}
\selectlanguage{latin}
\lettrine[lines=2]{R}{omæ} pássio sancti Cyríaci Diáconi, qui, post 
 longam cárceris maceratiónem, liquáta pice perfúsus et in catásta exténsus, 
 attráctus étiam nervis et fústibus cæsus, ad últimum, cum Largo et Smarágdo 
 et áliis vigínti, jubénte Maximiáno, cápite truncátus est. Sanctórum 
 vero Cyríaci, Largi et Smarágdi festívitas sexto Idus Augústi recólitur, quo 
 die a beáto Marcéllo Papa córpora eorúndem vigínti trium Mártyrum leváta 
 sunt ac venerabíliter tumuláta.
\switchcolumn
\selectlanguage{english}
\lettrine[lines=2]{A}{t} Rome the martyrdom of the deacon St. Cyriacus, who, after a long 
 imprisonment, had melted pitch poured over him, was stretched on the rack, 
 had his limbs pulled with ropes, was beaten with clubs, and finally was 
 beheaded by order of Maximian, together with Largus, Smaragdus, and twenty 
 others. Their feast, however, is kept on the 8th of August, the day on 
 which these twenty-three martyrs were exhumed by blessed Pope Marcellus and 
 reverently entombed.
\switchcolumn*
\selectlanguage{latin}
Aquiléjæ natális beáti Hilárii Epíscopi, et 
 Tatiáni Diáconi, qui, sub Numeriáno Imperatóre et Berónio Præside, post 
 equúleum atque ália torménta, una cum Felíce, Largo et Dionysio, martyrium 
 terminárunt.
\switchcolumn
\selectlanguage{english}
At Aquileia, in the time of Emperor Numerian and the governor Beronius, the 
 birthday of the holy bishop Hilary, and the deacon Tatian, who were martyred 
 with Felix, Largus, and Denis, after being subjected to the rack and other 
 tortures.
\switchcolumn*
\selectlanguage{latin}
In Lycaónia sancti Papæ Mártyris, qui, ob 
 Christi fidem, verbéribus cæsus, úngulis férreis lacerátus, clavátis cálceis 
 incédere jussus est; deínde árbori alligátus, eándem árborem, migrans ad 
 Dóminum, ex stérili réddidit fructuósam.
\switchcolumn
\selectlanguage{english}
In Lycaonia, the holy martyr Papas, who was scourged for the Christian 
 faith, had his flesh torn with iron hooks, and was compelled to walk with 
 shoes pierced with nails, and was finally bound to a barren tree. In 
 leaving this world to go to God, he rendered this same tree fruitful.
\switchcolumn*
\selectlanguage{latin}
Anazárbi, in Cilícia, sancti Juliáni Mártyris, qui, sub Marciáno Præside, 
 diutíssime cruciátus, demum, in sacco una cum serpéntibus inclúsus, in mare 
 demérsus est.
\switchcolumn
\selectlanguage{english}
At Anazarbum in Cilicia, under the governor Marcian, the martyr St. Julian, 
 who was a long time tortured, then put into a sack with serpents, and cast 
 into the sea.
\switchcolumn*
\selectlanguage{latin}
In dicióne Canadénsi sanctórum Mártyrum Joánnis de Brébeuf, 
 Gabriélis Lalemant, Antónii Daniel, Cároli Garnier et Natális Chabanel, 
 Presbyterórum Societátis Jesu, qui in Hurónica Missióne, hac aliísque diébus, 
 post multos labóres et sævíssimos cruciátus, mortem pro Christo fórtiter 
 obiérunt.
\switchcolumn
\selectlanguage{english}
In the territory of Canada, Saints John de Brébeuf, Gabriel Lalemant, 
 Anthony Daniel, Charles Garnier, and Noel Chabanel, priests of the Society 
 of Jesus, who in the mission of the Hurons, on this and other days, after 
 many labours and most cruel torments, bravely underwent death for Christ.
\switchcolumn*
\selectlanguage{latin}
Ravénnæ sancti Agapíti, Epíscopi et Confessóris.
\switchcolumn
\selectlanguage{english}
At Ravenna, St. Agapitus, bishop and confessor.
\switchcolumn*
\selectlanguage{latin}
Colóniæ Agrippínæ sancti Heribérti Epíscopi, 
 sanctitáte célebris.
\switchcolumn
\selectlanguage{english}
At Cologne, St. Heribert, bishop, celebrated for sanctity.
\switchcolumn*
\selectlanguage{latin}
Arvérnis, in Gállia, deposítio sancti Patrícii Epíscopi.
\switchcolumn
\selectlanguage{english}
In Auvergne, the death of St. Patrick, bishop.
\switchcolumn*
\selectlanguage{latin}
In Syria sancti Abrahæ Eremítæ, cujus res 
 gestas beátus Ephræm Diáconus conscrípsit.
\switchcolumn
\selectlanguage{english}
In Syria, St. Abraham, hermit, whose life has been written by the blessed 
 deacon Ephrem.
\switchcolumn*
\selectlanguage{latin}
\end{paracol}


% ---- martyrology/mart03/mart0317.htm
\needspace{10\baselineskip}
\begin{paracol}{2}
\selectlanguage{latin}
\begin{center}{\color{gregoriocolor} Sextodécimo Kaléndas Aprilis. 
 Luna\dots\ }\end{center}
\switchcolumn
\selectlanguage{english}
\begin{center}{\color{gregoriocolor} The Seventeenth Day of March. 
 The\dots\ Day of the Moon.}\end{center}
\end{paracol}

\noindent\begin{tabularx}{\linewidth}{*{19}{>{\centering\arraybackslash}X}}
 \textcolor{gregoriocolor}{a} & \textcolor{gregoriocolor}{b} & \textcolor{gregoriocolor}{c} & \textcolor{gregoriocolor}{d} & \textcolor{gregoriocolor}{e} & \textcolor{gregoriocolor}{f} & \textcolor{gregoriocolor}{g} & \textcolor{gregoriocolor}{h} & \textcolor{gregoriocolor}{i} & \textcolor{gregoriocolor}{k} & \textcolor{gregoriocolor}{l} & \textcolor{gregoriocolor}{m} & \textcolor{gregoriocolor}{n} & \textcolor{gregoriocolor}{p} & \textcolor{gregoriocolor}{q} & \textcolor{gregoriocolor}{r} & \textcolor{gregoriocolor}{s} & \textcolor{gregoriocolor}{t} & \textcolor{gregoriocolor}{u} \\
 18 & 19 & 20 & 21 & 22 & 23 & 24 & 25 & 26 & 27 & 28 & 29 & 30 & 1 & 2 & 3 & 4 & 5 & 6 \\
\end{tabularx}
\vspace{0.5\baselineskip}
\noindent\begin{tabularx}{\linewidth}{*{12}{>{\centering\arraybackslash}X}}
 \textcolor{gregoriocolor}{A} & \textcolor{gregoriocolor}{B} & \textcolor{gregoriocolor}{C} & \textcolor{gregoriocolor}{D} & \textcolor{gregoriocolor}{E} & F & \textcolor{gregoriocolor}{F} & \textcolor{gregoriocolor}{G} & \textcolor{gregoriocolor}{H} & \textcolor{gregoriocolor}{M} & \textcolor{gregoriocolor}{N} & \textcolor{gregoriocolor}{P} \\
 7 & 8 & 9 & 10 & 11 & 12 & 12 & 13 & 14 & 15 & 16 & 17 \\
\end{tabularx}

\begin{paracol}{2}
\selectlanguage{latin}
\lettrine[lines=2]{A}{pud} civitátem Dunum, in Hibérnia, natális sancti Patrícii, Epíscopi et 
 Confessóris, qui primus in ea ínsula Christum evangelizávit, et máximis 
 miráculis et virtútibus cláruit.
\switchcolumn
\selectlanguage{english}
\lettrine[lines=2]{A}{t} Downpatrick in Ireland, the birthday of St. Patrick, bishop and 
 confessor, who was the first to preach Christ in that country, and who 
 became illustrious by great miracles and virtues.
\switchcolumn*
\selectlanguage{latin}
Hierosólymis sancti Joseph ab Arimathǽa, qui 
 nóbilis Decúrio et discípulus Dómini éxstitit; atque ipsíus Dómini corpus, 
 de cruce depósitum, in monuménto suo novo sepelívit.
\switchcolumn
\selectlanguage{english}
At Jerusalem, St. Joseph of Arimathea, noble senator and disciple of our 
 Lord,. who took his Body down from the Cross and buried it in his own new 
 sepulchre.
\switchcolumn*
\selectlanguage{latin}
Romæ sanctórum Alexándri et Theodóri Mártyrum.
\switchcolumn
\selectlanguage{english}
At Rome, the Saints Alexander and Theodore, martyrs.
\switchcolumn*
\selectlanguage{latin}
Alexandríæ commemorátio plurimórum sanctórum 
 Mártyrum, qui a Serápidis cultóribus comprehénsi, et, cum adoráre idólum 
 constánter renuíssent, sævíssime cæsi sunt, témpore Theodósii Imperatóris; 
 qui mox rescríptum dedit, ut Serápidis templum destruerétur.
\switchcolumn
\selectlanguage{english}
At Alexandria, the commemoration of many holy martyrs, who, being seized by 
 the worshippers of Serapis, and refusing constantly to adore that idol, were 
 cruelly murdered. Emperor Theodosius, who issued the order, afterwards 
 commanded that the temple of Serapis should be destroyed.
\switchcolumn*
\selectlanguage{latin}
Constantinópoli sancti Pauli Mártyris, qui, sub Constantíno Coprónymo, 
 cum sanctárum Imáginum cultum defénderet, igne combústus est.
\switchcolumn
\selectlanguage{english}
At Constantinople, St. Paul, martyr, who was burned alive by Constantine 
 Copronymus, for defending the veneration of sacred images.
\switchcolumn*
\selectlanguage{latin}
Cabillóne, in Gálliis, sancti Agrícolæ Epíscopi.
\switchcolumn
\selectlanguage{english}
At Chalons in France, St. Agricola, bishop.
\switchcolumn*
\selectlanguage{latin}
Nivigéllæ, in Brabántia, sanctæ Gertrúdis 
 Vírginis, quæ claríssimo génere orta, despíciens mundum et toto vitæ suæ 
 cursu in ómnibus sanctitátis offíciis se exércens, Christum sponsum in cælis 
 habére méruit.
\switchcolumn
\selectlanguage{english}
At Nivelle in Brabant, St. Gertrude, a virgin of noble birth. Because 
 she despised the world, and during her whole life practised all kinds of 
 good works, she deserved to have Christ for her spouse in heaven.
\switchcolumn*
\selectlanguage{latin}
\end{paracol}


% ---- martyrology/mart03/mart0318.htm
\needspace{10\baselineskip}
\begin{paracol}{2}
\selectlanguage{latin}
\begin{center}{\color{gregoriocolor} Quintodécimo Kaléndas Aprilis. 
 Luna\dots\ }\end{center}
\switchcolumn
\selectlanguage{english}
\begin{center}{\color{gregoriocolor} The Eighteenth Day of March. 
 The\dots\ Day of the Moon.}\end{center}
\end{paracol}

\noindent\begin{tabularx}{\linewidth}{*{19}{>{\centering\arraybackslash}X}}
 \textcolor{gregoriocolor}{a} & \textcolor{gregoriocolor}{b} & \textcolor{gregoriocolor}{c} & \textcolor{gregoriocolor}{d} & \textcolor{gregoriocolor}{e} & \textcolor{gregoriocolor}{f} & \textcolor{gregoriocolor}{g} & \textcolor{gregoriocolor}{h} & \textcolor{gregoriocolor}{i} & \textcolor{gregoriocolor}{k} & \textcolor{gregoriocolor}{l} & \textcolor{gregoriocolor}{m} & \textcolor{gregoriocolor}{n} & \textcolor{gregoriocolor}{p} & \textcolor{gregoriocolor}{q} & \textcolor{gregoriocolor}{r} & \textcolor{gregoriocolor}{s} & \textcolor{gregoriocolor}{t} & \textcolor{gregoriocolor}{u} \\
 19 & 20 & 21 & 22 & 23 & 24 & 25 & 26 & 27 & 28 & 29 & 30 & 1 & 2 & 3 & 4 & 5 & 6 & 7 \\
\end{tabularx}
\vspace{0.5\baselineskip}
\noindent\begin{tabularx}{\linewidth}{*{12}{>{\centering\arraybackslash}X}}
 \textcolor{gregoriocolor}{A} & \textcolor{gregoriocolor}{B} & \textcolor{gregoriocolor}{C} & \textcolor{gregoriocolor}{D} & \textcolor{gregoriocolor}{E} & F & \textcolor{gregoriocolor}{F} & \textcolor{gregoriocolor}{G} & \textcolor{gregoriocolor}{H} & \textcolor{gregoriocolor}{M} & \textcolor{gregoriocolor}{N} & \textcolor{gregoriocolor}{P} \\
 8 & 9 & 10 & 11 & 12 & 13 & 13 & 14 & 15 & 16 & 17 & 18 \\
\end{tabularx}

\begin{paracol}{2}
\selectlanguage{latin}
\lettrine[lines=2]{H}{ierosólymis} sancti Cyrílli Epíscopi, Confessóris et Ecclésiæ 
 Doctóris; qui, ab Ariánis multas pro fidei causa perpéssus injúrias et ex 
 Ecclésia sua sæpe depúlsus, tandem, sanctitátis glória clarus, in pace 
 quiévit. Ipsíus porro intemerátam fidem prima Constantinopolitána 
 Synodus œcuménica, sancto Dámaso Papæ scribens, præcláro testimónio 
 commendávit.
\switchcolumn
\selectlanguage{english}
\lettrine[lines=2]{A}{t} Jerusalem, St. Cyril, bishop, who suffered many injuries from the Arians 
 for the faith. Often exiled from his church, he at length rested in 
 peace with a great reputation for sanctity. A magnificent testimony of 
 the purity of his faith is given by the first ecumenical Council of 
 Constantinople in a letter to Pope Damasus.
\switchcolumn*
\selectlanguage{latin}
Cæsaréæ, in Palæstína, natális beáti Alexándri 
 Epíscopi, qui de Cappadócia, ex própria civitáte, ubi erat Epíscopus, 
 sanctórum locórum desidério Hierosólymam pétiit; atque ibi, cum a Narcísso, 
 ejúsdem urbis Epíscopo, jam sene, illa regerétur Ecclésia, ipsíus 
 gubernácula, divína edóctus revelatióne, suscépit. Póstmodum vero, in 
 persecutióne Décii, cum jam longævæ ætátis veneránda canítie præfúlgeret, 
 ductus est Cæsaréam, et clausus in cárcere, ob confessiónem Christi, 
 martyrium complévit.
\switchcolumn
\selectlanguage{english}
At Caesarea in Palestine, the birthday of the blessed Bishop Alexander, who, 
 from his own city in Cappadocia, where he was bishop, coming to Jerusalem to 
 visit the holy places, took upon himself, by divine revelation, the 
 government of that church in place of the aged Narcissus. Sometime 
 afterwards, when he had become venerable by his age and gray hair, he was 
 led to Caesarea and shut up in prison, where he completed his martyrdom for 
 the confession of Christ during the persecution of Decius.
\switchcolumn*
\selectlanguage{latin}
Augústæ sancti Narcíssi Epíscopi, qui primus in 
 Rhǽtia Evangélium prædicávit; deínde in Hispániam profectus est, et, cum 
 Gerúndæ multos ad Christi fidem convertísset, ibídem, in persecutióne 
 Diocletiáni Imperatóris, una cum Felíce Diácono, martyrii palmam accépit.
\switchcolumn
\selectlanguage{english}
At Augsburg, St. Narcissus, bishop, who was the first to preach the Gospel 
 in the Tyrol. Afterwards, setting out for Spain, he converted many to 
 the faith of Christ at Gerona, and there, along with the deacon Felix, he 
 received the palm of martyrdom during the persecution of Diocletian.
\switchcolumn*
\selectlanguage{latin}
Nicomedíæ sanctórum decem míllium Mártyrum, 
 qui, pro Christi confessióne, gládio percússi sunt.
\switchcolumn
\selectlanguage{english}
At Nicomedia, ten thousand holy martyrs, who were put to the sword for the 
 confession of Christ.
\switchcolumn*
\selectlanguage{latin}
Ibídem sanctórum Mártyrum Tróphimi et Eucárpii.
\switchcolumn
\selectlanguage{english}
In the same place, the holy martyrs Trophimus and Eucarpius.
\switchcolumn*
\selectlanguage{latin}
In Británnia sancti Eduárdi Regis, qui, dolis novércæ 
 necátus, multis miráculis cláruit.
\switchcolumn
\selectlanguage{english}
In England, St. Edward, king, who was assassinated by order of his 
 treacherous stepmother, and became celebrated for many miracles.
\switchcolumn*
\selectlanguage{latin}
Lucæ, in Túscia, natális sancti Frigdiáni 
 Epíscopi, virtúte miraculórum illústris.
\switchcolumn
\selectlanguage{english}
At Lucca in Tuscany, the birthday of the holy bishop Fridian, who was 
 illustrious by the power of working miracles.
\switchcolumn*
\selectlanguage{latin}
Mántuæ sancti Ansélmi, Epíscopi Lucénsis et 
 Confessóris.
\switchcolumn
\selectlanguage{english}
At Mantua, St. Anselm, bishop and confessor.
\switchcolumn*
\selectlanguage{latin}
Cárali, in Sardínia, sancti Salvatóris ab Horta 
 Confessóris, ex Ordine Fratrum Minórum, qui virtútibus et singulári 
 miraculórum dono cláruit, et a Pio Papa Undécimo inter sanctos Cǽlites 
 adnumerátus est.
\switchcolumn
\selectlanguage{english}
At Cagliari in Sardinia, St. Salvatore of Orte, confessor, a member of the 
 Order of Friars Minor, who was numbered among the heavenly saints by Pope 
 Pius XI, because he was graced with every virtue and had been given by God 
 the gift of performing outstanding miracles.
\switchcolumn*
\selectlanguage{latin}
\end{paracol}


% ---- martyrology/mart03/mart0319.htm
\needspace{10\baselineskip}
\begin{paracol}{2}
\selectlanguage{latin}
\begin{center}{\color{gregoriocolor} Quartodécimo Kaléndas Aprilis. 
 Luna\dots\ }\end{center}
\switchcolumn
\selectlanguage{english}
\begin{center}{\color{gregoriocolor} The Nineteenth Day of March. 
 The\dots\ Day of the Moon.}\end{center}
\end{paracol}

\noindent\begin{tabularx}{\linewidth}{*{19}{>{\centering\arraybackslash}X}}
 \textcolor{gregoriocolor}{a} & \textcolor{gregoriocolor}{b} & \textcolor{gregoriocolor}{c} & \textcolor{gregoriocolor}{d} & \textcolor{gregoriocolor}{e} & \textcolor{gregoriocolor}{f} & \textcolor{gregoriocolor}{g} & \textcolor{gregoriocolor}{h} & \textcolor{gregoriocolor}{i} & \textcolor{gregoriocolor}{k} & \textcolor{gregoriocolor}{l} & \textcolor{gregoriocolor}{m} & \textcolor{gregoriocolor}{n} & \textcolor{gregoriocolor}{p} & \textcolor{gregoriocolor}{q} & \textcolor{gregoriocolor}{r} & \textcolor{gregoriocolor}{s} & \textcolor{gregoriocolor}{t} & \textcolor{gregoriocolor}{u} \\
 20 & 21 & 22 & 23 & 24 & 25 & 26 & 27 & 28 & 29 & 30 & 1 & 2 & 3 & 4 & 5 & 6 & 7 & 8 \\
\end{tabularx}
\vspace{0.5\baselineskip}
\noindent\begin{tabularx}{\linewidth}{*{12}{>{\centering\arraybackslash}X}}
 \textcolor{gregoriocolor}{A} & \textcolor{gregoriocolor}{B} & \textcolor{gregoriocolor}{C} & \textcolor{gregoriocolor}{D} & \textcolor{gregoriocolor}{E} & F & \textcolor{gregoriocolor}{F} & \textcolor{gregoriocolor}{G} & \textcolor{gregoriocolor}{H} & \textcolor{gregoriocolor}{M} & \textcolor{gregoriocolor}{N} & \textcolor{gregoriocolor}{P} \\
 9 & 10 & 11 & 12 & 13 & 14 & 14 & 15 & 16 & 17 & 18 & 19 \\
\end{tabularx}

\begin{paracol}{2}
\selectlanguage{latin}
\lettrine[lines=2]{I}{n} Judæa natális sancti Joseph, Sponsi 
 beatíssimæ Vírginis Maríæ, Confessóris et Ecclésiæ universális Patróni; quem Pius Nonus, Póntifex Máximus, 
 votis et précibus ánnuens totíus cathólici Orbis, universális Ecclésiæ 
 Patrónum declarávit.
\switchcolumn
\selectlanguage{english}
\lettrine[lines=2]{I}{n} Judea, the birthday of St. Joseph, spouse of the Most Blessed Virgin 
 Mary, Confessor and Patron of the Universal Church. Pope Pius IX, yielding to the desires and prayers of the whole 
 Catholic world, declared him Patron of the Universal Church.
\switchcolumn*
\selectlanguage{latin}
Surrénti sanctórum Mártyrum Quincti, Quinctíllæ, 
 Quartíllæ et Marci, cum áliis novem.
\switchcolumn
\selectlanguage{english}
At Sorrento, the holy martyrs Quinctus, Quinctilla, Quartilla, Mark, and 
 nine others.
\switchcolumn*
\selectlanguage{latin}
Nicomedíæ sancti Panchárii Románi, qui, sub 
 Diocletiáno Imperatóre, in hujus grátiam Christum pro diis inánibus ejurávit, 
 sed, matre ac soróre instántibus, ad veram fidem mox redívit, et ob immótam 
 in ea constántiam, nervis cæsus, et cápite truncátus, martyrii corónam 
 accépit.
\switchcolumn
\selectlanguage{english}
At Nicomedia, St. Pancharius, a Roman, who apostatized for the sake of 
 Emperor Diocletian, but by the persuasion of his mother and sister 
 immediately returned to the true faith. Because of his subsequent 
 constancy in it, he was beaten with clubs and beheaded, obtaining thus the 
 crown of martyrdom.
\switchcolumn*
\selectlanguage{latin}
Eódem die sanctórum Apollónii et Leóntii Episcopórum.
\switchcolumn
\selectlanguage{english}
The same day, the holy Bishops Apollonius and Leontius.
\switchcolumn*
\selectlanguage{latin}
Gandávi, in Flándria, sanctórum Landoáldi, Presbyteri Románi, et Amántii 
 Diáconi; qui, a sancto Martíno Papa ad prædicándum 
 Evangélium missi, ambo apostólicum sibi commíssum opus fidéliter implevérunt, 
 ac multis post óbitum sunt illustráti miráculis.
\switchcolumn
\selectlanguage{english}
At Ghent in Flanders, Saints Landoald, a Roman priest, and the deacon 
 Amantius, who were sent to preach the Gospel by Pope St. Martin. They 
 faithfully fulfilled this apostolic appointment, and after their deaths 
 became renowned for their miracles.
\switchcolumn*
\selectlanguage{latin}
Apud Pinnénsem civitátem natális beáti Joánnis, magnæ 
 sanctitátis viri; qui de Syria ad Itáliam venit, atque ibi, constrúcto 
 monastério, multórum servórum Dei per quátuor et quadragínta annos Pater éxstitit, et, clarus virtútibus, in pace quiévit.
\switchcolumn
\selectlanguage{english}
In the city of Pinna, the birthday of blessed John, a man of great sanctity, 
 who came from Syria into Italy, and there founded a monastery. After 
 being the spiritual guide for many of God's servants for forty-four years, 
 he rested in peace.
\switchcolumn*
\selectlanguage{latin}
\end{paracol}


% ---- martyrology/mart03/mart0320.htm
\needspace{10\baselineskip}
\begin{paracol}{2}
\selectlanguage{latin}
\begin{center}{\color{gregoriocolor} Tertiodécimo Kaléndas Aprilis. 
 Luna\dots\ }\end{center}
\switchcolumn
\selectlanguage{english}
\begin{center}{\color{gregoriocolor} The Twentieth Day of March. 
 The\dots\ Day of the Moon.}\end{center}
\end{paracol}

\noindent\begin{tabularx}{\linewidth}{*{19}{>{\centering\arraybackslash}X}}
 \textcolor{gregoriocolor}{a} & \textcolor{gregoriocolor}{b} & \textcolor{gregoriocolor}{c} & \textcolor{gregoriocolor}{d} & \textcolor{gregoriocolor}{e} & \textcolor{gregoriocolor}{f} & \textcolor{gregoriocolor}{g} & \textcolor{gregoriocolor}{h} & \textcolor{gregoriocolor}{i} & \textcolor{gregoriocolor}{k} & \textcolor{gregoriocolor}{l} & \textcolor{gregoriocolor}{m} & \textcolor{gregoriocolor}{n} & \textcolor{gregoriocolor}{p} & \textcolor{gregoriocolor}{q} & \textcolor{gregoriocolor}{r} & \textcolor{gregoriocolor}{s} & \textcolor{gregoriocolor}{t} & \textcolor{gregoriocolor}{u} \\
 21 & 22 & 23 & 24 & 25 & 26 & 27 & 28 & 29 & 30 & 1 & 2 & 3 & 4 & 5 & 6 & 7 & 8 & 9 \\
\end{tabularx}
\vspace{0.5\baselineskip}
\noindent\begin{tabularx}{\linewidth}{*{12}{>{\centering\arraybackslash}X}}
 \textcolor{gregoriocolor}{A} & \textcolor{gregoriocolor}{B} & \textcolor{gregoriocolor}{C} & \textcolor{gregoriocolor}{D} & \textcolor{gregoriocolor}{E} & F & \textcolor{gregoriocolor}{F} & \textcolor{gregoriocolor}{G} & \textcolor{gregoriocolor}{H} & \textcolor{gregoriocolor}{M} & \textcolor{gregoriocolor}{N} & \textcolor{gregoriocolor}{P} \\
 10 & 11 & 12 & 13 & 14 & 15 & 15 & 16 & 17 & 18 & 19 & 20 \\
\end{tabularx}

\begin{paracol}{2}
\selectlanguage{latin}
\lettrine[lines=2]{I}{n} Judǽa natális sancti Jóachim, patris 
 immaculátæ Vírginis Genitrícis Dei Maríæ, Confessóris. Ipsíus tamen 
 festum agitur décimo séptimo Kaléndas Septémbris.
\switchcolumn
\selectlanguage{english}
\lettrine[lines=2]{I}{n} Judea, St. Joachim, the father of the Most Blessed Virgin Mary, Mother of 
 God. His feast day is on the 16th of August.
\switchcolumn*
\selectlanguage{latin}
In Asia item natális sancti Archíppi, qui beáti Pauli Apóstoli 
 éxstitit 
 commílito, et cujus ipse in Epístola ad Philémonem 
 et ad Colossénses méminit.
\switchcolumn
\selectlanguage{english}
In Asia, the birthday of St. Archippus, fellow-labourer of the apostle St. 
 Paul, who is mentioned by him in his epistles to Philemon and the 
 Colossians.
\switchcolumn*
\selectlanguage{latin}
In Syria sanctórum Mártyrum Pauli, Cyrílli, Eugénii et aliórum quátuor.
\switchcolumn
\selectlanguage{english}
In Syria, the holy martyrs Paul, Cyril, Eugene, and four others.
\switchcolumn*
\selectlanguage{latin}
Eódem die sanctórum Photínæ Samaritánæ, Joseph 
 et Victóris filiórum, itémque Sebastiáni Ducis, Anatólii, Phótii, Phótidis, 
 Parascéves et Cyríacæ germanárum; qui omnes, Christum conféssi, martyrium 
 sunt assecúti.
\switchcolumn
\selectlanguage{english}
On the same day, the Saints Photina, a Samaritan, and her sons Joseph and 
 Victor; also, Sebastian, a military officer, Anatolius, and Photius; 
 Photides, Parasceves, and Cyriaca, sisters, all of whom were put to death 
 for the confession of the faith.
\switchcolumn*
\selectlanguage{latin}
Amísi, in Paphlagónia, sanctárum septem mulíerum, scílicet Alexándræ, 
 Cláudiæ, Euphrásiæ, Matrónæ, Juliánæ, Euphémiæ et Theodósiæ; quæ in fidei 
 confessióne sunt cæsæ, eásque secútæ sunt Derphúta et soror ipsíus.
\switchcolumn
\selectlanguage{english}
At Amisus in Paphlagonia, seven holy women, Alexandra, Claudia, Euphrasia, 
 Matrona, Juliana, Euphemia, and Theodosia, who were put to death for the 
 confession of the faith. They were followed by Dephuta and her sister.
\switchcolumn*
\selectlanguage{latin}
Apollóniæ sancti Nicétæ Epíscopi, qui, pro 
 sanctárum Imáginum cultu ejéctus in exsílium, illic réddidit spíritum.
\switchcolumn
\selectlanguage{english}
At Apollonia, Bishop St. Nicetas, who died in exile where he had been sent 
 for upholding the veneration of sacred images.
\switchcolumn*
\selectlanguage{latin}
In monastério Fontanéllæ, in Gállia, sancti 
 Wulfránni, Epíscopi Senonénsis, qui, relícto Episcopátu, ibídem, clarus 
 miráculis, decéssit e vita.
\switchcolumn
\selectlanguage{english}
In the monastery of Fontanelle in France, St. Wulfram, bishop of Sens, who 
 resigned his bishopric, and after having performed miracles, departed out of 
 this life.
\switchcolumn*
\selectlanguage{latin}
In Británnia deposítio sancti Cuthbérti, Epíscopi Lindisfarnénsis, qui, a 
 puerítia ad óbitum usque, sanctis opéribus et miraculórum signis effúlsit.
\switchcolumn
\selectlanguage{english}
In England, the death of St. Cuthbert, bishop of Lindisfarne, who from his 
 childhood until his death was renowned for good works and miracles.
\switchcolumn*
\selectlanguage{latin}
Senis, in Túscia, Beáti Ambrósii, ex Ordine Prædicatórum, 
 sanctitáte, prædicatióne et miráculis clari.
\switchcolumn
\selectlanguage{english}
At Sienna in Tuscany, blessed Ambrose of the Order of Preachers, celebrated 
 for sanctity, eloquence, and miracles.
\switchcolumn*
\selectlanguage{latin}
\end{paracol}


% ---- martyrology/mart03/mart0321.htm
\needspace{10\baselineskip}
\begin{paracol}{2}
\selectlanguage{latin}
\begin{center}{\color{gregoriocolor} Duodécimo Kaléndas Aprílis. 
 Luna\dots\ }\end{center}
\switchcolumn
\selectlanguage{english}
\begin{center}{\color{gregoriocolor} The Twenty-First Day of March. 
 The\dots\ Day of the Moon.}\end{center}
\end{paracol}

\noindent\begin{tabularx}{\linewidth}{*{19}{>{\centering\arraybackslash}X}}
 \textcolor{gregoriocolor}{a} & \textcolor{gregoriocolor}{b} & \textcolor{gregoriocolor}{c} & \textcolor{gregoriocolor}{d} & \textcolor{gregoriocolor}{e} & \textcolor{gregoriocolor}{f} & \textcolor{gregoriocolor}{g} & \textcolor{gregoriocolor}{h} & \textcolor{gregoriocolor}{i} & \textcolor{gregoriocolor}{k} & \textcolor{gregoriocolor}{l} & \textcolor{gregoriocolor}{m} & \textcolor{gregoriocolor}{n} & \textcolor{gregoriocolor}{p} & \textcolor{gregoriocolor}{q} & \textcolor{gregoriocolor}{r} & \textcolor{gregoriocolor}{s} & \textcolor{gregoriocolor}{t} & \textcolor{gregoriocolor}{u} \\
 22 & 23 & 24 & 25 & 26 & 27 & 28 & 29 & 30 & 1 & 2 & 3 & 4 & 5 & 6 & 7 & 8 & 9 & 10 \\
\end{tabularx}
\vspace{0.5\baselineskip}
\noindent\begin{tabularx}{\linewidth}{*{12}{>{\centering\arraybackslash}X}}
 \textcolor{gregoriocolor}{A} & \textcolor{gregoriocolor}{B} & \textcolor{gregoriocolor}{C} & \textcolor{gregoriocolor}{D} & \textcolor{gregoriocolor}{E} & F & \textcolor{gregoriocolor}{F} & \textcolor{gregoriocolor}{G} & \textcolor{gregoriocolor}{H} & \textcolor{gregoriocolor}{M} & \textcolor{gregoriocolor}{N} & \textcolor{gregoriocolor}{P} \\
 11 & 12 & 13 & 14 & 15 & 16 & 16 & 17 & 18 & 19 & 20 & 21 \\
\end{tabularx}

\begin{paracol}{2}
\selectlanguage{latin}
\lettrine[lines=2]{I}{n} monte Cassíno natális sancti Benedícti Abbátis, qui in Occidénte fere 
 collápsam Monachórum disciplínam restítuit ac mirífice propagávit; cujus 
 vitam, virtútibus et miráculis gloriósam, beátus Gregórius Papa conscrípsit.
\switchcolumn
\selectlanguage{english}
\lettrine[lines=2]{A}{t} Monte Cassino, the birthday of the holy abbot St. Benedict, who restored 
 and wonderfully extended the monastic discipline in the West, where it had 
 almost been destroyed. His life, brilliant in virtues and miracles, 
 was written by Pope St. Gregory.
\switchcolumn*
\selectlanguage{latin}
Cátanæ, in Sicília, sancti Birílli, qui, a 
 beáto Petro ordinátus Epíscopus, ibídem, cum multos Gentílium convertísset 
 ad fidem, in última senectúte quiévit in pace.
\switchcolumn
\selectlanguage{english}
At Catania, St. Birillus, who was consecrated bishop by St. Peter. 
 After converting many gentiles to the faith, he died in extreme old age.
\switchcolumn*
\selectlanguage{latin}
Alexandríæ commemorátio sanctórum Mártyrum, 
 qui, sub Constántio Imperatóre et Præfécto Philágrio, irruéntibus Ariánis et 
 Gentílibus in Ecclésias, in die Parascéves cæsi sunt.
\switchcolumn
\selectlanguage{english}
At Alexandria, under Emperor Constantine and the governor Philagrius, the 
 commemoration of the holy martyrs who were murdered by the Arians and the 
 heathens, being attacked by them while they were in church on Good Friday.
\switchcolumn*
\selectlanguage{latin}
Eódem die sanctórum Mártyrum Philémonis et 
 Domníni.
\switchcolumn
\selectlanguage{english}
On the same day, the holy martyrs Philemon and Domninus.
\switchcolumn*
\selectlanguage{latin}
Alexandríæ beáti Serapiónis, Anachorétæ et 
 Epíscopi Thmúeos, magnárum virtútum viri; qui, Arianórum furóre in exsílium 
 actus, Conféssor migrávit ad Dóminum.
\switchcolumn
\selectlanguage{english}
At Alexandria, blessed Serapion, anchoret and bishop of Thmuis, a man of 
 great virtue, who was driven into exile by the enraged Arians, where he 
 departed to heaven.
\switchcolumn*
\selectlanguage{latin}
In território Lugdunénsi sancti Lupicíni Abbátis, cujus vita ob sanctitátis 
 et miraculórum glóriam fuit illústris.
\switchcolumn
\selectlanguage{english}
In the territory of Lyons, St. Lupicinus, abbot, whose life was brilliant 
 with the glory of holiness and miracles.
\switchcolumn*
\selectlanguage{latin}
In loco Ranft, prope Sachseln, in Helvétia, sancti Nicolái de Flüe, 
 patris famílias, dein Anachorétæ, arctíssima pæniténtia et mundi contémptu 
 insígnis, ab Helvétiis pater pátriæ appelláti, quem Pius Papa Duodécimus 
 Sanctórum fastis adscrípsit.
\switchcolumn
\selectlanguage{english}
In the village of Ranft, near Sachseln in Switzerland, St. Nicholas of Flue, 
 a family man who became an anchoret, famed for his most ardent penitence and 
 contempt for the world, and known by the Swiss as the father of the 
 fatherland. He was numbered among the saints by Pope Pius XII.
\switchcolumn*
\selectlanguage{latin}
\end{paracol}


% ---- martyrology/mart03/mart0322.htm
\needspace{10\baselineskip}
\begin{paracol}{2}
\selectlanguage{latin}
\begin{center}{\color{gregoriocolor} Undécimo Kaléndas Aprílis. 
 Luna\dots\ }\end{center}
\switchcolumn
\selectlanguage{english}
\begin{center}{\color{gregoriocolor} The Twenty-Second Day of March. 
 The\dots\ Day of the Moon.}\end{center}
\end{paracol}

\noindent\begin{tabularx}{\linewidth}{*{19}{>{\centering\arraybackslash}X}}
 \textcolor{gregoriocolor}{a} & \textcolor{gregoriocolor}{b} & \textcolor{gregoriocolor}{c} & \textcolor{gregoriocolor}{d} & \textcolor{gregoriocolor}{e} & \textcolor{gregoriocolor}{f} & \textcolor{gregoriocolor}{g} & \textcolor{gregoriocolor}{h} & \textcolor{gregoriocolor}{i} & \textcolor{gregoriocolor}{k} & \textcolor{gregoriocolor}{l} & \textcolor{gregoriocolor}{m} & \textcolor{gregoriocolor}{n} & \textcolor{gregoriocolor}{p} & \textcolor{gregoriocolor}{q} & \textcolor{gregoriocolor}{r} & \textcolor{gregoriocolor}{s} & \textcolor{gregoriocolor}{t} & \textcolor{gregoriocolor}{u} \\
 23 & 24 & 25 & 26 & 27 & 28 & 29 & 30 & 1 & 2 & 3 & 4 & 5 & 6 & 7 & 8 & 9 & 10 & 11 \\
\end{tabularx}
\vspace{0.5\baselineskip}
\noindent\begin{tabularx}{\linewidth}{*{12}{>{\centering\arraybackslash}X}}
 \textcolor{gregoriocolor}{A} & \textcolor{gregoriocolor}{B} & \textcolor{gregoriocolor}{C} & \textcolor{gregoriocolor}{D} & \textcolor{gregoriocolor}{E} & F & \textcolor{gregoriocolor}{F} & \textcolor{gregoriocolor}{G} & \textcolor{gregoriocolor}{H} & \textcolor{gregoriocolor}{M} & \textcolor{gregoriocolor}{N} & \textcolor{gregoriocolor}{P} \\
 12 & 13 & 14 & 15 & 16 & 17 & 17 & 18 & 19 & 20 & 21 & 22 \\
\end{tabularx}

\begin{paracol}{2}
\selectlanguage{latin}
\lettrine[lines=2]{N}{arbóne,} in Gállia, natális sancti Pauli Epíscopi, Apostolórum discípuli, 
 quem tradunt fuísse Sérgium Paulum Procónsulem. Hic, a beáto Apóstolo 
 Paulo baptizátus, et ab eo, cum in Hispániam pérgeret, apud Narbónem 
 relíctus, ibídem Episcopáli dignitáte donátus est; ibíque, prædicatiónis 
 offício non ségniter expléto, clarus miráculis migrávit in cælum.
\switchcolumn
\selectlanguage{english}
\lettrine[lines=2]{A}{t} Narbonne in France, the birthday of the bishop St. Paul, a disciple of 
 the apostles. He is said to have been the proconsul Sergius Paulus, 
 who was baptized by the blessed apostle Paul, and left at Narbonne, where he 
 was raised to the episcopal dignity when the apostle went to Spain. 
 Having zealously discharged the office of preaching and having performed 
 miracles, he departed to heaven.
\switchcolumn*
\selectlanguage{latin}
Tarracínæ, in Campánia, sancti Epaphrodíti, 
 Apostolórum discípuli, qui a beáto Petro Apóstolo Epíscopus illíus civitátis 
 ordinátus fuit.
\switchcolumn
\selectlanguage{english}
At Terracina, St. Epaphroditus, a disciple of the apostles, who was 
 consecrated bishop of that city by the blessed apostle Peter.
\switchcolumn*
\selectlanguage{latin}
Ancyræ, in Galátia, sancti Basilíi, Presbyteri 
 et Mártyris, qui sub Juliáno Apóstata, gravíssimis cruciátibus afféctus, 
 ánimam Deo réddidit.
\switchcolumn
\selectlanguage{english}
At Ancyra, under Julian the Apostate, St. Basil, priest and martyr, who gave 
 up his soul to God after having endured grievous torments.
\switchcolumn*
\selectlanguage{latin}
Carthágine sancti Octaviáni Archidiáconi, et multórum míllium Mártyrum, qui, 
 ob fidem cathólicam, a Wándalis cæsi sunt.
\switchcolumn
\selectlanguage{english}
At Carthage, the archdeacon St. Octavian, and many thousands of martyrs, who 
 were slain by the Vandals for the Catholic faith.
\switchcolumn*
\selectlanguage{latin}
In Africa sanctórum Mártyrum Saturníni et aliórum novem.
\switchcolumn
\selectlanguage{english}
In Africa, the holy martyrs Saturninus and nine others.
\switchcolumn*
\selectlanguage{latin}
In Galátia natális sanctárum Mártyrum Callinícæ 
 et Basilíssæ.
\switchcolumn
\selectlanguage{english}
In Galatia, the birthday of the holy martyrs Callinica and Basilissa.
\switchcolumn*
\selectlanguage{latin}
Romæ sancti Zacharíæ Papæ, qui Dei Ecclésiam 
 summa vigilántia gubernávit, et clarus méritis quiévit in pace.
\switchcolumn
\selectlanguage{english}
At Rome, the birthday of Pope St. Zachary, who governed the Church of God 
 with vigilance, and at last, renowned for miracles, rested in peace.
\switchcolumn*
\selectlanguage{latin}
Carthágine sancti Deográtias, Epíscopi Carthaginénsis, qui plúrimos, a 
 Wándalis captívos ex Urbe ductos, redémit, aliísque sanctis opéribus 
 célebris quiévit in Dómino.
\switchcolumn
\selectlanguage{english}
At Carthage, St. Deogratias, bishop of Carthage, who ransomed many captives 
 taken from that city by the Vandals, and who performed many other good 
 works, after which he went to rest in the Lord.
\switchcolumn*
\selectlanguage{latin}
Auximi, in Picéno, sancti Benvenúti Epíscopi.
\switchcolumn
\selectlanguage{english}
At Osimo, in Piceno, the bishop St. Benvenuto.
\switchcolumn*
\selectlanguage{latin}
Romæ sanctæ Leæ Víduæ, cujus virtútes et 
 tránsitum ad Deum sanctus Hierónymus scribit.
\switchcolumn
\selectlanguage{english}
At Rome, the widow St. Lea, whose virtues and happy death are related by St. 
 Jerome.
\switchcolumn*
\selectlanguage{latin}
\end{paracol}


% ---- martyrology/mart03/mart0323.htm
\needspace{10\baselineskip}
\begin{paracol}{2}
\selectlanguage{latin}
\begin{center}{\color{gregoriocolor} Décimo Kaléndas Aprílis. 
 Luna\dots\ }\end{center}
\switchcolumn
\selectlanguage{english}
\begin{center}{\color{gregoriocolor} The Twenty-Third Day of March. 
 The\dots\ Day of the Moon.}\end{center}
\end{paracol}

\noindent\begin{tabularx}{\linewidth}{*{19}{>{\centering\arraybackslash}X}}
 \textcolor{gregoriocolor}{a} & \textcolor{gregoriocolor}{b} & \textcolor{gregoriocolor}{c} & \textcolor{gregoriocolor}{d} & \textcolor{gregoriocolor}{e} & \textcolor{gregoriocolor}{f} & \textcolor{gregoriocolor}{g} & \textcolor{gregoriocolor}{h} & \textcolor{gregoriocolor}{i} & \textcolor{gregoriocolor}{k} & \textcolor{gregoriocolor}{l} & \textcolor{gregoriocolor}{m} & \textcolor{gregoriocolor}{n} & \textcolor{gregoriocolor}{p} & \textcolor{gregoriocolor}{q} & \textcolor{gregoriocolor}{r} & \textcolor{gregoriocolor}{s} & \textcolor{gregoriocolor}{t} & \textcolor{gregoriocolor}{u} \\
 24 & 25 & 26 & 27 & 28 & 29 & 30 & 1 & 2 & 3 & 4 & 5 & 6 & 7 & 8 & 9 & 10 & 11 & 12 \\
\end{tabularx}
\vspace{0.5\baselineskip}
\noindent\begin{tabularx}{\linewidth}{*{12}{>{\centering\arraybackslash}X}}
 \textcolor{gregoriocolor}{A} & \textcolor{gregoriocolor}{B} & \textcolor{gregoriocolor}{C} & \textcolor{gregoriocolor}{D} & \textcolor{gregoriocolor}{E} & F & \textcolor{gregoriocolor}{F} & \textcolor{gregoriocolor}{G} & \textcolor{gregoriocolor}{H} & \textcolor{gregoriocolor}{M} & \textcolor{gregoriocolor}{N} & \textcolor{gregoriocolor}{P} \\
 13 & 14 & 15 & 16 & 17 & 18 & 18 & 19 & 20 & 21 & 22 & 23 \\
\end{tabularx}

\begin{paracol}{2}
\selectlanguage{latin}
\lettrine[lines=2]{I}{n} Africa sanctórum Mártyrum Victoriáni, Procónsulis Cartháginis, et duórum 
 germanórum, Aquisregénsium; item Fruméntii et altérius Fruméntii, mercatórum. 
 Hi omnes, in persecutióne Wandálica (ut scribit Victor, Africánus Epíscopus), 
 sub Ariáno Rege Hunneríco, pro constántia 
 cathólicæ confessiónis, immaníssimis supplíciis cruciáti, egrégie coronáti 
 sunt.
\switchcolumn
\selectlanguage{english}
\lettrine[lines=2]{I}{n} Africa, the holy martyrs Victorian, proconsul of Carthage, and two 
 brothers from Aquaregia. Also two merchants, both named Frementius, 
 who (as Bishop Victor Africanus relates) were subjected to the most 
 atrocious torments for their courageous confession of the Catholic faith, 
 and who were gloriously crowned martyrs under the Arian king Hunneric, 
 during the persecution of the Vandals.
\switchcolumn*
\selectlanguage{latin}
Item in Africa sancti Fidélis Mártyris.
\switchcolumn
\selectlanguage{english}
Also in Africa, St. Fidelis, martyr.
\switchcolumn*
\selectlanguage{latin}
Ibídem sancti Felícis et aliórum vigínti Mártyrum.
\switchcolumn
\selectlanguage{english}
In the same place, St. Felix and twenty other martyrs.
\switchcolumn*
\selectlanguage{latin}
Cæsaréæ, in Palæstína, sanctórum Mártyrum 
 Nicónis et aliórum nonagínta novem.
\switchcolumn
\selectlanguage{english}
At Caesarea in Palestine, the holy martyrs Nicon and ninety-nine others.
\switchcolumn*
\selectlanguage{latin}
Item corónæ sanctórum Mártyrum Domítii, Pelágiæ, 
 Aquilæ, Epárchii et Theodósiæ.
\switchcolumn
\selectlanguage{english}
Likewise, the crowning of the holy martyrs Domitius, Pelagia, Aquila, 
 Eparchius, and Theodosia.
\switchcolumn*
\selectlanguage{latin}
Limæ, in Perúvia, sancti Turíbii Epíscopi, 
 cujus virtúte fides et disciplína ecclesiástica per Amerícam diffúsæ sunt.
\switchcolumn
\selectlanguage{english}
At Lima in Peru, Archbishop St. Turibius, through whose labours both faith 
 and ecclesiastical discipline were spread through America.
\switchcolumn*
\selectlanguage{latin}
Antiochíæ sancti Theodúli Presbyteri.
\switchcolumn
\selectlanguage{english}
At Antioch, the priest St. Theodulus.
\switchcolumn*
\selectlanguage{latin}
Barcinóne, in Hispánia, sancti Joséphi Oriol Presbyteri, Ecclésiæ 
 sanctæ Maríæ Regum Beneficiárii, omnígena virtúte ac præsértim córporis 
 afflictatióne, paupertátis cultu atque in egénos et infírmos caritáte 
 célebris; quem, in vita et post mortem miráculis gloriósum, Pius Papa 
 Décimus in Sanctórum númerum recénsuit.
\switchcolumn
\selectlanguage{english}
At Barcelona in Spain, the priest St. Joseph Oriol, pastor of the church of 
 St. Mary of the Kings, famous for every virtue, especially mortification of 
 the body, his rule of poverty, and his love towards the poor and the sick. 
 Because he was known for his miracles both in life and after death, Pope 
 Pius X placed his name in the number of the saints.
\switchcolumn*
\selectlanguage{latin}
Cæsaréæ sancti Juliáni Confessóris.
\switchcolumn
\selectlanguage{english}
At Caesarea, St. Julian, confessor.
\switchcolumn*
\selectlanguage{latin}
In Campánia sancti Benedícti Mónachi, qui, a Gothis in ardénti clíbano 
 inclúsus, sequénti die invéntus est illæsus.
\switchcolumn
\selectlanguage{english}
In Campania, St. Benedict, monk, who was shut up in a burning furnace by the 
 Goths, but who was found uninjured the next day.
\switchcolumn*
\selectlanguage{latin}
\end{paracol}


% ---- martyrology/mart03/mart0324.htm
\needspace{10\baselineskip}
\begin{paracol}{2}
\selectlanguage{latin}
\begin{center}{\color{gregoriocolor} Nono Kaléndas Aprílis. 
 Luna\dots\ }\end{center}
\switchcolumn
\selectlanguage{english}
\begin{center}{\color{gregoriocolor} The Twenty-Fourth Day of March. 
 The\dots\ Day of the Moon.}\end{center}
\end{paracol}

\noindent\begin{tabularx}{\linewidth}{*{19}{>{\centering\arraybackslash}X}}
 \textcolor{gregoriocolor}{a} & \textcolor{gregoriocolor}{b} & \textcolor{gregoriocolor}{c} & \textcolor{gregoriocolor}{d} & \textcolor{gregoriocolor}{e} & \textcolor{gregoriocolor}{f} & \textcolor{gregoriocolor}{g} & \textcolor{gregoriocolor}{h} & \textcolor{gregoriocolor}{i} & \textcolor{gregoriocolor}{k} & \textcolor{gregoriocolor}{l} & \textcolor{gregoriocolor}{m} & \textcolor{gregoriocolor}{n} & \textcolor{gregoriocolor}{p} & \textcolor{gregoriocolor}{q} & \textcolor{gregoriocolor}{r} & \textcolor{gregoriocolor}{s} & \textcolor{gregoriocolor}{t} & \textcolor{gregoriocolor}{u} \\
 25 & 26 & 27 & 28 & 29 & 30 & 1 & 2 & 3 & 4 & 5 & 6 & 7 & 8 & 9 & 10 & 11 & 12 & 13 \\
\end{tabularx}
\vspace{0.5\baselineskip}
\noindent\begin{tabularx}{\linewidth}{*{12}{>{\centering\arraybackslash}X}}
 \textcolor{gregoriocolor}{A} & \textcolor{gregoriocolor}{B} & \textcolor{gregoriocolor}{C} & \textcolor{gregoriocolor}{D} & \textcolor{gregoriocolor}{E} & F & \textcolor{gregoriocolor}{F} & \textcolor{gregoriocolor}{G} & \textcolor{gregoriocolor}{H} & \textcolor{gregoriocolor}{M} & \textcolor{gregoriocolor}{N} & \textcolor{gregoriocolor}{P} \\
 14 & 15 & 16 & 17 & 18 & 19 & 19 & 20 & 21 & 22 & 23 & 24 \\
\end{tabularx}

\begin{paracol}{2}
\selectlanguage{latin}
\lettrine[lines=2]{F}{estum} sancti Gabriélis Archángeli, qui ad annuntiándum Incarnatiónis divíni 
 Verbi mystérium a Deo missus est.
\switchcolumn
\selectlanguage{english}
\lettrine[lines=2]{T}{he} Feast of St. Gabriel Archangel, who was sent by God to announce the 
 Incarnation of the Divine Word.
\switchcolumn*
\selectlanguage{latin}
Romæ sancti Epigménii Presbyteri, qui, in 
 persecutióne Diocletiáni, sub Túrpio Júdice, gládio cæsus, martyrium 
 consummávit.
\switchcolumn
\selectlanguage{english}
At Rome, the priest St. Epigmenius, who completed his martyrdom by the sword 
 in the persecution of Diocletian, under the judge Turpius.
\switchcolumn*
\selectlanguage{latin}
Ibídem pássio beáti Pigménii Presbyteri, qui, sub Juliáno Apóstata, pro fide 
 Christi, præcipitátus in Tíberim, necátus est.
\switchcolumn
\selectlanguage{english}
Also at Rome, in the time of Julian the Apostate, the passion of blessed 
 Pigmenius, a priest, who was killed for the faith of Christ by being drowned 
 in the Tiber.
\switchcolumn*
\selectlanguage{latin}
Item Romæ sanctórum Mártyrum Marci et Timóthei, 
 qui martyrio coronáti sunt sub Antoníno Imperatóre.
\switchcolumn
\selectlanguage{english}
At Rome, the holy martyrs Mark and Timothy, who were crowned with martyrdom 
 under Emperor Antoninus.
\switchcolumn*
\selectlanguage{latin}
Cæsaréæ, in Palæstína, natális sanctórum 
 Mártyrum Timolái, Dionysii, Páusidis, Rómuli, Alexándri, altérius Alexándri, 
 Agápii et altérius Dionysii; qui, in persecutióne Diocletiáni, sub Urbáno 
 Præside, secúris ictu percússi, vitæ corónas meruérunt.
\switchcolumn
\selectlanguage{english}
At Caesarea in Palestine, the birthday of the holy martyrs Timolaus, Denis, 
 Pausides, Romulus, Alexander, another Alexander, Agapius, and another Denis. 
 They merited the crown of life by being beheaded in the persecution of 
 Diocletian under the governor Urban.
\switchcolumn*
\selectlanguage{latin}
In Mauritánia item natális sanctórum fratrum Rómuli et Secúndi, qui pro 
 Christi fide passi sunt.
\switchcolumn
\selectlanguage{english}
In Morocco, the birthday of the saintly brothers Romulus and Secundus, who 
 suffered for the faith of Christ.
\switchcolumn*
\selectlanguage{latin}
Tridénti pássio sancti Simeónis púeri, a Judǽis 
 sævíssime trucidáti, qui multis póstea miráculis coruscávit.
\switchcolumn
\selectlanguage{english}
At Trent, the martyrdom of the boy St. Simeon, who was barbarously murdered 
 by the Jews, but who was afterwards glorified by many miracles.
\switchcolumn*
\selectlanguage{latin}
Synnadæ, in Phrygia, sancti Agapíti Epíscopi.
\switchcolumn
\selectlanguage{english}
At Synnadas in Phrygia, Bishop St. Agapitus.
\switchcolumn*
\selectlanguage{latin}
Bríxiæ sancti Latíni Epíscopi.
\switchcolumn
\selectlanguage{english}
At Brescia, the bishop St. Latinus.
\switchcolumn*
\selectlanguage{latin}
In Syria sancti Seléuci Confessóris.
\switchcolumn
\selectlanguage{english}
In Syria, St. Seleucus, confessor.
\switchcolumn*
\selectlanguage{latin}
In Suécia sanctæ Catharínæ Vírginis, quæ fuit 
 fília sanctæ Birgíttæ.
\switchcolumn
\selectlanguage{english}
In Sweden, the virgin St. Catherine, daughter of St. Bridget.
\switchcolumn*
\selectlanguage{latin}
\end{paracol}


% ---- martyrology/mart03/mart0325.htm
\needspace{10\baselineskip}
\begin{paracol}{2}
\selectlanguage{latin}
\begin{center}{\color{gregoriocolor} Octávo Kaléndas Aprílis. 
 Luna\dots\ }\end{center}
\switchcolumn
\selectlanguage{english}
\begin{center}{\color{gregoriocolor} The Twenty-Fifth Day of March. 
 The\dots\ Day of the Moon.}\end{center}
\end{paracol}

\noindent\begin{tabularx}{\linewidth}{*{19}{>{\centering\arraybackslash}X}}
 \textcolor{gregoriocolor}{a} & \textcolor{gregoriocolor}{b} & \textcolor{gregoriocolor}{c} & \textcolor{gregoriocolor}{d} & \textcolor{gregoriocolor}{e} & \textcolor{gregoriocolor}{f} & \textcolor{gregoriocolor}{g} & \textcolor{gregoriocolor}{h} & \textcolor{gregoriocolor}{i} & \textcolor{gregoriocolor}{k} & \textcolor{gregoriocolor}{l} & \textcolor{gregoriocolor}{m} & \textcolor{gregoriocolor}{n} & \textcolor{gregoriocolor}{p} & \textcolor{gregoriocolor}{q} & \textcolor{gregoriocolor}{r} & \textcolor{gregoriocolor}{s} & \textcolor{gregoriocolor}{t} & \textcolor{gregoriocolor}{u} \\
 26 & 27 & 28 & 29 & 30 & 1 & 2 & 3 & 4 & 5 & 6 & 7 & 8 & 9 & 10 & 11 & 12 & 13 & 14 \\
\end{tabularx}
\vspace{0.5\baselineskip}
\noindent\begin{tabularx}{\linewidth}{*{12}{>{\centering\arraybackslash}X}}
 \textcolor{gregoriocolor}{A} & \textcolor{gregoriocolor}{B} & \textcolor{gregoriocolor}{C} & \textcolor{gregoriocolor}{D} & \textcolor{gregoriocolor}{E} & F & \textcolor{gregoriocolor}{F} & \textcolor{gregoriocolor}{G} & \textcolor{gregoriocolor}{H} & \textcolor{gregoriocolor}{M} & \textcolor{gregoriocolor}{N} & \textcolor{gregoriocolor}{P} \\
 15 & 16 & 17 & 18 & 19 & 20 & 20 & 21 & 22 & 23 & 24 & 25 \\
\end{tabularx}

\begin{paracol}{2}
\selectlanguage{latin}
\lettrine[lines=2]{A}{nnuntiátio} beatíssimæ Vírginis Genitrícis Dei 
 Maríæ.
\switchcolumn
\selectlanguage{english}
\lettrine[lines=2]{T}{he} Annunciation of the Most Blessed Virgin Mary, Mother of God.
\switchcolumn*
\selectlanguage{latin}
Hierosólymis commemorátio sancti Latrónis, qui, in cruce Christum conféssus, 
 ab eo méruit audíre: « Hódie mecum eris in 
 paradíso ».
\switchcolumn
\selectlanguage{english}
At Jerusalem, the commemoration of the good thief who confessed Christ on 
 the cross, and who deserved to hear from him these words: ``This day shalt 
 thou be with me in paradise.''
\switchcolumn*
\selectlanguage{latin}
Romæ sancti Quiríni Mártyris, qui, sub Cláudio 
 Imperatóre, post facultátum amissiónem, post cárceris squalórem, post 
 multórum vérberum afflictiónem, gládio interféctus est et in Tíberim 
 projéctus; quem Christiáni, cum in ínsula Lycaónia (quæ póstea sancti 
 Bartholomæi dicta est) inveníssent, in cœmetério Pontiáni condidérunt.
\switchcolumn
\selectlanguage{english}
At Rome, St. Quirinus, martyr, who after losing his possessions, suffering 
 imprisonment in a dark dungeon, and being cruelly scourged, was put to death 
 with the sword, and thrown into the Tiber. The Christians found his 
 body on the island of Lycaonia (which was thereafter called St. 
 Bartholomew's), and buried it in the Pontian cemetery.
\switchcolumn*
\selectlanguage{latin}
Item Romæ sanctórum ducentórum sexagínta duórum 
 Mártyrum.
\switchcolumn
\selectlanguage{english}
Also at Rome, two hundred and sixty-two holy martyrs.
\switchcolumn*
\selectlanguage{latin}
Sírmii pássio sancti Irenǽi, Epíscopi et 
 Mártyris; qui, témpore Maximiáni Imperatóris, sub Præside Probo, 
 primum torméntis acérrimis vexátus, deínde diébus plúrimis cruciátus in cárcere, novíssime, abscísso cápite, consummátus est.
\switchcolumn
\selectlanguage{english}
At Sirmium, the martyrdom of St. Irenaeus, bishop. In the time of 
 Emperor Maximian, under the governor Probus, after undergoing bitter 
 torments and a painful imprisonment for may days, he was beheaded.
\switchcolumn*
\selectlanguage{latin}
Nicomedíæ sanctæ Dulæ, cujúsdam mílitis ancíllæ, 
 quæ, ob castitátem servándam occísa, martyrii corónam proméruit.
\switchcolumn
\selectlanguage{english}
At Nicomedia, St. Dula, the servant of a soldier, who was killed for the 
 preservation of her chastity, and deserved the crown of martyrdom.
\switchcolumn*
\selectlanguage{latin}
Laodicéæ, ad Líbanum, sancti Pelágii Epíscopi, 
 qui, ob fidem cathólicam, témpore Valéntis, exsílium et ália passus est; ac 
 tandem, in sedem suam restitútus, quiévit in Dómino.
\switchcolumn
\selectlanguage{english}
At Laodicea, St. Pelagius, bishop, who after having endured exile and other 
 afflictions for the Catholic faith under Valens, rested in the Lord.
\switchcolumn*
\selectlanguage{latin}
In Antro, ínsula Lígeris flúminis, sancti 
 Hermelándi Abbátis, cujus gloriósa conversátio insígni miraculórum præcónio 
 commendátur.
\switchcolumn
\selectlanguage{english}
At Indre, an island in the Loire, Abbot St. Hermeland, whose glorious life 
 was commended by outstanding miracles.
\switchcolumn*
\selectlanguage{latin}
Pistórii, in Túscia, sanctórum Confessórum Baróntii et Desidérii.
\switchcolumn
\selectlanguage{english}
At Pistoia, the holy confessors Barontius and Desiderius.
\switchcolumn*
\selectlanguage{latin}
Faliscodúni sanctæ 
 Lúciæ Filippíni, Fundatrícis Institúti Magistrárum Piárum ab ejus cognómine 
 nuncupatárum, de Christiána puellárum et mulíerum, præsértim páuperum, 
 eruditióne óptime méritæ, quam Pius Papa Undécimus inter sanctas Vírgines 
 rétulit.
\switchcolumn
\selectlanguage{english}
At Montefiascone, St. Lucia Filippini, founder of the Institute of Pious 
 Teachers, from whose surname they are known as Filippines. 
 Having merited greatly by the Christian education of girls and women, 
 especially of the poor, Pope Pius XI enrolled her among the holy virgins.
\switchcolumn*
\selectlanguage{latin}
\end{paracol}


% ---- martyrology/mart03/mart0326.htm
\needspace{10\baselineskip}
\begin{paracol}{2}
\selectlanguage{latin}
\begin{center}{\color{gregoriocolor} Séptimo Kaléndas Aprílis. 
 Luna\dots\ }\end{center}
\switchcolumn
\selectlanguage{english}
\begin{center}{\color{gregoriocolor} The Twenty-Sixth Day of March. 
 The\dots\ Day of the Moon.}\end{center}
\end{paracol}

\noindent\begin{tabularx}{\linewidth}{*{19}{>{\centering\arraybackslash}X}}
 \textcolor{gregoriocolor}{a} & \textcolor{gregoriocolor}{b} & \textcolor{gregoriocolor}{c} & \textcolor{gregoriocolor}{d} & \textcolor{gregoriocolor}{e} & \textcolor{gregoriocolor}{f} & \textcolor{gregoriocolor}{g} & \textcolor{gregoriocolor}{h} & \textcolor{gregoriocolor}{i} & \textcolor{gregoriocolor}{k} & \textcolor{gregoriocolor}{l} & \textcolor{gregoriocolor}{m} & \textcolor{gregoriocolor}{n} & \textcolor{gregoriocolor}{p} & \textcolor{gregoriocolor}{q} & \textcolor{gregoriocolor}{r} & \textcolor{gregoriocolor}{s} & \textcolor{gregoriocolor}{t} & \textcolor{gregoriocolor}{u} \\
 27 & 28 & 29 & 30 & 1 & 2 & 3 & 4 & 5 & 6 & 7 & 8 & 9 & 10 & 11 & 12 & 13 & 14 & 15 \\
\end{tabularx}
\vspace{0.5\baselineskip}
\noindent\begin{tabularx}{\linewidth}{*{12}{>{\centering\arraybackslash}X}}
 \textcolor{gregoriocolor}{A} & \textcolor{gregoriocolor}{B} & \textcolor{gregoriocolor}{C} & \textcolor{gregoriocolor}{D} & \textcolor{gregoriocolor}{E} & F & \textcolor{gregoriocolor}{F} & \textcolor{gregoriocolor}{G} & \textcolor{gregoriocolor}{H} & \textcolor{gregoriocolor}{M} & \textcolor{gregoriocolor}{N} & \textcolor{gregoriocolor}{P} \\
 16 & 17 & 18 & 19 & 20 & 21 & 21 & 22 & 23 & 24 & 25 & 26 \\
\end{tabularx}

\begin{paracol}{2}
\selectlanguage{latin}
\lettrine[lines=2]{R}{omæ,} via Lavicána, sancti Cástuli Mártyris, 
 qui, cum esset zetárius Palátii et hospes Sanctórum, a persecutóribus tértio 
 appénsus, tértio audítus, et, in confessióne Dómini persevérans, missus est 
 in fóveam, ac, dimíssa super eum massa arenária, martyrio coronátus est.
\switchcolumn
\selectlanguage{english}
\lettrine[lines=2]{A}{t} Rome, on the Via Lavicana, St. Castulus, martyr, chamberlain in the 
 palace of the emperor. For harbouring Christians, he was three times 
 suspended by the hands, three times cited before the tribunals. As he 
 persevered in the confession of the Lord, he was thrown into a pit, covered 
 with a mass of sand, and thus obtained the crown of martyrdom.
\switchcolumn*
\selectlanguage{latin}
Item Romæ corónæ sanctórum Mártyrum Petri, 
 Marciáni, Jovíni, Theclæ, Cassiáni et aliórum.
\switchcolumn
\selectlanguage{english}
Also at Rome, the crowning of the holy martyrs Peter, Marcian, Jovinus, 
 Thecla, Cassian, and others.
\switchcolumn*
\selectlanguage{latin}
Pentápoli, in Libya, natális sanctórum 
 Mártyrum Theodóri Epíscopi, Irenǽi Diáconi, 
 Serapiónis et Ammónii Lectórum.
\switchcolumn
\selectlanguage{english}
At Pentopolis in Libya, the birthday of the holy martyrs Theodore, bishop, 
 the deacon Irenæus, and the lectors Serapion and Ammonius.
\switchcolumn*
\selectlanguage{latin}
Sírmii sanctórum Mártyrum Montáni Presbyteri, et Máximæ, 
 qui, ob Christi fidem, in flumen demérsi sunt.
\switchcolumn
\selectlanguage{english}
At Sirmium, the holy martyrs Montanus, priest, and Maxima, who were drowned 
 in a river for the faith of Christ.
\switchcolumn*
\selectlanguage{latin}
Item sanctórum Mártyrum Quadráti, Theodósii, Emmanuélis et aliórum 
 quadragínta.
\switchcolumn
\selectlanguage{english}
Likewise, the holy martyrs Quadratus, Theodosius, Emmanuel, and forty 
 others.
\switchcolumn*
\selectlanguage{latin}
Alexandríæ sanctórum Mártyrum Eutychii et 
 aliórum; qui, Constántii témpore, sub Ariáno Epíscopo Geórgio, pro fide 
 cathólica gládio cæsi sunt.
\switchcolumn
\selectlanguage{english}
At Alexandria, the holy martyrs Eutychius and others, who died by the sword 
 for the Catholic faith, in the time of Constantine, under the Arian bishop 
 George.
\switchcolumn*
\selectlanguage{latin}
Eódem die sancti Ludgéri, Epíscopi Monasteriénsis, qui Saxónibus Evangélium 
 prædicávit.
\switchcolumn
\selectlanguage{english}
The same day, St. Ludger, bishop of Munster, who preached the Gospel to the 
 Saxons.
\switchcolumn*
\selectlanguage{latin}
Cæsaraugústæ, in Hispánia, sancti Bráulii, 
 Epíscopi et Confessóris.
\switchcolumn
\selectlanguage{english}
At Saragossa in Spain, St. Braulio, bishop and confessor.
\switchcolumn*
\selectlanguage{latin}
Tréviris sancti Felícis Epíscopi.
\switchcolumn
\selectlanguage{english}
At Treves, St. Felix, bishop.
\switchcolumn*
\selectlanguage{latin}
\end{paracol}


% ---- martyrology/mart03/mart0327.htm
\needspace{10\baselineskip}
\begin{paracol}{2}
\selectlanguage{latin}
\begin{center}{\color{gregoriocolor} Sexto Kaléndas Aprílis. 
 Luna\dots\ }\end{center}
\switchcolumn
\selectlanguage{english}
\begin{center}{\color{gregoriocolor} The Twenty-Seventh Day of March. 
 The\dots\ Day of the Moon.}\end{center}
\end{paracol}

\noindent\begin{tabularx}{\linewidth}{*{19}{>{\centering\arraybackslash}X}}
 \textcolor{gregoriocolor}{a} & \textcolor{gregoriocolor}{b} & \textcolor{gregoriocolor}{c} & \textcolor{gregoriocolor}{d} & \textcolor{gregoriocolor}{e} & \textcolor{gregoriocolor}{f} & \textcolor{gregoriocolor}{g} & \textcolor{gregoriocolor}{h} & \textcolor{gregoriocolor}{i} & \textcolor{gregoriocolor}{k} & \textcolor{gregoriocolor}{l} & \textcolor{gregoriocolor}{m} & \textcolor{gregoriocolor}{n} & \textcolor{gregoriocolor}{p} & \textcolor{gregoriocolor}{q} & \textcolor{gregoriocolor}{r} & \textcolor{gregoriocolor}{s} & \textcolor{gregoriocolor}{t} & \textcolor{gregoriocolor}{u} \\
 28 & 29 & 30 & 1 & 2 & 3 & 4 & 5 & 6 & 7 & 8 & 9 & 10 & 11 & 12 & 13 & 14 & 15 & 16 \\
\end{tabularx}
\vspace{0.5\baselineskip}
\noindent\begin{tabularx}{\linewidth}{*{12}{>{\centering\arraybackslash}X}}
 \textcolor{gregoriocolor}{A} & \textcolor{gregoriocolor}{B} & \textcolor{gregoriocolor}{C} & \textcolor{gregoriocolor}{D} & \textcolor{gregoriocolor}{E} & F & \textcolor{gregoriocolor}{F} & \textcolor{gregoriocolor}{G} & \textcolor{gregoriocolor}{H} & \textcolor{gregoriocolor}{M} & \textcolor{gregoriocolor}{N} & \textcolor{gregoriocolor}{P} \\
 17 & 18 & 19 & 20 & 21 & 22 & 22 & 23 & 24 & 25 & 26 & 27 \\
\end{tabularx}

\begin{paracol}{2}
\selectlanguage{latin}
\lettrine[lines=2]{S}{ancti} Joánnis Damascéni, Presbyteri, Confessóris et Ecclésiæ 
 Doctoris, cujus dies natális ágitur prídie Nonas Maji.
\switchcolumn
\selectlanguage{english}
\lettrine[lines=2]{S}{t.} John Damascene, priest, confessor, and doctor of the Church, whose 
 birthday is commemorated on the 6th of May.
\switchcolumn*
\selectlanguage{latin}
Drizíparæ, 
 in Pannónia, sancti Alexándri mílitis, qui, sub Maximiáno Imperatóre, post 
 multos pro Christo agónes superátos múltaque mirácula édita, cápitis 
 abscissióne martyrium complévit.
\switchcolumn
\selectlanguage{english}
At Drizipara in Hungary, St. Alexander, soldier, in the time of Emperor 
 Maximian. Having overcome many torments for the sake of Christ, and 
 performing many miracles, his martyrdom was completed by beheading.
\switchcolumn*
\selectlanguage{latin}
In Illyrico sanctórum Philéti Senatóris, Lydiæ 
 uxóris, et filiórum Macédonis et Theoprépii, itémque Amphilóchii Ducis, et 
 Crónidæ Commentariénsis; qui, pro Christi confessióne, torméntis plúribus 
 superátis, corónam glóriæ sunt adépti.
\switchcolumn
\selectlanguage{english}
In Illyria, the Saints Philetus, senator, his wife Lydia, and their sons 
 Macedon and Theoprepides; also Amphilochius, an officer in the army, and 
 Chronides, a notary, who were put to death for the confession of Christ 
 after suffering many things.
\switchcolumn*
\selectlanguage{latin}
In Pérside sanctórum Mártyrum Zanítæ, Lázari, 
 Marótæ, Narsétis et aliórum quinque, qui sub Rege Persárum Sápore, sævíssime 
 trucidáti, martyrii palmam meruérunt.
\switchcolumn
\selectlanguage{english}
In Persia, in the reign of King Sapor, the holy martyrs Zanitas, Lazarus, 
 Marotas, Narses, and five others, who were barbarously slain, having merited 
 the martyr's palm.
\switchcolumn*
\selectlanguage{latin}
Salisbúrgi, in Nórico, sancti Rupérti, Epíscopi 
 et Confessóris, qui apud Bávaros et Nóricos Evangélium mirífice propagávit.
\switchcolumn
\selectlanguage{english}
At Salzburg in Austria, St. Rupert, bishop and confessor, who spread the 
 Gospel extensively in Bavaria and Austria.
\switchcolumn*
\selectlanguage{latin}
In Ægypto sancti Joánnis Eremítæ, magnæ 
 sanctitátis viri, qui, inter cétera virtútum insígnia, étiam prophético 
 spíritu plenus, Theodósio Imperatóri victórias de tyránnis Máximo et Eugénio 
 prædíxit.
\switchcolumn
\selectlanguage{english}
In Egypt, the hermit St. John, a man of great sanctity, who, among other 
 virtues, was filled with the spirit of prophecy, and predicted to Emperor 
 Theodosius his victories over the tyrants Maximus and Eugene.
\switchcolumn*
\selectlanguage{latin}
\end{paracol}


% ---- martyrology/mart03/mart0328.htm
\needspace{10\baselineskip}
\begin{paracol}{2}
\selectlanguage{latin}
\begin{center}{\color{gregoriocolor} Quinto Kaléndas Aprílis. 
 Luna\dots\ }\end{center}
\switchcolumn
\selectlanguage{english}
\begin{center}{\color{gregoriocolor} The Twenty-Eighth Day of March. 
 The\dots\ Day of the Moon.}\end{center}
\end{paracol}

\noindent\begin{tabularx}{\linewidth}{*{19}{>{\centering\arraybackslash}X}}
 \textcolor{gregoriocolor}{a} & \textcolor{gregoriocolor}{b} & \textcolor{gregoriocolor}{c} & \textcolor{gregoriocolor}{d} & \textcolor{gregoriocolor}{e} & \textcolor{gregoriocolor}{f} & \textcolor{gregoriocolor}{g} & \textcolor{gregoriocolor}{h} & \textcolor{gregoriocolor}{i} & \textcolor{gregoriocolor}{k} & \textcolor{gregoriocolor}{l} & \textcolor{gregoriocolor}{m} & \textcolor{gregoriocolor}{n} & \textcolor{gregoriocolor}{p} & \textcolor{gregoriocolor}{q} & \textcolor{gregoriocolor}{r} & \textcolor{gregoriocolor}{s} & \textcolor{gregoriocolor}{t} & \textcolor{gregoriocolor}{u} \\
 29 & 30 & 1 & 2 & 3 & 4 & 5 & 6 & 7 & 8 & 9 & 10 & 11 & 12 & 13 & 14 & 15 & 16 & 17 \\
\end{tabularx}
\vspace{0.5\baselineskip}
\noindent\begin{tabularx}{\linewidth}{*{12}{>{\centering\arraybackslash}X}}
 \textcolor{gregoriocolor}{A} & \textcolor{gregoriocolor}{B} & \textcolor{gregoriocolor}{C} & \textcolor{gregoriocolor}{D} & \textcolor{gregoriocolor}{E} & F & \textcolor{gregoriocolor}{F} & \textcolor{gregoriocolor}{G} & \textcolor{gregoriocolor}{H} & \textcolor{gregoriocolor}{M} & \textcolor{gregoriocolor}{N} & \textcolor{gregoriocolor}{P} \\
 18 & 19 & 20 & 21 & 22 & 23 & 23 & 24 & 25 & 26 & 27 & 28 \\
\end{tabularx}

\begin{paracol}{2}
\selectlanguage{latin}
\lettrine[lines=2]{S}{ancti} Joánnis de Capistráno, Sacerdótis ex Ordine Minórum et Confessóris, 
 cujus memória recólitur décimo Kaléndas Novémbris.
\switchcolumn
\selectlanguage{english}
\lettrine[lines=2]{S}{t.} John Capistrano, confessor, a priest of the Order of Friars Minor, who 
 is mentioned on the 23rd of October.
\switchcolumn*
\selectlanguage{latin}
Cæsaréæ, in Palæstína, natális sanctórum 
 Mártyrum Prisci, Malchi et Alexándri. Hi tres, in persecutióne 
 Valeriáni, cum in suburbáno agéllo supradíctæ urbis habitárent, atque in ea 
 cæléstes martyrii proponeréntur corónæ, ultro Júdicem, divíno fídei calóre 
 succénsi, ádeunt, et cur tantum in sánguinem piórum desævíret, objúrgant; 
 quos ille contínuo, pro Christi nómine, béstiis trádidit devorándos.
\switchcolumn
\selectlanguage{english}
At Caesarea in Palestine, the birthday of the holy martyrs Priscus, Malchus, 
 and Alexander. In the persecution of Valerian, they were living the 
 suburbs of Caesarea, but knowing that in the city the heavenly crown of 
 martyrdom was to be gained, and burning with the divine ardour of faith, 
 they went to the judge of their own accord, rebuked him for shedding in 
 torrents the blood of the faithful, and were immediately condemned to be 
 devoured by beasts for the Name of Christ.
\switchcolumn*
\selectlanguage{latin}
Tarsi, in Cilícia, sanctórum Mártyrum Cástoris 
 et Doróthei.
\switchcolumn
\selectlanguage{english}
At Tarsus in Cilicia, the holy martyrs Castor and Dorotheus.
\switchcolumn*
\selectlanguage{latin}
In Africa sanctórum Mártyrum Rogáti, 
 Succéssi et aliórum séxdecim.
\switchcolumn
\selectlanguage{english}
In Africa, the holy martyrs Rogatus, Successus, and sixteen others.
\switchcolumn*
\selectlanguage{latin}
Apud Núrsiam sancti Spei Abbátis, miræ 
 patiéntiæ viri, cujus ánima (ut refert sanctus Gregórius Papa), cum ex hac 
 vita migráret, in colúmbæ spécie a cunctis frátribus visa est in cælum 
 ascéndere.
\switchcolumn
\selectlanguage{english}
At Norcia, Abbot St. Spes, a man of extraordinary patience, whose soul at 
 its departure from this life (as Pope St. Gregory relates) was seen by all 
 his brethren to ascend to heaven in the shape of a dove.
\switchcolumn*
\selectlanguage{latin}
Cabillóne, in Gálliis, deposítio sancti Gunthrámni, Regis Francórum, qui 
 spiritálibus actiónibus ita se mancipávit, ut, relíctis, sæculi 
 pompis, thesáuros suos lárgiter Ecclésiis et paupéribus erogáret.
\switchcolumn
\selectlanguage{english}
At Chalons in France, the death of St. Guntram, king of the Franks, who 
 devoted himself to exercises of piety, despising the ostentation of the 
 world, and who bestowed his treasures on the Church and the poor.
\switchcolumn*
\selectlanguage{latin}
\end{paracol}


% ---- martyrology/mart03/mart0329.htm
\needspace{10\baselineskip}
\begin{paracol}{2}
\selectlanguage{latin}
\begin{center}{\color{gregoriocolor} Quarto Kaléndas Aprílis. 
 Luna\dots\ }\end{center}
\switchcolumn
\selectlanguage{english}
\begin{center}{\color{gregoriocolor} The Twenty-Ninth Day of March. 
 The\dots\ Day of the Moon.}\end{center}
\end{paracol}

\noindent\begin{tabularx}{\linewidth}{*{19}{>{\centering\arraybackslash}X}}
 \textcolor{gregoriocolor}{a} & \textcolor{gregoriocolor}{b} & \textcolor{gregoriocolor}{c} & \textcolor{gregoriocolor}{d} & \textcolor{gregoriocolor}{e} & \textcolor{gregoriocolor}{f} & \textcolor{gregoriocolor}{g} & \textcolor{gregoriocolor}{h} & \textcolor{gregoriocolor}{i} & \textcolor{gregoriocolor}{k} & \textcolor{gregoriocolor}{l} & \textcolor{gregoriocolor}{m} & \textcolor{gregoriocolor}{n} & \textcolor{gregoriocolor}{p} & \textcolor{gregoriocolor}{q} & \textcolor{gregoriocolor}{r} & \textcolor{gregoriocolor}{s} & \textcolor{gregoriocolor}{t} & \textcolor{gregoriocolor}{u} \\
 30 & 1 & 2 & 3 & 4 & 5 & 6 & 7 & 8 & 9 & 10 & 11 & 12 & 13 & 14 & 15 & 16 & 17 & 18 \\
\end{tabularx}
\vspace{0.5\baselineskip}
\noindent\begin{tabularx}{\linewidth}{*{12}{>{\centering\arraybackslash}X}}
 \textcolor{gregoriocolor}{A} & \textcolor{gregoriocolor}{B} & \textcolor{gregoriocolor}{C} & \textcolor{gregoriocolor}{D} & \textcolor{gregoriocolor}{E} & F & \textcolor{gregoriocolor}{F} & \textcolor{gregoriocolor}{G} & \textcolor{gregoriocolor}{H} & \textcolor{gregoriocolor}{M} & \textcolor{gregoriocolor}{N} & \textcolor{gregoriocolor}{P} \\
 19 & 20 & 21 & 22 & 23 & 24 & 24 & 25 & 26 & 27 & 28 & 29 \\
\end{tabularx}

\begin{paracol}{2}
\selectlanguage{latin}
\lettrine[lines=2]{H}{eliópoli,} apud Líbanum, 
 sancti Cyrílli, Diáconi et Mártyris, cujus jecur, e discísso ventre avúlsum, 
 Gentíles, sub Juliáno Apóstata, feráliter depásti sunt.
\switchcolumn
\selectlanguage{english}
\lettrine[lines=2]{A}{t} Heliopolis in Lebanon, under Julian the Apostate, St. Cyril, deacon and 
 martyr, whose body was opened and his liver taken out by the heathens who 
 devoured it like wild beasts.
\switchcolumn*
\selectlanguage{latin}
In Pérside sanctórum Monachórum et Mártyrum Jonæ 
 et Barachísii fratrum, sub Rege Persárum Sápore. Ex ipsis Jonas, 
 préssus in cóchlea, confráctis óssibus, médius disséctus est; Barachísius 
 autem, opplétis fáucibus pice ardénti suffocátus.
\switchcolumn
\selectlanguage{english}
In Persia, the holy martyrs Jonas and Barachisius, under the Persian king 
 Sapor. Jonas was put under the pressure of a vice, his bones broken, 
 and cut asunder; Barachisius was suffocated by burning pitch being poured 
 into his throat.
\switchcolumn*
\selectlanguage{latin}
Nicomedíæ pássio sanctórum Mártyrum Pastóris, 
 Victoríni et Sociórum.
\switchcolumn
\selectlanguage{english}
At Nicomedia, the passion of the holy martyrs Pastor, Victorinus, and their 
 companions.
\switchcolumn*
\selectlanguage{latin}
In Africa sanctórum Confessórum Armogástis Cómitis, 
 Másculæ archimími, et Satúri, régiæ domus procuratóris; qui, témpore 
 Wandálicæ persecutiónis, sub Rege Ariáno Genséríco, pro confessióne 
 veritátis, multa et grávia perpéssi supplícia atque oppróbria, cursum 
 gloriósi certáminis implevérunt.
\switchcolumn
\selectlanguage{english}
In Africa, under the Arian king Genseric, during the persecution of the 
 Vandals, the holy confessors Armogastes, a count, Mascula, Archimimus, and 
 Saturus, master of the king's household. After enduring many severe 
 torments, as well as insults, for the confession of the truth, they 
 completed their tests with glory.
\switchcolumn*
\selectlanguage{latin}
In urbe Asténsi sancti Secúndi Mártyris.
\switchcolumn
\selectlanguage{english}
In the town of Asti, St. Secundus, martyr.
\switchcolumn*
\selectlanguage{latin}
In monastério Luxoviénsi, in Gállia, deposítio sancti Eustásii Abbátis, qui 
 sancti Columbáni discípulus et ferme sexcentórum Monachórum Pater fuit; ac, 
 vitæ sanctitáte conspícuus, étiam miráculis 
 cláruit.
\switchcolumn
\selectlanguage{english}
In the monastery of Luxeuil, the death of Abbot St. Eustasius, a disciple of 
 St. Columban, who had under his guidance nearly six hundred monks. 
 Eminent in sanctity, he was also renowned for miracles.
\switchcolumn*
\selectlanguage{latin}
\end{paracol}


% ---- martyrology/mart03/mart0330.htm
\needspace{10\baselineskip}
\begin{paracol}{2}
\selectlanguage{latin}
\begin{center}{\color{gregoriocolor} Tértio Kaléndas Aprílis. 
 Luna\dots\ }\end{center}
\switchcolumn
\selectlanguage{english}
\begin{center}{\color{gregoriocolor} The Thirtieth Day of March. 
 The\dots\ Day of the Moon.}\end{center}
\end{paracol}

\noindent\begin{tabularx}{\linewidth}{*{19}{>{\centering\arraybackslash}X}}
 \textcolor{gregoriocolor}{a} & \textcolor{gregoriocolor}{b} & \textcolor{gregoriocolor}{c} & \textcolor{gregoriocolor}{d} & \textcolor{gregoriocolor}{e} & \textcolor{gregoriocolor}{f} & \textcolor{gregoriocolor}{g} & \textcolor{gregoriocolor}{h} & \textcolor{gregoriocolor}{i} & \textcolor{gregoriocolor}{k} & \textcolor{gregoriocolor}{l} & \textcolor{gregoriocolor}{m} & \textcolor{gregoriocolor}{n} & \textcolor{gregoriocolor}{p} & \textcolor{gregoriocolor}{q} & \textcolor{gregoriocolor}{r} & \textcolor{gregoriocolor}{s} & \textcolor{gregoriocolor}{t} & \textcolor{gregoriocolor}{u} \\
 1 & 2 & 3 & 4 & 5 & 6 & 7 & 8 & 9 & 10 & 11 & 12 & 13 & 14 & 15 & 16 & 17 & 18 & 19 \\
\end{tabularx}
\vspace{0.5\baselineskip}
\noindent\begin{tabularx}{\linewidth}{*{12}{>{\centering\arraybackslash}X}}
 \textcolor{gregoriocolor}{A} & \textcolor{gregoriocolor}{B} & \textcolor{gregoriocolor}{C} & \textcolor{gregoriocolor}{D} & \textcolor{gregoriocolor}{E} & F & \textcolor{gregoriocolor}{F} & \textcolor{gregoriocolor}{G} & \textcolor{gregoriocolor}{H} & \textcolor{gregoriocolor}{M} & \textcolor{gregoriocolor}{N} & \textcolor{gregoriocolor}{P} \\
 20 & 21 & 22 & 23 & 24 & 25 & 25 & 26 & 27 & 28 & 29 & 30 \\
\end{tabularx}

\begin{paracol}{2}
\selectlanguage{latin}
\lettrine[lines=2]{R}{omæ,} via Appia, pássio beáti Quiríni Tribúni, 
 patris sanctæ Balbínæ Vírginis, qui a beáto Alexándro Papa, quem habébat in 
 custódia, cum omni domo sua baptizátus est; atque, sub Hadriáno Imperatóre, 
 cum esset tráditus Aureliáno Júdici, et in fídei confessióne persísteret, 
 invíctus Christi miles, post linguæ abscissiónem, equúlei suspensiónem, 
 manuúmque ac pedum detruncatiónem, martyrii agónem gládio consummávit.
\switchcolumn
\selectlanguage{english}
\lettrine[lines=2]{A}{t} Rome, on the Appian Way, the martyrdom of the tribune blessed Quirinus, 
 who had been baptized with all his household by Pope St. Alexander when he 
 was imprisoned in their house. Under Emperor Adrian, he was delivered 
 to the judge Aurelian, and because he persevered in the confession of faith, 
 his tongue was torn out, he was stretched on the rack, his hands and feet 
 were cut off, and the sword completed his course of martyrdom.
\switchcolumn*
\selectlanguage{latin}
Thessalonícæ 
 natális sanctórum Mártyrum Domníni, Victóris et Sociórum.
\switchcolumn
\selectlanguage{english}
At Thessalonica, the birthday of the holy martyrs Domninus, Victor, and 
 their companions.
\switchcolumn*
\selectlanguage{latin}
Constantinópoli commemorátio sanctórum plurimórum Mártyrum cathólicæ 
 communiónis, quos, Constántii témpore, Macedónius hæresiárcha, inaudítis 
 tormentórum genéribus cruciátos, occídit; nam, inter cétera, fidélium 
 mulíerum úbera inter compréssa arcárum labra dissécuit, et candénti ferro 
 combússit.
\switchcolumn
\selectlanguage{english}
At Constantinople, in the time of Constantius, the commemoration of many 
 holy martyrs of the Catholic communion, whom the heresiarch Macedonius put 
 to death by unheard-of kinds of torments. Among other tortures, they 
 were burned with red-hot irons, and the breasts of Christian women were cut 
 away between the lids of coffers.
\switchcolumn*
\selectlanguage{latin}
In castro Silvanecténsi, in Gállia, deposítio sancti Réguli, Arelaténsis 
 Epíscopi.
\switchcolumn
\selectlanguage{english}
At Senlis in France, the death of St. Regulus, bishop of Arles.
\switchcolumn*
\selectlanguage{latin}
Auréliæ, in Gállia, sancti Pastóris Epíscopi.
\switchcolumn
\selectlanguage{english}
At Orleans in France, Bishop St. Pastor.
\switchcolumn*
\selectlanguage{latin}
Syracúsis, in Sicília, sancti Zósimi, 
 Epíscopi et Confessóris.
\switchcolumn
\selectlanguage{english}
At Syracuse, St. Zosimus, bishop and confessor.
\switchcolumn*
\selectlanguage{latin}
In monte Sina sancti Joánnis Clímaci Abbátis.
\switchcolumn
\selectlanguage{english}
On Mount Sinai, Abbot St. John Climacus.
\switchcolumn*
\selectlanguage{latin}
Aquilériæ, in Hispánia, sancti Petri Regaláti, 
 in urbe Vallisoletána orti, Sacerdótis ex Ordine Minórum et Confessóris, 
 reguláris disciplínæ in Hispániæ cœnóbiis restitutóris; quem Benedíctus 
 Décimus quartus, Póntifex Máximus, Sanctórum fastis adscrípsit.
\switchcolumn
\selectlanguage{english}
At Aquileria in Spain, the confessor St. Peter Regalado, priest of the 
 Order of Friars Minor. He was born in Valladolid, and restored the 
 regular discipline in the Spanish monasteries. Pope Benedict XIV 
 placed him on the roll of saints.
\switchcolumn*
\selectlanguage{latin}
Apud Aquínum sancti Clínii Confessóris.
\switchcolumn
\selectlanguage{english}
At Aquino, St. Clinius confessor.
\switchcolumn*
\selectlanguage{latin}
\end{paracol}


% ---- martyrology/mart03/mart0331.htm
\needspace{10\baselineskip}
\begin{paracol}{2}
\selectlanguage{latin}
\begin{center}{\color{gregoriocolor} Prídie Kaléndas Aprílis. 
 Luna\dots\ }\end{center}
\switchcolumn
\selectlanguage{english}
\begin{center}{\color{gregoriocolor} The Thirty-First Day of March. 
 The\dots\ Day of the Moon.}\end{center}
\end{paracol}

\noindent\begin{tabularx}{\linewidth}{*{19}{>{\centering\arraybackslash}X}}
 \textcolor{gregoriocolor}{a} & \textcolor{gregoriocolor}{b} & \textcolor{gregoriocolor}{c} & \textcolor{gregoriocolor}{d} & \textcolor{gregoriocolor}{e} & \textcolor{gregoriocolor}{f} & \textcolor{gregoriocolor}{g} & \textcolor{gregoriocolor}{h} & \textcolor{gregoriocolor}{i} & \textcolor{gregoriocolor}{k} & \textcolor{gregoriocolor}{l} & \textcolor{gregoriocolor}{m} & \textcolor{gregoriocolor}{n} & \textcolor{gregoriocolor}{p} & \textcolor{gregoriocolor}{q} & \textcolor{gregoriocolor}{r} & \textcolor{gregoriocolor}{s} & \textcolor{gregoriocolor}{t} & \textcolor{gregoriocolor}{u} \\
 2 & 3 & 4 & 5 & 6 & 7 & 8 & 9 & 10 & 11 & 12 & 13 & 14 & 15 & 16 & 17 & 18 & 19 & 20 \\
\end{tabularx}
\vspace{0.5\baselineskip}
\noindent\begin{tabularx}{\linewidth}{*{12}{>{\centering\arraybackslash}X}}
 \textcolor{gregoriocolor}{A} & \textcolor{gregoriocolor}{B} & \textcolor{gregoriocolor}{C} & \textcolor{gregoriocolor}{D} & \textcolor{gregoriocolor}{E} & F & \textcolor{gregoriocolor}{F} & \textcolor{gregoriocolor}{G} & \textcolor{gregoriocolor}{H} & \textcolor{gregoriocolor}{M} & \textcolor{gregoriocolor}{N} & \textcolor{gregoriocolor}{P} \\
 21 & 22 & 23 & 24 & 25 & 26 & 26 & 27 & 28 & 29 & 30 & 1 \\
\end{tabularx}

\begin{paracol}{2}
\selectlanguage{latin}
\lettrine[lines=2]{T}{hécuæ,} in Palæstína, sancti Amos Prophétæ, qui 
 ab Amasía Sacerdóte frequénter plagis afflíctus est, atque ab hujus fílio 
 Ozía vecte per témpora transfíxus; et póstea, semivívus in pátriam devéctus, 
 ibídem exspirávit, sepultúsque est cum pátribus suis.
\switchcolumn
\selectlanguage{english}
\lettrine[lines=2]{A}{t} Thecua in Palestine, the holy prophet Amos, whom the priest Amasias 
 frequently had scourged. Ozias, that priest's son, pierced his head at 
 the temples with an iron spike. Being carried half dead to his own 
 country, he died there, and was buried with his family.
\switchcolumn*
\selectlanguage{latin}
In Pérside sancti Bénjamin Diáconi, qui, cum Dei verbum non desísteret prædicáre, 
 ídeo, sub Isdegérde Rege, arundínibus acútis confíxus únguibus, et spinósa 
 sude per alvum transmíssa, martyrium consummávit.
\switchcolumn
\selectlanguage{english}
In Persia, during the reign of King Isdegerdes, the deacon St. Benjamin. 
 Because he would not stop preaching the word of God, he had a sharp reed 
 forced under his nails, a thorny stake driven through his body, and thus 
 completed his martyrdom.
\switchcolumn*
\selectlanguage{latin}
In Africa sanctórum Mártyrum Theodúli, Anésii, Felícis, Cornéliæ 
 et Sociórum.
\switchcolumn
\selectlanguage{english}
In Africa, the holy martyrs Theodulus, Anesius, Felix, Cornelia, and their 
 companions.
\switchcolumn*
\selectlanguage{latin}
Romæ sanctæ Balbínæ Vírginis, fíliæ beáti 
 Quiríni Mártyris, quæ, a sancto Alexándro Papa baptizáta, in sancta 
 virginitáte Christum sibi sponsum elégit; et, post devíctum hujus sæculi 
 cursum, sepúlta est via Appia, juxta patrem suum.
\switchcolumn
\selectlanguage{english}
At Rome, the virgin St. Balbina, daughter of the blessed martyr Quirinus. 
 She was baptized by Pope Alexander, and she chose Christ as her spouse in 
 her virginity. After overcoming the world, she was buried at her 
 father's side on the Appian Way.
\switchcolumn*
\selectlanguage{latin}
\end{paracol}

\setrunningtitles{Aprilis}{April}

% ---- martyrology/mart04/mart0401.htm
\needspace{10\baselineskip}
\begin{paracol}{2}
\selectlanguage{latin}
\begin{center}{\color{gregoriocolor} Kaléndis Aprílis. 
 Luna\dots\ }\end{center}
\switchcolumn
\selectlanguage{english}
\begin{center}{\color{gregoriocolor} The First Day of April. 
 The\dots\ Day of the Moon.}\end{center}
\end{paracol}

\noindent\begin{tabularx}{\linewidth}{*{19}{>{\centering\arraybackslash}X}}
 \textcolor{gregoriocolor}{a} & \textcolor{gregoriocolor}{b} & \textcolor{gregoriocolor}{c} & \textcolor{gregoriocolor}{d} & \textcolor{gregoriocolor}{e} & \textcolor{gregoriocolor}{f} & \textcolor{gregoriocolor}{g} & \textcolor{gregoriocolor}{h} & \textcolor{gregoriocolor}{i} & \textcolor{gregoriocolor}{k} & \textcolor{gregoriocolor}{l} & \textcolor{gregoriocolor}{m} & \textcolor{gregoriocolor}{n} & \textcolor{gregoriocolor}{p} & \textcolor{gregoriocolor}{q} & \textcolor{gregoriocolor}{r} & \textcolor{gregoriocolor}{s} & \textcolor{gregoriocolor}{t} & \textcolor{gregoriocolor}{u} \\
 3 & 4 & 5 & 6 & 7 & 8 & 9 & 10 & 11 & 12 & 13 & 14 & 15 & 16 & 17 & 18 & 19 & 20 & 21 \\
\end{tabularx}
\vspace{0.5\baselineskip}
\noindent\begin{tabularx}{\linewidth}{*{12}{>{\centering\arraybackslash}X}}
 \textcolor{gregoriocolor}{A} & \textcolor{gregoriocolor}{B} & \textcolor{gregoriocolor}{C} & \textcolor{gregoriocolor}{D} & \textcolor{gregoriocolor}{E} & F & \textcolor{gregoriocolor}{F} & \textcolor{gregoriocolor}{G} & \textcolor{gregoriocolor}{H} & \textcolor{gregoriocolor}{M} & \textcolor{gregoriocolor}{N} & \textcolor{gregoriocolor}{P} \\
 22 & 23 & 24 & 25 & 26 & 27 & 27 & 28 & 29 & 30 & 1 & 2 \\
\end{tabularx}

\begin{paracol}{2}
\selectlanguage{latin}
\lettrine[lines=2]{R}{omæ} pássio sanctæ Theodóræ, soróris 
 illustríssimi Mártyris Hermétis, quæ, sub Hadriáno Imperatóre, ab Aureliáno 
 Júdice affécta martyrio, sepúlta est juxta fratrem suum, via Salária, non 
 longe ab Urbe.
\switchcolumn
\selectlanguage{english}
\lettrine[lines=2]{A}{t} Rome, the passion of St. Theodora, sister of the illustrious martyr 
 Hermes. She underwent martyrdom in the time of Emperor Adrian, under 
 the judge Aurelian, and was buried at the side of her brother, on the 
 Salarian Way, a short distance from the city.
\switchcolumn*
\selectlanguage{latin}
Eódem die sancti Venántii, Epíscopi et Mártyris.
\switchcolumn
\selectlanguage{english}
The same day, St. Venantius, bishop and martyr.
\switchcolumn*
\selectlanguage{latin}
In Ægypto sanctórum Mártyrum Victóris et 
 Stéphani.
\switchcolumn
\selectlanguage{english}
In Egypt, the holy martyrs Victor and Stephen.
\switchcolumn*
\selectlanguage{latin}
In Arménia sanctórum Mártyrum Quinctiáni et Irenǽi.
\switchcolumn
\selectlanguage{english}
In Armenia, the holy martyrs Quinctian and Irenæus.
\switchcolumn*
\selectlanguage{latin}
Constantinópoli sancti Macárii Confessóris, qui, sub Leóne Imperatóre, pro 
 assertióne sanctárum Imáginum, in exsílio vitam finívit.
\switchcolumn
\selectlanguage{english}
At Constantinople, under Emperor Leo, St. Macarius, confessor, who ended his 
 life in exile for defending the veneration of sacred images.
\switchcolumn*
\selectlanguage{latin}
Ardpatrícii, in Momónia Hibérniæ província, 
 sancti Celsi Epíscopi, qui beátum Malachíam in Episcopátu præcéssit.
\switchcolumn
\selectlanguage{english}
At Ard-Patrick in Munster, a province of Ireland, Bishop St. Celsus, who 
 preceded blessed Malachy in that bishopric.
\switchcolumn*
\selectlanguage{latin}
Gratianópoli, in Gállia, sancti Hugónis Epíscopi, qui multis annis in 
 solitúdine vitam exégit, et miraculórum glória clarus migrávit ad Dóminum.
\switchcolumn
\selectlanguage{english}
At Grenoble in France, Bishop St. Hugh, who spent many years of his life in 
 solitude, and departed for heaven with a great reputation for miracles.
\switchcolumn*
\selectlanguage{latin}
Apud Ambiánum, in Gállia, sancti Waleríci 
 Abbátis, cujus sepúlcrum crebris miráculis illustrátur.
\switchcolumn
\selectlanguage{english}
At Amiens in France, Abbot St. Valery, whose tomb is well known for its 
 frequent miracles.
\switchcolumn*
\selectlanguage{latin}
\end{paracol}


% ---- martyrology/mart04/mart0402.htm
\needspace{10\baselineskip}
\begin{paracol}{2}
\selectlanguage{latin}
\begin{center}{\color{gregoriocolor} Quarto Nonas Aprílis. 
 Luna\dots\ }\end{center}
\switchcolumn
\selectlanguage{english}
\begin{center}{\color{gregoriocolor} The Second Day of April. 
 The\dots\ Day of the Moon.}\end{center}
\end{paracol}

\noindent\begin{tabularx}{\linewidth}{*{19}{>{\centering\arraybackslash}X}}
 \textcolor{gregoriocolor}{a} & \textcolor{gregoriocolor}{b} & \textcolor{gregoriocolor}{c} & \textcolor{gregoriocolor}{d} & \textcolor{gregoriocolor}{e} & \textcolor{gregoriocolor}{f} & \textcolor{gregoriocolor}{g} & \textcolor{gregoriocolor}{h} & \textcolor{gregoriocolor}{i} & \textcolor{gregoriocolor}{k} & \textcolor{gregoriocolor}{l} & \textcolor{gregoriocolor}{m} & \textcolor{gregoriocolor}{n} & \textcolor{gregoriocolor}{p} & \textcolor{gregoriocolor}{q} & \textcolor{gregoriocolor}{r} & \textcolor{gregoriocolor}{s} & \textcolor{gregoriocolor}{t} & \textcolor{gregoriocolor}{u} \\
 4 & 5 & 6 & 7 & 8 & 9 & 10 & 11 & 12 & 13 & 14 & 15 & 16 & 17 & 18 & 19 & 20 & 21 & 22 \\
\end{tabularx}
\vspace{0.5\baselineskip}
\noindent\begin{tabularx}{\linewidth}{*{12}{>{\centering\arraybackslash}X}}
 \textcolor{gregoriocolor}{A} & \textcolor{gregoriocolor}{B} & \textcolor{gregoriocolor}{C} & \textcolor{gregoriocolor}{D} & \textcolor{gregoriocolor}{E} & F & \textcolor{gregoriocolor}{F} & \textcolor{gregoriocolor}{G} & \textcolor{gregoriocolor}{H} & \textcolor{gregoriocolor}{M} & \textcolor{gregoriocolor}{N} & \textcolor{gregoriocolor}{P} \\
 23 & 24 & 25 & 26 & 27 & 28 & 28 & 29 & 30 & 1 & 2 & 3 \\
\end{tabularx}

\begin{paracol}{2}
\selectlanguage{latin}
\lettrine[lines=2]{T}{urónis,} in Gállia, sancti Francísci de Paula Confessóris, qui Ordinis 
 Minimórum Institútor éxstitit; atque, virtútibus et miráculis clarus, a 
 Leóne Papa Décimo in Sanctórum númerum est adscríptus.
\switchcolumn
\selectlanguage{english}
\lettrine[lines=2]{A}{t} Tours in France, St. Francis of Paula, founder of the Order of Minims. 
 Because he was renowned for virtues and miracles, he was inscribed among the 
 saints by Pope Leo X.
\switchcolumn*
\selectlanguage{latin}
Cæsaréæ, in Palæstína, natális sancti Apphiáni 
 Mártyris, qui, ante sanctum Ædésium Mártyrem, fratrem suum, in persecutióne 
 Galérii Maximiáni, cum Prǽsidem Urbánum idólis immolántem arguísset, sæve 
 dilaniátus est, et, pédibus lino in óleum intíncto eóque incénso obvolútis, 
 acerbíssime cruciátus, ac demum in mare demérsus; atque ita, tránsiens per 
 ignem et aquam, edúctus est in refrigérium.
\switchcolumn
\selectlanguage{english}
At Caesarea in Palestine, during the persecution of Galerius Maximian, the 
 birthday of the martyr St. Amphian. He reproved the governor Urban for 
 sacrificing to idols, so his body was cruelly cut in shreds, his feet 
 wrapped in oil-soaked cloths, and set on fire. After these painful 
 torments, he was cast into the sea. Thus through fire and water, he 
 reached his everlasting repose.
\switchcolumn*
\selectlanguage{latin}
Ibídem pássio sanctæ Theodósiæ, Vírginis Tyriæ, 
 quæ, in eádem persecutióne, cum sanctos Confessóres ante tribúnal stantes 
 públice salutásset, atque rogásset eos, ut, cum ad Dóminum perveníssent, sui 
 recordaréntur, a milítibus tenta et ad Urbánum Prǽsidem ducta est; eóque 
 jubénte, latéribus et mammis ad interióra usque dilaniátis, in mare tandem 
 projícitur.
\switchcolumn
\selectlanguage{english}
In the same city, the passion of St. Theodosia, a virgin of Tyre. In 
 the same persecution, she publicly spoke to the holy confessors as they 
 stood before the tribunal, and begged of them to remember her when they 
 should be with God. She was arrested and led to the governor Urban, at 
 whose order her sides and breasts were deeply lacerated, and she was thrown 
 into the sea.
\switchcolumn*
\selectlanguage{latin}
Apud Língonas, in Gállia, sancti Urbáni 
 Epíscopi.
\switchcolumn
\selectlanguage{english}
At Langres in France, Bishop St. Urban.
\switchcolumn*
\selectlanguage{latin}
Apud Comum sancti Abúndii, Epíscopi et Confessóris.
\switchcolumn
\selectlanguage{english}
At Como, St. Abundius, bishop and confessor.
\switchcolumn*
\selectlanguage{latin}
Cápuæ sancti Victóris Epíscopi, eruditióne et 
 sanctitáte conspícui.
\switchcolumn
\selectlanguage{english}
At Capua, Bishop St. Victor, well known for his sanctity and learning.
\switchcolumn*
\selectlanguage{latin}
Lugdúni, in Gállia, sancti Nicétii, ejúsdem urbis Epíscopi, vita et 
 miráculis clari.
\switchcolumn
\selectlanguage{english}
At Lyons in France, St. Nicetus, bishop of that city, renowned for his life 
 and miracles.
\switchcolumn*
\selectlanguage{latin}
In Palæstína deposítio sanctæ Maríæ Ægyptíacæ, 
 quæ peccátrix appellátur.
\switchcolumn
\selectlanguage{english}
In Palestine, the death of St. Mary of Egypt, called the Sinner.
\switchcolumn*
\selectlanguage{latin}
\end{paracol}


% ---- martyrology/mart04/mart0403.htm
\needspace{10\baselineskip}
\begin{paracol}{2}
\selectlanguage{latin}
\begin{center}{\color{gregoriocolor} Tértio Nonas Aprílis. 
 Luna\dots\ }\end{center}
\switchcolumn
\selectlanguage{english}
\begin{center}{\color{gregoriocolor} The Third Day of April. 
 The\dots\ Day of the Moon.}\end{center}
\end{paracol}

\noindent\begin{tabularx}{\linewidth}{*{19}{>{\centering\arraybackslash}X}}
 \textcolor{gregoriocolor}{a} & \textcolor{gregoriocolor}{b} & \textcolor{gregoriocolor}{c} & \textcolor{gregoriocolor}{d} & \textcolor{gregoriocolor}{e} & \textcolor{gregoriocolor}{f} & \textcolor{gregoriocolor}{g} & \textcolor{gregoriocolor}{h} & \textcolor{gregoriocolor}{i} & \textcolor{gregoriocolor}{k} & \textcolor{gregoriocolor}{l} & \textcolor{gregoriocolor}{m} & \textcolor{gregoriocolor}{n} & \textcolor{gregoriocolor}{p} & \textcolor{gregoriocolor}{q} & \textcolor{gregoriocolor}{r} & \textcolor{gregoriocolor}{s} & \textcolor{gregoriocolor}{t} & \textcolor{gregoriocolor}{u} \\
 5 & 6 & 7 & 8 & 9 & 10 & 11 & 12 & 13 & 14 & 15 & 16 & 17 & 18 & 19 & 20 & 21 & 22 & 23 \\
\end{tabularx}
\vspace{0.5\baselineskip}
\noindent\begin{tabularx}{\linewidth}{*{12}{>{\centering\arraybackslash}X}}
 \textcolor{gregoriocolor}{A} & \textcolor{gregoriocolor}{B} & \textcolor{gregoriocolor}{C} & \textcolor{gregoriocolor}{D} & \textcolor{gregoriocolor}{E} & F & \textcolor{gregoriocolor}{F} & \textcolor{gregoriocolor}{G} & \textcolor{gregoriocolor}{H} & \textcolor{gregoriocolor}{M} & \textcolor{gregoriocolor}{N} & \textcolor{gregoriocolor}{P} \\
 24 & 25 & 26 & 27 & 28 & 29 & 29 & 30 & 1 & 2 & 3 & 4 \\
\end{tabularx}

\begin{paracol}{2}
\selectlanguage{latin}
\lettrine[lines=2]{R}{omæ} natális beáti Xysti Primi, Papæ et 
 Mártyris; qui, tempóribus Hadriáni Imperatóris, summa cum laude rexit 
 Ecclésiam, ac demum, sub Antoníno Pio, ut sibi Christum lucrifáceret, 
 libénter mortem sustínuit temporálem.
\switchcolumn
\selectlanguage{english}
\lettrine[lines=2]{A}{t} Rome, the birthday of blessed Pope Sixtus the First, martyr, who ruled 
 the Church with distinction during the reign of Emperor Hadrian, and finally 
 in the reign of Antoninus Pius he gladly accepted temporal death in order to 
 gain Christ for himself.
\switchcolumn*
\selectlanguage{latin}
Tauroménii, in Sicília, sancti Pancrátii Epíscopi, qui Christi Evangélium, 
 quod a sancto Petro Apóstolo illuc missus prædicáverat, 
 martyrii sánguine consignávit.
\switchcolumn
\selectlanguage{english}
At Taormina in Sicily, Bishop St. Pancras, who sealed with a martyr's blood 
 the Gospel of Christ that the apostle St. Peter had sent him there to 
 preach.
\switchcolumn*
\selectlanguage{latin}
Tomis, in Scythia, natális sanctórum Mártyrum Evágrii et Benígni.
\switchcolumn
\selectlanguage{english}
At Tomis in Scythia, the birthday of the holy martyrs Evagrius and Benignus.
\switchcolumn*
\selectlanguage{latin}
Tyri, in Phœnícia, sancti Vulpiáni Mártyris, 
 qui, in persecutióne Maximiáni Galérii, cum áspide et cane insútus cúleo, in 
 mare demérsus fuit.
\switchcolumn
\selectlanguage{english}
At Tyre, the martyr St. Vulpian, who was sewn up in a sack with a serpent 
 and a dog and drowned in the sea, during the persecution of Maximian 
 Galerius.
\switchcolumn*
\selectlanguage{latin}
Thessalonícæ 
 pássio sanctárum Vírginum Agapis et Chióniæ, Diocletiáno Imperatóre, sub quo 
 et sancta Virgo Iréne, eárum soror, póstmodum passúra erat. Ambæ vero, 
 cum Christum negáre nollent, primum in cárcere macerátæ sunt, póstea in 
 ignem missæ, sed a flammis intáctæ, ibi, oratióne ad Dóminum fusa, ánimas 
 reddidérunt.
\switchcolumn
\selectlanguage{english}
At Thessalonica, the martyrdom of the holy virgins Agape and Chionia, under 
 Emperor Diocletian. Because they would not deny Christ, they were 
 first detained in prison, then cast into the fire where, untouched by the 
 flames, they gave up their souls to their Creator while praying. Their 
 sister Irene had been imprisoned with them, but was to die later.
\switchcolumn*
\selectlanguage{latin}
In monastério Medícii, in Bithynia, deposítio sancti Nicétæ 
 Abbátis, qui ob cultum sanctárum Imáginum, sub Leóne Arméno, multa passus 
 est, ac tandem, juxta Constantinópolim, Conféssor quiévit in pace.
\switchcolumn
\selectlanguage{english}
In the monastery of Medicion in Bithynia, Abbot St. Nicetas, who suffered a 
 great deal for the veneration of sacred images in the time of Leo the 
 Armenian, and then died in peace as a confessor near Constantinople.
\switchcolumn*
\selectlanguage{latin}
In Anglia sancti Richárdii, Epíscopi Cicestrénsis, sanctitáte et miraculórum 
 glória conspícui.
\switchcolumn
\selectlanguage{english}
In England, St. Richard, bishop of Chichester, celebrated for his sanctity 
 and glorious miracles.
\switchcolumn*
\selectlanguage{latin}
Eboríaci, in território Meldénsi, sanctæ 
 Burgundofáræ, étiam Faræ nómine appellátæ, Abbatíssæ et Vírginis.
\switchcolumn
\selectlanguage{english}
At Faremoutiers, in the district of Meaux, St. Burgundofara, also known as 
 St. Fara, abbess and virgin.
\switchcolumn*
\selectlanguage{latin}
\end{paracol}


% ---- martyrology/mart04/mart0404.htm
\needspace{10\baselineskip}
\begin{paracol}{2}
\selectlanguage{latin}
\begin{center}{\color{gregoriocolor} Prídie Nonas Aprílis. 
 Luna\dots\ }\end{center}
\switchcolumn
\selectlanguage{english}
\begin{center}{\color{gregoriocolor} The Fourth Day of April. 
 The\dots\ Day of the Moon.}\end{center}
\end{paracol}

\noindent\begin{tabularx}{\linewidth}{*{19}{>{\centering\arraybackslash}X}}
 \textcolor{gregoriocolor}{a} & \textcolor{gregoriocolor}{b} & \textcolor{gregoriocolor}{c} & \textcolor{gregoriocolor}{d} & \textcolor{gregoriocolor}{e} & \textcolor{gregoriocolor}{f} & \textcolor{gregoriocolor}{g} & \textcolor{gregoriocolor}{h} & \textcolor{gregoriocolor}{i} & \textcolor{gregoriocolor}{k} & \textcolor{gregoriocolor}{l} & \textcolor{gregoriocolor}{m} & \textcolor{gregoriocolor}{n} & \textcolor{gregoriocolor}{p} & \textcolor{gregoriocolor}{q} & \textcolor{gregoriocolor}{r} & \textcolor{gregoriocolor}{s} & \textcolor{gregoriocolor}{t} & \textcolor{gregoriocolor}{u} \\
 6 & 7 & 8 & 9 & 10 & 11 & 12 & 13 & 14 & 15 & 16 & 17 & 18 & 19 & 20 & 21 & 22 & 23 & 24 \\
\end{tabularx}
\vspace{0.5\baselineskip}
\noindent\begin{tabularx}{\linewidth}{*{12}{>{\centering\arraybackslash}X}}
 \textcolor{gregoriocolor}{A} & \textcolor{gregoriocolor}{B} & \textcolor{gregoriocolor}{C} & \textcolor{gregoriocolor}{D} & \textcolor{gregoriocolor}{E} & F & \textcolor{gregoriocolor}{F} & \textcolor{gregoriocolor}{G} & \textcolor{gregoriocolor}{H} & \textcolor{gregoriocolor}{M} & \textcolor{gregoriocolor}{N} & \textcolor{gregoriocolor}{P} \\
 25 & 26 & 27 & 28 & 29 & 1 & 30 & 1 & 2 & 3 & 4 & 5 \\
\end{tabularx}

\begin{paracol}{2}
\selectlanguage{latin}
\lettrine[lines=2]{H}{íspali,} in Hispánia, sancti Isidóri Epíscopi, 
 Confessóris et Ecclésiæ Doctóris, sanctitáte et doctrína conspícui; qui zelo 
 cathólicæ fidei et ecclesiásticæ observántia disciplinæ Hispánias 
 illustrávit.
\switchcolumn
\selectlanguage{english}
\lettrine[lines=2]{A}{t} Seville in Spain, St. Isidore, bishop, confessor, and doctor of the 
 Church. He was conspicuous for sanctity and learning, and had 
 brightened all Spain by his zeal for the Catholic faith and his observance 
 of Church discipline.
\switchcolumn*
\selectlanguage{latin}
Medioláni deposítio sancti Ambrósii Epíscopi, Confessóris et Ecclésiæ 
 Doctóris; cujus stúdio, inter cétera doctrínæ et miraculórum insígnia, 
 témpore Ariánæ perfídiæ, tota fere Itália ad cathólicam fidem convérsa est. 
 Ipsíus tamen festívitas séptimo Idus Decémbris potíssimum recólitur, quo die 
 Epíscopus Mediolanénsis ordinátus est.
\switchcolumn
\selectlanguage{english}
At Milan, the death of St. Ambrose, bishop and confessor, doctor of the 
 Church. By his zeal, besides other monuments to his learning and 
 miracles, almost all Italy returned to the Catholic faith at the time of the 
 Arian heresy. His feast is properly kept on the seventh of December, 
 on which day he became Bishop of Milan.
\switchcolumn*
\selectlanguage{latin}
Thessalonícæ sanctórum Mártyrum Agathópodis 
 Diáconi, et Theodúli Lectóris, qui, sub Maximiáno Imperatóre et Faustíno Prǽside, ob Christiánæ fidei confessiónem, in mare, 
 alligáto ad collum saxo, 
 demérsi sunt.
\switchcolumn
\selectlanguage{english}
At Thessalonica, in the time of Emperor Maximian and the governor Faustinus, 
 the holy martyrs Agathopodes, a deacon, and Theodulus, a lector, who, for 
 the confession of the Catholic faith, had stones tied to their necks and 
 were drowned in the sea.
\switchcolumn*
\selectlanguage{latin}
Constantinópoli sancti Platónis Mónachi, qui plúribus annis advérsus hæréticos, 
 sanctárum Imáginum effractóres, invícto ánimo decertávit.
\switchcolumn
\selectlanguage{english}
At Constantinople, the monk St. Plato. For many years he combated with 
 dauntless courage the heretics bent on destroying sacred images.
\switchcolumn*
\selectlanguage{latin}
In Palæstína sancti Zósimi Anachorétæ, qui 
 funus sanctæ Maríæ Ægyptíacæ curávit.
\switchcolumn
\selectlanguage{english}
In Palestine, the anchoret St. Zosimus, who took care of the funeral 
 of St. Mary of Egypt.
\switchcolumn*
\selectlanguage{latin}
Panórmi sancti Benedícti a sancto Philadélpho, ob córporis nigrédinem 
 cognoménto Nigri, ex Ordine Minórum, Confessóris; qui, signis et virtútibus 
 clarus, in Dómino quiévit, et a Pio Séptimo, Pontífice Máximo, in Sanctórum 
 númerum relátus est.
\switchcolumn
\selectlanguage{english}
At Palermo, St. Benedict of St. Philadelphus, called the Black because of 
 the darkness of his body, a confessor of the Order of Friars Minor. 
 After becoming outstanding for signs and virtues, he went to rest in the 
 Lord, and was enrolled among the saints by Pope Pius VII.
\switchcolumn*
\selectlanguage{latin}
\end{paracol}


% ---- martyrology/mart04/mart0405.htm
\needspace{10\baselineskip}
\begin{paracol}{2}
\selectlanguage{latin}
\begin{center}{\color{gregoriocolor} Nonis Aprílis. 
 Luna\dots\ }\end{center}
\switchcolumn
\selectlanguage{english}
\begin{center}{\color{gregoriocolor} The Fifth Day of April. 
 The\dots\ Day of the Moon.}\end{center}
\end{paracol}

\noindent\begin{tabularx}{\linewidth}{*{19}{>{\centering\arraybackslash}X}}
 \textcolor{gregoriocolor}{a} & \textcolor{gregoriocolor}{b} & \textcolor{gregoriocolor}{c} & \textcolor{gregoriocolor}{d} & \textcolor{gregoriocolor}{e} & \textcolor{gregoriocolor}{f} & \textcolor{gregoriocolor}{g} & \textcolor{gregoriocolor}{h} & \textcolor{gregoriocolor}{i} & \textcolor{gregoriocolor}{k} & \textcolor{gregoriocolor}{l} & \textcolor{gregoriocolor}{m} & \textcolor{gregoriocolor}{n} & \textcolor{gregoriocolor}{p} & \textcolor{gregoriocolor}{q} & \textcolor{gregoriocolor}{r} & \textcolor{gregoriocolor}{s} & \textcolor{gregoriocolor}{t} & \textcolor{gregoriocolor}{u} \\
 7 & 8 & 9 & 10 & 11 & 12 & 13 & 14 & 15 & 16 & 17 & 18 & 19 & 20 & 21 & 22 & 23 & 24 & 25 \\
\end{tabularx}
\vspace{0.5\baselineskip}
\noindent\begin{tabularx}{\linewidth}{*{12}{>{\centering\arraybackslash}X}}
 \textcolor{gregoriocolor}{A} & \textcolor{gregoriocolor}{B} & \textcolor{gregoriocolor}{C} & \textcolor{gregoriocolor}{D} & \textcolor{gregoriocolor}{E} & F & \textcolor{gregoriocolor}{F} & \textcolor{gregoriocolor}{G} & \textcolor{gregoriocolor}{H} & \textcolor{gregoriocolor}{M} & \textcolor{gregoriocolor}{N} & \textcolor{gregoriocolor}{P} \\
 26 & 27 & 28 & 29 & 1 & 2 & 1 & 2 & 3 & 4 & 5 & 6 \\
\end{tabularx}

\begin{paracol}{2}
\selectlanguage{latin}
\lettrine[lines=2]{V}{enétiæ,} in Británnia minóre, sancti Vincéntii, 
 cognoménto Ferrérii, ex Ordine Prædicatórum, Confessóris, qui, potens ópere 
 et sermóne, multa míllia infidélium convértit ad Christum.
\switchcolumn
\selectlanguage{english}
\lettrine[lines=2]{A}{t} Vannes in Brittany, St. Vincent Ferrer, of the Order of Preachers, and 
 confessor. He was mighty in word and deed, and converted many 
 thousands of infidels to Christ.
\switchcolumn*
\selectlanguage{latin}
In Africa pássio sanctórum Mártyrum, qui, in persecutióne Regis Ariáni 
 Genseríci, in die Paschæ, in Ecclésia cæsi sunt, 
 quorum Lector, dum in púlpito « ALLELUJA » cantáret, sagítta in gútture 
 transfíxus est.
\switchcolumn
\selectlanguage{english}
In Africa, during the persecution of the Arian king Genseric, the holy 
 martyrs who were murdered in the church on Easter day. The lector, 
 while singing ``Alleluia'` at the lectern, was pierced through the throat by 
 an arrow.
\switchcolumn*
\selectlanguage{latin}
Eódem die sancti Zenónis Mártyris, qui, pice íllitus et in ignem conjéctus, 
 et hasta in médio rogo vulnerátus, martyrio coronátus est.
\switchcolumn
\selectlanguage{english}
The same day, the martyr St. Zeno, who was covered with pitch, cast into the 
 fire, and wounded by the thrust of a spear, thus gaining the crown of 
 martyrdom.
\switchcolumn*
\selectlanguage{latin}
In Lesbo ínsula pássio sanctárum quinque vírginum, quæ 
 gládio martyrium consummárunt.
\switchcolumn
\selectlanguage{english}
On the island of Lesbos, the martyrdom of five holy virgins, who were slain 
 by the sword.
\switchcolumn*
\selectlanguage{latin}
Thessalonícæ sanctæ Irénes Vírginis, quæ, cum 
 sacros Libros contra Diocletiáni edíctum occultásset, ideo, post tolerántiam cárceris, sagítta percússa est et igne cremáta, jussu Dulcétii Præsidis; sub 
 quo et soróres ejus simul Agape et Chiónia passæ ántea fúerant.
\switchcolumn
\selectlanguage{english}
At Thessalonica, the virgin St. Irene, who was imprisoned for hiding the 
 sacred books, contrary to the order of Diocletian. She was pierced 
 with an arrow, then burned to death by order of the governor Dulcetius, 
 under whom her sisters Agape and Chionia had previously suffered.
\switchcolumn*
\selectlanguage{latin}
Palmæ, in ínsula Majórica, sanctæ Catharínæ 
 Thomas, Vírginis, Canoníssæ Reguláris, ex Ordine sancti Augustíni, quam Pius 
 Papa Undécimus in sanctárum Vírginum númerum rétulit.
\switchcolumn
\selectlanguage{english}
In the monastery at Palma, in the diocese of Majorca, the birthday of St. 
 Catalina Tomas, Canoness Regular of the Order of St. Augustine, whom Pope 
 Pius XI, in the fiftieth year of his priesthood, placed among the number of 
 virgin saints.
\switchcolumn*
\selectlanguage{latin}
\end{paracol}


% ---- martyrology/mart04/mart0406.htm
\needspace{10\baselineskip}
\begin{paracol}{2}
\selectlanguage{latin}
\begin{center}{\color{gregoriocolor} Octávo Idus Aprílis. 
 Luna\dots\ }\end{center}
\switchcolumn
\selectlanguage{english}
\begin{center}{\color{gregoriocolor} The Sixth Day of April. 
 The\dots\ Day of the Moon.}\end{center}
\end{paracol}

\noindent\begin{tabularx}{\linewidth}{*{19}{>{\centering\arraybackslash}X}}
 \textcolor{gregoriocolor}{a} & \textcolor{gregoriocolor}{b} & \textcolor{gregoriocolor}{c} & \textcolor{gregoriocolor}{d} & \textcolor{gregoriocolor}{e} & \textcolor{gregoriocolor}{f} & \textcolor{gregoriocolor}{g} & \textcolor{gregoriocolor}{h} & \textcolor{gregoriocolor}{i} & \textcolor{gregoriocolor}{k} & \textcolor{gregoriocolor}{l} & \textcolor{gregoriocolor}{m} & \textcolor{gregoriocolor}{n} & \textcolor{gregoriocolor}{p} & \textcolor{gregoriocolor}{q} & \textcolor{gregoriocolor}{r} & \textcolor{gregoriocolor}{s} & \textcolor{gregoriocolor}{t} & \textcolor{gregoriocolor}{u} \\
 8 & 9 & 10 & 11 & 12 & 13 & 14 & 15 & 16 & 17 & 18 & 19 & 20 & 21 & 22 & 23 & 24 & 25 & 26 \\
\end{tabularx}
\vspace{0.5\baselineskip}
\noindent\begin{tabularx}{\linewidth}{*{12}{>{\centering\arraybackslash}X}}
 \textcolor{gregoriocolor}{A} & \textcolor{gregoriocolor}{B} & \textcolor{gregoriocolor}{C} & \textcolor{gregoriocolor}{D} & \textcolor{gregoriocolor}{E} & F & \textcolor{gregoriocolor}{F} & \textcolor{gregoriocolor}{G} & \textcolor{gregoriocolor}{H} & \textcolor{gregoriocolor}{M} & \textcolor{gregoriocolor}{N} & \textcolor{gregoriocolor}{P} \\
 27 & 28 & 29 & 1 & 2 & 3 & 2 & 3 & 4 & 5 & 6 & 7 \\
\end{tabularx}

\begin{paracol}{2}
\selectlanguage{latin}
\lettrine[lines=2]{M}{edioláni} pássio sancti Petri, ex Ordine Prædicatórum, 
 Mártyris, qui ab hæréticis, ob fidem cathólicam, interémptus est. 
 Ipsíus tamen festívitas recólitur tértio Kaléndas Maji.
\switchcolumn
\selectlanguage{english}
\lettrine[lines=2]{A}{t} Milan, the passion of St. Peter, a martyr belonging to the Order of 
 Preachers, who was slain by the heretics for his Catholic faith. His 
 feast, however, is kept on the 29th of April.
\switchcolumn*
\selectlanguage{latin}
Velehrádii, in Morávia, natális sancti Methódii, Epíscopi et Confessóris, 
 qui, una cum sancto Cyríllo, item Epíscopo et fratre suo, cujus natális 
 sextodécimo Kaléndas Mártii recensétur, multas Slávicas gentes earúmque 
 Reges ad fidem Christi perdúxit. Horum autem Sanctórum festum Nonis 
 Júlii celebrátur.
\switchcolumn
\selectlanguage{english}
In Moravia, the birthday of St. Methodius, bishop and confessor. 
 Together with his brother, the bishop St. Cyril, whose birthday was the 14th 
 of February, he converted many of the Slav races and their rulers to the 
 faith of Christ. Their feast is celebrated on the 7th day of July.
\switchcolumn*
\selectlanguage{latin}
In Macedónia sanctórum Mártyrum Timóthei et Diógenis.
\switchcolumn
\selectlanguage{english}
In Macedonia, the holy martyrs Timothy and Diogenes.
\switchcolumn*
\selectlanguage{latin}
In Pérside sanctórum centum vigínti Mártyrum.
\switchcolumn
\selectlanguage{english}
In Persia, one hundred and twenty holy martyrs.
\switchcolumn*
\selectlanguage{latin}
Ascalóne, in Palæstína, pássio sanctórum 
 Platónidis et aliórum duórum Mártyrum.
\switchcolumn
\selectlanguage{english}
At Ascalon in Palestine, the passion of St. Platonides and two other 
 martyrs.
\switchcolumn*
\selectlanguage{latin}
Carthágine sancti Marcellíni Mártyris, qui, ob cathólicæ 
 fidei defensiónem, ab hæréticis occísus est.
\switchcolumn
\selectlanguage{english}
At Carthage, St. Marcellin, who was slain by the heretics for defending the 
 Catholic faith.
\switchcolumn*
\selectlanguage{latin}
In Dánia sancti Guliélmi Abbátis, vita et miráculis clari.
\switchcolumn
\selectlanguage{english}
In Denmark, St. William, an abbot renowned for his saintly life and 
 miracles.
\switchcolumn*
\selectlanguage{latin}
\end{paracol}


% ---- martyrology/mart04/mart0407.htm
\needspace{10\baselineskip}
\begin{paracol}{2}
\selectlanguage{latin}
\begin{center}{\color{gregoriocolor} Séptimo Idus Aprílis. 
 Luna\dots\ }\end{center}
\switchcolumn
\selectlanguage{english}
\begin{center}{\color{gregoriocolor} The Seventh Day of April. The\dots\ Day of the Moon.}\end{center}
\end{paracol}

\noindent\begin{tabularx}{\linewidth}{*{19}{>{\centering\arraybackslash}X}}
 \textcolor{gregoriocolor}{a} & \textcolor{gregoriocolor}{b} & \textcolor{gregoriocolor}{c} & \textcolor{gregoriocolor}{d} & \textcolor{gregoriocolor}{e} & \textcolor{gregoriocolor}{f} & \textcolor{gregoriocolor}{g} & \textcolor{gregoriocolor}{h} & \textcolor{gregoriocolor}{i} & \textcolor{gregoriocolor}{k} & \textcolor{gregoriocolor}{l} & \textcolor{gregoriocolor}{m} & \textcolor{gregoriocolor}{n} & \textcolor{gregoriocolor}{p} & \textcolor{gregoriocolor}{q} & \textcolor{gregoriocolor}{r} & \textcolor{gregoriocolor}{s} & \textcolor{gregoriocolor}{t} & \textcolor{gregoriocolor}{u} \\
 9 & 10 & 11 & 12 & 13 & 14 & 15 & 16 & 17 & 18 & 19 & 20 & 21 & 22 & 23 & 24 & 25 & 26 & 27 \\
\end{tabularx}
\vspace{0.5\baselineskip}
\noindent\begin{tabularx}{\linewidth}{*{12}{>{\centering\arraybackslash}X}}
 \textcolor{gregoriocolor}{A} & \textcolor{gregoriocolor}{B} & \textcolor{gregoriocolor}{C} & \textcolor{gregoriocolor}{D} & \textcolor{gregoriocolor}{E} & F & \textcolor{gregoriocolor}{F} & \textcolor{gregoriocolor}{G} & \textcolor{gregoriocolor}{H} & \textcolor{gregoriocolor}{M} & \textcolor{gregoriocolor}{N} & \textcolor{gregoriocolor}{P} \\
 28 & 29 & 1 & 2 & 3 & 4 & 3 & 4 & 5 & 6 & 7 & 8 \\
\end{tabularx}

\begin{paracol}{2}
\selectlanguage{latin}
\lettrine[lines=2]{R}{otómagi} natális 
 sancti Joánnis Baptístæ de La Salle, Presbyteri et Confessóris, qui, in 
 erudiénda adolescéntia præsértim páupere excéllens, et de religióne 
 civilíque societáte præcláre méritus, Fratrum Scholárum Christianárum 
 Sodalitátem instítuit. Eum Pius Duodécimus, Póntifex Máximus, ómnium 
 Magistrórum púeris adolescentibúsque instituéndis præcípuum apud Deum 
 cæléstem Patrónum constítuit. Ipsíus tamen festum Idibus Maji 
 celebrátur.
\switchcolumn
\selectlanguage{english}
\lettrine[lines=2]{A}{t} Rouen, the birthday of St. John 
 Baptist de la Salle, priest and confessor. He was prominent in the 
 education of youth, especially those who were poor, for which he was 
 acclaimed both by religious and civil society. He was the founder of 
 the Society of the Brothers of the Christian Schools. Pius XII, 
 Supreme Pontiff, declared him patron of all those who teach children and 
 young people. His feast is celebrated on the 15th of May.
\switchcolumn*
\selectlanguage{latin}
In Africa item natális sanctórum Mártyrum Epiphánii Epíscopi, 
 Donáti, Rufíni et aliórum trédecim.
\switchcolumn
\selectlanguage{english}
In Africa, 
 the birthday of the holy martyrs Epiphanius bishop, Donatus, Rufinus and 
 thirteen others.
\switchcolumn*
\selectlanguage{latin}
Alexandríæ sancti Peléusii, Presbyteri, et Mártyris.
\switchcolumn
\selectlanguage{english}
At Alexandria, St. Peleusius, 
 priest and martyr.
\switchcolumn*
\selectlanguage{latin}
Synópe, in Ponto, sanctórum ducentórum Mártyrum.
\switchcolumn
\selectlanguage{english}
At Sinope, in Pontus, two hundred holy martyrs.
\switchcolumn*
\selectlanguage{latin}
In Cilícia sancti Calliópii Mártyris, qui, sub Maximiáno Præfécto, post 
 ália torménta, cápite in terram verso cruci affíxus, nóbili coróna martyrii 
 decorátus est.
\switchcolumn
\selectlanguage{english}
In Cilicia, under the prefect Maximus, St. Calliopius, martyr. 
 After undergoing other torments, he was fastened to a cross with his head 
 downward, and thus gained the noble crown of martyrdom.
\switchcolumn*
\selectlanguage{latin}
Nicomedíæ sancti Cyríaci et aliórum decem Mártyrum.
\switchcolumn
\selectlanguage{english}
At Nicomedia, St. Cyriacus and ten other martyrs.
\switchcolumn*
\selectlanguage{latin}
Verónæ sancti Saturníni, Epíscopi et Confessóris.
\switchcolumn
\selectlanguage{english}
At Verona, St. Saturninus, bishop and confessor.
\switchcolumn*
\selectlanguage{latin}
Romæ sancti Hegesíppi, qui vicínus Apostolórum tempóribus, Romam venit ad 
 Anicétum Pontíficem, ibíque mansit usque ad Eleuthérium, et Ecclesiasticórum 
 Actuum a passióne Dómini usque ad suam ætátem sermóne símplici téxuit 
 históriam; ut, quorum vitam sectabátur, dicéndi quoque exprímeret charactére.
\switchcolumn
\selectlanguage{english}
At Rome, St. Hegesippus, who lived close to the time of the apostles. 
 He came to Rome while Anicetus was pope, and remained until the time of 
 Eleutherius. He wrote a history of the Church, from the Passion of our 
 Lord to his own time, in a simple style, to make clear the character of 
 those whose life he imitated.
\switchcolumn*
\selectlanguage{latin}
In Syria sancti Aphraátis Anachorétæ, qui, Valéntis témpore, cathólicam 
 fidem virtúte miraculórum advérsus Ariános deféndit.
\switchcolumn
\selectlanguage{english}
In Syria, in the time of Valens, St. Aphraates, an anchoret, who defended 
 the Catholic faith against the Arians by the power of miracles.
\switchcolumn*
\selectlanguage{latin}
\end{paracol}


% ---- martyrology/mart04/mart0408.htm
\needspace{10\baselineskip}
\begin{paracol}{2}
\selectlanguage{latin}
\begin{center}{\color{gregoriocolor} Sexto Idus Aprílis. 
 Luna\dots\ }\end{center}
\switchcolumn
\selectlanguage{english}
\begin{center}{\color{gregoriocolor} The Eighth Day of April. The\dots\ Day of the Moon.}\end{center}
\end{paracol}

\noindent\begin{tabularx}{\linewidth}{*{19}{>{\centering\arraybackslash}X}}
 \textcolor{gregoriocolor}{a} & \textcolor{gregoriocolor}{b} & \textcolor{gregoriocolor}{c} & \textcolor{gregoriocolor}{d} & \textcolor{gregoriocolor}{e} & \textcolor{gregoriocolor}{f} & \textcolor{gregoriocolor}{g} & \textcolor{gregoriocolor}{h} & \textcolor{gregoriocolor}{i} & \textcolor{gregoriocolor}{k} & \textcolor{gregoriocolor}{l} & \textcolor{gregoriocolor}{m} & \textcolor{gregoriocolor}{n} & \textcolor{gregoriocolor}{p} & \textcolor{gregoriocolor}{q} & \textcolor{gregoriocolor}{r} & \textcolor{gregoriocolor}{s} & \textcolor{gregoriocolor}{t} & \textcolor{gregoriocolor}{u} \\
 10 & 11 & 12 & 13 & 14 & 15 & 16 & 17 & 18 & 19 & 20 & 21 & 22 & 23 & 24 & 25 & 26 & 27 & 28 \\
\end{tabularx}
\vspace{0.5\baselineskip}
\noindent\begin{tabularx}{\linewidth}{*{12}{>{\centering\arraybackslash}X}}
 \textcolor{gregoriocolor}{A} & \textcolor{gregoriocolor}{B} & \textcolor{gregoriocolor}{C} & \textcolor{gregoriocolor}{D} & \textcolor{gregoriocolor}{E} & F & \textcolor{gregoriocolor}{F} & \textcolor{gregoriocolor}{G} & \textcolor{gregoriocolor}{H} & \textcolor{gregoriocolor}{M} & \textcolor{gregoriocolor}{N} & \textcolor{gregoriocolor}{P} \\
 29 & 1 & 2 & 3 & 4 & 5 & 4 & 5 & 6 & 7 & 8 & 9 \\
\end{tabularx}

\begin{paracol}{2}
\selectlanguage{latin}
\lettrine[lines=2]{C}{ommemorátio} sanctórum 
 Herodiónis, Asyncriti et Phlegóntis, de quibus scribit beátus Paulus 
 Apóstolus in Epístola ad Romános.
\switchcolumn
\selectlanguage{english}
\lettrine[lines=2]{T}{he} commemoration of Saints 
 Herodian, Asyncritus, and Phlegon who are mentioned by blessed Paul the 
 Apostle in his Letter to the Romans.
\switchcolumn*
\selectlanguage{latin}
Alexandríæ sancti 
 Ædésii Mártyris, qui, sub Maximiáno Galério Imperatóre, cum esset beáti 
 Apphiáni frater, et ímpium Júdicem, quod Deo dicátas Vírgines lenónibus 
 tráderet, palam argúeret, idcírco, a milítibus tentus sævissimísque afféctus 
 supplíciis, in mare demérsus est pro Christo Dómino.
\switchcolumn
\selectlanguage{english}
At Alexandria, in the time of 
 Emperor Maximian Galerius, the martyr St. Aedesius, brother of the blessed 
 Apphian. Because he publicly reproved the wicked judge who delivered 
 to corruptors virgins consecrated to God, he was arrested by the soldiers, 
 exposed to the most severe torments, and thrown into the sea for the sake of 
 Christ our Lord.
\switchcolumn*
\selectlanguage{latin}
In Africa sanctórum 
 Mártyrum Januárii, Máximæ et Macáriæ.
\switchcolumn
\selectlanguage{english}
In Africa, the holy martyrs 
 Januarius, Maxima, and Macaria.
\switchcolumn*
\selectlanguage{latin}
Carthágine sanctæ 
 Concéssæ Mártyris.
\switchcolumn
\selectlanguage{english}
At Carthage, the martyr St. 
 Concessa.
\switchcolumn*
\selectlanguage{latin}
Apud Corínthum beáti 
 Dionysii Epíscopi, qui eruditióne et grátia, quam hábuit in verbo Dei, non 
 solum suæ civitátis et provínciæ pópulos, sed et aliárum provinciárum et 
 úrbium Epíscopos epístolis erudívit; Romanósque Pontífices 
 ádeo cóluit, ut 
 eórum epístolas públice légere in Ecclésia diébus Domínicis consuéverit. 
 Cláruit autem tempóribus Marci Antoníni Veri et Lúcii Aurélii Cómmodi.
\switchcolumn
\selectlanguage{english}
At Corinth, Bishop St. Denis, who 
 instructed not only the people of his own city and province by the learning 
 and charm with which he preached the word of God, but also the bishops of 
 other cities and provinces by the letters he wrote to them. His 
 devotion to the Roman Pontiffs was such that he was accustomed to read their 
 letters publicly in the church on Sundays. He lived in the time of 
 Marcus Antoninus Verus and Lucius Aurelius Commodus.
\switchcolumn*
\selectlanguage{latin}
Turónis, in Gállia, 
 sancti Perpétui Epíscopi, admirándæ sanctitátis viri.
\switchcolumn
\selectlanguage{english}
At Tours in France, the holy bishop 
 Perpetuus, a man of great sanctity.
\switchcolumn*
\selectlanguage{latin}
Ferentíni, in Hérnicis, 
 sancti Redémpti Epíscopi, cujus méminit beátus Gregórius Papa.
\switchcolumn
\selectlanguage{english}
At Ferentino in Campania, Bishop 
 St. Redemptus, who was mentioned by Pope St. Gregory.
\switchcolumn*
\selectlanguage{latin}
Apud Comum sancti 
 Amántii, Epíscopi et Confessóris.
\switchcolumn
\selectlanguage{english}
At Como, St. Amantius, bishop and 
 confessor.
\switchcolumn*
\selectlanguage{latin}
\end{paracol}


% ---- martyrology/mart04/mart0409.htm
\needspace{10\baselineskip}
\begin{paracol}{2}
\selectlanguage{latin}
\begin{center}{\color{gregoriocolor} Quinto Idus Aprílis. 
 Luna\dots\ }\end{center}
\switchcolumn
\selectlanguage{english}
\begin{center}{\color{gregoriocolor} The 
 Ninth Day of April. The\dots\ Day of the Moon.}\end{center}
\end{paracol}

\noindent\begin{tabularx}{\linewidth}{*{19}{>{\centering\arraybackslash}X}}
 \textcolor{gregoriocolor}{a} & \textcolor{gregoriocolor}{b} & \textcolor{gregoriocolor}{c} & \textcolor{gregoriocolor}{d} & \textcolor{gregoriocolor}{e} & \textcolor{gregoriocolor}{f} & \textcolor{gregoriocolor}{g} & \textcolor{gregoriocolor}{h} & \textcolor{gregoriocolor}{i} & \textcolor{gregoriocolor}{k} & \textcolor{gregoriocolor}{l} & \textcolor{gregoriocolor}{m} & \textcolor{gregoriocolor}{n} & \textcolor{gregoriocolor}{p} & \textcolor{gregoriocolor}{q} & \textcolor{gregoriocolor}{r} & \textcolor{gregoriocolor}{s} & \textcolor{gregoriocolor}{t} & \textcolor{gregoriocolor}{u} \\
 11 & 12 & 13 & 14 & 15 & 16 & 17 & 18 & 19 & 20 & 21 & 22 & 23 & 24 & 25 & 26 & 27 & 28 & 29 \\
\end{tabularx}
\vspace{0.5\baselineskip}
\noindent\begin{tabularx}{\linewidth}{*{12}{>{\centering\arraybackslash}X}}
 \textcolor{gregoriocolor}{A} & \textcolor{gregoriocolor}{B} & \textcolor{gregoriocolor}{C} & \textcolor{gregoriocolor}{D} & \textcolor{gregoriocolor}{E} & F & \textcolor{gregoriocolor}{F} & \textcolor{gregoriocolor}{G} & \textcolor{gregoriocolor}{H} & \textcolor{gregoriocolor}{M} & \textcolor{gregoriocolor}{N} & \textcolor{gregoriocolor}{P} \\
 1 & 2 & 3 & 4 & 5 & 6 & 5 & 6 & 7 & 8 & 9 & 10 \\
\end{tabularx}

\begin{paracol}{2}
\selectlanguage{latin}
\lettrine[lines=2]{I}{n} Judǽa sanctæ Maríæ 
 Cléophæ, quam beátus Joánnes Evangelísta sorórem sanctíssimæ Dei Genitrícis 
 Maríæ núncupat, et cum hac simul juxta crucem Jesu stetísse narrat.
\switchcolumn
\selectlanguage{english}
\lettrine[lines=2]{I}{n} Judea, St. Mary Cleophas, whom 
 St. John the Evangelist calls the sister of the Blessed Virgin Mary, Mother 
 of God, and says that she stood at her side beneath the Cross of Jesus.
\switchcolumn*
\selectlanguage{latin}
Antiochíæ sancti 
 Próchori, qui fuit unus de septem primis Diáconis; et, fide ac miráculis 
 clarus, martyrio coronátus est.
\switchcolumn
\selectlanguage{english}
At Antioch, St. Prochorus who was 
 one of the first seven deacons. Renowned for faith and miracles, he 
 received the crown of martyrdom.
\switchcolumn*
\selectlanguage{latin}
Romæ natális sanctórum 
 Mártyrum Demétrii, Concéssi, Hilárii et Sociórum.
\switchcolumn
\selectlanguage{english}
At Rome, the birthday of the holy 
 martyrs Demetrius, Concessus, Hilary, and their companions.
\switchcolumn*
\selectlanguage{latin}
Cæsarǽæ, in Cappadócia, 
 sancti Eupsychii Mártyris, qui ob evérsum Fortúnæ fanum, sub Juliáno 
 Apóstata, martyrium consummávit.
\switchcolumn
\selectlanguage{english}
At Caesarea in Cappadocia, St. 
 Eupsychius, martyr, who was persecuted under Julian the Apostate for having 
 overthrown the temple of Fortune.
\switchcolumn*
\selectlanguage{latin}
In Africa sanctórum 
 Mártyrum Massylitanórum, in quorum die natáli sanctus Augustínus tractátum 
 hábuit.
\switchcolumn
\selectlanguage{english}
In Africa the holy Massylitan 
 Martyrs, on whose birthday was written a tract by St. Augustine.
\switchcolumn*
\selectlanguage{latin}
Sírmii pássio 
 sanctárum septem Vírginum et Mártyrum, quæ, dato simul prétio sánguinis, 
 vitam mercátæ sunt ætérnam.
\switchcolumn
\selectlanguage{english}
At Sirmio, seven holy virgins and 
 martyrs, who purchased eternal life together at the price of their own 
 blood.
\switchcolumn*
\selectlanguage{latin}
Amidæ, in Mesopotámia, sancti Acátii Epíscopi, qui pro rediméndis captívis étiam vasa Ecclésiæ 
 conflávit ac véndidit.
\switchcolumn
\selectlanguage{english}
At Amida in Mesopotamia, St. 
 Acatius, bishop, who even melted down and sold the sacred vessels in order 
 to ransom captives.
\switchcolumn*
\selectlanguage{latin}
Rotómagi sancti 
 Hugónis, Epíscopi et Confessóris.
\switchcolumn
\selectlanguage{english}
At Rouen, St. Hugh, bishop and 
 confessor.
\switchcolumn*
\selectlanguage{latin}
In civitáte Diénsi, in 
 Gállia, sancti Marcélli Epíscopi, miráculis clari.
\switchcolumn
\selectlanguage{english}
In the city of Die, in France, St. 
 Marcellus, bishop, celebrated for miracles.
\switchcolumn*
\selectlanguage{latin}
Móntibus, in Hannónia, 
 beátæ Waldetrúdis, vitæ sanctimónia et miráculis claræ.
\switchcolumn
\selectlanguage{english}
At Mons in Hainaut, blessed 
 Waltrude, renowned for holiness and miracles.
\switchcolumn*
\selectlanguage{latin}
Romæ Translátio 
 córporis sanctæ Mónicæ, matris beáti Augustíni Epíscopi; quod, ex Ostiis 
 Tiberínis, Martíno Quinto Summo Pontífice, in Urbem delátum, in Ecclésia 
 ejúsdem beáti Augustíni honorífice recónditum fuit.
\switchcolumn
\selectlanguage{english}
At Rome, the transferring of the 
 body of St. Monica, mother of the bishop St. Augustine. It was brought 
 from Ostia to Rome, under the Sovereign Pontiff, Martin V, and buried with 
 due honours in the church of St. Augustine.
\switchcolumn*
\selectlanguage{latin}
\end{paracol}


% ---- martyrology/mart04/mart0410.htm
\needspace{10\baselineskip}
\begin{paracol}{2}
\selectlanguage{latin}
\begin{center}{\color{gregoriocolor} Quarto Idus Aprílis. 
 Luna\dots\ }\end{center}
\switchcolumn
\selectlanguage{english}
\begin{center}{\color{gregoriocolor} The Tenth Day of April. The\dots\ Day of the Moon.}\end{center}
\end{paracol}

\noindent\begin{tabularx}{\linewidth}{*{19}{>{\centering\arraybackslash}X}}
 \textcolor{gregoriocolor}{a} & \textcolor{gregoriocolor}{b} & \textcolor{gregoriocolor}{c} & \textcolor{gregoriocolor}{d} & \textcolor{gregoriocolor}{e} & \textcolor{gregoriocolor}{f} & \textcolor{gregoriocolor}{g} & \textcolor{gregoriocolor}{h} & \textcolor{gregoriocolor}{i} & \textcolor{gregoriocolor}{k} & \textcolor{gregoriocolor}{l} & \textcolor{gregoriocolor}{m} & \textcolor{gregoriocolor}{n} & \textcolor{gregoriocolor}{p} & \textcolor{gregoriocolor}{q} & \textcolor{gregoriocolor}{r} & \textcolor{gregoriocolor}{s} & \textcolor{gregoriocolor}{t} & \textcolor{gregoriocolor}{u} \\
 12 & 13 & 14 & 15 & 16 & 17 & 18 & 19 & 20 & 21 & 22 & 23 & 24 & 25 & 26 & 27 & 28 & 29 & 1 \\
\end{tabularx}
\vspace{0.5\baselineskip}
\noindent\begin{tabularx}{\linewidth}{*{12}{>{\centering\arraybackslash}X}}
 \textcolor{gregoriocolor}{A} & \textcolor{gregoriocolor}{B} & \textcolor{gregoriocolor}{C} & \textcolor{gregoriocolor}{D} & \textcolor{gregoriocolor}{E} & F & \textcolor{gregoriocolor}{F} & \textcolor{gregoriocolor}{G} & \textcolor{gregoriocolor}{H} & \textcolor{gregoriocolor}{M} & \textcolor{gregoriocolor}{N} & \textcolor{gregoriocolor}{P} \\
 2 & 3 & 4 & 5 & 6 & 7 & 6 & 7 & 8 & 9 & 10 & 11 \\
\end{tabularx}

\begin{paracol}{2}
\selectlanguage{latin}
\lettrine[lines=2]{A}{pud} Babylónem sancti 
 Ezechiélis Prophétæ, qui, a Júdice pópuli Israël, quod eum de cultu idolórum 
 argúeret, interféctus, in sepúlcro Sem et Arpháxad, Abrahæ progenitórum, 
 sepúltus est; ad quod sepúlcrum, oratiónis causa, multi conflúere 
 consuevérunt.
\switchcolumn
\selectlanguage{english}
\lettrine[lines=2]{A}{t} Babylon, the prophet Ezechiel, 
 who was put to death by a judge of the people of Israel because he reproved 
 him for worshipping idols. He was buried in the sepulchre of Sem and 
 Arphaxad, ancestors of Abraham. Many people were in the habit of going 
 to his tomb to pray.
\switchcolumn*
\selectlanguage{latin}
Romæ natális 
 plurimórum sanctórum Mártyrum, quos sanctus Alexánder Papa, cum detinerétur 
 in cárcere, baptizávit. Hos autem omnes Aureliánus Præféctus, navi 
 vetústæ impósitos, in altum mare dedúci, et illic, ligátis ad colla 
 lapídibus, mergi præcépit.
\switchcolumn
\selectlanguage{english}
At Rome, the birthday of many holy 
 martyrs, whom Pope St. Alexander baptized while he was in prison. The 
 prefect Aurelian had them all put in an old ship, taken to the deep sea, and 
 drowned with stones tied to their necks.
\switchcolumn*
\selectlanguage{latin}
Alexandríæ sanctórum 
 Mártyrum Apollónii Presbyteri, et aliórum quinque, qui, in persecutióne 
 Maximiáni, in mare demérsi sunt.
\switchcolumn
\selectlanguage{english}
At Alexandria, during the 
 persecution of Maximian, the holy martyrs Apollonius, a priest, and five 
 others who were drowned in the sea.
\switchcolumn*
\selectlanguage{latin}
In Africa sanctórum 
 Mártyrum Teréntii, Africáni, Pompéji et Sociórum; qui, sub Décio Imperatóre 
 et Fortuniáno Præfécto, virgis cæsi, equúleis torti aliísque supplíciis 
 cruciáti, demum cápitis obtruncatióne martyrium complevérunt.
\switchcolumn
\selectlanguage{english}
In Africa, under Emperor Decius and 
 the prefect Fortunian, the holy martyrs Terence, Africanus, Pompey, and 
 their companions, who were scourged, racked and subjected to other torments. 
 Their martyrdom ended by beheading.
\switchcolumn*
\selectlanguage{latin}
Gandávi, in Flándria, 
 sancti Macárii, Epíscopi Antiochéni, virtútibus et miráculis clari.
\switchcolumn
\selectlanguage{english}
At Ghent in Flanders, St. Macarius, 
 bishop of Antioch, celebrated for virtues and miracles.
\switchcolumn*
\selectlanguage{latin}
Vallisoléti, in 
 Hispánia, sancti Michaélis de Sanctis, ex Ordine Discalceatórum sanctíssimæ 
 Trinitátis redemptiónis captivórum, Confessóris, innocéntia vitæ, admirábili 
 pæniténtia et caritáte in Deum exímii; quem Pius Nonus, Póntifex Máximus, 
 inter Sanctos rétulit.
\switchcolumn
\selectlanguage{english}
At Valladolid in Spain, St. Michael 
 of the Saints, confessor, of the Order of Discalced Trinitarians for the 
 Redemption of Captives, a man known for his upright life, his penitential 
 spirit, and his great love of God. He was placed on the roll of the 
 saints by Pope Pius IX.
\switchcolumn*
\selectlanguage{latin}
\end{paracol}


% ---- martyrology/mart04/mart0411.htm
\needspace{10\baselineskip}
\begin{paracol}{2}
\selectlanguage{latin}
\begin{center}{\color{gregoriocolor} Tértio Idus Aprílis. 
 Luna\dots\ }\end{center}
\switchcolumn
\selectlanguage{english}
\begin{center}{\color{gregoriocolor} The Eleventh Day of April. The\dots\ Day of the Moon.}\end{center}
\end{paracol}

\noindent\begin{tabularx}{\linewidth}{*{19}{>{\centering\arraybackslash}X}}
 \textcolor{gregoriocolor}{a} & \textcolor{gregoriocolor}{b} & \textcolor{gregoriocolor}{c} & \textcolor{gregoriocolor}{d} & \textcolor{gregoriocolor}{e} & \textcolor{gregoriocolor}{f} & \textcolor{gregoriocolor}{g} & \textcolor{gregoriocolor}{h} & \textcolor{gregoriocolor}{i} & \textcolor{gregoriocolor}{k} & \textcolor{gregoriocolor}{l} & \textcolor{gregoriocolor}{m} & \textcolor{gregoriocolor}{n} & \textcolor{gregoriocolor}{p} & \textcolor{gregoriocolor}{q} & \textcolor{gregoriocolor}{r} & \textcolor{gregoriocolor}{s} & \textcolor{gregoriocolor}{t} & \textcolor{gregoriocolor}{u} \\
 13 & 14 & 15 & 16 & 17 & 18 & 19 & 20 & 21 & 22 & 23 & 24 & 25 & 26 & 27 & 28 & 29 & 1 & 2 \\
\end{tabularx}
\vspace{0.5\baselineskip}
\noindent\begin{tabularx}{\linewidth}{*{12}{>{\centering\arraybackslash}X}}
 \textcolor{gregoriocolor}{A} & \textcolor{gregoriocolor}{B} & \textcolor{gregoriocolor}{C} & \textcolor{gregoriocolor}{D} & \textcolor{gregoriocolor}{E} & F & \textcolor{gregoriocolor}{F} & \textcolor{gregoriocolor}{G} & \textcolor{gregoriocolor}{H} & \textcolor{gregoriocolor}{M} & \textcolor{gregoriocolor}{N} & \textcolor{gregoriocolor}{P} \\
 3 & 4 & 5 & 6 & 7 & 8 & 7 & 8 & 9 & 10 & 11 & 12 \\
\end{tabularx}

\begin{paracol}{2}
\selectlanguage{latin}
\lettrine[lines=2]{S}{ancti} Leónis Papæ 
 Primi, cognoménto Magni, Confessóris et Ecclésiæ Doctóris, cujus dies 
 natális recólitur quarto Idus Novémbris.
\switchcolumn
\selectlanguage{english}
\lettrine[lines=2]{S}{t.} Leo the First, pope and 
 confessor, who was surnamed the Great. His birthday falls on the 10th 
 of November.
\switchcolumn*
\selectlanguage{latin}
Pérgami, in Asia, 
 sancti Antípæ, testis fidélis, cujus méminit sanctus Joánnes in Apocalypsi. 
 Ipse autem Antípas, sub Domitiáno Imperatóre, in bovem æneum candéntem 
 conjéctus, martyrium consummávit.
\switchcolumn
\selectlanguage{english}
At Pergamum in Asia, the faithful 
 witness, St. Antipas, who was mentioned by St. John in the Apocalypse. 
 Under Emperor Domitian, he was enclosed in an ox made of brass that had been 
 heated to redness, and thus completed his martyrdom.
\switchcolumn*
\selectlanguage{latin}
Salónæ, in Dalmátia, 
 sanctórum Mártyrum Domniónis Epíscopi, cum milítibus octo.
\switchcolumn
\selectlanguage{english}
At Salona in Dalmatia, the holy 
 martyrs Domnio, bishop, and eight soldiers.
\switchcolumn*
\selectlanguage{latin}
Gortynæ, in Creta, 
 sancti Philíppi Epíscopi, vita et doctrína claríssimi, qui, tempóribus Marci 
 Antoníni Veri et Lúcii Aurélii Cómmodi, Ecclésiam sibi créditam regens, a 
 furóre Gentílium et ab hæreticórum insídiis eándem tutátus est.
\switchcolumn
\selectlanguage{english}
At Gortina in Crete, during the 
 reign of Marcus Antoninus Verus and Lucius Aurelius Commodus, St. Philip, 
 bishop, well known for his life and his teaching. He had defended the 
 Church entrusted to his care against the fury of the heathen and the snares 
 of the heretics.
\switchcolumn*
\selectlanguage{latin}
Nicomedíæ sancti 
 Eustórgii Presbyteri.
\switchcolumn
\selectlanguage{english}
At Nicomedia, the priest St. 
 Eustorgius.
\switchcolumn*
\selectlanguage{latin}
Spoléti sancti Isaac, 
 Mónachi et Confessóris, cujus virtútes sanctus Gregórius Papa commémorat.
\switchcolumn
\selectlanguage{english}
At Spoleto, St. Isaac, monk and 
 confessor, whose virtues are recorded by Pope St. Gregory.
\switchcolumn*
\selectlanguage{latin}
Apud Gazam in 
 Palæstína, sancti Barsanúphii Anachorétæ, sub Justiniáno Imperatóre.
\switchcolumn
\selectlanguage{english}
At Gaza in Palestine, in the time 
 of Emperor Justinian, St. Barsanuphius, an anchoret.
\switchcolumn*
\selectlanguage{latin}
Lucæ, in Etrúria, 
 sanctæ Gemmæ Galgáni, Vírginis, contemplatióne Domínicæ Passiónis et vitæ 
 sanctitátis mirábilis, quam Pius Papa Duodécimus in Sanctárum númerum 
 rétulit.
\switchcolumn
\selectlanguage{english}
At Luca in Etruria, St. Gemma 
 Galgani, virgin, renowned for her contemplation of the Passion of our Lord, 
 and for a life of holiness, and whom Pope Pius XII joined to the number of 
 the Saints.
\switchcolumn*
\selectlanguage{latin}
\end{paracol}


% ---- martyrology/mart04/mart0412.htm
\needspace{10\baselineskip}
\begin{paracol}{2}
\selectlanguage{latin}
\begin{center}{\color{gregoriocolor} Prídie Idus Aprílis. 
 Luna\dots\ }\end{center}
\switchcolumn
\selectlanguage{english}
\begin{center}{\color{gregoriocolor} The Twelfth Day of April. The\dots\ Day of the Moon.}\end{center}
\end{paracol}

\noindent\begin{tabularx}{\linewidth}{*{19}{>{\centering\arraybackslash}X}}
 \textcolor{gregoriocolor}{a} & \textcolor{gregoriocolor}{b} & \textcolor{gregoriocolor}{c} & \textcolor{gregoriocolor}{d} & \textcolor{gregoriocolor}{e} & \textcolor{gregoriocolor}{f} & \textcolor{gregoriocolor}{g} & \textcolor{gregoriocolor}{h} & \textcolor{gregoriocolor}{i} & \textcolor{gregoriocolor}{k} & \textcolor{gregoriocolor}{l} & \textcolor{gregoriocolor}{m} & \textcolor{gregoriocolor}{n} & \textcolor{gregoriocolor}{p} & \textcolor{gregoriocolor}{q} & \textcolor{gregoriocolor}{r} & \textcolor{gregoriocolor}{s} & \textcolor{gregoriocolor}{t} & \textcolor{gregoriocolor}{u} \\
 14 & 15 & 16 & 17 & 18 & 19 & 20 & 21 & 22 & 23 & 24 & 25 & 26 & 27 & 28 & 29 & 1 & 2 & 3 \\
\end{tabularx}
\vspace{0.5\baselineskip}
\noindent\begin{tabularx}{\linewidth}{*{12}{>{\centering\arraybackslash}X}}
 \textcolor{gregoriocolor}{A} & \textcolor{gregoriocolor}{B} & \textcolor{gregoriocolor}{C} & \textcolor{gregoriocolor}{D} & \textcolor{gregoriocolor}{E} & F & \textcolor{gregoriocolor}{F} & \textcolor{gregoriocolor}{G} & \textcolor{gregoriocolor}{H} & \textcolor{gregoriocolor}{M} & \textcolor{gregoriocolor}{N} & \textcolor{gregoriocolor}{P} \\
 4 & 5 & 6 & 7 & 8 & 9 & 8 & 9 & 10 & 11 & 12 & 13 \\
\end{tabularx}

\begin{paracol}{2}
\selectlanguage{latin}
\lettrine[lines=2]{V}{erónæ} pássio sancti 
 Zenónis Epíscopi, qui inter persecutiónis procéllas eam Ecclésiam mira 
 constántia gubernávit, et, Galliéni témpore, martyrio coronátus est.
\switchcolumn
\selectlanguage{english}
\lettrine[lines=2]{A}{t} Verona, the passion of Bishop 
 St. Zeno, who governed that Church with great fortitude amid the storms of 
 persecution, and was crowned with martyrdom in the time of Gallienus.
\switchcolumn*
\selectlanguage{latin}
In Cappadócia sancti 
 Sabæ Gothi, qui, sub Valénte Imperatóre, cum Rex Gothórum Athanarícus 
 Christiános insequerétur, in flumen, post dira torménta, projéctus est; quo 
 étiam témpore (ut sanctus Augustínus scribit) quamplúrimi ex Gothis 
 orthodóxis coróna martyrii sunt insigníti.
\switchcolumn
\selectlanguage{english}
In Cappadocia, in the reign of 
 Emperor Valens, during the persecution raised against the Christians by 
 Atanaric, king of the Goths, St. Sabas, himself a Goth, who was cast into a 
 river after undergoing cruel torments. Many orthodox Goths, as St. 
 Augustine relates, received at that time the crown of martyrdom.
\switchcolumn*
\selectlanguage{latin}
Brácari, in Lusitánia, 
 sancti Victóris Mártyris, qui, adhuc catechúmenus, cum noluísset idólum 
 adoráre, et Christum Jesum magna constántia conféssus fuísset, ídeo, post 
 multa torménta, cápite abscísso, méruit próprio sánguine baptizári.
\switchcolumn
\selectlanguage{english}
At Braga in Portugal, the martyr 
 St. Victor. Although only a catechumen, he refused to adore an idol, 
 and confessed Jesus Christ with great constancy. After suffering many 
 tortures, he was beheaded, and thus merited to be baptized in his own blood.
\switchcolumn*
\selectlanguage{latin}
Firmi, in Picéno, 
 sanctæ Víssiæ, Vírginis et Mártyris.
\switchcolumn
\selectlanguage{english}
At Fermo, in Piceno, St. Vissia, 
 virgin and martyr.
\switchcolumn*
\selectlanguage{latin}
Romæ, via Aurélia, natális sancti Júlii Papæ Primi, qui advérsus Ariános pro fide cathólica 
 plúrimum laborávit, ac, multis præcláre gestis, sanctitáte célebris quiévit 
 in pace.
\switchcolumn
\selectlanguage{english}
At Rome, on the Aurelian Way, the 
 birthday of Pope St. Julius, who vigorously defended the Catholic faith 
 against the Arians. After a life of brilliant accomplishments, he 
 rested in peace, famed for his sanctity.
\switchcolumn*
\selectlanguage{latin}
Apud Vapíngum óppidum, 
 in Gállia, sancti Constantíni, Epíscopi et Confessóris.
\switchcolumn
\selectlanguage{english}
At the town of Gap in France, St. 
 Constantine, bishop and confessor.
\switchcolumn*
\selectlanguage{latin}
Papíæ sancti Damiáni 
 Epíscopi.
\switchcolumn
\selectlanguage{english}
At Pavia, Bishop St. Damian.
\switchcolumn*
\selectlanguage{latin}
\end{paracol}


% ---- martyrology/mart04/mart0413.htm
\needspace{10\baselineskip}
\begin{paracol}{2}
\selectlanguage{latin}
\begin{center}{\color{gregoriocolor} Idibus Aprílis. 
 Luna\dots\ }\end{center}
\switchcolumn
\selectlanguage{english}
\begin{center}{\color{gregoriocolor} The Thirteenth Day of April. The\dots\ Day of the Moon.}\end{center}
\end{paracol}

\noindent\begin{tabularx}{\linewidth}{*{19}{>{\centering\arraybackslash}X}}
 \textcolor{gregoriocolor}{a} & \textcolor{gregoriocolor}{b} & \textcolor{gregoriocolor}{c} & \textcolor{gregoriocolor}{d} & \textcolor{gregoriocolor}{e} & \textcolor{gregoriocolor}{f} & \textcolor{gregoriocolor}{g} & \textcolor{gregoriocolor}{h} & \textcolor{gregoriocolor}{i} & \textcolor{gregoriocolor}{k} & \textcolor{gregoriocolor}{l} & \textcolor{gregoriocolor}{m} & \textcolor{gregoriocolor}{n} & \textcolor{gregoriocolor}{p} & \textcolor{gregoriocolor}{q} & \textcolor{gregoriocolor}{r} & \textcolor{gregoriocolor}{s} & \textcolor{gregoriocolor}{t} & \textcolor{gregoriocolor}{u} \\
 15 & 16 & 17 & 18 & 19 & 20 & 21 & 22 & 23 & 24 & 25 & 26 & 27 & 28 & 29 & 1 & 2 & 3 & 4 \\
\end{tabularx}
\vspace{0.5\baselineskip}
\noindent\begin{tabularx}{\linewidth}{*{12}{>{\centering\arraybackslash}X}}
 \textcolor{gregoriocolor}{A} & \textcolor{gregoriocolor}{B} & \textcolor{gregoriocolor}{C} & \textcolor{gregoriocolor}{D} & \textcolor{gregoriocolor}{E} & F & \textcolor{gregoriocolor}{F} & \textcolor{gregoriocolor}{G} & \textcolor{gregoriocolor}{H} & \textcolor{gregoriocolor}{M} & \textcolor{gregoriocolor}{N} & \textcolor{gregoriocolor}{P} \\
 5 & 6 & 7 & 8 & 9 & 10 & 9 & 10 & 11 & 12 & 13 & 14 \\
\end{tabularx}

\begin{paracol}{2}
\selectlanguage{latin}
\lettrine[lines=2]{H}{íspali,} in Hispánia, sancti Hermenegíldi Mártyris, qui fuit fílius Leovigíldi, Regis Visigothórum 
 Ariáni; atque ob cathólicæ fidei confessiónem conjéctus in cárcerem, et, cum 
 in solemnitáte Pascháli Communiónem ab Epíscopo Ariáno accípere noluísset, 
 pérfidi patris jussu secúri percússus est, ac regnum cæléste pro terréno Rex 
 et Martyr intrávit.
\switchcolumn
\selectlanguage{english}
\lettrine[lines=2]{A}{t} Seville in Spain, St. 
 Hermenegild, son of Leovigild, Arian king of the Visigoths, who was 
 imprisoned for the confession of the Catholic faith. By order of his 
 wicked father he was beheaded because he had refused to receive communion 
 from an Arian bishop on the feast of Easter. Thus exchanging an 
 earthly for a heavenly kingdom, he entered the abode of the saints, both as 
 a king and as a martyr.
\switchcolumn*
\selectlanguage{latin}
Romæ, in persecutióne 
 Marci Antoníni Veri et Lúcii Aurélii Cómmodi, pássio sancti Justíni, 
 Philósophi et Mártyris; qui, cum secúndum librum pro religiónis nostræ 
 defensióne præfátis Imperatóribus porrexísset, eámque ibídem disputándo 
 strénue propugnásset, Crescéntis Cynici, cujus vitam et mores nefários 
 redargúerat, insídiis accusátus quod Christiánus esset, remuneratiónem 
 linguæ fidélis, martyrii munus accépit. Ipsíus tamen festum sequénti 
 die recólitur.
\switchcolumn
\selectlanguage{english}
At Rome, in the persecution of 
 Marcus Antoninus Verus and Lucius Aurelius Commodus, St. Justin, philosopher 
 and martyr. He had addressed to the emperors his second Apology in 
 defence of our religion, and upheld it by strong arguments. By the 
 intrigue of Crescens the Cynic, whose conduct and immorality he had 
 reproved, he was accused of professing Christianity, and thus he obtained 
 the reward of martyrdom in payment for his faithful confession. His 
 feast is kept on the following day.
\switchcolumn*
\selectlanguage{latin}
Pérgami, in Asia, in 
 eádem persecutióne, natális sanctórum Mártyrum Carpi, Thyatirénsis Epíscopi, 
 Pápyli Diáconi et Agathonícæ, ejúsdem Pápyli soróris et 
 óptimæ féminæ, atque 
 Agathadóri, eórum fámuli, aliorúmque multórum. Hi omnes, post vários 
 cruciátus, pro beátis confessiónibus martyrio coronáti sunt.
\switchcolumn
\selectlanguage{english}
At Pergamum in Asia, during the 
 same persecution, the birthday of the holy martyrs Carpus, bishop of 
 Thyatira, the deacon Papylus, and his sister Agathonica, an excellent woman, 
 Agathadorus, their servant, and many others. After many torments they 
 received their crowns of martyrdom for their worthy confessions.
\switchcolumn*
\selectlanguage{latin}
Doróstori, in Mysia 
 inferióre, pássio sanctórum Máximi, Quinctiliáni et Dadæ, in persecutióne 
 Diocletiáni.
\switchcolumn
\selectlanguage{english}
At Silistria in Bulgaria, the 
 passion of Saints Maximus, Quinctilian, and Dadas, during the persecution of 
 Diocletian.
\switchcolumn*
\selectlanguage{latin}
Ravénnæ sancti Ursi, 
 Epíscopi et Confessóris.
\switchcolumn
\selectlanguage{english}
At Ravenna, St. Ursus, bishop and 
 confessor.
\switchcolumn*
\selectlanguage{latin}
\end{paracol}


% ---- martyrology/mart04/mart0414.htm
\needspace{10\baselineskip}
\begin{paracol}{2}
\selectlanguage{latin}
\begin{center}{\color{gregoriocolor} Décimo octávo Kaléndas Maji. 
 Luna\dots\ }\end{center}
\switchcolumn
\selectlanguage{english}
\begin{center}{\color{gregoriocolor} The Fourteenth Day of April. The\dots\ Day of the Moon.}\end{center}
\end{paracol}

\noindent\begin{tabularx}{\linewidth}{*{19}{>{\centering\arraybackslash}X}}
 \textcolor{gregoriocolor}{a} & \textcolor{gregoriocolor}{b} & \textcolor{gregoriocolor}{c} & \textcolor{gregoriocolor}{d} & \textcolor{gregoriocolor}{e} & \textcolor{gregoriocolor}{f} & \textcolor{gregoriocolor}{g} & \textcolor{gregoriocolor}{h} & \textcolor{gregoriocolor}{i} & \textcolor{gregoriocolor}{k} & \textcolor{gregoriocolor}{l} & \textcolor{gregoriocolor}{m} & \textcolor{gregoriocolor}{n} & \textcolor{gregoriocolor}{p} & \textcolor{gregoriocolor}{q} & \textcolor{gregoriocolor}{r} & \textcolor{gregoriocolor}{s} & \textcolor{gregoriocolor}{t} & \textcolor{gregoriocolor}{u} \\
 16 & 17 & 18 & 19 & 20 & 21 & 22 & 23 & 24 & 25 & 26 & 27 & 28 & 29 & 1 & 2 & 3 & 4 & 5 \\
\end{tabularx}
\vspace{0.5\baselineskip}
\noindent\begin{tabularx}{\linewidth}{*{12}{>{\centering\arraybackslash}X}}
 \textcolor{gregoriocolor}{A} & \textcolor{gregoriocolor}{B} & \textcolor{gregoriocolor}{C} & \textcolor{gregoriocolor}{D} & \textcolor{gregoriocolor}{E} & F & \textcolor{gregoriocolor}{F} & \textcolor{gregoriocolor}{G} & \textcolor{gregoriocolor}{H} & \textcolor{gregoriocolor}{M} & \textcolor{gregoriocolor}{N} & \textcolor{gregoriocolor}{P} \\
 6 & 7 & 8 & 9 & 10 & 11 & 10 & 11 & 12 & 13 & 14 & 15 \\
\end{tabularx}

\begin{paracol}{2}
\selectlanguage{latin}
\lettrine[lines=2]{S}{ancti} Justíni, Philósophi et Mártyris, cujus memória 
 prídie hujus diéi recensétur.
\switchcolumn
\selectlanguage{english}
\lettrine[lines=2]{T}{he} feast of St. Justin, philosopher and martyr, who was 
 yesterday mentioned.
\switchcolumn*
\selectlanguage{latin}
Romæ, via Appia, 
 natális sanctórum Mártyrum Tibúrtii, Valeriáni et Máximi, sub Alexándro 
 Imperatóre et Almáchio Præfécto. Horum duo primi, beátæ Cæcíliæ 
 exhortatióne ad Christum convérsi et a sancto Urbáno Papa baptizáti, 
 póstmodum, ob fidei confessiónem, fústibus cæsi, gládio percússi sunt; 
 Máximus vero, Præfécti 
 cubiculárius, cum et ipse, eórum permótus constántia 
 et angélica visióne firmátus, in Christum credidísset, támdiu plumbátis 
 verberátus est, donec spíritum exhaláret.
\switchcolumn
\selectlanguage{english}
At Rome, on the Appian Way, the 
 birthday of the holy martyrs Tiburtius, Valerian, and Maximus, who suffered 
 in the time of Emperor Alexander and the prefect Almachius. The first 
 two were converted to Christ by the exhortations of blessed Cecilia, and 
 baptized by Pope St. Urban. They were beaten with clubs, then beheaded 
 for the sake of the true faith. Maximus, who had been the prefect's 
 chamberlain, was touched by their constancy, and confirmed by the vision of 
 an angel, believed in Christ, and was scourged with leaded whips until he 
 died.
\switchcolumn*
\selectlanguage{latin}
Interámnæ sancti 
 Próculi, Epíscopi et Mártyris.
\switchcolumn
\selectlanguage{english}
At Teramo, St. Proculus, bishop and 
 martyr.
\switchcolumn*
\selectlanguage{latin}
Eódem die sancti 
 Ardaliónis mimi, qui dum Sacris Christianórum in theatro illúderet, 
 derepénte mutátus est, et ea non solum verbis, sed étiam testimónio sui 
 sánguinis comprobávit.
\switchcolumn
\selectlanguage{english}
Also St. Ardalion, an actor. 
 One day in the theatre, while scoffing at the holy rites of the Christian 
 religion, he was suddenly converted and bore testimony to it, not only by 
 his words, but also with his blood.
\switchcolumn*
\selectlanguage{latin}
Interámnæ sanctæ 
 Domnínæ, Vírginis et Mártyris, cum Sóciis Virgínibus coronátæ.
\switchcolumn
\selectlanguage{english}
At Teramo, St. Domnina, virgin and 
 martyr, who received the crown with her virgin companions.
\switchcolumn*
\selectlanguage{latin}
Alexandríæ sanctæ 
 Thomáidis Mártyris, quæ a sócero, cujus impudícis nolúerat consentíre votis, 
 gládio percússa est atque in duas partes per médium discíssa.
\switchcolumn
\selectlanguage{english}
At Alexandria, St. Thomais, martyr. 
 Because she would not consent to the impure wishes of her father-in-law, she 
 was struck with a sword dividing her body from head to foot.
\switchcolumn*
\selectlanguage{latin}
Lugdúni, in Gállia, 
 sancti Lambérti, Epíscopi et Confessóris.
\switchcolumn
\selectlanguage{english}
At Lyons, in France, St. Lambert, bishop and 
 confessor.
\switchcolumn*
\selectlanguage{latin}
Alexandríæ sancti 
 Frontónis Abbátis, cujus vita sanctitáte et miráculis cláruit.
\switchcolumn
\selectlanguage{english}
At Alexandria, St. Fronto, an abbot 
 whose life was graced by sanctity and his miracles.
\switchcolumn*
\selectlanguage{latin}
Romæ sancti Abúndii, 
 Mansionárii Ecclésiæ sancti Petri.
\switchcolumn
\selectlanguage{english}
At Rome, St. Abundius, sacristan of 
 the church of St. Peter.
\switchcolumn*
\selectlanguage{latin}
\end{paracol}


% ---- martyrology/mart04/mart0415.htm
\needspace{10\baselineskip}
\begin{paracol}{2}
\selectlanguage{latin}
\begin{center}{\color{gregoriocolor} Décimo séptimo Kaléndas Maji. 
 Luna\dots\ }\end{center}
\switchcolumn
\selectlanguage{english}
\begin{center}{\color{gregoriocolor} The Fifteenth Day of April. The\dots\ Day of the Moon.}\end{center}
\end{paracol}

\noindent\begin{tabularx}{\linewidth}{*{19}{>{\centering\arraybackslash}X}}
 \textcolor{gregoriocolor}{a} & \textcolor{gregoriocolor}{b} & \textcolor{gregoriocolor}{c} & \textcolor{gregoriocolor}{d} & \textcolor{gregoriocolor}{e} & \textcolor{gregoriocolor}{f} & \textcolor{gregoriocolor}{g} & \textcolor{gregoriocolor}{h} & \textcolor{gregoriocolor}{i} & \textcolor{gregoriocolor}{k} & \textcolor{gregoriocolor}{l} & \textcolor{gregoriocolor}{m} & \textcolor{gregoriocolor}{n} & \textcolor{gregoriocolor}{p} & \textcolor{gregoriocolor}{q} & \textcolor{gregoriocolor}{r} & \textcolor{gregoriocolor}{s} & \textcolor{gregoriocolor}{t} & \textcolor{gregoriocolor}{u} \\
 17 & 18 & 19 & 20 & 21 & 22 & 23 & 24 & 25 & 26 & 27 & 28 & 29 & 1 & 2 & 3 & 4 & 5 & 6 \\
\end{tabularx}
\vspace{0.5\baselineskip}
\noindent\begin{tabularx}{\linewidth}{*{12}{>{\centering\arraybackslash}X}}
 \textcolor{gregoriocolor}{A} & \textcolor{gregoriocolor}{B} & \textcolor{gregoriocolor}{C} & \textcolor{gregoriocolor}{D} & \textcolor{gregoriocolor}{E} & F & \textcolor{gregoriocolor}{F} & \textcolor{gregoriocolor}{G} & \textcolor{gregoriocolor}{H} & \textcolor{gregoriocolor}{M} & \textcolor{gregoriocolor}{N} & \textcolor{gregoriocolor}{P} \\
 7 & 8 & 9 & 10 & 11 & 12 & 11 & 12 & 13 & 14 & 15 & 16 \\
\end{tabularx}

\begin{paracol}{2}
\selectlanguage{latin}
\lettrine[lines=2]{R}{omæ} sanctárum 
 Basilíssæ et Anastásiæ, nobílium feminárum, quæ, cum essent Apostolórum 
 discípulæ et constántes in fidei confessióne persísterent, sub Neróne 
 Imperatóre, lingua pedibúsque præcísis, percússæ gládio, martyrii corónam 
 adéptæ sunt.
\switchcolumn
\selectlanguage{english}
\lettrine[lines=2]{A}{t} Rome, the Saints Basilissa and 
 Anastasia, noble women who were disciples of the apostles. Because 
 they persevered courageously in the profession of their faith during 
 the time of the Emperor Nero, they had their tongues and feet cut off, were 
 put to the sword, and thus obtained the crown of martyrdom.
\switchcolumn*
\selectlanguage{latin}
Eódem die sanctórum 
 Mártyrum Marónis, Eutychétis et Victoríni; qui, primo cum beáta Flávia 
 Domitílla apud ínsulam Póntiam in Christi confessióne 
 éxsules, póstmodum, 
 sub Príncipe Nerva, liberáti, tandem, cum plúrimos ad fidem convertíssent, 
 in persecutióne Trajáni, a Valeriáno Júdice váriis pœnis jussi sunt 
 intérfici.
\switchcolumn
\selectlanguage{english}
The same day, the holy martyrs Maro, 
 Eutyches, and Victorinus, who, along with blessed Flavia Domitilla, had been 
 banished to the island of Pontia for the confession of Christ. Being 
 recalled in the reign of Nerva, and having converted many to the faith, they 
 were put to death in different ways by the judge Valerian, during the 
 persecution of Trajan.
\switchcolumn*
\selectlanguage{latin}
In Pérside sanctórum 
 Mártyrum Máximi et Olympíadis, qui, sub Décio Imperatóre, fústibus et 
 plumbátis cæsi sunt; et ad últimum cápita eórum sunt fústibus tunsa, donec 
 ambo emítterent spíritum.
\switchcolumn
\selectlanguage{english}
In Persia, in the reign of Emperor 
 Decius, the holy martyrs Maximus and Olympias, who were beaten with rods and 
 whips, and struck on their heads with clubs until they breathed no more.
\switchcolumn*
\selectlanguage{latin}
Ferentíni, in Hérnicis, 
 sancti Eutychii Mártyris.
\switchcolumn
\selectlanguage{english}
At Ferentino in Campania, the 
 martyr St. Eutychius.
\switchcolumn*
\selectlanguage{latin}
Myræ, in Lycia, sancti 
 Crescéntis, qui per ignem martyrium consummávit.
\switchcolumn
\selectlanguage{english}
At Myra in Lycia, St. Crescens, who 
 was martyred by fire.
\switchcolumn*
\selectlanguage{latin}
In Thrácia sanctórum 
 Mártyrum Theodóri et Pausilíppi, qui sub Hadriáno Imperatóre passi sunt.
\switchcolumn
\selectlanguage{english}
In Thrace, the holy martyrs 
 Theodorus and Pausilippus, who suffered under Emperor Hadrian.
\switchcolumn*
\selectlanguage{latin}
\end{paracol}


% ---- martyrology/mart04/mart0416.htm
\needspace{10\baselineskip}
\begin{paracol}{2}
\selectlanguage{latin}
\begin{center}{\color{gregoriocolor} Sextodécimo Kaléndas Maji. 
 Luna\dots\ }\end{center}
\switchcolumn
\selectlanguage{english}
\begin{center}{\color{gregoriocolor} The Sixteenth Day of April. The\dots\ Day of the Moon.}\end{center}
\end{paracol}

\noindent\begin{tabularx}{\linewidth}{*{19}{>{\centering\arraybackslash}X}}
 \textcolor{gregoriocolor}{a} & \textcolor{gregoriocolor}{b} & \textcolor{gregoriocolor}{c} & \textcolor{gregoriocolor}{d} & \textcolor{gregoriocolor}{e} & \textcolor{gregoriocolor}{f} & \textcolor{gregoriocolor}{g} & \textcolor{gregoriocolor}{h} & \textcolor{gregoriocolor}{i} & \textcolor{gregoriocolor}{k} & \textcolor{gregoriocolor}{l} & \textcolor{gregoriocolor}{m} & \textcolor{gregoriocolor}{n} & \textcolor{gregoriocolor}{p} & \textcolor{gregoriocolor}{q} & \textcolor{gregoriocolor}{r} & \textcolor{gregoriocolor}{s} & \textcolor{gregoriocolor}{t} & \textcolor{gregoriocolor}{u} \\
 18 & 19 & 20 & 21 & 22 & 23 & 24 & 25 & 26 & 27 & 28 & 29 & 1 & 2 & 3 & 4 & 5 & 6 & 7 \\
\end{tabularx}
\vspace{0.5\baselineskip}
\noindent\begin{tabularx}{\linewidth}{*{12}{>{\centering\arraybackslash}X}}
 \textcolor{gregoriocolor}{A} & \textcolor{gregoriocolor}{B} & \textcolor{gregoriocolor}{C} & \textcolor{gregoriocolor}{D} & \textcolor{gregoriocolor}{E} & F & \textcolor{gregoriocolor}{F} & \textcolor{gregoriocolor}{G} & \textcolor{gregoriocolor}{H} & \textcolor{gregoriocolor}{M} & \textcolor{gregoriocolor}{N} & \textcolor{gregoriocolor}{P} \\
 8 & 9 & 10 & 11 & 12 & 13 & 12 & 13 & 14 & 15 & 16 & 17 \\
\end{tabularx}

\begin{paracol}{2}
\selectlanguage{latin}
\lettrine[lines=2]{C}{orínthi} natális 
 sanctórum Mártyrum Callísti et Charísii, cum áliis septem, qui omnes, post 
 ália toleráta supplícia, in mare demérsi sunt.
\switchcolumn
\selectlanguage{english}
\lettrine[lines=2]{A}{t} Corinth, the birthday of the 
 holy martyrs Callistus and Charistius, with seven others, who were all cast 
 into the sea.
\switchcolumn*
\selectlanguage{latin}
Cæsaraugústæ, in 
 Hispánia, item natális sanctórum decem et octo Mártyrum, scílicet Optáti, 
 Lupérci, Succéssi, Martiális, Urbáni, Júliæ, Quinctiliáni, Públii, Frontónis, 
 Felícis, Cæciliáni, Evéntii, Primitívi, Apodémii, et aliórum quátuor qui 
 Saturníni vocáti esse referúntur. Hi omnes sub Daciáno, Hispaniárum 
 Præside, simul pœnis affécti atque interémpti sunt; quorum illústre 
 martyrium Prudéntius vérsibus exornávit.
\switchcolumn
\selectlanguage{english}
At Saragossa, in Spain, the 
 birthday of eighteen holy martyrs, Optatus, Lupercus, Successus, Martial, 
 Urban, Julia, Quinctilian, Publius, Fronto, Felix, Cecilian, Eventius, 
 Primitivus, Apodemius, and four others who are said to have been Saturninus. 
 They were all tortured and slain together under Dacian, governor of Spain. 
 The glory of their martyrdom has been celebrated in verse by Prudentius.
\switchcolumn*
\selectlanguage{latin}
In eádem civitáte 
 sanctórum Caji et Creméntii, qui secúndo conféssi et in Christi fide 
 perseverántes, martyrii cálicem gustavérunt.
\switchcolumn
\selectlanguage{english}
In the same city, the Saints Caius 
 and Crementius, who twice confessed the faith of Christ, and persevering in 
 it, drank of the chalice of martyrdom.
\switchcolumn*
\selectlanguage{latin}
Ibídem sancti Lambérti 
 Mártyris.
\switchcolumn
\selectlanguage{english}
In the same place, the martyr St. 
 Lambert.
\switchcolumn*
\selectlanguage{latin}
Item Cæsaraugústæ 
 sanctæ Encrátidis, Vírginis et Mártyris, quæ, laniáto córpore, mamílla 
 abscíssa et jécore avúlso, adhuc supérstes in cárcere inclúsa et reténta est, 
 donec ulcerátum ipsíus corpus putrésceret.
\switchcolumn
\selectlanguage{english}
Also at Saragossa, St. Encratis, 
 virgin and martyr, whose body was lacerated, her breasts cut away, and her 
 bowels torn out. Still alive after these torments, she was confined in 
 prison until her body, covered with wounds, began to decompose.
\switchcolumn*
\selectlanguage{latin}
Paléntiæ sancti 
 Turíbii, Epíscopi Asturicénsis, qui, ope sancti Leónis Papæ, Priscilliáni 
 hæresim ex Hispánia pénitus profligávit, clarúsque miráculis quiévit in 
 pace.
\switchcolumn
\selectlanguage{english}
At Palentia, St. Turibius, bishop 
 of Astorga. With the aid of Pope St. Leo, he drove out of Spain 
 completely the Priscillian heresy. He went to rest in the Lord with a 
 great renown for miracles.
\switchcolumn*
\selectlanguage{latin}
Brácari, in Lusitánia, 
 sancti Fructuósi Epíscopi.
\switchcolumn
\selectlanguage{english}
At Braga in Portugal, the bishop 
 St. Fructuosus.
\switchcolumn*
\selectlanguage{latin}
Sescíaci, in 
 Constantiénsi Gálliæ território, tránsitus sancti Patérni, Epíscopi 
 Abrincénsis et Confessóris.
\switchcolumn
\selectlanguage{english}
At Scicy, in the district of 
 Coutances in France, the death of St. Paternus, bishop of Avranches and 
 confessor.
\switchcolumn*
\selectlanguage{latin}
Romæ natális sancti 
 Benedícti-Joséphi Labre Confessóris, qui contémptu sui et extrémæ voluntáriæ 
 paupertátis laude exstitit insígnis.
\switchcolumn
\selectlanguage{english}
At Rome, the birthday of St. 
 Benedict Joseph Labre, confessor, who was famed for his contempt of self and 
 his great voluntary poverty.
\switchcolumn*
\selectlanguage{latin}
Apud Valencénas, in 
 Gállia, sancti Drogónis Confessóris.
\switchcolumn
\selectlanguage{english}
In Belgium, near Valenciennes, St. 
 Drogo, confessor.
\switchcolumn*
\selectlanguage{latin}
Nivérnis, in Gállia, 
 sanctæ Maríæ-Bernárdæ Soubirous, Vírginis, e Congregatióne Sorórum a 
 Caritáte et Institutióne Christiána, Lapúrdi, adhuc juvénculæ, iterátis 
 apparitiónibus Immaculátæ Dei Genitrícis Maríæ recreátæ; quam Pius Papa 
 Undécimus, inter sanctas Vírgines adscrípsit.
\switchcolumn
\selectlanguage{english}
In the city of Nevers in France, 
 St. Mary Bernard Soubirous of the Congregation of the Sisters of Charity, 
 also called the Christian Institute. She was favoured with frequent 
 apparitions and conversations at Lourdes with Mary Immaculate, the Mother of 
 God. In 1933 her name was added to the roll of holy virgins by Pope 
 Pius XI.
\switchcolumn*
\selectlanguage{latin}
Senis, in Túscia, 
 beáti Jóachim, ex Ordine Servórum beátæ Maríæ Vírginis.
\switchcolumn
\selectlanguage{english}
At Siena in Tuscany, blessed 
 Joachim of the Order of Servites of the Blessed Virgin Mary.
\switchcolumn*
\selectlanguage{latin}
\end{paracol}


% ---- martyrology/mart04/mart0417.htm
\needspace{10\baselineskip}
\begin{paracol}{2}
\selectlanguage{latin}
\begin{center}{\color{gregoriocolor} Quintodécimo Kaléndas Maji. 
 Luna\dots\ }\end{center}
\switchcolumn
\selectlanguage{english}
\begin{center}{\color{gregoriocolor} The Seventeenth Day of April. The\dots\ Day of the Moon.}\end{center}
\end{paracol}

\noindent\begin{tabularx}{\linewidth}{*{19}{>{\centering\arraybackslash}X}}
 \textcolor{gregoriocolor}{a} & \textcolor{gregoriocolor}{b} & \textcolor{gregoriocolor}{c} & \textcolor{gregoriocolor}{d} & \textcolor{gregoriocolor}{e} & \textcolor{gregoriocolor}{f} & \textcolor{gregoriocolor}{g} & \textcolor{gregoriocolor}{h} & \textcolor{gregoriocolor}{i} & \textcolor{gregoriocolor}{k} & \textcolor{gregoriocolor}{l} & \textcolor{gregoriocolor}{m} & \textcolor{gregoriocolor}{n} & \textcolor{gregoriocolor}{p} & \textcolor{gregoriocolor}{q} & \textcolor{gregoriocolor}{r} & \textcolor{gregoriocolor}{s} & \textcolor{gregoriocolor}{t} & \textcolor{gregoriocolor}{u} \\
 19 & 20 & 21 & 22 & 23 & 24 & 25 & 26 & 27 & 28 & 29 & 1 & 2 & 3 & 4 & 5 & 6 & 7 & 8 \\
\end{tabularx}
\vspace{0.5\baselineskip}
\noindent\begin{tabularx}{\linewidth}{*{12}{>{\centering\arraybackslash}X}}
 \textcolor{gregoriocolor}{A} & \textcolor{gregoriocolor}{B} & \textcolor{gregoriocolor}{C} & \textcolor{gregoriocolor}{D} & \textcolor{gregoriocolor}{E} & F & \textcolor{gregoriocolor}{F} & \textcolor{gregoriocolor}{G} & \textcolor{gregoriocolor}{H} & \textcolor{gregoriocolor}{M} & \textcolor{gregoriocolor}{N} & \textcolor{gregoriocolor}{P} \\
 9 & 10 & 11 & 12 & 13 & 14 & 13 & 14 & 15 & 16 & 17 & 18 \\
\end{tabularx}

\begin{paracol}{2}
\selectlanguage{latin}
\lettrine[lines=2]{R}{omæ} sancti Anicéti, 
 Papæ et Mártyris; qui martyrii palmam in persecutióne Marci Aurélii Antoníni 
 et Lúcii Veri accépit.
\switchcolumn
\selectlanguage{english}
\lettrine[lines=2]{A}{t} Rome, St. Anicetus, pope and 
 martyr, who received the palm of martyrdom in the persecution of Marcus 
 Aurelius Antoninus and Lucius Verus.
\switchcolumn*
\selectlanguage{latin}
Córdubæ, in Hispánia, sanctórum Mártyrum Elíæ Presbyteri, Pauli et Isidóri Monachórum, qui, in 
 persecutióne Arábica, ob Christiánæ fidei professiónem, interémpti sunt.
\switchcolumn
\selectlanguage{english}
At Cordova in Spain, the holy 
 martyrs Elias, a priest, and the monks Paul and Isidore, who were slain in 
 the Arab persecution for the profession of the Christian faith.
\switchcolumn*
\selectlanguage{latin}
Antiochíæ sanctórum 
 Mártyrum Petri Diáconi, et Hermógenis, qui ejúsdem Petri erat miníster.
\switchcolumn
\selectlanguage{english}
At Antioch, the holy martyrs Peter, 
 a deacon, and Hermogenes, who was his servant.
\switchcolumn*
\selectlanguage{latin}
In Africa natális 
 beáti Mappálici Mártyris, qui (ut scribit sanctus Cypriánus in epístola ad 
 Mártyres et Confessóres), cum áliis plúribus, martyrio coronátus est.
\switchcolumn
\selectlanguage{english}
In Africa, the birthday of blessed 
 Mappalicus, martyr. St. Cyprian relates in his Epistle to the Martyrs 
 and Confessors that he, along with many others, was crowned with martyrdom.
\switchcolumn*
\selectlanguage{latin}
Ibídem sanctórum 
 Mártyrum Fortunáti et Marciáni.
\switchcolumn
\selectlanguage{english}
In the same place, the holy martyrs 
 Fortunatus and Marcian.
\switchcolumn*
\selectlanguage{latin}
Viénnæ, in Gállia, 
 sancti Pantágathi Epíscopi.
\switchcolumn
\selectlanguage{english}
At Vienne in France, Bishop St. 
 Pantagathus.
\switchcolumn*
\selectlanguage{latin}
Dertónæ sancti 
 Innocéntii, Epíscopi et Confessóris.
\switchcolumn
\selectlanguage{english}
At Tortona, St. Innocent, bishop 
 and confessor.
\switchcolumn*
\selectlanguage{latin}
Cistércii, in Gállia, 
 sancti Stéphani Abbátis, qui primus erémum Cisterciénsem incóluit, et 
 sanctum Bernárdum, cum sóciis ad se veniéntem, lætus excépit.
\switchcolumn
\selectlanguage{english}
At Citeaux in France, St. Stephen Harding, 
 abbot, who was first to live in the Cistercian desert and who joyfully 
 welcomed St. Bernard and his companions when they came to him.
\switchcolumn*
\selectlanguage{latin}
In monastério Casæ 
 Dei, Claromontánæ diœcésis, in Gállia, sancti Robérti Confessóris, qui 
 ejúsdem monastérii cónditor et primus Abbas éxstitit.
\switchcolumn
\selectlanguage{english}
In the monastery of Chaise-Dieu, in 
 the diocese of Clermont, St. Robert, confessor, the founder and first abbot 
 of the monastery.
\switchcolumn*
\selectlanguage{latin}
\end{paracol}


% ---- martyrology/mart04/mart0418.htm
\needspace{10\baselineskip}
\begin{paracol}{2}
\selectlanguage{latin}
\begin{center}{\color{gregoriocolor} Quartodécimo Kaléndas Maji. 
 Luna\dots\ }\end{center}
\switchcolumn
\selectlanguage{english}
\begin{center}{\color{gregoriocolor} The Eighteenth Day of April. The\dots\ Day of the Moon.}\end{center}
\end{paracol}

\noindent\begin{tabularx}{\linewidth}{*{19}{>{\centering\arraybackslash}X}}
 \textcolor{gregoriocolor}{a} & \textcolor{gregoriocolor}{b} & \textcolor{gregoriocolor}{c} & \textcolor{gregoriocolor}{d} & \textcolor{gregoriocolor}{e} & \textcolor{gregoriocolor}{f} & \textcolor{gregoriocolor}{g} & \textcolor{gregoriocolor}{h} & \textcolor{gregoriocolor}{i} & \textcolor{gregoriocolor}{k} & \textcolor{gregoriocolor}{l} & \textcolor{gregoriocolor}{m} & \textcolor{gregoriocolor}{n} & \textcolor{gregoriocolor}{p} & \textcolor{gregoriocolor}{q} & \textcolor{gregoriocolor}{r} & \textcolor{gregoriocolor}{s} & \textcolor{gregoriocolor}{t} & \textcolor{gregoriocolor}{u} \\
 20 & 21 & 22 & 23 & 24 & 25 & 26 & 27 & 28 & 29 & 1 & 2 & 3 & 4 & 5 & 6 & 7 & 8 & 9 \\
\end{tabularx}
\vspace{0.5\baselineskip}
\noindent\begin{tabularx}{\linewidth}{*{12}{>{\centering\arraybackslash}X}}
 \textcolor{gregoriocolor}{A} & \textcolor{gregoriocolor}{B} & \textcolor{gregoriocolor}{C} & \textcolor{gregoriocolor}{D} & \textcolor{gregoriocolor}{E} & F & \textcolor{gregoriocolor}{F} & \textcolor{gregoriocolor}{G} & \textcolor{gregoriocolor}{H} & \textcolor{gregoriocolor}{M} & \textcolor{gregoriocolor}{N} & \textcolor{gregoriocolor}{P} \\
 10 & 11 & 12 & 13 & 14 & 15 & 14 & 15 & 16 & 17 & 18 & 19 \\
\end{tabularx}

\begin{paracol}{2}
\selectlanguage{latin}
\lettrine[lines=2]{A}{pud} montem Senárium, 
 in Etrúria, natális sancti Amidǽi Confessóris, e septem Fundatóribus Ordinis 
 Servórum beátæ Maríæ Vírginis, flagrantíssima in Deum caritáte præclári. 
 Ipsíus tamen ac Sociórum festum prídie Idus Februárii celebrátur.
\switchcolumn
\selectlanguage{english}
\lettrine[lines=2]{O}{n} Mount Senario in Tuscany, St. 
 Amadeo, confessor, one of the seven founders of the Order of Servites of the 
 Blessed Virgin Mary, famous for his ardent love for God. His feast, 
 together with that of his companions, is kept on the 12th of February.
\switchcolumn*
\selectlanguage{latin}
Romæ beáti Apollónii 
 Senatóris, qui, sub Cómmodo Príncipe et Perénnio Præfécto, a servo próditus 
 quod Christiánus esset, jussúsque ut ratiónem fídei suæ rédderet, insígne volúmen compósuit, quod in Senátu legit; et nihilóminus pro Christo, 
 senténtia Senátus, cápite truncátus est.
\switchcolumn
\selectlanguage{english}
At Rome, blessed Apollonius, a 
 senator under Emperor Commodus and the prefect Perennius. He was 
 denounced as a Christian by one of his slaves, and being commanded to give 
 an account of his faith, he composed an able work which he read in the 
 Senate. He was nevertheless beheaded for Christ by their sentence.
\switchcolumn*
\selectlanguage{latin}
Messánæ, in Sicília, 
 natális sanctórum Mártyrum Eleuthérii, Epíscopi Illyrici, et Anthíæ matris. 
 Ipse, cum esset vitæ sanctimónia et miraculórum virtúte illústris, sub 
 Hadriáno Príncipe, lectum férreum ignítum, cratículam et sartáginem, óleo et 
 pice ac resína fervéntem, súperans, projéctus quoque leónibus sed ab illis 
 nil læsus, novíssime simul cum matre jugulátur.
\switchcolumn
\selectlanguage{english}
At Messina in Sicily, the birthday 
 of the holy martyrs Eleutherius, bishop of Illyria, and Anthia, his mother. 
 He was famous for holiness of life and the power of miracles. During 
 the reign of Hadrian, he was placed on a bed of red-hot iron, on a gridiron, 
 in a vessel filled with boiling oil, pitch, and resin, and also cast to the 
 lions; but remaining unhurt through all of this, they finally cut his throat 
 with a sword. His mother suffered the same torments.
\switchcolumn*
\selectlanguage{latin}
Córdubæ, in Hispánia, sancti Perfécti, Presbyteri et Mártyris; qui a Mauris, eo quod inveherétur 
 in Mahumétis sectam et fírmiter Christi fidem profiterétur, gládio 
 trucidátus est.
\switchcolumn
\selectlanguage{english}
At Cordova, St. Perfectus, priest 
 and martyr, who was slain with the sword by the Moors, because he argued 
 against the sect of Mohammed and firmly insisted on the Catholic faith.
\switchcolumn*
\selectlanguage{latin}
Messánæ, in Sicília, 
 sancti Corébi Præfécti, qui, per sanctum Eleuthérium convérsus ad fidem, 
 gládio percússus est.
\switchcolumn
\selectlanguage{english}
At Messina in Sicily, St. Corebus, 
 the prefect, who was converted to the faith by St. Eleutherius, and died by 
 the sword.
\switchcolumn*
\selectlanguage{latin}
Bríxiæ sancti Calóceri 
 Mártyris, qui, a sanctis Faustíno et Jovíta convérsus ad Christum, sub 
 Hadriáno Príncipe gloriósum confessiónis certámen complévit.
\switchcolumn
\selectlanguage{english}
At Brescia, the martyr St. 
 Calocerus, who was converted to Christ by Saints Faustinus and Jovita, and 
 who gloriously triumphed in the test of his confession, in the time of 
 Hadrian.
\switchcolumn*
\selectlanguage{latin}
Medioláni sancti 
 Galdíni, Cardinális et ejúsdem civitátis Epíscopi, qui, concióne advérsus 
 hæréticos expléta, spíritum Deo réddidit.
\switchcolumn
\selectlanguage{english}
At Milan, St. Galdino, cardinal 
 bishop of that city, who at the very end of a sermon against heretics, gave 
 up his soul to God.
\switchcolumn*
\selectlanguage{latin}
\end{paracol}


% ---- martyrology/mart04/mart0419.htm
\needspace{10\baselineskip}
\begin{paracol}{2}
\selectlanguage{latin}
\begin{center}{\color{gregoriocolor} Tertiodécimo Kaléndas Maji. 
 Luna\dots\ }\end{center}
\switchcolumn
\selectlanguage{english}
\begin{center}{\color{gregoriocolor} The Nineteenth Day of April. The\dots\ Day of the Moon.}\end{center}
\end{paracol}

\noindent\begin{tabularx}{\linewidth}{*{19}{>{\centering\arraybackslash}X}}
 \textcolor{gregoriocolor}{a} & \textcolor{gregoriocolor}{b} & \textcolor{gregoriocolor}{c} & \textcolor{gregoriocolor}{d} & \textcolor{gregoriocolor}{e} & \textcolor{gregoriocolor}{f} & \textcolor{gregoriocolor}{g} & \textcolor{gregoriocolor}{h} & \textcolor{gregoriocolor}{i} & \textcolor{gregoriocolor}{k} & \textcolor{gregoriocolor}{l} & \textcolor{gregoriocolor}{m} & \textcolor{gregoriocolor}{n} & \textcolor{gregoriocolor}{p} & \textcolor{gregoriocolor}{q} & \textcolor{gregoriocolor}{r} & \textcolor{gregoriocolor}{s} & \textcolor{gregoriocolor}{t} & \textcolor{gregoriocolor}{u} \\
 21 & 22 & 23 & 24 & 25 & 26 & 27 & 28 & 29 & 1 & 2 & 3 & 4 & 5 & 6 & 7 & 8 & 9 & 10 \\
\end{tabularx}
\vspace{0.5\baselineskip}
\noindent\begin{tabularx}{\linewidth}{*{12}{>{\centering\arraybackslash}X}}
 \textcolor{gregoriocolor}{A} & \textcolor{gregoriocolor}{B} & \textcolor{gregoriocolor}{C} & \textcolor{gregoriocolor}{D} & \textcolor{gregoriocolor}{E} & F & \textcolor{gregoriocolor}{F} & \textcolor{gregoriocolor}{G} & \textcolor{gregoriocolor}{H} & \textcolor{gregoriocolor}{M} & \textcolor{gregoriocolor}{N} & \textcolor{gregoriocolor}{P} \\
 11 & 12 & 13 & 14 & 15 & 16 & 15 & 16 & 17 & 18 & 19 & 20 \\
\end{tabularx}

\begin{paracol}{2}
\selectlanguage{latin}
\lettrine[lines=2]{C}{orínthi} natális 
 sancti Timónis, qui fuit unus de septem primis Diáconis. Hic primo 
 apud Berœam Doctor resédit, ac deínde, verbum Dómini disséminans, venit 
 Corínthum; ibíque, a Judǽis et Græcis (ut tráditur) injéctus flammis, sed 
 nihil læsus, demum, cruci affíxus, martyrium suum implévit.
\switchcolumn
\selectlanguage{english}
\lettrine[lines=2]{A}{t} Corinth, the birthday of St. 
 Timon, one of the first seven deacons, who was first a teacher at Berea. 
 Afterwards, while preaching the word of the Lord at Corinth, he was 
 delivered to the flames by the Jews and the Greeks, but remaining uninjured, 
 he ended his martyrdom by crucifixion.
\switchcolumn*
\selectlanguage{latin}
Cantuáriæ, in Anglia, 
 sancti Elphégi, Epíscopi et Mártyris.
\switchcolumn
\selectlanguage{english}
At Canterbury in England, St. 
 Alphege, bishop and martyr.
\switchcolumn*
\selectlanguage{latin}
Melitínæ, in Arménia, sanctórum Mártyrum Hermógenis, Caji, Expedíti, Aristónici, Rufi et Galátæ, 
 qui omnes una die sunt coronáti.
\switchcolumn
\selectlanguage{english}
At Melitine in Armenia, the holy 
 martyrs Hermogenes, Caius, Expeditus, Aristonicus, Rufus, and Galatas, all 
 crowned on the same day.
\switchcolumn*
\selectlanguage{latin}
Caucolíberi, in 
 Hispánia Tarraconénsi, pássio sancti Vincéntii Mártyris.
\switchcolumn
\selectlanguage{english}
At Collioure in Spain, the martyr 
 St. Vincent.
\switchcolumn*
\selectlanguage{latin}
Eódem die sanctórum 
 Mártyrum Socrátis et Dionysii, qui lánceis confóssi sunt.
\switchcolumn
\selectlanguage{english}
On the same day, the holy martyrs 
 Socrates and Denis, who were killed with spears.
\switchcolumn*
\selectlanguage{latin}
Hierosólymis sancti 
 Paphnútii Mártyris.
\switchcolumn
\selectlanguage{english}
At Jerusalem, the martyr St. 
 Paphnutius.
\switchcolumn*
\selectlanguage{latin}
Romæ sancti Leónis 
 Papæ Noni, virtútum et miraculórum laude insígnis.
\switchcolumn
\selectlanguage{english}
At Rome, Pope St. Leo IX, 
 illustrious for his virtues and his miracles.
\switchcolumn*
\selectlanguage{latin}
Antiochíæ Pisídiæ 
 sancti Geórgii Epíscopi, qui, ob sanctárum Imáginum cultum, exsul occúbuit.
\switchcolumn
\selectlanguage{english}
At Antioch in Pisidia, St. George, 
 a bishop, who died in exile for the veneration of sacred images.
\switchcolumn*
\selectlanguage{latin}
In cœnóbio Laubiénsi, 
 in Bélgio, sancti Ursmári Epíscopi.
\switchcolumn
\selectlanguage{english}
In the monastery of Lobbes in Belgium, the 
 bishop St. Ursmar.
\switchcolumn*
\selectlanguage{latin}
Floréntiæ sancti 
 Crescéntii Confessóris, qui fuit discípulus beáti Zenóbii Epíscopi.
\switchcolumn
\selectlanguage{english}
At Florence, St. Crescent, 
 confessor, a disciple of the blessed Bishop Zenobius.
\switchcolumn*
\selectlanguage{latin}
\end{paracol}


% ---- martyrology/mart04/mart0420.htm
\needspace{10\baselineskip}
\begin{paracol}{2}
\selectlanguage{latin}
\begin{center}{\color{gregoriocolor} Duodécimo Kaléndas Maji. 
 Luna\dots\ }\end{center}
\switchcolumn
\selectlanguage{english}
\begin{center}{\color{gregoriocolor} The 
 Twentieth Day of April. The\dots\ Day of the Moon.}\end{center}
\end{paracol}

\noindent\begin{tabularx}{\linewidth}{*{19}{>{\centering\arraybackslash}X}}
 \textcolor{gregoriocolor}{a} & \textcolor{gregoriocolor}{b} & \textcolor{gregoriocolor}{c} & \textcolor{gregoriocolor}{d} & \textcolor{gregoriocolor}{e} & \textcolor{gregoriocolor}{f} & \textcolor{gregoriocolor}{g} & \textcolor{gregoriocolor}{h} & \textcolor{gregoriocolor}{i} & \textcolor{gregoriocolor}{k} & \textcolor{gregoriocolor}{l} & \textcolor{gregoriocolor}{m} & \textcolor{gregoriocolor}{n} & \textcolor{gregoriocolor}{p} & \textcolor{gregoriocolor}{q} & \textcolor{gregoriocolor}{r} & \textcolor{gregoriocolor}{s} & \textcolor{gregoriocolor}{t} & \textcolor{gregoriocolor}{u} \\
 22 & 23 & 24 & 25 & 26 & 27 & 28 & 29 & 1 & 2 & 3 & 4 & 5 & 6 & 7 & 8 & 9 & 10 & 11 \\
\end{tabularx}
\vspace{0.5\baselineskip}
\noindent\begin{tabularx}{\linewidth}{*{12}{>{\centering\arraybackslash}X}}
 \textcolor{gregoriocolor}{A} & \textcolor{gregoriocolor}{B} & \textcolor{gregoriocolor}{C} & \textcolor{gregoriocolor}{D} & \textcolor{gregoriocolor}{E} & F & \textcolor{gregoriocolor}{F} & \textcolor{gregoriocolor}{G} & \textcolor{gregoriocolor}{H} & \textcolor{gregoriocolor}{M} & \textcolor{gregoriocolor}{N} & \textcolor{gregoriocolor}{P} \\
 12 & 13 & 14 & 15 & 16 & 17 & 16 & 17 & 18 & 19 & 20 & 21 \\
\end{tabularx}

\begin{paracol}{2}
\selectlanguage{latin}
\lettrine[lines=2]{R}{omæ} sanctórum 
 Mártyrum Sulpícii et Serviliáni, qui, prædicatióne et miráculis beátæ 
 Domitíllæ Vírginis ad Christi fidem convérsi, ambo, cum nollent idólis 
 immoláre, in persecutióne Trajáni, a Præfécto Urbis Aniáno sunt cápite cæsi.
\switchcolumn
\selectlanguage{english}
\lettrine[lines=2]{A}{t} Rome, the holy martyrs Sulpicius 
 and Servilian, who were converted to the faith of Christ by the speeches and 
 the miracles of the holy virgin Domitilla. Because they refused to 
 sacrifice to the idols, they were beheaded by Anian, prefect of the city, in 
 the persecution of Trajan.
\switchcolumn*
\selectlanguage{latin}
Nicomedíæ sanctórum 
 Mártyrum Victóris, Zótici, Zenónis, Acíndyni, Cæsárei, Severiáni, 
 Chrysóphori, Theónæ et Antoníni, qui, sub Diocletiáno Imperatóre, passióne 
 ac signis beáti Geórgii ad Christum convérsi sunt, et ob intrépidam fídei 
 confessiónem, várie tentáti, martyrium complevérunt.
\switchcolumn
\selectlanguage{english}
At Nicomedia, the holy martyrs 
 Victor, Zoticus, Zeno, Acindynus, Caesareus, Severian, Chrysophorus, Theonas, 
 and Antonine. They were converted to Christ by the miracles and the 
 martyrdom of St. George, and because of their own dauntless confession of 
 the faith, they were tortured in various ways under the Emperor Diocletian, 
 and thus completed their martyrdom.
\switchcolumn*
\selectlanguage{latin}
Tomis, in Scythia, 
 sancti Theótimi Epíscopi, quem, ob insígnem ipsíus sanctitátem atque 
 mirácula, étiam infidéles bárbari veneráti sunt.
\switchcolumn
\selectlanguage{english}
At Tomis in Scythia, Bishop St. 
 Theotimus, whose great sanctity and miracles procured him the respect even 
 of unbelieving barbarians.
\switchcolumn*
\selectlanguage{latin}
Ebredúni, in Gálliis, 
 sancti Marcellíni, qui fuit primus ejúsdem urbis Epíscopus. Hic, Dei 
 mónitu, cum sanctis Sóciis Vincéntio et Domníno, ex Africa venit, et máximam 
 Alpium maritimárum partem verbo et signis admirándis, quibus usque hódie 
 refúlget, ad Christi fidem convértit.
\switchcolumn
\selectlanguage{english}
At Embrun in France, St. Marcellin, 
 first bishop of that city. By divine inspiration he came from Africa 
 with his holy companions Vincent and Domninus, and converted the greater 
 portion of the inhabitants of the Maritime Alps by his preaching, and by the 
 wonderful prodigies which he still continues to work.
\switchcolumn*
\selectlanguage{latin}
Antisiodóri sancti 
 Marciáni Presbyteri.
\switchcolumn
\selectlanguage{english}
At Auxerre, the priest St. Marcian.
\switchcolumn*
\selectlanguage{latin}
Apud Constantinópolim 
 sancti Theodóri Confessóris, ab áspera cilícii veste, qua tegebátur, 
 cognoménto Tríchinas, qui multis virtútibus, præsértim advérsus dæmones, 
 cláruit; ex cujus córpore scatúriens unguéntum ægrótis sanitátem impértit.
\switchcolumn
\selectlanguage{english}
At Constantinople, St. Theodore, 
 confessor, surnamed Trichinas, from the rough garment of hair which he wore. 
 He was renowned for many miracles, but especially for his power over the 
 demons. From his body issues a liquid that imparts health to the sick.
\switchcolumn*
\selectlanguage{latin}
In Monte Politiáno, in 
 Túscia, sanctæ Agnétis Vírginis, ex Ordine sancti Domínici, miráculis claræ.
\switchcolumn
\selectlanguage{english}
At Monte Pulciano, St. Agnes, a 
 virgin of the Order of St. Dominic, celebrated for her miracles.
\switchcolumn*
\selectlanguage{latin}
\end{paracol}


% ---- martyrology/mart04/mart0421.htm
\needspace{10\baselineskip}
\begin{paracol}{2}
\selectlanguage{latin}
\begin{center}{\color{gregoriocolor} Undécimo Kaléndas Maji. 
 Luna\dots\ }\end{center}
\switchcolumn
\selectlanguage{english}
\begin{center}{\color{gregoriocolor} The 
 Twenty-First Day of April. The\dots\ Day of the Moon.}\end{center}
\end{paracol}

\noindent\begin{tabularx}{\linewidth}{*{19}{>{\centering\arraybackslash}X}}
 \textcolor{gregoriocolor}{a} & \textcolor{gregoriocolor}{b} & \textcolor{gregoriocolor}{c} & \textcolor{gregoriocolor}{d} & \textcolor{gregoriocolor}{e} & \textcolor{gregoriocolor}{f} & \textcolor{gregoriocolor}{g} & \textcolor{gregoriocolor}{h} & \textcolor{gregoriocolor}{i} & \textcolor{gregoriocolor}{k} & \textcolor{gregoriocolor}{l} & \textcolor{gregoriocolor}{m} & \textcolor{gregoriocolor}{n} & \textcolor{gregoriocolor}{p} & \textcolor{gregoriocolor}{q} & \textcolor{gregoriocolor}{r} & \textcolor{gregoriocolor}{s} & \textcolor{gregoriocolor}{t} & \textcolor{gregoriocolor}{u} \\
 23 & 24 & 25 & 26 & 27 & 28 & 29 & 1 & 2 & 3 & 4 & 5 & 6 & 7 & 8 & 9 & 10 & 11 & 12 \\
\end{tabularx}
\vspace{0.5\baselineskip}
\noindent\begin{tabularx}{\linewidth}{*{12}{>{\centering\arraybackslash}X}}
 \textcolor{gregoriocolor}{A} & \textcolor{gregoriocolor}{B} & \textcolor{gregoriocolor}{C} & \textcolor{gregoriocolor}{D} & \textcolor{gregoriocolor}{E} & F & \textcolor{gregoriocolor}{F} & \textcolor{gregoriocolor}{G} & \textcolor{gregoriocolor}{H} & \textcolor{gregoriocolor}{M} & \textcolor{gregoriocolor}{N} & \textcolor{gregoriocolor}{P} \\
 13 & 14 & 15 & 16 & 17 & 18 & 17 & 18 & 19 & 20 & 21 & 22 \\
\end{tabularx}

\begin{paracol}{2}
\selectlanguage{latin}
\lettrine[lines=2]{C}{antuáriæ,} in Anglia, 
 sancti Ansélmi Epíscopi, Confessóris et Ecclésiæ Doctóris, sanctitáte et 
 doctrína conspícui.
\switchcolumn
\selectlanguage{english}
\lettrine[lines=2]{A}{t} Canterbury, England, St. Anselm, 
 bishop, confessor, and doctor of the Church, renowned for sanctity and 
 learning.
\switchcolumn*
\selectlanguage{latin}
In Pérside natális 
 sancti Simeónis, Epíscopi Seleucíæ et Ctesiphóntis, qui, jubénte Rege 
 Persárum Sápore, comprehénsus ferróque onústus, iníquis tribunálibus 
 exhíbitus, et, cum Solem ipsum adoráre nollet et de Jesu Christo voce líbera 
 et constantíssima testarétur, primum carceráli ergástulo, cum áliis centum 
 (ex quibus álii Epíscopi, álii erant Presbyteri, álii diversórum órdinum 
 Clérici), longo témpore macerátus est. Deínde, cum Usthazánes, Regis 
 nutrítius, qui ante jam lapsus a fide, sed per eum ad pæniténtiam fúerat 
 revocátus, martyrium constánter subiísset, postrídie, qui erat ánnuus 
 Domínicæ passiónis dies, ómnibus ante Simeónis óculos, qui unumquémque eórum 
 strénue exhortabátur, gládio jugulátis, novíssime et ipse decollátus est. 
 Passi sunt étiam cum ipso claríssimi viri Abdéchalas et Ananías, qui ejus 
 erant Presbyteri, Pusícius quoque, Præféctus artíficum Regis, eo quod 
 Ananíam titubántem corroborásset, ídeo, collo circa téndinem perforáto et 
 lingua exínde extrácta, crudéli morte occúbuit; post quem et fília 
 cruciátibus ac demum ense decollátus est.
\switchcolumn
\selectlanguage{english}
In Persia, the birthday of St. 
 Simeon, bishop of Seleucia and Ctesiphon. He was arrested by order of 
 Sapor, king of Persia, loaded with irons, and presented to the iniquitous 
 tribunals. As he refused to adore the sun, and openly and constantly 
 bore testimony to Jesus Christ, he was confined for a long time in a dungeon 
 with one hundred other confessors, some of whom were bishops. others 
 priests, others clerics of various ranks. Afterwards, Usthazanes, the 
 king's foster-father, who had been converted from apostasy by Simeon, 
 endured martyrdom with great constancy. The day after, which was the 
 anniversary of our Lord's Passion, the companions of Simeon whom he had 
 feelingly exhorted, were beheaded before his eyes, after which he met the 
 same fate. With him suffered also several distinguished men: 
 Abdechalas and Ananias, his priests, with Pusicius, the head of the royal 
 workmen. This last having encouraged Ananias, who seemed to falter, 
 died a cruel death, having his tongue drawn out through a perforation made 
 in his neck. After him, his daughter, who was a consecrated virgin, 
 was put to death.
\switchcolumn*
\selectlanguage{latin}
Alexandríæ sanctórum 
 Mártyrum Aratóris Presbyteri, Fortunáti, Felícis, Sílvii et Vitális, qui in cárcere quievérunt.
\switchcolumn
\selectlanguage{english}
At Alexandria, the holy martyrs 
 Arator, a priest, Fortunatus, Felix, Silvius, and Vitalis, who all died in 
 prison.
\switchcolumn*
\selectlanguage{latin}
Nicomedíæ sanctórum 
 Mártyrum Apóllinis, Isácii et Codráti; e quibus, sub Diocletiáno Imperatóre, 
 últimus cápite plexus, et, paucis post illum diébus, duo primi in vínculis 
 fame confécti, martyrii corónam meruérunt.
\switchcolumn
\selectlanguage{english}
At Nicomedia, the holy martyrs 
 Apollo, Isacius, and Codratus, who suffered under the Emperor Diocletian. 
 The last of these was slain by the sword, and a few days later the other two 
 died from starvation in prison, meriting also the crown of martyrdom.
\switchcolumn*
\selectlanguage{latin}
Antiochíæ sancti 
 Anastásii Sinaítæ Epíscopi.
\switchcolumn
\selectlanguage{english}
At Antioch, St. Anastasius the 
 Sinaite, bishop.
\switchcolumn*
\selectlanguage{latin}
Œttingæ Véteris, in 
 Bavária, sancti Conrádi a Parzham, Confessóris, Ordinis Minórum Capuccinórum, 
 caritáte et oratióne insígnis; quem, miráculis clarum, Pius Papa Undécimus 
 Sanctórum número adscrípsit.
\switchcolumn
\selectlanguage{english}
At Wertingen in Bavaria, St. Conrad 
 of Parzham, confessor, of the Order of Friars Minor Capuchin, outstanding 
 both for prayer and for love of neighbour. Being renowned for 
 miracles, Pope Pius XI enrolled him among the number of the saints.
\switchcolumn*
\selectlanguage{latin}
\end{paracol}


% ---- martyrology/mart04/mart0422.htm
\needspace{10\baselineskip}
\begin{paracol}{2}
\selectlanguage{latin}
\begin{center}{\color{gregoriocolor} Décimo Kaléndas Maji. 
 Luna\dots\ }\end{center}
\switchcolumn
\selectlanguage{english}
\begin{center}{\color{gregoriocolor} The 
 Twenty-Second Day of April. The\dots\ Day of the Moon.}\end{center}
\end{paracol}

\noindent\begin{tabularx}{\linewidth}{*{19}{>{\centering\arraybackslash}X}}
 \textcolor{gregoriocolor}{a} & \textcolor{gregoriocolor}{b} & \textcolor{gregoriocolor}{c} & \textcolor{gregoriocolor}{d} & \textcolor{gregoriocolor}{e} & \textcolor{gregoriocolor}{f} & \textcolor{gregoriocolor}{g} & \textcolor{gregoriocolor}{h} & \textcolor{gregoriocolor}{i} & \textcolor{gregoriocolor}{k} & \textcolor{gregoriocolor}{l} & \textcolor{gregoriocolor}{m} & \textcolor{gregoriocolor}{n} & \textcolor{gregoriocolor}{p} & \textcolor{gregoriocolor}{q} & \textcolor{gregoriocolor}{r} & \textcolor{gregoriocolor}{s} & \textcolor{gregoriocolor}{t} & \textcolor{gregoriocolor}{u} \\
 24 & 25 & 26 & 27 & 28 & 29 & 1 & 2 & 3 & 4 & 5 & 6 & 7 & 8 & 9 & 10 & 11 & 12 & 13 \\
\end{tabularx}
\vspace{0.5\baselineskip}
\noindent\begin{tabularx}{\linewidth}{*{12}{>{\centering\arraybackslash}X}}
 \textcolor{gregoriocolor}{A} & \textcolor{gregoriocolor}{B} & \textcolor{gregoriocolor}{C} & \textcolor{gregoriocolor}{D} & \textcolor{gregoriocolor}{E} & F & \textcolor{gregoriocolor}{F} & \textcolor{gregoriocolor}{G} & \textcolor{gregoriocolor}{H} & \textcolor{gregoriocolor}{M} & \textcolor{gregoriocolor}{N} & \textcolor{gregoriocolor}{P} \\
 14 & 15 & 16 & 17 & 18 & 19 & 18 & 19 & 20 & 21 & 22 & 23 \\
\end{tabularx}

\begin{paracol}{2}
\selectlanguage{latin}
\lettrine[lines=2]{R}{omæ,} via Appia, 
 natális sancti Sotéris, Papæ et Mártyris.
\switchcolumn
\selectlanguage{english}
\lettrine[lines=2]{A}{t} Rome, on the Appian Way, the 
 birthday of St. Soter, pope and martyr.
\switchcolumn*
\selectlanguage{latin}
Item Romæ sancti Caji, 
 Papæ et Mártyris; qui martyrio coronátus est sub Diocletiáno Príncipe.
\switchcolumn
\selectlanguage{english}
In the same city, Pope St. Caius, 
 who was crowned with martyrdom under Emperor Diocletian.
\switchcolumn*
\selectlanguage{latin}
Smyrnæ sanctórum 
 Apéllis et Lúcii, ex primis Christi discípulis.
\switchcolumn
\selectlanguage{english}
At Smyrna, the Saints Apelles and 
 Lucius, who were among the first disciples of Christ.
\switchcolumn*
\selectlanguage{latin}
Eódem die sanctórum 
 plurimórum Mártyrum, qui, sequénti anno post óbitum Simeónis, ánnuo item die 
 quo passiónis Domínicæ memória celebrabátur, per totam Pérsidis regiónem, 
 pro Christi nómine, sub Rege Sápore, gládio cædi jussi sunt. In quo 
 fídei certámine passus est Azades eunúchus, Regi caríssimus; Milles 
 Epíscopus, sanctitáte et miraculórum virtúte insígnis; Acépsimas Epíscopus, 
 cum Presbytero suo Jacóbo, item Aíthala et Josépho Presbyteris, Azadáne et 
 Abdiéso Diáconis, et complúribus áliis Cléricis; Maréas quoque et Bicor 
 Epíscopi, cum áliis vigínti Epíscopis, et Cléricis fere ducéntis 
 quinquagínta, Mónachis étiam et sacris Virgínibus plúrimis. Has inter 
 Vírgines fuit étiam sancti Simeónis Epíscopi soror, nómine Tárbula, cum 
 pedíssequa sua; quæ, stipítibus alligátæ serráque scissæ, crudelíssime 
 necátæ sunt.
\switchcolumn
\selectlanguage{english}
The same day, many holy martyrs 
 who, the year following the death of St. Simeon, and on the anniversary of 
 the Passion of our Lord, were put to the sword for the name of Christ 
 throughout Persia, under King Sapor. Among those who then suffered for 
 the faith were the eunuch Azades, a favorite of the king; Milles, a bishop 
 renowned for sanctity and miracles; Bishop Acepsimas with one of his priests 
 named James; also Aithalas and Joseph, priests; Azadan and Abdiesus, 
 deacons, and many other clerics; Mareas and Bicor, bishop, with twenty other 
 bishops, and nearly two hundred and fifty clerics; many monks and 
 consecrated virgins, among whom was the sister of St. Simeon, called Tarbula, 
 with her maid, who were both killed in a most cruel manner by being tied to 
 stakes and sawn asunder.
\switchcolumn*
\selectlanguage{latin}
Item in Pérside 
 sanctórum Parménii, Heliménæ et Chrysóteli Presbyterórum, Lucæ et Múcii 
 Diaconórum; quorum triúmphus martyrii in passióne sanctórum Abdon et Sennen 
 habétur.
\switchcolumn
\selectlanguage{english}
Also in Persia, Saints Parmenius, 
 Helimenas, and Chrysotelus, priests; Lucas and Mucius, deacons, whose 
 triumph is related in the Acts of Saints Abdon and Sennen.
\switchcolumn*
\selectlanguage{latin}
Alexandríæ natális 
 sancti Leónidæ Mártyris, qui sub Sevéro passus est.
\switchcolumn
\selectlanguage{english}
At Alexandria, the birthday of the 
 martyr St. Leonides, who suffered under Severus.
\switchcolumn*
\selectlanguage{latin}
Lugdúni, in Gállia, 
 sancti Epipódii, qui, in persecutióne Antoníni Veri, cum Alexándro colléga 
 tentus, ibídem, post dira torménta, martyrium abscissióne cápitis complévit.
\switchcolumn
\selectlanguage{english}
At Lyons in France, in the 
 persecution of Antoninus Verus, St. Epipodius, who was arrested with his 
 companion Alexander, and after undergoing severe torments, completed his 
 martyrdom by being beheaded.
\switchcolumn*
\selectlanguage{latin}
Constantinópoli sancti 
 Agapíti Papæ Primi, cujus sánctitas a beáto Gregório Magno commendátur. 
 Ipsíus autem corpus, póstea Romam relátum, in Vaticáno cónditum est.
\switchcolumn
\selectlanguage{english}
At Constantinople, Pope St. 
 Agapitus the First, whose sanctity was praised by St. Gregory the Great. 
 His body was afterwards taken to Rome and buried in the Vatican.
\switchcolumn*
\selectlanguage{latin}
Apud Senónas sancti 
 Leónis, Epíscopi et Confessóris.
\switchcolumn
\selectlanguage{english}
At Sens, St. Leo, bishop and 
 confessor.
\switchcolumn*
\selectlanguage{latin}
Anastasiópoli, in 
 Galátia, sancti Theodóri Epíscopi, miráculis clari.
\switchcolumn
\selectlanguage{english}
At Anastasiopolis in Galatia, St. 
 Theodore, a bishop well known for his miracles.
\switchcolumn*
\selectlanguage{latin}
\end{paracol}


% ---- martyrology/mart04/mart0423.htm
\needspace{10\baselineskip}
\begin{paracol}{2}
\selectlanguage{latin}
\begin{center}{\color{gregoriocolor} Nono Kaléndas Maji. 
 Luna\dots\ }\end{center}
\switchcolumn
\selectlanguage{english}
\begin{center}{\color{gregoriocolor} The 
 Twenty-Third Day of April. The\dots\ Day of the Moon.}\end{center}
\end{paracol}

\noindent\begin{tabularx}{\linewidth}{*{19}{>{\centering\arraybackslash}X}}
 \textcolor{gregoriocolor}{a} & \textcolor{gregoriocolor}{b} & \textcolor{gregoriocolor}{c} & \textcolor{gregoriocolor}{d} & \textcolor{gregoriocolor}{e} & \textcolor{gregoriocolor}{f} & \textcolor{gregoriocolor}{g} & \textcolor{gregoriocolor}{h} & \textcolor{gregoriocolor}{i} & \textcolor{gregoriocolor}{k} & \textcolor{gregoriocolor}{l} & \textcolor{gregoriocolor}{m} & \textcolor{gregoriocolor}{n} & \textcolor{gregoriocolor}{p} & \textcolor{gregoriocolor}{q} & \textcolor{gregoriocolor}{r} & \textcolor{gregoriocolor}{s} & \textcolor{gregoriocolor}{t} & \textcolor{gregoriocolor}{u} \\
 25 & 26 & 27 & 28 & 29 & 1 & 2 & 3 & 4 & 5 & 6 & 7 & 8 & 9 & 10 & 11 & 12 & 13 & 14 \\
\end{tabularx}
\vspace{0.5\baselineskip}
\noindent\begin{tabularx}{\linewidth}{*{12}{>{\centering\arraybackslash}X}}
 \textcolor{gregoriocolor}{A} & \textcolor{gregoriocolor}{B} & \textcolor{gregoriocolor}{C} & \textcolor{gregoriocolor}{D} & \textcolor{gregoriocolor}{E} & F & \textcolor{gregoriocolor}{F} & \textcolor{gregoriocolor}{G} & \textcolor{gregoriocolor}{H} & \textcolor{gregoriocolor}{M} & \textcolor{gregoriocolor}{N} & \textcolor{gregoriocolor}{P} \\
 15 & 16 & 17 & 18 & 19 & 20 & 19 & 20 & 21 & 22 & 23 & 24 \\
\end{tabularx}

\begin{paracol}{2}
\selectlanguage{latin}
\lettrine[lines=2]{N}{atális} sancti Geórgii 
 Mártyris, cujus illústre martyrium inter Mártyrum corónas Ecclésia Dei 
 venerátur.
\switchcolumn
\selectlanguage{english}
\lettrine[lines=2]{T}{he} birthday of St. George, whose 
 illustrious martyrdom is honoured by the Church of God among the triumphs of 
 the other martyrs.
\switchcolumn*
\selectlanguage{latin}
In vico Tenkítten, ad 
 sinum Venédicum, in Borússia, item natális sancti Adalbérti, Epíscopi 
 Pragénsis et Mártyris, qui Polónis et Húngaris Evangélium prædicávit.
\switchcolumn
\selectlanguage{english}
At Danzig in Prussia, the birthday 
 of St. Adalbert, bishop of Prague, and martyr, who preached the Gospel to 
 the Poles and the Hungarians.
\switchcolumn*
\selectlanguage{latin}
Valéntiæ, in Gállia, 
 pássio sanctórum Mártyrum Felícis Presbyteri, Fortunáti et Achíllei 
 Diaconórum. Hi, cum fuíssent a beáto Irenæo, Lugdunénsi Episcopo, 
 missi ad prædicándum verbum Dei, et máximam illíus civitátis partem ad 
 Christi fidem convertíssent, a Duce Cornélio sunt in cárcerem trusi; deínde, 
 diutíssime verberáti, cruribúsque confráctis, circa rotárum vertíginem 
 stricti, fumum quoque in equúlei suspensióne perpéssi; ad extrémum gládio 
 consummáti sunt.
\switchcolumn
\selectlanguage{english}
At Valence in France, the holy 
 martyrs Felix, a priest, Fortunatus and Achilleus, deacons, who were sent 
 there to preach the word of God by blessed Irenæus, bishop of Lyons. 
 They converted the greater portion of that city to the faith of Christ. 
 These martyrs were cast into prison by the commander Cornelius, were for a 
 long time scourged, had their legs crushed, were bound to wheels in motion, 
 and stifled with smoke while stretched on the rack, and finally died by the 
 sword.
\switchcolumn*
\selectlanguage{latin}
Medioláni sancti 
 Mároli, Epíscopi et Confessóris.
\switchcolumn
\selectlanguage{english}
At Milan, St. Marolus, bishop and 
 confessor.
\switchcolumn*
\selectlanguage{latin}
Tulli, in Gállia, 
 sancti Gerárdi, ejúsdem civitátis Epíscopi.
\switchcolumn
\selectlanguage{english}
At Toul in France, St. Gerard, 
 bishop of that city.
\switchcolumn*
\selectlanguage{latin}
\end{paracol}


% ---- martyrology/mart04/mart0424.htm
\needspace{10\baselineskip}
\begin{paracol}{2}
\selectlanguage{latin}
\begin{center}{\color{gregoriocolor} Octávo Kaléndas Maji. 
 Luna\dots\ }\end{center}
\switchcolumn
\selectlanguage{english}
\begin{center}{\color{gregoriocolor} The Twenty-Fourth Day of April. The\dots\ 
 Day of the Moon.}\end{center}
\end{paracol}

\noindent\begin{tabularx}{\linewidth}{*{19}{>{\centering\arraybackslash}X}}
 \textcolor{gregoriocolor}{a} & \textcolor{gregoriocolor}{b} & \textcolor{gregoriocolor}{c} & \textcolor{gregoriocolor}{d} & \textcolor{gregoriocolor}{e} & \textcolor{gregoriocolor}{f} & \textcolor{gregoriocolor}{g} & \textcolor{gregoriocolor}{h} & \textcolor{gregoriocolor}{i} & \textcolor{gregoriocolor}{k} & \textcolor{gregoriocolor}{l} & \textcolor{gregoriocolor}{m} & \textcolor{gregoriocolor}{n} & \textcolor{gregoriocolor}{p} & \textcolor{gregoriocolor}{q} & \textcolor{gregoriocolor}{r} & \textcolor{gregoriocolor}{s} & \textcolor{gregoriocolor}{t} & \textcolor{gregoriocolor}{u} \\
 26 & 27 & 28 & 29 & 1 & 2 & 3 & 4 & 5 & 6 & 7 & 8 & 9 & 10 & 11 & 12 & 13 & 14 & 15 \\
\end{tabularx}
\vspace{0.5\baselineskip}
\noindent\begin{tabularx}{\linewidth}{*{12}{>{\centering\arraybackslash}X}}
 \textcolor{gregoriocolor}{A} & \textcolor{gregoriocolor}{B} & \textcolor{gregoriocolor}{C} & \textcolor{gregoriocolor}{D} & \textcolor{gregoriocolor}{E} & F & \textcolor{gregoriocolor}{F} & \textcolor{gregoriocolor}{G} & \textcolor{gregoriocolor}{H} & \textcolor{gregoriocolor}{M} & \textcolor{gregoriocolor}{N} & \textcolor{gregoriocolor}{P} \\
 16 & 17 & 18 & 19 & 20 & 21 & 20 & 21 & 22 & 23 & 24 & 25 \\
\end{tabularx}

\begin{paracol}{2}
\selectlanguage{latin}
\lettrine[lines=2]{S}{evísii,} in Rhǽtia, 
 sancti Fidélis a Sigmarínga, Sacerdótis ex Ordine Minórum Capuccinórum et 
 Mártyris; qui, illuc ad prædicándam cathólicam fidem missus, ibídem, ab 
 hæréticis interémptus, martyrium consummávit; et a Benedícto Décimo quarto, 
 Pontifice Máximo, inter sanctos Mártyres relátus est.
\switchcolumn
\selectlanguage{english}
\lettrine[lines=2]{A}{t} Gruch in Switzerland, St. 
 Fidelis of Sigmaringen, priest and martyr, of the Order of Friars Minor 
 Capuchin. He was sent there to preach the Catholic faith, but was put 
 to death by the heretics. He was numbered among the holy martyrs by 
 the Sovereign Pontiff, Benedict XIV.
\switchcolumn*
\selectlanguage{latin}
Romæ sancti Sabæ, 
 ductóris mílitum, qui, accusátus quod Christiános in cárcere deténtos 
 visitáret, coram Júdice Christum líbere conféssus est. Hinc ab eódem 
 Júdice fácibus adústus et in lebétem picis fervéntis est immíssus, et, cum 
 inde evasísset illæsus, eo miráculo septuagínta viros ad Christum convértit; 
 qui omnes, constánter in confessióne fídei permanéntes, gládio cæsi sunt. 
 Postrémo et ipse, demérsus in flumen, martyrium consummávit.
\switchcolumn
\selectlanguage{english}
At Rome, St. Sabas, a military 
 officer, who bravely confessed Christ before the judge when he was accused 
 of visiting the Christians kept in prison. For this he was burned with 
 torches and thrown into a cauldron of boiling pitch, out of which he came 
 uninjured. Seventy men were converted to Christ at the sight of this 
 miracle, and as they all remained unshaken in the confession of the faith, 
 they were put to the sword. Sabas, however, completed his martyrdom by 
 being cast into the river.
\switchcolumn*
\selectlanguage{latin}
Lugdúni, in Gállia, 
 natális sancti Alexándri Mártyris, qui, in persecutióne Antoníni Veri, post cárceris custódiam, primo ita laniátus est crudelitáte verberántium, ut, 
 crate solúta costárum, patefáctis viscéribus, interióra córporis panderéntur; 
 deínde, crucis affixus patíbulo, beátum spíritum exanimátus emísit. 
 Passi sunt cum ipso et álii, número trigínta quátuor, quorum memória áliis 
 diébus ágitur.
\switchcolumn
\selectlanguage{english}
At Lyons in France, during the 
 persecution of Antoninus Verus, the birthday of St. Alexander, martyr. 
 After being imprisoned, he was so lacerated by the cruelty of those who 
 scourged him, that his ribs and the interior of his body were exposed to 
 view. Then he was fastened to the gibbet of the cross, on which he 
 yielded up his blessed soul. Thirty-four others who suffered with him 
 are commemorated on other days.
\switchcolumn*
\selectlanguage{latin}
Nicomedíæ sanctórum 
 Mártyrum Eusébii, Neónis, Leóntii, Longíni et aliórum quátuor; qui, in 
 persecutióne Diocletiáni, post diros cruciátus, gládio percússi sunt.
\switchcolumn
\selectlanguage{english}
At Nicomedia, during the 
 persecution of Diocletian, the holy martyrs Eusebius, Neon, Leontius, 
 Longinus, and four others, all of whom were slain with the sword after 
 enduring great torments.
\switchcolumn*
\selectlanguage{latin}
In Anglia deposítio 
 sancti Mellíti Epíscopi, qui, a sancto Gregório Papa in Angliam missus, 
 orientáles Saxónes et ipsórum Regem ad fidem convértit.
\switchcolumn
\selectlanguage{english}
In England, the death of St. 
 Mellitus, bishop. He was sent there by St. Gregory, and he converted 
 to the faith the East Saxons and their king.
\switchcolumn*
\selectlanguage{latin}
Illíberi, in Hispánia, sancti Gregórii, Epíscopi et Confessóris.
\switchcolumn
\selectlanguage{english}
At Elvira, in Spain, St. Gregory, 
 bishop and confessor.
\switchcolumn*
\selectlanguage{latin}
Bríxiæ sancti Honórii 
 Epíscopi.
\switchcolumn
\selectlanguage{english}
At Brescia, St. Honorius, bishop.
\switchcolumn*
\selectlanguage{latin}
In Ióna, Scótiæ 
 ínsula, 
 sancti Egbérti, Presbyteri et Mónachi, admirándæ humilitátis et continéntiæ 
 viri.
\switchcolumn
\selectlanguage{english}
In Iona, an island of Scotland, St. 
 Egbert, priest and monk, a man of admirable humility and continency.
\switchcolumn*
\selectlanguage{latin}
Rhemis, in Gállia, 
 sanctárum Vírginum Bovæ et Dodæ.
\switchcolumn
\selectlanguage{english}
At Rheims in France, the holy 
 virgins Bova and Doda.
\switchcolumn*
\selectlanguage{latin}
Andégavi, in Gállia, 
 sanctæ Maríæ a sancta Euphrásia Pelletier, Vírginis, Institúti Sorórum a 
 Bono Pastóre Fundatrícis, quam Pius Duodécimus, Póntifex Máximus, in 
 Sanctárum númerum rétulit.
\switchcolumn
\selectlanguage{english}
At Angers in France, St. Mary Euphrasia Pelletier, virgin and foundress of the Institute of the Good 
 Shepherd Sisters, whom Pius XII, Sovereign Pontiff, enrolled among the 
 number of the saints.
\switchcolumn*
\selectlanguage{latin}
Medioláni Convérsio 
 sancti Augustíni Epíscopi, Confessóris et Ecclésiæ Doctóris; quem beátum 
 Ambrósius Epíscopus veritátem fídei cathólicæ dócuit, et hac die baptizávit.
\switchcolumn
\selectlanguage{english}
At Milan, the Conversion of St. 
 Augustine, bishop, confessor, and doctor of the Church, whom the bishop St. 
 Ambrose had instructed in the truth of the Catholic faith, and baptized on 
 this day.
\switchcolumn*
\selectlanguage{latin}
\end{paracol}


% ---- martyrology/mart04/mart0425.htm
\needspace{10\baselineskip}
\begin{paracol}{2}
\selectlanguage{latin}
\begin{center}{\color{gregoriocolor} Séptimo Kaléndas Maji. 
 Luna\dots\ }\end{center}
\switchcolumn
\selectlanguage{english}
\begin{center}{\color{gregoriocolor} The 
 Twenty-Fifth Day of April. The\dots\ Day of the Moon.}\end{center}
\end{paracol}

\noindent\begin{tabularx}{\linewidth}{*{19}{>{\centering\arraybackslash}X}}
 \textcolor{gregoriocolor}{a} & \textcolor{gregoriocolor}{b} & \textcolor{gregoriocolor}{c} & \textcolor{gregoriocolor}{d} & \textcolor{gregoriocolor}{e} & \textcolor{gregoriocolor}{f} & \textcolor{gregoriocolor}{g} & \textcolor{gregoriocolor}{h} & \textcolor{gregoriocolor}{i} & \textcolor{gregoriocolor}{k} & \textcolor{gregoriocolor}{l} & \textcolor{gregoriocolor}{m} & \textcolor{gregoriocolor}{n} & \textcolor{gregoriocolor}{p} & \textcolor{gregoriocolor}{q} & \textcolor{gregoriocolor}{r} & \textcolor{gregoriocolor}{s} & \textcolor{gregoriocolor}{t} & \textcolor{gregoriocolor}{u} \\
 27 & 28 & 29 & 1 & 2 & 3 & 4 & 5 & 6 & 7 & 8 & 9 & 10 & 11 & 12 & 13 & 14 & 15 & 16 \\
\end{tabularx}
\vspace{0.5\baselineskip}
\noindent\begin{tabularx}{\linewidth}{*{12}{>{\centering\arraybackslash}X}}
 \textcolor{gregoriocolor}{A} & \textcolor{gregoriocolor}{B} & \textcolor{gregoriocolor}{C} & \textcolor{gregoriocolor}{D} & \textcolor{gregoriocolor}{E} & F & \textcolor{gregoriocolor}{F} & \textcolor{gregoriocolor}{G} & \textcolor{gregoriocolor}{H} & \textcolor{gregoriocolor}{M} & \textcolor{gregoriocolor}{N} & \textcolor{gregoriocolor}{P} \\
 17 & 18 & 19 & 20 & 21 & 22 & 21 & 22 & 23 & 24 & 25 & 26 \\
\end{tabularx}

\begin{paracol}{2}
\selectlanguage{latin}
\lettrine[lines=1]{R}{omæ} Litaníæ majóres ad sanctum Petrum.
\switchcolumn
\selectlanguage{english}
\lettrine[lines=1]{A}{t} Rome, the Greater Litanies at St. Peter's.
\switchcolumn*
\selectlanguage{latin}
Alexandríæ natális 
 beáti Marci Evangelístæ. Hic, discípulus et intérpres Apóstoli Petri, 
 rogátus Romæ a frátribus scripsit Evangélium, quo assúmpto, perréxit in 
 Ægyptum, primúsque Alexandríæ Christum annúntians, constítuit Ecclésiam; ac 
 póstea, pro fide Christi tentus, fúnibus vinctus et per saxa raptátus, 
 gráviter afflíctus est; deínde, reclúsus in cárcere, primo angélica visitatióne confortátus est, et demum, ipso Dómino sibi apparénte, ad 
 cæléstia regna vocátus, octávo Nerónis anno.
\switchcolumn
\selectlanguage{english}
At Alexandria, the birthday of St. 
 Mark the Evangelist, disciple and interpreter of the apostle St. Peter. 
 He wrote his gospel at the request of the faithful at Rome, and taking it 
 with him, proceeded to Egypt and founded a church at Alexandria, where he 
 was the first to preach Christ. Afterwards, being arrested for the 
 faith, he was bound, dragged over stones, and endured great afflictions. 
 Finally he was confined to prison, where, being comforted by the visit of an 
 angel, and even by an apparition of our Lord himself, he was called to the 
 heavenly kingdom in the eighth year of the reign of Nero.
\switchcolumn*
\selectlanguage{latin}
Item Alexandríæ sancti 
 Aniáni Epíscopi, qui, beáti Marci discípulus ejúsque in Episcopátu 
 succéssor, clarus virtútibus quiévit in Dómino.
\switchcolumn
\selectlanguage{english}
Also at Alexandria, Bishop St. 
 Anian, disciple of blessed Mark, and his successor in the episcopate. 
 With a great renown for virtue, he rested in the Lord.
\switchcolumn*
\selectlanguage{latin}
Antiochíæ sancti 
 Stéphani, Epíscopi et Mártyris, qui ab hæréticis Synodum Chalcedonénsem 
 impugnántibus, multa passus, in Oróntem flúvium præcipitátus est, témpore 
 Zenónis Imperatóris.
\switchcolumn
\selectlanguage{english}
At Antioch, St. Stephen, bishop and 
 martyr, who suffered a great deal from the heretics opposed to the Council 
 of Chalcedon, and was cast into the river Orontes, in the time of Emperor 
 Zeno.
\switchcolumn*
\selectlanguage{latin}
Syracúsis, in Sicília, 
 sanctórum Mártyrum fratrum Evódii, Hermógenis et Callístæ.
\switchcolumn
\selectlanguage{english}
At Syracuse in Sicily, the holy 
 martyrs Evodius, Hermogenes, and Callista.
\switchcolumn*
\selectlanguage{latin}
Láubiis, in Bélgio, 
 natális sancti Ermíni, Epíscopi et Confessóris.
\switchcolumn
\selectlanguage{english}
At Lobbes in Belgium, the birthday 
 of St. Ermin, bishop and confessor.
\switchcolumn*
\selectlanguage{latin}
Antiochíæ sanctórum 
 Philónis et Agathópodis Diaconórum, de quibus beátus Ignátius, Epíscopus et 
 Martyr, laudábilem in suis epístolis mentiónem facit.
\switchcolumn
\selectlanguage{english}
At Antioch, the deacons Saints 
 Philo and Agathopodes, who were praised in the letters of blessed Ignatius, 
 bishop and martyr.
\switchcolumn*
\selectlanguage{latin}
\end{paracol}


% ---- martyrology/mart04/mart0426.htm
\needspace{10\baselineskip}
\begin{paracol}{2}
\selectlanguage{latin}
\begin{center}{\color{gregoriocolor} Sexto Kaléndas Maji. 
 Luna\dots\ }\end{center}
\switchcolumn
\selectlanguage{english}
\begin{center}{\color{gregoriocolor} The 
 Twenty-Sixth Day of April. The\dots\ Day of the Moon.}\end{center}
\end{paracol}

\noindent\begin{tabularx}{\linewidth}{*{19}{>{\centering\arraybackslash}X}}
 \textcolor{gregoriocolor}{a} & \textcolor{gregoriocolor}{b} & \textcolor{gregoriocolor}{c} & \textcolor{gregoriocolor}{d} & \textcolor{gregoriocolor}{e} & \textcolor{gregoriocolor}{f} & \textcolor{gregoriocolor}{g} & \textcolor{gregoriocolor}{h} & \textcolor{gregoriocolor}{i} & \textcolor{gregoriocolor}{k} & \textcolor{gregoriocolor}{l} & \textcolor{gregoriocolor}{m} & \textcolor{gregoriocolor}{n} & \textcolor{gregoriocolor}{p} & \textcolor{gregoriocolor}{q} & \textcolor{gregoriocolor}{r} & \textcolor{gregoriocolor}{s} & \textcolor{gregoriocolor}{t} & \textcolor{gregoriocolor}{u} \\
 28 & 29 & 1 & 2 & 3 & 4 & 5 & 6 & 7 & 8 & 9 & 10 & 11 & 12 & 13 & 14 & 15 & 16 & 17 \\
\end{tabularx}
\vspace{0.5\baselineskip}
\noindent\begin{tabularx}{\linewidth}{*{12}{>{\centering\arraybackslash}X}}
 \textcolor{gregoriocolor}{A} & \textcolor{gregoriocolor}{B} & \textcolor{gregoriocolor}{C} & \textcolor{gregoriocolor}{D} & \textcolor{gregoriocolor}{E} & F & \textcolor{gregoriocolor}{F} & \textcolor{gregoriocolor}{G} & \textcolor{gregoriocolor}{H} & \textcolor{gregoriocolor}{M} & \textcolor{gregoriocolor}{N} & \textcolor{gregoriocolor}{P} \\
 18 & 19 & 20 & 21 & 22 & 23 & 22 & 23 & 24 & 25 & 26 & 27 \\
\end{tabularx}

\begin{paracol}{2}
\selectlanguage{latin}
\lettrine[lines=2]{R}{omæ} natális beáti 
 Cleti, Papæ et Mártyris; qui, secúndus post Apóstolum Petrum, rexit 
 Ecclésiam, et martyrio in persecutióne Domitiáni coronátus est.
\switchcolumn
\selectlanguage{english}
\lettrine[lines=2]{A}{t} Rome, the birthday of St. 
 Cletus, the pope who governed the Church the second after the apostle St. 
 Peter, and was crowned with martyrdom in the persecution of Domitian.
\switchcolumn*
\selectlanguage{latin}
Sancti Marcellíni, 
 Papæ et Mártyris, cujus dies natális octávo Kaléndas Novémbris recensétur.
\switchcolumn
\selectlanguage{english}
St. Marcellinus, pope and martyr, 
 whose birthday is commemorated on the 25th of October.
\switchcolumn*
\selectlanguage{latin}
Amaséæ, in Ponto, 
 sancti Basiléi, Epíscopi et Mártyris, qui, sub Licínio Imperatóre, illústre 
 martyrium consummávit. Ipsíus autem corpus, in mare projéctum, et ab 
 Elpidíphoro, Angeli mónitu repértum, honorífice tumulátum fuit.
\switchcolumn
\selectlanguage{english}
At Amasea in Pontus, St. Basileus, 
 bishop and martyr, whose illustrious martyrdom occurred under Emperor 
 Licinius. His body was thrown into the sea, but was found by 
 Elpidiphorus, through the revelation of an angel, and was honorably buried.
\switchcolumn*
\selectlanguage{latin}
Brácari, in Lusitánia, 
 sancti Petri Mártyris, qui fuit primus ejúsdem civitátis Epíscopus.
\switchcolumn
\selectlanguage{english}
At Braga in Portugal, St. Peter, 
 martyr, the first bishop of that city.
\switchcolumn*
\selectlanguage{latin}
Viénnæ, in Gállia, 
 sancti Claréntii, Epíscopi et Confessóris.
\switchcolumn
\selectlanguage{english}
At Vienne in France, St. Clarence, 
 bishop and confessor.
\switchcolumn*
\selectlanguage{latin}
Verónæ sancti Lucídii 
 Epíscopi.
\switchcolumn
\selectlanguage{english}
At Verona, St. Lucidius, bishop.
\switchcolumn*
\selectlanguage{latin}
In monastério Céntula, 
 in Gállia, sancti Richárii, Presbyteri et Confessóris.
\switchcolumn
\selectlanguage{english}
In the monastery of Centula in 
 France, St. Richarius, priest and confessor.
\switchcolumn*
\selectlanguage{latin}
Trecis, in Gállia, 
 sanctæ Exsuperántiæ Vírginis.
\switchcolumn
\selectlanguage{english}
At Troyes in France, St. 
 Exuperantia, virgin.
\switchcolumn*
\selectlanguage{latin}
\end{paracol}


% ---- martyrology/mart04/mart0427.htm
\needspace{10\baselineskip}
\begin{paracol}{2}
\selectlanguage{latin}
\begin{center}{\color{gregoriocolor} Quinto Kaléndas Maji. 
 Luna\dots\ }\end{center}
\switchcolumn
\selectlanguage{english}
\begin{center}{\color{gregoriocolor} The 
 Twenty-Seventh Day of April. The\dots\ Day of the Moon.}\end{center}
\end{paracol}

\noindent\begin{tabularx}{\linewidth}{*{19}{>{\centering\arraybackslash}X}}
 \textcolor{gregoriocolor}{a} & \textcolor{gregoriocolor}{b} & \textcolor{gregoriocolor}{c} & \textcolor{gregoriocolor}{d} & \textcolor{gregoriocolor}{e} & \textcolor{gregoriocolor}{f} & \textcolor{gregoriocolor}{g} & \textcolor{gregoriocolor}{h} & \textcolor{gregoriocolor}{i} & \textcolor{gregoriocolor}{k} & \textcolor{gregoriocolor}{l} & \textcolor{gregoriocolor}{m} & \textcolor{gregoriocolor}{n} & \textcolor{gregoriocolor}{p} & \textcolor{gregoriocolor}{q} & \textcolor{gregoriocolor}{r} & \textcolor{gregoriocolor}{s} & \textcolor{gregoriocolor}{t} & \textcolor{gregoriocolor}{u} \\
 29 & 1 & 2 & 3 & 4 & 5 & 6 & 7 & 8 & 9 & 10 & 11 & 12 & 13 & 14 & 15 & 16 & 17 & 18 \\
\end{tabularx}
\vspace{0.5\baselineskip}
\noindent\begin{tabularx}{\linewidth}{*{12}{>{\centering\arraybackslash}X}}
 \textcolor{gregoriocolor}{A} & \textcolor{gregoriocolor}{B} & \textcolor{gregoriocolor}{C} & \textcolor{gregoriocolor}{D} & \textcolor{gregoriocolor}{E} & F & \textcolor{gregoriocolor}{F} & \textcolor{gregoriocolor}{G} & \textcolor{gregoriocolor}{H} & \textcolor{gregoriocolor}{M} & \textcolor{gregoriocolor}{N} & \textcolor{gregoriocolor}{P} \\
 19 & 20 & 21 & 22 & 23 & 24 & 23 & 24 & 25 & 26 & 27 & 28 \\
\end{tabularx}

\begin{paracol}{2}
\selectlanguage{latin}
\lettrine[lines=2]{S}{ancti} Petri Canísii, 
 Sacerdótis e Societáte Jesu et Confessóris atque Ecclésiæ Doctóris; qui 
 duodécimo Kaléndas Januárii migrávit ad Dóminum.
\switchcolumn
\selectlanguage{english}
\lettrine[lines=2]{S}{t.} Peter Canisius, priest of the 
 Society of Jesus, confessor and doctor of the Church, who departed to the 
 Lord on the 21st of December.
\switchcolumn*
\selectlanguage{latin}
Nicomedíæ natális 
 sancti Anthimi, Epíscopi et Mártyris; qui in persecutióne Diocletiáni, ob 
 confessiónem Christi, martyrii glóriam obtruncatióne cápitis accépit. 
 Secúta est quoque illum univérsa ferme gregis sui multitúdo; quorum álios 
 Judex gládio obtruncári, álios conflagrári ígnibus, álios, navículis 
 impósitos, pélago immérgi fecit.
\switchcolumn
\selectlanguage{english}
At Nicomedia, during the 
 persecution of Diocletian, the birthday of St. Anthimus, bishop and martyr, 
 who obtained the glory of martyrdom by being beheaded for the faith. 
 Nearly all his numerous flock followed him. The judge ordered some to 
 be beheaded, others to be burned alive, others to be put in boats and sunk 
 in the sea.
\switchcolumn*
\selectlanguage{latin}
Tarsi, in Cilícia, 
 sanctórum Cástoris et Stéphani Mártyrum.
\switchcolumn
\selectlanguage{english}
At Tarsus in Cilicia, the Saints 
 Castor and Stephen, martyrs.
\switchcolumn*
\selectlanguage{latin}
Bonóniæ sancti 
 Tertulliáni, Epíscopi et Confessóris.
\switchcolumn
\selectlanguage{english}
At Bologna, St. Tertullian, bishop 
 and confessor.
\switchcolumn*
\selectlanguage{latin}
Bríxiæ sancti 
 Theóphili Epíscopi.
\switchcolumn
\selectlanguage{english}
At Brescia, St. Theophilus, bishop.
\switchcolumn*
\selectlanguage{latin}
In Ægypto sancti 
 Theodóri Abbátis, qui fuit discípulus sancti Pachómii.
\switchcolumn
\selectlanguage{english}
In Egypt, St. Theodore, abbot, who 
 was a disciple of St. Pachomius.
\switchcolumn*
\selectlanguage{latin}
Constantinópoli sancti 
 Joánnis Abbátis, qui pro cultu sacrárum Imáginum, sub Leóne Isáurico, 
 plúrimum decertávit.
\switchcolumn
\selectlanguage{english}
At Constantinople, the abbot St. 
 John, who valiantly defended the veneration of sacred images, under Leo the 
 Isaurian.
\switchcolumn*
\selectlanguage{latin}
Tarracóne, in 
 Hispánia, beáti Petri Armengáudii, ex Ordine beátæ Maríæ de Mercéde 
 redemptiónis captivórum; qui, multa pro fidélibus rediméndis in Africa 
 passus, tandem, in convéntu sanctæ Maríæ Pratórum, beáto fine quiévit.
\switchcolumn
\selectlanguage{english}
At Tarragona in Spain, the blessed 
 Peter Armengaudius, of the Order of Blessed Mary of Mercy for the Redemption 
 of Captives. He endured many tribulations in Africa in ransoming the 
 faithful, and finally closed his career peacefully in the convent of St. 
 Mary of the Meadows.
\switchcolumn*
\selectlanguage{latin}
Lucæ, in Túscia, beátæ 
 Zitæ Vírginis, virtútum et miraculórum fama conspícuæ.
\switchcolumn
\selectlanguage{english}
At Lucca in Tuscany, blessed Zita, a 
 virgin renowned for virtues and miracles.
\switchcolumn*
\selectlanguage{latin}
\end{paracol}


% ---- martyrology/mart04/mart0428.htm
\needspace{10\baselineskip}
\begin{paracol}{2}
\selectlanguage{latin}
\begin{center}{\color{gregoriocolor} Quarto Kaléndas Maji. 
 Luna\dots\ }\end{center}
\switchcolumn
\selectlanguage{english}
\begin{center}{\color{gregoriocolor} The 
 Twenty-Eighth Day of April. The\dots\ Day of the Moon.}\end{center}
\end{paracol}

\noindent\begin{tabularx}{\linewidth}{*{19}{>{\centering\arraybackslash}X}}
 \textcolor{gregoriocolor}{a} & \textcolor{gregoriocolor}{b} & \textcolor{gregoriocolor}{c} & \textcolor{gregoriocolor}{d} & \textcolor{gregoriocolor}{e} & \textcolor{gregoriocolor}{f} & \textcolor{gregoriocolor}{g} & \textcolor{gregoriocolor}{h} & \textcolor{gregoriocolor}{i} & \textcolor{gregoriocolor}{k} & \textcolor{gregoriocolor}{l} & \textcolor{gregoriocolor}{m} & \textcolor{gregoriocolor}{n} & \textcolor{gregoriocolor}{p} & \textcolor{gregoriocolor}{q} & \textcolor{gregoriocolor}{r} & \textcolor{gregoriocolor}{s} & \textcolor{gregoriocolor}{t} & \textcolor{gregoriocolor}{u} \\
 1 & 2 & 3 & 4 & 5 & 6 & 7 & 8 & 9 & 10 & 11 & 12 & 13 & 14 & 15 & 16 & 17 & 18 & 19 \\
\end{tabularx}
\vspace{0.5\baselineskip}
\noindent\begin{tabularx}{\linewidth}{*{12}{>{\centering\arraybackslash}X}}
 \textcolor{gregoriocolor}{A} & \textcolor{gregoriocolor}{B} & \textcolor{gregoriocolor}{C} & \textcolor{gregoriocolor}{D} & \textcolor{gregoriocolor}{E} & F & \textcolor{gregoriocolor}{F} & \textcolor{gregoriocolor}{G} & \textcolor{gregoriocolor}{H} & \textcolor{gregoriocolor}{M} & \textcolor{gregoriocolor}{N} & \textcolor{gregoriocolor}{P} \\
 20 & 21 & 22 & 23 & 24 & 25 & 24 & 25 & 26 & 27 & 28 & 29 \\
\end{tabularx}

\begin{paracol}{2}
\selectlanguage{latin}
\lettrine[lines=2]{S}{ancti} Pauli a Cruce, 
 Presbyteri et Confessóris; qui Congregatiónis a Cruce et Passióne Dómini 
 nostri Jesu Christi nuncupátæ Institútor fuit, atque in Dómino obdormívit 
 quintodécimo Kaléndas Novémbris.
\switchcolumn
\selectlanguage{english}
\lettrine[lines=2]{S}{t.} Paul of the Cross, priest and 
 confessor, founder of the Congregation of the Cross and Passion of our Lord 
 Jesus Christ. He went to his repose in the Lord on the 18th of 
 October.
\switchcolumn*
\selectlanguage{latin}
Ravénnæ natális sancti 
 Vitális Mártyris, viri sanctæ Valériæ ac patris sanctórum Gervásii et 
 Protásii; qui, cum beáti Ursicíni corpus sublátum débita honestáte 
 sepelísset, tentus est a Paulíno Consulári, et, post equúlei torménta, 
 jussus depóni in profúndam fóveam, et terra ac lapídibus óbrui; talíque 
 martyrio migrávit ad Christum.
\switchcolumn
\selectlanguage{english}
At Ravenna, the birthday of St. 
 Vitalis, martyr, father of the Saints Gervase and Protase. When he had 
 taken up and reverently buried the body of blessed Ursicinus, he was 
 arrested by the governor Paulinus, and after being racked and thrown into a 
 deep pit, was covered with earth and stones, and by this kind of martyrdom went 
 to Christ.
\switchcolumn*
\selectlanguage{latin}
Atínæ, in Campánia, 
 sancti Marci, qui, a beáto Petro Apóstolo Epíscopus ordinátus, Æquícolis 
 primus Evangélium prædicávit, et in persecutióne Domitiáni, sub Máximo 
 Præside, martyrii corónam accépit.
\switchcolumn
\selectlanguage{english}
At Atino in Campania, St. Mark, who 
 was made bishop by the blessed apostle Peter. He was the first to 
 preach the Gospel to the Equicoli, and received the crown of martyrdom in 
 the persecution of Domitian, under the governor Maximus.
\switchcolumn*
\selectlanguage{latin}
Prusæ, in Bithynia, 
 sanctórum Mártyrum Patrícii Epíscopi, Acátii, Menándri et Polyǽni.
\switchcolumn
\selectlanguage{english}
At Broussa in Bithynia, the holy 
 martyrs Patrick, a bishop, Acatius, Menander, and Polyaenus.
\switchcolumn*
\selectlanguage{latin}
Eódem die sanctórum 
 Mártyrum Aphrodísii, Caralíppi, Agápii et Eusébii.
\switchcolumn
\selectlanguage{english}
On the same day, the holy martyrs 
 Aphrodisius, Caralippus, Agapius, and Eusebius.
\switchcolumn*
\selectlanguage{latin}
In Pannónia sancti 
 Polliónis Mártyris, sub Diocletiáno Imperatóre.
\switchcolumn
\selectlanguage{english}
In Hungary, St. Pollio, martyr, 
 under the Emperor Diocletian.
\switchcolumn*
\selectlanguage{latin}
Medioláni sanctæ 
 Valériæ Mártyris, uxóris sancti Vitális ac matris sanctórum Gervásii et 
 Protásii.
\switchcolumn
\selectlanguage{english}
At Milan, the martyr St. Valeria, 
 who was the wife of St. Vitalis and the mother of Saints Gervase and Protase.
\switchcolumn*
\selectlanguage{latin}
Alexandríæ pássio 
 sanctæ Theodóræ, Vírginis et Mártyris. Hæc, idólis sacrificáre 
 contémnens, in lupánar est trádita, sed repénte quidam ex frátribus, nómine 
 Dídymus, miro Dei favóre, commutátis véstibus, illam erípuit; qui póstea, in 
 persecutióne Diocletiáni, sub Eustrátio Prǽside, simul cum eádem Vírgine 
 percússus, simul coronátus est.
\switchcolumn
\selectlanguage{english}
At Alexandria, the martyrdom of the 
 virgin St. Theodora. For refusing to sacrifice to idols, she was sent 
 to a place of debauchery; but one of the brethren, named Didymus, through 
 the admirable providence of God, delivered her by quickly exchanging 
 garments with her. He was afterwards beheaded and crowned with her in 
 the persecution of Diocletian, under the governor Eustratius.
\switchcolumn*
\selectlanguage{latin}
Turiasóne, in Hispánia 
 Tarraconénsi, sancti Prudéntii, Epíscopi et Confessóris.
\switchcolumn
\selectlanguage{english}
At Tarrazona in Spain, St. 
 Prudentius, bishop and confessor.
\switchcolumn*
\selectlanguage{latin}
Corfínii, in Pelígnis, 
 sancti Pámphili, Valvénsis Epíscopi, caritáte in páuperes et virtúte 
 miraculórum illústris; cujus corpus Sulmóne cónditum est.
\switchcolumn
\selectlanguage{english}
At Corfinio in Peligno, St. 
 Pamphilus, bishop of Valva, illustrious for his charity towards the poor and 
 the gift of miracles. His body was buried at Solmona.
\switchcolumn*
\selectlanguage{latin}
In pago sancti 
 Lauréntii ad Séparim, diœcésis Lucionénsis, sancti Ludovíci Maríæ Grignion a 
 Montfort, Confessóris, Fundatóris Missionariórum Societátis Maríæ et 
 Filiárum a Sapiéntia, apostólicæ vitæ forma, prædicatióne et devotióne 
 mariáli insígnis, quem Pius Papa Duodécimus Sanctórum catálogo adscrípsit.
\switchcolumn
\selectlanguage{english}
At St. Laurent sur Sèvres, in the 
 diocese of Luçon, St. Louis-Marie Grignion de Montfort, confessor and 
 founder of the Missionaries of the Company of Mary and the Sisters of 
 Wisdom, a form of apostolic life. He was renowned for his preaching 
 and devotion to the Blessed Mother, and was added to the number of the 
 saints by Pope Pius XII.
\switchcolumn*
\selectlanguage{latin}
\end{paracol}


% ---- martyrology/mart04/mart0429.htm
\needspace{10\baselineskip}
\begin{paracol}{2}
\selectlanguage{latin}
\begin{center}{\color{gregoriocolor} Tértio Kaléndas Maji. 
 Luna\dots\ }\end{center}
\switchcolumn
\selectlanguage{english}
\begin{center}{\color{gregoriocolor} The 
 Twenty-Ninth Day of April. The\dots\ Day of the Moon.}\end{center}
\end{paracol}

\noindent\begin{tabularx}{\linewidth}{*{19}{>{\centering\arraybackslash}X}}
 \textcolor{gregoriocolor}{a} & \textcolor{gregoriocolor}{b} & \textcolor{gregoriocolor}{c} & \textcolor{gregoriocolor}{d} & \textcolor{gregoriocolor}{e} & \textcolor{gregoriocolor}{f} & \textcolor{gregoriocolor}{g} & \textcolor{gregoriocolor}{h} & \textcolor{gregoriocolor}{i} & \textcolor{gregoriocolor}{k} & \textcolor{gregoriocolor}{l} & \textcolor{gregoriocolor}{m} & \textcolor{gregoriocolor}{n} & \textcolor{gregoriocolor}{p} & \textcolor{gregoriocolor}{q} & \textcolor{gregoriocolor}{r} & \textcolor{gregoriocolor}{s} & \textcolor{gregoriocolor}{t} & \textcolor{gregoriocolor}{u} \\
 2 & 3 & 4 & 5 & 6 & 7 & 8 & 9 & 10 & 11 & 12 & 13 & 14 & 15 & 16 & 17 & 18 & 19 & 20 \\
\end{tabularx}
\vspace{0.5\baselineskip}
\noindent\begin{tabularx}{\linewidth}{*{12}{>{\centering\arraybackslash}X}}
 \textcolor{gregoriocolor}{A} & \textcolor{gregoriocolor}{B} & \textcolor{gregoriocolor}{C} & \textcolor{gregoriocolor}{D} & \textcolor{gregoriocolor}{E} & F & \textcolor{gregoriocolor}{F} & \textcolor{gregoriocolor}{G} & \textcolor{gregoriocolor}{H} & \textcolor{gregoriocolor}{M} & \textcolor{gregoriocolor}{N} & \textcolor{gregoriocolor}{P} \\
 21 & 22 & 23 & 24 & 25 & 26 & 25 & 26 & 27 & 28 & 29 & 1 \\
\end{tabularx}

\begin{paracol}{2}
\selectlanguage{latin}
\lettrine[lines=2]{S}{ancti} Petri, ex 
 Ordine Prædicatórum, Mártyris, qui octávo Idus Aprílis pro fide cathólica 
 martyrium súbiit.
\switchcolumn
\selectlanguage{english}
\lettrine[lines=2]{S}{t.} Peter, a martyr of the Order of 
 Preachers, who was slain for the Catholic faith on the 6th day of April.
\switchcolumn*
\selectlanguage{latin}
Romæ natális sanctæ 
 Catharínæ Senénsis Vírginis, ex tértio Ordine sancti Domínici, vita et 
 miráculis claræ, quam Pius Secúndus, Póntifex Máximus, sanctárum Vírginum 
 número adscrípsit. Ipsíus tamen festum sequénti die celebrátur.
\switchcolumn
\selectlanguage{english}
At Rome, the birthday of St. 
 Catherine of Siena, virgin of the Third Order of St. Dominic, renowned for 
 her holy life and her miracles. She was inscribed among the canonized 
 virgins by Pope Pius II. Her feast, however, is celebrated on the 
 following day.
\switchcolumn*
\selectlanguage{latin}
Apud Paphum, in Cypro, 
 sancti Tychici, qui fuit discípulus beáti Pauli Apóstoli, et ab ipso 
 Apóstolo in suis Epístolis caríssimus frater, miníster fidélis suúsque in 
 Dómino consérvus appellátur.
\switchcolumn
\selectlanguage{english}
At Paphos in Cyprus, St. Tychicus, 
 a disciple of the blessed Apostle Paul, who called him in his Epistles, 
 ``most dear brother,'' ``faithful minister,'' and ``fellow-servant in the Lord'`.
\switchcolumn*
\selectlanguage{latin}
Pisis, in Túscia, 
 sancti Torpétis Mártyris, qui magnus in offício Nerónis primum fuit, unusque 
 ex his, de quibus idem Paulus Apóstolus ab urbe Roma ad Philippénses scribit: 
 « Salútant vos omnes sancti, máxime autem qui de Cǽsaris domo sunt ». 
 Sed póstea, pro fide Christi, jubénte Satéllico, 
 álapis cæditur, verbéribus 
 duríssime affícitur, ac béstiis devorándus tráditur, sed mínime læditur; 
 tandem martyrium suum decollatióne complévit.
\switchcolumn
\selectlanguage{english}
At Pisa in Tuscany, the martyr St. 
 Torpes, who filled a high office in the court of Nero, and was one of those 
 of whom the apostle wrote from Rome to the Philippians: ``All the saints 
 salute you, especially those that are of the house of Caesar.'' For the 
 faith of Christ, he was, by order of Satellicus, beaten, cruelly scourged, 
 and delivered to the beasts to be devoured, but remained uninjured. He 
 completed his martyrdom by being beheaded.
\switchcolumn*
\selectlanguage{latin}
Cirthæ, in Numídia, 
 natális sanctórum Mártyrum Agápii et Secundíni Episcopórum, qui, post longum 
 apud præfátam urbem exsílium, in persecutióne Valeriáni, in qua tunc máxime 
 Gentílium rábies ad tentándam justórum fidem inhiábat, ex illústri 
 sacerdótio effécti sunt Mártyres gloriósi. Passi sunt in eódem 
 collégio Æmiliánus miles, Tertúlla et Antónia, quæ erant sacræ 
 Vírgines, et quædam múlier cum suis géminis.
\switchcolumn
\selectlanguage{english}
At Cirta in Numidia, the birthday 
 of the holy martyrs Apapius and Secundinus, bishops, who, after a long exile 
 in that city, added to the glory of their priesthood the crown of martyrdom. 
 They suffered in the persecution of Valerian, during which the enraged 
 Gentiles made every effort to shake the faith of the just. In their 
 company suffered Aemilian, a soldier, Tertulla and Antonia, consecrated 
 virgins, and a woman with her twin children.
\switchcolumn*
\selectlanguage{latin}
In ínsula Corcyra 
 sanctórum septem Latrónum, qui, a sancto Jásone ad Christum convérsi, 
 martyrio vitam adépti sunt sempitérnam.
\switchcolumn
\selectlanguage{english}
In the island of Codyra, the seven 
 holy thieves who were converted to Christ by St. Jason, and gained eternal 
 life by martyrdom.
\switchcolumn*
\selectlanguage{latin}
Neápoli, in Campánia, 
 sancti Sevéri Epíscopi, qui, inter ália admiránda, mórtuum de sepúlcro 
 excitávit ad tempus, ut mendácem víduæ et pupíllórum creditórum argúeret 
 falsitátis.
\switchcolumn
\selectlanguage{english}
At Naples in Campania, Bishop St. 
 Severus, who, among other prodigies, raised for a short time a dead man from 
 the grave in order to convict of falsehood the lying creditor of a widow and 
 her children.
\switchcolumn*
\selectlanguage{latin}
Bríxiæ sancti Paulíni, 
 Epíscopi et Confessóris.
\switchcolumn
\selectlanguage{english}
At Brescia, St. Paulinus, bishop 
 and confessor.
\switchcolumn*
\selectlanguage{latin}
In cœnóbio Cluniacénsi, 
 in Gállia, sancti Hugónis Abbátis.
\switchcolumn
\selectlanguage{english}
In the monastery of Cluny in 
 France, St. Hugh Abbot.
\switchcolumn*
\selectlanguage{latin}
In monastério 
 Molisménsi, in Gállia, sancti Robérti, qui fuit primus Abbas Cistércii.
\switchcolumn
\selectlanguage{english}
In the monastery of Molesmes in 
 France, St. Robert, the first abbot of the Cistercians.
\switchcolumn*
\selectlanguage{latin}
\end{paracol}


% ---- martyrology/mart04/mart0430.htm
\needspace{10\baselineskip}
\begin{paracol}{2}
\selectlanguage{latin}
\begin{center}{\color{gregoriocolor} Prídie Kaléndas Maji. 
 Luna\dots\ }\end{center}
\switchcolumn
\selectlanguage{english}
\begin{center}{\color{gregoriocolor} The 
 Thirtieth Day of April. The\dots\ Day of the Moon.}\end{center}
\end{paracol}

\noindent\begin{tabularx}{\linewidth}{*{19}{>{\centering\arraybackslash}X}}
 \textcolor{gregoriocolor}{a} & \textcolor{gregoriocolor}{b} & \textcolor{gregoriocolor}{c} & \textcolor{gregoriocolor}{d} & \textcolor{gregoriocolor}{e} & \textcolor{gregoriocolor}{f} & \textcolor{gregoriocolor}{g} & \textcolor{gregoriocolor}{h} & \textcolor{gregoriocolor}{i} & \textcolor{gregoriocolor}{k} & \textcolor{gregoriocolor}{l} & \textcolor{gregoriocolor}{m} & \textcolor{gregoriocolor}{n} & \textcolor{gregoriocolor}{p} & \textcolor{gregoriocolor}{q} & \textcolor{gregoriocolor}{r} & \textcolor{gregoriocolor}{s} & \textcolor{gregoriocolor}{t} & \textcolor{gregoriocolor}{u} \\
 3 & 4 & 5 & 6 & 7 & 8 & 9 & 10 & 11 & 12 & 13 & 14 & 15 & 16 & 17 & 18 & 19 & 20 & 21 \\
\end{tabularx}
\vspace{0.5\baselineskip}
\noindent\begin{tabularx}{\linewidth}{*{12}{>{\centering\arraybackslash}X}}
 \textcolor{gregoriocolor}{A} & \textcolor{gregoriocolor}{B} & \textcolor{gregoriocolor}{C} & \textcolor{gregoriocolor}{D} & \textcolor{gregoriocolor}{E} & F & \textcolor{gregoriocolor}{F} & \textcolor{gregoriocolor}{G} & \textcolor{gregoriocolor}{H} & \textcolor{gregoriocolor}{M} & \textcolor{gregoriocolor}{N} & \textcolor{gregoriocolor}{P} \\
 22 & 23 & 24 & 25 & 26 & 27 & 26 & 27 & 28 & 29 & 1 & 2 \\
\end{tabularx}

\begin{paracol}{2}
\selectlanguage{latin}
\lettrine[lines=2]{S}{anctæ} Catharínæ 
 Senénsis Vírginis, ex tértio Ordine sancti Domínici, quæ ad cæléstem Sponsum 
 transívit prídie hujus diéi.
\switchcolumn
\selectlanguage{english}
\lettrine[lines=2]{S}{t.} Catherine of Siena, virgin of 
 the Third Order of St. Dominic, who on the previous day went to her heavenly 
 Spouse.
\switchcolumn*
\selectlanguage{latin}
Apud Sántonas, in 
 Gállia, beáti Eutrópii, Epíscopi et Mártyris, qui, a sancto Cleménte Papa, 
 órdinis pontificális grátia consecrátus, in Gálliam diréctus est, ibíque, 
 perácta diu prædicatióne, tandem, ob Christi testimónium, collíso cápite, 
 victor occúbuit.
\switchcolumn
\selectlanguage{english}
At Saintes in France, blessed 
 Eutropius, bishop and martyr, who was consecrated bishop and sent to France 
 by St. Clement. After preaching for many years, he had his skull 
 crushed for bearing testimony to Christ, and thus gained victory by his 
 death.
\switchcolumn*
\selectlanguage{latin}
Córdubæ, in Hispánia, sanctórum Mártyrum Amatóris Presbyteri, Petri Mónachi, et Ludovíci.
\switchcolumn
\selectlanguage{english}
At Cordova in Spain, the holy 
 martyrs Amator, a priest, Peter, a monk, and Louis.
\switchcolumn*
\selectlanguage{latin}
Nováriæ sancti 
 Lauréntii Presbyteri, et puerórum Mártyrum, quos ille suscéperat educándos.
\switchcolumn
\selectlanguage{english}
At Novara, the martyrdom of the 
 holy priest Laurence, and some boys whom he was teaching.
\switchcolumn*
\selectlanguage{latin}
Alexandríæ sanctórum 
 Mártyrum Aphrodísii Presbyteri, et aliórum trigínta.
\switchcolumn
\selectlanguage{english}
At Alexandria, the holy martyrs 
 Aphrodisius, a priest, and thirty martyrs.
\switchcolumn*
\selectlanguage{latin}
Lambésæ, in Numídia, 
 natális sanctórum Mártyrum Mariáni Lectóris, et Jacóbi Diáconi. Horum 
 prior, cum infestatiónes jamprídem Deciánæ persecutiónis in confessióne 
 Christi vicísset, íterum cum claríssimo colléga tentus est; et ambo, post 
 dira et exquisíta supplícia, mirabíliter divínis revelatiónibus secúndo 
 confortáti, novíssime, cum multis áliis, gládio consummáti sunt.
\switchcolumn
\selectlanguage{english}
At Lambesa in Numidia, the birthday 
 of the holy martyrs Marian, a lector, and James, a deacon. The former, 
 after having successfully endured many trials for the confession of Christ 
 in the persecution of Decius, was again arrested with his noble companions, 
 and both were subjected to severe and cruel torments, during which they were 
 twice miraculously comforted by heaven, but finally fell by the sword along 
 with many others.
\switchcolumn*
\selectlanguage{latin}
Ephesi sancti Máximi 
 Mártyris, qui in persecutióne Décii coronátus est.
\switchcolumn
\selectlanguage{english}
At Ephesus, the martyr St. Maximus, 
 who received his crown during the persecution of Decius.
\switchcolumn*
\selectlanguage{latin}
Firmi, in Picéno, 
 sanctæ Sophíæ, Vírginis et Mártyris.
\switchcolumn
\selectlanguage{english}
At Ferno in Piceno, St. Sophia, 
 virgin and martyr.
\switchcolumn*
\selectlanguage{latin}
Evóreæ, in Epíro, 
 sancti Donáti Epíscopi, qui, témpore Theodósii Imperatóris, exímia 
 sanctitáte refúlsit.
\switchcolumn
\selectlanguage{english}
At Evorea in Epirus, St. Donatus, a 
 bishop, who was eminent for sanctity in the time of Emperor Theodosius.
\switchcolumn*
\selectlanguage{latin}
Neápoli, in Campánia, sancti Pompónii Epíscopi.
\switchcolumn
\selectlanguage{english}
At Naples in Campania, St. 
 Pomponius, bishop.
\switchcolumn*
\selectlanguage{latin}
Londíni, in Anglia, 
 sancti Erconváldi Epíscopi, qui multis miráculis cláruit.
\switchcolumn
\selectlanguage{english}
At London in England, St. Erkenwald, 
 a bishop celebrated for many miracles.
\switchcolumn*
\selectlanguage{latin}
Chérii, apud Augústam 
 Taurinórum, sancti Joséphi Benedícti Cottoléngo, Confessóris, Parvæ Domus a 
 Divína Providéntia Fundatóris, summa in Deum confidéntia et caritáte in 
 páuperes insígnis, quem Pius Papa Undécimus Sanctórum fastis adscrípsit.
\switchcolumn
\selectlanguage{english}
At Chieri, near Turin, St. Joseph 
 Cottolengo, confessor, founder of the Little House of Divine Providence, 
 full of trust in God and remarkable for his charity toward the poor, whom 
 Pope Pius XI enrolled among the saints.
\switchcolumn*
\selectlanguage{latin}
\end{paracol}


\setrunningtitles{Majus}{May}

% ---- martyrology/mart05/mart0501.htm
\needspace{10\baselineskip}
\begin{paracol}{2}
\selectlanguage{latin}
\begin{center}{\color{gregoriocolor} Kaléndis Maji. 
 Luna\dots\ }\end{center}
\switchcolumn
\selectlanguage{english}
\begin{center}{\color{gregoriocolor} The 
 First Day of May. The\dots\ Day of the Moon.}\end{center}
\end{paracol}

\noindent\begin{tabularx}{\linewidth}{*{19}{>{\centering\arraybackslash}X}}
 \textcolor{gregoriocolor}{a} & \textcolor{gregoriocolor}{b} & \textcolor{gregoriocolor}{c} & \textcolor{gregoriocolor}{d} & \textcolor{gregoriocolor}{e} & \textcolor{gregoriocolor}{f} & \textcolor{gregoriocolor}{g} & \textcolor{gregoriocolor}{h} & \textcolor{gregoriocolor}{i} & \textcolor{gregoriocolor}{k} & \textcolor{gregoriocolor}{l} & \textcolor{gregoriocolor}{m} & \textcolor{gregoriocolor}{n} & \textcolor{gregoriocolor}{p} & \textcolor{gregoriocolor}{q} & \textcolor{gregoriocolor}{r} & \textcolor{gregoriocolor}{s} & \textcolor{gregoriocolor}{t} & \textcolor{gregoriocolor}{u} \\
 4 & 5 & 6 & 7 & 8 & 9 & 10 & 11 & 12 & 13 & 14 & 15 & 16 & 17 & 18 & 19 & 20 & 21 & 22 \\
\end{tabularx}
\vspace{0.5\baselineskip}
\noindent\begin{tabularx}{\linewidth}{*{12}{>{\centering\arraybackslash}X}}
 \textcolor{gregoriocolor}{A} & \textcolor{gregoriocolor}{B} & \textcolor{gregoriocolor}{C} & \textcolor{gregoriocolor}{D} & \textcolor{gregoriocolor}{E} & F & \textcolor{gregoriocolor}{F} & \textcolor{gregoriocolor}{G} & \textcolor{gregoriocolor}{H} & \textcolor{gregoriocolor}{M} & \textcolor{gregoriocolor}{N} & \textcolor{gregoriocolor}{P} \\
 23 & 24 & 25 & 26 & 27 & 28 & 27 & 28 & 29 & 1 & 2 & 3 \\
\end{tabularx}

\begin{paracol}{2}
\selectlanguage{latin}
\lettrine[lines=2]{N}{atális} beatórum Philíppi et Jacóbi Apostolórum. Ex 
 his Philíppus, cum omnem fere Scythiam ad Christi fidem convertísset, tandem 
 apud Hierápolim, Asiæ civitátem, cruci affíxus et lapídibus 
 óbrutus, 
 glorióso fine quiévit; Jacóbus vero, qui et frater Dómini légitur et primus 
 Hierosolymórum Epíscopus, e pinna Templi præcipitátus, confráctis inde 
 crúribus, ac fullónis fuste in cérebro percússus, intériit, ibique, non 
 longe a Templo, sepúltus est.
\switchcolumn
\selectlanguage{english}
\lettrine[lines=2]{T}{he} birthday of the blessed apostles Philip and James. 
 Philip, after having converted nearly all of Scythia to the faith of Christ, 
 went to Hieropolis, a city in Asia, where he was fastened to a cross and 
 stoned, and thus ended his life gloriously. James, who is also called 
 the brother of our Lord, was the first bishop of Jerusalem. Being 
 hurled down from a pinnacle of the temple, his legs were broken, and being 
 struck on the head with a dyer's staff, he expired and was buried near the 
 temple.
\switchcolumn*
\selectlanguage{latin}
Romæ item natális sancti Pii Quinti, ex Ordine 
 Prædicatórum, Papæ et Confessóris; qui, ecclesiásticæ disciplínæ restituéndæ, 
 hærésibus exstirpándis et Christiáni nóminis hóstibus conteréndis strénue ac 
 felíciter incúmbens, vitæ ac legum sanctitáte cathólicam Ecclésiam 
 gubernávit. Ipsíus autem festívitas tértio Nonas mensis hujus 
 celebrátur.
\switchcolumn
\selectlanguage{english}
At Rome, Pope St. Pius V of the Order of Preachers, who 
 laboured zealously and successfully for the re-establishment of church 
 discipline, the stamping out of heresies, and the destruction of the enemies 
 of the Christian name. He governed the Catholic Church by holy laws, 
 and the example of a saintly life. His feast is observed on the fifth 
 day of May.
\switchcolumn*
\selectlanguage{latin}
In Ægypto sancti Jeremíæ Prophétæ, qui, a pópulo 
 lapídibus óbrutus, apud Taphnas occúbuit, ibíque sepúltus est; ad cujus 
 sepúlcrum fidéles (ut refert sanctus Epiphánius) supplicáre consuevérunt, 
 índeque sumpto púlvere, áspidum mórsibus medéntur.
\switchcolumn
\selectlanguage{english}
In Egypt, St. Jeremias, prophet, who was stoned to death 
 by the people at Taphnas, where he was buried. St. Epiphanius tells 
 that the faithful were accustomed to pray at his grave, and to take away 
 from it dust to heal those who were stung by serpents.
\switchcolumn*
\selectlanguage{latin}
In território Vivariénsi, in Gálliis, beáti Andéoli, 
 Subdiáconi, qui, una cum áliis, a beáto Polycárpo, Smyrnénsi Epíscopo, ex 
 Oriénte in Gálliam ad prædicándum verbum Dei missus est. Hic, sub 
 Sevéro Imperatóre, spinósis fústibus cæsus, demum, per ensem lígneum cápite 
 in quátuor partes in modum crucis conscísso, martyrium consummávit.
\switchcolumn
\selectlanguage{english}
In France, in the Province of Vivarias, blessed Andeol, 
 subdeacon, who was sent from the East into Gaul with others by St. Polycarp 
 to preach the word of God. Under Emperor Severus he was scourged with 
 thorny sticks, and having his head split with a wooden sword into four 
 parts, in the shape of a cross, he completed his martyrdom.
\switchcolumn*
\selectlanguage{latin}
Oscæ, in Hispánia, sanctórum Mártyrum Oréntii et 
 Patiéntiæ.
\switchcolumn
\selectlanguage{english}
At Huesca in Spain, the holy martyrs Orentius and 
 Patience.
\switchcolumn*
\selectlanguage{latin}
Apud Colúmnam vicum, in Aurelianénsi Gálliæ território, 
 pássio sancti Sigismúndi, Regis Burgundiónum, qui in púteum demérsus 
 occúbuit, ac póstea miráculis cláruit. Sacrum vero ipsíus corpus, e 
 púteo tandem extráctum, ad Ecclésiam Agaunénsis monastérii, intra Sedunénsis 
 diœcésis términos siti, delátum est ibíque honorífice collocátum.
\switchcolumn
\selectlanguage{english}
In the town of Columna, in the province of Orleans in 
 France, the martyrdom of St. Sigismund, king of Burgundy. He met death 
 by being drowned in a well, and was afterwards famous for his miracles. 
 His venerable body was later recovered and taken to the monastery of Agaune 
 in the diocese of Sitten where it was honorably entombed.
\switchcolumn*
\selectlanguage{latin}
Antisiodóri sancti Amatóris, Epíscopi et Confessóris.
\switchcolumn
\selectlanguage{english}
At Auxerre, St. Amator, bishop and confessor.
\switchcolumn*
\selectlanguage{latin}
Auscii, in Gállia, sancti Oriéntii Epíscopi.
\switchcolumn
\selectlanguage{english}
At Auch in France, Bishop St. Orientius.
\switchcolumn*
\selectlanguage{latin}
Elviæ, in Anglia, sancti Asaphi Epíscopi, cujus nómine 
 ipsa cívitas Episcopális póstmodum est insigníta.
\switchcolumn
\selectlanguage{english}
At Llanelwy in Wales, Bishop St. Asaph, in whose memory 
 the cathedral city was later named.
\switchcolumn*
\selectlanguage{latin}
Foro Lívii sancti Peregríni, ex Ordine Servórum beátæ 
 Maríæ Vírginis.
\switchcolumn
\selectlanguage{english}
At Forli, St. Peregrinus of the Order of Servites of the 
 Blessed Virgin Mary.
\switchcolumn*
\selectlanguage{latin}
Bérgomi sanctæ Gratæ 
 Víduæ.
\switchcolumn
\selectlanguage{english}
At Bergamo, St. Grata, widow.
\switchcolumn*
\selectlanguage{latin}
\end{paracol}


% ---- martyrology/mart05/mart0502.htm
\needspace{10\baselineskip}
\begin{paracol}{2}
\selectlanguage{latin}
\begin{center}{\color{gregoriocolor} Sexto Nonas Maji. 
 Luna\dots\ }\end{center}
\switchcolumn
\selectlanguage{english}
\begin{center}{\color{gregoriocolor} The 
 Second Day of May. The\dots\ Day of the Moon.}\end{center}
\end{paracol}

\noindent\begin{tabularx}{\linewidth}{*{19}{>{\centering\arraybackslash}X}}
 \textcolor{gregoriocolor}{a} & \textcolor{gregoriocolor}{b} & \textcolor{gregoriocolor}{c} & \textcolor{gregoriocolor}{d} & \textcolor{gregoriocolor}{e} & \textcolor{gregoriocolor}{f} & \textcolor{gregoriocolor}{g} & \textcolor{gregoriocolor}{h} & \textcolor{gregoriocolor}{i} & \textcolor{gregoriocolor}{k} & \textcolor{gregoriocolor}{l} & \textcolor{gregoriocolor}{m} & \textcolor{gregoriocolor}{n} & \textcolor{gregoriocolor}{p} & \textcolor{gregoriocolor}{q} & \textcolor{gregoriocolor}{r} & \textcolor{gregoriocolor}{s} & \textcolor{gregoriocolor}{t} & \textcolor{gregoriocolor}{u} \\
 5 & 6 & 7 & 8 & 9 & 10 & 11 & 12 & 13 & 14 & 15 & 16 & 17 & 18 & 19 & 20 & 21 & 22 & 23 \\
\end{tabularx}
\vspace{0.5\baselineskip}
\noindent\begin{tabularx}{\linewidth}{*{12}{>{\centering\arraybackslash}X}}
 \textcolor{gregoriocolor}{A} & \textcolor{gregoriocolor}{B} & \textcolor{gregoriocolor}{C} & \textcolor{gregoriocolor}{D} & \textcolor{gregoriocolor}{E} & F & \textcolor{gregoriocolor}{F} & \textcolor{gregoriocolor}{G} & \textcolor{gregoriocolor}{H} & \textcolor{gregoriocolor}{M} & \textcolor{gregoriocolor}{N} & \textcolor{gregoriocolor}{P} \\
 24 & 25 & 26 & 27 & 28 & 29 & 28 & 29 & 1 & 2 & 3 & 4 \\
\end{tabularx}

\begin{paracol}{2}
\selectlanguage{latin}
\lettrine[lines=2]{A}{lexandríæ} natális 
 sancti Athanásii, ejúsdem urbis Epíscopi, Confessóris et Ecclésiæ Doctóris, 
 sanctitáte et doctrína claríssimi; in cujus persecutiónem univérsus fere 
 Orbis conjuráverat. Ipse tamen cathólicam fidem, a témpore Constantíni 
 usque ad Valéntem, advérsus Imperatóres ac Prǽsides et innúmeros Epíscopos 
 Ariános strénue propugnávit; a quibus plúrimas perpéssus insídias, prófugus 
 toto Orbe actus est, nec ullus ei tutus ad laténdum supérerat locus. 
 Tandem, ad suam Ecclésiam revérsus, illic, post multos agones multásque 
 patiéntiæ corónas, quadragésimo sexto sui sacerdótii anno migrávit ad 
 Dóminum, témpore Valentiniáni et Valéntis Imperatórum.
\switchcolumn
\selectlanguage{english}
\lettrine[lines=2]{A}{t} Alexandria, the birthday of St. 
 Athanasius, bishop of that city, confessor and doctor of the Church, most 
 celebrated for sanctity and learning. Although almost all of the world 
 had formed a conspiracy to persecute him, he courageously defended the 
 Catholic faith, from the reign of Constantine to that of Valens, against 
 emperors, governors, and a multitude of Arian bishops, whose underhanded 
 attacks forced him to wander as an exile over the whole earth without 
 finding a place of security. At length, however, he was restored to 
 his church, and after overcoming many trials, and winning many crowns by his 
 patience, he departed for heaven in the forty-sixth year of his priesthood, 
 in the time of the emperors Valentinian and Valens.
\switchcolumn*
\selectlanguage{latin}
Floréntiæ item natális 
 sancti Antoníni, ex Ordine Prædicatórum, Epíscopi et Confessóris, doctrína 
 et sanctitáte célebris. Ipsíus autem festívitas sexto Idus mensis 
 hujus recólitur.
\switchcolumn
\selectlanguage{english}
At Florence, Bishop St. Antoninus 
 of the Order of Preachers, renowned for sanctity and learning. His 
 feast is kept on the 10th of this month.
\switchcolumn*
\selectlanguage{latin}
Romæ sanctórum 
 Mártyrum Saturníni, Neópoli, Germáni et Cælestíni, qui, multa passi, in cárcerem demum conjécti, ibi in Dómino quievérunt.
\switchcolumn
\selectlanguage{english}
At Rome, the holy martyrs 
 Saturninus, Neopolus, Germanus, and Celestine, who after much suffering were 
 thrown into prison, where they found rest in the Lord.
\switchcolumn*
\selectlanguage{latin}
Eódem die sancti 
 Vindemiális, Epíscopi et Mártyris, qui, una cum sanctis Epíscopis Eugénio et 
 Longíno doctrína et miráculis advérsus Ariános decértans, a Rege Wandalórum 
 Hunneríco jubétur váriis torméntis afflígi ad tandem cápite obtruncári.
\switchcolumn
\selectlanguage{english}
The same day, St. Vindemial, bishop 
 and martyr, who with the holy bishops Eugene and Longinus, combated the 
 Arians by his teaching and miracles, and was beheaded by order of Hunneric, 
 king of the Vandals.
\switchcolumn*
\selectlanguage{latin}
Híspali, in Hispánia, sancti Felícis, Diáconi et Mártyris.
\switchcolumn
\selectlanguage{english}
At Seville in Spain, St. Felix, 
 deacon and martyr.
\switchcolumn*
\selectlanguage{latin}
Attalíæ, in Pamphylia, 
 sanctórum Mártyrum Exsupérii, et Zoes uxóris, atque Cyríaci et Theodúli 
 filiórum; qui, sub Hadriáno Imperatóre, cum servi essent cujúsdam viri 
 Pagáni, omnes, ipso hero jubénte, ob líberam Christiánæ fídei professiónem, 
 primum verberáti sunt ac veheménter torti, deínde, in accénsum clíbanum 
 injécti, ánimas suas Deo tradidérunt.
\switchcolumn
\selectlanguage{english}
At Attalia in Pamphylia, the holy 
 martyrs Exuperius and Zoe, his wife, with their sons, Cyriacus and Theodulus. 
 They were the slaves of a man named Paganus. During the reign of 
 Emperor Hadrian, because of their outspoken profession of the Christian 
 faith, their master ordered them to be scourged and severely tortured. 
 They were finally cast into an oven, and in this way gave up their souls to 
 God.
\switchcolumn*
\selectlanguage{latin}
\end{paracol}


% ---- martyrology/mart05/mart0503.htm
\needspace{10\baselineskip}
\begin{paracol}{2}
\selectlanguage{latin}
\begin{center}{\color{gregoriocolor} Quinto Nonas Maji. 
 Luna\dots\ }\end{center}
\switchcolumn
\selectlanguage{english}
\begin{center}{\color{gregoriocolor} The 
 Third Day of May. The\dots\ Day of the Moon.}\end{center}
\end{paracol}

\noindent\begin{tabularx}{\linewidth}{*{19}{>{\centering\arraybackslash}X}}
 \textcolor{gregoriocolor}{a} & \textcolor{gregoriocolor}{b} & \textcolor{gregoriocolor}{c} & \textcolor{gregoriocolor}{d} & \textcolor{gregoriocolor}{e} & \textcolor{gregoriocolor}{f} & \textcolor{gregoriocolor}{g} & \textcolor{gregoriocolor}{h} & \textcolor{gregoriocolor}{i} & \textcolor{gregoriocolor}{k} & \textcolor{gregoriocolor}{l} & \textcolor{gregoriocolor}{m} & \textcolor{gregoriocolor}{n} & \textcolor{gregoriocolor}{p} & \textcolor{gregoriocolor}{q} & \textcolor{gregoriocolor}{r} & \textcolor{gregoriocolor}{s} & \textcolor{gregoriocolor}{t} & \textcolor{gregoriocolor}{u} \\
 6 & 7 & 8 & 9 & 10 & 11 & 12 & 13 & 14 & 15 & 16 & 17 & 18 & 19 & 20 & 21 & 22 & 23 & 24 \\
\end{tabularx}
\vspace{0.5\baselineskip}
\noindent\begin{tabularx}{\linewidth}{*{12}{>{\centering\arraybackslash}X}}
 \textcolor{gregoriocolor}{A} & \textcolor{gregoriocolor}{B} & \textcolor{gregoriocolor}{C} & \textcolor{gregoriocolor}{D} & \textcolor{gregoriocolor}{E} & F & \textcolor{gregoriocolor}{F} & \textcolor{gregoriocolor}{G} & \textcolor{gregoriocolor}{H} & \textcolor{gregoriocolor}{M} & \textcolor{gregoriocolor}{N} & \textcolor{gregoriocolor}{P} \\
 25 & 26 & 27 & 28 & 29 & 30 & 29 & 1 & 2 & 3 & 4 & 5 \\
\end{tabularx}

\begin{paracol}{2}
\selectlanguage{latin}
\lettrine[lines=2]{H}{ierosólymis} Invéntio 
 Sacrosánctæ Crucis Domínicæ, sub Constantíno Imperatóre.
\switchcolumn
\selectlanguage{english}
\lettrine[lines=2]{A}{t} Jerusalem, in the time of 
 Emperor Constantine, the finding of the holy Cross of our Lord.
\switchcolumn*
\selectlanguage{latin}
Romæ, via Nomentána, 
 pássio sanctórum Mártyrum Alexándri Papæ Primi, Evéntii et Theodúli 
 Presbyterórum. Ex his Alexánder, sub Hadriáno Príncipe et Aureliáno 
 Júdice, post víncula, cárceres, equúleum, úngulas et ignes, punctis 
 crebérrimis per tota membra confóssus ac perémptus est; Evéntius vero et 
 Theodúlus, post longos cárceres, ígnibus examináti, ad 
 últimum decolláti 
 sunt.
\switchcolumn
\selectlanguage{english}
At Rome, on the Via Nomentana, the 
 holy martyrs Pope Alexander and the priests Eventius and Theodulus. 
 Alexander was bound, imprisoned, racked, lacerated with hooks, burned, and 
 had all his limbs pierced with pointed instruments, and finally met death, 
 under Emperor Hadrian and the judge Aurelian. Eventius and Theodulus 
 after a long imprisonment were exposed to flames and then beheaded.
\switchcolumn*
\selectlanguage{latin}
Nárniæ sancti Juvenáli, 
 Epíscopi et Confessóris.
\switchcolumn
\selectlanguage{english}
At Narni, St. Juvenal, bishop and 
 confessor.
\switchcolumn*
\selectlanguage{latin}
Apud montem Senárium, 
 in Etrúria, natális sanctórum Sostenǽi et Ugucciónis Confessórum, e septem 
 Fundatóribus Ordinis Servórum beátæ Maríæ Vírginis; qui, cælitus admóniti, 
 eádem die et hora, salutatiónem Angélicam recitántes, e vita migrárunt. 
 Ipsórum autem ac Sociórum festum prídie Idus Februárii celebrátur.
\switchcolumn
\selectlanguage{english}
On Mount Senario in Etruria, Saints 
 Sosteneo and Ugoccio, confessors, of the seven founders of the Order of 
 Servites of the Blessed Virgin Mary. Responding to a voice from 
 heaven, they departed this life on the same day and at the same hour, while 
 reciting the angelical salutation. Their feast is observed with the 
 rest of their companions on the 12th day of February.
\switchcolumn*
\selectlanguage{latin}
Constantinópoli 
 sanctórum Mártyrum Alexándri mílitis, et Antonínæ Vírginis. Hæc, in 
 persecutióne Maximiáni, sub Prǽside Festo, ad lupánar damnáta, et ab 
 Alexándro, qui pro ipsa ibi remánserat, mutátis véstibus clam edúcta, cum eo 
 póstmodum jussa est torquéri; et ambo simul in ignem, præcísis mánibus, pro 
 Christo sunt injécti, atque ita, egrégio perácto certámine, coronántur.
\switchcolumn
\selectlanguage{english}
At Constantinople, the holy martyrs 
 Alexander, soldier, and Antonina, virgin. In the persecution of 
 Maximian, under the governor Festus, Antonina, having been condemned to 
 remain in a place of debauchery, was delivered by Alexander, who secretly 
 exchanged garments with her, and took her place. They were tortured 
 together, both had their hands cut off, were cast into the fire, and 
 received their crowns at the end of their heroic combat for the faith.
\switchcolumn*
\selectlanguage{latin}
In Thebáide sanctórum 
 Mártyrum Timóthei et Mauræ cónjugis, quos Ariánus Præféctus, post multa 
 torménta, cruci jussit affígi; in qua, per novem dies cum vivi pependíssent 
 ac se ipsos in fide roborássent, martyrium consummárunt.
\switchcolumn
\selectlanguage{english}
In Thebais, the holy martyrs 
 Timothy and his wife Maura. The Arian prefect caused them to be 
 tortured, and then fastened to a cross, on which they remained alive for 
 nine days, encouraging each other to persevere in the faith, until they 
 completed their martyrdom.
\switchcolumn*
\selectlanguage{latin}
Aphrodísiæ, in Cária, 
 sanctórum Mártyrum Diodóri et Rhodopiáni, qui, in persecutióne Diocletiáni 
 Imperatóris, a cívibus suis lapidáti sunt.
\switchcolumn
\selectlanguage{english}
At Aphrodisia in Caria, the holy 
 martyrs Diodorus and Rodopian, who were stoned to death by their fellow 
 citizens, in the persecution of Diocletian.
\switchcolumn*
\selectlanguage{latin}
\end{paracol}


% ---- martyrology/mart05/mart0504.htm
\needspace{10\baselineskip}
\begin{paracol}{2}
\selectlanguage{latin}
\begin{center}{\color{gregoriocolor} Quarto Nonas Maji. 
 Luna\dots\ }\end{center}
\switchcolumn
\selectlanguage{english}
\begin{center}{\color{gregoriocolor} The 
 Fourth Day of May. The\dots\ Day of the Moon.}\end{center}
\end{paracol}

\noindent\begin{tabularx}{\linewidth}{*{19}{>{\centering\arraybackslash}X}}
 \textcolor{gregoriocolor}{a} & \textcolor{gregoriocolor}{b} & \textcolor{gregoriocolor}{c} & \textcolor{gregoriocolor}{d} & \textcolor{gregoriocolor}{e} & \textcolor{gregoriocolor}{f} & \textcolor{gregoriocolor}{g} & \textcolor{gregoriocolor}{h} & \textcolor{gregoriocolor}{i} & \textcolor{gregoriocolor}{k} & \textcolor{gregoriocolor}{l} & \textcolor{gregoriocolor}{m} & \textcolor{gregoriocolor}{n} & \textcolor{gregoriocolor}{p} & \textcolor{gregoriocolor}{q} & \textcolor{gregoriocolor}{r} & \textcolor{gregoriocolor}{s} & \textcolor{gregoriocolor}{t} & \textcolor{gregoriocolor}{u} \\
 7 & 8 & 9 & 10 & 11 & 12 & 13 & 14 & 15 & 16 & 17 & 18 & 19 & 20 & 21 & 22 & 23 & 24 & 25 \\
\end{tabularx}
\vspace{0.5\baselineskip}
\noindent\begin{tabularx}{\linewidth}{*{12}{>{\centering\arraybackslash}X}}
 \textcolor{gregoriocolor}{A} & \textcolor{gregoriocolor}{B} & \textcolor{gregoriocolor}{C} & \textcolor{gregoriocolor}{D} & \textcolor{gregoriocolor}{E} & F & \textcolor{gregoriocolor}{F} & \textcolor{gregoriocolor}{G} & \textcolor{gregoriocolor}{H} & \textcolor{gregoriocolor}{M} & \textcolor{gregoriocolor}{N} & \textcolor{gregoriocolor}{P} \\
 26 & 27 & 28 & 29 & 30 & 1 & 1 & 2 & 3 & 4 & 5 & 6 \\
\end{tabularx}

\begin{paracol}{2}
\selectlanguage{latin}
\lettrine[lines=2]{A}{pud} Ostia Tiberína 
 sanctæ Mónicæ, beáti Augustíni matris, cujus ille præcláram vitam, in libro 
 nono Confessiónum, testátam relíquit.
\switchcolumn
\selectlanguage{english}
\lettrine[lines=2]{A}{t} Ostia, the birthday of St. 
 Monica, mother of blessed Augustine. He has left us in the ninth book 
 of his Confessions a beautiful sketch of her life.
\switchcolumn*
\selectlanguage{latin}
In metállo Phennénsi 
 Palæstínæ natális beáti Silváni, Gazæ Epíscopi, qui, in Diocletiáni 
 Imperatóris persecutióne, Galérii Maximiáni Cǽsaris mandáto, cum plúrimis 
 suis Cléricis, martyrio coronátus est.
\switchcolumn
\selectlanguage{english}
At the metal mines of Phennes in 
 Palestine, the birthday of blessed Silvanus, bishop of Gaza, who was crowned 
 with martyrdom with many of his clerics by the command of Caesar Galerius 
 Maximian, in the persecution of Diocletian.
\switchcolumn*
\selectlanguage{latin}
Hierosólymis sancti 
 Cyríaci Epíscopi, qui, cum loca sancta visitáret, ibídem, sub Juliáno 
 Apóstata, cæsus est.
\switchcolumn
\selectlanguage{english}
At Jerusalem, in the reign of 
 Julian the Apostate, St. Cyriacus, bishop, who was murdered while visiting 
 the holy places.
\switchcolumn*
\selectlanguage{latin}
Cameríni sancti 
 Porphyrii Presbyteri et Mártyris, qui, sub Décio Imperatóre et Antíocho 
 Prǽside, cum plúrimos (in quibus fuit Venántius) ad Christi fidem 
 convertísset, cápite amputátus est.
\switchcolumn
\selectlanguage{english}
At Camerinum, St. Porphyry, priest 
 and martyr. Because he converted many to the faith (among them 
 Venantius), he was beheaded during the reign of Emperor Decius and the 
 governor Antiochus.
\switchcolumn*
\selectlanguage{latin}
In metállo Phennénsi 
 Palæstínæ sanctórum trigínta novem Mártyrum, qui, ad metálla damnáti, demum, 
 post adustiónem candéntis ferri et ália torménta, simul cápite cæsi sunt.
\switchcolumn
\selectlanguage{english}
Also in the mines of Phennes, 
 thirty-nine holy martyrs, who were condemned to work there, to be branded 
 with hot irons, to undergo other torments, and finally all to be beheaded at 
 the same time.
\switchcolumn*
\selectlanguage{latin}
Lauréaci, in Nórico 
 Ripénsi, sancti Floriáni Mártyris, qui, sub Diocletiáno Imperatóre, Aquilíni 
 Prǽsidis jussu, in flumen Anísum, ligáto ad collum saxo, præcipitátus est.
\switchcolumn
\selectlanguage{english}
At Lorch in Austria, under Emperor 
 Diocletian and the governor Aquilinus, the martyr St. Florian, who was 
 thrown into the River Enns, with a stone tied about his neck.
\switchcolumn*
\selectlanguage{latin}
Colóniæ Agrippínæ 
 sancti Paulíni Mártyris.
\switchcolumn
\selectlanguage{english}
At Cologne, the martyr St. Paulinus.
\switchcolumn*
\selectlanguage{latin}
Tarsi, in Cilícia, 
 sanctæ Pelágiæ, Vírginis et Mártyris, quæ, sub Diocletiáno Imperatóre, in 
 bovem æneum candéntem inclúsa, martyrium complévit.
\switchcolumn
\selectlanguage{english}
At Tarsus, St. Pelagia, virgin, who 
 endured martyrdom under Diocletian by being shut up inside an ox made of 
 brass that had been heated to redness.
\switchcolumn*
\selectlanguage{latin}
Nicomedíæ natális 
 sanctæ Antóniæ Mártyris, quæ, nímium torta et váriis afflícta cruciátibus, 
 áltero bráchio tribus suspénsa diébus, et in cárcere biénnio deténta, ad 
 últimum a Priscilliáno Prǽside, in confessióne Dómini, flammis exústa est.
\switchcolumn
\selectlanguage{english}
At Nicomedia, the birthday of St. 
 Antonia, martyr, who was cruelly tortured, subjected to various torments, 
 suspended by one arm for three days, kept two years in prison, and finally 
 delivered to the flames for the confession of Christ by the governor 
 Priscillian.
\switchcolumn*
\selectlanguage{latin}
Medioláni sancti 
 Venérii Epíscopi, cujus virtútes sanctus Joánnes Chrysóstomus, in epístola 
 ad eum scripta, testátas relíquit.
\switchcolumn
\selectlanguage{english}
At Milan, St. Venerius, a bishop 
 whose virtues are attested to by St. John Chrysostom in the epistle which he 
 had written to him.
\switchcolumn*
\selectlanguage{latin}
In território 
 Petragoricénsi sancti Sacerdótis, Epíscopi Lemovicénsis.
\switchcolumn
\selectlanguage{english}
In the province of Perigord, St. 
 Sacerdos, bishop of Limoges.
\switchcolumn*
\selectlanguage{latin}
Hildeshémii, in 
 Saxónia, sancti Godehárdi, Epíscopi et Confessóris, qui ab Innocéntio Papa 
 Secúndo in Sanctórum censum relátus est.
\switchcolumn
\selectlanguage{english}
At Hildesheim in Saxony, St. Gothard, 
 bishop and confessor, who was ranked among the saints by Innocent II.
\switchcolumn*
\selectlanguage{latin}
Antisiodóri sancti 
 Curcódomi Diáconi.
\switchcolumn
\selectlanguage{english}
At Auxerre, St. Curcodomus, deacon.
\switchcolumn*
\selectlanguage{latin}
\end{paracol}


% ---- martyrology/mart05/mart0505.htm
\needspace{10\baselineskip}
\begin{paracol}{2}
\selectlanguage{latin}
\begin{center}{\color{gregoriocolor} Tértio Nonas Maji. 
 Luna\dots\ }\end{center}
\switchcolumn
\selectlanguage{english}
\begin{center}{\color{gregoriocolor} The 
 Fifth Day of May. The\dots\ Day of the Moon.}\end{center}
\end{paracol}

\noindent\begin{tabularx}{\linewidth}{*{19}{>{\centering\arraybackslash}X}}
 \textcolor{gregoriocolor}{a} & \textcolor{gregoriocolor}{b} & \textcolor{gregoriocolor}{c} & \textcolor{gregoriocolor}{d} & \textcolor{gregoriocolor}{e} & \textcolor{gregoriocolor}{f} & \textcolor{gregoriocolor}{g} & \textcolor{gregoriocolor}{h} & \textcolor{gregoriocolor}{i} & \textcolor{gregoriocolor}{k} & \textcolor{gregoriocolor}{l} & \textcolor{gregoriocolor}{m} & \textcolor{gregoriocolor}{n} & \textcolor{gregoriocolor}{p} & \textcolor{gregoriocolor}{q} & \textcolor{gregoriocolor}{r} & \textcolor{gregoriocolor}{s} & \textcolor{gregoriocolor}{t} & \textcolor{gregoriocolor}{u} \\
 8 & 9 & 10 & 11 & 12 & 13 & 14 & 15 & 16 & 17 & 18 & 19 & 20 & 21 & 22 & 23 & 24 & 25 & 26 \\
\end{tabularx}
\vspace{0.5\baselineskip}
\noindent\begin{tabularx}{\linewidth}{*{12}{>{\centering\arraybackslash}X}}
 \textcolor{gregoriocolor}{A} & \textcolor{gregoriocolor}{B} & \textcolor{gregoriocolor}{C} & \textcolor{gregoriocolor}{D} & \textcolor{gregoriocolor}{E} & F & \textcolor{gregoriocolor}{F} & \textcolor{gregoriocolor}{G} & \textcolor{gregoriocolor}{H} & \textcolor{gregoriocolor}{M} & \textcolor{gregoriocolor}{N} & \textcolor{gregoriocolor}{P} \\
 27 & 28 & 29 & 30 & 1 & 2 & 2 & 3 & 4 & 5 & 6 & 7 \\
\end{tabularx}

\begin{paracol}{2}
\selectlanguage{latin}
\lettrine[lines=2]{S}{ancti} Pii Quinti, ex 
 Ordine Prædicatórum, Papæ et Confessóris, qui Kaléndis mensis hujus 
 obdormívit in Dómino.
\switchcolumn
\selectlanguage{english}
\lettrine[lines=2]{P}{ope} St. Pius V, confessor of the 
 Order of Preachers, who went to sleep in the Lord on the 1st of May.
\switchcolumn*
\selectlanguage{latin}
Romæ sancti Silváni 
 Mártyris.
\switchcolumn
\selectlanguage{english}
At Rome, the martyr St. Silvanus.
\switchcolumn*
\selectlanguage{latin}
Item Romæ sanctæ 
 Crescentiánæ Mártyris.
\switchcolumn
\selectlanguage{english}
Also at Rome, St. Crescentia, 
 martyr.
\switchcolumn*
\selectlanguage{latin}
Leocátæ, in Sicília, 
 sancti Angeli, ex Ordine Carmelitárum, Presbyteri et Mártyris, qui ab 
 hæréticis, ob defensiónem cathólicæ fídei, trucidátus est.
\switchcolumn
\selectlanguage{english}
At Leocata in Sicily, St. Angelus, 
 priest of the Order of Carmelites, who was murdered by the heretics because 
 of his defence of the Catholic faith.
\switchcolumn*
\selectlanguage{latin}
Alexandríæ sancti 
 Euthymii Diáconi, qui ob Christum quiévit in cárcere.
\switchcolumn
\selectlanguage{english}
At Alexandria, St. Euthymius, 
 deacon, who died in prison for the sake of Christ.
\switchcolumn*
\selectlanguage{latin}
Antisiodóri pássio 
 sancti Joviniáni Lectóris.
\switchcolumn
\selectlanguage{english}
At Auxerre, the martyrdom of St. 
 Jovinian, lector.
\switchcolumn*
\selectlanguage{latin}
Thessalonícæ natális 
 sanctórum Mártyrum Irenǽi, Peregríni et Irénes, qui, 
 ígnibus combústi, 
 palmas martyrii percepérunt.
\switchcolumn
\selectlanguage{english}
At Thessalonica, the birthday of 
 the holy martyrs Irenæus, Peregrinus, and Irene, who were burned alive.
\switchcolumn*
\selectlanguage{latin}
Hierosólymis sancti 
 Máximi Epíscopi, qui a Maximiáno Galério Cǽsare, post óculum effóssum 
 pedémque igníto ferro adústum, ad metálla damnátus est; atque, liber inde 
 abíre permíssus et Ecclésiæ Hierosolymitánæ præpósitus, ibi, confessiónis 
 glória præclárus, in pace tandem quiévit.
\switchcolumn
\selectlanguage{english}
At Jerusalem, St. Maximus, bishop, 
 whom Maximian Galerius Caesar condemned to the mines, after having plucked 
 out one of his eyes and branded him on the foot with a hot iron. He 
 was afterwards freed, and allowed to rule the church at Jerusalem, where he 
 died in peace, renowned for the glory of his confession.
\switchcolumn*
\selectlanguage{latin}
Edéssæ, in Syria, 
 sancti Eulógii, Epíscopi et Confessóris.
\switchcolumn
\selectlanguage{english}
At Edessa in Syria, St. Eulogius, 
 bishop and confessor.
\switchcolumn*
\selectlanguage{latin}
Areláte, in Gállia, 
 sancti Hilárii Epíscopi, doctrína et sanctitáte conspícui.
\switchcolumn
\selectlanguage{english}
At Arles in France, the bishop St. 
 Hilary, noted for his learning and sanctity.
\switchcolumn*
\selectlanguage{latin}
Viénnæ, in Gállia, 
 sancti Nicéti Epíscopi, venerábilis sanctitátis viri.
\switchcolumn
\selectlanguage{english}
At Vienne in France, the bishop St. 
 Nicetus, a man venerable for his piety.
\switchcolumn*
\selectlanguage{latin}
Bonóniæ sancti 
 Theodóri Epíscopi, méritis clari.
\switchcolumn
\selectlanguage{english}
At Bologna, St. Theodore, a bishop 
 who was eminent for merits.
\switchcolumn*
\selectlanguage{latin}
Medioláni sancti 
 Gerúntii Epíscopi.
\switchcolumn
\selectlanguage{english}
At Milan, the bishop St. Geruntius.
\switchcolumn*
\selectlanguage{latin}
Eódem die sancti 
 Sacerdótis, Epíscopi Saguntíni.
\switchcolumn
\selectlanguage{english}
On the same day, St. Sacerdos, 
 bishop of Saguntum.
\switchcolumn*
\selectlanguage{latin}
\end{paracol}


% ---- martyrology/mart05/mart0506.htm
\needspace{10\baselineskip}
\begin{paracol}{2}
\selectlanguage{latin}
\begin{center}{\color{gregoriocolor} Prídie Nonas Maji. 
 Luna\dots\ }\end{center}
\switchcolumn
\selectlanguage{english}
\begin{center}{\color{gregoriocolor} The 
 Sixth Day of May. The\dots\ Day of the Moon.}\end{center}
\end{paracol}

\noindent\begin{tabularx}{\linewidth}{*{19}{>{\centering\arraybackslash}X}}
 \textcolor{gregoriocolor}{a} & \textcolor{gregoriocolor}{b} & \textcolor{gregoriocolor}{c} & \textcolor{gregoriocolor}{d} & \textcolor{gregoriocolor}{e} & \textcolor{gregoriocolor}{f} & \textcolor{gregoriocolor}{g} & \textcolor{gregoriocolor}{h} & \textcolor{gregoriocolor}{i} & \textcolor{gregoriocolor}{k} & \textcolor{gregoriocolor}{l} & \textcolor{gregoriocolor}{m} & \textcolor{gregoriocolor}{n} & \textcolor{gregoriocolor}{p} & \textcolor{gregoriocolor}{q} & \textcolor{gregoriocolor}{r} & \textcolor{gregoriocolor}{s} & \textcolor{gregoriocolor}{t} & \textcolor{gregoriocolor}{u} \\
 9 & 10 & 11 & 12 & 13 & 14 & 15 & 16 & 17 & 18 & 19 & 20 & 21 & 22 & 23 & 24 & 25 & 26 & 27 \\
\end{tabularx}
\vspace{0.5\baselineskip}
\noindent\begin{tabularx}{\linewidth}{*{12}{>{\centering\arraybackslash}X}}
 \textcolor{gregoriocolor}{A} & \textcolor{gregoriocolor}{B} & \textcolor{gregoriocolor}{C} & \textcolor{gregoriocolor}{D} & \textcolor{gregoriocolor}{E} & F & \textcolor{gregoriocolor}{F} & \textcolor{gregoriocolor}{G} & \textcolor{gregoriocolor}{H} & \textcolor{gregoriocolor}{M} & \textcolor{gregoriocolor}{N} & \textcolor{gregoriocolor}{P} \\
 28 & 29 & 30 & 1 & 2 & 3 & 3 & 4 & 5 & 6 & 7 & 8 \\
\end{tabularx}

\begin{paracol}{2}
\selectlanguage{latin}
\lettrine[lines=2]{R}{omæ} sancti Joánnis, 
 Apóstoli et Evangelístæ, ante Portam Latínam; qui, ab Epheso, jussu 
 Domitiáni, vinctus Romam est perdúctus, et, judicánte Senátu, ante eándem 
 portam in ólei fervéntis dólium missus, exívit inde púrior et vegétior quam 
 intrávit.
\switchcolumn
\selectlanguage{english}
\lettrine[lines=2]{A}{t} Rome, the Apostle and Evangelist 
 St. John before the Latin Gate. He was bound and brought to Rome from 
 Ephesus by the order of Domitian, and the Senate condemned him to be taken 
 to that gate and placed in a cauldron of boiling oil, from which he came 
 forth more healthy and vigorous than before.
\switchcolumn*
\selectlanguage{latin}
Damásci natális beáti 
 Joánnis Damascéni, Presbyteri, Confessóris et Ecclésiæ Doctóris, doctrína et 
 sanctitáte célebris. Hic, pro cultu sanctárum Imáginum, verbo et 
 scriptis advérsus Leónem Isáuricum strénue decertávit; cujus Imperatóris ob 
 calúmnias cum ipsi Joánni déxtera manus e Saracenórum Príncipe amputáta 
 esset, idem, beátæ Maríæ Vírgini, cujus Imágines 
 defénderat, se comméndans, 
 prótinus déxteram íntegram sanámque recépit. Ejus autem festívitas 
 sexto Kaléndas Aprílis celebrátur.
\switchcolumn
\selectlanguage{english}
At Damascus, the birthday of St. 
 John Damascene, priest and doctor of the Church, renowned for sanctity and 
 learning. By means of his writing and preaching, he courageously 
 resisted Leo the Isaurian, in defending the veneration paid to sacred 
 images. By order of this emperor his right hand was cut off, but 
 commending himself before an image of the Blessed Virgin Mary, which he had 
 defended, his hand was immediately restored to him, entire and sound. 
 His feast day is the 27th of March.
\switchcolumn*
\selectlanguage{latin}
Cyréne, in Líbya, sancti Lúcii Epíscopi, quem in Actibus Apostolórum sanctus Lucas commémorat.
\switchcolumn
\selectlanguage{english}
At Cyrene in Africa, Bishop St. 
 Lucius, who is mentioned by St. Luke in the Acts of the Apostles.
\switchcolumn*
\selectlanguage{latin}
Antiochíæ sancti 
 Evódii, qui (ut beátus Ignátius ad Antiochénses scribit), primus ibídem a 
 sancto Petro Apóstolo ordinátus Epíscopus, glorióso martyrio vitam finívit.
\switchcolumn
\selectlanguage{english}
At Antioch, St. Evodius, who, as 
 the blessed Ignatius wrote to the people of Antioch, was consecrated first 
 bishop of that city by the apostle St. Peter, and ended his life by a 
 glorious martyrdom.
\switchcolumn*
\selectlanguage{latin}
In Africa sanctórum 
 Mártyrum Heliodóri et Venústi, cum áliis septuagínta quinque
\switchcolumn
\selectlanguage{english}
In Africa, the holy martyrs 
 Heliodorus and Venustus and seventy-five others.
\switchcolumn*
\selectlanguage{latin}
In Cypro sancti 
 Theódoti, Epíscopi Cyríniæ, qui, sub Licínio Imperatóre, gravíssima passus 
 est, ac tandem, in Ecclésiæ pace, spíritum Deo réddidit.
\switchcolumn
\selectlanguage{english}
In Cyprus, St. Theodotus, bishop of 
 Cyrinia, who having undergone grievous afflictions under Emperor Licinius, 
 at length yielded his soul to God when peace was restored to the Church.
\switchcolumn*
\selectlanguage{latin}
Carrhis, in 
 Mesopotámia, sancti Protógenis, Epíscopi et Confessóris.
\switchcolumn
\selectlanguage{english}
At Carrhae in Mesopotamia, St. 
 Protogenes, bishop and confessor.
\switchcolumn*
\selectlanguage{latin}
In Anglia, sancti 
 Eadbérti, Epíscopi Lindisfarnénsis, doctrína et pietáte insígnis.
\switchcolumn
\selectlanguage{english}
In England, St. Eadbert, bishop of 
 Lindisfarne, famed for his teachings and his piety.
\switchcolumn*
\selectlanguage{latin}
Romæ sanctæ Benedíctæ 
 Vírginis.
\switchcolumn
\selectlanguage{english}
At Rome, the virgin St. Benedicta.
\switchcolumn*
\selectlanguage{latin}
Salérni Translátio 
 sancti Matthǽi, Apóstoli et Evangelístæ; cujus sacrum corpus, olim ex 
 Æthiópia ad divérsas regiónes et demum ad eam urbem delátum, ibídem, in 
 dedicáta ejus nómine Ecclésia, summo honóre cónditum fuit.
\switchcolumn
\selectlanguage{english}
At Salerno, the translation of St. 
 Matthew, apostle and evangelist. His revered body, previously 
 transferred from Ethiopia to various countries, was finally taken to 
 Salerno, and with great pomp was there placed in a church dedicated to his 
 name.
\switchcolumn*
\selectlanguage{latin}
\end{paracol}


% ---- martyrology/mart05/mart0507.htm
\needspace{10\baselineskip}
\begin{paracol}{2}
\selectlanguage{latin}
\begin{center}{\color{gregoriocolor} Nonis Maji. 
 Luna\dots\ }\end{center}
\switchcolumn
\selectlanguage{english}
\begin{center}{\color{gregoriocolor} The 
 Seventh Day of May. The\dots\ Day of the Moon.}\end{center}
\end{paracol}

\noindent\begin{tabularx}{\linewidth}{*{19}{>{\centering\arraybackslash}X}}
 \textcolor{gregoriocolor}{a} & \textcolor{gregoriocolor}{b} & \textcolor{gregoriocolor}{c} & \textcolor{gregoriocolor}{d} & \textcolor{gregoriocolor}{e} & \textcolor{gregoriocolor}{f} & \textcolor{gregoriocolor}{g} & \textcolor{gregoriocolor}{h} & \textcolor{gregoriocolor}{i} & \textcolor{gregoriocolor}{k} & \textcolor{gregoriocolor}{l} & \textcolor{gregoriocolor}{m} & \textcolor{gregoriocolor}{n} & \textcolor{gregoriocolor}{p} & \textcolor{gregoriocolor}{q} & \textcolor{gregoriocolor}{r} & \textcolor{gregoriocolor}{s} & \textcolor{gregoriocolor}{t} & \textcolor{gregoriocolor}{u} \\
 10 & 11 & 12 & 13 & 14 & 15 & 16 & 17 & 18 & 19 & 20 & 21 & 22 & 23 & 24 & 25 & 26 & 27 & 28 \\
\end{tabularx}
\vspace{0.5\baselineskip}
\noindent\begin{tabularx}{\linewidth}{*{12}{>{\centering\arraybackslash}X}}
 \textcolor{gregoriocolor}{A} & \textcolor{gregoriocolor}{B} & \textcolor{gregoriocolor}{C} & \textcolor{gregoriocolor}{D} & \textcolor{gregoriocolor}{E} & F & \textcolor{gregoriocolor}{F} & \textcolor{gregoriocolor}{G} & \textcolor{gregoriocolor}{H} & \textcolor{gregoriocolor}{M} & \textcolor{gregoriocolor}{N} & \textcolor{gregoriocolor}{P} \\
 29 & 30 & 1 & 2 & 3 & 4 & 4 & 5 & 6 & 7 & 8 & 9 \\
\end{tabularx}

\begin{paracol}{2}
\selectlanguage{latin}
\lettrine[lines=2]{S}{ancti} Stanislái, 
 Epíscopi Cracoviénsis et Mártyris, qui sequénti die, corónam martyrii 
 consecútus est.
\switchcolumn
\selectlanguage{english}
\lettrine[lines=2]{S}{t.} Stanislas, bishop of Cracow 
 and martyr, who received the crown of martyrdom on the day following this.
\switchcolumn*
\selectlanguage{latin}
Tarracínæ, in Campánia, 
 natális beátæ Fláviæ Domitíllæ, Vírginis et Mártyris, quæ, cum esset fília 
 sanctæ Plautíllæ, soróris sancti Mártyris Flávii Cleméntis Cónsulis, et 
 sacro velámine fuísset a sancto Cleménte Pontífice consecráta, primum, in 
 persecutióne Domitiáni, ob testimónium Christi, in ínsulam Póntiam, cum 
 áliis plúrimis, exsílio deportáta, longum illic martyrium duxit. 
 Novíssime vero, Tarracínam dedúcta, ibi, cum plúrimos doctrína et miráculis 
 ad Christi fidem convertísset. Júdicis jussu incénso cubículo, in quo 
 simul cum suis virgínibus Euphrósyna et Theodóra morabátur, cursum gloriósi 
 martyrii consummávit. Ipsa vero Domitílla, una cum sanctis Martyribus 
 Néreo et Achílleo atque Pancrátio, festíva celebritáte quarto Idus mensis 
 hujus recólitur.
\switchcolumn
\selectlanguage{english}
At Terracina in Campania, the 
 birthday of blessed Flavia Domitilla, virgin and martyr, and niece of the 
 holy martyr, the Consul Flavius Clemens. She received the religious 
 veil at the hands of St. Clement, and in the persecution of Domitian was exiled with many others to the island of Pontia, where endured a long 
 martyrdom for Christ. Taken afterwards to Terracina, she converted 
 many to the faith of Christ by her teachings and miracles. The judge 
 ordered the room in which she was with the virgins Euphrosina and Theodora, 
 to be set on fire, and she thus completed her glorious martyrdom. She 
 is also mentioned with the holy martyrs Nereus, Achilleus and Pancras, on 
 the 12th day of this month.
\switchcolumn*
\selectlanguage{latin}
Eódem die sancti 
 Juvenális Mártyris.
\switchcolumn
\selectlanguage{english}
On the same day, St. Juvenal, 
 martyr.
\switchcolumn*
\selectlanguage{latin}
Nicomedíæ sanctórum 
 Mártyrum fratrum Flávii, Augústi et Augustíni.
\switchcolumn
\selectlanguage{english}
At Nicomedia, the holy martyrs 
 Flavius, Augustus and Augustine, all brothers.
\switchcolumn*
\selectlanguage{latin}
Ibídem sancti Quadráti 
 Mártyris, qui, in persecutióne Décii Imperatóris, sæpius ad torménta 
 repetítus, demum, cápite truncátus, martyrium complévit.
\switchcolumn
\selectlanguage{english}
In the same city, St. Quadratus, 
 martyr, who was frequently tortured in the persecution of Decius, and at 
 last beheaded.
\switchcolumn*
\selectlanguage{latin}
Eboráci, in Anglia, 
 sancti Joánnis Epíscopi, vita et miráculis clari.
\switchcolumn
\selectlanguage{english}
At York in England, St. John, 
 bishop, renowned for a saintly life and miracles.
\switchcolumn*
\selectlanguage{latin}
Papíæ sancti Petri 
 Epíscopi.
\switchcolumn
\selectlanguage{english}
At Pavia, Bishop St. Peter.
\switchcolumn*
\selectlanguage{latin}
Romæ Translátio 
 córporis sancti Stéphani Protomártyris, quod, Pelágio Primo Summo Pontífice, 
 e Constantinópoli ad Urbem allátum atque in sepúlcro sancti Lauréntii 
 Mártyris in agro Veráno pósitum, ibídem magna piórum religióne cólitur.
\switchcolumn
\selectlanguage{english}
At Rome, the translation of the 
 body of St. Stephen protomartyr, which was brought from Constantinople to 
 Rome by Pope Pelagius I, and laid in the sepulchre of the martyr St. Lawrence in the Agro 
 Verano, where it is honoured with great devotion by the pious faithful.
\switchcolumn*
\selectlanguage{latin}
\end{paracol}


% ---- martyrology/mart05/mart0508.htm
\needspace{10\baselineskip}
\begin{paracol}{2}
\selectlanguage{latin}
\begin{center}{\color{gregoriocolor} Octávo Idus Maji. 
 Luna\dots\ }\end{center}
\switchcolumn
\selectlanguage{english}
\begin{center}{\color{gregoriocolor} The 
 Eighth Day of May. The\dots\ Day of the Moon.}\end{center}
\end{paracol}

\noindent\begin{tabularx}{\linewidth}{*{19}{>{\centering\arraybackslash}X}}
 \textcolor{gregoriocolor}{a} & \textcolor{gregoriocolor}{b} & \textcolor{gregoriocolor}{c} & \textcolor{gregoriocolor}{d} & \textcolor{gregoriocolor}{e} & \textcolor{gregoriocolor}{f} & \textcolor{gregoriocolor}{g} & \textcolor{gregoriocolor}{h} & \textcolor{gregoriocolor}{i} & \textcolor{gregoriocolor}{k} & \textcolor{gregoriocolor}{l} & \textcolor{gregoriocolor}{m} & \textcolor{gregoriocolor}{n} & \textcolor{gregoriocolor}{p} & \textcolor{gregoriocolor}{q} & \textcolor{gregoriocolor}{r} & \textcolor{gregoriocolor}{s} & \textcolor{gregoriocolor}{t} & \textcolor{gregoriocolor}{u} \\
 11 & 12 & 13 & 14 & 15 & 16 & 17 & 18 & 19 & 20 & 21 & 22 & 23 & 24 & 25 & 26 & 27 & 28 & 29 \\
\end{tabularx}
\vspace{0.5\baselineskip}
\noindent\begin{tabularx}{\linewidth}{*{12}{>{\centering\arraybackslash}X}}
 \textcolor{gregoriocolor}{A} & \textcolor{gregoriocolor}{B} & \textcolor{gregoriocolor}{C} & \textcolor{gregoriocolor}{D} & \textcolor{gregoriocolor}{E} & F & \textcolor{gregoriocolor}{F} & \textcolor{gregoriocolor}{G} & \textcolor{gregoriocolor}{H} & \textcolor{gregoriocolor}{M} & \textcolor{gregoriocolor}{N} & \textcolor{gregoriocolor}{P} \\
 30 & 1 & 2 & 3 & 4 & 5 & 5 & 6 & 7 & 8 & 9 & 10 \\
\end{tabularx}

\begin{paracol}{2}
\selectlanguage{latin}
\lettrine[lines=2]{I}{n} monte Gargáno 
 Apparítio sancti Michaélis Archángeli, quem Pius Papa Duodécimus Radiólogis 
 et Radiumtherapéuticis Patrónum et Protectórum constítuit.
\switchcolumn
\selectlanguage{english}
\lettrine[lines=2]{O}{n} Mount Gargano, the apparition of 
 St. Michael Archangel, whom Pope Pius XII named the patron and protector of 
 radiologists and radiotherapists.
\switchcolumn*
\selectlanguage{latin}
Cracóviæ, in Polónia, 
 natális sancti Stanislái, Epíscopi et Mártyris, qui a Bolesláo, ímpio Rege, 
 necátus est. Ipsíus autem festum prídie hujus diéi celebrátur.
\switchcolumn
\selectlanguage{english}
At Cracow in Poland, the birthday 
 of St. Stanislas, bishop and martyr, who was slain by the wicked King 
 Boleslas. His feast was celebrated on the previous day.
\switchcolumn*
\selectlanguage{latin}
Medioláni item natális 
 sancti Victóris Mártyris, qui, natióne Maurus et a primæva ætate Christiánus, 
 a Maximiáno, cum esset in castris imperiálibus miles, compúlsus ut idólis 
 sacrificáret, et in confessióne Dómini fortíssime persevérans, ideo, primum 
 gráviter fústibus cæsus, sed, Deo protegénte, dolóris expers; deínde 
 liquénti plumbo perfúsus, sed nihil pénitus læsus; novíssime gloriósi 
 martyrii cursum, cápite abscíssus, implévit.
\switchcolumn
\selectlanguage{english}
At Milan, the birthday of the holy 
 martyr Victor, a Moor. He became a Christian in his youth and served 
 in the imperial army. When Maximian wished to force him to offer 
 sacrifice to idols, he persevered with the greatest fortitude in the 
 confession of the Lord. He was first beaten with rods, but by God's 
 protection without feeling any pain. Following this, melted lead was 
 poured over him, which did him no injury whatever. The career of his 
 glorious martyrdom was finally ended by his being beheaded.
\switchcolumn*
\selectlanguage{latin}
Constantinópoli sancti 
 Agáthii Centuriónis, qui, in persecutióne Diocletiáni et Maximiáni, a Firmo 
 Tribúno delátus quod Christiánus esset, et a Júdice Perínthi Bibiáno 
 sævíssime tortus, Byzántii demum a Procónsule Flaccíno cápitis damnátus est. 
 Ipsíus corpus ad Scyllácium littus, in Calábria, divínitus póstea delátum 
 est, atque ibi honorífice asservátum.
\switchcolumn
\selectlanguage{english}
At Constantinople, St. Acathius, 
 who, being denounced as a Christian by the tribune Firmus, and cruelly 
 tortured at Perinthus by the judge Bibian, was finally condemned to death at 
 Byzantium by the procunsul Flaccinus. His body was afterwards 
 miraculously brought to the shore of Squillace in Calabria, where it is 
 preserved with honour.
\switchcolumn*
\selectlanguage{latin}
Romæ sancti Bonifátii 
 Papæ Quarti, qui Pántheon in honórem beátæ Maríæ ad Mártyres dedicávit.
\switchcolumn
\selectlanguage{english}
At Rome, Pope St. Boniface IV, who 
 dedicated the Pantheon to the honour of our Lady and the martyrs.
\switchcolumn*
\selectlanguage{latin}
Item Romæ sancti 
 Benedícti Secúndi, Papæ et Confessóris.
\switchcolumn
\selectlanguage{english}
Also at Rome, St. Benedict II, pope 
 and confessor.
\switchcolumn*
\selectlanguage{latin}
Viénnæ, in Gállia, 
 sancti Dionysii, Epíscopi et Confessóris.
\switchcolumn
\selectlanguage{english}
At Vienne in France, St. Denis, 
 bishop and confessor.
\switchcolumn*
\selectlanguage{latin}
Antisiodóri sancti 
 Helládii Epíscopi.
\switchcolumn
\selectlanguage{english}
At Auxerre, St. Helladius, bishop.
\switchcolumn*
\selectlanguage{latin}
In monastério Bellæ 
 Vallis, in território Bisuntíno, sancti Petri, qui ex Mónacho Cisterciénsi 
 factus est Tarentasiénsis in Sabáudia Epíscopus.
\switchcolumn
\selectlanguage{english}
In the monastery of Bella Vallis, 
 in the diocese of Besançon, St. Peter, Cistercian monk, who was made bishop 
 of Tarantaise in Savoy.
\switchcolumn*
\selectlanguage{latin}
Apud Ruræmóndam, in 
 Géldria, sancti Wirónis, Epíscopi Scoti.
\switchcolumn
\selectlanguage{english}
At Ruremonde in Holland, St. Wiro, 
 bishop of Scotland.
\switchcolumn*
\selectlanguage{latin}
\end{paracol}


% ---- martyrology/mart05/mart0509.htm
\needspace{10\baselineskip}
\begin{paracol}{2}
\selectlanguage{latin}
\begin{center}{\color{gregoriocolor} Séptimo Idus Maji. 
 Luna\dots\ }\end{center}
\switchcolumn
\selectlanguage{english}
\begin{center}{\color{gregoriocolor} The 
 Ninth Day of May. The\dots\ Day of the Moon.}\end{center}
\end{paracol}

\noindent\begin{tabularx}{\linewidth}{*{19}{>{\centering\arraybackslash}X}}
 \textcolor{gregoriocolor}{a} & \textcolor{gregoriocolor}{b} & \textcolor{gregoriocolor}{c} & \textcolor{gregoriocolor}{d} & \textcolor{gregoriocolor}{e} & \textcolor{gregoriocolor}{f} & \textcolor{gregoriocolor}{g} & \textcolor{gregoriocolor}{h} & \textcolor{gregoriocolor}{i} & \textcolor{gregoriocolor}{k} & \textcolor{gregoriocolor}{l} & \textcolor{gregoriocolor}{m} & \textcolor{gregoriocolor}{n} & \textcolor{gregoriocolor}{p} & \textcolor{gregoriocolor}{q} & \textcolor{gregoriocolor}{r} & \textcolor{gregoriocolor}{s} & \textcolor{gregoriocolor}{t} & \textcolor{gregoriocolor}{u} \\
 12 & 13 & 14 & 15 & 16 & 17 & 18 & 19 & 20 & 21 & 22 & 23 & 24 & 25 & 26 & 27 & 28 & 29 & 30 \\
\end{tabularx}
\vspace{0.5\baselineskip}
\noindent\begin{tabularx}{\linewidth}{*{12}{>{\centering\arraybackslash}X}}
 \textcolor{gregoriocolor}{A} & \textcolor{gregoriocolor}{B} & \textcolor{gregoriocolor}{C} & \textcolor{gregoriocolor}{D} & \textcolor{gregoriocolor}{E} & F & \textcolor{gregoriocolor}{F} & \textcolor{gregoriocolor}{G} & \textcolor{gregoriocolor}{H} & \textcolor{gregoriocolor}{M} & \textcolor{gregoriocolor}{N} & \textcolor{gregoriocolor}{P} \\
 1 & 2 & 3 & 4 & 5 & 6 & 6 & 7 & 8 & 9 & 10 & 11 \\
\end{tabularx}

\begin{paracol}{2}
\selectlanguage{latin}
\lettrine[lines=2]{N}{aziánzi,} in 
 Cappadócia, natális beáti Gregórii Epíscopi, Confessóris et Ecclésiæ 
 Doctóris, ob singulárem divinárum rerum doctrínam cognoménto Theólogi; qui 
 collápsam Constantinópoli cathólicam fidem, ipsíus urbis Episcopátum gerens, 
 restítuit, hæresésque insurgéntes compréssit.
\switchcolumn
\selectlanguage{english}
\lettrine[lines=2]{A}{t} Nazianzum, the birthday of St. 
 Gregory, bishop, confessor, and doctor of the Church, surnamed the 
 Theologian because of his remarkable knowledge of divinity. At 
 Constantinople, he restored the Catholic faith which was fast waning, and 
 repressed the rising heresies.
\switchcolumn*
\selectlanguage{latin}
Romæ sancti Hermæ, 
 cujus Apóstolus Paulus in Epístola ad Romános méminit. Ipse autem 
 Hermas, digne semetípsum sacríficans acceptabilísque Deo hóstia factus, 
 virtútibus clarus cæléstia regna petívit.
\switchcolumn
\selectlanguage{english}
At Rome, St. Hermas, mentioned by 
 the apostle St. Paul in the Epistle to the Romans. Generously 
 sacrificing himself, he became an offering acceptable to God, and 
 outstanding for his virtues he took his departure for the heavenly kingdom.
\switchcolumn*
\selectlanguage{latin}
Cállii, via Flamínia, 
 pássio sancti Geróntii, Epíscopi Ficoclénsis.
\switchcolumn
\selectlanguage{english}
At Cagli, on the Flaminian Way, the 
 passion of St. Gerontius, bishop of Cervia.
\switchcolumn*
\selectlanguage{latin}
In Pérside sanctórum 
 Mártyrum trecentórum et decem.
\switchcolumn
\selectlanguage{english}
In Persia, three hundred and ten 
 holy martyrs.
\switchcolumn*
\selectlanguage{latin}
In Ægypto sancti 
 Pachómii Abbátis, qui plúrima eréxit in ea regióne monasteria, et régulam 
 Monachórum scripsit, quam ab Angelo dictánte didícerat.
\switchcolumn
\selectlanguage{english}
In Egypt, the abbot St. Pachomius, 
 who founded many monasteries in that country, and wrote a rule for monks 
 which he had learned from the dictation of an angel.
\switchcolumn*
\selectlanguage{latin}
In castro Vindecíno, 
 in Gállia, deposítio sancti Beáti Confessóris.
\switchcolumn
\selectlanguage{english}
In the town of Windisch in France, 
 the death of St. Beatus, confessor.
\switchcolumn*
\selectlanguage{latin}
Bonóniæ beáti Nicolái 
 Albergáti, Mónachi Carthusiáni, ejúsdem civitátis Epíscopi et sanctæ Románæ 
 Ecclésiæ Cardinális, sanctitáte et Apostólicis Legatiónibus clari; cujus 
 corpus Floréntiæ, apud Carthusiános, cónditum est.
\switchcolumn
\selectlanguage{english}
At Bologna, blessed Nicholas 
 Albergati, a Carthusian monk, bishop of that city, and cardinal of the Holy 
 Roman Church, celebrated for his sanctity and and for his work as an 
 apostolic legate. His body was buried at Florence in the monastery of 
 the Carthusians.
\switchcolumn*
\selectlanguage{latin}
Constantinópoli 
 Translátio sanctórum Andréæ Apóstoli, et Lucæ Evangelístæ de Achája; et 
 Timóthei, uníus ex discípulis beáti Pauli Apóstoli, ab Epheso. Corpus 
 autem sancti Andréæ, longo post témpore Amálphim devéctum, ibi pio fidélium 
 concúrsu honorátur; ex cujus sepúlcro liquor ad languóres curándos júgiter 
 manat.
\switchcolumn
\selectlanguage{english}
At Constantinople, the translation 
 of the apostle St. Andrew and the evangelist St. Luke, out of Achaia, and of 
 Timothy, disciple of the blessed apostle Paul, from Ephesus. The body 
 of St. Andrew, long after, was conveyed to Amalfi, where it is honoured by 
 the pious gatherings of the faithful. From his tomb there continually 
 flows a liquid which heals diseases.
\switchcolumn*
\selectlanguage{latin}
Romæ item Translátio 
 sancti Hierónymi Presbyteri, Confessóris et Ecclésiæ Doctóris, ex Béthlehem Judæ ad Basílicam sanctæ Maríæ ad Præsépe.
\switchcolumn
\selectlanguage{english}
At Rome, also, the translation of 
 St. Jerome, priest, confessor, and doctor of the Church. His body was 
 taken from Bethlehem of Judea to the basilica of St. Mary of the Manger.
\switchcolumn*
\selectlanguage{latin}
Bárii quoque, in 
 Apúlia, Translátio sancti Nicolái, Epíscopi et Confessóris, ex Myra, 
 civitáte Lyciæ.
\switchcolumn
\selectlanguage{english}
At Bari in Apulia, the translation 
 also of St. Nicholas, bishop and confessor, from Myra, a city of Lycia.
\switchcolumn*
\selectlanguage{latin}
\end{paracol}


% ---- martyrology/mart05/mart0510.htm
\needspace{10\baselineskip}
\begin{paracol}{2}
\selectlanguage{latin}
\begin{center}{\color{gregoriocolor} Sexto Idus Maji. 
 Luna\dots\ }\end{center}
\switchcolumn
\selectlanguage{english}
\begin{center}{\color{gregoriocolor} The 
 Tenth Day of May. The\dots\ Day of the Moon.}\end{center}
\end{paracol}

\noindent\begin{tabularx}{\linewidth}{*{19}{>{\centering\arraybackslash}X}}
 \textcolor{gregoriocolor}{a} & \textcolor{gregoriocolor}{b} & \textcolor{gregoriocolor}{c} & \textcolor{gregoriocolor}{d} & \textcolor{gregoriocolor}{e} & \textcolor{gregoriocolor}{f} & \textcolor{gregoriocolor}{g} & \textcolor{gregoriocolor}{h} & \textcolor{gregoriocolor}{i} & \textcolor{gregoriocolor}{k} & \textcolor{gregoriocolor}{l} & \textcolor{gregoriocolor}{m} & \textcolor{gregoriocolor}{n} & \textcolor{gregoriocolor}{p} & \textcolor{gregoriocolor}{q} & \textcolor{gregoriocolor}{r} & \textcolor{gregoriocolor}{s} & \textcolor{gregoriocolor}{t} & \textcolor{gregoriocolor}{u} \\
 13 & 14 & 15 & 16 & 17 & 18 & 19 & 20 & 21 & 22 & 23 & 24 & 25 & 26 & 27 & 28 & 29 & 30 & 1 \\
\end{tabularx}
\vspace{0.5\baselineskip}
\noindent\begin{tabularx}{\linewidth}{*{12}{>{\centering\arraybackslash}X}}
 \textcolor{gregoriocolor}{A} & \textcolor{gregoriocolor}{B} & \textcolor{gregoriocolor}{C} & \textcolor{gregoriocolor}{D} & \textcolor{gregoriocolor}{E} & F & \textcolor{gregoriocolor}{F} & \textcolor{gregoriocolor}{G} & \textcolor{gregoriocolor}{H} & \textcolor{gregoriocolor}{M} & \textcolor{gregoriocolor}{N} & \textcolor{gregoriocolor}{P} \\
 2 & 3 & 4 & 5 & 6 & 7 & 7 & 8 & 9 & 10 & 11 & 12 \\
\end{tabularx}

\begin{paracol}{2}
\selectlanguage{latin}
\lettrine[lines=2]{S}{ancti} Antoníni, ex 
 Ordine Prædicatórum, Epíscopi Florentíni et Confessóris, cujus dies natális 
 sexto Nonas mensis hujus recensétur.
\switchcolumn
\selectlanguage{english}
\lettrine[lines=2]{S}{t.} Antoninus of the Order of 
 Preachers, confessor and archbishop of Florence, whose birthday is the 2nd 
 of May.
\switchcolumn*
\selectlanguage{latin}
Romæ, via Latína, natális sanctórum Mártyrum Gordiáni et Epímachi, quorum prior, pro 
 confessióne nóminis Christi, témpore Juliáni Apóstatæ, diu plumbátis cæsus 
 et ad últimum cápite truncátus, noctu a Christiánis sepúltus eádem via fuit 
 in crypta, in quam beáti Epímachi Mártyris relíquiæ paulo ante translátæ 
 fúerant ab Alexandría, ubi ipse, pro Christi fide, martyrium compléverat 
 prídie Idus Decémbris.
\switchcolumn
\selectlanguage{english}
At Rome, on the Via Latina, the 
 birthday of the holy martyrs Gordian and Epimachus. In the time of 
 Julian the Apostate, the former was a long time scourged and finally 
 beheaded for confessing the name of Christ. He was buried at night by 
 the Christians, in a crypt to which, shortly before, the remains of the 
 blessed martyr Epimachus had been transferred from Alexandria, where he had 
 been martyred for the faith of Christ on the 12th of December.
\switchcolumn*
\selectlanguage{latin}
In terra Hus sancti 
 Job Prophétæ, admirándæ patiéntiæ viri.
\switchcolumn
\selectlanguage{english}
In the land of Hus, the holy 
 prophet Job, a man of wonderful patience.
\switchcolumn*
\selectlanguage{latin}
Romæ beáti Calepódii, 
 Presbyteri et Mártyris; quem Alexánder Imperátor gládio fecit occídi, et 
 corpus ejus per civitátem trahi, atque in Tíberim jactári, quod invéntum 
 Callístus Papa sepelívit. Decollátus est étiam Palmátius Consul cum 
 uxóre et fíliis et áliis promíscui sexus quadragínta duóbus de domo sua, 
 Simplícius quoque Senátor cum uxóre et sexagínta octo de famila sua, item et 
 Felix cum uxóre sua Blanda; quorum cápita suspénsa sunt per divérsas portas 
 Urbis, ad exémplum Christianórum.
\switchcolumn
\selectlanguage{english}
At Rome, the blessed priest and 
 martyr Calepodius, who was killed with the sword by order of Emperor 
 Alexander. His body was dragged through the city and thrown into the 
 Tiber. It was afterwards found and buried by Pope Callistus. The 
 consul Palmatius was also beheaded with his wife, his sons, and forty-two of 
 both sexes belonging to his household; likewise the senator Simplicius with 
 his wife, and sixty-eight of his house; Felix also with his wife Blanda. 
 The heads of all these martyrs were exposed over different gates of the city 
 in order to terrify the Christians.
\switchcolumn*
\selectlanguage{latin}
Item Romæ, via Latína, 
 ad Centum Aulas, natális sanctórum Mártyrum Quarti et Quincti, quorum 
 córpora Cápuam transláta sunt.
\switchcolumn
\selectlanguage{english}
Also at Rome, on the Via Latina, 
 the birthday of the holy martyrs Quartus and Quinctus, whose bodies were 
 translated to Capua.
\switchcolumn*
\selectlanguage{latin}
Apud Leontínos, in 
 Sicília, sanctórum Mártyrum Alphii, Philadélphi et Cyríni.
\switchcolumn
\selectlanguage{english}
At Lentini in Sicily, the holy 
 martyrs Alphius, Philadelphis, and Cyrinus.
\switchcolumn*
\selectlanguage{latin}
Smyrnæ sancti 
 Dioscóridis Mártyris.
\switchcolumn
\selectlanguage{english}
At Smyrna, St. Dioscorides, martyr.
\switchcolumn*
\selectlanguage{latin}
Apud Taréntum sancti 
 Catáldi Epíscopi, miráculis clari.
\switchcolumn
\selectlanguage{english}
At Taranto, St. Cataldus, a bishop 
 renowned for miracles.
\switchcolumn*
\selectlanguage{latin}
Matríti sancti Isidóri 
 Agrícolæ, quem, miráculis clarum, Gregórius Papa Décimus quintus, una cum 
 sanctis Ignátio, Francísco Xavério, Terésia et Philíppo Nério, in Sanctórum 
 númerum rétulit.
\switchcolumn
\selectlanguage{english}
At Madrid, St. Isidore the Farmer. 
 Being well known for his miracles, Pope Gregory XV placed him in the number 
 of saints at the same time with St. Ignatius, St. Francis Xavier, St. 
 Teresa, and St. Philip Neri.
\switchcolumn*
\selectlanguage{latin}
Medioláni Invéntio 
 sanctórum Mártyrum Nazárii et Celsi, in qua beátus Ambrósius Epíscopus 
 corpus sancti Nazárii recénti adhuc sánguine conspérsum réperit, atque ad 
 Basílicam Apostolórum tránstulit, una cum córpore beáti Celsi púeri, quem 
 idem ipse Nazárius nutríerat, et Anolínus, in Nerónis persecutióne, simul 
 cum eo feríri gládio jússerat quinto Kaléndas Augústi; quo die festívitas 
 gloriósi eórum martyrii celebrátur.
\switchcolumn
\selectlanguage{english}
At Milan, the finding of the bodies 
 of the holy martyrs Nazarius and Celsus. The blessed bishop Ambrose 
 found the body of St. Nazarius covered with blood still fresh, and 
 transferred it to the Basilica of the Apostles, together with the body of 
 the blessed Celsus, a youth whom Nazarius had taken care of, and whom 
 Anolinus, in the persecution of Nero, had ordered to be slain with the sword 
 on the 28th of July, on which day their martyrdom is commemorated.
\switchcolumn*
\selectlanguage{latin}
\end{paracol}


% ---- martyrology/mart05/mart0511.htm
\needspace{10\baselineskip}
\begin{paracol}{2}
\selectlanguage{latin}
\begin{center}{\color{gregoriocolor} Quinto Idus Maji. 
 Luna\dots\ }\end{center}
\switchcolumn
\selectlanguage{english}
\begin{center}{\color{gregoriocolor} The 
 Eleventh Day of May. The\dots\ Day of the Moon.}\end{center}
\end{paracol}

\noindent\begin{tabularx}{\linewidth}{*{19}{>{\centering\arraybackslash}X}}
 \textcolor{gregoriocolor}{a} & \textcolor{gregoriocolor}{b} & \textcolor{gregoriocolor}{c} & \textcolor{gregoriocolor}{d} & \textcolor{gregoriocolor}{e} & \textcolor{gregoriocolor}{f} & \textcolor{gregoriocolor}{g} & \textcolor{gregoriocolor}{h} & \textcolor{gregoriocolor}{i} & \textcolor{gregoriocolor}{k} & \textcolor{gregoriocolor}{l} & \textcolor{gregoriocolor}{m} & \textcolor{gregoriocolor}{n} & \textcolor{gregoriocolor}{p} & \textcolor{gregoriocolor}{q} & \textcolor{gregoriocolor}{r} & \textcolor{gregoriocolor}{s} & \textcolor{gregoriocolor}{t} & \textcolor{gregoriocolor}{u} \\
 14 & 15 & 16 & 17 & 18 & 19 & 20 & 21 & 22 & 23 & 24 & 25 & 26 & 27 & 28 & 29 & 30 & 1 & 2 \\
\end{tabularx}
\vspace{0.5\baselineskip}
\noindent\begin{tabularx}{\linewidth}{*{12}{>{\centering\arraybackslash}X}}
 \textcolor{gregoriocolor}{A} & \textcolor{gregoriocolor}{B} & \textcolor{gregoriocolor}{C} & \textcolor{gregoriocolor}{D} & \textcolor{gregoriocolor}{E} & F & \textcolor{gregoriocolor}{F} & \textcolor{gregoriocolor}{G} & \textcolor{gregoriocolor}{H} & \textcolor{gregoriocolor}{M} & \textcolor{gregoriocolor}{N} & \textcolor{gregoriocolor}{P} \\
 3 & 4 & 5 & 6 & 7 & 8 & 8 & 9 & 10 & 11 & 12 & 13 \\
\end{tabularx}

% NOTE: check that this layout is ok on p. 126-127
\begin{paracol}{2}
\selectlanguage{latin}
\lettrine[lines=2]{R}{omæ,} via Salária, 
 natális beáti Anthimi Presbyteri, qui, post virtútum et prædicatiónis 
 insígnia, in persecutióne Diocletiáni, in Tíberim præcipitátus, et ab Angelo 
 exínde eréptus, oratório próprio restitútus est; deínde, cápite punítus, 
 victor migrávit ad cælos.
\switchcolumn
\selectlanguage{english}
\lettrine[lines=2]{A}{t} Rome, on the Salarian Way, the 
 birthday of blessed Anthimus, priest, who, after having distinguished 
 himself by his virtues and preaching, was cast into the Tiber during the 
 persecution of Diocletian. He was rescued by an angel and restored to 
 his oratory. Afterwards he was beheaded, and went victoriously to 
 heaven.
\switchcolumn*
\selectlanguage{latin}
Ibídem sancti Evéllii 
 Mártyris, qui, cum esset de família Nerónis, ad passiónem sancti Torpétis in 
 Christum crédidit, pro quo et decollátus est.
\switchcolumn
\selectlanguage{english}
In the same place, St. Evelius, 
 martyr, who belonged to the household of Nero. By witnessing the 
 martyrdom of St. Torpes, he also believed in Christ, and for him was 
 beheaded.
\switchcolumn*
\selectlanguage{latin}
Item Romæ sanctórum 
 Mártyrum Máximi, Bassi et Fábii; qui sub Diocletiáno, via Salária, cæsi sunt.
\switchcolumn
\selectlanguage{english}
Also at Rome, on the Salarian Way, 
 the holy martyrs Maximus, Bassus, and Fabius, who were put to death during 
 the reign of Diocletian.
\switchcolumn*
\selectlanguage{latin}
Auximi, in Picéno, 
 sanctórum Mártyrum Sisínii Diáconi, Dioclétii et Floréntii, discipulórum 
 sancti Anthimi Presbyteri; qui, sub Diocletiáno, lapídibus óbruti, martyrium 
 complevérunt.
\switchcolumn
\selectlanguage{english}
At Osimo in Piceno, the holy 
 martyrs Sisinius, a deacon, Diocletius and Florentius, disciples of the 
 priest St. Anthimus, whose martyrdom was completed under Diocletian by their 
 being stoned.
\switchcolumn*
\selectlanguage{latin}
Cameríni sanctórum 
 Mártyrum Anastásii et Sociórum; qui, in persecutióne Décii, sub Antíocho Prǽside, cæsi sunt.
\switchcolumn
\selectlanguage{english}
At Camerino, the holy martyrs 
 Anastasius and his companions who were killed in the persecution of Decius, 
 under the governor Antiochus.
\switchcolumn*
\selectlanguage{latin}
Varénnis, in Gállia, 
 sancti Gangúlfi Mártyris.
\switchcolumn
\selectlanguage{english}
At Varennes in France, St. 
 Gangulphus, martyr.
\switchcolumn*
\selectlanguage{latin}
Viénnæ, in Gállia, 
 sancti Mamérti Epíscopi, qui, ob imminéntem cladem, solémnes ante 
 Ascensiónem Dómini triduánas in ea urbe Litanías instítuit; quem ritum 
 póstea universális Ecclésia recípiens comprobávit.
\switchcolumn
\selectlanguage{english}
At Vienne in France, St. Mamertus, 
 bishop, who, to avert an impending calamity, instituted in that city the 
 three days' Litanies immediately before the Ascension of our Lord. 
 This rite was afterwards received and approved by the universal Church.
\switchcolumn*
\selectlanguage{latin}
Apud Silviníacum, in 
 Gállia, deposítio sancti Majóli, Abbátis Cluniacénsis, cujus vita sanctis 
 méritis fuit præclára.
\switchcolumn
\selectlanguage{english}
At Souvigny in France, the death of 
 St. Maieul, abbot of Cluny, whose life was distinguished for merits and 
 sanctity.
\switchcolumn*
\selectlanguage{latin}
Neápoli, in Campánia, 
 sancti Francísci de Hierónymo, in Tarentínæ diœcésis 
 óppido Cryptaleárum 
 orti, Sacerdótis e Societáte Jesu et Confessóris, exímiæ in salúte animárum 
 procuránda caritátis et patiéntiæ viri; quem Gregórius Papa Décimus sextus 
 in Sanctórum cánonem rétulit.
\switchcolumn
\selectlanguage{english}
At Naples in Campania, St. Francis 
 of Jerome, priest of the Society of Jesus, and confessor. He was born 
 in the town of Grottaglia, in the diocese of Taranto. Having been a 
 man of great patience and zeal for the salvation of souls, he was canonized 
 by Pope Gregory XVI.
\switchcolumn*
\selectlanguage{latin}
Apud Septempedános, in 
 Picéno, sancti Illumináti Confessóris.
\switchcolumn
\selectlanguage{english}
At San Severino in Piceno, St. 
 Illuminatus, confessor.
\switchcolumn*
\selectlanguage{latin}
Cálari, in Sardínia, sancti Ignátii a Lacóni, Confessóris, ex Ordine Minórum Capuccinórum, 
 humilitáte, caritáte et miráculis præclári; quem Pius Papa Duodécimus 
 Sanctórum honóribus decorávit.
\switchcolumn
\selectlanguage{english}
At Cagliari in Sardinia, St. 
 Ignatius of Laconi, confessor, of the Minor Order of Capuchins, 
 distinguished for his humility, charity and miracles. He was accorded 
 the honour of canonization by Pope Pius XII.
\switchcolumn*
\selectlanguage{latin}
\end{paracol}


% ---- martyrology/mart05/mart0512.htm
\needspace{10\baselineskip}
\begin{paracol}{2}
\selectlanguage{latin}
\begin{center}{\color{gregoriocolor} Quarto Idus Maji. 
 Luna\dots\ }\end{center}
\switchcolumn
\selectlanguage{english}
\begin{center}{\color{gregoriocolor} The 
 Twelfth Day of May. The\dots\ Day of the Moon.}\end{center}
\end{paracol}

\noindent\begin{tabularx}{\linewidth}{*{19}{>{\centering\arraybackslash}X}}
 \textcolor{gregoriocolor}{a} & \textcolor{gregoriocolor}{b} & \textcolor{gregoriocolor}{c} & \textcolor{gregoriocolor}{d} & \textcolor{gregoriocolor}{e} & \textcolor{gregoriocolor}{f} & \textcolor{gregoriocolor}{g} & \textcolor{gregoriocolor}{h} & \textcolor{gregoriocolor}{i} & \textcolor{gregoriocolor}{k} & \textcolor{gregoriocolor}{l} & \textcolor{gregoriocolor}{m} & \textcolor{gregoriocolor}{n} & \textcolor{gregoriocolor}{p} & \textcolor{gregoriocolor}{q} & \textcolor{gregoriocolor}{r} & \textcolor{gregoriocolor}{s} & \textcolor{gregoriocolor}{t} & \textcolor{gregoriocolor}{u} \\
 15 & 16 & 17 & 18 & 19 & 20 & 21 & 22 & 23 & 24 & 25 & 26 & 27 & 28 & 29 & 30 & 1 & 2 & 3 \\
\end{tabularx}
\vspace{0.5\baselineskip}
\noindent\begin{tabularx}{\linewidth}{*{12}{>{\centering\arraybackslash}X}}
 \textcolor{gregoriocolor}{A} & \textcolor{gregoriocolor}{B} & \textcolor{gregoriocolor}{C} & \textcolor{gregoriocolor}{D} & \textcolor{gregoriocolor}{E} & F & \textcolor{gregoriocolor}{F} & \textcolor{gregoriocolor}{G} & \textcolor{gregoriocolor}{H} & \textcolor{gregoriocolor}{M} & \textcolor{gregoriocolor}{N} & \textcolor{gregoriocolor}{P} \\
 4 & 5 & 6 & 7 & 8 & 9 & 9 & 10 & 11 & 12 & 13 & 14 \\
\end{tabularx}

\begin{paracol}{2}
\selectlanguage{latin}
\lettrine[lines=2]{R}{omæ,} via Ardeatína, sanctórum Mártyrum Nérei et Achíllei 
 fratrum, qui primo cum Flávia Domitílla, cujus erant eunúchi, in ínsula 
 Póntia longum pro Christo duxérunt exsílium; póstmodum gravíssimis 
 verbéribus attrectáti sunt; deínde, cum a Minútio Rufo, viro Consulári, 
 equúleo et flammis ad immolándum compelleréntur, diceréntque se, a beáto 
 Petro Apóstolo baptizátos, nulla ratióne posse idólis immoláre, cápite cæsi 
 sunt. Horum sacræ relíquiæ, simúlque Fláviæ Domitíllæ, ex Diaconía 
 sancti Hadriáni in antíquum eórum Títulum, ubi asservabántur olim recónditæ, 
 dénuo restaurátum, solémniter translátæ sunt prídie hujus diéi, jussu 
 Cleméntis Papæ Octávi; qui exínde hodiérna celebrándum die indíxit étiam 
 festum ipsíus beátæ Domitíllæ Vírginis, cujus pássio Nonis hujus mensis 
 recensétur.
\switchcolumn
\selectlanguage{english}
\lettrine[lines=2]{A}{t} Rome, on the Ardeatine Way, the holy martyrs Nereus 
 and Achilleus, brothers, who underwent a long exile for Christ in the island 
 of Pontia with Flavia Domitilla, whose chamberlains they were. 
 Afterwards they endured a most severe scourging. Finally, as the 
 judge, Minutius Rufus, endeavoured by using the rack and fire to force them 
 to offer sacrifices, they said that having been baptized by the blessed 
 apostle Peter, they could by no means sacrifice to idols. They were 
 beheaded, and their revered remains, with those of Flavia Domitilla, were, 
 by order of Pope Clement VIII, solemnly transferred the day before this, 
 from the sacristy of St. Adrian to the church in which they had been kept in 
 the first place, and which was now repaired. He also ordered today's 
 observance of the feast of St. Domitilla, the virgin, whose martyrdom was 
 mentioned on the 7th of May.
\switchcolumn*
\selectlanguage{latin}
Item Romæ, via Aurélia, sancti Pancrátii Mártyris, qui, 
 cum esset annórum quatuórdecim, sub Diocletiáno, cápitis obtruncatióne 
 martyrium complévit.
\switchcolumn
\selectlanguage{english}
In the same place, on the Aurelian Way, the holy martyr 
 Pancras who at fourteen years of age endured martyrdom by being beheaded 
 under Diocletian.
\switchcolumn*
\selectlanguage{latin}
Salamínæ, in Cypro, sancti Epiphánii Epíscopi, qui, 
 multíplici eruditióne et sacrárum sciéntia litterárum excéllens, vitæ quoque 
 sanctitáte, zelo cathólicæ fídei, munificéntia in páuperes et virtúte 
 miraculórum éxstitit admirándus.
\switchcolumn
\selectlanguage{english}
At Salamis in Cyprus, St. Epiphanius, a bishop of great 
 erudition, with a profound knowledge of the Holy Scriptures. He is to 
 be admired for the sanctity of his life, his zeal for the Catholic faith, 
 his charity to the poor, and the gift of miracles.
\switchcolumn*
\selectlanguage{latin}
Constantinópoli sancti Germáni Epíscopi, doctrína et 
 virtútibus insígnis, qui Leónem Isáuricum, edíctum advérsus sacras Imágines 
 promulgántem, magna fidúcia redárguit.
\switchcolumn
\selectlanguage{english}
At Constantinople, St. Germanus, a bishop distinguished 
 by his virtues and learning, who faithfully opposed Leo the Isaurian for 
 publishing an edict against sacred images.
\switchcolumn*
\selectlanguage{latin}
Tréviris sancti Modoáldi Epíscopi.
\switchcolumn
\selectlanguage{english}
At Treves, St. Modoaldus, bishop.
\switchcolumn*
\selectlanguage{latin}
Romæ sancti Dionysii, qui éxstitit pátruus sancti 
 Pancrátii Mártyris.
\switchcolumn
\selectlanguage{english}
At Rome, St. Denis, uncle of the martyr St. Pancras.
\switchcolumn*
\selectlanguage{latin}
Argyrii, in Sicília, sancti Philíppi Presbyteri, qui, a 
 Románo Pontífice ad eándem Sicíliam ínsulam missus, magnam illíus partem 
 convértit ad Christum. Ipsíus vero sánctitas in liberándis energúmenis 
 máxime declarátur.
\switchcolumn
\selectlanguage{english}
At Agirone in Sicily, St. Philip, a priest who was sent 
 to that island by the Roman Pontiff, and converted to Christ a great portion 
 of it. His sanctity is particularly manifested by the deliverance of 
 persons possessed.
\switchcolumn*
\selectlanguage{latin}
In civitáte 
 Calciaténsi, in Hispánia, sancti Domínici Confessóris.
\switchcolumn
\selectlanguage{english}
In the city of Calzada in Spain, 
 St. Dominic, confessor.
\switchcolumn*
\selectlanguage{latin}
\end{paracol}


% ---- martyrology/mart05/mart0513.htm
\needspace{10\baselineskip}
\begin{paracol}{2}
\selectlanguage{latin}
\begin{center}{\color{gregoriocolor} Tértio Idus Maji. 
 Luna\dots\ }\end{center}
\switchcolumn
\selectlanguage{english}
\begin{center}{\color{gregoriocolor} The 
 Thirteenth Day of May. The\dots\ Day of the Moon.}\end{center}
\end{paracol}

\noindent\begin{tabularx}{\linewidth}{*{19}{>{\centering\arraybackslash}X}}
 \textcolor{gregoriocolor}{a} & \textcolor{gregoriocolor}{b} & \textcolor{gregoriocolor}{c} & \textcolor{gregoriocolor}{d} & \textcolor{gregoriocolor}{e} & \textcolor{gregoriocolor}{f} & \textcolor{gregoriocolor}{g} & \textcolor{gregoriocolor}{h} & \textcolor{gregoriocolor}{i} & \textcolor{gregoriocolor}{k} & \textcolor{gregoriocolor}{l} & \textcolor{gregoriocolor}{m} & \textcolor{gregoriocolor}{n} & \textcolor{gregoriocolor}{p} & \textcolor{gregoriocolor}{q} & \textcolor{gregoriocolor}{r} & \textcolor{gregoriocolor}{s} & \textcolor{gregoriocolor}{t} & \textcolor{gregoriocolor}{u} \\
 16 & 17 & 18 & 19 & 20 & 21 & 22 & 23 & 24 & 25 & 26 & 27 & 28 & 29 & 30 & 1 & 2 & 3 & 4 \\
\end{tabularx}
\vspace{0.5\baselineskip}
\noindent\begin{tabularx}{\linewidth}{*{12}{>{\centering\arraybackslash}X}}
 \textcolor{gregoriocolor}{A} & \textcolor{gregoriocolor}{B} & \textcolor{gregoriocolor}{C} & \textcolor{gregoriocolor}{D} & \textcolor{gregoriocolor}{E} & F & \textcolor{gregoriocolor}{F} & \textcolor{gregoriocolor}{G} & \textcolor{gregoriocolor}{H} & \textcolor{gregoriocolor}{M} & \textcolor{gregoriocolor}{N} & \textcolor{gregoriocolor}{P} \\
 5 & 6 & 7 & 8 & 9 & 10 & 10 & 11 & 12 & 13 & 14 & 15 \\
\end{tabularx}

\begin{paracol}{2}
\selectlanguage{latin}
\lettrine[lines=2]{S}{ancti} Robérti 
 Bellarmíno, e Societáte Jesu, Cardinális atque olim Epíscopi Capuáni, 
 Confessóris et Ecclésiæ Doctóris, cujus dies natális décimo quinto Kaléndas 
 Octóbris recensétur.
\switchcolumn
\selectlanguage{english}
\lettrine[lines=2]{S}{t.} Robert Bellarmine, of the 
 Society of Jesus, cardinal and one time bishop of Capua, confessor and 
 doctor of the Church, whose birthday is kept on the 17th of September.
\switchcolumn*
\selectlanguage{latin}
Romæ Dedicátio 
 Ecclésiæ sanctæ Maríæ ad Mártyres, quam beátus Bonifátius Papa Quartus, 
 expurgáto deórum ómnium véteri fano, quod Pántheon vocabátur, in honórem 
 beátæ semper Vírginis Maríæ et ómnium Mártyrum dedicávit, témpore Phocæ 
 Imperatóris. Ipsíus vero Dedicatiónis ánnuam solemnitátem póstmodum 
 Summus Póntifex Gregórius item Quartus ab univérsa Ecclésia, et in honórem 
 quidem ómnium Sanctórum, Kaléndis Novémbris agéndam esse constítuit.
\switchcolumn
\selectlanguage{english}
At Rome, in the time of Emperor 
 Phocas, the dedication of the church of St. Mary of the Martyrs, formerly a 
 temple of all the gods, called the Pantheon, which was purified and 
 dedicated by the blessed Pope Boniface IV to the honour of the Blessed Mary 
 ever Virgin, and of all the martyrs. The solemn anniversary of this 
 dedication was later ordered to be kept by Pope Gregory IV as the Feast of 
 All Saints on the 1st of November.
\switchcolumn*
\selectlanguage{latin}
Constantinópoli beáti 
 Múcii, Presbyteri et Mártyris; qui, sub Diocletiáno Imperatóre et Laudício 
 Procónsule, prius Amphípoli, in Macedónia, multis pœnis atque cruciátibus ob 
 Christi confessiónem afflíctus, póstea, Byzántium usque perdúctus, capitáli 
 senténtia occúbuit.
\switchcolumn
\selectlanguage{english}
At Constantinople, under Emperor 
 Diocletian and the proconsul Laudicius, the blessed Mucius, priest and 
 martyr, who endured many tribulations and torments for the confession of 
 Christ at Amphipolis, and then being taken to Byzantium, suffered death.
\switchcolumn*
\selectlanguage{latin}
Alexandríæ 
 commemorátio plurimórum sanctórum Mártyrum, qui, ob fidem cathólicam, ab 
 Ariánis in Ecclésia Theónæ occísi sunt.
\switchcolumn
\selectlanguage{english}
At Alexandria, the commemoration of 
 many holy martyrs, who were put to death for the Catholic faith by the 
 Arians in the church of St. Theonas.
\switchcolumn*
\selectlanguage{latin}
Heracléæ, in Thrácia, 
 sanctæ Glycériæ, Mártyris Románæ, quæ, sub Antoníno Imperatóre et Sabíno 
 Prǽside, cum fuísset plúrimis ac diris tentáta supplíciis et ab his divína 
 ope incólumis evasísset, tandem feris est objécta, earúmque una, morsum 
 córpori ipsíus infigénte, spíritum Deo réddidit.
\switchcolumn
\selectlanguage{english}
At Heraclea in Thrace, St. Glyceria, 
 a Roman martyr who suffered many severe torments under Emperor Antonius and 
 the governor Sabinus. By the help of God having escaped them all 
 unharmed, she was finally thrown to the wild beasts, and when the first one had 
 bitten her body, she rendered her soul to God.
\switchcolumn*
\selectlanguage{latin}
Apud Trajéctum sancti 
 Servátii, Tungrénsis Ecclésiæ Epíscopi; ad cujus méritum ómnibus 
 demonstrándum, dum témpore híemis ómnia in circúitu nix repléret, sepúlcrum 
 ejus numquam opéruit, donec, ex indústria cívium, Basílica super illud 
 ædificáta est.
\switchcolumn
\selectlanguage{english}
At Utrecht, St. Servatius, bishop 
 of Tongres, whose grave, as a public sign of his merit, was free from snow 
 during winter (although everything around was covered with it), until the 
 inhabitants built a church over it.
\switchcolumn*
\selectlanguage{latin}
In Palæstína sancti 
 Joánnis Silentiárii, qui, Coloniénsi in Arménia Episcopátu dimísso, in 
 sancti Sabbæ laura monásticam vitam duxit, et sancto fine quiévit.
\switchcolumn
\selectlanguage{english}
In Palestine, St. John the Silent, 
 who resigned the see of Colonia in Armenia and retired to the monastery of 
 St. Sabbas until his saintly death.
\switchcolumn*
\selectlanguage{latin}
Pódii, in diœcési 
 Pictaviénsi, sancti Andréæ Hubérti Fournet, Confessóris, olim párochi, 
 Institúti Filiárum a Cruce una cum sancta Elísabeth Bichier des Ages 
 Fundatóris, quem Pius Papa Undécimus Sanctórum fastis adscrípsit.
\switchcolumn
\selectlanguage{english}
At La Puye in the diocese of 
 Poitiers, St. André-Hubert Fournet, confessor and one time parish priest, 
 and founder with St. Elizabeth-Lucie Bichier des Ages of the Institute of 
 the Daughters of the Holy Cross. He was placed on the roll of the 
 saints by Pope Pius XI.
\switchcolumn*
\selectlanguage{latin}
\end{paracol}


% ---- martyrology/mart05/mart0514.htm
\needspace{10\baselineskip}
\begin{paracol}{2}
\selectlanguage{latin}
\begin{center}{\color{gregoriocolor} Prídie Idus Maji. 
 Luna\dots\ }\end{center}
\switchcolumn
\selectlanguage{english}
\begin{center}{\color{gregoriocolor} The 
 Fourteenth Day of May. The\dots\ Day of the Moon.}\end{center}
\end{paracol}

\noindent\begin{tabularx}{\linewidth}{*{19}{>{\centering\arraybackslash}X}}
 \textcolor{gregoriocolor}{a} & \textcolor{gregoriocolor}{b} & \textcolor{gregoriocolor}{c} & \textcolor{gregoriocolor}{d} & \textcolor{gregoriocolor}{e} & \textcolor{gregoriocolor}{f} & \textcolor{gregoriocolor}{g} & \textcolor{gregoriocolor}{h} & \textcolor{gregoriocolor}{i} & \textcolor{gregoriocolor}{k} & \textcolor{gregoriocolor}{l} & \textcolor{gregoriocolor}{m} & \textcolor{gregoriocolor}{n} & \textcolor{gregoriocolor}{p} & \textcolor{gregoriocolor}{q} & \textcolor{gregoriocolor}{r} & \textcolor{gregoriocolor}{s} & \textcolor{gregoriocolor}{t} & \textcolor{gregoriocolor}{u} \\
 17 & 18 & 19 & 20 & 21 & 22 & 23 & 24 & 25 & 26 & 27 & 28 & 29 & 30 & 1 & 2 & 3 & 4 & 5 \\
\end{tabularx}
\vspace{0.5\baselineskip}
\noindent\begin{tabularx}{\linewidth}{*{12}{>{\centering\arraybackslash}X}}
 \textcolor{gregoriocolor}{A} & \textcolor{gregoriocolor}{B} & \textcolor{gregoriocolor}{C} & \textcolor{gregoriocolor}{D} & \textcolor{gregoriocolor}{E} & F & \textcolor{gregoriocolor}{F} & \textcolor{gregoriocolor}{G} & \textcolor{gregoriocolor}{H} & \textcolor{gregoriocolor}{M} & \textcolor{gregoriocolor}{N} & \textcolor{gregoriocolor}{P} \\
 6 & 7 & 8 & 9 & 19 & 11 & 11 & 12 & 13 & 14 & 15 & 16 \\
\end{tabularx}

\begin{paracol}{2}
\selectlanguage{latin}
\lettrine[lines=2]{T}{arsi,} in Cilícia, 
 natális sancti Bonifátii Mártyris, qui, sub Diocletiáno et Maximiáno passus, 
 deínde Romam advéctus, via Latína sepúltus est.
\switchcolumn
\selectlanguage{english}
\lettrine[lines=2]{A}{t} Tarsus in Cilicia, the birthday 
 of the holy martyr Boniface, who suffered under Diocletian and Maximian. 
 His body was subsequently taken to Rome and buried on the Via Latina.
\switchcolumn*
\selectlanguage{latin}
In Gállia sancti 
 Póntii Mártyris, cujus prædicatióne et indústria postquam duo Philíppi Cǽsares ad Christi fidem convérsi sunt, ipse, sub Valeriáno et Galliéno 
 Princípibus, martyrii palmam adéptus est.
\switchcolumn
\selectlanguage{english}
In France, St. Pontius, martyr. 
 Having by his preaching and his zeal converted to the faith of Christ the 
 two Caesars Philippi, he obtained the palm of martyrdom under the emperors 
 Valerian and Gallienus.
\switchcolumn*
\selectlanguage{latin}
In Syria sanctórum 
 Mártyrum Victóris et Corónæ, sub Antoníno Imperatóre; ex quibus Victor a 
 Sebastiáno Júdice váriis et horréndis afféctus est cruciátibus. Cum 
 autem ipsum Coróna, uxor cujúsdam mílitis, cœpísset beátum prædicáre ob 
 martyrii constántiam, vidit duas corónas de cælo lapsas, unam Victóri et 
 álteram sibi missam; cumque hoc audiéntibus cunctis testarétur, ipsa quidem 
 inter árbores scissa, Victor vero decollátus est.
\switchcolumn
\selectlanguage{english}
In Syria, the holy martyrs Victor 
 and Corona, under Emperor Antoninus. Victor was subjected to diverse 
 and horrible torments by the judge Sebastian. Just then, as Corona, 
 the the wife of a certain soldier, proclaimed him blessed for his 
 constancy in his sufferings, she saw two crowns falling from heaven, one for 
 Victor, the other for herself. She related this to all present, and 
 was torn to pieces between two trees, while Victor was beheaded.
\switchcolumn*
\selectlanguage{latin}
In Sardínia sanctárum 
 Mártyrum Justæ, Justínæ et Henedínæ.
\switchcolumn
\selectlanguage{english}
In Sardinia, the holy martyrs Justa, 
 Justina, and Henedina.
\switchcolumn*
\selectlanguage{latin}
Ferénti, in Túscia, 
 sancti Bonifátii Epíscopi, qui (ut refert beátus Gregórius Papa) sanctitáte 
 et miráculis a puerítia cláruit.
\switchcolumn
\selectlanguage{english}
At Ferentino in Tuscany, Bishop St. 
 Boniface, who was renowned for sanctity and miracles from his childhood as 
 is told by the blessed Pope Gregory.
\switchcolumn*
\selectlanguage{latin}
In pago Bethárram, 
 diœcésis Bajonénsis, sancti Michaélis Garicoïts, Confessóris, Congregatiónis 
 Presbyterórum Missionariórum a Sacro Corde Jesu Fundatóris, apostólico zelo 
 insígnis, quem Pius Papa Duodécimus Sanctórum fastis adscrípsit.
\switchcolumn
\selectlanguage{english}
In the town of Betharram in the 
 diocese of Bayonne, St. Michael Garricoits, confessor, and founder of the 
 Congregation of the Priests of the Sacred Heart, renowned for his apostolic 
 fervour. Pope Pius XII added him to the roll of saints.
\switchcolumn*
\selectlanguage{latin}
Níciæ, in Subalpínis, 
 sanctæ Maríæ Domínicæ Mazzaréllo, Confundatrícis Institúti Filiárum Maríæ 
 Auxiliatrícis, quæ humilitáte, prudéntia et caritáte præclára, in album 
 sanctárum Vírginum a Pio Papa Duodécimo fuit reláta.
\switchcolumn
\selectlanguage{english}
At Nizza Monferrato in Italy, St. 
 Mary Dominica Mazzarello, co-founder of the Daughters of Mary Help of 
 Christians, and renowned for her humility, prudence and charity. She 
 was added to the book of Virgins by Pope Pius XII.
\switchcolumn*
\selectlanguage{latin}
\end{paracol}


% ---- martyrology/mart05/mart0515.htm
\needspace{10\baselineskip}
\begin{paracol}{2}
\selectlanguage{latin}
\begin{center}{\color{gregoriocolor} Idibus Maji. 
 Luna\dots\ }\end{center}
\switchcolumn
\selectlanguage{english}
\begin{center}{\color{gregoriocolor} The 
 Fifteenth Day of May. The\dots\ Day of the Moon.}\end{center}
\end{paracol}

\noindent\begin{tabularx}{\linewidth}{*{19}{>{\centering\arraybackslash}X}}
 \textcolor{gregoriocolor}{a} & \textcolor{gregoriocolor}{b} & \textcolor{gregoriocolor}{c} & \textcolor{gregoriocolor}{d} & \textcolor{gregoriocolor}{e} & \textcolor{gregoriocolor}{f} & \textcolor{gregoriocolor}{g} & \textcolor{gregoriocolor}{h} & \textcolor{gregoriocolor}{i} & \textcolor{gregoriocolor}{k} & \textcolor{gregoriocolor}{l} & \textcolor{gregoriocolor}{m} & \textcolor{gregoriocolor}{n} & \textcolor{gregoriocolor}{p} & \textcolor{gregoriocolor}{q} & \textcolor{gregoriocolor}{r} & \textcolor{gregoriocolor}{s} & \textcolor{gregoriocolor}{t} & \textcolor{gregoriocolor}{u} \\
 18 & 19 & 20 & 21 & 22 & 23 & 24 & 25 & 26 & 27 & 28 & 29 & 30 & 1 & 2 & 3 & 4 & 5 & 6 \\
\end{tabularx}
\vspace{0.5\baselineskip}
\noindent\begin{tabularx}{\linewidth}{*{12}{>{\centering\arraybackslash}X}}
 \textcolor{gregoriocolor}{A} & \textcolor{gregoriocolor}{B} & \textcolor{gregoriocolor}{C} & \textcolor{gregoriocolor}{D} & \textcolor{gregoriocolor}{E} & F & \textcolor{gregoriocolor}{F} & \textcolor{gregoriocolor}{G} & \textcolor{gregoriocolor}{H} & \textcolor{gregoriocolor}{M} & \textcolor{gregoriocolor}{N} & \textcolor{gregoriocolor}{P} \\
 7 & 8 & 9 & 10 & 11 & 12 & 12 & 13 & 14 & 15 & 16 & 17 \\
\end{tabularx}

\begin{paracol}{2}
\selectlanguage{latin}
\lettrine[lines=2]{S}{ancti} Joánnis 
 Baptístæ de La Salle, Presbyteri et Confessóris, qui Sodalitátis Fratrum 
 Scholárum Christianárum fuit Institútor, ac séptimo Idus Aprílis obdormívit 
 in Dómino.
\switchcolumn
\selectlanguage{english}
\lettrine[lines=2]{S}{t.} John Baptist de la Salle, 
 priest and confessor, who founded the Society of Brothers of the Christian 
 Schools. He went to rest in the Lord on the 7th of April.
\switchcolumn*
\selectlanguage{latin}
In Hispánia sanctórum 
 Torquáti, Ctesiphóntis, Secúndi, Indalétii, Cæcílii, Hesychii et Euphrásii; 
 qui Romæ a sanctis Apóstolis Epíscopi ordináti, et ad prædicándum verbum Dei 
 in Hispánias dirécti sunt. Cum autem hi váriis úrbibus evangelizássent, 
 et innúmeras multitúdines ad Christi fidem perduxíssent, divérsis in ea 
 província locis quievérunt; scílicet Torquátus Acci, Ctésiphon Vérgii, 
 Secúndus Abulæ, Indalétius Urci, Cæcílius Illíberi, Hesychius Cartéjæ et 
 Euphrásius Illitúrgi.
\switchcolumn
\selectlanguage{english}
In Spain, the Saints Torquatus, 
 Ctesiphon, Secundus, Indaletius, Cecilius, Hesychius, and Euphrasius, who 
 were consecrated bishops at Rome by the holy apostles, and sent to Spain to 
 preach the word of God. When they had evangelized various cities, and 
 brought innumerable multitudes under the yoke of Christ, they rested in 
 peace in different places in that country: Torquatus at Cadiz, Ctesiphon at 
 Vierco, Secundus at Avila, Indaletius at Portilla, Cecílius at Elvira, 
 Hesychius at Gibraltar, and Euphrasius at Anduxar.
\switchcolumn*
\selectlanguage{latin}
Fausínæ, in Sardínia, sancti Simplícii, Epíscopi et Mártyris, qui, Diocletiáni témpore, sub 
 Bárbaro Prǽside, perfóssus láncea martyrium consummávit.
\switchcolumn
\selectlanguage{english}
At Fausina in Sardinia, in the time 
 of Diocletian and the governor Barbarus, Bishop St. Simplicius, who was 
 pierced with a lance and thus gained martyrdom.
\switchcolumn*
\selectlanguage{latin}
Eboræ, in Lusitánia, 
 sancti Máncii Mártyris.
\switchcolumn
\selectlanguage{english}
At Evora in Portugal, St. Mancius, 
 martyr.
\switchcolumn*
\selectlanguage{latin}
In ínsula Chio natális 
 beáti Isidóri Mártyris, in cujus Basílica exstat púteus, in quem fertur 
 fuísse injéctus, de cujus aqua infírmi potáti sæpius sanántur.
\switchcolumn
\selectlanguage{english}
In the island of Chio, the birthday 
 of blessed Isidore, martyr, in whose church is a well into which he is said 
 to have been thrown. By drinking of the water of this well, the sick 
 are frequently cured.
\switchcolumn*
\selectlanguage{latin}
Lampásci, in 
 Hellespónto, pássio sanctórum Petri, Andréæ, Pauli et Dionysiæ.
\switchcolumn
\selectlanguage{english}
At Lampascum in the Hellespont, the 
 martyrdom of the Saints Peter, Andrew, Paul, and Dionysia.
\switchcolumn*
\selectlanguage{latin}
Arvérnis, in Gállia, 
 sanctórum Mártyrum Cássii, Victoríni, Máximi et Sociórum.
\switchcolumn
\selectlanguage{english}
In the Auvergne in France, the holy martyrs 
 Cassius, Victorinus, Maximus, and their companions.
\switchcolumn*
\selectlanguage{latin}
Ghelæ, in Brabántia, 
 sanctæ Dympnæ, Vírginis et Mártyris, fíliæ Regis Hibernórum; quæ, cum in 
 fide Christi et virginitáte servánda permanéret immóbilis, a patre jussa est 
 decollári.
\switchcolumn
\selectlanguage{english}
At Gheel in Brabant, St. Dymphna, 
 virgin and martyr, daughter of the king of Ireland. By order of her 
 father, she was beheaded for the faith of Christ and the preservation of her 
 virginity.
\switchcolumn*
\selectlanguage{latin}
\end{paracol}


% ---- martyrology/mart05/mart0516.htm
\needspace{10\baselineskip}
\begin{paracol}{2}
\selectlanguage{latin}
\begin{center}{\color{gregoriocolor} Décimo séptimo Kaléndas Júnii. 
 Luna\dots\ }\end{center}
\switchcolumn
\selectlanguage{english}
\begin{center}{\color{gregoriocolor} The Sixteenth Day of May. 
 The\dots\ Day of the Moon.}\end{center}
\end{paracol}

\noindent\begin{tabularx}{\linewidth}{*{19}{>{\centering\arraybackslash}X}}
 \textcolor{gregoriocolor}{a} & \textcolor{gregoriocolor}{b} & \textcolor{gregoriocolor}{c} & \textcolor{gregoriocolor}{d} & \textcolor{gregoriocolor}{e} & \textcolor{gregoriocolor}{f} & \textcolor{gregoriocolor}{g} & \textcolor{gregoriocolor}{h} & \textcolor{gregoriocolor}{i} & \textcolor{gregoriocolor}{k} & \textcolor{gregoriocolor}{l} & \textcolor{gregoriocolor}{m} & \textcolor{gregoriocolor}{n} & \textcolor{gregoriocolor}{p} & \textcolor{gregoriocolor}{q} & \textcolor{gregoriocolor}{r} & \textcolor{gregoriocolor}{s} & \textcolor{gregoriocolor}{t} & \textcolor{gregoriocolor}{u} \\
 19 & 20 & 21 & 22 & 23 & 24 & 25 & 26 & 27 & 28 & 29 & 30 & 1 & 2 & 3 & 4 & 5 & 6 & 7 \\
\end{tabularx}
\vspace{0.5\baselineskip}
\noindent\begin{tabularx}{\linewidth}{*{12}{>{\centering\arraybackslash}X}}
 \textcolor{gregoriocolor}{A} & \textcolor{gregoriocolor}{B} & \textcolor{gregoriocolor}{C} & \textcolor{gregoriocolor}{D} & \textcolor{gregoriocolor}{E} & F & \textcolor{gregoriocolor}{F} & \textcolor{gregoriocolor}{G} & \textcolor{gregoriocolor}{H} & \textcolor{gregoriocolor}{M} & \textcolor{gregoriocolor}{N} & \textcolor{gregoriocolor}{P} \\
 8 & 9 & 10 & 11 & 12 & 13 & 13 & 14 & 15 & 16 & 17 & 18 \\
\end{tabularx}

\begin{paracol}{2}
\selectlanguage{latin}
\lettrine[lines=2]{E}{ugúbii} sancti Ubáldi, 
 Epíscopi et Confessóris, miráculis clari.
\switchcolumn
\selectlanguage{english}
\lettrine[lines=2]{A}{t} Gubbio, St. Ubaldus, bishop and 
 confessor renowned for his miracles.
\switchcolumn*
\selectlanguage{latin}
Antisiodóri pássio 
 sancti Peregríni, qui fuit primus ejúsdem civitátis Epíscopus. Hic, a 
 beáto Xysto Papa Secúndo in Gállias, una cum áliis Cléricis, missus, ibi, 
 Evangélicæ prædicatiónis múnere expléto, capitáli senténtia damnátus corónam 
 méruit sempitérnam.
\switchcolumn
\selectlanguage{english}
At Auxerre, the passion of St. 
 Peregrinus, first bishop of that city. He was sent into France with 
 other clerics by the blessed Pope Sixtus II, and having accomplished his 
 work of preaching the Gospel, he was condemned to capital punishment, and 
 merited for himself an everlasting crown.
\switchcolumn*
\selectlanguage{latin}
In Pérside sanctórum 
 Mártyrum Audæ Epíscopi, septem Presbyterórum, novem Diaconórum, et septem 
 Vírginum; qui sub Isdegérde Rege, variis tormentórum genéribus cruciáti, 
 gloriósum martyrium complevérunt.
\switchcolumn
\selectlanguage{english}
In Persia, the holy martyrs Audas, 
 a bishop, seven priests, nine deacons and seven virgins, who endured various 
 kins of torments under King Isdegerdes, and thus gloriously completed their 
 martyrdom.
\switchcolumn*
\selectlanguage{latin}
Pragæ, in Bohémia, sancti Joánnis Nepomucéni, Metropolitánæ Ecclésiæ Canónici; qui, frustra 
 tentátus ut sigílli sacramentális fidem próderet, martyrii palmam, in flumen 
 Moldávam dejéctus, eméruit.
\switchcolumn
\selectlanguage{english}
At Prague in Bohemia, St. John 
 Nepomucene, a canon of the cathedral church, who, being tempted in vain to 
 betray the secret of confession, was cast into the River Moldau, and thus 
 won the palm of martyrdom.
\switchcolumn*
\selectlanguage{latin}
In Isáuria natális 
 sanctórum Mártyrum Aquilíni et Victoriáni.
\switchcolumn
\selectlanguage{english}
In Isauria, the birthday of the 
 holy martyrs Aquilinus and Victorian.
\switchcolumn*
\selectlanguage{latin}
Uzali, in Africa, 
 sanctórum Mártyrum Felícis et Gennádii.
\switchcolumn
\selectlanguage{english}
At Uzalis in Africa, the holy 
 martyrs Felix and Gennadius.
\switchcolumn*
\selectlanguage{latin}
In Palæstína pássio 
 sanctórum Monachórum, a Saracénis in laura sancti Sabbæ interfectórum.
\switchcolumn
\selectlanguage{english}
In Palestine, the martyrdom of the 
 holy monks massacred by the Saracens in the monastery of St. Sabbas.
\switchcolumn*
\selectlanguage{latin}
Janóviæ, apud Pinsk, 
 in Polésia, sancti Andréæ Bobóla, Sacerdótis e Societáte Jesu, qui, a 
 schismáticis innumerabília tormentórum génera perpéssus, illústri martyrio 
 coronátus est.
\switchcolumn
\selectlanguage{english}
At Janow, near Pinsk in Lithuania, 
 St. Andrew Bobola, priest of the Society of Jesus, who having suffered many 
 kinds of torments at the hands of the schismatics, was crowned with an 
 illustrious martyrdom.
\switchcolumn*
\selectlanguage{latin}
Ambiáni, in Gállia, 
 sancti Honoráti Epíscopi.
\switchcolumn
\selectlanguage{english}
At Amiens in France, St. Honoratus, 
 bishop.
\switchcolumn*
\selectlanguage{latin}
Apud Cenómanos, in 
 Gállia, sancti Dómnoli Epíscopi.
\switchcolumn
\selectlanguage{english}
At Le Mans in France, St. Domnolus, 
 bishop.
\switchcolumn*
\selectlanguage{latin}
Mirándulæ, in Æmília, 
 sancti Possídii, Calaménsis in Numídia Epíscopi, qui fuit sancti Augustíni 
 discípulus, ejúsque præcláram vitam scripsit.
\switchcolumn
\selectlanguage{english}
At Mirandola in Aemilia, St. 
 Possidius, bishop of Calamae, and disciple of St. Augustine, of whose 
 glorious life he wrote a history.
\switchcolumn*
\selectlanguage{latin}
In monastério 
 Enachduinénsi, in Hibérnia, tránsitus sancti Brendáni Presbyteri et 
 Clonferténsis Abbátis.
\switchcolumn
\selectlanguage{english}
In the monastery of Enachduin in 
 Ireland, the death of St. Brendan, abbot of Clonfert.
\switchcolumn*
\selectlanguage{latin}
Trecis, in Gállia, 
 sancti Fídoli Confessóris.
\switchcolumn
\selectlanguage{english}
At Treves in France, St. Fidolus, 
 confessor.
\switchcolumn*
\selectlanguage{latin}
Apud Forum Júlii, in 
 Gállia, sanctæ Máximæ Vírginis, quæ, multis clara virtútibus, in pace 
 quiévit.
\switchcolumn
\selectlanguage{english}
At Fréjus in France, St. Maxima, 
 virgin, who died in peace with a reputation for many virtues.
\switchcolumn*
\selectlanguage{latin}
\end{paracol}


% ---- martyrology/mart05/mart0517.htm
\needspace{10\baselineskip}
\begin{paracol}{2}
\selectlanguage{latin}
\begin{center}{\color{gregoriocolor} Sextodécimo Kaléndas Júnii. 
 Luna\dots\ }\end{center}
\switchcolumn
\selectlanguage{english}
\begin{center}{\color{gregoriocolor} The 
 Seventeenth Day of May. The\dots\ Day of the Moon.}\end{center}
\end{paracol}

\noindent\begin{tabularx}{\linewidth}{*{19}{>{\centering\arraybackslash}X}}
 \textcolor{gregoriocolor}{a} & \textcolor{gregoriocolor}{b} & \textcolor{gregoriocolor}{c} & \textcolor{gregoriocolor}{d} & \textcolor{gregoriocolor}{e} & \textcolor{gregoriocolor}{f} & \textcolor{gregoriocolor}{g} & \textcolor{gregoriocolor}{h} & \textcolor{gregoriocolor}{i} & \textcolor{gregoriocolor}{k} & \textcolor{gregoriocolor}{l} & \textcolor{gregoriocolor}{m} & \textcolor{gregoriocolor}{n} & \textcolor{gregoriocolor}{p} & \textcolor{gregoriocolor}{q} & \textcolor{gregoriocolor}{r} & \textcolor{gregoriocolor}{s} & \textcolor{gregoriocolor}{t} & \textcolor{gregoriocolor}{u} \\
 20 & 21 & 22 & 23 & 24 & 25 & 26 & 27 & 28 & 29 & 30 & 1 & 2 & 3 & 4 & 5 & 6 & 7 & 8 \\
\end{tabularx}
\vspace{0.5\baselineskip}
\noindent\begin{tabularx}{\linewidth}{*{12}{>{\centering\arraybackslash}X}}
 \textcolor{gregoriocolor}{A} & \textcolor{gregoriocolor}{B} & \textcolor{gregoriocolor}{C} & \textcolor{gregoriocolor}{D} & \textcolor{gregoriocolor}{E} & F & \textcolor{gregoriocolor}{F} & \textcolor{gregoriocolor}{G} & \textcolor{gregoriocolor}{H} & \textcolor{gregoriocolor}{M} & \textcolor{gregoriocolor}{N} & \textcolor{gregoriocolor}{P} \\
 8 & 10 & 11 & 12 & 13 & 14 & 14 & 15 & 16 & 17 & 18 & 19 \\
\end{tabularx}

\begin{paracol}{2}
\selectlanguage{latin}
\lettrine[lines=2]{A}{pud} Villam Regálem, 
 in Hispánia, sancti Paschális, ex Ordine Minórum, Confessóris, miræ 
 innocéntiæ et pæniténtiæ viri; quem Leo Papa Décimus tértius cæléstem 
 eucharisticórum Cœtuum et Societátum a sanctíssima Eucharístia Patrónum 
 declarávit.
\switchcolumn
\selectlanguage{english}
\lettrine[lines=2]{A}{t} Villareal in Spain, St. Paschal 
 of the Order of Friars Minor, confessor. He was a man remarkable for 
 innocence of life and the spirit of penance, whom Pope Leo XIII declared to 
 be the heavenly patron of Eucharistic Congresses and of societies formed to 
 honour the Most Blessed Sacrament.
\switchcolumn*
\selectlanguage{latin}
Noviodúni, in Gálliis, 
 sanctórum Mártyrum Herádii, Pauli et Aquilíni, cum duóbus áliis.
\switchcolumn
\selectlanguage{english}
At Noyon in France, the holy 
 martyrs Heradius, Paul, and Aquilinus, with two others.
\switchcolumn*
\selectlanguage{latin}
Chalcédone sanctórum 
 Mártyrum Solochónis et Sociórum mílitum, sub Maximiáno Imperatóre.
\switchcolumn
\selectlanguage{english}
At Chalcedon, the holy martyrs 
 Solochan and his companions.
\switchcolumn*
\selectlanguage{latin}
Alexandríæ sanctórum 
 Mártyrum Adriónis, Victóris et Basíllæ.
\switchcolumn
\selectlanguage{english}
At Alexandria, the holy martyrs 
 Adrion, Victor, and Basilla.
\switchcolumn*
\selectlanguage{latin}
Eódem die sanctæ 
 Restitútæ, Vírginis et Mártyris; quæ, in Africa, Valeriáno imperánte, a 
 Próculo Júdice várie torta, et in navículam pice et stupa refértam, ut in mari comburerétur, impósita, tandem, cum in incensóres, immísso igne, flamma 
 converterétur, in oratióne spíritum Deo réddidit. Ipsíus corpus cum 
 eádem navícula, Dei nutu, ad Ænáriam ínsulam, prope Neápolim, in Campánia, 
 devéctum est, et a Christiánis magna veneratióne suscéptum; ac póstmodum in 
 ejus honórem Constantínus Magnus Basílicam in ipsa urbe Neápoli erigéndam 
 curávit.
\switchcolumn
\selectlanguage{english}
Also St. Restituta, virgin and 
 martyr, who was subjected to various kinds of tortures in Africa by the 
 judge Proculus, in the reign of Valerian, and then put in a boat filled with 
 pitch and oakum, to be burned to death on the sea. But the flame 
 turned on those who kindled it, and the saint yielded her soul to God in 
 prayer. Her body was, by Divine Providence, carried in the boat to the 
 island of Ischia, near Naples, where it was received by the Christians with 
 great veneration. A church was afterwards erected in her honour at 
 Naples by Constantine the Great.
\switchcolumn*
\selectlanguage{latin}
\end{paracol}


% ---- martyrology/mart05/mart0518.htm
\needspace{10\baselineskip}
\begin{paracol}{2}
\selectlanguage{latin}
\begin{center}{\color{gregoriocolor} Quintodécimo Kaléndas Júnii. 
 Luna\dots\ }\end{center}
\switchcolumn
\selectlanguage{english}
\begin{center}{\color{gregoriocolor} The 
 Eighteenth Day of May. The\dots\ Day of the Moon.}\end{center}
\end{paracol}

\noindent\begin{tabularx}{\linewidth}{*{19}{>{\centering\arraybackslash}X}}
 \textcolor{gregoriocolor}{a} & \textcolor{gregoriocolor}{b} & \textcolor{gregoriocolor}{c} & \textcolor{gregoriocolor}{d} & \textcolor{gregoriocolor}{e} & \textcolor{gregoriocolor}{f} & \textcolor{gregoriocolor}{g} & \textcolor{gregoriocolor}{h} & \textcolor{gregoriocolor}{i} & \textcolor{gregoriocolor}{k} & \textcolor{gregoriocolor}{l} & \textcolor{gregoriocolor}{m} & \textcolor{gregoriocolor}{n} & \textcolor{gregoriocolor}{p} & \textcolor{gregoriocolor}{q} & \textcolor{gregoriocolor}{r} & \textcolor{gregoriocolor}{s} & \textcolor{gregoriocolor}{t} & \textcolor{gregoriocolor}{u} \\
 21 & 22 & 23 & 24 & 25 & 26 & 27 & 28 & 29 & 30 & 1 & 2 & 3 & 4 & 5 & 6 & 7 & 8 & 9 \\
\end{tabularx}
\vspace{0.5\baselineskip}
\noindent\begin{tabularx}{\linewidth}{*{12}{>{\centering\arraybackslash}X}}
 \textcolor{gregoriocolor}{A} & \textcolor{gregoriocolor}{B} & \textcolor{gregoriocolor}{C} & \textcolor{gregoriocolor}{D} & \textcolor{gregoriocolor}{E} & F & \textcolor{gregoriocolor}{F} & \textcolor{gregoriocolor}{G} & \textcolor{gregoriocolor}{H} & \textcolor{gregoriocolor}{M} & \textcolor{gregoriocolor}{N} & \textcolor{gregoriocolor}{P} \\
 10 & 11 & 12 & 13 & 14 & 15 & 15 & 16 & 17 & 18 & 19 & 20 \\
\end{tabularx}

\begin{paracol}{2}
\selectlanguage{latin}
\lettrine[lines=2]{C}{ameríni} sancti 
 Venántii Mártyris, qui, annos quíndecim natus, sub Décio Imperatóre et 
 Antíocho Præside, una cum áliis decem, gloriósi certáminis cursum, 
 cervícibus abscíssis implévit.
\switchcolumn
\selectlanguage{english}
\lettrine[lines=2]{A}{t} Camerino, the holy martyr Venantius, who, at fifteen years of age, along with ten others, ended a 
 glorious ordeal by being beheaded under Emperor Decius and the governor 
 Antiochus.
\switchcolumn*
\selectlanguage{latin}
Ravénnæ, natális 
 sancti Joánnis Primi, Papæ et Mártyris; qui, ab Ariáno Itáliæ Rege 
 Theodoríco illuc dolo evocátus, ibídem, ab eo propter orthodóxam fidem diu 
 macerátus in cárcere, vitam finívit. Ipsíus autem festum recólitur 
 sexto Kaléndas Júnii, quo die sacrum ejus corpus, Romam relátum, in Basílica sancti Petri, Apostolórum Príncipis, sepúltum est.
\switchcolumn
\selectlanguage{english}
The birthday of St. John I, pope 
 and martyr, who was called to Ravenna by the Arian king of Italy, Theodoric, 
 and died there after being in prison a long time for the true faith. 
 His feast, however, is celebrated on the 27th of May, the day on which his 
 revered body was taken to Rome and buried in the basilica of St. Peter, 
 prince of the apostles.
\switchcolumn*
\selectlanguage{latin}
Spoléti sancti 
 Felícis Epíscopi, qui, sub Maximiáno Imperatóre, martyrii palmam adéptus 
 est.
\switchcolumn
\selectlanguage{english}
At Spoleto, St. Felix, a bishop who 
 obtained the palm of martyrdom under Emperor Maximian.
\switchcolumn*
\selectlanguage{latin}
Heracléæ, in Ægypto, 
 sancti Potamónis Epíscopi, qui primum, sub Maximiáno Galério, Confessor fuit; 
 deínde, sub Constántio Imperatóre et Ariáno Prǽside Philágrio, martyrio 
 coronátus est. Ipsum vero beátum virum sancti Ecclésiæ Patres 
 Athanásius et Epiphánius suis láudibus celebrárunt.
\switchcolumn
\selectlanguage{english}
At Heraclea in Egypt, Bishop St. 
 Potamon, first a confessor under Maximian Galerius, and afterwards, a martyr 
 under Emperor Constantius, and the Arian governor Philagrius. 
 Athanasius and Epiphanius, Fathers of the Church, have sung the praises of 
 this holy man.
\switchcolumn*
\selectlanguage{latin}
In Ægypto sancti 
 Dióscori Lectóris, in quem Præses multa et vária torménta ita exércuit, ut 
 ungues ejus effóderet et lampádibus látera inflammáret, sed, cæléstis 
 lúminis fulgóre pertérriti, cecidérunt minístri; novíssime autem ipse 
 Dióscorus, láminis ardéntibus adústus, martyrium consummávit.
\switchcolumn
\selectlanguage{english}
In Egypt, St. Dioscorus, a lector, 
 who was subjected by the governor to many and diverse torments, such as the 
 tearing off of his nails and the burning of his sides with torches; but a 
 light from heaven having prostrated the executioners, the saint's martyrdom 
 was finally ended by having red-hot metal plates applied to his body.
\switchcolumn*
\selectlanguage{latin}
Ancyræ, in Galátia, sancti Theódoti Mártyris, et sanctárum septem Vírginum et Mártyrum, scílicet 
 Thecúsæ, quæ ipsíus Theódoti erat 
 ámita, Alexándræ, Cláudiæ, Phaínæ, 
 Euphrásiæ, Matrónæ et Julíttæ. Hæ a Prǽside primum prostitútæ, sed Dei 
 virtúte servátæ, deínde, lapídibus ad colla ligátis, in palúdem mersæ sunt; quarum relíquias colléctas cum Theódotus honorífice sepelísset, 
 ídeo, 
 comprehénsus a Prǽside, sævíssime dilaniátus est, ac demum, ense percússus, 
 martyrii corónam accépit.
\switchcolumn
\selectlanguage{english}
At Ancyra in Galatia, the martyr 
 St. Theodotus, and the holy virgins Thecusa, his aunt, Alexandra, Claudia, 
 Faina, Euphrasia, Matrona, and Julitta. They were at first taken to a 
 place of debauchery, but the power of God prevented them from evil, and they 
 later had stones fastened to their necks and were plunged into a lake. 
 For gathering the remains and burying them honorably, Theodotus was arrested 
 by the governor, and after having been horribly lacerated, was put to the 
 sword, and thus received the crown of martyrdom.
\switchcolumn*
\selectlanguage{latin}
Upsali, in Suécia, 
 sancti Eríci, Regis et Mártyris.
\switchcolumn
\selectlanguage{english}
At Upsal in Sweden, St. Erik, king 
 and martyr.
\switchcolumn*
\selectlanguage{latin}
Romæ sancti Felícis 
 Confessóris, ex Ordine Minórum Capuccinórum, evangélica simplicitáte et 
 caritáte conspícui; quem Clemens Undécimus, Póntifex Máximus, Sanctórum 
 fastis adscrípsit.
\switchcolumn
\selectlanguage{english}
At Rome, St. Felix, confessor of 
 the Order of Friars Minor Capuchin, celebrated for his evangelical 
 simplicity and charity. He was inscribed on the roll of the saints by 
 the Sovereign Pontiff Clement XI.
\switchcolumn*
\selectlanguage{latin}
\end{paracol}


% ---- martyrology/mart05/mart0519.htm
\needspace{10\baselineskip}
\begin{paracol}{2}
\selectlanguage{latin}
\begin{center}{\color{gregoriocolor} Quartodécimo Kaléndas Júnii. 
 Luna\dots\ }\end{center}
\switchcolumn
\selectlanguage{english}
\begin{center}{\color{gregoriocolor} The 
 Nineteenth Day of May. The\dots\ Day of the Moon.}\end{center}
\end{paracol}

\noindent\begin{tabularx}{\linewidth}{*{19}{>{\centering\arraybackslash}X}}
 \textcolor{gregoriocolor}{a} & \textcolor{gregoriocolor}{b} & \textcolor{gregoriocolor}{c} & \textcolor{gregoriocolor}{d} & \textcolor{gregoriocolor}{e} & \textcolor{gregoriocolor}{f} & \textcolor{gregoriocolor}{g} & \textcolor{gregoriocolor}{h} & \textcolor{gregoriocolor}{i} & \textcolor{gregoriocolor}{k} & \textcolor{gregoriocolor}{l} & \textcolor{gregoriocolor}{m} & \textcolor{gregoriocolor}{n} & \textcolor{gregoriocolor}{p} & \textcolor{gregoriocolor}{q} & \textcolor{gregoriocolor}{r} & \textcolor{gregoriocolor}{s} & \textcolor{gregoriocolor}{t} & \textcolor{gregoriocolor}{u} \\
 22 & 23 & 24 & 25 & 26 & 27 & 28 & 29 & 30 & 1 & 2 & 3 & 4 & 5 & 6 & 7 & 8 & 9 & 10 \\
\end{tabularx}
\vspace{0.5\baselineskip}
\noindent\begin{tabularx}{\linewidth}{*{12}{>{\centering\arraybackslash}X}}
 \textcolor{gregoriocolor}{A} & \textcolor{gregoriocolor}{B} & \textcolor{gregoriocolor}{C} & \textcolor{gregoriocolor}{D} & \textcolor{gregoriocolor}{E} & F & \textcolor{gregoriocolor}{F} & \textcolor{gregoriocolor}{G} & \textcolor{gregoriocolor}{H} & \textcolor{gregoriocolor}{M} & \textcolor{gregoriocolor}{N} & \textcolor{gregoriocolor}{P} \\
 11 & 12 & 13 & 14 & 15 & 16 & 16 & 17 & 18 & 19 & 20 & 21 \\
\end{tabularx}

\begin{paracol}{2}
\selectlanguage{latin}
\lettrine[lines=2]{N}{atalis} sancti Petri 
 de Moróno, Confessóris, qui ex Anachoréta Summus Póntifex creátus, dictus 
 est Cælestínus Quintus. Sed Pontificátu se póstmodum abdicávit, et in 
 solitúdine religiósam vitam agens, virtútibus et miráculis clarus, migrávit 
 ad Dóminum.
\switchcolumn
\selectlanguage{english}
\lettrine[lines=2]{T}{he} birthday of St. Peter of Moroni 
 who, while leading the life of an anchoret, was created Sovereign Pontiff 
 and called Celestine V. He later abdicated the pontificate, and led a 
 religious life in solitude, where, renowned for virtues and miracles, he 
 went to the Lord.
\switchcolumn*
\selectlanguage{latin}
Romæ sanctæ Pudentiánæ 
 Vírginis, quæ, post innúmeros agónes, post multórum Mártyrum venerabíliter 
 exhíbitas sepultúras, post omnes facultátes suas pro Christo paupéribus 
 erogátas, tandem de terris ad cælos migrávit.
\switchcolumn
\selectlanguage{english}
At Rome, the saintly virgin 
 Pudentiana, who, after numberless tribulations, after burying with respect 
 many martyrs, and distributing all her goods to the poor for Christ's sake, 
 departed from this world to go to heaven.
\switchcolumn*
\selectlanguage{latin}
Ibídem sancti Pudéntis 
 Senatóris, qui ejúsdem sanctæ Pudentiánæ Vírginis ac sanctæ item Praxédis 
 Vírginis fuit pater; atque, ab Apostólicis Christo in baptísmo vestítus, 
 innocéntem túnicam usque ad vitæ corónam immaculáte custodívit.
\switchcolumn
\selectlanguage{english}
In the same city, St. Pudens, 
 senator, father of the virgins Pudentiana and Praxedes. He was clothed 
 with Christ in baptism by the apostles, and preserved the robe of innocence 
 unspotted until he received the crown of life.
\switchcolumn*
\selectlanguage{latin}
Item Romæ, via Appia, 
 natális sanctórum Calóceri et Parthénii eunuchórum; quorum prior, cum esset 
 præpósitus cubículi uxóris Décii Imperatóris, postérior vero altérius 
 múneris primicérius, et ambo nollent idólis sacrificáre, idcírco, ejúsdem 
 Imperatóris jussu váriis ac diris sunt excruciáti supplíciis, ac tandem, 
 cervícibus ardénti fuste contúsis, spíritum Deo tradidérunt.
\switchcolumn
\selectlanguage{english}
Also at Rome, on the Appian Way, 
 the birthday of the Saints Calocerus and Parthenius, eunuchs. The 
 former was chamberlain of the wife of Emperor Decius, and the latter chief 
 officer in another department. Because they refused to offer sacrifice 
 to idols they were tortured in many cruel ways, and finally when their necks 
 were broken with cudgels, they gave up their souls to God.
\switchcolumn*
\selectlanguage{latin}
Nicomedíæ sancti 
 Philóteri Mártyris, qui Paciáni Procónsulis fílius 
 éxstitit, et, sub 
 Diocletiáno Imperatóre, multa passus, martyrii corónam accépit.
\switchcolumn
\selectlanguage{english}
At Nicomedia, the martyr St. 
 Philoterus, son of the proconsul Pacian, who after suffering much under 
 Emperor Diocletian, received the crown of martyrdom.
\switchcolumn*
\selectlanguage{latin}
Ibídem sanctárum sex 
 Vírginum et Mártyrum, quarum præcípua erat Cyríaca, quæ, cum líbere 
 Maximiánum impietátis objurgásset, diríssime cæsa et dilaniáta est, atque ad 
 últimum, igne cremáta, martyrium consummávit.
\switchcolumn
\selectlanguage{english}
In the same city, six holy virgins 
 and martyrs. The principal one, named Cyriaca, having boldly reproved 
 Maximian for his impiety, was severely scourged and lacerated, and then 
 consumed by fire.
\switchcolumn*
\selectlanguage{latin}
Cantuáriæ, in Anglia, 
 sancti Dunstáni Epíscopi.
\switchcolumn
\selectlanguage{english}
At Canterbury in England, St. 
 Dunstan, bishop.
\switchcolumn*
\selectlanguage{latin}
Lohanéti, in Británnia minóre, sancti Ivónis Presbyteri et Confessóris; qui, pro Christi amóre, 
 causas pupillórum, viduárum ac páuperum defendébat.
\switchcolumn
\selectlanguage{english}
In Brittany, St. Ivo, priest and 
 confessor, who for the love of Christ, defended the interests of orphans, 
 widows and the poor.
\switchcolumn*
\selectlanguage{latin}
Ficécli, in Etrúria, 
 sancti Theóphili a Curte, Confessóris, Sacerdótis Ordinis Fratrum Minórum, 
 sacrórum recéssum propagatóris, quem Pius Papa Undécimus inter Sanctos 
 rétulit.
\switchcolumn
\selectlanguage{english}
At Fucecchio in Etruria, St. 
 Theophilus of Curte, confessor and priest of the Order of Friars Minor, who 
 was canonized by Pope Pius XI.
\switchcolumn*
\selectlanguage{latin}
\end{paracol}


% ---- martyrology/mart05/mart0520.htm
\needspace{10\baselineskip}
\begin{paracol}{2}
\selectlanguage{latin}
\begin{center}{\color{gregoriocolor} Tertiodécimo Kaléndas Júnii. 
 Luna\dots\ }\end{center}
\switchcolumn
\selectlanguage{english}
\begin{center}{\color{gregoriocolor} The 
 Twentieth Day of May. The\dots\ Day of the Moon.}\end{center}
\end{paracol}

\noindent\begin{tabularx}{\linewidth}{*{19}{>{\centering\arraybackslash}X}}
 \textcolor{gregoriocolor}{a} & \textcolor{gregoriocolor}{b} & \textcolor{gregoriocolor}{c} & \textcolor{gregoriocolor}{d} & \textcolor{gregoriocolor}{e} & \textcolor{gregoriocolor}{f} & \textcolor{gregoriocolor}{g} & \textcolor{gregoriocolor}{h} & \textcolor{gregoriocolor}{i} & \textcolor{gregoriocolor}{k} & \textcolor{gregoriocolor}{l} & \textcolor{gregoriocolor}{m} & \textcolor{gregoriocolor}{n} & \textcolor{gregoriocolor}{p} & \textcolor{gregoriocolor}{q} & \textcolor{gregoriocolor}{r} & \textcolor{gregoriocolor}{s} & \textcolor{gregoriocolor}{t} & \textcolor{gregoriocolor}{u} \\
 23 & 24 & 25 & 26 & 27 & 28 & 29 & 30 & 1 & 2 & 3 & 4 & 5 & 6 & 7 & 8 & 9 & 10 & 11 \\
\end{tabularx}
\vspace{0.5\baselineskip}
\noindent\begin{tabularx}{\linewidth}{*{12}{>{\centering\arraybackslash}X}}
 \textcolor{gregoriocolor}{A} & \textcolor{gregoriocolor}{B} & \textcolor{gregoriocolor}{C} & \textcolor{gregoriocolor}{D} & \textcolor{gregoriocolor}{E} & F & \textcolor{gregoriocolor}{F} & \textcolor{gregoriocolor}{G} & \textcolor{gregoriocolor}{H} & \textcolor{gregoriocolor}{M} & \textcolor{gregoriocolor}{N} & \textcolor{gregoriocolor}{P} \\
 12 & 13 & 14 & 15 & 16 & 17 & 17 & 18 & 19 & 20 & 21 & 22 \\
\end{tabularx}

\begin{paracol}{2}
\selectlanguage{latin}
\lettrine[lines=2]{A}{quilæ,} in Vestínis, 
 sancti Bernardíni Senénsis, Sacerdótis ex Ordine Minórum et Confessóris, qui 
 verbo et exémplo Itáliam illustrávit.
\switchcolumn
\selectlanguage{english}
\lettrine[lines=2]{A}{t} Aquila in Abruzzi, St. Bernardin 
 of Siena, priest of the Order of Friars Minor, who added to the glory of 
 Italy by his preaching and his example.
\switchcolumn*
\selectlanguage{latin}
Romæ sanctæ Plautíllæ, 
 féminæ Consuláris, quæ beatórum Mártyrum Flávii Cleméntis Cónsulis soror et 
 Fláviæ Domitíllæ Vírginis mater 
 éxstitit; atque, a sancto Petro Apóstolo 
 baptizáta, ómnium virtútum laude refúlgens, quiévit in pace.
\switchcolumn
\selectlanguage{english}
At Rome, St. Plautilla, wife of a 
 consul, sister of the consul Flavius Clemens, and mother of the holy virgin 
 Flavia Domitilla, both martyrs. She was baptized by the apostle St. 
 Peter, and after giving an example of all the virtues, she rested in peace.
\switchcolumn*
\selectlanguage{latin}
Item Romæ, via Salária, 
 natális sanctæ Basíllæ Vírginis, quæ, cum esset ex régio génere, et 
 illustríssimum sponsum habéret illúmque dimisísset, accusáta fuit ab eo quod 
 esset Christiána, et mox decrétum est a Galliéno Augústo ut aut sponsum 
 recíperet, aut gládio interíret; cumque ipsa Virgo, convénta de hoc, 
 respondísset se Regem regum habére sponsum, gládio transverberáta est.
\switchcolumn
\selectlanguage{english}
Also at Rome, on the Salarian Way, 
 the birthday of St. Basilla, virgin, who was of a royal family and betrothed 
 to a nobleman. When she refused to marry him, he accused her of being 
 a Christian. Emperor Gallienus gave orders that she should accept the 
 person to whom she had been engaged, or die by the sword. Being 
 informed of this, and answering that she had for her spouse the King of 
 kings, she was pierced with a sword.
\switchcolumn*
\selectlanguage{latin}
Nemáusi, in Gálliis, 
 sancti Baudélii Mártyris, qui comprehénsus est a Pagánis, et cum sacrificáre 
 nollet idólis et in Christi fide inter vérbera et torménta immóbilis 
 persísteret, martyrii palmam pretiósa morte suscépit.
\switchcolumn
\selectlanguage{english}
At Nimes in France, St. Baudelius, 
 martyr. Being arrested, but refusing to sacrifice to idols, and 
 remaining immovable in the faith of Christ, despite blows and tortures, he 
 gained the palm of martyrdom by his praiseworthy death.
\switchcolumn*
\selectlanguage{latin}
Edéssæ, apud Ægas, in 
 Cilícia, sanctórum Mártyrum Thalelǽi, Astérii, Alexándri et Sociórum, qui 
 sub Numeriáno Imperatóre passi sunt.
\switchcolumn
\selectlanguage{english}
At Edessa near Aegea in Cilicia, 
 the holy martyrs Thalalaeus, Asterius, Alexander, and their companions, who 
 suffered under Emperor Numerian.
\switchcolumn*
\selectlanguage{latin}
In Thebáide sancti 
 Aquilæ Mártyris, qui pectínibus pro Christo dilaniátus fuit.
\switchcolumn
\selectlanguage{english}
In Thebais, St. Aquila, martyr to 
 the faith, whose body was torn with iron combs.
\switchcolumn*
\selectlanguage{latin}
Apud Bitúrcias, in 
 Aquitánia, sancti Austregísili, Epíscopi et Confessóris.
\switchcolumn
\selectlanguage{english}
At Bourges in France, St. 
 Austregisil, bishop and confessor.
\switchcolumn*
\selectlanguage{latin}
Bríxiæ sancti 
 Anastásii Epíscopi.
\switchcolumn
\selectlanguage{english}
At Brescia, St. Anastasius, bishop.
\switchcolumn*
\selectlanguage{latin}
Papíæ sancti Theodóri 
 Epíscopi.
\switchcolumn
\selectlanguage{english}
At Pavia, St. Theodore, bishop.
\switchcolumn*
\selectlanguage{latin}
\end{paracol}


% ---- martyrology/mart05/mart0521.htm
\needspace{10\baselineskip}
\begin{paracol}{2}
\selectlanguage{latin}
\begin{center}{\color{gregoriocolor} Duodécimo Kaléndas Júnii. 
 Luna\dots\ }\end{center}
\switchcolumn
\selectlanguage{english}
\begin{center}{\color{gregoriocolor} The 
 Twenty-First Day of May. The\dots\ Day of the Moon.}\end{center}
\end{paracol}

\noindent\begin{tabularx}{\linewidth}{*{19}{>{\centering\arraybackslash}X}}
 \textcolor{gregoriocolor}{a} & \textcolor{gregoriocolor}{b} & \textcolor{gregoriocolor}{c} & \textcolor{gregoriocolor}{d} & \textcolor{gregoriocolor}{e} & \textcolor{gregoriocolor}{f} & \textcolor{gregoriocolor}{g} & \textcolor{gregoriocolor}{h} & \textcolor{gregoriocolor}{i} & \textcolor{gregoriocolor}{k} & \textcolor{gregoriocolor}{l} & \textcolor{gregoriocolor}{m} & \textcolor{gregoriocolor}{n} & \textcolor{gregoriocolor}{p} & \textcolor{gregoriocolor}{q} & \textcolor{gregoriocolor}{r} & \textcolor{gregoriocolor}{s} & \textcolor{gregoriocolor}{t} & \textcolor{gregoriocolor}{u} \\
 24 & 25 & 26 & 27 & 28 & 29 & 30 & 1 & 2 & 3 & 4 & 5 & 6 & 7 & 8 & 9 & 10 & 11 & 12 \\
\end{tabularx}
\vspace{0.5\baselineskip}
\noindent\begin{tabularx}{\linewidth}{*{12}{>{\centering\arraybackslash}X}}
 \textcolor{gregoriocolor}{A} & \textcolor{gregoriocolor}{B} & \textcolor{gregoriocolor}{C} & \textcolor{gregoriocolor}{D} & \textcolor{gregoriocolor}{E} & F & \textcolor{gregoriocolor}{F} & \textcolor{gregoriocolor}{G} & \textcolor{gregoriocolor}{H} & \textcolor{gregoriocolor}{M} & \textcolor{gregoriocolor}{N} & \textcolor{gregoriocolor}{P} \\
 13 & 14 & 15 & 16 & 17 & 18 & 18 & 19 & 20 & 21 & 22 & 23 \\
\end{tabularx}

\begin{paracol}{2}
\selectlanguage{latin}
\lettrine[lines=2]{S}{ancti} Valéntis 
 Epíscopi, qui, una cum tribus púeris, necátus est.
\switchcolumn
\selectlanguage{english}
\lettrine[lines=2]{S}{t.} Valens, bishop, who was put to 
 death along with three children.
\switchcolumn*
\selectlanguage{latin}
Alexandríæ 
 commemorátio sanctórum Mártyrum Secúndi Presbyteri, et aliórum; quos sacris 
 diébus Pentecóstes, sub Constántio Imperatóre, Ariánus Epíscopus Geórgius 
 sævíssime occídi præcépit.
\switchcolumn
\selectlanguage{english}
At Alexandria, the commemoration of 
 the holy martyrs Secundus, a priest, and others, whom the Arian bishop 
 George ordered to be barbarously slain during the holy days of Pentecost, 
 under Emperor Constantius.
\switchcolumn*
\selectlanguage{latin}
In Mauritánia 
 Cæsariénsi natális sanctórum Mártyrum Diaconórum Timóthei, Pólii et Eutychii, 
 qui, in eádem regióne dísseminántes verbum Dei, páriter coronári meruérunt.
\switchcolumn
\selectlanguage{english}
In Morocco, the birthday of the 
 holy martyrs Timothy, Polius, and Eutychius, deacons, who merited to be 
 crowned together for spreading the word of God in that region.
\switchcolumn*
\selectlanguage{latin}
Cæsaréæ, in Cappadócia, 
 item natális sanctórum Mártyrum Polyéucti, Victórii et Donáti.
\switchcolumn
\selectlanguage{english}
At Caesarea in Cappadocia, the 
 birthday of the holy martyrs Polyeuctus, Victorinus, and Donatus.
\switchcolumn*
\selectlanguage{latin}
Córdubæ, in Hispánia, 
 sancti Secundíni Mártyris.
\switchcolumn
\selectlanguage{english}
At Cordova, the martyr St. 
 Secundinus.
\switchcolumn*
\selectlanguage{latin}
Eódem die sanctórum 
 Mártyrum Synésii et Theopómpi.
\switchcolumn
\selectlanguage{english}
The same day, the holy martyrs 
 Synesius and Theopompus.
\switchcolumn*
\selectlanguage{latin}
Cæsaréæ Philíppi 
 natális sanctórum Mártyrum Nicóstrati et Antíochi Tribunórum, cum áliis 
 milítibus.
\switchcolumn
\selectlanguage{english}
At Caesarea Philippi, the holy 
 martyrs Nicostratus and Antiochus, tribunes, with other soldiers.
\switchcolumn*
\selectlanguage{latin}
Alexandríæ 
 commemorátio sanctórum Episcopórum et Presbyterórum; qui, ab Ariánis exsílio 
 relegáti, sanctis Confessóribus sociári meruérunt.
\switchcolumn
\selectlanguage{english}
At Alexandria, the commemoration of 
 the saintly bishops and priests, who were banished by the Arians, and 
 merited to be numbered among the holy confessors.
\switchcolumn*
\selectlanguage{latin}
Níciæ, apud Varum 
 flúvium, sancti Hospítii Confessóris, abstinéntiæ virtúte ac prophetíæ 
 spíritu insígnis.
\switchcolumn
\selectlanguage{english}
At Nice in France, St. Hospitius, 
 confessor, distinguished by the virtue of abstinence and the spirit of 
 prophecy.
\switchcolumn*
\selectlanguage{latin}
\end{paracol}


% ---- martyrology/mart05/mart0522.htm
\needspace{10\baselineskip}
\begin{paracol}{2}
\selectlanguage{latin}
\begin{center}{\color{gregoriocolor} Undécimo Kaléndas Júnii. 
 Luna\dots\ }\end{center}
\switchcolumn
\selectlanguage{english}
\begin{center}{\color{gregoriocolor} The 
 Twenty-Second Day of May. The\dots\ Day of the Moon.}\end{center}
\end{paracol}

\noindent\begin{tabularx}{\linewidth}{*{19}{>{\centering\arraybackslash}X}}
 \textcolor{gregoriocolor}{a} & \textcolor{gregoriocolor}{b} & \textcolor{gregoriocolor}{c} & \textcolor{gregoriocolor}{d} & \textcolor{gregoriocolor}{e} & \textcolor{gregoriocolor}{f} & \textcolor{gregoriocolor}{g} & \textcolor{gregoriocolor}{h} & \textcolor{gregoriocolor}{i} & \textcolor{gregoriocolor}{k} & \textcolor{gregoriocolor}{l} & \textcolor{gregoriocolor}{m} & \textcolor{gregoriocolor}{n} & \textcolor{gregoriocolor}{p} & \textcolor{gregoriocolor}{q} & \textcolor{gregoriocolor}{r} & \textcolor{gregoriocolor}{s} & \textcolor{gregoriocolor}{t} & \textcolor{gregoriocolor}{u} \\
 25 & 26 & 27 & 28 & 29 & 30 & 30 & 1 & 2 & 3 & 4 & 5 & 6 & 7 & 8 & 10 & 11 & 12 & 13 \\
\end{tabularx}
\vspace{0.5\baselineskip}
\noindent\begin{tabularx}{\linewidth}{*{12}{>{\centering\arraybackslash}X}}
 \textcolor{gregoriocolor}{A} & \textcolor{gregoriocolor}{B} & \textcolor{gregoriocolor}{C} & \textcolor{gregoriocolor}{D} & \textcolor{gregoriocolor}{E} & F & \textcolor{gregoriocolor}{F} & \textcolor{gregoriocolor}{G} & \textcolor{gregoriocolor}{H} & \textcolor{gregoriocolor}{M} & \textcolor{gregoriocolor}{N} & \textcolor{gregoriocolor}{P} \\
 14 & 15 & 16 & 17 & 18 & 19 & 19 & 20 & 21 & 22 & 23 & 24 \\
\end{tabularx}

\begin{paracol}{2}
\selectlanguage{latin}
\lettrine[lines=2]{R}{omæ} sanctórum 
 Mártyrum Faustíni, Timóthei et Venústi.
\switchcolumn
\selectlanguage{english}
\lettrine[lines=2]{A}{t} Rome, the holy martyrs Faustinus, 
 Timothy, and Venustus.
\switchcolumn*
\selectlanguage{latin}
In Africa sanctórum 
 Mártyrum Casti et Æmílii, qui per passiónis ignem martyrium consummárunt. 
 Hos (ut beátus Cypriánus scribit), in prima congressióne devíctos, Dóminus 
 victóres in secúndo prælio réddidit, ut fortióres ígnibus fíerent qui 
 ígnibus ante cessíssent.
\switchcolumn
\selectlanguage{english}
In Africa, the holy martyrs Castus 
 and Aemilius, who met their martyrdom by fire, St. Cyprian says that there 
 were overcome by the first trial, but that in the second God made them 
 victorious, so that those who had first weakened in the face of the fire 
 were made mightier than the flames.
\switchcolumn*
\selectlanguage{latin}
Cománæ, in Ponto, 
 sancti Basilísci Mártyris, qui, sub Maximiáno Imperatóre et Agríppa Prǽside, 
 férreas calceátus crépidas, ignítis clavis confíxas, múltaque ália passus, 
 demum, cápite obtruncátus et in flumen projéctus, martyrii glóriam 
 consecútus est.
\switchcolumn
\selectlanguage{english}
At Comana in Pontus, under Emperor 
 Maximian and the governor Agrippa, the holy martyr Basiliscus, who was 
 forced to wear iron shoes pierced with heated nails, and who endured many 
 other trials. He was finally beheaded and thrown into the river, which 
 gained for him the crown of martyrdom.
\switchcolumn*
\selectlanguage{latin}
In Córsica sanctæ 
 Júliæ Vírginis, quæ crucis supplício coronáta est.
\switchcolumn
\selectlanguage{english}
In Corsica, St. Julia, virgin, who 
 won her crown by being crucified.
\switchcolumn*
\selectlanguage{latin}
In Hispánia sanctæ 
 Quitériæ, Vírginis et Mártyris.
\switchcolumn
\selectlanguage{english}
In Spain, St. Quiteria, virgin and 
 martyr.
\switchcolumn*
\selectlanguage{latin}
Ravénnæ sancti 
 Marciáni, Epíscopi et Confessóris.
\switchcolumn
\selectlanguage{english}
At Ravenna, St. Marcian, bishop and 
 confessor.
\switchcolumn*
\selectlanguage{latin}
Pistórii, in Túscia, 
 beáti Atthónis Epíscopi, ex Ordine Vallis Umbrósæ.
\switchcolumn
\selectlanguage{english}
At Pistoia in Tuscany, the bishop, 
 blessed Attho, of the Order of Vallombrosa.
\switchcolumn*
\selectlanguage{latin}
In pago Antisiodorénsi 
 beáti Románi Abbátis, qui sancto Benedícto ministrávit in specu; inde, in 
 Gállias proféctus, ibi, ædificáto monastério relictísque multis sanctitátis 
 alúmnis, quiévit in Dómino.
\switchcolumn
\selectlanguage{english}
In the diocese of Auxerre, Abbot 
 St. Romanus, who ministered to St. Benedict in his cave. Going later 
 to France, he built a monastery there, and leaving many disciples and 
 imitators of his sanctity, went to rest in the Lord.
\switchcolumn*
\selectlanguage{latin}
Apud Aquínum sancti 
 Fulci Confessóris.
\switchcolumn
\selectlanguage{english}
At Aquino, St. Fulk, confessor.
\switchcolumn*
\selectlanguage{latin}
Antisiodóri sanctæ 
 Hélenæ Vírginis.
\switchcolumn
\selectlanguage{english}
At Auxerre, St. Helen, virgin.
\switchcolumn*
\selectlanguage{latin}
Cássiæ, in Umbria, 
 sanctæ Ritæ Víduæ, Moniális ex Ordine Eremitárum sancti Augustíni; quæ, post 
 sæculi núptias, ætérnum sponsum Christum únice diléxit.
\switchcolumn
\selectlanguage{english}
At Cascia in Umbria, St. Rita, a 
 widow and nun of the Order of the Hermits of St. Augustine, who, after being 
 disengaged from her earthly marriage, loved only her eternal spouse Christ.
\switchcolumn*
\selectlanguage{latin}
\end{paracol}


% ---- martyrology/mart05/mart0523.htm
\needspace{10\baselineskip}
\begin{paracol}{2}
\selectlanguage{latin}
\begin{center}{\color{gregoriocolor} Décimo Kaléndas Júnii. 
 Luna\dots\ }\end{center}
\switchcolumn
\selectlanguage{english}
\begin{center}{\color{gregoriocolor} The 
 Twenty-Third Day of May. The\dots\ Day of the Moon.}\end{center}
\end{paracol}

\noindent\begin{tabularx}{\linewidth}{*{19}{>{\centering\arraybackslash}X}}
 \textcolor{gregoriocolor}{a} & \textcolor{gregoriocolor}{b} & \textcolor{gregoriocolor}{c} & \textcolor{gregoriocolor}{d} & \textcolor{gregoriocolor}{e} & \textcolor{gregoriocolor}{f} & \textcolor{gregoriocolor}{g} & \textcolor{gregoriocolor}{h} & \textcolor{gregoriocolor}{i} & \textcolor{gregoriocolor}{k} & \textcolor{gregoriocolor}{l} & \textcolor{gregoriocolor}{m} & \textcolor{gregoriocolor}{n} & \textcolor{gregoriocolor}{p} & \textcolor{gregoriocolor}{q} & \textcolor{gregoriocolor}{r} & \textcolor{gregoriocolor}{s} & \textcolor{gregoriocolor}{t} & \textcolor{gregoriocolor}{u} \\
 26 & 27 & 28 & 29 & 30 & 1 & 2 & 3 & 4 & 5 & 6 & 7 & 8 & 9 & 10 & 11 & 12 & 13 & 14 \\
\end{tabularx}
\vspace{0.5\baselineskip}
\noindent\begin{tabularx}{\linewidth}{*{12}{>{\centering\arraybackslash}X}}
 \textcolor{gregoriocolor}{A} & \textcolor{gregoriocolor}{B} & \textcolor{gregoriocolor}{C} & \textcolor{gregoriocolor}{D} & \textcolor{gregoriocolor}{E} & F & \textcolor{gregoriocolor}{F} & \textcolor{gregoriocolor}{G} & \textcolor{gregoriocolor}{H} & \textcolor{gregoriocolor}{M} & \textcolor{gregoriocolor}{N} & \textcolor{gregoriocolor}{P} \\
 15 & 16 & 17 & 18 & 19 & 20 & 20 & 21 & 22 & 23 & 24 & 25 \\
\end{tabularx}

\begin{paracol}{2}
\selectlanguage{latin}
\lettrine[lines=2]{A}{pud} Língonas, in 
 Gállia, pássio sancti Desidérii Epíscopi, qui, cum plebem suam ab exércitu 
 Wandalórum vexári cérneret, ad Regem eórum pro ea supplicatúrus accéssit; a 
 quo statim jugulári jussus, pro óvibus sibi créditis cervícem libénter 
 exténdit, et, percússus gládio, migrávit ad Christum. Passi sunt autem 
 cum ipso et álii plures de número gregis sui, qui apud eándem urbem cónditi 
 sunt.
\switchcolumn
\selectlanguage{english}
\lettrine[lines=2]{A}{t} Langres in France, the martyrdom 
 of the holy bishop Desiderius, who visited the king to offer entreaties in 
 behalf of his people who were mistreated by the Vandal army. He was 
 immediately condemned to beheading, and willingly presenting his head to 
 receive the blow of the sword, he died for the sheep committed to his charge 
 and departed for heaven. With him suffered many of his flock, who are 
 buried in the same city.
\switchcolumn*
\selectlanguage{latin}
In Hispánia sanctórum 
 Mártyrum Epitácii Epíscopi, et Basiléi.
\switchcolumn
\selectlanguage{english}
In Spain, the holy martyrs 
 Epitacius, a bishop, and Basileus.
\switchcolumn*
\selectlanguage{latin}
In território 
 Lugdunénsi sancti Desidérii, Epíscopi Viennénsis; qui, Theodoríci Regis 
 jussu lapídibus óbrutus, martyrio coronátur.
\switchcolumn
\selectlanguage{english}
In the territory of Lyons, St. 
 Desiderius, bishop of Vienne, who was crowned with martyrdom by being 
 stoned at the order of King Theodoric.
\switchcolumn*
\selectlanguage{latin}
In Africa sanctórum 
 Mártyrum Quinctiáni, Lúcii et Juliáni, qui, in persecutióne Wandálica passi, 
 ætérnas corónas meruérunt.
\switchcolumn
\selectlanguage{english}
In Africa, the holy martyrs 
 Quintian, Lucius, and Julian, who merited eternal crowns by their 
 sufferings, during the persecution of the Vandals.
\switchcolumn*
\selectlanguage{latin}
In Cappadócia 
 commemorátio sanctórum Mártyrum, qui, in persecutióne Maximiáni Galérii, 
 confráctis crúribus, necáti sunt; item eórum, qui eódem témpore, in 
 Mesopotámia, appénsi pédibus in sublíme, cápite verso deórsum, suffocáti 
 fumo et lento igne consúmpti, martyrium complevérunt.
\switchcolumn
\selectlanguage{english}
In Cappadocia, the commemoration of 
 the holy martyrs who died by having their legs crushed, in the persecution 
 of Maximian Galerius. Also in Mesopotamia, those martyrs who, at the 
 same time, were suspended in the air with their heads downward, suffocated 
 with smoke, and consumed by a slow fire, thus fulfilling their martyrdom.
\switchcolumn*
\selectlanguage{latin}
Synnadæ, in Phrygia, 
 sancti Michaélis Epíscopi.
\switchcolumn
\selectlanguage{english}
At Synnada in Phrygia, St. Michael, 
 bishop.
\switchcolumn*
\selectlanguage{latin}
Ipso die sancti 
 Mercuriális Epíscopi.
\switchcolumn
\selectlanguage{english}
The same day, St. Mercurialis, 
 bishop.
\switchcolumn*
\selectlanguage{latin}
Neápoli, in Campánia, 
 sancti Euphébii Epíscopi.
\switchcolumn
\selectlanguage{english}
At Naples in Campania, St. 
 Euphebius, bishop.
\switchcolumn*
\selectlanguage{latin}
Romæ sancti Joánnis 
 Baptístæ de Rossi, Presbyteri et Confessóris, patiéntia et caritáte in 
 evangelizándis paupéribus insígnis.
\switchcolumn
\selectlanguage{english}
At Rome, St. John Baptist de Rossi, 
 priest and confessor, a man illustrious for his patience and his zeal in 
 preaching the Gospel to the poor.
\switchcolumn*
\selectlanguage{latin}
Apud Núrsiam sanctórum 
 Eutychii et Floréntii Monachórum, quorum méminit beátus Gregórius Papa.
\switchcolumn
\selectlanguage{english}
At Norcia, Saints Eutychius and 
 Florentius, monks, mentioned by the blessed Pope Gregory.
\switchcolumn*
\selectlanguage{latin}
\end{paracol}


% ---- martyrology/mart05/mart0524.htm
\needspace{10\baselineskip}
\begin{paracol}{2}
\selectlanguage{latin}
\begin{center}{\color{gregoriocolor} Nono Kaléndas Júnii. 
 Luna\dots\ }\end{center}
\switchcolumn
\selectlanguage{english}
\begin{center}{\color{gregoriocolor} The 
 Twenty-Fourth Day of May. The\dots\ Day of the Moon.}\end{center}
\end{paracol}

\noindent\begin{tabularx}{\linewidth}{*{19}{>{\centering\arraybackslash}X}}
 \textcolor{gregoriocolor}{a} & \textcolor{gregoriocolor}{b} & \textcolor{gregoriocolor}{c} & \textcolor{gregoriocolor}{d} & \textcolor{gregoriocolor}{e} & \textcolor{gregoriocolor}{f} & \textcolor{gregoriocolor}{g} & \textcolor{gregoriocolor}{h} & \textcolor{gregoriocolor}{i} & \textcolor{gregoriocolor}{k} & \textcolor{gregoriocolor}{l} & \textcolor{gregoriocolor}{m} & \textcolor{gregoriocolor}{n} & \textcolor{gregoriocolor}{p} & \textcolor{gregoriocolor}{q} & \textcolor{gregoriocolor}{r} & \textcolor{gregoriocolor}{s} & \textcolor{gregoriocolor}{t} & \textcolor{gregoriocolor}{u} \\
 27 & 28 & 29 & 30 & 1 & 2 & 3 & 4 & 5 & 6 & 7 & 8 & 9 & 10 & 11 & 12 & 13 & 14 & 15 \\
\end{tabularx}
\vspace{0.5\baselineskip}
\noindent\begin{tabularx}{\linewidth}{*{12}{>{\centering\arraybackslash}X}}
 \textcolor{gregoriocolor}{A} & \textcolor{gregoriocolor}{B} & \textcolor{gregoriocolor}{C} & \textcolor{gregoriocolor}{D} & \textcolor{gregoriocolor}{E} & F & \textcolor{gregoriocolor}{F} & \textcolor{gregoriocolor}{G} & \textcolor{gregoriocolor}{H} & \textcolor{gregoriocolor}{M} & \textcolor{gregoriocolor}{N} & \textcolor{gregoriocolor}{P} \\
 16 & 17 & 18 & 19 & 20 & 21 & 21 & 22 & 23 & 24 & 25 & 26 \\
\end{tabularx}

\begin{paracol}{2}
\selectlanguage{latin}
\lettrine[lines=2]{A}{ntiochíæ} natális 
 sancti Mánahen, qui fuit Heródis Tetrárchæ collactáneus, atque, Doctor et 
 Prophéta exsístens sub grátia novi Testaménti, in eádem urbe quiévit.
\switchcolumn
\selectlanguage{english}
\lettrine[lines=2]{A}{t} Antioch, the birthday of St. 
 Manahen, foster-brother of Herod the Tetrach. He was a doctor and 
 prophet under the grace of the New Testament, and his remains now lie in the 
 city of Antioch.
\switchcolumn*
\selectlanguage{latin}
Item beátæ Joánnæ, 
 uxóris Chuzæ, procuratóris Heródis, quam Lucas Evangelísta commémorat.
\switchcolumn
\selectlanguage{english}
Also, blessed Joanna, wife of Chuza, 
 Herod's steward, mentioned by the evangelist St. Luke.
\switchcolumn*
\selectlanguage{latin}
In Portu Románo 
 natális sancti Vincéntii Mártyris.
\switchcolumn
\selectlanguage{english}
At Porto, the birthday of St. 
 Vincent, martyr.
\switchcolumn*
\selectlanguage{latin}
Nannéte, in Británnia minóre, beatórum Mártyrum Donatiáni et Rogatiáni fratrum, qui, sub 
 Diocletiáno Imperatóre, pro constántia fídei, in cárcerem missi, et in 
 equúleo suspénsi ac laniáti, deínde láncea militári confóssi, novíssime 
 cápite præcísi sunt.
\switchcolumn
\selectlanguage{english}
At Nantes in Brittany, in the time 
 of Emperor Diocletian, the blessed martyrs Donatian and Rogatian, brothers, 
 who, because of their constancy in the faith, were sent to prison, stretched 
 on the rack, and lacerated. Finally, they were pierced through with a 
 soldier's lance, and then beheaded.
\switchcolumn*
\selectlanguage{latin}
In Istria sanctórum 
 Mártyrum Zoélli, Servílii, Felícis, Silváni, et Dióclis.
\switchcolumn
\selectlanguage{english}
In Istria, the holy martyrs Zoellus, 
 Servilius, Felix, Silvanus, and Diocles.
\switchcolumn*
\selectlanguage{latin}
Eódem die sanctórum 
 Mártyrum Melétii, ducis exércitus, ac Sociórum ejus ducentórum et 
 quinquagínta duórum mílitum; qui divérso mortis génere martyrium 
 complevérunt.
\switchcolumn
\selectlanguage{english}
Also, the holy martyrs Meletius, 
 who was a military officer, and two hundred and fifty-two of his companions, 
 who achieved their martyrdom by various kinds of deaths.
\switchcolumn*
\selectlanguage{latin}
Item sanctárum 
 Mártyrum Susánnæ, Marciánæ et Palládiæ, eorúndem mílitum uxórum, quæ, una 
 cum párvulis suis, confráctæ sunt.
\switchcolumn
\selectlanguage{english}
Also, the holy martyrs Susanna, 
 Marciana, and Palladia, wives of the soldiers just mentioned, who were put 
 to death with their young children.
\switchcolumn*
\selectlanguage{latin}
Medioláni sancti 
 Robustiáni Mártyris.
\switchcolumn
\selectlanguage{english}
At Milan, St. Robustian, martyr.
\switchcolumn*
\selectlanguage{latin}
Bríxiæ sanctæ Afræ 
 Mártyris, quæ sub Hadriáno Imperatóre passa est.
\switchcolumn
\selectlanguage{english}
At Brescia, St. Afra, martyr, who 
 suffered under Emperor Hadrian.
\switchcolumn*
\selectlanguage{latin}
In monastério 
 Lirinénsi, in Gállia, sancti Vincéntii Presbyteri, doctrína et sanctitáte 
 conspícui.
\switchcolumn
\selectlanguage{english}
In the monastery of Lerins, St. 
 Vincent, a priest eminent for learning and sanctity.
\switchcolumn*
\selectlanguage{latin}
Marróchii, in Africa, 
 beáti Joánnis de Prado, Sacerdótis ex Ordine Minórum et Mártyris; qui, in 
 prædicatióne Evangélii, post víncula, cárceres, flagélla plurimáque ália 
 torménta, pro Christo fórtiter toleráta, per ignem martyrium consummávit.
\switchcolumn
\selectlanguage{english}
At Morocco in Africa, the passion 
 of blessed John of Prado, priest and martyr of the Order of Friars Minor. 
 While preaching the Gospel, he was bound, imprisoned, and scourged; and 
 after enduring with fortitude many other torments for Christ, fulfilled his 
 martyrdom by fire.
\switchcolumn*
\selectlanguage{latin}
Bonóniæ Translátio 
 sancti Domínici Confessóris, témpore Gregórii Papæ Noni.
\switchcolumn
\selectlanguage{english}
At Bologna, the translation of St. 
 Dominic, confessor, in the time of Pope Gregory IX.
\switchcolumn*
\selectlanguage{latin}
\end{paracol}


% ---- martyrology/mart05/mart0525.htm
\needspace{10\baselineskip}
\begin{paracol}{2}
\selectlanguage{latin}
\begin{center}{\color{gregoriocolor} Octávo Kaléndas Júnii. 
 Luna\dots\ }\end{center}
\switchcolumn
\selectlanguage{english}
\begin{center}{\color{gregoriocolor} The 
 Twenty-Fifth Day of May. The\dots\ Day of the Moon.}\end{center}
\end{paracol}

\noindent\begin{tabularx}{\linewidth}{*{19}{>{\centering\arraybackslash}X}}
 \textcolor{gregoriocolor}{a} & \textcolor{gregoriocolor}{b} & \textcolor{gregoriocolor}{c} & \textcolor{gregoriocolor}{d} & \textcolor{gregoriocolor}{e} & \textcolor{gregoriocolor}{f} & \textcolor{gregoriocolor}{g} & \textcolor{gregoriocolor}{h} & \textcolor{gregoriocolor}{i} & \textcolor{gregoriocolor}{k} & \textcolor{gregoriocolor}{l} & \textcolor{gregoriocolor}{m} & \textcolor{gregoriocolor}{n} & \textcolor{gregoriocolor}{p} & \textcolor{gregoriocolor}{q} & \textcolor{gregoriocolor}{r} & \textcolor{gregoriocolor}{s} & \textcolor{gregoriocolor}{t} & \textcolor{gregoriocolor}{u} \\
 28 & 29 & 30 & 1 & 2 & 3 & 4 & 5 & 6 & 7 & 8 & 9 & 10 & 11 & 12 & 13 & 14 & 15 & 16 \\
\end{tabularx}
\vspace{0.5\baselineskip}
\noindent\begin{tabularx}{\linewidth}{*{12}{>{\centering\arraybackslash}X}}
 \textcolor{gregoriocolor}{A} & \textcolor{gregoriocolor}{B} & \textcolor{gregoriocolor}{C} & \textcolor{gregoriocolor}{D} & \textcolor{gregoriocolor}{E} & F & \textcolor{gregoriocolor}{F} & \textcolor{gregoriocolor}{G} & \textcolor{gregoriocolor}{H} & \textcolor{gregoriocolor}{M} & \textcolor{gregoriocolor}{N} & \textcolor{gregoriocolor}{P} \\
 17 & 18 & 19 & 20 & 21 & 22 & 22 & 23 & 24 & 25 & 26 & 27 \\
\end{tabularx}

\begin{paracol}{2}
\selectlanguage{latin}
\lettrine[lines=2]{S}{alérni} deposítio 
 beáti Gregórii Séptimi, Papæ et Confessóris, ecclesiásticæ libertátis 
 propugnatóris ac defensóris acérrimi.
\switchcolumn
\selectlanguage{english}
\lettrine[lines=2]{A}{t} Salerno, the death of blessed 
 Pope Gregory VII, a most zealous protector and champion of Church liberty.
\switchcolumn*
\selectlanguage{latin}
Romæ, via Nomentána, 
 natális beáti Urbáni Primi, Papæ et Mártyris, cujus exhortatióne et doctrína 
 complúres (in quibus fuére Tibúrtius et Valeriánus) Christi suscepérunt 
 fidem, et pro ea martyrium subiérunt. Ipse quoque, in persecutióne 
 Alexándri Sevéri, multa passus pro Ecclésia Dei, tandem, cervícibus 
 abscíssis, martyrio coronátus est.
\switchcolumn
\selectlanguage{english}
At Rome, on the Via Nomentana, the 
 birthday of blessed Urban, pope and martyr, by whose exhortation and 
 teaching many persons, among whom were Tiburtius and Valerian, received the 
 faith of Christ and suffered martyrdom for it. He himself endured many 
 afflictions for the Church of God, and was crowned with martyrdom by being 
 beheaded in the persecution of Alexander Severus.
\switchcolumn*
\selectlanguage{latin}
Girvi, in Anglia, 
 tránsitus sancti Bedæ Venerábilis, Presbyteri, Confessóris et Ecclésiæ 
 Doctóris, sanctitáte et eruditióne celebérrimi. Ipsíus autem festum 
 recólitur sexto Kaléndas Júnii.
\switchcolumn
\selectlanguage{english}
At Jarrow in England, the death of 
 St. Venerable Bede, priest, confessor and doctor of the Church, well known 
 for his sanctity and scholarship. His feast, however, is celebrated on 
 the 27th day of May.
\switchcolumn*
\selectlanguage{latin}
Floréntiæ natális 
 sanctæ Maríæ-Magdalénæ de Pazzis, Vírginis, ex Ordine Carmelitárum, vita et 
 sanctitáte illústris. Ejus vero festívitas quarto Kaléndas Júnii 
 celebrátur.
\switchcolumn
\selectlanguage{english}
At Florence, the birthday of St. 
 Mary Magdalen de Pazzi, a virgin of the Order of the Carmelites, who is 
 famed for her holy life. Her feast is observed on the 29th of May.
\switchcolumn*
\selectlanguage{latin}
Doróstori, in Mysia 
 inferióre, item natális sanctórum Mártyrum Pasícratis, Valentiónis et 
 aliórum duórum, simul coronatórum.
\switchcolumn
\selectlanguage{english}
At Silistria in Bulgaria, the 
 birthday of the holy martyrs Pasicrates, Valentio, and two others crowned 
 with them.
\switchcolumn*
\selectlanguage{latin}
Medioláni sancti 
 Dionysii Epíscopi, qui, ab Ariáno Imperatóre Constántio in Cappadóciam pro 
 fide cathólica relegátus, ibídem, propióre Martyribus título, spíritum Deo 
 réddidit. Ejus sacrum corpus ab Aurélio Epíscopo transmíssum est 
 Mediolánum ad beátum Ambrósium Epíscopum; cui piæ actióni offícium quoque 
 sancti Basilíi Magni accessísse tráditur.
\switchcolumn
\selectlanguage{english}
At Milan, Bishop St. Denis, who for 
 the Catholic faith was exiled into Cappadocia by the Arian emperor 
 Constantius, where he yielded his soul to God in a manner almost like that 
 of the martyrs. His revered body was sent to blessed Bishop Ambrose at 
 Milan, by Bishop Aurelius, with the help, it is said, of St. Basil the 
 Great.
\switchcolumn*
\selectlanguage{latin}
Floréntiæ natális 
 sancti Zenóbii, ejúsdem civitátis Epíscopi, vitæ sanctitáte ac miraculórum 
 glória conspícui.
\switchcolumn
\selectlanguage{english}
At Florence, the birthday of St. 
 Zenobius, bishop of that city, renowned for the sanctity of his life and his 
 glorious miracles.
\switchcolumn*
\selectlanguage{latin}
In Británnia sancti 
 Aldélmi, Epíscopi Schireburgénsis.
\switchcolumn
\selectlanguage{english}
In England, St. Aldhelm, bishop of 
 Sherborne.
\switchcolumn*
\selectlanguage{latin}
In território 
 Tricassíno sancti Leónis Confessóris.
\switchcolumn
\selectlanguage{english}
In the territory of Troyes, St. 
 Leo, confessor.
\switchcolumn*
\selectlanguage{latin}
Lutétiæ Parisiórum 
 sanctæ Magdalénæ-Sophíæ Barat, Fundatrícis Institúti Sorórum a Sacro Corde 
 Jesu; quæ pro Christiána puellárum informatióne valde adlaborávit, et a Pio 
 Papa Undécimo in sanctárum Vírginum catálogum fuit reláta.
\switchcolumn
\selectlanguage{english}
At Paris, St. Madeleine-Sophie 
 Barat, foundress of the Congregation of the Sisters of the Sacred Heart, who 
 devoted her labours for the Christian education of girls. She was 
 added to the list of holy virgins by Pope Pius XI.
\switchcolumn*
\selectlanguage{latin}
Vérulis, in Hérnicis, 
 Translátio sanctæ Maríæ Jacóbi; cujus sacrum corpus 
 plúrimis miráculis 
 illustrátur.
\switchcolumn
\selectlanguage{english}
At Veroli in Campania, the 
 translation of St. Mary, the mother of James, whose revered body is noted 
 for many miracles.
\switchcolumn*
\selectlanguage{latin}
Assísii, in Umbria, 
 item Translátio sancti Francísci Confessóris, témpore Gregórii Papæ Noni.
\switchcolumn
\selectlanguage{english}
At Assisi in Umbria, the 
 translation of St. Francis, confessor, in the time of Pope Gregory IX.
\switchcolumn*
\selectlanguage{latin}
\end{paracol}


% ---- martyrology/mart05/mart0526.htm
\needspace{10\baselineskip}
\begin{paracol}{2}
\selectlanguage{latin}
\begin{center}{\color{gregoriocolor} Séptimo Kaléndas Júnii. 
 Luna\dots\ }\end{center}
\switchcolumn
\selectlanguage{english}
\begin{center}{\color{gregoriocolor} The 
 Twenty-Sixth Day of May. The\dots\ Day of the Moon.}\end{center}
\end{paracol}

\noindent\begin{tabularx}{\linewidth}{*{19}{>{\centering\arraybackslash}X}}
 \textcolor{gregoriocolor}{a} & \textcolor{gregoriocolor}{b} & \textcolor{gregoriocolor}{c} & \textcolor{gregoriocolor}{d} & \textcolor{gregoriocolor}{e} & \textcolor{gregoriocolor}{f} & \textcolor{gregoriocolor}{g} & \textcolor{gregoriocolor}{h} & \textcolor{gregoriocolor}{i} & \textcolor{gregoriocolor}{k} & \textcolor{gregoriocolor}{l} & \textcolor{gregoriocolor}{m} & \textcolor{gregoriocolor}{n} & \textcolor{gregoriocolor}{p} & \textcolor{gregoriocolor}{q} & \textcolor{gregoriocolor}{r} & \textcolor{gregoriocolor}{s} & \textcolor{gregoriocolor}{t} & \textcolor{gregoriocolor}{u} \\
 29 & 30 & 1 & 2 & 3 & 4 & 5 & 6 & 7 & 8 & 9 & 10 & 11 & 12 & 13 & 14 & 15 & 16 & 17 \\
\end{tabularx}
\vspace{0.5\baselineskip}
\noindent\begin{tabularx}{\linewidth}{*{12}{>{\centering\arraybackslash}X}}
 \textcolor{gregoriocolor}{A} & \textcolor{gregoriocolor}{B} & \textcolor{gregoriocolor}{C} & \textcolor{gregoriocolor}{D} & \textcolor{gregoriocolor}{E} & F & \textcolor{gregoriocolor}{F} & \textcolor{gregoriocolor}{G} & \textcolor{gregoriocolor}{H} & \textcolor{gregoriocolor}{M} & \textcolor{gregoriocolor}{N} & \textcolor{gregoriocolor}{P} \\
 18 & 19 & 20 & 21 & 22 & 23 & 23 & 24 & 25 & 26 & 27 & 28 \\
\end{tabularx}

\begin{paracol}{2}
\selectlanguage{latin}
\lettrine[lines=2]{R}{omæ} sancti Philíppi 
 Nérii, Presbyteri et Confessóris, qui Congregatiónis Oratórii Fundátor fuit, 
 ac virginitáte, prophetíæ dono, et miráculis éxstitit insígnis.
\switchcolumn
\selectlanguage{english}
\lettrine[lines=2]{A}{t} Rome, St. Philip Neri, priest and 
 confessor, founder of the Congregation of the Oratory, celebrated for his 
 virginal purity, the gift of prophecy, and miracles.
\switchcolumn*
\selectlanguage{latin}
Item Romæ sancti 
 Eleuthérii, Papæ et Mártyris, qui multos nóbiles Romanórum ad fidem Christi 
 perdúxit, et sanctos Damiánum et Fugátium in Británniam misit, qui Lúcium 
 Regem, cum uxóre ipsíus ac toto fere pópulo, baptizárunt.
\switchcolumn
\selectlanguage{english}
Also at Rome, St. Eleutherius, pope 
 and martyr, who converted to the Christian faith many noble Romans. He 
 sent Saints Damian and Fugatius to England, and they baptized King Lucius, 
 his wife, and almost all his people.
\switchcolumn*
\selectlanguage{latin}
Cantuáriæ, in Anglia, 
 natális sancti Augustíni, Epíscopi et Confessóris; qui, una cum áliis, a 
 beáto Gregório Papa missus, genti Anglórum sacrum Christi Evangélium 
 prædicávit., ibíque, virtútibus et miráculis gloriósus, obdormívit in 
 Dómino. Ejus tamen festívitas quinto Kaléndas Júnii recólitur.
\switchcolumn
\selectlanguage{english}
At Canterbury in England, St. 
 Augustine, bishop, who was sent there with others by blessed Pope Gregory, 
 and who preached the Gospel of Christ to the English nation. 
 Celebrated for virtues and miracles, he went peacefully to his rest in the 
 Lord. The 28th of May is observed as his feast.
\switchcolumn*
\selectlanguage{latin}
Athénis item natális 
 beáti Quadráti, Apostolórum discípuli, qui, in persecutióne Hadriáni, fide 
 et indústria sua cóngregans Ecclésiam grándi terróre dispérsam, librum pro 
 Christiánæ religiónis defensióne, valde 
 útilem et Apostólica doctrína dignum, 
 eídem Imperatóri porréxit.
\switchcolumn
\selectlanguage{english}
At Athens, during the persecution 
 of Hadrian, the birthday of blessed Quadratus, a disciple of the apostles, 
 who collected by his zealous work the faithful who had dispersed through 
 terror, and presented to the emperor a book which was an excellent apology 
 of the Christian religion, worthy of an apostle.
\switchcolumn*
\selectlanguage{latin}
Romæ sanctórum 
 Mártyrum Simítrii Presbyteri, et aliórum vigínti duórum; qui sub Antoníno 
 Pio passi sunt.
\switchcolumn
\selectlanguage{english}
At Rome, the holy martyrs Simitrius, 
 priest, and twenty-two others who suffered under Antoninus Pius.
\switchcolumn*
\selectlanguage{latin}
Viénnæ, in Gállia, 
 sancti Zacharíæ, Epíscopi et Mártyris, qui sub Trajáno passus est.
\switchcolumn
\selectlanguage{english}
At Vienne, St. Zacharias, bishop and 
 martyr, who suffered under Trajan.
\switchcolumn*
\selectlanguage{latin}
In Africa sancti 
 Quadráti Mártyris, in cujus solemnitáte sanctus Augustínus sermónem hábuit.
\switchcolumn
\selectlanguage{english}
In Africa, St. Quadratus, martyr, 
 on whose feast day St. Augustine preached a sermon.
\switchcolumn*
\selectlanguage{latin}
Tudérti, in Umbria, 
 natális sanctórum Mártyrum Felicíssimi, Heráclii et Paulíni.
\switchcolumn
\selectlanguage{english}
At Todi in Umbria, the birthday of 
 the holy martyrs Felicissimus, Heraclius, and Paulinus.
\switchcolumn*
\selectlanguage{latin}
In território Antisiodorénsi pássio sancti Prisci Mártyris, qui, cum ingénti multitúdine 
 fidélium Christi, cápite cæsus est.
\switchcolumn
\selectlanguage{english}
In the territory of Auxerre, the 
 passion of St. Priscus, martyr, along with a great multitude of other 
 Christians.
\switchcolumn*
\selectlanguage{latin}
In civitáte Quiténsi, 
 Æquatoriánæ Ditiónis, sanctæ Maríæ Annæ a Jesu de Parédes Vírginis, e tértio 
 Ordine sancti Francísci, austeritáte et in próximum caritáte præcláræ, quam 
 Pius Papa Duodécimus sanctárum Vírginum catálogo adnumerávit.
\switchcolumn
\selectlanguage{english}
In the city of Quito in Ecuador, 
 St. María Ana de Jesù de Paredes, a third order Franciscan, well known for 
 her austerity and charity towards her neighbour. Pope Pius XII 
 numbered her in the book of Virgins.
\switchcolumn*
\selectlanguage{latin}
\end{paracol}


% ---- martyrology/mart05/mart0527.htm
\needspace{10\baselineskip}
\begin{paracol}{2}
\selectlanguage{latin}
\begin{center}{\color{gregoriocolor} Sexto Kaléndas Júnii. 
 Luna\dots\ }\end{center}
\switchcolumn
\selectlanguage{english}
\begin{center}{\color{gregoriocolor} The 
 Twenty-Seventh Day of May. The\dots\ Day of the Moon.}\end{center}
\end{paracol}

\noindent\begin{tabularx}{\linewidth}{*{19}{>{\centering\arraybackslash}X}}
 \textcolor{gregoriocolor}{a} & \textcolor{gregoriocolor}{b} & \textcolor{gregoriocolor}{c} & \textcolor{gregoriocolor}{d} & \textcolor{gregoriocolor}{e} & \textcolor{gregoriocolor}{f} & \textcolor{gregoriocolor}{g} & \textcolor{gregoriocolor}{h} & \textcolor{gregoriocolor}{i} & \textcolor{gregoriocolor}{k} & \textcolor{gregoriocolor}{l} & \textcolor{gregoriocolor}{m} & \textcolor{gregoriocolor}{n} & \textcolor{gregoriocolor}{p} & \textcolor{gregoriocolor}{q} & \textcolor{gregoriocolor}{r} & \textcolor{gregoriocolor}{s} & \textcolor{gregoriocolor}{t} & \textcolor{gregoriocolor}{u} \\
 30 & 1 & 2 & 3 & 4 & 5 & 6 & 7 & 8 & 9 & 10 & 11 & 12 & 13 & 14 & 15 & 16 & 17 & 18 \\
\end{tabularx}
\vspace{0.5\baselineskip}
\noindent\begin{tabularx}{\linewidth}{*{12}{>{\centering\arraybackslash}X}}
 \textcolor{gregoriocolor}{A} & \textcolor{gregoriocolor}{B} & \textcolor{gregoriocolor}{C} & \textcolor{gregoriocolor}{D} & \textcolor{gregoriocolor}{E} & F & \textcolor{gregoriocolor}{F} & \textcolor{gregoriocolor}{G} & \textcolor{gregoriocolor}{H} & \textcolor{gregoriocolor}{M} & \textcolor{gregoriocolor}{N} & \textcolor{gregoriocolor}{P} \\
 19 & 20 & 21 & 22 & 23 & 24 & 24 & 25 & 26 & 27 & 28 & 29 \\
\end{tabularx}

\begin{paracol}{2}
\selectlanguage{latin}
\lettrine[lines=2]{S}{ancti} Bedæ Venerábilis, 
 Presbyteri, Confessóris et Ecclésiæ Doctóris; qui migrávit in cælum octávo 
 Kaléndas Júnii.
\switchcolumn
\selectlanguage{english}
\lettrine[lines=2]{S}{t.} Venerable Bede, priest, 
 confessor, and doctor of the Church, who went to heaven on the 25th of May.
\switchcolumn*
\selectlanguage{latin}
Sancti Joánnis Primi, 
 Papæ et Mártyris; cujus dies natális quintodécimo Kaléndas Júnii refértur, 
 sed festívitas hac die, ob Translatiónem córporis ejus, potíssimum 
 celebrátur.
\switchcolumn
\selectlanguage{english}
St. John I, pope and martyr. 
 His birthday is observed on the 18th of May, but his feast is celebrated 
 today because of the translation of his revered body.
\switchcolumn*
\selectlanguage{latin}
Doróstori, in Mysia 
 inferióre, pássio beáti Júlii, qui, témpore Alexándri Imperatóris, cum esset 
 veteránus et eméritæ milítiæ, comprehénsus est ab officiálibus, et Máximo 
 Prǽsidi oblátus; quo præsénte, cum exsecrarétur idóla, et Christi nomen 
 constantíssime confiterétur, capitáli senténtia punítus est.
\switchcolumn
\selectlanguage{english}
At Silistria in Bulgaria, during 
 the reign of Emperor Alexander, the martyrdom of blessed Julius, a veteran 
 soldier in retirement, who was arrested by the officials and presented to 
 the governor Maximus. Having denounced the idols in his presence, and 
 confessed the name of Christ with utmost constancy, he was condemned to 
 capital punishment.
\switchcolumn*
\selectlanguage{latin}
In pago Atrebaténsi 
 sancti Ranúlfi Mártyris.
\switchcolumn
\selectlanguage{english}
In the district of Arras, St. 
 Ralph, martyr.
\switchcolumn*
\selectlanguage{latin}
Apud Soram sanctæ 
 Restitútæ, Vírginis et Mártyris; quæ, sub Aureliáno Imperatóre et Agáthio 
 Procónsule, fídei certámine suscépto, dǽmonum ímpetus, paréntum blandítias 
 et tortórum sævítiam superávit, ac demum, cum áliis Christiánis, cápite 
 truncáta, martyrio decoráta est.
\switchcolumn
\selectlanguage{english}
At Sora, in the time of Emperor 
 Aurelian and the proconsul Agathius, St. Restituta, virgin and martyr, who 
 overcame in a trial for the faith the violence of the demons, the affections 
 of her family, and the cruelty of the executioners. Being finally 
 beheaded with other Christians, she obtained the honour of martyrdom.
\switchcolumn*
\selectlanguage{latin}
Aráusicæ, in Gálliis, 
 sancti Eutrópii Epíscopi, virtútibus atque miráculis illústris.
\switchcolumn
\selectlanguage{english}
At Orange in France, St. Eutropius, 
 a bishop illustrious for virtues and miracles.
\switchcolumn*
\selectlanguage{latin}
Herbípoli, in Germánia, 
 sancti Brunónis, Epíscopi et Confessóris.
\switchcolumn
\selectlanguage{english}
At Wurzburg in Germany, St. Bruno, 
 bishop and confessor.
\switchcolumn*
\selectlanguage{latin}
\end{paracol}


% ---- martyrology/mart05/mart0528.htm
\needspace{10\baselineskip}
\begin{paracol}{2}
\selectlanguage{latin}
\begin{center}{\color{gregoriocolor} Quinto Kaléndas Júnii. 
 Luna\dots\ }\end{center}
\switchcolumn
\selectlanguage{english}
\begin{center}{\color{gregoriocolor} The 
 Twenty-Eighth Day of May. The\dots\ Day of the Moon.}\end{center}
\end{paracol}

\noindent\begin{tabularx}{\linewidth}{*{19}{>{\centering\arraybackslash}X}}
 \textcolor{gregoriocolor}{a} & \textcolor{gregoriocolor}{b} & \textcolor{gregoriocolor}{c} & \textcolor{gregoriocolor}{d} & \textcolor{gregoriocolor}{e} & \textcolor{gregoriocolor}{f} & \textcolor{gregoriocolor}{g} & \textcolor{gregoriocolor}{h} & \textcolor{gregoriocolor}{i} & \textcolor{gregoriocolor}{k} & \textcolor{gregoriocolor}{l} & \textcolor{gregoriocolor}{m} & \textcolor{gregoriocolor}{n} & \textcolor{gregoriocolor}{p} & \textcolor{gregoriocolor}{q} & \textcolor{gregoriocolor}{r} & \textcolor{gregoriocolor}{s} & \textcolor{gregoriocolor}{t} & \textcolor{gregoriocolor}{u} \\
 1 & 2 & 3 & 4 & 5 & 6 & 7 & 8 & 9 & 10 & 11 & 12 & 13 & 14 & 15 & 16 & 17 & 18 & 19 \\
\end{tabularx}
\vspace{0.5\baselineskip}
\noindent\begin{tabularx}{\linewidth}{*{12}{>{\centering\arraybackslash}X}}
 \textcolor{gregoriocolor}{A} & \textcolor{gregoriocolor}{B} & \textcolor{gregoriocolor}{C} & \textcolor{gregoriocolor}{D} & \textcolor{gregoriocolor}{E} & F & \textcolor{gregoriocolor}{F} & \textcolor{gregoriocolor}{G} & \textcolor{gregoriocolor}{H} & \textcolor{gregoriocolor}{M} & \textcolor{gregoriocolor}{N} & \textcolor{gregoriocolor}{P} \\
 20 & 21 & 22 & 23 & 24 & 25 & 25 & 26 & 27 & 28 & 29 & 30 \\
\end{tabularx}

\begin{paracol}{2}
\selectlanguage{latin}
\lettrine[lines=2]{S}{ancti} Augustíni, 
 Epíscopi Cantuariénsis et Confessóris, cujus dies natális ágitur séptimo 
 Kaléndas Júnii.
\switchcolumn
\selectlanguage{english}
\lettrine[lines=2]{S}{t.} Augustine, bishop of Canterbury 
 and confessor, whose birthday is mentioned on the 26th of May.
\switchcolumn*
\selectlanguage{latin}
In Sardínia sanctórum 
 Mártyrum Æmílii, Felícis, Priámi et Luciáni, qui, pro Christo certántes, ab 
 eo glorióse coronáti sunt.
\switchcolumn
\selectlanguage{english}
In Sardinia, the holy martyrs 
 Aemilius, Priamus, and Lucian, who gained their crowns after being in the 
 combat for Christ.
\switchcolumn*
\selectlanguage{latin}
Carnóti, in Gállia, 
 sancti Caráuni Mártyris, qui sub Domitiáno Imperatóre, cápite amputátus, 
 martyrium sumpsit.
\switchcolumn
\selectlanguage{english}
At Chartres in France, under 
 Emperor Domitian, St. Caraunus, martyr, who was beheaded, and thus acquired 
 the glory of martyrdom.
\switchcolumn*
\selectlanguage{latin}
Item pássio sanctórum 
 Crescéntis, Dioscóridis, Pauli et Helládii.
\switchcolumn
\selectlanguage{english}
Also the martyrdom of the Saints 
 Crescens, Dioscorides, Paul, and Helladius.
\switchcolumn*
\selectlanguage{latin}
Thécuæ, in Palæstína, 
 sanctórum Monachórum Mártyrum, qui, témpore Theodósii junióris, a Saracénis 
 occísi sunt, quorum sacras relíquias collegérunt áccolæ, et summa 
 veneratióne illas habuérunt.
\switchcolumn
\selectlanguage{english}
At Thecua in Palestine, the saintly 
 monks who became martyrs by being killed by the Saracens, in the time of 
 Theodosius the Younger. Their venerable remains were gathered by the 
 inhabitants and preserved with greatest reverence.
\switchcolumn*
\selectlanguage{latin}
Corínthi sanctæ 
 Helicónidis Mártyris, témpore Gordiáni Imperatóris. Hæc primum, sub 
 Perénnio Prǽside, multis torméntis afflícta, deínde, sub ejus successóre 
 Justíno, íterum cruciáta, sed ab Angelis liberáta est; demum, disséctis 
 mammis, ferísque objécta atque igne probáta, cápitis obtruncatióne martyrium 
 complévit.
\switchcolumn
\selectlanguage{english}
At Corinth, St. Helconides, martyr, 
 who was first subjected to torments in the reign of Emperor Gordian, under 
 the governor Perennius, and then again tortured under his successor Justin, 
 but was delivered by an angel. Her breasts were cut away, she was 
 exposed to wild beasts and to fire, and finally her martyrdom was fulfilled 
 by beheading.
\switchcolumn*
\selectlanguage{latin}
Lutétiæ Parisiórum 
 sancti Germáni, Epíscopi et Confessóris; qui quantæ sanctitátis quantíque 
 fúerit mériti, quibus étiam miráculis clarúerit, litterárum moniméntis 
 Fortunátus Epíscopus consignávit.
\switchcolumn
\selectlanguage{english}
At Paris, St. Germanus, bishop and 
 confessor, whose fame for holiness, merit, and miracles has been handed down 
 to us by the writings of Bishop Fortunatus.
\switchcolumn*
\selectlanguage{latin}
Medioláni sancti 
 Senatóris Epíscopi, virtútibus et eruditióne claríssimi.
\switchcolumn
\selectlanguage{english}
At Milan, St. Senator, bishop, who 
 was very well known for his virtues and his learning.
\switchcolumn*
\selectlanguage{latin}
Urgéllæ, in Hispánia 
 Tarraconénsi, sancti Justi Epíscopi.
\switchcolumn
\selectlanguage{english}
At Urgel in Spain, Bishop St. 
 Justus.
\switchcolumn*
\selectlanguage{latin}
Floréntiæ sancti Pódii, 
 Epíscopi et Confessóris.
\switchcolumn
\selectlanguage{english}
At Florence, St. Podius, bishop and 
 confessor.
\switchcolumn*
\selectlanguage{latin}
Apud Nováriam sancti 
 Bernárdi a Menthóne, Confessóris, qui in monte Jovis, super Alpes, in 
 Valésia, celebérrimum cœnóbium et hospítium exstrúxit, atque a Pio Papa 
 Undécimo, non modo Alpium íncolis vel viatóribus, sed iis étiam qui eárum 
 juga ascendéndo se exércent, cæléstis Patrónus est attribútus.
\switchcolumn
\selectlanguage{english}
At Novara, St. Bernard of Menton, 
 confessor. On Mount Jou in the Alps of Valais in Switzerland, he 
 founded the famous monastery and hospice. Pope Pius XI appointed him 
 the heavenly patron not only of those who live in or travel across the Alps, 
 but of all mountain climbers.
\switchcolumn*
\selectlanguage{latin}
\end{paracol}


% ---- martyrology/mart05/mart0529.htm
\needspace{10\baselineskip}
\begin{paracol}{2}
\selectlanguage{latin}
\begin{center}{\color{gregoriocolor} Quarto Kaléndas Júnii. 
 Luna\dots\ }\end{center}
\switchcolumn
\selectlanguage{english}
\begin{center}{\color{gregoriocolor} The 
 Twenty-Ninth Day of May. The\dots\ Day of the Moon.}\end{center}
\end{paracol}

\noindent\begin{tabularx}{\linewidth}{*{19}{>{\centering\arraybackslash}X}}
 \textcolor{gregoriocolor}{a} & \textcolor{gregoriocolor}{b} & \textcolor{gregoriocolor}{c} & \textcolor{gregoriocolor}{d} & \textcolor{gregoriocolor}{e} & \textcolor{gregoriocolor}{f} & \textcolor{gregoriocolor}{g} & \textcolor{gregoriocolor}{h} & \textcolor{gregoriocolor}{i} & \textcolor{gregoriocolor}{k} & \textcolor{gregoriocolor}{l} & \textcolor{gregoriocolor}{m} & \textcolor{gregoriocolor}{n} & \textcolor{gregoriocolor}{p} & \textcolor{gregoriocolor}{q} & \textcolor{gregoriocolor}{r} & \textcolor{gregoriocolor}{s} & \textcolor{gregoriocolor}{t} & \textcolor{gregoriocolor}{u} \\
 2 & 3 & 4 & 5 & 6 & 7 & 8 & 9 & 10 & 11 & 12 & 13 & 14 & 15 & 16 & 17 & 18 & 19 & 20 \\
\end{tabularx}
\vspace{0.5\baselineskip}
\noindent\begin{tabularx}{\linewidth}{*{12}{>{\centering\arraybackslash}X}}
 \textcolor{gregoriocolor}{A} & \textcolor{gregoriocolor}{B} & \textcolor{gregoriocolor}{C} & \textcolor{gregoriocolor}{D} & \textcolor{gregoriocolor}{E} & F & \textcolor{gregoriocolor}{F} & \textcolor{gregoriocolor}{G} & \textcolor{gregoriocolor}{H} & \textcolor{gregoriocolor}{M} & \textcolor{gregoriocolor}{N} & \textcolor{gregoriocolor}{P} \\
 21 & 22 & 23 & 24 & 25 & 26 & 26 & 27 & 28 & 29 & 30 & 1 \\
\end{tabularx}

\begin{paracol}{2}
\selectlanguage{latin}
\lettrine[lines=2]{S}{anctæ} Maríæ-Magdalénæ 
 de Pazzis, ex Ordine Carmelitárum, Vírginis, cujus dies natális octávo 
 Kaléndas Júnii recensétur.
\switchcolumn
\selectlanguage{english}
\lettrine[lines=2]{S}{t.} Mary Magdalene of Pazzi of the 
 Order of Carmelites, and virgin. Her birthday was mentioned on the 
 25th of May.
\switchcolumn*
\selectlanguage{latin}
Romæ, via Aurélia, natális sancti Restitúti Mártyris.
\switchcolumn
\selectlanguage{english}
At Rome, on the Via Aurelia, the 
 birthday of St. Restitutus, martyr.
\switchcolumn*
\selectlanguage{latin}
Apud Icónium, in 
 Lycaónia, pássio sanctórum Conónis, et fílii annórum duódecim, qui, sub 
 Aureliáno Imperatóre, cratículæ, prunis suppósitis et óleo superinfúso 
 candéntis, suspensiónis in equúleo atque ignis pœnam constánter passi, ad 
 extrémum, málleo lígneo mánibus eórum contrítis, spíritum emisérunt.
\switchcolumn
\selectlanguage{english}
At Iconium in Lycaonia, in the time 
 of Emperor Aurelian, the martyrdom of the Saints Conon and his son, a child 
 twelve years of age, who were laid on a grate over burning coals sprinkled 
 with oil, placed on the rack, and exposed to the fire. Finally their 
 hands were crushed with a mallet, and they breathed their last.
\switchcolumn*
\selectlanguage{latin}
In agro Tridentíno 
 natális sanctórum Mártyrum Sisínii, Martyrii et Alexándri; qui, témpore 
 Honórii Imperatóris, in Anáuniæ pártibus (ut scribit in vita sancti Ambrósii 
 Paulínus), persequéntibus Gentílibus, martyrii corónam adépti sunt.
\switchcolumn
\selectlanguage{english}
In the district of Trent, in the 
 time of Emperor Honorius, the birthday of the holy martyrs Sisinius, 
 Martyrius, and Alexander, who were persecuted by the heathens of Anaunia, 
 and obtained the crown of martyrdom, all of which is told by Paulinus in the
 Life of Ambrose.
\switchcolumn*
\selectlanguage{latin}
Cameríni pássio 
 sanctórum mille quingentórum et vigínti quinque Mártyrum.
\switchcolumn
\selectlanguage{english}
At Camerino, the passion of fifteen 
 hundred and twenty-five holy martyrs.
\switchcolumn*
\selectlanguage{latin}
Cæsaréæ Philíppi 
 sanctárum Mártyrum Theodósiæ, quæ sancti Procópii Mártyris 
 éxstitit mater, 
 et aliárum duódecim nobílium matronárum; quæ, in persecutióne Diocletiáni, 
 cápitis obtruncatióne consummátæ sunt.
\switchcolumn
\selectlanguage{english}
At Caesarea Philippi, the holy 
 martyrs Theodosia, mother of the martyr St. Procopius, and twelve other 
 noble women, whose lives were ended by their being beheaded in the 
 persecution of Diocletian.
\switchcolumn*
\selectlanguage{latin}
Tréviris beáti 
 Maximíni, Epíscopi et Confessóris; a quo sanctus Athanásius Epíscopus, ob 
 persecutiónem Arianórum éxsulans honorífice suscéptus fuit.
\switchcolumn
\selectlanguage{english}
At Treves, blessed Maximinus, 
 bishop and confessor, who received with honour the patriarch St. Athanasius 
 when he was banished by the Arian persecutors.
\switchcolumn*
\selectlanguage{latin}
Verónæ sancti Máximi 
 Epíscopi.
\switchcolumn
\selectlanguage{english}
At Verona, St. Maximus, bishop.
\switchcolumn*
\selectlanguage{latin}
Arcáni, in Látio, 
 sancti Eleuthérii Confessóris.
\switchcolumn
\selectlanguage{english}
At Arcano in Lazio, St. Eleutherius, 
 confessor.
\switchcolumn*
\selectlanguage{latin}
\end{paracol}


% ---- martyrology/mart05/mart0530.htm
\needspace{10\baselineskip}
\begin{paracol}{2}
\selectlanguage{latin}
\begin{center}{\color{gregoriocolor} Tértio Kaléndas Júnii. 
 Luna\dots\ }\end{center}
\switchcolumn
\selectlanguage{english}
\begin{center}{\color{gregoriocolor} The 
 Thirtieth Day of May. The\dots\ Day of the Moon.}\end{center}
\end{paracol}

\noindent\begin{tabularx}{\linewidth}{*{19}{>{\centering\arraybackslash}X}}
 \textcolor{gregoriocolor}{a} & \textcolor{gregoriocolor}{b} & \textcolor{gregoriocolor}{c} & \textcolor{gregoriocolor}{d} & \textcolor{gregoriocolor}{e} & \textcolor{gregoriocolor}{f} & \textcolor{gregoriocolor}{g} & \textcolor{gregoriocolor}{h} & \textcolor{gregoriocolor}{i} & \textcolor{gregoriocolor}{k} & \textcolor{gregoriocolor}{l} & \textcolor{gregoriocolor}{m} & \textcolor{gregoriocolor}{n} & \textcolor{gregoriocolor}{p} & \textcolor{gregoriocolor}{q} & \textcolor{gregoriocolor}{r} & \textcolor{gregoriocolor}{s} & \textcolor{gregoriocolor}{t} & \textcolor{gregoriocolor}{u} \\
 3 & 4 & 5 & 6 & 7 & 8 & 9 & 10 & 11 & 12 & 13 & 14 & 15 & 16 & 17 & 18 & 19 & 20 & 21 \\
\end{tabularx}
\vspace{0.5\baselineskip}
\noindent\begin{tabularx}{\linewidth}{*{12}{>{\centering\arraybackslash}X}}
 \textcolor{gregoriocolor}{A} & \textcolor{gregoriocolor}{B} & \textcolor{gregoriocolor}{C} & \textcolor{gregoriocolor}{D} & \textcolor{gregoriocolor}{E} & F & \textcolor{gregoriocolor}{F} & \textcolor{gregoriocolor}{G} & \textcolor{gregoriocolor}{H} & \textcolor{gregoriocolor}{M} & \textcolor{gregoriocolor}{N} & \textcolor{gregoriocolor}{P} \\
 22 & 23 & 24 & 25 & 26 & 27 & 27 & 28 & 29 & 30 & 1 & 2 \\
\end{tabularx}

\begin{paracol}{2}
\selectlanguage{latin}
\lettrine[lines=2]{S}{ancti} Felícis Primi, 
 Papæ et Mártyris, cujus dies natális tértio Kaléndas Januárii recensétur.
\switchcolumn
\selectlanguage{english}
\lettrine[lines=2]{P}{ope} St. Felix I, martyr, whose 
 birthday is commemorated on the 30th of December.
\switchcolumn*
\selectlanguage{latin}
Túrribus, in Sardínia, sanctórum Mártyrum Gabíni et Críspuli.
\switchcolumn
\selectlanguage{english}
At Torres in Sardinia, the holy 
 martyrs Gabinus and Crispulus.
\switchcolumn*
\selectlanguage{latin}
Antiochíæ sanctórum 
 Syci et Palatíni, qui, pro Christi nómine, multa torménta passi sunt.
\switchcolumn
\selectlanguage{english}
At Antioch, Saints Sycus and 
 Palatinus, who endured many torments for the name of Christ.
\switchcolumn*
\selectlanguage{latin}
Ravénnæ sancti 
 Exsuperántii, Epíscopi et Confessóris.
\switchcolumn
\selectlanguage{english}
At Ravenna, St. Exuperantius, 
 bishop and confessor.
\switchcolumn*
\selectlanguage{latin}
Papíæ sancti Anastásii 
 Epíscopi.
\switchcolumn
\selectlanguage{english}
At Pavia, St. Anastasius, bishop.
\switchcolumn*
\selectlanguage{latin}
Cæsaréæ, in Cappadócia, 
 sanctórum Basilíi et Emméliæ uxóris, qui fuérunt paréntes beatórum Basilíi 
 Magni et Gregórii Nysséni ac Petri Sebasténsis Episcopórum, atque Macrínæ 
 Vírginis. Hi vero sancti cónjuges, témpore Galérii Maximiáni, extórres 
 facti, Pónticas solitúdines incolúere; et post persecutiónem, fíliis suárum 
 relíctis virtútum herédibus, in pace quievérunt.
\switchcolumn
\selectlanguage{english}
At Caesarea in Cappadocia, the 
 Saints Basil and his wife Emmelia, parents of St. Basil the Great, St. 
 Gregory of Nyssa, St. Peter of Sebastopol, bishops, and St. Macrina, virgin. 
 They lived in exile in the deserts of Pontus during the reign of Galerius 
 Maximian, and after the persecution they died in peace, leaving their 
 children as heirs of their virtues.
\switchcolumn*
\selectlanguage{latin}
Híspali, in Hispánia, 
 sancti Ferdinándi Tértii, Castéllæ et Legiónis Regis, ob virtútum 
 præstántiam cognoménto Sancti; qui, fídei propagándæ zelo clarus, tandem, 
 devíctis Mauris, ad cæléste regnum, terréno relícto, felíciter evolávit.
\switchcolumn
\selectlanguage{english}
At Seville in Spain, St. Ferdinand 
 III, king of Castile and Leon. He was surnamed the Saint on account of 
 his eminent virtues; he was celebrated for his zeal in spreading the faith. 
 After conquering the Moors he left his kingdom on earth to pass happily to 
 that of heaven.
\switchcolumn*
\selectlanguage{latin}
Rotómagi sanctæ Joánnæ 
 Arcénsis Vírginis, Puéllæ Aurelianénsis appellátæ, quæ cum fórtiter pro 
 pátria dimicásset, tandem, in hóstium potestátem trádita, iníquo judício 
 condemnáta est et igne combústa; atque a Benedícto Décimo quinto, Pontífice 
 Máximo, Sanctórum fastis adscrípta.
\switchcolumn
\selectlanguage{english}
At Rouen, St. Joan of Arc, virgin, 
 called the Maid of Orleans. After fighting heroically for her 
 fatherland, she was at the end delivered into the hands of the enemies, 
 condemned by an unjust judge, and burned at the stake. The supreme 
 Pontiff Benedict XV placed her name on the canon of the saints.
\switchcolumn*
\selectlanguage{latin}
\end{paracol}


% ---- martyrology/mart05/mart0531.htm
\needspace{10\baselineskip}
\begin{paracol}{2}
\selectlanguage{latin}
\begin{center}{\color{gregoriocolor} Prídie Kaléndas Júnii. 
 Luna\dots\ }\end{center}
\switchcolumn
\selectlanguage{english}
\begin{center}{\color{gregoriocolor} The 
 Thirty-First Day of May. The\dots\ Day of the Moon.}\end{center}
\end{paracol}

\noindent\begin{tabularx}{\linewidth}{*{19}{>{\centering\arraybackslash}X}}
 \textcolor{gregoriocolor}{a} & \textcolor{gregoriocolor}{b} & \textcolor{gregoriocolor}{c} & \textcolor{gregoriocolor}{d} & \textcolor{gregoriocolor}{e} & \textcolor{gregoriocolor}{f} & \textcolor{gregoriocolor}{g} & \textcolor{gregoriocolor}{h} & \textcolor{gregoriocolor}{i} & \textcolor{gregoriocolor}{k} & \textcolor{gregoriocolor}{l} & \textcolor{gregoriocolor}{m} & \textcolor{gregoriocolor}{n} & \textcolor{gregoriocolor}{p} & \textcolor{gregoriocolor}{q} & \textcolor{gregoriocolor}{r} & \textcolor{gregoriocolor}{s} & \textcolor{gregoriocolor}{t} & \textcolor{gregoriocolor}{u} \\
 4 & 5 & 6 & 7 & 8 & 9 & 10 & 11 & 12 & 13 & 14 & 15 & 16 & 17 & 18 & 19 & 20 & 21 & 22 \\
\end{tabularx}
\vspace{0.5\baselineskip}
\noindent\begin{tabularx}{\linewidth}{*{12}{>{\centering\arraybackslash}X}}
 \textcolor{gregoriocolor}{A} & \textcolor{gregoriocolor}{B} & \textcolor{gregoriocolor}{C} & \textcolor{gregoriocolor}{D} & \textcolor{gregoriocolor}{E} & F & \textcolor{gregoriocolor}{F} & \textcolor{gregoriocolor}{G} & \textcolor{gregoriocolor}{H} & \textcolor{gregoriocolor}{M} & \textcolor{gregoriocolor}{N} & \textcolor{gregoriocolor}{P} \\
 23 & 24 & 25 & 26 & 27 & 28 & 28 & 29 & 30 & 1 & 2 & 3 \\
\end{tabularx}

\begin{paracol}{2}
\selectlanguage{latin}
\lettrine[lines=1]{F}{estum} beátæ Maríæ Vírginis Regínæ.
\switchcolumn
\selectlanguage{english}
\lettrine[lines=1]{F}{east} of the Queenship of the Blessed Virgin Mary.
\switchcolumn*
\selectlanguage{latin}
Romæ sanctæ Petroníllæ 
 Vírginis, fíliæ beáti Petri Apóstoli, quæ, conjúgium nóbilis viri Flacci 
 spernens, et, accéptis triduánis ad deliberándum indúciis, ínterim jejúniis 
 et oratiónibus vacans, tértia die, mox ut Christi Sacraméntum accépit, 
 emísit spíritum.
\switchcolumn
\selectlanguage{english}
At Rome, St. Petronilla, virgin, 
 disciple of the blessed apostle Peter. She refused to marry Flaccus, a 
 nobleman, and was granted three days for deliberation. She spent these 
 days in fasting and in prayer, and on the third day, after having received 
 the Sacrament of the Body of Christ, she yielded up her soul.
\switchcolumn*
\selectlanguage{latin}
Aquiléjæ sanctórum 
 Mártyrum fratrum Cántii, Cantiáni et Cantianíllæ, qui, cum essent ex 
 illústri Aniciórum progénie, sub Diocletiáno et Maximiáno Imperatóribus, ob 
 Christiánæ fídei constántiam, una cum pædagógo suo Proto, cápite plexi sunt.
\switchcolumn
\selectlanguage{english}
At Aquileia, the holy martyrs 
 Cantius, Cantian, and Cantianilla, members of one family, which belonged to 
 the illustrious line of the Anicii. For their attachment to the 
 Christian faith, they were condemned to capital punishment with their tutor, 
 Protus, in the time of Emperors Diocletian and Maximian.
\switchcolumn*
\selectlanguage{latin}
Túrribus, in Sardínia, sancti Crescentiáni Mártyris.
\switchcolumn
\selectlanguage{english}
At Torres in Sardinia, St. 
 Crescentian, martyr.
\switchcolumn*
\selectlanguage{latin}
Apud Comános, in Ponto, 
 sancti Hérmiæ mílitis, qui, sub Antoníno Imperatóre, de innúmeris et 
 sævíssimis torméntis divína ope liberátus, carníficem convértit ad Christum, 
 et ejúsdem martyrii corónæ partícipem fecit; quam tamen ipse primus, gládio 
 obtruncátus, accépit.
\switchcolumn
\selectlanguage{english}
At Comana in Pontus during the 
 reign of Emperor Antoninus, St. Hermias, a soldier. Being miraculously 
 delivered from many horrible torments, he converted his executioner to 
 Christ, and made him partaker of the crown which he was first to receive by 
 being beheaded.
\switchcolumn*
\selectlanguage{latin}
Verónæ sancti Lupicíni 
 Epíscopi.
\switchcolumn
\selectlanguage{english}
At Verona, St. Lupicinus, bishop.
\switchcolumn*
\selectlanguage{latin}
Romæ sancti Paschásii, 
 Diáconi et Confessóris, cujus méminit beátus Gregórius Papa.
\switchcolumn
\selectlanguage{english}
At Rome, St. Paschasius, deacon and 
 confessor, who is mentioned by blessed Pope Gregory.
\switchcolumn*
\selectlanguage{latin}
\end{paracol}

\setrunningtitles{Junius}{June}

% ---- martyrology/mart06/mart0601.htm
\needspace{10\baselineskip}
\begin{paracol}{2}
\selectlanguage{latin}
\begin{center}{\color{gregoriocolor} Kaléndis Júnii. 
 Luna\dots\ }\end{center}
\switchcolumn
\selectlanguage{english}
\begin{center}{\color{gregoriocolor} The First Day of 
 June. The\dots\ Day of the Moon.}\end{center}
\end{paracol}

\noindent\begin{tabularx}{\linewidth}{*{19}{>{\centering\arraybackslash}X}}
 \textcolor{gregoriocolor}{a} & \textcolor{gregoriocolor}{b} & \textcolor{gregoriocolor}{c} & \textcolor{gregoriocolor}{d} & \textcolor{gregoriocolor}{e} & \textcolor{gregoriocolor}{f} & \textcolor{gregoriocolor}{g} & \textcolor{gregoriocolor}{h} & \textcolor{gregoriocolor}{i} & \textcolor{gregoriocolor}{k} & \textcolor{gregoriocolor}{l} & \textcolor{gregoriocolor}{m} & \textcolor{gregoriocolor}{n} & \textcolor{gregoriocolor}{p} & \textcolor{gregoriocolor}{q} & \textcolor{gregoriocolor}{r} & \textcolor{gregoriocolor}{s} & \textcolor{gregoriocolor}{t} & \textcolor{gregoriocolor}{u} \\
 5 & 6 & 7 & 8 & 9 & 10 & 11 & 12 & 13 & 14 & 15 & 16 & 17 & 18 & 19 & 20 & 21 & 22 & 23 \\
\end{tabularx}
\vspace{0.5\baselineskip}
\noindent\begin{tabularx}{\linewidth}{*{12}{>{\centering\arraybackslash}X}}
 \textcolor{gregoriocolor}{A} & \textcolor{gregoriocolor}{B} & \textcolor{gregoriocolor}{C} & \textcolor{gregoriocolor}{D} & \textcolor{gregoriocolor}{E} & F & \textcolor{gregoriocolor}{F} & \textcolor{gregoriocolor}{G} & \textcolor{gregoriocolor}{H} & \textcolor{gregoriocolor}{M} & \textcolor{gregoriocolor}{N} & \textcolor{gregoriocolor}{P} \\
 24 & 25 & 26 & 27 & 28 & 29 & 29 & 30 & 1 & 2 & 3 & 4 \\
\end{tabularx}

\begin{paracol}{2}
\selectlanguage{latin}
\lettrine[lines=2]{S}{anctæ} Angelæ Meríci 
 Vírginis, ex tértio Ordine sancti Francísci; quæ fuit Institútrix Societátis 
 Vírginum sanctæ Ursulæ, et ad accipiéndam corónam immarcescíbilem, sexto 
 Kaléndas Februárii a cælésti Sponso vocáta est.
\switchcolumn
\selectlanguage{english}
\lettrine[lines=2]{S}{t.} Angela Merici, virgin of the 
 Third Order of St. Francis. She was the foundress of the Nuns of St. 
 Ursula, and was called by her heavenly Spouse on the 27th of January in 
 order to receive an incorruptible crown.
\switchcolumn*
\selectlanguage{latin}
Romæ sancti Juvéntii 
 Mártyris.
\switchcolumn
\selectlanguage{english}
At Rome, St. Juventius, martyr.
\switchcolumn*
\selectlanguage{latin}
Augustodúni sanctórum 
 Reveriáni Epíscopi, et Pauli Presbyteri, cum áliis decem, qui, sub Aureliáno 
 Príncipe, martyrio coronáti sunt.
\switchcolumn
\selectlanguage{english}
At Autun, the Saints Reverian, 
 bishop, and Paul, a priest, along with ten others, who were crowned with 
 martyrdom under Emperor Aurelian.
\switchcolumn*
\selectlanguage{latin}
Cæsaréæ, in Palæstína, 
 beáti Pámphili, Presbyteri et Mártyris, viri admirándæ sanctitátis et 
 doctrínæ, atque in páuperes munífici; qui, ob Christi fidem, in persecutióne 
 Galérii Maximiáni, primum, sub Urbáno Præside, cruciátus et in cárcerem 
 trusus, deínde, sub Firmiliáno, íterum revocátus ad pœnas, una cum áliis 
 martyrium consummávit. Passi sunt étiam tunc Valens Diáconus, et 
 Paulus, aliíque novem; quorum memória áliis diébus celebrátur.
\switchcolumn
\selectlanguage{english}
At Caesarea in Palestine, blessed 
 Pamphilus, priest and martyr, a man of remarkable sanctity and learning, and 
 great charity to the poor. In the persecution of Galerius Maximian, he 
 was tortured for the faith of Christ, under Governor Urbanus, and thrown 
 into prison. Later he was again subjected to torments under Firmilian, 
 and he completed his martyrdom with others. At the same time, there 
 suffered Valens, a deacon, and Paul, and nine others, whose commemoration 
 occurs on other days.
\switchcolumn*
\selectlanguage{latin}
In Cappadócia sancti 
 Thespésii Mártyris, qui, sub Alexándro Imperatóre et Simplício Præfécto, post 
 ália torménta, decollátus est.
\switchcolumn
\selectlanguage{english}
In Cappadocia, in the time of 
 Emperor Alexander and the prefect Simplicius, the holy martyr Thespesius, 
 who, after undergoing many torments, was beheaded.
\switchcolumn*
\selectlanguage{latin}
In Ægypto sanctórum 
 Mártyrum Ischyriónis, ductóris mílitum, et aliórum quinque mílitum; qui, sub 
 Diocletiáno Imperatóre, pro fide Christi, divérso mortis génere perémpti 
 sunt.
\switchcolumn
\selectlanguage{english}
In Egypt, under Emperor Diocletian, 
 the holy martyrs Ischyrion, a military officer, and five other soldiers, who 
 were put to death in various ways for the faith of Christ.
\switchcolumn*
\selectlanguage{latin}
Item sancti Firmi 
 Mártyris, qui, in persecutióne Maximiáni, acerbíssimis plagis afféctus, 
 lapídibus percússus, ac demum cápite cæsus est.
\switchcolumn
\selectlanguage{english}
Also, St. Firmus, martyr, who was 
 scourged most severely, struck with stones, and finally beheaded during the 
 persecution of Maximian.
\switchcolumn*
\selectlanguage{latin}
Perúsiæ sanctórum 
 Mártyrum Felíni et Gratiniáni mílitum, qui, sub Décio, váriis torméntis 
 cruciáti, martyrii palmam gloriósa morte percepérunt.
\switchcolumn
\selectlanguage{english}
At Perugia, the holy martyrs 
 Felinus and Gratinian, soldiers under Decius, who were tortured in several 
 ways, and by a glorious death won the palm of martyrdom.
\switchcolumn*
\selectlanguage{latin}
Bonóniæ sancti Próculi 
 Mártyris, qui sub Maximiáno Imperatóre passus est.
\switchcolumn
\selectlanguage{english}
At Bologna, St. Proculus, martyr, 
 who suffered under Emperor Maximian.
\switchcolumn*
\selectlanguage{latin}
Amériæ, in Umbria, 
 sancti Secúndi Mártyris, qui, sub Diocletiáno, in Tíberim projéctus, 
 martyrium consummávit.
\switchcolumn
\selectlanguage{english}
At Amelia in Umbria, in the reign 
 of Diocletian, St. Secundus, martyr, who fulfilled his martyrdom when thrown 
 into the Tiber.
\switchcolumn*
\selectlanguage{latin}
Apud Tiférnum, in 
 Umbria, sancti Crescentiáni, mílitis Románi, qui sub eódem Imperatóre, 
 martyrio coronátus est.
\switchcolumn
\selectlanguage{english}
At Tiferno in Umbria, St. 
 Crescentian, a Roman soldier, crowned with martyrdom under the same emperor.
\switchcolumn*
\selectlanguage{latin}
In monastério 
 Lirinénsi, in Gállia, sancti Caprásii Abbátis.
\switchcolumn
\selectlanguage{english}
In the monastery of Lerins, the 
 abbot St. Caprasius.
\switchcolumn*
\selectlanguage{latin}
In monastério Onniénsi, 
 apud Burgos, in Hispánia, sancti Enecónis, Abbátis Benedictíni, ob 
 sanctitátis et miraculórum glóriam illústris.
\switchcolumn
\selectlanguage{english}
At Burgos in Spain, in the 
 monastery of Onia, St. Eneco, Benedictine abbot, made illustrious by his 
 sanctity and miracles.
\switchcolumn*
\selectlanguage{latin}
Apud Montem Falcum, in 
 Umbria, sancti Fortunáti Presbyteri, virtútibus et miráculis clari.
\switchcolumn
\selectlanguage{english}
At Montefalco in Umbria, St. 
 Fortunatus, a priest renowned for his virtues and his miracles.
\switchcolumn*
\selectlanguage{latin}
Tréviris sancti 
 Simeónis Mónachi, qui a Benedícto Papa Nono in Sanctórum númerum relátus 
 est.
\switchcolumn
\selectlanguage{english}
At Treves, St. Simeon, a monk, whom 
 Pope Benedict IX numbered among the saints.
\switchcolumn*
\selectlanguage{latin}
\end{paracol}


% ---- martyrology/mart06/mart0602.htm
\needspace{10\baselineskip}
\begin{paracol}{2}
\selectlanguage{latin}
\begin{center}{\color{gregoriocolor} Quarto Nonas Júnii. 
 Luna\dots\ }\end{center}
\switchcolumn
\selectlanguage{english}
\begin{center}{\color{gregoriocolor} The Second Day of 
 June. The\dots\ Day of the Moon.}\end{center}
\end{paracol}

\noindent\begin{tabularx}{\linewidth}{*{19}{>{\centering\arraybackslash}X}}
 \textcolor{gregoriocolor}{a} & \textcolor{gregoriocolor}{b} & \textcolor{gregoriocolor}{c} & \textcolor{gregoriocolor}{d} & \textcolor{gregoriocolor}{e} & \textcolor{gregoriocolor}{f} & \textcolor{gregoriocolor}{g} & \textcolor{gregoriocolor}{h} & \textcolor{gregoriocolor}{i} & \textcolor{gregoriocolor}{k} & \textcolor{gregoriocolor}{l} & \textcolor{gregoriocolor}{m} & \textcolor{gregoriocolor}{n} & \textcolor{gregoriocolor}{p} & \textcolor{gregoriocolor}{q} & \textcolor{gregoriocolor}{r} & \textcolor{gregoriocolor}{s} & \textcolor{gregoriocolor}{t} & \textcolor{gregoriocolor}{u} \\
 6 & 7 & 8 & 9 & 10 & 11 & 12 & 13 & 14 & 15 & 16 & 17 & 18 & 19 & 20 & 21 & 22 & 23 & 24 \\
\end{tabularx}
\vspace{0.5\baselineskip}
\noindent\begin{tabularx}{\linewidth}{*{12}{>{\centering\arraybackslash}X}}
 \textcolor{gregoriocolor}{A} & \textcolor{gregoriocolor}{B} & \textcolor{gregoriocolor}{C} & \textcolor{gregoriocolor}{D} & \textcolor{gregoriocolor}{E} & F & \textcolor{gregoriocolor}{F} & \textcolor{gregoriocolor}{G} & \textcolor{gregoriocolor}{H} & \textcolor{gregoriocolor}{M} & \textcolor{gregoriocolor}{N} & \textcolor{gregoriocolor}{P} \\
 25 & 26 & 27 & 28 & 29 & 1 & 30 & 1 & 2 & 3 & 4 & 5 \\
\end{tabularx}

\begin{paracol}{2}
\selectlanguage{latin}
\lettrine[lines=2]{R}{omæ} natális sanctórum 
 Mártyrum Marcellíni Presbyteri, et Petri Exorcístæ, qui, sub Diocletiáno, 
 cum multos in cárcere ad fidem erudírent, ídeo post dira víncula et plúrima 
 torménta, a Seréno Júdice decolláti sunt in loco qui dicebátur Silva Nigra; 
 quæ deínde, in honórem Sanctórum, commutáto nómine Silva Cándida appelláta 
 est. Horum córpora in crypta, juxta sanctum Tibúrtium, sepúlta sunt; 
 eorúmque sepúlcrum sanctus Dámasus Papa vérsibus póstea exornávit.
\switchcolumn
\selectlanguage{english}
\lettrine[lines=2]{A}{t} Rome, the birthday of the holy 
 martyr Marcellinus, priest, and Peter, exorcist, who instructed in the faith 
 many persons kept in prison. Under Diocletian, they were loaded with 
 chains, and after enduring many torments, were beheaded by Judge Serenus, in 
 a place which was then called the Black Forest, but which was in their 
 honour afterwards known as the White Forest. Their bodies were buried 
 in a crypt near St. Tiburtius, and Pope St. Damasus composed an epitaph in 
 verse for their tomb.
\switchcolumn*
\selectlanguage{latin}
In Campánia sancti 
 Erásmi, Epíscopi et Mártyris. Hic, sub Diocletiáno Augústo, primum 
 plumbátis cæsus, deínde fústibus gráviter mactátus, post resína, súlphure, 
 plumbo, pice, cera oleóque perfúsus, illæsus appáruit; deínde Fórmiis, sub 
 Maximiáno, divérsis atque immaníssimis supplíciis íterum tortus, sed ad 
 confirmándum céteros a Deo servátus; tandem, vocánte Dómino, martyrio clarus, 
 sancto fine quiévit. Ipsíus autem corpus Cajétam póstea translátum 
 est.
\switchcolumn
\selectlanguage{english}
In Campania, during the reign of 
 Decius, St. Erasmus, bishop and martyr, who was first scourged with leaded 
 whips and then severely beaten with rods. He also had resin, 
 brimstone, lead, pitch, wax, and oil poured over him, without receiving any 
 injury. Afterwards, under Maximian, he was again subjected to various 
 and most horrible tortures at Mola, but still was preserved from death by 
 the power of God in order to confirm others in the faith. Finally, 
 celebrated for his sufferings, and called by God, he closed his life by a 
 peaceful and holy death. His body was afterwards transferred to Gaeta.
\switchcolumn*
\selectlanguage{latin}
Lugdúni, in Gállia, 
 sanctórum Mártyrum Pothíni Epíscopi, Sancti Diáconi, Vétii, Epágathi, Matúri, 
 Póntici, Bíblidis, Attali, Alexándri et Blandínæ, cum áliis multis; quorum 
 fórtia et iteráta certámina, témpore Marci Aurélii Antoníni et Lúcii Veri, 
 Ecclésiæ Lugdunénsis epístola, ad Ecclésias Asiæ et Phrygiæ scripta, 
 recénset. In his sancta Blandína, sexu infírmior, córpore imbecíllior, 
 conditióne dejéctior, diuturnióra et acerbióra certámina súbiens, et fortis 
 adhuc pérmanens, gládio juguláta, céteros secúta est, quos hortabátur ad 
 palmam.
\switchcolumn
\selectlanguage{english}
At Lyons, many holy martyrs (Photinus, 
 a bishop, Sanctus, a deacon, Vetius, Epagathus, Maturus, Ponticus, Biblis, 
 Attalus, Alexander, and Blandina, with many others), whose many valiant 
 trials in the time of Marcus Aurelius Antoninus and Lucius Verus are 
 recorded in a letter from the church at Lyons to the churches of Asia and 
 Phrygia. St. Blandina, one of these martyrs, was weaker by reason of 
 her sex, more infirm in body, and of a lower station in life, and yet she 
 encountered longer and more terrible trials than the rest. But 
 remaining unshaken, she was put to the sword, and followed those whom she 
 had exhorted to win the palm of martyrdom.
\switchcolumn*
\selectlanguage{latin}
In ínsula Proconnéso, 
 in Propóntide, sancti Nicéphori, Epíscopi Constantinopolitáni, qui, 
 paternárum traditiónum propugnátor acérrimus, pro cultu sanctárum Imáginum 
 se Leóni Arméno, Imperatóri Iconoclástæ, constánter oppósuit; a quo 
 mulctátus exsílio, ibídem, cum per quatuórdecim annos longum duxísset 
 martyrium, migrávit ad Dóminum.
\switchcolumn
\selectlanguage{english}
In the island of Marmara, in the 
 Sea of Marmara, St. Nicephorus, bishop of Constantinople. In defence 
 of the traditions of the Fathers and of the veneration of sacred images, he 
 set himself firmly against the Iconoclast emperor Leo the Armenian, by whom 
 he was sent into exile. There he underwent a long martyrdom of 
 fourteen years and then departed for the kingdom of God.
\switchcolumn*
\selectlanguage{latin}
Romæ sancti Eugénii 
 Primi, Papæ et Confessóris.
\switchcolumn
\selectlanguage{english}
At Rome, Pope St. Eugene I, 
 Confessor.
\switchcolumn*
\selectlanguage{latin}
Trani, in Apúlia, 
 sancti Nicolái Peregríni Confessóris, cujus mirácula in Concílio Románo, cui 
 beátus Urbánus Papa Secúndus præfuit, recitáta sunt.
\switchcolumn
\selectlanguage{english}
At Trani in Apulia, St. Nicholas 
 Peregrinus, confessor, whose miracles were recounted in the Roman Council 
 under Pope Urban II.
\switchcolumn*
\selectlanguage{latin}
\end{paracol}


% ---- martyrology/mart06/mart0603.htm
\needspace{10\baselineskip}
\begin{paracol}{2}
\selectlanguage{latin}
\begin{center}{\color{gregoriocolor} Tértio Nonas Júnii. 
 Luna\dots\ }\end{center}
\switchcolumn
\selectlanguage{english}
\begin{center}{\color{gregoriocolor} The Third Day of 
 June. The\dots\ Day of the Moon.}\end{center}
\end{paracol}

\noindent\begin{tabularx}{\linewidth}{*{19}{>{\centering\arraybackslash}X}}
 \textcolor{gregoriocolor}{a} & \textcolor{gregoriocolor}{b} & \textcolor{gregoriocolor}{c} & \textcolor{gregoriocolor}{d} & \textcolor{gregoriocolor}{e} & \textcolor{gregoriocolor}{f} & \textcolor{gregoriocolor}{g} & \textcolor{gregoriocolor}{h} & \textcolor{gregoriocolor}{i} & \textcolor{gregoriocolor}{k} & \textcolor{gregoriocolor}{l} & \textcolor{gregoriocolor}{m} & \textcolor{gregoriocolor}{n} & \textcolor{gregoriocolor}{p} & \textcolor{gregoriocolor}{q} & \textcolor{gregoriocolor}{r} & \textcolor{gregoriocolor}{s} & \textcolor{gregoriocolor}{t} & \textcolor{gregoriocolor}{u} \\
 7 & 8 & 9 & 10 & 11 & 12 & 13 & 14 & 15 & 16 & 17 & 18 & 19 & 20 & 21 & 22 & 23 & 24 & 25 \\
\end{tabularx}
\vspace{0.5\baselineskip}
\noindent\begin{tabularx}{\linewidth}{*{12}{>{\centering\arraybackslash}X}}
 \textcolor{gregoriocolor}{A} & \textcolor{gregoriocolor}{B} & \textcolor{gregoriocolor}{C} & \textcolor{gregoriocolor}{D} & \textcolor{gregoriocolor}{E} & F & \textcolor{gregoriocolor}{F} & \textcolor{gregoriocolor}{G} & \textcolor{gregoriocolor}{H} & \textcolor{gregoriocolor}{M} & \textcolor{gregoriocolor}{N} & \textcolor{gregoriocolor}{P} \\
 26 & 27 & 28 & 29 & 1 & 2 & 1 & 2 & 3 & 4 & 5 & 6 \\
\end{tabularx}

\begin{paracol}{2}
\selectlanguage{latin}
\lettrine[lines=2]{A}{rétii,} in Túscia, 
 sanctórum Mártyrum Pergentíni et Laurentíni fratrum, qui, in persecutióne 
 Décii, sub Tibúrtio Præside, cum essent púeri, ibídem, post dira supplícia 
 toleráta et magna mirácula osténsa, gládio cæsi sunt.
\switchcolumn
\selectlanguage{english}
\lettrine[lines=2]{A}{t} Arezzo in Tuscany, during the 
 persecution of Decius, under Governor Tiburtius, the holy martyrs 
 Pergentinus and Laurentinus, brothers, who being as yet children, were put 
 to the sword after they had endured cruel torments and performed many 
 miracles.
\switchcolumn*
\selectlanguage{latin}
Constantinópoli 
 sanctórum Mártyrum Lucilliáni et quátuor puerórum, scílicet Cláudii, Hypátii, 
 Pauli et Dionysii. His cum púeris Lucilliánus, ex idolórum sacerdóte 
 Christiánus factus, in fornácem, post vária torménta, injéctus est, sed, 
 flamma imbre exstíncta, omnes illæsi evasérunt; dénique, ipse cruci affíxus, 
 púeri autem gládio obtruncáti, sub Silváno Præside, consummáti sunt.
\switchcolumn
\selectlanguage{english}
At Constantinople, the holy martyrs 
 Lucillian and four boys, Claudius, Hypatius, Paul, and Denis. 
 Lucillian, formerly a pagan priest, but now a Christian, was cast with them 
 into a furnace after undergoing many torments, but the flames were 
 extinguished by the rain and all escaped injury. Finally their lives 
 were ended under the governor Silvanus, Lucillian by crucifixion, the 
 children by beheading.
\switchcolumn*
\selectlanguage{latin}
Córdubæ, in Hispánia, 
 beáti Isaac Mónachi, qui, pro Christi fide, gládio necátus est.
\switchcolumn
\selectlanguage{english}
At Cordova in Spain, blessed Isaac, 
 a monk who was slain by the sword for the faith of Christ.
\switchcolumn*
\selectlanguage{latin}
Constantinópoli sanctæ 
 Paulæ, Vírginis et Mártyris, quæ, cum prædictórum Mártyrum Lucilliáni ac 
 Sociórum sánguinem collígeret, ídeo, comprehénsa, virgis percússa, et in 
 ignem conjécta sed liberáta, demum et ipsa eódem loco, ubi sanctus 
 Lucilliánus crucifíxus fúerat, decolláta est.
\switchcolumn
\selectlanguage{english}
At Constantinople, St. Paula, virgin 
 and martyr, who was arrested while gathering the blood of the martyrs just 
 mentioned. She was beaten with rods and thrown into the fire, but was 
 delivered from it. She was at length beheaded in the same place where 
 St. Lucillian had been crucified.
\switchcolumn*
\selectlanguage{latin}
Carthágine sancti 
 Cæcílii Presbyteri, qui sanctum Cypriánum ad Christi fidem perdúxit.
\switchcolumn
\selectlanguage{english}
At Carthage, St. Caecilius, the 
 priest who converted St. Cyprian to the faith of Christ.
\switchcolumn*
\selectlanguage{latin}
In território 
 Aurelianénsi sancti Liphárdi, Presbyteri et Confessóris.
\switchcolumn
\selectlanguage{english}
In the diocese of Orleans, St. 
 Lifard, priest and confessor.
\switchcolumn*
\selectlanguage{latin}
Lucæ, in Túscia, sancti 
 Davíni Confessóris.
\switchcolumn
\selectlanguage{english}
At Lucca in Tuscany, St. Davinus, 
 confessor.
\switchcolumn*
\selectlanguage{latin}
Anágniæ sanctæ Olívæ 
 Vírginis.
\switchcolumn
\selectlanguage{english}
At Anagni, St. Olive, virgin.
\switchcolumn*
\selectlanguage{latin}
Lutétiæ Parisiórum 
 sanctæ Clotíldis Regínæ, cujus précibus vir ejus Clodovéus, Rex Francórum, 
 Christi fidem suscépit.
\switchcolumn
\selectlanguage{english}
At Paris, St. Clotilde, queen, by 
 whose prayers her husband, King Clovis, was converted to the faith of 
 Christ.
\switchcolumn*
\selectlanguage{latin}
\end{paracol}


% ---- martyrology/mart06/mart0604.htm
\needspace{10\baselineskip}
\begin{paracol}{2}
\selectlanguage{latin}
\begin{center}{\color{gregoriocolor} Prídie Nonas Júnii. 
 Luna\dots\ }\end{center}
\switchcolumn
\selectlanguage{english}
\begin{center}{\color{gregoriocolor} The Fourth Day of 
 June. The\dots\ Day of the Moon.}\end{center}
\end{paracol}

\noindent\begin{tabularx}{\linewidth}{*{19}{>{\centering\arraybackslash}X}}
 \textcolor{gregoriocolor}{a} & \textcolor{gregoriocolor}{b} & \textcolor{gregoriocolor}{c} & \textcolor{gregoriocolor}{d} & \textcolor{gregoriocolor}{e} & \textcolor{gregoriocolor}{f} & \textcolor{gregoriocolor}{g} & \textcolor{gregoriocolor}{h} & \textcolor{gregoriocolor}{i} & \textcolor{gregoriocolor}{k} & \textcolor{gregoriocolor}{l} & \textcolor{gregoriocolor}{m} & \textcolor{gregoriocolor}{n} & \textcolor{gregoriocolor}{p} & \textcolor{gregoriocolor}{q} & \textcolor{gregoriocolor}{r} & \textcolor{gregoriocolor}{s} & \textcolor{gregoriocolor}{t} & \textcolor{gregoriocolor}{u} \\
 8 & 9 & 10 & 11 & 12 & 13 & 14 & 15 & 16 & 17 & 18 & 19 & 20 & 21 & 22 & 23 & 24 & 25 & 26 \\
\end{tabularx}
\vspace{0.5\baselineskip}
\noindent\begin{tabularx}{\linewidth}{*{12}{>{\centering\arraybackslash}X}}
 \textcolor{gregoriocolor}{A} & \textcolor{gregoriocolor}{B} & \textcolor{gregoriocolor}{C} & \textcolor{gregoriocolor}{D} & \textcolor{gregoriocolor}{E} & F & \textcolor{gregoriocolor}{F} & \textcolor{gregoriocolor}{G} & \textcolor{gregoriocolor}{H} & \textcolor{gregoriocolor}{M} & \textcolor{gregoriocolor}{N} & \textcolor{gregoriocolor}{P} \\
 27 & 28 & 29 & 1 & 2 & 3 & 2 & 3 & 4 & 5 & 6 & 7 \\
\end{tabularx}

\begin{paracol}{2}
\selectlanguage{latin}
\lettrine[lines=2]{A}{gnóni,} in Aprútio 
 citerióre, sancti Francísci, ex nóbili Neapolitána família Carácciolo, 
 Confessóris, Congregatiónis Clericórum Regulárium Minórum Fundatóris, qui 
 mira in Deum et próximum caritáte et ardentíssimo sacræ Eucharístiæ cultus 
 propagándi stúdio flagrávit; atque a Pio Séptimo, Pontífice Máximo, 
 Sanctórum cánoni adscríptus est. Ipsíus corpus Neápolim, in Campánia, 
 translátum fuit, ibíque religiosíssime cólitur.
\switchcolumn
\selectlanguage{english}
\lettrine[lines=2]{A}{t} Agnone in Abruzzo, St. Francis of 
 the noble Neapolitan family Caracciolo, confessor, and founder of the 
 Congregation of Minor Clerks Regular. He burned with an admirable love 
 of God and of neighbour, and a most ardent desire to spread devotion to the 
 Most Holy Eucharist. His body was taken to Naples in Campania, where 
 it is religiously honoured. He was inscribed in the catalogue of the 
 saints by Pius VII.
\switchcolumn*
\selectlanguage{latin}
Romæ sanctórum Mártyrum 
 Arétii et Daciáni.
\switchcolumn
\selectlanguage{english}
At Rome, the holy martyrs Aretius 
 and Dacian.
\switchcolumn*
\selectlanguage{latin}
Sísciæ, in Illyrico, 
 sancti Quiríni Epíscopi, qui, sub Galério Præside, pro fide Christi (ut 
 Prudéntius scribit), molári saxo ad collum ligáto, in flumen præcipitátus 
 est; sed, lápide supernatánte, cum circumstántes Christiános, ne ejus 
 terreréntur supplício neve titubárent in fide, diu fuísset hortátus, ipse, 
 ut martyrii glóriam assequerétur, précibus a Deo, ut mergerétur, obtínuit.
\switchcolumn
\selectlanguage{english}
At Sissek in Illyria, in the time of 
 Governor Galerius, St. Quirinus, bishop. Prudentius relates that for 
 the faith of Christ he was thrown into a river with a millstone tied to his 
 neck. But the stone floated, and he for a long time exhorted the 
 Christians who were present not to be terrified by his punishment, nor to 
 waver in the faith, and then obtained of God by his prayers that he should 
 be drowned in order to attain the glory of martyrdom.
\switchcolumn*
\selectlanguage{latin}
Medioláni, sancti 
 Clatéi, Epíscopi Brixiénsis et Mártyris, qui, sub Neróne Imperatóre, jussu 
 Præfécti urbis illíus tentus, et, cum renuntiáre Christo nollet, multis 
 verbéribus afflíctus et cápite obtruncátus est.
\switchcolumn
\selectlanguage{english}
At Milan, in the reign of Emperor 
 Nero, St. Clateus, bishop of Brescia and martyr. By order of the 
 prefect of the city he was arrested, and when he would not deny Christ he 
 was cruelly scourged and beheaded.
\switchcolumn*
\selectlanguage{latin}
In Pannónia sanctórum 
 Mártyrum Rútili et Sociórum.
\switchcolumn
\selectlanguage{english}
In Hungary, the holy martyrs Rutilus 
 and his companions.
\switchcolumn*
\selectlanguage{latin}
Tíbure sancti Quiríni 
 Mártyris.
\switchcolumn
\selectlanguage{english}
At Tivoli, St Quirinus, martyr.
\switchcolumn*
\selectlanguage{latin}
Atrébati, in Gálliis, 
 sanctæ Saturnínæ, Vírginis et Mártyris.
\switchcolumn
\selectlanguage{english}
At Arras in France, St. Saturnina, 
 virgin and martyr.
\switchcolumn*
\selectlanguage{latin}
Constantinópoli sancti 
 Metróphanis, Epíscopi et Confessóris insígnis.
\switchcolumn
\selectlanguage{english}
At Constantinople, St. Metrophanes, 
 bishop and renowned confessor.
\switchcolumn*
\selectlanguage{latin}
Milévi, in Numídia, 
 sancti Optáti Epíscopi, doctrína et sanctitáte conspícui, quem sancti 
 Ecclésiæ Patres Augustínus et Fulgéntius suis láudibus celebrárunt.
\switchcolumn
\selectlanguage{english}
At Milevi in Numidia, St. Optatus, 
 bishop, celebrated for his learning and holiness. The holy Fathers of 
 the Church, Augustine and Fulgentius, prasied him highly.
\switchcolumn*
\selectlanguage{latin}
Verónæ sancti 
 Alexándri Epíscopi.
\switchcolumn
\selectlanguage{english}
At Verona, St. Alexander, bishop.
\switchcolumn*
\selectlanguage{latin}
\end{paracol}


% ---- martyrology/mart06/mart0605.htm
\needspace{10\baselineskip}
\begin{paracol}{2}
\selectlanguage{latin}
\begin{center}{\color{gregoriocolor} Nonis Júnii. 
 Luna\dots\ }\end{center}
\switchcolumn
\selectlanguage{english}
\begin{center}{\color{gregoriocolor} The Fifth Day of 
 June. The\dots\ Day of the Moon.}\end{center}
\end{paracol}

\noindent\begin{tabularx}{\linewidth}{*{19}{>{\centering\arraybackslash}X}}
 \textcolor{gregoriocolor}{a} & \textcolor{gregoriocolor}{b} & \textcolor{gregoriocolor}{c} & \textcolor{gregoriocolor}{d} & \textcolor{gregoriocolor}{e} & \textcolor{gregoriocolor}{f} & \textcolor{gregoriocolor}{g} & \textcolor{gregoriocolor}{h} & \textcolor{gregoriocolor}{i} & \textcolor{gregoriocolor}{k} & \textcolor{gregoriocolor}{l} & \textcolor{gregoriocolor}{m} & \textcolor{gregoriocolor}{n} & \textcolor{gregoriocolor}{p} & \textcolor{gregoriocolor}{q} & \textcolor{gregoriocolor}{r} & \textcolor{gregoriocolor}{s} & \textcolor{gregoriocolor}{t} & \textcolor{gregoriocolor}{u} \\
 9 & 10 & 11 & 12 & 13 & 14 & 15 & 16 & 17 & 18 & 19 & 20 & 21 & 22 & 23 & 24 & 25 & 26 & 27 \\
\end{tabularx}
\vspace{0.5\baselineskip}
\noindent\begin{tabularx}{\linewidth}{*{12}{>{\centering\arraybackslash}X}}
 \textcolor{gregoriocolor}{A} & \textcolor{gregoriocolor}{B} & \textcolor{gregoriocolor}{C} & \textcolor{gregoriocolor}{D} & \textcolor{gregoriocolor}{E} & F & \textcolor{gregoriocolor}{F} & \textcolor{gregoriocolor}{G} & \textcolor{gregoriocolor}{H} & \textcolor{gregoriocolor}{M} & \textcolor{gregoriocolor}{N} & \textcolor{gregoriocolor}{P} \\
 28 & 29 & 1 & 2 & 3 & 4 & 3 & 4 & 5 & 6 & 7 & 8 \\
\end{tabularx}

\begin{paracol}{2}
\selectlanguage{latin}
\lettrine[lines=2]{I}{n} Frísia sancti 
 Bonifátii, Epíscopi Moguntíni et Mártyris. Hic de Anglia Romam venit, 
 índeque a beáto Gregório Papa Secúndo in Germániam missus est ut Christi 
 fidem illis géntibus evangelizáret, et, cum máximum ibi multitúdinem, 
 præsértim Frísonum, Christiánæ religióni subjugásset, Germanórum Apóstolus 
 méruit appellári; novíssime in Frísia, a furéntibus Gentílibus gládio 
 perémptus, una cum Eóbano Coepíscopo et quibúsdam áliis servis Dei, martyrium 
 consummávit.
\switchcolumn
\selectlanguage{english}
\lettrine[lines=2]{I}{n} Friesland, St. Boniface, bishop 
 of Mainz, and martyr. He went from England to Rome, and was then sent 
 by Pope Gregory II to Germany to preach the faith of Christ to the people of 
 that country. After converting large multitudes to the Christian 
 religion, especially in Friesland, he merited the title Apostle of the 
 Germans. His martyrdom was fulfilled by being put to the sword by the 
 furious heathens, along with his fellow bishop Eobanus and some other 
 servants of God.
\switchcolumn*
\selectlanguage{latin}
Tyri, in Phœnícia, 
 sancti Doróthei Presbyteri, qui, sub Diocletiáno, multa passus est; et, 
 usque ad Juliáni témpora supérstes, sub eo, annum agens séptimum supra 
 centésimum, venerándam senéctam martyrio honestávit.
\switchcolumn
\selectlanguage{english}
At Tyre, St. Dorotheus, a priest, 
 who suffered greatly under Diocletian, but survived until the reign of 
 Julian, under whom his venerable age of one hundred and seven years was 
 crowned with martyrdom.
\switchcolumn*
\selectlanguage{latin}
In Ægypto natális 
 sanctórum Mártyrum Marciáni, Nicánoris, Apollónii et aliórum, qui, in 
 persecutióne Galérii Maximiáni, illústre martyrium consummárunt.
\switchcolumn
\selectlanguage{english}
In Egypt, the birthday of the holy 
 martyrs Marcian, Nicanor, Apollonius, and others, who suffered a glorious 
 martyrdom.
\switchcolumn*
\selectlanguage{latin}
Perúsiæ sanctórum 
 Mártyrum Floréntii, Juliáni, Cyríaci, Marcellíni et Faustíni, qui omnes, in 
 persecutióne Décii Imperatóris, cápite cæsi sunt.
\switchcolumn
\selectlanguage{english}
At Perugia, the holy martyrs 
 Florentius, Julian, Cyriacus, Marcellinus, and Faustinus, who were beheaded 
 in the persecution of Decius.
\switchcolumn*
\selectlanguage{latin}
Córdubæ, in Hispánia, 
 beáti Sáncii adolescéntis, qui, etsi in aula régia educátus, pro Christi 
 tamen fide, in persecutióne Arábica, martyrium subíre non dubitávit.
\switchcolumn
\selectlanguage{english}
At Cordova in Spain, blessed Sancho, 
 a youth brought up in the royal court, who did not hesitate to undergo 
 martyrdom for the faith of Christ during the persecution by the Arabs.
\switchcolumn*
\selectlanguage{latin}
Cæsaréæ, in Palæstína, 
 pássio sanctárum Zenáidis, Cyriæ, Valériæ et Márciæ; quæ, per multa torménta, 
 gaudéntes ad martyrium pervenérunt.
\switchcolumn
\selectlanguage{english}
At Caesarea in Palestine, the 
 martyrdom of the Saints Zenaides, Cyria, Valeria, and Marcia, who joyfully 
 attained martyrdom through many torments.
\switchcolumn*
\selectlanguage{latin}
\end{paracol}


% ---- martyrology/mart06/mart0606.htm
\needspace{10\baselineskip}
\begin{paracol}{2}
\selectlanguage{latin}
\begin{center}{\color{gregoriocolor} Octávo Idus Júnii. 
 Luna\dots\ }\end{center}
\switchcolumn
\selectlanguage{english}
\begin{center}{\color{gregoriocolor} The Sixth Day of 
 June. The\dots\ Day of the Moon.}\end{center}
\end{paracol}

\noindent\begin{tabularx}{\linewidth}{*{19}{>{\centering\arraybackslash}X}}
 \textcolor{gregoriocolor}{a} & \textcolor{gregoriocolor}{b} & \textcolor{gregoriocolor}{c} & \textcolor{gregoriocolor}{d} & \textcolor{gregoriocolor}{e} & \textcolor{gregoriocolor}{f} & \textcolor{gregoriocolor}{g} & \textcolor{gregoriocolor}{h} & \textcolor{gregoriocolor}{i} & \textcolor{gregoriocolor}{k} & \textcolor{gregoriocolor}{l} & \textcolor{gregoriocolor}{m} & \textcolor{gregoriocolor}{n} & \textcolor{gregoriocolor}{p} & \textcolor{gregoriocolor}{q} & \textcolor{gregoriocolor}{r} & \textcolor{gregoriocolor}{s} & \textcolor{gregoriocolor}{t} & \textcolor{gregoriocolor}{u} \\
 10 & 11 & 12 & 13 & 14 & 15 & 16 & 17 & 18 & 19 & 20 & 21 & 22 & 23 & 24 & 25 & 26 & 27 & 28 \\
\end{tabularx}
\vspace{0.5\baselineskip}
\noindent\begin{tabularx}{\linewidth}{*{12}{>{\centering\arraybackslash}X}}
 \textcolor{gregoriocolor}{A} & \textcolor{gregoriocolor}{B} & \textcolor{gregoriocolor}{C} & \textcolor{gregoriocolor}{D} & \textcolor{gregoriocolor}{E} & F & \textcolor{gregoriocolor}{F} & \textcolor{gregoriocolor}{G} & \textcolor{gregoriocolor}{H} & \textcolor{gregoriocolor}{M} & \textcolor{gregoriocolor}{N} & \textcolor{gregoriocolor}{P} \\
 29 & 1 & 2 & 3 & 4 & 5 & 4 & 5 & 6 & 7 & 8 & 9 \\
\end{tabularx}

\begin{paracol}{2}
\selectlanguage{latin}
\lettrine[lines=2]{M}{agdebúrgi} sancti 
 Norbérti, ejúsdem civitátis Epíscopi et Confessóris, qui Fundátor exstitit 
 Ordinis Præmonstraténsis.
\switchcolumn
\selectlanguage{english}
\lettrine[lines=2]{A}{t} Magdeburg, St. Norbert, bishop of 
 that city, confessor. He was the founder of the Premonstratensian 
 Order.
\switchcolumn*
\selectlanguage{latin}
Cæsaréæ, in Palæstína, 
 natális beáti Philíppi, qui fuit unus de septem primis Diáconis. Hic, 
 signis et prodígiis clarus, Samaríam ad Christi fidem convértit, et Regínæ 
 Æthíopum Cándacis Eunúchum baptizávit, ac demum apud Cæsaréam requiévit. 
 Juxta ipsum tres Vírgines, ejus fíliæ ac Prophetíssæ, tumulátæ jacent; nam 
 quarta fília ejus, plena Spíritu Sancto, Ephesi occúbuit.
\switchcolumn
\selectlanguage{english}
At Caesarea in Palestine, the 
 birthday of blessed Philip, one of the first seven deacons. He was 
 renowned for miracles and prodigies. He converted Samaria to the faith 
 of Christ, baptized the eunuch of Candace, queen of Ethiopia, and finally 
 rested in peace at Caesarea. Near him are buried three of his 
 daughters, virgins and prophetesses. His fourth daughter died at 
 Ephesus, filled with the Holy Ghost.
\switchcolumn*
\selectlanguage{latin}
Romæ sancti Artémii, 
 cum uxóre sua Cándida, et fília Paulína. Ex his Artémius, ad 
 prædicatiónem et mirácula sancti Petri Exorcístæ, ad Christum convérsus, et, 
 cum omni domo sua, a sancto Marcellíno Presbytero baptizátus, Seréni Judicis 
 jussu plumbátis cæsus et gládio percússus est; uxor vero ejus et fília, in 
 cryptam impúlsæ, lapídibus ruderibúsque sunt óbrutæ.
\switchcolumn
\selectlanguage{english}
At Rome, St. Artemius, with his wife 
 Candida and his daughter Paulina. Artemius became a believer through 
 the preaching and miracles of St. Peter the Exorcist, who was baptized with 
 all his household by the priest St. Marcellinus. By order of Judge 
 Serenus, he was scourged with leaded whips, and then slain with the sword. 
 His wife and daughter were forced into a pit and covered with stones and 
 earth.
\switchcolumn*
\selectlanguage{latin}
In agro Bononiénsi 
 sancti Alexándri, Epíscopi Fæsuláni et Mártyris; qui, rédiens ex urbe Papía, 
 ubi Ecclésiæ suæ bona apud Longobardórum Regem ab usurpatóribus vindicáverat, 
 ab his in Rhenum flúvium est dejéctus et aquis præfocátus.
\switchcolumn
\selectlanguage{english}
In the district of Bologna, St. 
 Alexander, bishop of Fiesole and martyr. While returning from the town 
 of Pavia, where he had defended the title to the goods of his church before 
 the Lombard king against those taking them away, he was seized by the 
 usurpers, cast into the Rhine river, and drowned.
\switchcolumn*
\selectlanguage{latin}
Tarsi, in Cilícia, 
 sanctórum Mártyrum vigínti, qui, tempóribus Diocletiáni et Maximiáni, sub 
 Simplício Júdice, per divérsa torménta glorificavérunt Deum in corpóribus 
 suis.
\switchcolumn
\selectlanguage{english}
At Tarsus in Cilicia, in the time of 
 Emperors Diocletian and Maximian, and the governor Simplicius, twenty holy 
 martyrs, who, through various torments to their bodies, glorified God.
\switchcolumn*
\selectlanguage{latin}
Noviodúni, in Gálliis, 
 sanctórum Mártyrum Amántii, Alexándri et Sociórum.
\switchcolumn
\selectlanguage{english}
At Noyon in France, the holy martyrs 
 Amantius, Alexander, and their companions.
\switchcolumn*
\selectlanguage{latin}
Medioláni deposítio 
 sancti Eustórgii Secúndi, Epíscopi et Confessóris.
\switchcolumn
\selectlanguage{english}
At Milan, the death of St. 
 Eustorgius II, bishop and confessor.
\switchcolumn*
\selectlanguage{latin}
Verónæ sancti Joánnis 
 Epíscopi.
\switchcolumn
\selectlanguage{english}
At Verona, the bishop St. John.
\switchcolumn*
\selectlanguage{latin}
Vesontióne, in Gálliis, 
 sancti Cláudii Epíscopi.
\switchcolumn
\selectlanguage{english}
At Besançon, France, the bishop St. 
 Claudius.
\switchcolumn*
\selectlanguage{latin}
\end{paracol}


% ---- martyrology/mart06/mart0607.htm
\needspace{10\baselineskip}
\begin{paracol}{2}
\selectlanguage{latin}
\begin{center}{\color{gregoriocolor} Séptimo Idus Júnii. 
 Luna\dots\ }\end{center}
\switchcolumn
\selectlanguage{english}
\begin{center}{\color{gregoriocolor} The Seventh Day of 
 June. The\dots\ Day of the Moon.}\end{center}
\end{paracol}

\noindent\begin{tabularx}{\linewidth}{*{19}{>{\centering\arraybackslash}X}}
 \textcolor{gregoriocolor}{a} & \textcolor{gregoriocolor}{b} & \textcolor{gregoriocolor}{c} & \textcolor{gregoriocolor}{d} & \textcolor{gregoriocolor}{e} & \textcolor{gregoriocolor}{f} & \textcolor{gregoriocolor}{g} & \textcolor{gregoriocolor}{h} & \textcolor{gregoriocolor}{i} & \textcolor{gregoriocolor}{k} & \textcolor{gregoriocolor}{l} & \textcolor{gregoriocolor}{m} & \textcolor{gregoriocolor}{n} & \textcolor{gregoriocolor}{p} & \textcolor{gregoriocolor}{q} & \textcolor{gregoriocolor}{r} & \textcolor{gregoriocolor}{s} & \textcolor{gregoriocolor}{t} & \textcolor{gregoriocolor}{u} \\
 11 & 12 & 13 & 14 & 15 & 16 & 17 & 18 & 19 & 20 & 21 & 22 & 23 & 24 & 25 & 26 & 27 & 28 & 29 \\
\end{tabularx}
\vspace{0.5\baselineskip}
\noindent\begin{tabularx}{\linewidth}{*{12}{>{\centering\arraybackslash}X}}
 \textcolor{gregoriocolor}{A} & \textcolor{gregoriocolor}{B} & \textcolor{gregoriocolor}{C} & \textcolor{gregoriocolor}{D} & \textcolor{gregoriocolor}{E} & F & \textcolor{gregoriocolor}{F} & \textcolor{gregoriocolor}{G} & \textcolor{gregoriocolor}{H} & \textcolor{gregoriocolor}{M} & \textcolor{gregoriocolor}{N} & \textcolor{gregoriocolor}{P} \\
 1 & 2 & 3 & 4 & 5 & 6 & 5 & 6 & 7 & 8 & 9 & 10 \\
\end{tabularx}

\begin{paracol}{2}
\selectlanguage{latin}
\lettrine[lines=2]{C}{onstantinópoli} natális 
 sancti Pauli, ejúsdem civitátis Epíscopi, qui sæpe ab Ariánis ob fidem 
 cathólicam pulsus, et a sancto Júlio Primo, Románo Pontífice, restitútus est; 
 ac demum, ab Ariáno Imperatóre Constántio relegátus ad Cucúsum, Cappadóciæ 
 oppídulum, ibídem, Arianórum insídiis crudéliter strangulátus, ad cæléstia 
 regna migrávit. Ipsíus autem corpus, Theodósio Imperatóre, 
 Constantinópolim summo honóre, translátum fuit.
\switchcolumn
\selectlanguage{english}
\lettrine[lines=2]{A}{t} Constantinople, the birthday of 
 St. Paul, bishop of that city. For the Catholic faith, he was often 
 driven out of his see by the Arians, but restored to it by the Roman 
 Pontiff, St. Julius I. Finally the Arian emperor Constantius banished 
 him to Cucusum, a small town of Cappadocia. There, by the intrigue of 
 the Arians, he was barbarously strangled, and thus departed for the heavenly 
 kingdom. His body was taken to Constantinople with the greatest honour 
 during the reign of Emperor Theodosius.
\switchcolumn*
\selectlanguage{latin}
Córdubæ, in Hispánia, 
 sanctórum Monachórum et Mártyrum Petri Presbyteri, Wallabónsi Diáconi, 
 Sabiniáni, Wistremúndi, Habéntii et Jeremíæ, qui pro Christo, in 
 persecutióne Arábica, sunt juguláti.
\switchcolumn
\selectlanguage{english}
At Cordova in Spain, the holy 
 martyrs Peter, a priest, Wallabonsus, a deacon, Sabinianus, Wistremund, 
 Habentius, and Jeremias, all of whom were monks. Their throats were 
 cut at the time of the Arab persecution because they had confessed Christ.
\switchcolumn*
\selectlanguage{latin}
Hermópoli, in Ægypto, 
 sancti Lycariónis Mártyris, qui laniátus, virgis férreis ignítis cæsus, 
 áliaque sævíssima passus est, ac demum, gládio percússus, martyrium 
 consummávit.
\switchcolumn
\selectlanguage{english}
At Hermopolis in Egypt, St. Licarion, 
 martyr, who had his body lacerated, was scourged with heated iron rods, and 
 endured other horrible torments, after which his martyrdom was completed by 
 beheading.
\switchcolumn*
\selectlanguage{latin}
Placéntiæ sancti 
 Antónii Maríæ Gianélli, Bobiénsis Epíscopi, Fundatóris Congregatiónis 
 Filiárum Maríæ sanctíssimæ ab Horto nuncupatárum, quem Pius Papa Duodécimus 
 inter sanctos Cælites adnumerávit.
\switchcolumn
\selectlanguage{english}
At Placentia, St. Anthony Mary 
 Gianelli, bishop of Bobbio, and founder of the Congregation of Sisters of 
 our Lady of the Garden. Pope Pius XII numbered him among the saints of 
 heaven.
\switchcolumn*
\selectlanguage{latin}
In Anglia sancti 
 Robérti Abbátis, ex Ordine Cisterciénsi.
\switchcolumn
\selectlanguage{english}
In England, St. Robert, an abbot of 
 the Cistercian Order.
\switchcolumn*
\selectlanguage{latin}
\end{paracol}


% ---- martyrology/mart06/mart0608.htm
\needspace{10\baselineskip}
\begin{paracol}{2}
\selectlanguage{latin}
\begin{center}{\color{gregoriocolor} Sexto Idus Júnii. 
 Luna\dots\ }\end{center}
\switchcolumn
\selectlanguage{english}
\begin{center}{\color{gregoriocolor} The Eighth Day of 
 June. The\dots\ Day of the Moon.}\end{center}
\end{paracol}

\noindent\begin{tabularx}{\linewidth}{*{19}{>{\centering\arraybackslash}X}}
 \textcolor{gregoriocolor}{a} & \textcolor{gregoriocolor}{b} & \textcolor{gregoriocolor}{c} & \textcolor{gregoriocolor}{d} & \textcolor{gregoriocolor}{e} & \textcolor{gregoriocolor}{f} & \textcolor{gregoriocolor}{g} & \textcolor{gregoriocolor}{h} & \textcolor{gregoriocolor}{i} & \textcolor{gregoriocolor}{k} & \textcolor{gregoriocolor}{l} & \textcolor{gregoriocolor}{m} & \textcolor{gregoriocolor}{n} & \textcolor{gregoriocolor}{p} & \textcolor{gregoriocolor}{q} & \textcolor{gregoriocolor}{r} & \textcolor{gregoriocolor}{s} & \textcolor{gregoriocolor}{t} & \textcolor{gregoriocolor}{u} \\
 12 & 13 & 14 & 15 & 16 & 17 & 18 & 19 & 20 & 21 & 22 & 23 & 24 & 25 & 26 & 27 & 28 & 29 & 1 \\
\end{tabularx}
\vspace{0.5\baselineskip}
\noindent\begin{tabularx}{\linewidth}{*{12}{>{\centering\arraybackslash}X}}
 \textcolor{gregoriocolor}{A} & \textcolor{gregoriocolor}{B} & \textcolor{gregoriocolor}{C} & \textcolor{gregoriocolor}{D} & \textcolor{gregoriocolor}{E} & F & \textcolor{gregoriocolor}{F} & \textcolor{gregoriocolor}{G} & \textcolor{gregoriocolor}{H} & \textcolor{gregoriocolor}{M} & \textcolor{gregoriocolor}{N} & \textcolor{gregoriocolor}{P} \\
 2 & 3 & 4 & 5 & 6 & 7 & 6 & 7 & 8 & 9 & 10 & 11 \\
\end{tabularx}

\begin{paracol}{2}
\selectlanguage{latin}
\lettrine[lines=2]{A}{quis} in Gállia, sancti 
 Maximíni, qui éxstitit primus ejúsdem civitátis Epíscopus, ac Dómini 
 discípulus fuísse tráditur.
\switchcolumn
\selectlanguage{english}
\lettrine[lines=2]{A}{t} Aix in France, St. Maximin, first 
 bishop of that city, who is said to have been a disciple of the Lord.
\switchcolumn*
\selectlanguage{latin}
Eódem die sanctæ 
 Callíopæ Mártyris, quæ, ob Christi fidem, abscíssis mammis atque adústis 
 cárnibus, super téstulas volutáta, demum, truncáta cápite, martyrii palmam 
 accépit.
\switchcolumn
\selectlanguage{english}
On the same day, St. Calliopa, 
 martyr, who for the faith of Christ received the palm of martyrdom. 
 Her breasts had been cut away, her flesh burned, she was rolled on broken 
 pottery, and was at last beheaded.
\switchcolumn*
\selectlanguage{latin}
Eboráci, in Anglia, 
 sancti Willhélmi, Epíscopi et Confessóris, qui, inter cétera ad ejus 
 sepúlcrum patráta mirácula, tres mórtuos suscitávit, atque ab Honório Papa 
 Tértio in Sanctórum cánonem relátus est.
\switchcolumn
\selectlanguage{english}
At York in England, St. William, 
 archbishop and confessor, who, among other miracles wrought at his tomb, 
 raised three persons from the dead. He was placed in the calendar of 
 the saints by Pope Honorius III.
\switchcolumn*
\selectlanguage{latin}
Apud Suessiónes, in 
 Gálliis, natális sancti Medárdi, Epíscopi Novioménsis; cujus vita et mors 
 pretiósa gloriósis miráculis commendátur.
\switchcolumn
\selectlanguage{english}
At Soissons in France, the birthday 
 of St. Medard, bishop of Noyon, whose life and precious death are commended 
 by glorious miracles.
\switchcolumn*
\selectlanguage{latin}
Rotómagi sancti 
 Gildárdi Epíscopi, qui fuit frater ejúsdem sancti Medárdi. Ambo autem 
 fratres, eódem die nati eodémque die Epíscopi consecráti, uno quoque die de 
 hac vita subtrácti, simul in cælum migrárunt.
\switchcolumn
\selectlanguage{english}
At Rouen, St. Gildard, bishop, 
 brother of this same St. Medard. They were born on the same day, 
 consecrated bishops at the same time, and were taken from this life on the 
 same day, entering heaven together.
\switchcolumn*
\selectlanguage{latin}
Apud Sénonas sancti 
 Heráclii Epíscopi.
\switchcolumn
\selectlanguage{english}
At Sens, the bishop St. Heraclius.
\switchcolumn*
\selectlanguage{latin}
Metis, in Gállia, 
 sancti Clodúlphi Epíscopi.
\switchcolumn
\selectlanguage{english}
At Metz, the bishop St. Clodulph.
\switchcolumn*
\selectlanguage{latin}
In Picéno sancti 
 Severíni, Septempedáni Epíscopi.
\switchcolumn
\selectlanguage{english}
In Piceno, St. Severin, bishop of 
 Septempeda.
\switchcolumn*
\selectlanguage{latin}
In Sardínia sancti 
 Sallustiáni Confessóris.
\switchcolumn
\selectlanguage{english}
In Sardinia, St. Sallustian, 
 confessor.
\switchcolumn*
\selectlanguage{latin}
Cameríni sancti 
 Victoríni Confessóris, qui fuit prædícti sancti Severíni, Septempedáni 
 Epíscopi, germánus frater.
\switchcolumn
\selectlanguage{english}
At Camerino, St. Victorinus, 
 confessor, the twin brother of St. Severin, bishop of Septempeda.
\switchcolumn*
\selectlanguage{latin}
\end{paracol}


% ---- martyrology/mart06/mart0609.htm
\needspace{10\baselineskip}
\begin{paracol}{2}
\selectlanguage{latin}
\begin{center}{\color{gregoriocolor} Quinto Idus Júnii. 
 Luna\dots\ }\end{center}
\switchcolumn
\selectlanguage{english}
\begin{center}{\color{gregoriocolor} The Ninth Day of 
 June. The\dots\ Day of the Moon.}\end{center}
\end{paracol}

\noindent\begin{tabularx}{\linewidth}{*{19}{>{\centering\arraybackslash}X}}
 \textcolor{gregoriocolor}{a} & \textcolor{gregoriocolor}{b} & \textcolor{gregoriocolor}{c} & \textcolor{gregoriocolor}{d} & \textcolor{gregoriocolor}{e} & \textcolor{gregoriocolor}{f} & \textcolor{gregoriocolor}{g} & \textcolor{gregoriocolor}{h} & \textcolor{gregoriocolor}{i} & \textcolor{gregoriocolor}{k} & \textcolor{gregoriocolor}{l} & \textcolor{gregoriocolor}{m} & \textcolor{gregoriocolor}{n} & \textcolor{gregoriocolor}{p} & \textcolor{gregoriocolor}{q} & \textcolor{gregoriocolor}{r} & \textcolor{gregoriocolor}{s} & \textcolor{gregoriocolor}{t} & \textcolor{gregoriocolor}{u} \\
 13 & 14 & 15 & 16 & 17 & 18 & 19 & 20 & 21 & 22 & 23 & 24 & 25 & 26 & 27 & 28 & 29 & 1 & 2 \\
\end{tabularx}
\vspace{0.5\baselineskip}
\noindent\begin{tabularx}{\linewidth}{*{12}{>{\centering\arraybackslash}X}}
 \textcolor{gregoriocolor}{A} & \textcolor{gregoriocolor}{B} & \textcolor{gregoriocolor}{C} & \textcolor{gregoriocolor}{D} & \textcolor{gregoriocolor}{E} & F & \textcolor{gregoriocolor}{F} & \textcolor{gregoriocolor}{G} & \textcolor{gregoriocolor}{H} & \textcolor{gregoriocolor}{M} & \textcolor{gregoriocolor}{N} & \textcolor{gregoriocolor}{P} \\
 3 & 4 & 5 & 6 & 7 & 8 & 7 & 8 & 9 & 10 & 11 & 12 \\
\end{tabularx}

\begin{paracol}{2}
\selectlanguage{latin}
\lettrine[lines=2]{N}{oménti} in Sabínis, 
 natális sanctórum Mártyrum Primi et Feliciáni fratrum, sub Diocletiáno et 
 Maximiáno Imperatóribus. Hi gloriósi Mártyres, cum longævam in Dómino 
 vitam duxíssent, et nunc simul pária, nunc singillátim divérsa et exquisíta 
 pertulíssent torménta, ambo tandem felícis pugnæ cursum, a Nomentáno Præside 
 Promóto animadvérsi gládio, consummavérunt. Ipsórum autem Mártyrum 
 córpora, póstea Romam transláta, in Ecclésia sancti Stéphani Protomártyris, 
 in monte Cælio, honorífice collocáta sunt.
\switchcolumn
\selectlanguage{english}
\lettrine[lines=2]{A}{t} Nomento in the Sabine Hills, the 
 birthday of the holy martyrs Primus and Felician, under the emperors 
 Diocletian and Maximian. These glorious martyrs lived long in the 
 service of the Lord, and endured sometimes together, sometimes separately, 
 various cruel torments. They were finally beheaded by Promotus, 
 governor of Nomento, and thus happily ended their trial. Their bodies 
 were afterwards translated to Rome and honorably buried in the Church of St. 
 Stephen the Protomartyr on the Cælian Hill.
\switchcolumn*
\selectlanguage{latin}
Agénni in Gállia, 
 pássio sancti Vincéntii, Levítæ et Mártyris, qui, ob Christi fidem, 
 verbéribus diríssime cæsus et gládio decollátus est.
\switchcolumn
\selectlanguage{english}
At Agen in France, the passion of St. Vincent, deacon and 
 martyr. For the faith of Christ, he was cruelly scourged and then 
 beheaded.
\switchcolumn*
\selectlanguage{latin}
Apud Antiochíam sanctæ 
 Pelágiæ, Vírginis et Mártyris, quam sancti Ambrósius et Joánnes Chrysóstomus 
 magnis éfferunt láudibus.
\switchcolumn
\selectlanguage{english}
At Antioch, St. Pelagia, virgin and 
 martyr, who has been eulogized by St. Ambrose and St. John Chrysostom.
\switchcolumn*
\selectlanguage{latin}
Syracúsis, in Sicília, 
 sancti Maximiáni Epíscopi, cujus sanctus Gregórius Papa sæpius méminit.
\switchcolumn
\selectlanguage{english}
At Syracuse in Sicily, Bishop St. 
 Maximian, who is frequently mentioned by Pope St. Gregory.
\switchcolumn*
\selectlanguage{latin}
Andriæ, in Apúlia, 
 sancti Richárdi, qui fuit primus ejúsdem civitátis Epíscopus, et miráculis 
 cláruit.
\switchcolumn
\selectlanguage{english}
At Andria in Apulia, St. Richard, 
 first bishop of that city, who is famed for his miracles.
\switchcolumn*
\selectlanguage{latin}
In Ióna, Scótiæ ínsula, 
 sancti Colúmbæ, Presbyteri et Abbátis.
\switchcolumn
\selectlanguage{english}
In the island of Iona in Scotland, 
 St. Columba, priest and confessor.
\switchcolumn*
\selectlanguage{latin}
Edéssæ, in Syria, 
 sancti Juliáni Mónachi, cujus præclára gesta sanctus Ephræm Diáconus 
 scripsit.
\switchcolumn
\selectlanguage{english}
At Edessa in Syria, St. Julian, a 
 monk whose memorable deeds have been related by the deacon St. Ephraem.
\switchcolumn*
\selectlanguage{latin}
\end{paracol}


% ---- martyrology/mart06/mart0610.htm
\needspace{10\baselineskip}
\begin{paracol}{2}
\selectlanguage{latin}
\begin{center}{\color{gregoriocolor} Quarto Idus Júnii. 
 Luna\dots\ }\end{center}
\switchcolumn
\selectlanguage{english}
\begin{center}{\color{gregoriocolor} The Tenth Day of 
 June. The\dots\ Day of the Moon.}\end{center}
\end{paracol}

\noindent\begin{tabularx}{\linewidth}{*{19}{>{\centering\arraybackslash}X}}
 \textcolor{gregoriocolor}{a} & \textcolor{gregoriocolor}{b} & \textcolor{gregoriocolor}{c} & \textcolor{gregoriocolor}{d} & \textcolor{gregoriocolor}{e} & \textcolor{gregoriocolor}{f} & \textcolor{gregoriocolor}{g} & \textcolor{gregoriocolor}{h} & \textcolor{gregoriocolor}{i} & \textcolor{gregoriocolor}{k} & \textcolor{gregoriocolor}{l} & \textcolor{gregoriocolor}{m} & \textcolor{gregoriocolor}{n} & \textcolor{gregoriocolor}{p} & \textcolor{gregoriocolor}{q} & \textcolor{gregoriocolor}{r} & \textcolor{gregoriocolor}{s} & \textcolor{gregoriocolor}{t} & \textcolor{gregoriocolor}{u} \\
 14 & 15 & 16 & 17 & 18 & 19 & 20 & 21 & 22 & 23 & 24 & 25 & 26 & 27 & 28 & 29 & 1 & 2 & 3 \\
\end{tabularx}
\vspace{0.5\baselineskip}
\noindent\begin{tabularx}{\linewidth}{*{12}{>{\centering\arraybackslash}X}}
 \textcolor{gregoriocolor}{A} & \textcolor{gregoriocolor}{B} & \textcolor{gregoriocolor}{C} & \textcolor{gregoriocolor}{D} & \textcolor{gregoriocolor}{E} & F & \textcolor{gregoriocolor}{F} & \textcolor{gregoriocolor}{G} & \textcolor{gregoriocolor}{H} & \textcolor{gregoriocolor}{M} & \textcolor{gregoriocolor}{N} & \textcolor{gregoriocolor}{P} \\
 4 & 5 & 6 & 7 & 8 & 9 & 8 & 9 & 10 & 11 & 12 & 13 \\
\end{tabularx}

\begin{paracol}{2}
\selectlanguage{latin}
\lettrine[lines=2]{S}{anctæ} Margarítæ Víduæ, 
 Scotórum Regínæ, quæ sextodécimo Kaléndas Decémbris obdormívit in Dómino.
\switchcolumn
\selectlanguage{english}
\lettrine[lines=2]{S}{t.} Margaret, widow, queen of 
 Scotland, who slept in the Lord on the 16th of November.
\switchcolumn*
\selectlanguage{latin}
Romæ, via Salária, 
 pássio beáti Getúlii, claríssimi et doctíssimi viri, ac susceptórum e sancta 
 uxóre Symphorósa beatórum septem fratrum Mártyrum patris, ejúsque Sociórum 
 Cæreális, Amántii et Primitívi. Hi omnes, Hadriáni Imperatóris jussu, 
 a Licínio Consulári tenti, primum cæsi sunt, deínde in cárcerem trusi; 
 postrémum, incéndio tráditi, sed nullo modo ab igne læsi, martyrium suum, 
 fústibus illíso cápite, complevérunt. Ipsórum autem córpora Symphorósa, 
 beáti Getúlii uxor, collégit, et in arenário prædii sui honorífice sepelívit.
\switchcolumn
\selectlanguage{english}
At Rome, on the Salarian Way, the 
 martyrdom of blessed Getulius, a very learned nobleman, and his companions, 
 Caerealis, Amantius, and Primitivus. By order of Emperor Hadrian they 
 were arrested by the ex-consul Licinius, scourged, thrown into prison, and 
 then delivered to the flames. But the fire did not injure them, and 
 their heads were crushed with clubs, thus ending their martyrdom. 
 Their bodies were taken by Symphorosa, wife of blessed Getulius, and 
 reverently interred on her own estate.
\switchcolumn*
\selectlanguage{latin}
Item Romæ, via Aurélia, natális sanctórum Basílidis, Trípodis, Mándalis et aliórum vigínti Mártyrum, 
 sub Imperatóre Aureliáno et Urbis Præfécto Platóne.
\switchcolumn
\selectlanguage{english}
Also at Rome, on the Aurelian Way, 
 the birthday of the Saints Basilides, Tripos, Mandal, and twenty other 
 martyrs, under Emperor Aurelian and Plato, the governor of the city.
\switchcolumn*
\selectlanguage{latin}
Neápoli, in Campánia, 
 sancti Máximi, Epíscopi et Mártyris; qui, ob strénuam Nicǽnæ fídei 
 confessiónem, a Constántio Imperatóre in exsílium pulsus, ibídem, ærúmnis 
 conféctus, decéssit.
\switchcolumn
\selectlanguage{english}
At Naples in Campania, St. Maximus, 
 bishop and martyr. For having vigorously defended the Nicene Creed he 
 was exiled by Emperor Constantius, where he died worn out by his trials.
\switchcolumn*
\selectlanguage{latin}
Prusæ, in Bithynia, 
 sancti Timóthei, Epíscopi et Mártyris, qui, sub Juliáno Apóstata, cum 
 Christum ejuráre noluísset, idcírco, ipsíus Imperatóris jussu, cápite 
 abscíssus est.
\switchcolumn
\selectlanguage{english}
At Prusias in Bithynia, St. Timothy, 
 bishop and martyr. He was beheaded during the reign of Julian the 
 Apostate because he refused to deny Christ.
\switchcolumn*
\selectlanguage{latin}
Colóniæ Agrippínæ 
 sancti Mauríni, Abbátis et Mártyris.
\switchcolumn
\selectlanguage{english}
At Cologne, St. Maurinus, abbot and martyr
\switchcolumn*
\selectlanguage{latin}
Nicomedíæ sancti 
 Zacharíæ Mártyris.
\switchcolumn
\selectlanguage{english}
At Nicomedia, the martyr St. 
 Zachary.
\switchcolumn*
\selectlanguage{latin}
In Hispánia sanctórum 
 Mártyrum Críspuli et Restitúti.
\switchcolumn
\selectlanguage{english}
In Spain, the holy martyrs Crispulus 
 and Restitutus.
\switchcolumn*
\selectlanguage{latin}
In Africa sanctórum 
 Mártyrum Arésii, Rogáti et aliórum quíndecim.
\switchcolumn
\selectlanguage{english}
In Africa, the holy martyrs Aresius, 
 Rogatus, and fifteen others.
\switchcolumn*
\selectlanguage{latin}
Petræ, in Arábia, 
 sancti Astérii Epíscopi, qui, ob fidem cathólicam, ab Ariánis multa passus 
 et ab Imperatóre Constántio in Africam relegátus est, ac tandem, in 
 Ecclésiam suam restitútus, Conféssor gloriósus occúbuit.
\switchcolumn
\selectlanguage{english}
At Petra in Africa, St. Asterius, a 
 bishop who suffered greatly for the Catholic faith at the hands of the 
 Arians. He was banished to Africa by Emperor Constantius, and there 
 died as a glorious confessor.
\switchcolumn*
\selectlanguage{latin}
Antisiodóri sancti 
 Censúrii Epíscopi.
\switchcolumn
\selectlanguage{english}
At Auxerre, St. Censurius, bishop.
\switchcolumn*
\selectlanguage{latin}
\end{paracol}


% ---- martyrology/mart06/mart0611.htm
\needspace{10\baselineskip}
\begin{paracol}{2}
\selectlanguage{latin}
\begin{center}{\color{gregoriocolor} Tértio Idus Júnii. 
 Luna\dots\ }\end{center}
\switchcolumn
\selectlanguage{english}
\begin{center}{\color{gregoriocolor} The Eleventh Day of 
 June. The\dots\ Day of the Moon.}\end{center}
\end{paracol}

\noindent\begin{tabularx}{\linewidth}{*{19}{>{\centering\arraybackslash}X}}
 \textcolor{gregoriocolor}{a} & \textcolor{gregoriocolor}{b} & \textcolor{gregoriocolor}{c} & \textcolor{gregoriocolor}{d} & \textcolor{gregoriocolor}{e} & \textcolor{gregoriocolor}{f} & \textcolor{gregoriocolor}{g} & \textcolor{gregoriocolor}{h} & \textcolor{gregoriocolor}{i} & \textcolor{gregoriocolor}{k} & \textcolor{gregoriocolor}{l} & \textcolor{gregoriocolor}{m} & \textcolor{gregoriocolor}{n} & \textcolor{gregoriocolor}{p} & \textcolor{gregoriocolor}{q} & \textcolor{gregoriocolor}{r} & \textcolor{gregoriocolor}{s} & \textcolor{gregoriocolor}{t} & \textcolor{gregoriocolor}{u} \\
 15 & 16 & 17 & 18 & 19 & 20 & 21 & 22 & 23 & 24 & 25 & 26 & 27 & 28 & 29 & 1 & 2 & 3 & 4 \\
\end{tabularx}
\vspace{0.5\baselineskip}
\noindent\begin{tabularx}{\linewidth}{*{12}{>{\centering\arraybackslash}X}}
 \textcolor{gregoriocolor}{A} & \textcolor{gregoriocolor}{B} & \textcolor{gregoriocolor}{C} & \textcolor{gregoriocolor}{D} & \textcolor{gregoriocolor}{E} & F & \textcolor{gregoriocolor}{F} & \textcolor{gregoriocolor}{G} & \textcolor{gregoriocolor}{H} & \textcolor{gregoriocolor}{M} & \textcolor{gregoriocolor}{N} & \textcolor{gregoriocolor}{P} \\
 5 & 6 & 7 & 8 & 9 & 10 & 9 & 10 & 11 & 12 & 13 & 14 \\
\end{tabularx}

\begin{paracol}{2}
\selectlanguage{latin}
\lettrine[lines=2]{S}{alamínæ,} in Cypro, 
 natális sancti Bárnabæ Apóstoli, qui, natióne Cyprius, cum Paulo Géntium 
 Apóstolus a discípulis ordinátus, multas regiónes cum eo peragrávit, 
 injúnctum sibi opus Evangélicæ prædicatiónis exércens; postrémo, Cyprum 
 proféctus, ibi Apostolátum suum glorióso martyrio decorávit. Ejus 
 corpus, témpore Zenónis Imperatóris, ipso Bárnaba revelánte, repértum est, 
 una cum códice Evangélii sancti Matthǽi, ejúsdem Bárnabæ manu descrípto.
\switchcolumn
\selectlanguage{english}
\lettrine[lines=2]{A}{t} Salamina in Cyprus, the birthday 
 of the apostle St. Barnabas, a native of that place. He was ordained 
 by the disciples as Apostle of the Gentiles with St. Paul, and travelled 
 with him over many regions, exercising the function committed unto him of 
 preaching the Gospel. At last he went back to Cyprus, where he 
 ennobled his apostolate by a glorious martyrdom. His body was found by 
 his own revelation, in the time of Emperor Zeno, together with a copy of St. 
 Matthew's Gospel written with his own hand.
\switchcolumn*
\selectlanguage{latin}
Salmánticæ, in 
 Hispánia, item natális sancti Joánnis a sancto Facúndo, ex Eremitárum sancti 
 Augustíni Ordine, Confessóris; qui fídei zelo, sanctimónia vitæ ac miráculis 
 cláruit. Ipsíus autem festívitas sequénti die celebrátur.
\switchcolumn
\selectlanguage{english}
At Salamanca in Spain, St. John of St. Facundus, a confessor of the Order of the Hermits of St. Augustine, 
 renowned for his zeal for the faith, for holiness of life, and for miracles. 
 His feast is celebrated on the day following.
\switchcolumn*
\selectlanguage{latin}
Aquiléjæ pássio 
 sanctórum Felícis et Fortunáti fratrum, qui, in persecutióne Diocletiáni et 
 Maximiáni, equúleo suspénsi, atque, ardéntibus lampádibus circa eórum látera 
 appósitis et mox divína virtúte exstínctis, per ventrem fervénti óleo sunt 
 perfúsi; et ad últimum, cum in Christi confessióne persísterent, gloriósi 
 certáminis cursum, obtruncáti cápite, implevérunt.
\switchcolumn
\selectlanguage{english}
At Aquileia, the martyrdom of the 
 Saints Felix and Fortunatus, brothers. In the persecution of 
 Diocletian and Maximian, they were placed on the rack, and had flaming 
 torches held against their sides. These were extinguished by the power 
 of God, and boiling oil was poured over them. As they persevered in 
 confessing Christ, they were beheaded.
\switchcolumn*
\selectlanguage{latin}
Bremæ natális sancti 
 Rembérti, Hamburgénsis ac Breménsis Epíscopi.
\switchcolumn
\selectlanguage{english}
At Bremen, the birthday of St. 
 Rembert, bishop of Hamburg and Bremen.
\switchcolumn*
\selectlanguage{latin}
Tarvísii sancti Parísii, 
 civis Bononiénsis, Confessóris et Mónachi, ex Ordine Camaldulénsi.
\switchcolumn
\selectlanguage{english}
At Treviso, St. Parisius, a citizen 
 of Bologna, confessor and monk of the Camaldolese Order.
\switchcolumn*
\selectlanguage{latin}
Romæ Translátio sancti 
 Gregórii Nazianzéni, Epíscopi, Confessóris atque Ecclésiæ Doctóris; cujus 
 sacrum corpus, e Constantinópoli ántea delátum ad Urbem, atque in Ecclésia 
 sanctæ Dei Genitrícis ad Campum Mártium diu asservátum, Gregórius Décimus 
 tértius, Póntifex Máximus, in Capéllum, a se in Basílica sancti Petri 
 magnificentíssime exornátum, summa celebritáte tránstulit, ac postrídie 
 digno honóre sub altári cóndidit.
\switchcolumn
\selectlanguage{english}
At Rome, the translation of St. 
 Gregory Nazianzen, whose revered body was brought from Constantinople to 
 Rome, and kept for a long time in the Church of the Mother of God. It 
 was then transferred with great solemnity by Pope Gregory XIII to a chapel 
 of the basilica of St. Peter, magnificently decorated by His Holiness, and 
 the next day placed with due honour beneath the altar.
\switchcolumn*
\selectlanguage{latin}
\end{paracol}


% ---- martyrology/mart06/mart0612.htm
\needspace{10\baselineskip}
\begin{paracol}{2}
\selectlanguage{latin}
\begin{center}{\color{gregoriocolor} Prídie Idus Júnii. 
 Luna\dots\ }\end{center}
\switchcolumn
\selectlanguage{english}
\begin{center}{\color{gregoriocolor} The Twelfth Day of 
 June. The\dots\ Day of the Moon.}\end{center}
\end{paracol}

\noindent\begin{tabularx}{\linewidth}{*{19}{>{\centering\arraybackslash}X}}
 \textcolor{gregoriocolor}{a} & \textcolor{gregoriocolor}{b} & \textcolor{gregoriocolor}{c} & \textcolor{gregoriocolor}{d} & \textcolor{gregoriocolor}{e} & \textcolor{gregoriocolor}{f} & \textcolor{gregoriocolor}{g} & \textcolor{gregoriocolor}{h} & \textcolor{gregoriocolor}{i} & \textcolor{gregoriocolor}{k} & \textcolor{gregoriocolor}{l} & \textcolor{gregoriocolor}{m} & \textcolor{gregoriocolor}{n} & \textcolor{gregoriocolor}{p} & \textcolor{gregoriocolor}{q} & \textcolor{gregoriocolor}{r} & \textcolor{gregoriocolor}{s} & \textcolor{gregoriocolor}{t} & \textcolor{gregoriocolor}{u} \\
 16 & 17 & 18 & 19 & 20 & 21 & 22 & 23 & 24 & 25 & 26 & 27 & 28 & 29 & 1 & 2 & 3 & 4 & 5 \\
\end{tabularx}
\vspace{0.5\baselineskip}
\noindent\begin{tabularx}{\linewidth}{*{12}{>{\centering\arraybackslash}X}}
 \textcolor{gregoriocolor}{A} & \textcolor{gregoriocolor}{B} & \textcolor{gregoriocolor}{C} & \textcolor{gregoriocolor}{D} & \textcolor{gregoriocolor}{E} & F & \textcolor{gregoriocolor}{F} & \textcolor{gregoriocolor}{G} & \textcolor{gregoriocolor}{H} & \textcolor{gregoriocolor}{M} & \textcolor{gregoriocolor}{N} & \textcolor{gregoriocolor}{P} \\
 6 & 7 & 8 & 9 & 10 & 11 & 10 & 11 & 12 & 13 & 14 & 15 \\
\end{tabularx}

\begin{paracol}{2}
\selectlanguage{latin}
\lettrine[lines=2]{S}{ancti} Joánnis a sancto Facúndo, ex Eremitárum sancti 
 Augustíni Ordine, Confessóris, qui migrávit in cælum prídie hujus diéi.
\switchcolumn
\selectlanguage{english}
\lettrine[lines=2]{S}{t.} John of St. Facundus, confessor of the Order of the 
 Hermits of St. Augustine, who died on the 11th of June.
\switchcolumn*
\selectlanguage{latin}
Romæ, via Aurélia, natális sanctórum Mártyrum Basílidis, 
 Cyríni, Náboris et Nazárii militum, qui, in persecutióne Diocletiáni et 
 Maximiáni, sub Aurélio Præfécto, ob Christiáni nóminis confessiónem, detrúsi 
 in cárcerem et scorpiónibus laceráti, tandem cápite truncáti sunt.
\switchcolumn
\selectlanguage{english}
At Rome, on the Aurelian Way, during the persecution of 
 Diocletian and Maximian, and under the prefect Aurelius, the birthday of the 
 holy martyrs Basilides, Cyrinus, Nabor, and Nazarius, all soldiers who were 
 cast into prison for the confession of the Christian name, scourged with 
 knotted whips, and finally beheaded.
\switchcolumn*
\selectlanguage{latin}
Nicǽæ, in Bithynia, sanctæ Antonínæ Mártyris, quæ, in 
 eádem persecutióne, a Priscilliáno Præside jussa est fústibus cædi, suspéndi 
 in equúleo, latéribus laniári et flammis incéndi, ac demum gládio necári.
\switchcolumn
\selectlanguage{english}
At Nicaea in Bithynia, St. Antonina, martyr. She 
 was scourged by order of the governor Priscillian during the same 
 persecution, then racked, lacerated, exposed to the fire, and finally put to 
 the sword.
\switchcolumn*
\selectlanguage{latin}
Romæ, in Basílica Vaticána, sancti Leónis Papæ Tértii, 
 cui erútos ab ímpiis óculos et præcísam linguam Deus mirabíliter restítuit.
\switchcolumn
\selectlanguage{english}
At Rome, in the Vatican basilica, Pope St. Leo III, to 
 whom God miraculously restored his eyes and his tongue after they had been 
 torn out by impious men.
\switchcolumn*
\selectlanguage{latin}
In Thrácia sancti Olympii Epíscopi, qui, ab Ariánis sede 
 pulsus, Conféssor occúbuit.
\switchcolumn
\selectlanguage{english}
In Thrace, St. Olympius, a bishop, who was driven out of 
 his diocese by the Arians, and died a confessor.
\switchcolumn*
\selectlanguage{latin}
In Cilícia sancti Amphiónis Epíscopi, qui, témpore 
 Galérii Maximiáni, Conféssor fuit egrégius.
\switchcolumn
\selectlanguage{english}
In Cilicia, Bishop St. Amphion, a celebrated confessor of 
 the time of Galerius Maximian.
\switchcolumn*
\selectlanguage{latin}
In Ægypto sancti Onúphrii Anachorétæ, qui in vasta erémo 
 sexagínta annis vitam religióse perégit, et, magnis virtútibus ac méritis 
 clarus, migrávit in cælum. Ipsíus vero insígnia gesta Paphnútius Abbas 
 conscrípsit.
\switchcolumn
\selectlanguage{english}
In Egypt, St. Onuphrius, an anchoret, who for sixty years 
 led a religious life in the desert, and renowned for great virtues and 
 merits departed for heaven. His admirable deeds have been recorded by 
 Abbot Paphnutius.
\switchcolumn*
\selectlanguage{latin}
\end{paracol}


% ---- martyrology/mart06/mart0613.htm
\needspace{10\baselineskip}
\begin{paracol}{2}
\selectlanguage{latin}
\begin{center}{\color{gregoriocolor} Idibus Júnii. 
 Luna\dots\ }\end{center}
\switchcolumn
\selectlanguage{english}
\begin{center}{\color{gregoriocolor} The Thirteenth Day of 
 June. The\dots\ Day of the Moon.}\end{center}
\end{paracol}

\noindent\begin{tabularx}{\linewidth}{*{19}{>{\centering\arraybackslash}X}}
 \textcolor{gregoriocolor}{a} & \textcolor{gregoriocolor}{b} & \textcolor{gregoriocolor}{c} & \textcolor{gregoriocolor}{d} & \textcolor{gregoriocolor}{e} & \textcolor{gregoriocolor}{f} & \textcolor{gregoriocolor}{g} & \textcolor{gregoriocolor}{h} & \textcolor{gregoriocolor}{i} & \textcolor{gregoriocolor}{k} & \textcolor{gregoriocolor}{l} & \textcolor{gregoriocolor}{m} & \textcolor{gregoriocolor}{n} & \textcolor{gregoriocolor}{p} & \textcolor{gregoriocolor}{q} & \textcolor{gregoriocolor}{r} & \textcolor{gregoriocolor}{s} & \textcolor{gregoriocolor}{t} & \textcolor{gregoriocolor}{u} \\
 17 & 18 & 19 & 20 & 21 & 22 & 23 & 24 & 25 & 26 & 27 & 28 & 29 & 1 & 2 & 3 & 4 & 5 & 6 \\
\end{tabularx}
\vspace{0.5\baselineskip}
\noindent\begin{tabularx}{\linewidth}{*{12}{>{\centering\arraybackslash}X}}
 \textcolor{gregoriocolor}{A} & \textcolor{gregoriocolor}{B} & \textcolor{gregoriocolor}{C} & \textcolor{gregoriocolor}{D} & \textcolor{gregoriocolor}{E} & F & \textcolor{gregoriocolor}{F} & \textcolor{gregoriocolor}{G} & \textcolor{gregoriocolor}{H} & \textcolor{gregoriocolor}{M} & \textcolor{gregoriocolor}{N} & \textcolor{gregoriocolor}{P} \\
 7 & 8 & 9 & 10 & 11 & 12 & 11 & 12 & 13 & 14 & 15 & 16 \\
\end{tabularx}

\begin{paracol}{2}
\selectlanguage{latin}
\lettrine[lines=2]{P}{atávii} sancti Antónii 
 Lusitáni, Sacerdótis ex Ordine Minórum et Confessóris, atque Ecclésiæ 
 Doctóris, vita et miráculis, ac prædicatióne illústris, quem, uno post illíus óbitum anno nondum expléto, Gregórius Papa Nonus in Sanctórum cánonem 
 rétulit.
\switchcolumn
\selectlanguage{english}
\lettrine[lines=2]{A}{t} Padua, St. Anthony, a native of 
 Portugal, priest of the Order of Friars Minor and confessor, illustrious for 
 the sanctity of his life, his miracles, and his preaching. Pope 
 Gregory IX placed him on the canon of the saints within a year after his 
 death.
\switchcolumn*
\selectlanguage{latin}
Romæ, via Ardeatína, 
 natális sanctæ Felículæ, Vírginis et Mártyris; quæ, nec Flacco núbere neque 
 idólis immoláre volens, trádita est cuídam Júdici, qui eam in confessióne 
 Christi perseverántem, post tenebricósam custódiam et famis inédiam, támdiu 
 fecit in equúleo torquéri, donec illa emítteret spíritum, et sic demum 
 depóni et in cloácam præcipitári. Ipsíus vero corpus, inde extráctum, 
 sanctus Nicomédes Présbyter eádem via sepelívit.
\switchcolumn
\selectlanguage{english}
At Rome, on the Ardeatine Way, the 
 birthday of St. Felicula, virgin and martyr, who was delivered to the judge 
 for refusing to marry Flaccus and to sacrifice to idols. As she 
 persevered in the confession of Christ, he confined her in a dark dungeon 
 without food, and afterwards caused her to be stretched on the rack until 
 she expired. She was then thrown into a sewer, but St. Nicomedes the 
 Priest recovered her body and buried it on this road.
\switchcolumn*
\selectlanguage{latin}
In Pelígnis sancti 
 Peregríni, Epíscopi et Mártyris, qui, pro fide cathólica, a Longobárdis in 
 Atérnum flumen demérsus est.
\switchcolumn
\selectlanguage{english}
In Abruzzi, St. Peregrinus, bishop 
 and martyr. For the Catholic faith he was thrown into the river Aterno 
 by the Lombards.
\switchcolumn*
\selectlanguage{latin}
Córdubæ, in Hispánia, 
 sancti Fándilæ, Presbyteri et Mónachi; qui, in persecutióne Arábica, 
 amputáto cápite, pro Christi fide martyrium súbiit.
\switchcolumn
\selectlanguage{english}
At Cordova in Spain, in the 
 persecution of the Arabs, St. Fandila, a priest and monk, who underwent 
 martyrdom by beheading for the faith of Christ.
\switchcolumn*
\selectlanguage{latin}
In Africa sanctórum 
 Mártyrum Fortunáti et Luciáni.
\switchcolumn
\selectlanguage{english}
In Africa, the holy martyrs 
 Fortunatus and Lucian.
\switchcolumn*
\selectlanguage{latin}
Bybli, in Phœnícia, 
 sanctæ Aquilínæ, Vírginis et Mártyris, quæ, annos duódecim nata, sub 
 Diocletiáno Imperatóre et Volusiáno Júdice, ob fídei confessiónem cólaphis 
 et verbéribus cæsa, et súbulis candéntibus perforáta, demum, percússa gládio, 
 virginitátem martyrio consecrávit.
\switchcolumn
\selectlanguage{english}
At Byblos in Phoenicia, St. Aquilina, 
 virgin and martyr, at the age of twelve years, under Emperor Diocletian and 
 the judge Volusian. For the confession of the faith, she was beaten, 
 scourged, pierced with heated stakes, and finally being struck with a sword, 
 consecrated her virginity by martyrdom.
\switchcolumn*
\selectlanguage{latin}
In Cypro sancti 
 Triphylii Epíscopi.
\switchcolumn
\selectlanguage{english}
In Cyprus, St. Triphyllius, bishop.
\switchcolumn*
\selectlanguage{latin}
\end{paracol}


% ---- martyrology/mart06/mart0614.htm
\needspace{10\baselineskip}
\begin{paracol}{2}
\selectlanguage{latin}
\begin{center}{\color{gregoriocolor} Décimo octávo Kaléndas Júlii. 
 Luna\dots\ }\end{center}
\switchcolumn
\selectlanguage{english}
\begin{center}{\color{gregoriocolor} The Fourteenth Day of 
 June. The\dots\ Day of the Moon.}\end{center}
\end{paracol}

\noindent\begin{tabularx}{\linewidth}{*{19}{>{\centering\arraybackslash}X}}
 \textcolor{gregoriocolor}{a} & \textcolor{gregoriocolor}{b} & \textcolor{gregoriocolor}{c} & \textcolor{gregoriocolor}{d} & \textcolor{gregoriocolor}{e} & \textcolor{gregoriocolor}{f} & \textcolor{gregoriocolor}{g} & \textcolor{gregoriocolor}{h} & \textcolor{gregoriocolor}{i} & \textcolor{gregoriocolor}{k} & \textcolor{gregoriocolor}{l} & \textcolor{gregoriocolor}{m} & \textcolor{gregoriocolor}{n} & \textcolor{gregoriocolor}{p} & \textcolor{gregoriocolor}{q} & \textcolor{gregoriocolor}{r} & \textcolor{gregoriocolor}{s} & \textcolor{gregoriocolor}{t} & \textcolor{gregoriocolor}{u} \\
 18 & 19 & 20 & 21 & 22 & 23 & 24 & 25 & 26 & 27 & 28 & 29 & 1 & 2 & 3 & 4 & 5 & 6 & 7 \\
\end{tabularx}
\vspace{0.5\baselineskip}
\noindent\begin{tabularx}{\linewidth}{*{12}{>{\centering\arraybackslash}X}}
 \textcolor{gregoriocolor}{A} & \textcolor{gregoriocolor}{B} & \textcolor{gregoriocolor}{C} & \textcolor{gregoriocolor}{D} & \textcolor{gregoriocolor}{E} & F & \textcolor{gregoriocolor}{F} & \textcolor{gregoriocolor}{G} & \textcolor{gregoriocolor}{H} & \textcolor{gregoriocolor}{M} & \textcolor{gregoriocolor}{N} & \textcolor{gregoriocolor}{P} \\
 8 & 9 & 10 & 11 & 12 & 13 & 12 & 13 & 14 & 15 & 16 & 17 \\
\end{tabularx}

\begin{paracol}{2}
\selectlanguage{latin}
\lettrine[lines=2]{S}{ancti} Basilíi, 
 cognoménto Magni, Confessóris et Ecclésiæ Doctóris, qui Kaléndis Januárii 
 obdormívit in Dómino, sed hac die potíssimum cólitur, qua Cæsariénsis in 
 Cappadócia Epíscopus ordinátus est.
\switchcolumn
\selectlanguage{english}
\lettrine[lines=2]{S}{t.} Basil, surnamed the Great, 
 confessor and doctor of the Church. He died on the 1st of January, but 
 his feast is celebrated today, for it was on this day that he was 
 consecrated bishop of Caesarea in Cappadocia.
\switchcolumn*
\selectlanguage{latin}
Samaríæ, in Palæstína, 
 sancti Eliséi Prophétæ, cujus sepúlcrum, ubi et Prophéta quiéscit Abdías, a 
 dæmónibus perhorrésci sanctus Hierónymus scribit.
\switchcolumn
\selectlanguage{english}
At Samaria in Palestine, the holy 
 prophet Eliseus, whose grave, says St. Jerome, makes the demons tremble. 
 With him also rests the prophet Abdias.
\switchcolumn*
\selectlanguage{latin}
Syracúsis, in Sicília, 
 sancti Marciáni Epíscopi, qui, a beáto Petro Apóstolo ordinátus Epíscopus, 
 ibídem, post Evangélii prædicatiónem, a Judæis occísus est.
\switchcolumn
\selectlanguage{english}
At Syracuse in Sicily, St. Marcian, 
 bishop, who was made bishop by blessed Peter, and killed by the Jews after 
 he had preached the Gospel.
\switchcolumn*
\selectlanguage{latin}
Córdubæ, in Hispánia, 
 sanctórum Mártyrum Anastásii Presbyteri, Felícis Mónachi, et Dignæ Vírginis.
\switchcolumn
\selectlanguage{english}
At Cordova in Spain, the holy 
 martyrs Anastasius, a priest, Felix, a monk, and Digna, virgin.
\switchcolumn*
\selectlanguage{latin}
Apud Suessiónes, in 
 Gálliis, sanctórum mártyrum Valérii et Rufíni, qui, in persecutióne 
 Diocletiáni, a Præside Rictiováro, post multa torménta, jussi sunt decollári.
\switchcolumn
\selectlanguage{english}
At Soissons in France, the holy 
 martyrs Valerius and Rufinus, who, after enduring many torments, were 
 condemned to be beheaded by the governor Rictiovarus, in the persecution of 
 Diocletian.
\switchcolumn*
\selectlanguage{latin}
Constantinópoli sancti 
 Methódii Epíscopi.
\switchcolumn
\selectlanguage{english}
At Constantinople, St. Methodius, 
 bishop.
\switchcolumn*
\selectlanguage{latin}
Viénnæ, in Gállia, 
 sancti Æthérii Epíscopi.
\switchcolumn
\selectlanguage{english}
At Vienne, St. Aetherius, bishop.
\switchcolumn*
\selectlanguage{latin}
Apud Rhuténos, in 
 Gállia, sancti Quinctiáni Epíscopi.
\switchcolumn
\selectlanguage{english}
At Rodez in France, St. Quinctian, 
 bishop.
\switchcolumn*
\selectlanguage{latin}
\end{paracol}


% ---- martyrology/mart06/mart0615.htm
\needspace{10\baselineskip}
\begin{paracol}{2}
\selectlanguage{latin}
\begin{center}{\color{gregoriocolor} Décimo séptimo Kaléndas Júlii. 
 Luna\dots\ }\end{center}
\switchcolumn
\selectlanguage{english}
\begin{center}{\color{gregoriocolor} The Fifteenth Day of 
 June. The\dots\ Day of the Moon.}\end{center}
\end{paracol}

\noindent\begin{tabularx}{\linewidth}{*{19}{>{\centering\arraybackslash}X}}
 \textcolor{gregoriocolor}{a} & \textcolor{gregoriocolor}{b} & \textcolor{gregoriocolor}{c} & \textcolor{gregoriocolor}{d} & \textcolor{gregoriocolor}{e} & \textcolor{gregoriocolor}{f} & \textcolor{gregoriocolor}{g} & \textcolor{gregoriocolor}{h} & \textcolor{gregoriocolor}{i} & \textcolor{gregoriocolor}{k} & \textcolor{gregoriocolor}{l} & \textcolor{gregoriocolor}{m} & \textcolor{gregoriocolor}{n} & \textcolor{gregoriocolor}{p} & \textcolor{gregoriocolor}{q} & \textcolor{gregoriocolor}{r} & \textcolor{gregoriocolor}{s} & \textcolor{gregoriocolor}{t} & \textcolor{gregoriocolor}{u} \\
 19 & 20 & 21 & 22 & 23 & 24 & 25 & 26 & 27 & 28 & 29 & 1 & 2 & 3 & 4 & 5 & 6 & 7 & 8 \\
\end{tabularx}
\vspace{0.5\baselineskip}
\noindent\begin{tabularx}{\linewidth}{*{12}{>{\centering\arraybackslash}X}}
 \textcolor{gregoriocolor}{A} & \textcolor{gregoriocolor}{B} & \textcolor{gregoriocolor}{C} & \textcolor{gregoriocolor}{D} & \textcolor{gregoriocolor}{E} & F & \textcolor{gregoriocolor}{F} & \textcolor{gregoriocolor}{G} & \textcolor{gregoriocolor}{H} & \textcolor{gregoriocolor}{M} & \textcolor{gregoriocolor}{N} & \textcolor{gregoriocolor}{P} \\
 9 & 10 & 11 & 12 & 13 & 14 & 13 & 14 & 15 & 16 & 17 & 18 \\
\end{tabularx}

\begin{paracol}{2}
\selectlanguage{latin}
\lettrine[lines=2]{A}{pud} Silárum flumen, in 
 Lucánia, natális sanctórum Mártyrum Viti, Modésti atque Crescéntiæ, qui, sub 
 Diocletiáno Imperatóre, e Sicília illuc deláti, ibídem, post ollam fervéntis 
 plumbi, post béstias et catástas divína virtúte superátas, cursum gloriósi 
 certáminis peregérunt.
\switchcolumn
\selectlanguage{english}
\lettrine[lines=2]{N}{ear} the river Silaro in Lucania, 
 the birthday of the holy martyrs Vitus, Modestus, and Crescentia, who were 
 brought there from Sicily in the reign of the emperor Diocletian. They 
 were plunged into a vessel of molten lead, exposed to the beasts, and 
 stretched on the rack, but after having survived these torments through the 
 power of God, they came to the end of their glorious trials.
\switchcolumn*
\selectlanguage{latin}
Doróstori, in Mysia 
 inferióre, sancti Hesychii mílitis, qui, cum beáto Júlio comprehénsus, post 
 eum, sub Máximo Præside, martyrio coronátus est.
\switchcolumn
\selectlanguage{english}
At Silistria in Rumania, St. 
 Hesychius, a soldier, who was arrested with blessed Julius, and under the 
 governor Máximus followed him to the crown of martyrdom.
\switchcolumn*
\selectlanguage{latin}
Zephyrii, in Cilícia, 
 sancti Dulæ Mártyris, qui sub Præside Máximo, ob Christi nomen, virgis cæsus, 
 in cratícula pósitus, fervénti óleo incénsus aliáque passus, martyrii palmam 
 victor accépit.
\switchcolumn
\selectlanguage{english}
At Zephirium in Cilicia, St. Dulas, 
 martyr under the governor Maximus. For the name of Christ, he was 
 scourged, laid on the gridiron, scalded with boiling oil, and after enduring 
 other trials, received for his victory the palm of martyrdom.
\switchcolumn*
\selectlanguage{latin}
Córdubæ, in Hispánia, 
 sanctæ Beníldis Mártyris.
\switchcolumn
\selectlanguage{english}
At Cordova in Spain, St. Benildes, 
 martyr.
\switchcolumn*
\selectlanguage{latin}
Sibápoli, in 
 Mesopotámia, sanctárum Vírginum et Mártyrum Lybes et Leónidis sorórum, et 
 Eutrópiæ, puéllæ annórum duódecim; quæ per divérsa torménta ad corónam 
 martyrii pervenérunt.
\switchcolumn
\selectlanguage{english}
At Palmyra in Sicily, the holy 
 martyrs Libya and Leonides, sisters, and Eutropia, a girl of twelve years, 
 who won the crown of martyrdom by various torments.
\switchcolumn*
\selectlanguage{latin}
Apud Valencénas, in 
 Gállia, deposítio sancti Landelíni Abbátis.
\switchcolumn
\selectlanguage{english}
At Valenciennes in France, the 
 death of St. Landelin, abbot.
\switchcolumn*
\selectlanguage{latin}
Arvérnis, in Gállia, 
 sancti Abrahæ Confessóris, sanctitáte ac miráculis illústris.
\switchcolumn
\selectlanguage{english}
In Auvergne in France, St. Abraham, 
 confessor, illustrious by his holiness and miracles.
\switchcolumn*
\selectlanguage{latin}
Pibráci, in diœcési 
 Tolosáni, sanctæ Germánæ Cousin, Vírginis, quæ custodiéndis grégibus addícta 
 permánsit, et, cum húmilis ac pauper vixísset, et, multis ærúmnis 
 patientíssime tolerátis, migrásset ad Sponsum, plúrimis post óbitum 
 miráculis cláruit; atque a Pio Nono, Pontífice Máximo, in sanctárum Vírginum 
 número adscrípta fuit.
\switchcolumn
\selectlanguage{english}
At Pibrac in the diocese of 
 Toulouse, St. Germaine Cousin, virgin. After a life of poverty, 
 humility, and patient suffering amidst many trials as shepherdess of her 
 flocks, she went to her heavenly spouse, and became renowned for numerous 
 miracles after her death. Pope Pius IX placed her in the number of 
 holy virgins.
\switchcolumn*
\selectlanguage{latin}
\end{paracol}


% ---- martyrology/mart06/mart0616.htm
\needspace{10\baselineskip}
\begin{paracol}{2}
\selectlanguage{latin}
\begin{center}{\color{gregoriocolor} Sextodécimo Kaléndas Júlii. 
 Luna\dots\ }\end{center}
\switchcolumn
\selectlanguage{english}
\begin{center}{\color{gregoriocolor} The Sixteenth Day of 
 June. The\dots\ Day of the Moon.}\end{center}
\end{paracol}

\noindent\begin{tabularx}{\linewidth}{*{19}{>{\centering\arraybackslash}X}}
 \textcolor{gregoriocolor}{a} & \textcolor{gregoriocolor}{b} & \textcolor{gregoriocolor}{c} & \textcolor{gregoriocolor}{d} & \textcolor{gregoriocolor}{e} & \textcolor{gregoriocolor}{f} & \textcolor{gregoriocolor}{g} & \textcolor{gregoriocolor}{h} & \textcolor{gregoriocolor}{i} & \textcolor{gregoriocolor}{k} & \textcolor{gregoriocolor}{l} & \textcolor{gregoriocolor}{m} & \textcolor{gregoriocolor}{n} & \textcolor{gregoriocolor}{p} & \textcolor{gregoriocolor}{q} & \textcolor{gregoriocolor}{r} & \textcolor{gregoriocolor}{s} & \textcolor{gregoriocolor}{t} & \textcolor{gregoriocolor}{u} \\
 20 & 21 & 22 & 23 & 24 & 25 & 26 & 27 & 28 & 29 & 1 & 2 & 3 & 4 & 5 & 6 & 7 & 8 & 9 \\
\end{tabularx}
\vspace{0.5\baselineskip}
\noindent\begin{tabularx}{\linewidth}{*{12}{>{\centering\arraybackslash}X}}
 \textcolor{gregoriocolor}{A} & \textcolor{gregoriocolor}{B} & \textcolor{gregoriocolor}{C} & \textcolor{gregoriocolor}{D} & \textcolor{gregoriocolor}{E} & F & \textcolor{gregoriocolor}{F} & \textcolor{gregoriocolor}{G} & \textcolor{gregoriocolor}{H} & \textcolor{gregoriocolor}{M} & \textcolor{gregoriocolor}{N} & \textcolor{gregoriocolor}{P} \\
 10 & 11 & 12 & 13 & 14 & 15 & 14 & 15 & 16 & 17 & 18 & 19 \\
\end{tabularx}

\begin{paracol}{2}
\selectlanguage{latin}
\lettrine[lines=2]{M}{ogúntiæ} pássio 
 sanctórum Auræi Epíscopi, et Justínæ soróris, ac ceterórum Mártyrum; qui, 
 synáxim in Ecclésia agéntes, ab Hunnis, Germániam diripiéntibus, trucidáti 
 sunt.
\switchcolumn
\selectlanguage{english}
\lettrine[lines=2]{A}{t} Mainz, the passion of the Saints 
 Aureus and Justina, his sister, and other martyrs who were massacred by the 
 Huns, at that time devastating Germany, while they were in church at Mass.
\switchcolumn*
\selectlanguage{latin}
Vesontióne, in Gálliis, 
 sanctórum Mártyrum Ferreóli Presbyteri, et Ferruntiónis Diáconi, qui, a 
 beáto Irenæo Epíscopo missi ad prædicándum verbum Dei, póstea, sub Cláudio 
 Júdice, divérsis pœnis excruciáti, gládio feriúntur.
\switchcolumn
\selectlanguage{english}
At Besançon in France, the holy 
 martyrs Ferreol, a priest, and Ferruntion, a deacon, who were sent by the 
 blessed bishop Irenæus to preach the word of God, and after being exposed 
 to various torments under Judge Claudius, were put to the sword.
\switchcolumn*
\selectlanguage{latin}
Tarsi, in Cilícia, 
 sanctórum Mártyrum Quírici et Julíttæ matris, sub Diocletiáno Imperatóre. 
 Ex eis Quíricus, triénnis puéllus, cum matrem, quæ ante Alexándrum Præsidem 
 nervis diríssime cædebátur, implacábili luctu lugéret, ad gradus tribunális 
 illísus intériit; Julítta vero, post dira vérbera et grávia torménta, 
 martyrii sui cursum obtruncatióne cápitis implévit.
\switchcolumn
\selectlanguage{english}
At Tarsus in Cilicia, in the reign 
 of Emperor Diocletian, the holy martyrs Cyricus and Julitta, his mother. 
 Cyricus, a child of three years, seeing his mother cruelly scourged with 
 whips in the presence of the governor Alexander, and crying bitterly, was 
 killed by being dashed against the steps of the tribunal. Julitta, 
 after being subjected to severe lashings and grievous torments, closed the 
 course of her martyrdom by beheading.
\switchcolumn*
\selectlanguage{latin}
Amathúnte, in Cypro, 
 sancti Tychónis Epíscopi, témpore Theodósii junióris.
\switchcolumn
\selectlanguage{english}
At Amathus in Cyprus, St. Tychon, a 
 bishop in the time of Theodosius the Younger.
\switchcolumn*
\selectlanguage{latin}
Lugdúni, in Gállia, 
 deposítio beáti Aureliáni, Epíscopi Arelaténsis.
\switchcolumn
\selectlanguage{english}
At Lyons, the death of blessed 
 Aurelian, bishop of Arles.
\switchcolumn*
\selectlanguage{latin}
Nannéte, in Británnia minóre, sancti Similiáni, Epíscopi et Confessóris.
\switchcolumn
\selectlanguage{english}
At Nantes in Brittany, St. Similian, 
 bishop and confessor.
\switchcolumn*
\selectlanguage{latin}
Misnæ, in Germánia, 
 sancti Bennónis Epíscopi.
\switchcolumn
\selectlanguage{english}
At Meissen in Germany, St. Benno, 
 bishop.
\switchcolumn*
\selectlanguage{latin}
In monastério 
 Aquiriénsi, in Brabántia, sanctæ Lutgárdis Vírginis.
\switchcolumn
\selectlanguage{english}
In the monastery of Aywieres in 
 Brabant, St. Lutgard, virgin.
\switchcolumn*
\selectlanguage{latin}
\end{paracol}


% ---- martyrology/mart06/mart0617.htm
\needspace{10\baselineskip}
\begin{paracol}{2}
\selectlanguage{latin}
\begin{center}{\color{gregoriocolor} Quintodécimo Kaléndas Júlii. 
 Luna\dots\ }\end{center}
\switchcolumn
\selectlanguage{english}
\begin{center}{\color{gregoriocolor} The Seventeenth Day of 
 June. The\dots\ Day of the Moon.}\end{center}
\end{paracol}

\noindent\begin{tabularx}{\linewidth}{*{19}{>{\centering\arraybackslash}X}}
 \textcolor{gregoriocolor}{a} & \textcolor{gregoriocolor}{b} & \textcolor{gregoriocolor}{c} & \textcolor{gregoriocolor}{d} & \textcolor{gregoriocolor}{e} & \textcolor{gregoriocolor}{f} & \textcolor{gregoriocolor}{g} & \textcolor{gregoriocolor}{h} & \textcolor{gregoriocolor}{i} & \textcolor{gregoriocolor}{k} & \textcolor{gregoriocolor}{l} & \textcolor{gregoriocolor}{m} & \textcolor{gregoriocolor}{n} & \textcolor{gregoriocolor}{p} & \textcolor{gregoriocolor}{q} & \textcolor{gregoriocolor}{r} & \textcolor{gregoriocolor}{s} & \textcolor{gregoriocolor}{t} & \textcolor{gregoriocolor}{u} \\
 21 & 22 & 23 & 24 & 25 & 26 & 27 & 28 & 29 & 1 & 2 & 3 & 4 & 5 & 6 & 7 & 8 & 9 & 10 \\
\end{tabularx}
\vspace{0.5\baselineskip}
\noindent\begin{tabularx}{\linewidth}{*{12}{>{\centering\arraybackslash}X}}
 \textcolor{gregoriocolor}{A} & \textcolor{gregoriocolor}{B} & \textcolor{gregoriocolor}{C} & \textcolor{gregoriocolor}{D} & \textcolor{gregoriocolor}{E} & F & \textcolor{gregoriocolor}{F} & \textcolor{gregoriocolor}{G} & \textcolor{gregoriocolor}{H} & \textcolor{gregoriocolor}{M} & \textcolor{gregoriocolor}{N} & \textcolor{gregoriocolor}{P} \\
 11 & 12 & 13 & 14 & 15 & 16 & 15 & 16 & 17 & 18 & 19 & 20 \\
\end{tabularx}

\begin{paracol}{2}
\selectlanguage{latin}
\lettrine[lines=2]{R}{omæ} natális sanctórum 
 ducentórum et sexagínta duórum Mártyrum, qui, in persecutióne Diocletiáni, 
 pro fide Christi sunt necáti, ac pósiti via Salária véteri, ad clivum 
 Cucúmeris.
\switchcolumn
\selectlanguage{english}
\lettrine[lines=2]{A}{t} Rome, during the persecution of 
 Diocletian, the birthday of two hundred and sixty-two martyrs, who were put 
 to death for the faith of Christ, and buried on the old Salarian Way, at the 
 foot of Cucumer Hill.
\switchcolumn*
\selectlanguage{latin}
Vesontióne, in Gálliis, 
 sancti Antídii, Epíscopi et Mártyris, qui, ob Christi fidem, a Wándalis 
 occísus fuit.
\switchcolumn
\selectlanguage{english}
At Besançon in France, St. Antidius, 
 bishop and martyr, who was slain by the Vandals for the faith of Christ.
\switchcolumn*
\selectlanguage{latin}
Apollóniæ, in 
 Macedónia, sanctórum Mártyrum Atheniénsium Isáuri Diáconi, Innocéntii, 
 Felícis, Jeremíæ et Peregríni, qui, a Tripóntio Tribúno várie torti, cápite 
 obtruncáti sunt.
\switchcolumn
\selectlanguage{english}
At Apollonia in Macedonia, the holy 
 martyrs Isaurus, a deacon, Innocent, Felix, Jeremias, and Peregrínus, all of 
 them Athenians who were tortured in various ways by the tribune Tripontius, 
 and beheaded.
\switchcolumn*
\selectlanguage{latin}
Tarracínæ, in Campánia, 
 sancti Montáni mílitis, qui sub Hadriáno Imperatóre et Leóntio Consulári, 
 post multa torménta, martyrii corónam accépit.
\switchcolumn
\selectlanguage{english}
At Terracina in Campania, St. 
 Montanus, a soldier, who received the crown of martyrdom after suffering 
 many torments, in the time of Emperor Hadrian and the governor Leontius.
\switchcolumn*
\selectlanguage{latin}
Apud Venáfrum, in 
 Campánia, sanctórum Mártyrum Nicándri et Marciáni, qui, in persecutióne 
 Maximiáni, cápite cæsi sunt.
\switchcolumn
\selectlanguage{english}
At Venafro in Campania, the holy 
 martyrs Nicander and Marcian, who were beheaded in the persecution of 
 Maximian.
\switchcolumn*
\selectlanguage{latin}
Chalcédone sanctórum 
 Mártyrum Manuélis, Sabélis et Ismaélis, qui, pacis causa apud Juliánum 
 Apóstatam pro Persárum Rege legatióne fungéntes, ab ipso Imperatóre, cum 
 idóla venerári compelleréntur idque constánti ánimo recusárent, gládio 
 feríri jubéntur.
\switchcolumn
\selectlanguage{english}
At Chalcedon, the holy martyrs 
 Manuel, Sabel, and Ismael, whom the king of Persia sent as ambassadors to 
 Julian the Apostate to obtain peace. Having firmly refused to worship 
 idols when commanded by the emperor, they were put to the sword.
\switchcolumn*
\selectlanguage{latin}
Amériæ, in Umbria, 
 sancti Himérii Epíscopi, cujus corpus Cremónam, in Insúbria, translátum est.
\switchcolumn
\selectlanguage{english}
At Amelia in Umbria, Bishop St. 
 Himerius, whose body was translated to Cremona.
\switchcolumn*
\selectlanguage{latin}
In pago Bituricénsi, 
 sancti Gundúlphi Epíscopi.
\switchcolumn
\selectlanguage{english}
In the territory of Bourges, St. 
 Gundulphus, bishop.
\switchcolumn*
\selectlanguage{latin}
Aureliánis, in Gállia, 
 sancti Avíti, Presbyteri et Confessóris.
\switchcolumn
\selectlanguage{english}
At Orleans in France, St. Avitus, 
 priest and confessor.
\switchcolumn*
\selectlanguage{latin}
In Phrygia sancti 
 Hypátii Confessóris.
\switchcolumn
\selectlanguage{english}
In Phrygia, St. Hypatius, confessor.
\switchcolumn*
\selectlanguage{latin}
Item sancti Bessariónis 
 Anachorétæ.
\switchcolumn
\selectlanguage{english}
Also, St. Bessarion, anchoret.
\switchcolumn*
\selectlanguage{latin}
Pisis, in Túscia, 
 sancti Rainérii Confessóris.
\switchcolumn
\selectlanguage{english}
At Pisa in Tuscany, St. Rainerius, 
 confessor.
\switchcolumn*
\selectlanguage{latin}
\end{paracol}


% ---- martyrology/mart06/mart0618.htm
\needspace{10\baselineskip}
\begin{paracol}{2}
\selectlanguage{latin}
\begin{center}{\color{gregoriocolor} Quartodécimo Kaléndas Júlii. 
 Luna\dots\ }\end{center}
\switchcolumn
\selectlanguage{english}
\begin{center}{\color{gregoriocolor} The Eighteenth Day of 
 June. The\dots\ Day of the Moon.}\end{center}
\end{paracol}

\noindent\begin{tabularx}{\linewidth}{*{19}{>{\centering\arraybackslash}X}}
 \textcolor{gregoriocolor}{a} & \textcolor{gregoriocolor}{b} & \textcolor{gregoriocolor}{c} & \textcolor{gregoriocolor}{d} & \textcolor{gregoriocolor}{e} & \textcolor{gregoriocolor}{f} & \textcolor{gregoriocolor}{g} & \textcolor{gregoriocolor}{h} & \textcolor{gregoriocolor}{i} & \textcolor{gregoriocolor}{k} & \textcolor{gregoriocolor}{l} & \textcolor{gregoriocolor}{m} & \textcolor{gregoriocolor}{n} & \textcolor{gregoriocolor}{p} & \textcolor{gregoriocolor}{q} & \textcolor{gregoriocolor}{r} & \textcolor{gregoriocolor}{s} & \textcolor{gregoriocolor}{t} & \textcolor{gregoriocolor}{u} \\
 22 & 23 & 24 & 25 & 26 & 27 & 28 & 29 & 1 & 2 & 3 & 4 & 5 & 6 & 7 & 8 & 9 & 10 & 11 \\
\end{tabularx}
\vspace{0.5\baselineskip}
\noindent\begin{tabularx}{\linewidth}{*{12}{>{\centering\arraybackslash}X}}
 \textcolor{gregoriocolor}{A} & \textcolor{gregoriocolor}{B} & \textcolor{gregoriocolor}{C} & \textcolor{gregoriocolor}{D} & \textcolor{gregoriocolor}{E} & F & \textcolor{gregoriocolor}{F} & \textcolor{gregoriocolor}{G} & \textcolor{gregoriocolor}{H} & \textcolor{gregoriocolor}{M} & \textcolor{gregoriocolor}{N} & \textcolor{gregoriocolor}{P} \\
 12 & 13 & 14 & 15 & 16 & 17 & 16 & 17 & 18 & 19 & 20 & 21 \\
\end{tabularx}

\begin{paracol}{2}
\selectlanguage{latin}
\lettrine[lines=2]{E}{déssæ,} in Mesopotámia, sancti Ephræm, Diáconi Edesséni et Confessóris, qui, post multos labóres pro 
 Christi fide suscéptos, doctrína et sanctitáte conspícuus, sub Valénte 
 Imperatóre quiévit in Dómino, et a Benedícto Papa Décimo quinto Doctor 
 Ecclésiæ universális est declarátus.
\switchcolumn
\selectlanguage{english}
\lettrine[lines=2]{A}{t} Edessa in Mesopotamia, St. 
 Ephraem, deacon of the church of Edessa in the time of Emperor Valens and 
 confessor. After suffering many trials for the faith of Christ and 
 gaining great renown for holiness and learning, he went to rest in the Lord. 
 He was declared a doctor of the Universal Church by Pope Benedict XV
\switchcolumn*
\selectlanguage{latin}
Romæ, via Ardeatína, 
 natális sanctórum Mártyrum Marci et Marcelliáni fratrum, qui, in 
 persecutióne Diocletiáni, a Júdice Fabiáno tenti et ad stípitem alligáti, in 
 pédibus clavos auctos accepérunt; cumque non cessárent laudáre Christum, 
 lánceis per látera transfíxi sunt, et cum glória martyrii ad sidérea regna 
 migrárunt.
\switchcolumn
\selectlanguage{english}
At Rome, on the Ardeatine Way, in 
 the persecution of Diocletian, the birthday of the saintly brothers Mark and 
 Marcellian, martyrs, who were arrested by the judge Fabian, tied to a stake, 
 and had sharp nails driven into their feet. Because they would not 
 cease praising the name of Christ they were pierced through the sides with 
 lances, and thus went to the kingdom of heaven with the glory of martyrdom.
\switchcolumn*
\selectlanguage{latin}
Málacæ, in Hispánia, 
 sanctórum Mártyrum Cyríaci, et Paulæ Vírginis, qui, lapídibus óbruti, inter 
 saxa cælo ánimas reddidérunt.
\switchcolumn
\selectlanguage{english}
At Malaga in Spain, the holy martyrs 
 Cyriacus and the virgin Paula, who were overwhelmed with stones, and yielded 
 up their souls to God.
\switchcolumn*
\selectlanguage{latin}
Trípoli, in Phœnícia, 
 sancti Leóntii mílitis, qui, sub Hadriáno Præside, una cum Hypátio Tribúno 
 et Theodúlo, quos convértit ad Christum, per acérba torménta pervénit ad 
 corónam martyrii.
\switchcolumn
\selectlanguage{english}
At Tripoli in Phoenicia, in the time 
 of the governor Adrian, St. Leontius, a soldier, who attained the crown of 
 martyrdom through bitter torments together with the tribune Hypatius and 
 Theodulus, whom he had converted to Christ.
\switchcolumn*
\selectlanguage{latin}
Eódem die sancti 
 Æthérii Mártyris, qui, in persecutióne Diocletiáni, post ignes et álios 
 cruciátus, gládio cæsus est.
\switchcolumn
\selectlanguage{english}
The same day, St. Aetherius, martyr, 
 in the persecution of Diocletian. After enduring fire and other 
 torments, he was put to death with the sword.
\switchcolumn*
\selectlanguage{latin}
Alexandríæ pássio 
 sanctæ Marínæ Vírginis.
\switchcolumn
\selectlanguage{english}
At Alexandria, the passion of St. 
 Marina, virgin.
\switchcolumn*
\selectlanguage{latin}
Burdígalæ sancti Amándi, 
 Epíscopi et Confessóris.
\switchcolumn
\selectlanguage{english}
At Bordeaux, St. Amandus, bishop and 
 confessor.
\switchcolumn*
\selectlanguage{latin}
Apud Saccam, in Sicília, 
 sancti Calógeri Eremítæ, cujus sánctitas in energúmenis liberándis máxime 
 effúlget.
\switchcolumn
\selectlanguage{english}
At Sacca in Sicily, St. Calogerus, 
 hermit, whose holiness is shewn especially in the deliverance of possessed 
 persons.
\switchcolumn*
\selectlanguage{latin}
Schonáugiæ, in Germánia, sanctæ Elísabeth Vírginis, ob monásticæ vitæ observántiam célebris.
\switchcolumn
\selectlanguage{english}
At Schongau in Germany, St. 
 Elizabeth, virgin, celebrated for her observance of the monastic life.
\switchcolumn*
\selectlanguage{latin}
\end{paracol}


% ---- martyrology/mart06/mart0619.htm
\needspace{10\baselineskip}
\begin{paracol}{2}
\selectlanguage{latin}
\begin{center}{\color{gregoriocolor} Tertiodécimo Kaléndas Júlii. 
 Luna\dots\ }\end{center}
\switchcolumn
\selectlanguage{english}
\begin{center}{\color{gregoriocolor} The Nineteenth Day of 
 June. The\dots\ Day of the Moon.}\end{center}
\end{paracol}

\noindent\begin{tabularx}{\linewidth}{*{19}{>{\centering\arraybackslash}X}}
 \textcolor{gregoriocolor}{a} & \textcolor{gregoriocolor}{b} & \textcolor{gregoriocolor}{c} & \textcolor{gregoriocolor}{d} & \textcolor{gregoriocolor}{e} & \textcolor{gregoriocolor}{f} & \textcolor{gregoriocolor}{g} & \textcolor{gregoriocolor}{h} & \textcolor{gregoriocolor}{i} & \textcolor{gregoriocolor}{k} & \textcolor{gregoriocolor}{l} & \textcolor{gregoriocolor}{m} & \textcolor{gregoriocolor}{n} & \textcolor{gregoriocolor}{p} & \textcolor{gregoriocolor}{q} & \textcolor{gregoriocolor}{r} & \textcolor{gregoriocolor}{s} & \textcolor{gregoriocolor}{t} & \textcolor{gregoriocolor}{u} \\
 23 & 24 & 25 & 26 & 27 & 28 & 29 & 1 & 2 & 3 & 4 & 5 & 6 & 7 & 8 & 9 & 10 & 11 & 12 \\
\end{tabularx}
\vspace{0.5\baselineskip}
\noindent\begin{tabularx}{\linewidth}{*{12}{>{\centering\arraybackslash}X}}
 \textcolor{gregoriocolor}{A} & \textcolor{gregoriocolor}{B} & \textcolor{gregoriocolor}{C} & \textcolor{gregoriocolor}{D} & \textcolor{gregoriocolor}{E} & F & \textcolor{gregoriocolor}{F} & \textcolor{gregoriocolor}{G} & \textcolor{gregoriocolor}{H} & \textcolor{gregoriocolor}{M} & \textcolor{gregoriocolor}{N} & \textcolor{gregoriocolor}{P} \\
 13 & 14 & 15 & 16 & 17 & 18 & 17 & 18 & 19 & 20 & 21 & 22 \\
\end{tabularx}

\begin{paracol}{2}
\selectlanguage{latin}
\lettrine[lines=2]{F}{loréntiæ} sanctæ 
 Juliánæ Falconériæ Vírginis, quæ Sorórum Ordinis Servórum beátæ Maríæ 
 Vírginis fuit Institútrix, et a Cleménte Duodécimo, Pontífice Máximo, in 
 sanctárum Vírginum númerum reláta est.
\switchcolumn
\selectlanguage{english}
\lettrine[lines=2]{A}{t} Florence, St. Juliana Falconieri, 
 virgin, foundress of the Sisters of the Order of the Servants of the Blessed 
 Virgin Mary, who was placed among the holy virgins by the Sovereign Pontiff, 
 Clement XII.
\switchcolumn*
\selectlanguage{latin}
Medioláni sanctórum 
 Mártyrum Gervásii et Protásii fratrum, ex quibus priórem támdiu jussit 
 Astásius Judex plumbátis cædi, quoúsque ille spíritum exhaláret; posteriórem 
 vero, fústibus cæsum, cápite truncári. Horum córpora, Dómino revelánte, 
 beátus Ambrósius sánguine conspérsa et ita incorrúpta réperit, ac si eo die 
 ipsi fuíssent interémpti; in quorum translatióne cæcus, féretri tactu, lumen 
 recépit, et plúrimi, vexáti a dæmónibus, liberáti sunt.
\switchcolumn
\selectlanguage{english}
At Milan, the holy martyrs Gervase 
 and Protase, brothers. The former, by order of the judge Astasius, was 
 scourged with leaded whips for so long that he expired. The latter, 
 after being scourged with rods, was beheaded. Through divine 
 revelation their bodies were found by St. Ambrose. They were partly 
 covered with blood, and as free from corruption as if they had been put to 
 death that very day. When the translation took place, a blind man 
 recovered his sight by touching their relics, and many persons possessed by 
 demons were delivered.
\switchcolumn*
\selectlanguage{latin}
In monastério Vallis 
 Castri, in Picéno, natális sancti Romuáldi Ravennátis, Anachorétæ et 
 Monachórum Camaldulénsium Patris; qui collápsam in Itália eremíticam 
 disciplínam restítuit ac mirífice propagávit. Ejus tamen festívitas 
 recólitur séptimo Idus Februárii, quo die sacræ ipsíus relíquiæ Fabriánum 
 sunt translátæ.
\switchcolumn
\selectlanguage{english}
At the monastery in the valley of 
 Castro in Piceno, the birthday of St. Romuald, anchoret, a native of 
 Ravenna. He was the founder of the Camaldolese monks, and he restored 
 and greatly extended monastic discipline, which was much relaxed in Italy. 
 His feast is observed on the 7th of February, on which day his sacred relics 
 were transferred to Fabriano.
\switchcolumn*
\selectlanguage{latin}
Arétii, in Túscia, 
 sanctórum Mártyrum Gaudéntii Epíscopi, et Culmátii Diáconi, qui, témpore 
 Valentiniáni, furóre Gentílium cæsi sunt.
\switchcolumn
\selectlanguage{english}
At Arezzo in Tuscany, the holy 
 martyrs Gaudentius, bishop, and Culmatius, deacon, who were murdered by the 
 furious heathen, during the reign of Valentinian.
\switchcolumn*
\selectlanguage{latin}
Eódem die sancti 
 Bonifátii, Epíscopi et Mártyris; qui fuit beáti Romuáldi discípulus. 
 Hic, a Gregório Quinto, Románo Pontífice, ad prædicándum Evangélium in 
 Rússiam missus, ibi, cum per ignem transísset illæsus, Regémque ac pópulum 
 baptizásset, a furénte Regis fratre necátus est, atque sic optátam martyrii 
 corónam accépit.
\switchcolumn
\selectlanguage{english}
Also, St. Boniface, martyr, a 
 disciple of blessed Romuald, who was sent by the Roman Pontiff, Gregory V, 
 to preach the Gospel in Russia. Having passed through fire uninjured, 
 and baptized the king and his people, he was killed by the enraged brother 
 of the king, and thus gained the palm of martyrdom which he ardently 
 desired.
\switchcolumn*
\selectlanguage{latin}
Ravénnæ sancti Ursicíni 
 Mártyris, qui, sub Paulíno Júdice, post plúrima torménta, in Dómini 
 confessióne pérmanens immóbilis, cápitis abscissióne martyrium complévit.
\switchcolumn
\selectlanguage{english}
At Ravenna, St. Ursicinus, martyr, 
 who remained constant through many torments in the confession of martyrdom 
 by being beheaded.
\switchcolumn*
\selectlanguage{latin}
Sozópoli, in Pisídia, 
 sancti Zósimi Mártyris, qui, in persecutióne Trajáni, sub Domitiáno Præside, 
 post acérbos cruciátus, cápite amputáto, victor migrávit ad Dóminum.
\switchcolumn
\selectlanguage{english}
At Sozopolis, under the governor 
 Domitian, during the persecution of Trajan, St. Zosimus, martyr, who 
 suffered bitter tortures, was beheaded, and thus triumphantly went to 
 heaven.
\switchcolumn*
\selectlanguage{latin}
\end{paracol}


% ---- martyrology/mart06/mart0620.htm
\needspace{10\baselineskip}
\begin{paracol}{2}
\selectlanguage{latin}
\begin{center}{\color{gregoriocolor} Duodécimo Kaléndas Júlii. 
 Luna\dots\ }\end{center}
\switchcolumn
\selectlanguage{english}
\begin{center}{\color{gregoriocolor} The 
 Twentieth Day of 
 June. The\dots\ Day of the Moon.}\end{center}
\end{paracol}

\noindent\begin{tabularx}{\linewidth}{*{19}{>{\centering\arraybackslash}X}}
 \textcolor{gregoriocolor}{a} & \textcolor{gregoriocolor}{b} & \textcolor{gregoriocolor}{c} & \textcolor{gregoriocolor}{d} & \textcolor{gregoriocolor}{e} & \textcolor{gregoriocolor}{f} & \textcolor{gregoriocolor}{g} & \textcolor{gregoriocolor}{h} & \textcolor{gregoriocolor}{i} & \textcolor{gregoriocolor}{k} & \textcolor{gregoriocolor}{l} & \textcolor{gregoriocolor}{m} & \textcolor{gregoriocolor}{n} & \textcolor{gregoriocolor}{p} & \textcolor{gregoriocolor}{q} & \textcolor{gregoriocolor}{r} & \textcolor{gregoriocolor}{s} & \textcolor{gregoriocolor}{t} & \textcolor{gregoriocolor}{u} \\
 24 & 25 & 26 & 27 & 28 & 29 & 1 & 2 & 3 & 4 & 5 & 6 & 7 & 8 & 9 & 10 & 11 & 12 & 13 \\
\end{tabularx}
\vspace{0.5\baselineskip}
\noindent\begin{tabularx}{\linewidth}{*{12}{>{\centering\arraybackslash}X}}
 \textcolor{gregoriocolor}{A} & \textcolor{gregoriocolor}{B} & \textcolor{gregoriocolor}{C} & \textcolor{gregoriocolor}{D} & \textcolor{gregoriocolor}{E} & F & \textcolor{gregoriocolor}{F} & \textcolor{gregoriocolor}{G} & \textcolor{gregoriocolor}{H} & \textcolor{gregoriocolor}{M} & \textcolor{gregoriocolor}{N} & \textcolor{gregoriocolor}{P} \\
 14 & 15 & 16 & 17 & 18 & 19 & 18 & 19 & 20 & 21 & 22 & 23 \\
\end{tabularx}

\begin{paracol}{2}
\selectlanguage{latin}
\lettrine[lines=2]{I}{n} ínsula Póntia 
 natális sancti Silvérii, Papæ et Mártyris, qui, cum Anthimum, Epíscopum hæréticum, a suo prædecessóre Agapíto depósitum, restitúere noluísset, a 
 Belisário, agénte ímpia Theodóra Augústa, in exsílium pulsus est, et ibídem, 
 pro fide cathólica multis ærúmnis conféctus, defécit.
\switchcolumn
\selectlanguage{english}
\lettrine[lines=2]{O}{n} the island of Pontia, the 
 birthday of St. Silverius, pope and martyr. For refusing to reinstate 
 the heretical bishop Anthimus who had been deposed by his predecessor 
 Agapitus, he was banished to the isle of Pontia by Belisarius, prompted by 
 the wicked empress Theodora. He died there, consumed by many 
 tribulations for the Catholic faith.
\switchcolumn*
\selectlanguage{latin}
Romæ deposítio sancti 
 Nováti, qui fuit fílius beáti Pudéntis Senatóris, et frater sancti Timóthei 
 Presbyteri et sanctárum Christi Vírginum Pudentiánæ et Praxédis, qui ab 
 Apóstolis erudíti sunt in fide. Ipsórum vero domus, in Ecclésiam 
 commutáta, Pastóris Títulus appellátur.
\switchcolumn
\selectlanguage{english}
At Rome, the death of St. Novatius, 
 son of the blessed senator Pudens, and brother of the saintly priest Timothy 
 and the holy virgins of Christ, Pudentiana and Praxedes, who had been 
 instructed in the faith by the apostles. Their house was converted 
 into a church, and bore the title of the Shepherd.
\switchcolumn*
\selectlanguage{latin}
Tomis, in Ponto, 
 sanctórum Mártyrum Pauli et Cyríaci.
\switchcolumn
\selectlanguage{english}
At Tomis in Pontus, the holy martyrs 
 Paul and Cyriacus.
\switchcolumn*
\selectlanguage{latin}
Petræ, in Palæstína, 
 sancti Macárii Epíscopi, qui, ab Ariánis passus multa et in Africa relegátus, 
 Conféssor quiévit in Dómino.
\switchcolumn
\selectlanguage{english}
At Petra in Palestine, St. Macarius, 
 a bishop, who suffered many things from the Arians, and was banished to 
 Africa where he rested in the Lord.
\switchcolumn*
\selectlanguage{latin}
Híspali, in Hispánia, 
 sanctæ Florentínæ Vírginis, soróris sanctórum Leándri et Isidóri Episcopórum.
\switchcolumn
\selectlanguage{english}
At Seville in Spain, the holy virgin 
 Florentina, sister of the sainted bishops Leander and Isidore.
\switchcolumn*
\selectlanguage{latin}
\end{paracol}


% ---- martyrology/mart06/mart0621.htm
\needspace{10\baselineskip}
\begin{paracol}{2}
\selectlanguage{latin}
\begin{center}{\color{gregoriocolor} Undécimo Kaléndas Júlii. 
 Luna\dots\ }\end{center}
\switchcolumn
\selectlanguage{english}
\begin{center}{\color{gregoriocolor} The 
 Twenty-First Day of 
 June. The\dots\ Day of the Moon.}\end{center}
\end{paracol}

\noindent\begin{tabularx}{\linewidth}{*{19}{>{\centering\arraybackslash}X}}
 \textcolor{gregoriocolor}{a} & \textcolor{gregoriocolor}{b} & \textcolor{gregoriocolor}{c} & \textcolor{gregoriocolor}{d} & \textcolor{gregoriocolor}{e} & \textcolor{gregoriocolor}{f} & \textcolor{gregoriocolor}{g} & \textcolor{gregoriocolor}{h} & \textcolor{gregoriocolor}{i} & \textcolor{gregoriocolor}{k} & \textcolor{gregoriocolor}{l} & \textcolor{gregoriocolor}{m} & \textcolor{gregoriocolor}{n} & \textcolor{gregoriocolor}{p} & \textcolor{gregoriocolor}{q} & \textcolor{gregoriocolor}{r} & \textcolor{gregoriocolor}{s} & \textcolor{gregoriocolor}{t} & \textcolor{gregoriocolor}{u} \\
 25 & 26 & 27 & 28 & 29 & 1 & 2 & 3 & 4 & 5 & 6 & 7 & 8 & 9 & 10 & 11 & 12 & 13 & 14 \\
\end{tabularx}
\vspace{0.5\baselineskip}
\noindent\begin{tabularx}{\linewidth}{*{12}{>{\centering\arraybackslash}X}}
 \textcolor{gregoriocolor}{A} & \textcolor{gregoriocolor}{B} & \textcolor{gregoriocolor}{C} & \textcolor{gregoriocolor}{D} & \textcolor{gregoriocolor}{E} & F & \textcolor{gregoriocolor}{F} & \textcolor{gregoriocolor}{G} & \textcolor{gregoriocolor}{H} & \textcolor{gregoriocolor}{M} & \textcolor{gregoriocolor}{N} & \textcolor{gregoriocolor}{P} \\
 15 & 16 & 17 & 18 & 19 & 20 & 19 & 20 & 21 & 22 & 23 & 24 \\
\end{tabularx}

\begin{paracol}{2}
\selectlanguage{latin}
\lettrine[lines=2]{R}{omæ} sancti Aloísii 
 Gonzágæ, Clérici e Societáte Jesu et Confessóris, principátus contémptu et 
 innocéntia vitæ claríssimi, quem, a Summo Pontífice Benedícto Décimo tértio 
 adscríptum Sanctórum fastis et Protectórem juvénibus præsértim studiósis datum, Pius Papa Undécimus cæléstem Christiánæ juventútis univérsæ Patrónum 
 confirmávit solémniter atque íterum declarávit.
\switchcolumn
\selectlanguage{english}
\lettrine[lines=2]{A}{t} Rome, St. Aloysius Gonzaga, 
 cleric of the Society of Jesus and confessor, most renowned for his contempt 
 of the princely dignity and the innocence of his life. Pope Benedict 
 XIII placed him on the canon of the saints as special protector of young 
 students; Pope Pius XI confirmed this and again solemnly declared him to be 
 the heavenly patron of all Christian youth.
\switchcolumn*
\selectlanguage{latin}
Item Romæ sanctæ 
 Demétriæ Vírginis, quæ, sanctórum Mártyrum Flaviáni et Dafrósæ fília ac 
 sanctæ Vírginis et Mártyris Bibiánæ soror, ipsa quoque, sub Juliáno Apóstata, 
 martyrio coronáta est.
\switchcolumn
\selectlanguage{english}
Also at Rome, St. Demetria, virgin, 
 daughter of the holy martyrs Flavian and Dafrosa, and the sister of St. 
 Bibiana, virgin and martyr. She was crowned with martyrdom under 
 Julian the Apostate.
\switchcolumn*
\selectlanguage{latin}
Eódem die sancti 
 Eusébii, Samosaténi Epíscopi; qui, témpore Constántii, Imperatóris Ariáni, 
 sub hábitu militári incógnitus Ecclésias Dei visitábat, ut in fide cathólica 
 illas confirmáret. Deínde, sub Valénte, in Thráciam relegátur; sed, 
 réddita pace Ecclésiæ, témpore Theodósii, ab exsílio revocátus, tandem, cum 
 íterum Ecclésias visitáret, ei caput, tégula per mulíerem Ariánam désuper 
 immíssa, confráctum est, sicque Martyr occúbuit.
\switchcolumn
\selectlanguage{english}
The same day, St. Eusebius, bishop 
 of Samosata. In the time of the Arian emperor Constantius, he 
 disguised himself in military dress and visited the churches of God to 
 confirm them in the faith. He was banished into Thrace by Valens, but 
 when peace was restored to the Church in the reign of Theodosius, he was 
 recalled. When he again visited the churches, an Arian woman threw a 
 tile down upon him, which fractured his skull and made him a martyr.
\switchcolumn*
\selectlanguage{latin}
Icónii, in Lycaónia, 
 sancti Teréntii, Epíscopi et Mártyris.
\switchcolumn
\selectlanguage{english}
At Iconium in Lycaonia, St. Terence, 
 bishop and martyr.
\switchcolumn*
\selectlanguage{latin}
Syracúsis, in Sicília, 
 natális sanctórum Mártyrum Rufíni et Mártiæ.
\switchcolumn
\selectlanguage{english}
At Syracuse in Sicily, the birthday 
 of the holy martyrs Rufinus and Martia.
\switchcolumn*
\selectlanguage{latin}
In Africa sanctórum 
 Mártyrum Cyríaci et Apollináris.
\switchcolumn
\selectlanguage{english}
In Africa, the holy martyrs Cyriacus 
 and Apollinaris.
\switchcolumn*
\selectlanguage{latin}
Mogúntiæ sancti Albáni 
 Mártyris, qui ob Christi fidem, post longos labóres et dura certámina, 
 factus est dignus coróna vitæ.
\switchcolumn
\selectlanguage{english}
At Mainz, St. Alban, martyr, who was 
 made worthy of the crown of life, after long labors and severe combats.
\switchcolumn*
\selectlanguage{latin}
Papíæ sancti Urciscéni, 
 Epíscopi et Confessóris.
\switchcolumn
\selectlanguage{english}
At Pavia, St. Ursiscenus, bishop and 
 confessor.
\switchcolumn*
\selectlanguage{latin}
Apud Tungrénses sancti 
 Martíni Epíscopi.
\switchcolumn
\selectlanguage{english}
At Tongres, St. Martin, bishop.
\switchcolumn*
\selectlanguage{latin}
In pago Ebroicénsi 
 sancti Leutfrídi Abbátis.
\switchcolumn
\selectlanguage{english}
In the parts of Evreux, St. Leutfrid, 
 abbot.
\switchcolumn*
\selectlanguage{latin}
\end{paracol}


% ---- martyrology/mart06/mart0622.htm
\needspace{10\baselineskip}
\begin{paracol}{2}
\selectlanguage{latin}
\begin{center}{\color{gregoriocolor} Décimo Kaléndas Júlii. 
 Luna\dots\ }\end{center}
\switchcolumn
\selectlanguage{english}
\begin{center}{\color{gregoriocolor} The 
 Twenty-Second Day of 
 June. The\dots\ Day of the Moon.}\end{center}
\end{paracol}

\noindent\begin{tabularx}{\linewidth}{*{19}{>{\centering\arraybackslash}X}}
 \textcolor{gregoriocolor}{a} & \textcolor{gregoriocolor}{b} & \textcolor{gregoriocolor}{c} & \textcolor{gregoriocolor}{d} & \textcolor{gregoriocolor}{e} & \textcolor{gregoriocolor}{f} & \textcolor{gregoriocolor}{g} & \textcolor{gregoriocolor}{h} & \textcolor{gregoriocolor}{i} & \textcolor{gregoriocolor}{k} & \textcolor{gregoriocolor}{l} & \textcolor{gregoriocolor}{m} & \textcolor{gregoriocolor}{n} & \textcolor{gregoriocolor}{p} & \textcolor{gregoriocolor}{q} & \textcolor{gregoriocolor}{r} & \textcolor{gregoriocolor}{s} & \textcolor{gregoriocolor}{t} & \textcolor{gregoriocolor}{u} \\
 26 & 27 & 28 & 29 & 1 & 2 & 3 & 4 & 5 & 6 & 7 & 8 & 9 & 10 & 11 & 12 & 13 & 14 & 15 \\
\end{tabularx}
\vspace{0.5\baselineskip}
\noindent\begin{tabularx}{\linewidth}{*{12}{>{\centering\arraybackslash}X}}
 \textcolor{gregoriocolor}{A} & \textcolor{gregoriocolor}{B} & \textcolor{gregoriocolor}{C} & \textcolor{gregoriocolor}{D} & \textcolor{gregoriocolor}{E} & F & \textcolor{gregoriocolor}{F} & \textcolor{gregoriocolor}{G} & \textcolor{gregoriocolor}{H} & \textcolor{gregoriocolor}{M} & \textcolor{gregoriocolor}{N} & \textcolor{gregoriocolor}{P} \\
 16 & 17 & 18 & 19 & 20 & 21 & 20 & 21 & 22 & 23 & 24 & 25 \\
\end{tabularx}

\begin{paracol}{2}
\selectlanguage{latin}
\lettrine[lines=2]{A}{pud} Nolam, Campániæ 
 urbem, natális beáti Paulíni, Epíscopi et Confessóris, qui ex nobilíssimo et 
 opulentíssimo factus est pro Christo pauper et húmilis, et, quod supérerat, 
 seípsum pro rediméndo víduæ fílio, quem Wándali, Campánia devastáta, 
 captívum in Africam abdúxerant, in servitútem dedit. Cláruit autem non 
 solum eruditióne et copiósa vitæ sanctitáte, sed étiam poténtia advérsus 
 dæmones; ejúsque præcláras laudes sancti Ambrósius, Hierónymus, Augustínus 
 et Gregórius Papa scriptis suis celebrárunt. Ipsíus corpus, póstea 
 Benevéntum et inde Romam translátum, tandem, Summi Pontíficis Pii Décimi 
 jussu, Nolæ restitútum fuit.
\switchcolumn
\selectlanguage{english}
\lettrine[lines=2]{A}{t} Nola in Campania, the birthday of 
 blessed Paulinus, bishop and confessor, who, although a noble and wealthy 
 man, made himself poor and humble for Christ; and what is still more 
 admirable, became a slave to liberate a widow's son who had been carried to 
 Africa by the Vandals when they devastated Campania. He was 
 celebrated, not only for his learning and great holiness of life, but also 
 for his power over demons. His great merit has been extolled by Saints 
 Ambrose, Jerome, Augustine, and Gregory in their writings. His body 
 was translated to Benevento, and later to Rome, but was taken back to Nola 
 by the order of Pope Pius X.
\switchcolumn*
\selectlanguage{latin}
Londíni in Anglia, 
 sancti Joánnis Fisher, Epíscopi Roffénsis et Cardinális, qui pro fide 
 cathólica et Románi Pontificis primátu, jubénte Henríco Octávo Rege, 
 decollátus est.
\switchcolumn
\selectlanguage{english}
At London in England, on Tower Hill, 
 St. John Fisher, bishop of Rochester and cardinal of the Holy Roman Church. 
 For the defence of the Catholic faith and the primacy of the Roman Pontiff 
 he was beheaded by order of King Henry VIII.
\switchcolumn*
\selectlanguage{latin}
In Monte Ararath pássio 
 sanctórum Mártyrum decem míllium, crucifixórum.
\switchcolumn
\selectlanguage{english}
On Mt. Ararat, the martyrdom of ten 
 thousand holy martyrs, who were crucified.
\switchcolumn*
\selectlanguage{latin}
Verolámii, in 
 Británnia, sancti Albáni Mártyris, qui, témpore Diocletiáni, pro Clérico 
 hóspite, quem domi excéperat et a quo Christiánæ fídei præceptiónibus 
 imbútus fúerat, seípsum, commutáta veste, trádidit; et hanc ob causam, post 
 vérbera et acérba torménta, cápite plexus est. Passus est étiam cum 
 illo unus de milítibus, qui, dum eum dúceret ad supplícium, in via convérsus 
 est ad Christum, et mox, gládio decollátus, próprio sánguine méruit 
 baptizári. Hoc autem nóbile sancti Albáni ac Sócii durátum pro Deo 
 certámen sanctus Beda Venerábilis descrípsit.
\switchcolumn
\selectlanguage{english}
At Verulam in England, in the time 
 of Diocletian, St. Alban, martyr, who gave himself up in order to save a 
 cleric whom he had harboured. After being scourged and subjected to 
 bitter torments, he was sentenced to capital punishment. With him also 
 suffered one of the soldiers who led him to execution, for he was converted 
 to Christ on the way and merited to be baptized in his own blood. St. 
 Venerable Bede has left an account of the noble combat of St. Alban and his 
 companion.
\switchcolumn*
\selectlanguage{latin}
Samaríæ, in Palæstína, 
 sanctórum mille quadringentórum octogínta Mártyrum, qui, sub Rege Persárum 
 Chósroa, pro Christi fide interfécti sunt.
\switchcolumn
\selectlanguage{english}
At Samaria in Palestine, fourteen 
 hundred and eighty holy martyrs, under Chosroes, king of Persia.
\switchcolumn*
\selectlanguage{latin}
Eódem die sancti Nicétæ, 
 Romantiánæ civitátis Epíscopi, doctrína sanctísque móribus clari.
\switchcolumn
\selectlanguage{english}
The same day, St. Nicaeas, bishop of 
 the town of Romatia, celebrated for his learning and holy life.
\switchcolumn*
\selectlanguage{latin}
Neápoli, in Campánia, 
 sancti Joánnis Epíscopi, quem beátus Paulínus, Epíscopus Nolánus, ad 
 cæléstia regna vocávit.
\switchcolumn
\selectlanguage{english}
At Naples in Campania, St. John, 
 bishop, who was called to the kingdom of heaven by blessed Paulinus, bishop 
 of Nola.
\switchcolumn*
\selectlanguage{latin}
In monastério 
 Cluniacénsi, in Gállia, deposítio sanctæ Consórtiæ Vírginis.
\switchcolumn
\selectlanguage{english}
In the monastery of Cluny, St. 
 Consortia, virgin.
\switchcolumn*
\selectlanguage{latin}
Romæ beáti Innocéntii 
 Papæ Quinti, ex Ordine Prædicatórum, Confessóris, qui ad tuéndam Ecclésiæ 
 libertátem et Christianórum concórdiam suávi prudéntia adlaborávit. 
 Cultum autem, ei exhíbitum, Leo Décimus tértius, Póntifex Máximus, ratum 
 hábuit et confirmávit.
\switchcolumn
\selectlanguage{english}
At Rome, blessed Pope Innocent V, 
 who laboured with mildness and prudence to maintain liberty for the Church 
 and harmony among the Christians. The veneration paid to him was 
 approved and confirmed by Pope Leo XIII.
\switchcolumn*
\selectlanguage{latin}
Item Romæ Translátio 
 sancti Flávii Cleméntis, viri Consuláris et Mártyris; qui, sanctæ Plautíllæ 
 frater ac beátæ Vírginis et Mártyris Fláviæ Domitíllæ avúnculus, a Domitiáno 
 Imperatóre, quocum Consulátum gésserat, ob Christi fidem interémptus est. 
 Ipsíus porro corpus, in Basílica sancti Cleméntis Papæ invéntum, ibídem 
 solémni pompa recónditum est.
\switchcolumn
\selectlanguage{english}
Likewise at Rome, the translation of 
 St. Flavius Clemens, exconsul and martyr, brother of St. Plautilla and uncle 
 of St. Flavia Domitilla, virgin and martyr. He was put to death for 
 the faith of Christ by Emperor Domitian. His body was found in the 
 Basilica of Pope St. Clement, and buried there with great pomp.
\switchcolumn*
\selectlanguage{latin}
\end{paracol}


% ---- martyrology/mart06/mart0623.htm
\needspace{10\baselineskip}
\begin{paracol}{2}
\selectlanguage{latin}
\begin{center}{\color{gregoriocolor} Nono Kaléndas Júlii. 
 Luna\dots\ }\end{center}
\switchcolumn
\selectlanguage{english}
\begin{center}{\color{gregoriocolor} The 
 Twenty-Third Day of 
 June. The\dots\ Day of the Moon.}\end{center}
\end{paracol}

\noindent\begin{tabularx}{\linewidth}{*{19}{>{\centering\arraybackslash}X}}
 \textcolor{gregoriocolor}{a} & \textcolor{gregoriocolor}{b} & \textcolor{gregoriocolor}{c} & \textcolor{gregoriocolor}{d} & \textcolor{gregoriocolor}{e} & \textcolor{gregoriocolor}{f} & \textcolor{gregoriocolor}{g} & \textcolor{gregoriocolor}{h} & \textcolor{gregoriocolor}{i} & \textcolor{gregoriocolor}{k} & \textcolor{gregoriocolor}{l} & \textcolor{gregoriocolor}{m} & \textcolor{gregoriocolor}{n} & \textcolor{gregoriocolor}{p} & \textcolor{gregoriocolor}{q} & \textcolor{gregoriocolor}{r} & \textcolor{gregoriocolor}{s} & \textcolor{gregoriocolor}{t} & \textcolor{gregoriocolor}{u} \\
 27 & 28 & 29 & 1 & 2 & 3 & 4 & 5 & 6 & 7 & 8 & 9 & 10 & 11 & 12 & 13 & 14 & 15 & 16 \\
\end{tabularx}
\vspace{0.5\baselineskip}
\noindent\begin{tabularx}{\linewidth}{*{12}{>{\centering\arraybackslash}X}}
 \textcolor{gregoriocolor}{A} & \textcolor{gregoriocolor}{B} & \textcolor{gregoriocolor}{C} & \textcolor{gregoriocolor}{D} & \textcolor{gregoriocolor}{E} & F & \textcolor{gregoriocolor}{F} & \textcolor{gregoriocolor}{G} & \textcolor{gregoriocolor}{H} & \textcolor{gregoriocolor}{M} & \textcolor{gregoriocolor}{N} & \textcolor{gregoriocolor}{P} \\
 17 & 18 & 19 & 20 & 21 & 22 & 21 & 22 & 23 & 24 & 25 & 26 \\
\end{tabularx}

\begin{paracol}{2}
\selectlanguage{latin}
\lettrine[lines=1]{V}{igília} Nativitátis 
 sancti Joánnis Baptístæ.
\switchcolumn
\selectlanguage{english}
\lettrine[lines=1]{T}{he} Vigil of St. John Baptist.
\switchcolumn*
\selectlanguage{latin}
Romæ sancti Joánnis 
 Presbyteri, qui, sub Juliáno Apóstata, via Salária véteri, ante simulácrum 
 Solis decollátus est, et corpus ejus a beáto Concórdio Presbytero juxta 
 Mártyrum Concília sepúltum.
\switchcolumn
\selectlanguage{english}
At Rome, in the reign of Julian the 
 Apostate, St. John, a priest who was beheaded on the old Salarian Way before 
 an idol of the sun. His body was buried near those of other martyrs by 
 the blessed priest Concordius.
\switchcolumn*
\selectlanguage{latin}
Item Romæ sanctæ 
 Agrippínæ, Vírginis et Mártyris, quæ sub Valeriáno Imperatóre martyrium 
 consummávit. Ipsíus autem corpus, in Sicíliam translátum ac Menis 
 cónditum, multis miráculis corúscat.
\switchcolumn
\selectlanguage{english}
Also at Rome, St. Agrippina, virgin 
 and martyr, under the emperor Valerian. Her body was taken to Sicily, 
 where it works many miracles.
\switchcolumn*
\selectlanguage{latin}
Sútrii, in Túscia, 
 sancti Felícis Presbyteri, cujus os támdiu jussit Túrcius Præféctus lápide 
 contúndi, donec ipse Felix emítteret spíritum.
\switchcolumn
\selectlanguage{english}
At Sutri in Tuscany, St. Felix, 
 priest. By the command of the prefect Turcius, he was struck on the 
 mouth with a stone until he breathed no more.
\switchcolumn*
\selectlanguage{latin}
Nicomedíæ commemorátio 
 plurimórum sanctórum Mártyrum, qui, témpore Diocletiáni, in móntibus et 
 spelúncis laténtes, pro Christi nómine martyrium læto ánimo subiérunt.
\switchcolumn
\selectlanguage{english}
At Nicomedia, in the time of 
 Diocletian, the commemoration of many holy martyrs who concealed themselves 
 in mountains and caverns, but joyfully underwent martyrdom for the name of 
 Christ.
\switchcolumn*
\selectlanguage{latin}
Philadelphíæ, in 
 Arábia, sanctórum Mártyrum Zenónis, ejúsque servi Zenæ. Hic dómini sui 
 vincti caténas exósculans, eúmque rogans ut se in torméntis partícipem 
 dignarétur habére, a milítibus tentus est, et cum ipso dómino parem martyrii 
 corónam accépit.
\switchcolumn
\selectlanguage{english}
At Philadelphia in Arabia, the holy 
 martyrs Zeno and his slave Zenas. When the latter kissed the chains of 
 his master, begging to be a partner in his torments, he was arrested by the 
 soldiers, and received the crown of martyrdom with him.
\switchcolumn*
\selectlanguage{latin}
Augústæ Taurinórum 
 sancti Joséphi Cafásso, Sacerdótis, qui levítis pietáte et sciéntiæ augéndis 
 atque damnátis cápite Deo conciliándis fuit illústris, et a Pio Papa 
 Duodécimo inter sanctos Cælites adscríptus est.
\switchcolumn
\selectlanguage{english}
At Turin, St. Joseph Cafasso, 
 priest, renowned for his piety and learning, and for his work with 
 prisoners, reconciling to God those who were preparing for execution. 
 He was added to the number of the Saints by Pope Pius XII.
\switchcolumn*
\selectlanguage{latin}
In monastério Elyénsi, 
 in Británnia, sanctæ Ediltrúdis, Regínæ et Vírginis, quæ sanctitáte et 
 miráculis clara migrávit ad Dóminum. Ipsíus autem corpus, úndecim post 
 annis, invéntum est incorrúptum.
\switchcolumn
\selectlanguage{english}
In England, in the monastery of Ely, 
 St. Etheldreda, queen and virgin, who departed for heaven with a great 
 renown for sanctity and miracles. Her body was found without 
 corruption eleven years afterwards.
\switchcolumn*
\selectlanguage{latin}
\end{paracol}


% ---- martyrology/mart06/mart0624.htm
\needspace{10\baselineskip}
\begin{paracol}{2}
\selectlanguage{latin}
\begin{center}{\color{gregoriocolor} Octávo Kaléndas Júlii. 
 Luna\dots\ }\end{center}
\switchcolumn
\selectlanguage{english}
\begin{center}{\color{gregoriocolor} The 
 Twenty-Fourth Day of 
 June. The\dots\ Day of the Moon.}\end{center}
\end{paracol}

\noindent\begin{tabularx}{\linewidth}{*{19}{>{\centering\arraybackslash}X}}
 \textcolor{gregoriocolor}{a} & \textcolor{gregoriocolor}{b} & \textcolor{gregoriocolor}{c} & \textcolor{gregoriocolor}{d} & \textcolor{gregoriocolor}{e} & \textcolor{gregoriocolor}{f} & \textcolor{gregoriocolor}{g} & \textcolor{gregoriocolor}{h} & \textcolor{gregoriocolor}{i} & \textcolor{gregoriocolor}{k} & \textcolor{gregoriocolor}{l} & \textcolor{gregoriocolor}{m} & \textcolor{gregoriocolor}{n} & \textcolor{gregoriocolor}{p} & \textcolor{gregoriocolor}{q} & \textcolor{gregoriocolor}{r} & \textcolor{gregoriocolor}{s} & \textcolor{gregoriocolor}{t} & \textcolor{gregoriocolor}{u} \\
 28 & 29 & 1 & 2 & 3 & 4 & 5 & 6 & 7 & 8 & 9 & 10 & 11 & 12 & 13 & 14 & 15 & 16 & 17 \\
\end{tabularx}
\vspace{0.5\baselineskip}
\noindent\begin{tabularx}{\linewidth}{*{12}{>{\centering\arraybackslash}X}}
 \textcolor{gregoriocolor}{A} & \textcolor{gregoriocolor}{B} & \textcolor{gregoriocolor}{C} & \textcolor{gregoriocolor}{D} & \textcolor{gregoriocolor}{E} & F & \textcolor{gregoriocolor}{F} & \textcolor{gregoriocolor}{G} & \textcolor{gregoriocolor}{H} & \textcolor{gregoriocolor}{M} & \textcolor{gregoriocolor}{N} & \textcolor{gregoriocolor}{P} \\
 18 & 19 & 20 & 21 & 22 & 23 & 22 & 23 & 24 & 25 & 26 & 27 \\
\end{tabularx}

\begin{paracol}{2}
\selectlanguage{latin}
\lettrine[lines=2]{N}{atívitas} sancti 
 Joánnis Baptístæ, Præcursóris Dómini, ac sanctórum Zacharíæ et Elísabeth fílii, qui Spíritu Sancto replétus est adhuc in útero matris suæ.
\switchcolumn
\selectlanguage{english}
\lettrine[lines=2]{T}{he} Nativity of St. John Baptist, 
 precursor of our Lord, son of Zachary and Elizabeth, who, while yet in the 
 womb of his mother, was filled with the Holy Ghost.
\switchcolumn*
\selectlanguage{latin}
Romæ commemorátio 
 sanctórum plurimórum Mártyrum, qui a Neróne Imperatóre, ut a se incénsæ 
 Urbis ódium avérteret, calumnióse accusáti, divérso mortis génere jussi sunt 
 sævíssime intérfici. Horum síquidem álii, ferárum tergis contécti, 
 laniátibus canum expósiti sunt; álii crúcibus affíxi; aliíque incéndio 
 tráditi, ut, ubi defecísset dies, in usum noctúrni lúminis deservírent. 
 Erant hi omnes Apostolórum discípuli, et primítiæ Mártyrum, quas Romána 
 Ecclésia, fértilis ager Mártyrum, ante Apostolórum necem transmísit ad 
 Dóminum.
\switchcolumn
\selectlanguage{english}
At Rome, in the time of Nero, the 
 commemoration of many holy martyrs. Being falsely accused of having 
 set fire to the city, they were cruelly put to death in various manners by 
 the emperor's order. Some were covered with the skins of wild beasts 
 and torn to pieces by dogs, other were fastened to crosses, others again 
 were delivered to the flames to serve as torches in the night. All 
 these were disciples of the apostles, and the first fruits of the martyrs 
 which the Roman Church, a field so fertile in martyrs, offered to God even 
 before the death of the Apostles.
\switchcolumn*
\selectlanguage{latin}
Item Romæ sanctórum 
 Mártyrum Fausti et aliórum vigínti trium.
\switchcolumn
\selectlanguage{english}
In the same city, the holy martyrs 
 Faustus and twenty-three others.
\switchcolumn*
\selectlanguage{latin}
Mechlíniæ, in Brabántia, 
 pássio sancti Rumóldi, Epíscopi Dublinénsis et Mártyris, e Scotórum Rege 
 progéniti.
\switchcolumn
\selectlanguage{english}
At Mechlin in Brabant, the passion 
 of St. Rumold, bishop of Dublin and martyr. He had been the son of the 
 king of the Scots.
\switchcolumn*
\selectlanguage{latin}
Sátalis, in Arménia, sanctórum Mártyrum septem fratrum, scílicet Oréntii, Heróis, Pharnácii, 
 Firmíni, Firmi, Cyríaci et Longíni mílitum; qui a Maximiáno Imperatóre, eo 
 quod Christiáni essent, cíngulo militári priváti sunt, et, ab ínvicem 
 separáti atque in divérsa loca abdúcti, in dolóribus et ærúmnis pósiti, 
 quievérunt in Dómino.
\switchcolumn
\selectlanguage{english}
At Satalis in Armenia, seven saintly 
 brothers, all martyrs: Orentius, Heros, Pharnacius, Firminus, Firmus, 
 Cyriacus and Longinus, who owe their martyrdom to Emperor Maximian. 
 Because they were Christians, they were deprived of the military belt by his 
 command, then separated from one another, hurried away to different places, 
 and in the midst of painful trials found their repose in the Lord.
\switchcolumn*
\selectlanguage{latin}
In vico Christólio, in 
 território Parisiénsi, pássio sanctórum Mártyrum Agoárdi et Aglibérti, cum 
 áliis innúmeris promíscui sexus.
\switchcolumn
\selectlanguage{english}
In the diocese of Paris, at Creteil, 
 the martyrdom of the Saints Agoard and Aglibert, with a great multitude of 
 others of both sexes.
\switchcolumn*
\selectlanguage{latin}
Augustodúni deposítio 
 sancti Simplícii, Epíscopi et Confessóris.
\switchcolumn
\selectlanguage{english}
At Autun, the death of St. 
 Simplicius, bishop and confessor.
\switchcolumn*
\selectlanguage{latin}
Láubiis, in Bélgio, 
 sancti Theodúlphi Epíscopi.
\switchcolumn
\selectlanguage{english}
At Lobbes in Belgium, St. 
 Theodulphus, bishop.
\switchcolumn*
\selectlanguage{latin}
\end{paracol}


% ---- martyrology/mart06/mart0625.htm
\needspace{10\baselineskip}
\begin{paracol}{2}
\selectlanguage{latin}
\begin{center}{\color{gregoriocolor} Séptimo Kaléndas Júlii. 
 Luna\dots\ }\end{center}
\switchcolumn
\selectlanguage{english}
\begin{center}{\color{gregoriocolor} The 
 Twenty-Fifth Day of 
 June. The\dots\ Day of the Moon.}\end{center}
\end{paracol}

\noindent\begin{tabularx}{\linewidth}{*{19}{>{\centering\arraybackslash}X}}
 \textcolor{gregoriocolor}{a} & \textcolor{gregoriocolor}{b} & \textcolor{gregoriocolor}{c} & \textcolor{gregoriocolor}{d} & \textcolor{gregoriocolor}{e} & \textcolor{gregoriocolor}{f} & \textcolor{gregoriocolor}{g} & \textcolor{gregoriocolor}{h} & \textcolor{gregoriocolor}{i} & \textcolor{gregoriocolor}{k} & \textcolor{gregoriocolor}{l} & \textcolor{gregoriocolor}{m} & \textcolor{gregoriocolor}{n} & \textcolor{gregoriocolor}{p} & \textcolor{gregoriocolor}{q} & \textcolor{gregoriocolor}{r} & \textcolor{gregoriocolor}{s} & \textcolor{gregoriocolor}{t} & \textcolor{gregoriocolor}{u} \\
 29 & 1 & 2 & 3 & 4 & 5 & 6 & 7 & 8 & 9 & 10 & 11 & 12 & 13 & 14 & 15 & 16 & 17 & 18 \\
\end{tabularx}
\vspace{0.5\baselineskip}
\noindent\begin{tabularx}{\linewidth}{*{12}{>{\centering\arraybackslash}X}}
 \textcolor{gregoriocolor}{A} & \textcolor{gregoriocolor}{B} & \textcolor{gregoriocolor}{C} & \textcolor{gregoriocolor}{D} & \textcolor{gregoriocolor}{E} & F & \textcolor{gregoriocolor}{F} & \textcolor{gregoriocolor}{G} & \textcolor{gregoriocolor}{H} & \textcolor{gregoriocolor}{M} & \textcolor{gregoriocolor}{N} & \textcolor{gregoriocolor}{P} \\
 19 & 20 & 21 & 22 & 23 & 24 & 23 & 24 & 25 & 26 & 27 & 28 \\
\end{tabularx}

\begin{paracol}{2}
\selectlanguage{latin}
\lettrine[lines=2]{I}{n} território Guléti, 
 prope Nuscum, sancti Guliélmi Confessóris, Patris Eremitárum Montis Vírginis.
\switchcolumn
\selectlanguage{english}
\lettrine[lines=2]{I}{n} the territory of Guletto near 
 Nusco, St. William, confessor, founder of the hermits of Monte Vergine.
\switchcolumn*
\selectlanguage{latin}
Apud Berœam natális 
 sancti Sosípatris, qui fuit discípulus beáti Pauli Apóstoli.
\switchcolumn
\selectlanguage{english}
At Beraea, the birthday of St. 
 Sosipater, disciple of the blessed apostle Paul.
\switchcolumn*
\selectlanguage{latin}
Romæ sanctæ Lúciæ, 
 Vírginis et Mártyris, cum áliis vigínti duóbus.
\switchcolumn
\selectlanguage{english}
At Rome, St. Lucy, virgin and 
 martyr, with twenty-two others.
\switchcolumn*
\selectlanguage{latin}
Alexandríæ sancti 
 Gallicáni Mártyris, viri Consuláris, qui, sublimátus ínfulis triumphálibus 
 et Constantíno Augústo carus, a sanctis Joánne et Paulo ad Christi fidem 
 convérsus est; eáque suscépta, cum sancto Hilaríno secéssit ad Ostia 
 Tiberína, atque ibi hospitalitáti et infirmórum servítio totum se dedit. 
 Cujus rei fama in toto Orbe divulgáta, multi úndique illuc veniéntes 
 vidébant virum ex Patrício et Cónsule laváre páuperum pedes, pónere mensam, 
 aquam mánibus effúndere, languéntibus sollícite ministráre et cétera pietátis offícia exhibére. Ipse póstmodum, sub Juliána Apóstata, inde 
 expúlsus, Alexandríam perréxit; ibíque, cum a Rauciáno Júdice sacrificáre 
 cogerétur et contémneret, ideo, percússus gládio, Christi Martyr efféctus 
 est.
\switchcolumn
\selectlanguage{english}
At Alexandria, St. Gallicanus, 
 exconsul and martyr who had been honoured with a triumph, and was held in 
 affection by the emperor Constantine. Converted by Saints John and 
 Paul, he withdrew to Ostia with St. Hilarinus, and consecrated himself 
 entirely to the duties of hospitality and to the service of the sick. 
 The report of such an event spread throughout the whole world, and from all 
 sides many people came to see a man who had been a senator and consul now 
 washing the feet of the poor, preparing their table, serving them, carefully 
 waiting on the infirm, and exercising other works of mercy. Driven 
 from this place by Julian the Apostate, he repaired to Alexandria, where, 
 for refusing to sacrifice to idols, at the command of the judge Raucian, he 
 was put to the sword, and thus became a martyr of Christ.
\switchcolumn*
\selectlanguage{latin}
Sibápoli, in 
 Mesopotámia, sanctæ Febróniæ, Vírginis et Mártyris; quæ, in persecutióne 
 Diocletiáni, sub Siléno Júdice, ob fidem et pudicítiam servándam, primo 
 virgis cæsa et equúleo torta, deínde pectínibus laniáta atque igne succénsa, 
 demum, excússis déntibus ac mammis pedibúsque abscíssis, cápitis damnáta, 
 tot passiónum ornáta monílibus migrávit ad Sponsum.
\switchcolumn
\selectlanguage{english}
At Sibapolis in Syria, under the 
 governor Silenus, in the persecution of Diocletian, St. Febronia, virgin and 
 martyr. She was scourged and racked for defending her faith and her 
 chastity, then torn with iron combs and exposed to fire. Finally her 
 teeth were broken out, her breasts and feet cut away, and she was condemned 
 to capital punishment, going to her Spouse adorned with sufferings as with 
 so many jewels.
\switchcolumn*
\selectlanguage{latin}
Apud Rhégium sancti 
 Prósperi Aquitáni, ejúsdem urbis Epíscopi, eruditióne ac pietáte insígnis, 
 qui advérsus Pelagiános, pro fide cathólica, strénue decertávit.
\switchcolumn
\selectlanguage{english}
At Reggio, St. Prosper of Aquitaine, 
 bishop of that city, distinguished by his learning and piety. He 
 valiantly combated the Pelagians in defence of the Catholic faith.
\switchcolumn*
\selectlanguage{latin}
Tauríni natális sancti 
 Máximi, Epíscopi et Confessóris, doctrína et sanctitáte celebérrimi.
\switchcolumn
\selectlanguage{english}
At Turin, the birthday of St. 
 Maximus, bishop and confessor, most celebrated for his sanctity and 
 scholarship.
\switchcolumn*
\selectlanguage{latin}
In Hollándia sancti 
 Adelbérti Confessóris, qui fuit discípulus sancti Willibrórdi Epíscopi.
\switchcolumn
\selectlanguage{english}
In Holland, St. Adalbert, confessor, 
 disciple of St. Willibrord, bishop.
\switchcolumn*
\selectlanguage{latin}
\end{paracol}


% ---- martyrology/mart06/mart0626.htm
\needspace{10\baselineskip}
\begin{paracol}{2}
\selectlanguage{latin}
\begin{center}{\color{gregoriocolor} Sexto Kaléndas Júlii. 
 Luna\dots\ }\end{center}
\switchcolumn
\selectlanguage{english}
\begin{center}{\color{gregoriocolor} The 
 Twenty-Sixth Day of 
 June. The\dots\ Day of the Moon.}\end{center}
\end{paracol}

\noindent\begin{tabularx}{\linewidth}{*{19}{>{\centering\arraybackslash}X}}
 \textcolor{gregoriocolor}{a} & \textcolor{gregoriocolor}{b} & \textcolor{gregoriocolor}{c} & \textcolor{gregoriocolor}{d} & \textcolor{gregoriocolor}{e} & \textcolor{gregoriocolor}{f} & \textcolor{gregoriocolor}{g} & \textcolor{gregoriocolor}{h} & \textcolor{gregoriocolor}{i} & \textcolor{gregoriocolor}{k} & \textcolor{gregoriocolor}{l} & \textcolor{gregoriocolor}{m} & \textcolor{gregoriocolor}{n} & \textcolor{gregoriocolor}{p} & \textcolor{gregoriocolor}{q} & \textcolor{gregoriocolor}{r} & \textcolor{gregoriocolor}{s} & \textcolor{gregoriocolor}{t} & \textcolor{gregoriocolor}{u} \\
 1 & 2 & 3 & 4 & 5 & 6 & 7 & 8 & 9 & 10 & 11 & 12 & 13 & 14 & 15 & 16 & 17 & 18 & 19 \\
\end{tabularx}
\vspace{0.5\baselineskip}
\noindent\begin{tabularx}{\linewidth}{*{12}{>{\centering\arraybackslash}X}}
 \textcolor{gregoriocolor}{A} & \textcolor{gregoriocolor}{B} & \textcolor{gregoriocolor}{C} & \textcolor{gregoriocolor}{D} & \textcolor{gregoriocolor}{E} & F & \textcolor{gregoriocolor}{F} & \textcolor{gregoriocolor}{G} & \textcolor{gregoriocolor}{H} & \textcolor{gregoriocolor}{M} & \textcolor{gregoriocolor}{N} & \textcolor{gregoriocolor}{P} \\
 20 & 21 & 22 & 23 & 24 & 25 & 24 & 25 & 26 & 27 & 28 & 29 \\
\end{tabularx}

\begin{paracol}{2}
\selectlanguage{latin}
\lettrine[lines=2]{R}{omæ,} in monte Cælio, 
 sanctórum Mártyrum Joánnis et Pauli fratrum, quorum primus erat præpósitus 
 domus, secúndus primicérius Constántiæ Vírginis, fíliæ Constantíni 
 Imperatóris, et ambo póstea, sub Juliáno Apóstata, martyrii palmam, cædénte 
 gládio, percepérunt.
\switchcolumn
\selectlanguage{english}
\lettrine[lines=2]{A}{t} Rome on Mt. Coelius, the holy 
 martyrs John and Paul, brothers. The former was steward, the other 
 secretary of the virgin Constantia, daughter of Emperor Constantine. 
 Afterwards, under Julian the Apostate, they received the palm of martyrdom 
 by being beheaded.
\switchcolumn*
\selectlanguage{latin}
Apud Tridéntum sancti 
 Vigílii Epíscopi, qui, cum relíquias idololatríæ pénitus exstirpáre 
 conarétur, ideo, pro Christi nómine, a feris et bárbaris homínibus percússus 
 lápidum imbre, martyrium implévit.
\switchcolumn
\selectlanguage{english}
At Trent, St. Vigilius, bishop, who, 
 while he endeavoured to root out the remains of idolatry, was overwhelmed 
 with a shower of stones by cruel and barbarous men, and thus endured 
 martyrdom for the name of Christ.
\switchcolumn*
\selectlanguage{latin}
Apud Valencénas, in 
 Gállia, pássio sanctórum Mártyrum Sálvii, qui éxstitit Engolisménsis 
 Epíscopus, et Supérii.
\switchcolumn
\selectlanguage{english}
At Valenciennes, the holy martyrs Salvius, bishop of Angoulême, and Superius.
\switchcolumn*
\selectlanguage{latin}
Córdubæ, in Hispánia, 
 natális sancti Pelágii adolescéntuli, qui, ob confessiónem fídei, Regis 
 Saracenórum Abdaraméni jussu forcípibus férreis membrátim præcísus, 
 martyrium suum glorióse consummávit.
\switchcolumn
\selectlanguage{english}
At Cordova in Spain, under the 
 Saracen king Abderaliman, the birthday of St. Pelagius, a young man who 
 gloriously completed his martyrdom for the faith by having his flesh torn to 
 pieces with iron pincers.
\switchcolumn*
\selectlanguage{latin}
Bellícii, in Gállia, 
 sancti Anthélmi, qui ex majóris Carthúsiæ Prióre, factus est ejúsdem 
 civitátis Epíscopus.
\switchcolumn
\selectlanguage{english}
At Belley in France, St. Anthelmus, 
 prior of the Grande Chartreuse, who became bishop of that city.
\switchcolumn*
\selectlanguage{latin}
In pago Pictaviénsi 
 sancti Maxéntii, Presbyteri et Confessóris; qui miráculis cláruit.
\switchcolumn
\selectlanguage{english}
In the country of Poitiers, St. 
 Maxentius, priest and confessor, renowned for miracles.
\switchcolumn*
\selectlanguage{latin}
Thessalonícæ sancti 
 David Eremítæ.
\switchcolumn
\selectlanguage{english}
At Thessalonica, St. David, hermit.
\switchcolumn*
\selectlanguage{latin}
Eódem die sanctæ 
 Perseverándæ Vírginis.
\switchcolumn
\selectlanguage{english}
The same day, St. Perseveranda, 
 virgin.
\switchcolumn*
\selectlanguage{latin}
\end{paracol}


% ---- martyrology/mart06/mart0627.htm
\needspace{10\baselineskip}
\begin{paracol}{2}
\selectlanguage{latin}
\begin{center}{\color{gregoriocolor} Quinto Kaléndas Júlii. 
 Luna\dots\ }\end{center}
\switchcolumn
\selectlanguage{english}
\begin{center}{\color{gregoriocolor} The 
 Twenty-Seventh Day of 
 June. The\dots\ Day of the Moon.}\end{center}
\end{paracol}

\noindent\begin{tabularx}{\linewidth}{*{19}{>{\centering\arraybackslash}X}}
 \textcolor{gregoriocolor}{a} & \textcolor{gregoriocolor}{b} & \textcolor{gregoriocolor}{c} & \textcolor{gregoriocolor}{d} & \textcolor{gregoriocolor}{e} & \textcolor{gregoriocolor}{f} & \textcolor{gregoriocolor}{g} & \textcolor{gregoriocolor}{h} & \textcolor{gregoriocolor}{i} & \textcolor{gregoriocolor}{k} & \textcolor{gregoriocolor}{l} & \textcolor{gregoriocolor}{m} & \textcolor{gregoriocolor}{n} & \textcolor{gregoriocolor}{p} & \textcolor{gregoriocolor}{q} & \textcolor{gregoriocolor}{r} & \textcolor{gregoriocolor}{s} & \textcolor{gregoriocolor}{t} & \textcolor{gregoriocolor}{u} \\
 2 & 3 & 4 & 5 & 6 & 7 & 8 & 9 & 10 & 11 & 12 & 13 & 14 & 15 & 16 & 17 & 18 & 19 & 20 \\
\end{tabularx}
\vspace{0.5\baselineskip}
\noindent\begin{tabularx}{\linewidth}{*{12}{>{\centering\arraybackslash}X}}
 \textcolor{gregoriocolor}{A} & \textcolor{gregoriocolor}{B} & \textcolor{gregoriocolor}{C} & \textcolor{gregoriocolor}{D} & \textcolor{gregoriocolor}{E} & F & \textcolor{gregoriocolor}{F} & \textcolor{gregoriocolor}{G} & \textcolor{gregoriocolor}{H} & \textcolor{gregoriocolor}{M} & \textcolor{gregoriocolor}{N} & \textcolor{gregoriocolor}{P} \\
 21 & 22 & 23 & 24 & 25 & 26 & 25 & 26 & 27 & 28 & 29 & 1 \\
\end{tabularx}

\begin{paracol}{2}
\selectlanguage{latin}
\lettrine[lines=2]{I}{n} Galátia sancti 
 Crescéntis, qui fuit beáti Pauli Apóstoli discípulus. Hic, in Gállias 
 tránsitum fáciens, verbo prædicatiónis multos ad fidem Christi convértit; 
 regréssus vero ad gentem cui speciáliter datus erat Epíscopus, demum, sub 
 Trajáno, cum Gálatas ipsos usque ad finem vitæ suæ in ópere Dómini 
 confirmásset, martyrium consummávit.
\switchcolumn
\selectlanguage{english}
\lettrine[lines=2]{I}{n} Galatia, St. Crescens, disciple 
 of the blessed apostle Paul. In passing through Gaul he converted many 
 to the Christian faith by his preaching. Returning to the people for 
 whom he had been especially made bishop, he confirmed the Galatians in the 
 service of the Lord to the end of his life. He finally completed his 
 martyrdom under Trajan.
\switchcolumn*
\selectlanguage{latin}
Córdubæ, in Hispánia, 
 sanctórum Mártyrum Zóili, et aliórum decem et novem.
\switchcolumn
\selectlanguage{english}
At Cordova in Spain, St. Zoilus and 
 nineteen other martyrs.
\switchcolumn*
\selectlanguage{latin}
Cæsaréæ, in Palæstína, 
 sancti Anécti Mártyris, qui, in persecutióne Diocletiáni, sub Urbáno Præside, 
 cum álios ad martyrium hortátus esset et idóla oratióne sua prostrásset, a 
 decem milítibus verberári jussus est; ac tandem, mánibus pedibúsque præcísis, 
 truncátus cápite, martyrii corónam accépit.
\switchcolumn
\selectlanguage{english}
At Caesarea in Palestine, in the 
 persecution of Diocletian, under the governor Urban, St. Anectus, martyr. 
 For having exhorted others to suffer martyrdom, and having overthrown idols 
 by his prayers, he was scourged by ten soldiers, had his hands and feet cut 
 off, and merited the crown of martyrdom by beheading.
\switchcolumn*
\selectlanguage{latin}
Constantinópoli sancti 
 Sampsónis Presbyteri, páuperum exceptóris.
\switchcolumn
\selectlanguage{english}
At Constantinople, St. Sampson, a 
 priest, who harboured the poor.
\switchcolumn*
\selectlanguage{latin}
In castro Cainóne, in 
 Gállia, sancti Joánnis, Presbyteri et Confessóris.
\switchcolumn
\selectlanguage{english}
In the town of Chinon in France, St. 
 John, priest and confessor.
\switchcolumn*
\selectlanguage{latin}
Varadíni, in Hungária, 
 sancti Ladislái Regis, qui claríssimis miráculis usque ad diem hodiérnum 
 corúscat.
\switchcolumn
\selectlanguage{english}
At Grosswardein in Hungary, the holy 
 king Ladislaus, greatly renowned for his miracles even to this day.
\switchcolumn*
\selectlanguage{latin}
\end{paracol}


% ---- martyrology/mart06/mart0628.htm
\needspace{10\baselineskip}
\begin{paracol}{2}
\selectlanguage{latin}
\begin{center}{\color{gregoriocolor} Quarto Kaléndas Júlii. 
 Luna\dots\ }\end{center}
\switchcolumn
\selectlanguage{english}
\begin{center}{\color{gregoriocolor} The 
 Twenty-Eighth Day of 
 June. The\dots\ Day of the Moon.}\end{center}
\end{paracol}

\noindent\begin{tabularx}{\linewidth}{*{19}{>{\centering\arraybackslash}X}}
 \textcolor{gregoriocolor}{a} & \textcolor{gregoriocolor}{b} & \textcolor{gregoriocolor}{c} & \textcolor{gregoriocolor}{d} & \textcolor{gregoriocolor}{e} & \textcolor{gregoriocolor}{f} & \textcolor{gregoriocolor}{g} & \textcolor{gregoriocolor}{h} & \textcolor{gregoriocolor}{i} & \textcolor{gregoriocolor}{k} & \textcolor{gregoriocolor}{l} & \textcolor{gregoriocolor}{m} & \textcolor{gregoriocolor}{n} & \textcolor{gregoriocolor}{p} & \textcolor{gregoriocolor}{q} & \textcolor{gregoriocolor}{r} & \textcolor{gregoriocolor}{s} & \textcolor{gregoriocolor}{t} & \textcolor{gregoriocolor}{u} \\
 3 & 4 & 5 & 6 & 7 & 8 & 9 & 10 & 11 & 12 & 13 & 14 & 15 & 16 & 17 & 18 & 19 & 20 & 21 \\
\end{tabularx}
\vspace{0.5\baselineskip}
\noindent\begin{tabularx}{\linewidth}{*{12}{>{\centering\arraybackslash}X}}
 \textcolor{gregoriocolor}{A} & \textcolor{gregoriocolor}{B} & \textcolor{gregoriocolor}{C} & \textcolor{gregoriocolor}{D} & \textcolor{gregoriocolor}{E} & F & \textcolor{gregoriocolor}{F} & \textcolor{gregoriocolor}{G} & \textcolor{gregoriocolor}{H} & \textcolor{gregoriocolor}{M} & \textcolor{gregoriocolor}{N} & \textcolor{gregoriocolor}{P} \\
 22 & 23 & 24 & 25 & 26 & 27 & 26 & 27 & 28 & 29 & 1 & 2 \\
\end{tabularx}

\begin{paracol}{2}
\selectlanguage{latin}
\lettrine[lines=1]{V}{igília} sanctórum 
 Apostolórum Petri et Pauli.
\switchcolumn
\selectlanguage{english}
\lettrine[lines=1]{T}{he} vigil of the holy apostles Peter 
 and Paul.
\switchcolumn*
\selectlanguage{latin}
Lugdúni, in Gállia, 
 sancti Irenæi, Epíscopi et Mártyris; qui (ut scribit sanctus Hierónymus) 
 beáti Polycárpi, Smyrnénsis Epíscopi, discípulus fuit, et Apostolicórum 
 témporum vicínis. Is, cum advérsus hæréticos verbis ac scriptis 
 plúrimum decertásset, tandem, in persecutióne Sevéri, cum omni fere 
 civitátis suæ pópulo, coronátus est glorióso martyrio.
\switchcolumn
\selectlanguage{english}
At Lyons in France, St. Irenaeus, 
 bishop and martyr. St. Jerome relates that he was the disciple of 
 blessed Polycarp, bishop of Smyrna, and lived near the time of the apostles. 
 After having strenuously opposed the heretics by word and by writing, he was 
 crowned with a glorious martyrdom along with almost all the people of his 
 city, during the persecution of Severus.
\switchcolumn*
\selectlanguage{latin}
Trajécti sancti Benígni, 
 Epíscopi et Mártyris.
\switchcolumn
\selectlanguage{english}
At Utrecht, St. Benignus, bishop and 
 martyr.
\switchcolumn*
\selectlanguage{latin}
Alexandríæ, in 
 persecutióne Sevéri, sanctórum Mártyrum Plutárchi, Seréni, Heraclídis 
 catechúmeni, Herónis neóphyti, et altérius Seréni, Rháidis catechúmenæ, et 
 Potamiœnæ cum ipsíus matre Marcélla; inter quos præcípue enítuit Potamiœna 
 Virgo, quæ, primo imménsos innumerósque agónes pro virginitáte desudávit, 
 deínde étiam exquisíta et inaudíta torménta pro fide sustínuit, ad últimum, 
 simul cum matre, igne consúmpta est.
\switchcolumn
\selectlanguage{english}
At Alexandria, in the persecution of 
 Severus, the holy martyrs Plutarch, Serenus, Heraclides, catechumen, Heron, 
 a neophyte, another Serenus, Rhais, a catechumen, Potamioena and Marcella 
 her mother. Among them the virgin Potamioena is particularly 
 distinguished. She first endured many painful trials for the 
 preservation of her virginity, and then cruel and unheard-of torments for 
 the faith, after which both she and her mother were consumed with fire.
\switchcolumn*
\selectlanguage{latin}
Eódem die sancti Pápii 
 Mártyris, qui, in persecutióne Diocletiáni Imperatóris, flagris cæsus, atque 
 in lebétem, óleo et ádipe fervénti plenum, immíssus, áliaque horrénda 
 supplícia perpéssus, demum, datis cervícibus, coronátur.
\switchcolumn
\selectlanguage{english}
Also during the persecution of 
 Diocletian, St. Papius, martyr, who was scourged with knotted cords, cast 
 into a cauldron of seething oil and grease, and after other horrible 
 torments was beheaded, and thus won an eternal crown.
\switchcolumn*
\selectlanguage{latin}
Córdubæ, in Hispánia, 
 sancti Argymíri, Mónachi et Mártyris, qui in persecutióne Arábica, pro 
 Christi fide, in equúleo pósitus et ense transfóssus est.
\switchcolumn
\selectlanguage{english}
At Cordova in Spain, St. Argymirus, 
 monk and martyr, who was slain for the faith of Christ during the 
 persecution of the Arabs.
\switchcolumn*
\selectlanguage{latin}
Romæ sancti Pauli Primi, 
 Papæ et Confessóris.
\switchcolumn
\selectlanguage{english}
At Rome, Pope St. Paul I, confessor.
\switchcolumn*
\selectlanguage{latin}
Lúere, in diœcési 
 Brixiénsi, sanctæ Vincéntiæ Gerósa Vírginis, Institúti Sorórum a Caritáte 
 una cum sancta Bartholomǽa Capitánio Fundatrícis, quam Pius Papa Duodécimus 
 albo sanctárum Vírginum accénsuit.
\switchcolumn
\selectlanguage{english}
At Lovere, in the diocese of 
 Brescia, St. Vincenza Gerosa, virgin, who co-founded the Institute of the 
 Sisters of Charity with St. Bartolomea Capitanio, and whom Pope Pius XII 
 added to the list of holy virgins.
\switchcolumn*
\selectlanguage{latin}
\end{paracol}


% ---- martyrology/mart06/mart0629.htm
\needspace{10\baselineskip}
\begin{paracol}{2}
\selectlanguage{latin}
\begin{center}{\color{gregoriocolor} Tértio Kaléndas Júlii. 
 Luna\dots\ }\end{center}
\switchcolumn
\selectlanguage{english}
\begin{center}{\color{gregoriocolor} The 
 Twenty-Ninth Day of 
 June. The\dots\ Day of the Moon.}\end{center}
\end{paracol}

\noindent\begin{tabularx}{\linewidth}{*{19}{>{\centering\arraybackslash}X}}
 \textcolor{gregoriocolor}{a} & \textcolor{gregoriocolor}{b} & \textcolor{gregoriocolor}{c} & \textcolor{gregoriocolor}{d} & \textcolor{gregoriocolor}{e} & \textcolor{gregoriocolor}{f} & \textcolor{gregoriocolor}{g} & \textcolor{gregoriocolor}{h} & \textcolor{gregoriocolor}{i} & \textcolor{gregoriocolor}{k} & \textcolor{gregoriocolor}{l} & \textcolor{gregoriocolor}{m} & \textcolor{gregoriocolor}{n} & \textcolor{gregoriocolor}{p} & \textcolor{gregoriocolor}{q} & \textcolor{gregoriocolor}{r} & \textcolor{gregoriocolor}{s} & \textcolor{gregoriocolor}{t} & \textcolor{gregoriocolor}{u} \\
 4 & 5 & 6 & 7 & 8 & 9 & 10 & 11 & 12 & 13 & 14 & 15 & 16 & 17 & 18 & 19 & 20 & 21 & 22 \\
\end{tabularx}
\vspace{0.5\baselineskip}
\noindent\begin{tabularx}{\linewidth}{*{12}{>{\centering\arraybackslash}X}}
 \textcolor{gregoriocolor}{A} & \textcolor{gregoriocolor}{B} & \textcolor{gregoriocolor}{C} & \textcolor{gregoriocolor}{D} & \textcolor{gregoriocolor}{E} & F & \textcolor{gregoriocolor}{F} & \textcolor{gregoriocolor}{G} & \textcolor{gregoriocolor}{H} & \textcolor{gregoriocolor}{M} & \textcolor{gregoriocolor}{N} & \textcolor{gregoriocolor}{P} \\
 23 & 24 & 25 & 26 & 27 & 28 & 27 & 28 & 29 & 1 & 2 & 3 \\
\end{tabularx}

\begin{paracol}{2}
\selectlanguage{latin}
\lettrine[lines=2]{R}{omæ} natális sanctórum 
 Apostolórum Petri et Pauli, qui eódem anno eodémque die passi sunt, sub 
 Neróne Imperatóre. Horum prior, in eádem Urbe, cápite ad terram verso 
 cruci affíxus, et in Vaticáno juxta viam Triumphálem sepúltus, totíus Orbis 
 veneratióne celebrátur; postérior autem, gládio animadvérsus, et via 
 Ostiénsi sepúltus, pari honóre habétur.
\switchcolumn
\selectlanguage{english}
\lettrine[lines=2]{A}{t} Rome, the birthday of the holy 
 apostles Peter and Paul, who suffered martyrdom on the same day, under 
 Emperor Nero. Within the city the former was crucified with his head 
 downwards, and buried in the Vatican, near the Triumphal Way, where he is 
 venerated by the whole world. The latter was put to the sword and 
 buried on the Ostian Way, where he received similar honours.
\switchcolumn*
\selectlanguage{latin}
In Cypro sanctæ Maríæ, 
 matris Joánnis qui cognominátus est Marcus.
\switchcolumn
\selectlanguage{english}
In Cyprus, St. Mary, mother of John, 
 surnamed Mark.
\switchcolumn*
\selectlanguage{latin}
In castro Argentómacho, 
 in Gállia, sancti Marcélli Mártyris, qui, pro fide Christi, una cum 
 Anastásio, viro militári, cápite plexus est.
\switchcolumn
\selectlanguage{english}
At Argenton in France, St. 
 Marcellus, martyr, who was beheaded for the faith of Christ together with 
 the soldier Anastasius.
\switchcolumn*
\selectlanguage{latin}
Génuæ natális sancti 
 Syri Epíscopi.
\switchcolumn
\selectlanguage{english}
At Genoa, the birthday of St. Syrius, 
 bishop.
\switchcolumn*
\selectlanguage{latin}
Nárniæ sancti Cássii, 
 ejúsdem cívitiátis Epíscopi, de quo sanctus Gregórius Papa refert quod 
 nullus ferme dies vitæ ejus abscedébat, quo placatiónis hóstias omnipoténti 
 Deo non offérret; cui et concordábat vita, quia, cuncta quæ habébat, in 
 eleemósynis tríbuens, in hora sacrifícii totus in lácrimis diffluébat. 
 Demum, natalítio Apostolórum die, quo síngulis annis Romam veníre 
 consuéverat, in eádem Narniénsi urbe, cum Missárum solémnia celebrásset, et 
 corpus Domínicum pacémque ómnibus dedísset, migrávit ad Dóminum.
\switchcolumn
\selectlanguage{english}
At Narni, St. Cassius, bishop of 
 that city. St. Gregory relates that he permitted scarcely any day of 
 his life to pass without offering the Victim of propitiation to Almighty 
 God. It was in character with his life for he distributed in alms all 
 he possessed, and his devotion was such that abundant tears flowed from his 
 eyes during the holy Sacrifice. At last, coming to Rome on the 
 birthday of the apostles, as was his yearly custom, after having solemnly 
 celebrated Mass and given the Lord's Body and the kiss of peace to all, he 
 departed for heaven.
\switchcolumn*
\selectlanguage{latin}
In território Senonénsi 
 sanctæ Benedíctæ Vírginis.
\switchcolumn
\selectlanguage{english}
In the territory of Sens, St. 
 Benedicta, virgin.
\switchcolumn*
\selectlanguage{latin}
\end{paracol}


% ---- martyrology/mart06/mart0630.htm
\needspace{10\baselineskip}
\begin{paracol}{2}
\selectlanguage{latin}
\begin{center}{\color{gregoriocolor} Prídie Kaléndas Júlii. 
 Luna\dots\ }\end{center}
\switchcolumn
\selectlanguage{english}
\begin{center}{\color{gregoriocolor} The 
 Thirtieth Day of 
 June. The\dots\ Day of the Moon.}\end{center}
\end{paracol}

\noindent\begin{tabularx}{\linewidth}{*{19}{>{\centering\arraybackslash}X}}
 \textcolor{gregoriocolor}{a} & \textcolor{gregoriocolor}{b} & \textcolor{gregoriocolor}{c} & \textcolor{gregoriocolor}{d} & \textcolor{gregoriocolor}{e} & \textcolor{gregoriocolor}{f} & \textcolor{gregoriocolor}{g} & \textcolor{gregoriocolor}{h} & \textcolor{gregoriocolor}{i} & \textcolor{gregoriocolor}{k} & \textcolor{gregoriocolor}{l} & \textcolor{gregoriocolor}{m} & \textcolor{gregoriocolor}{n} & \textcolor{gregoriocolor}{p} & \textcolor{gregoriocolor}{q} & \textcolor{gregoriocolor}{r} & \textcolor{gregoriocolor}{s} & \textcolor{gregoriocolor}{t} & \textcolor{gregoriocolor}{u} \\
 5 & 6 & 7 & 8 & 9 & 10 & 11 & 12 & 13 & 14 & 15 & 16 & 17 & 18 & 19 & 20 & 21 & 22 & 23 \\
\end{tabularx}
\vspace{0.5\baselineskip}
\noindent\begin{tabularx}{\linewidth}{*{12}{>{\centering\arraybackslash}X}}
 \textcolor{gregoriocolor}{A} & \textcolor{gregoriocolor}{B} & \textcolor{gregoriocolor}{C} & \textcolor{gregoriocolor}{D} & \textcolor{gregoriocolor}{E} & F & \textcolor{gregoriocolor}{F} & \textcolor{gregoriocolor}{G} & \textcolor{gregoriocolor}{H} & \textcolor{gregoriocolor}{M} & \textcolor{gregoriocolor}{N} & \textcolor{gregoriocolor}{P} \\
 24 & 25 & 26 & 27 & 28 & 29 & 28 & 29 & 1 & 2 & 3 & 4 \\
\end{tabularx}

\begin{paracol}{2}
\selectlanguage{latin}
\lettrine[lines=2]{C}{ommemorátio} sancti 
 Pauli Apóstoli.
\switchcolumn
\selectlanguage{english}
\lettrine[lines=2]{T}{he} commemoration of the holy 
 apostle Paul.
\switchcolumn*
\selectlanguage{latin}
Romæ sanctæ Lucínæ, 
 Apostolórum discípulæ, quæ, de facultátibus suis commúnicans Sanctórum 
 necessitátibus, Christiános in cárcere deténtos visitábat, ac sepultúræ 
 Mártyrum inserviébat; juxta quos et ipsa, in crypta a se constrúcta, sepúlta 
 est.
\switchcolumn
\selectlanguage{english}
At Rome, St. Lucina, a disciple of 
 the apostles, who relieved the necessities of the saints with her goods, 
 visited the Christians detained in prison, buried the martyrs, and was laid 
 by their side in a crypt which she herself had constructed.
\switchcolumn*
\selectlanguage{latin}
Item Romæ sanctæ 
 Æmiliánæ Mártyris.
\switchcolumn
\selectlanguage{english}
In the same city, St. Aemiliana, 
 martyr.
\switchcolumn*
\selectlanguage{latin}
Ipso die sanctórum 
 Mártyrum Caji Presbyteri, et Leónis Subdiáconi.
\switchcolumn
\selectlanguage{english}
The same day, the saints Caius, 
 priest, and Leo, subdeacon.
\switchcolumn*
\selectlanguage{latin}
Alexandríæ pássio 
 sancti Basílidis, qui, sub Sevéro Imperatóre, cum sanctam Potamiœnam 
 Vírginem, quam ad supplícium ducébat, ab impudicórum hóminum petulántia 
 tutátus esset, religiósi offícii mercédem ab ea recépit; nam ipsa, post 
 tríduum illi appárens et ejus cápiti corónam impónens, non tantum convértit 
 eum ad Christum, sed étiam, brevi agóne certántem, suis précibus Mártyrem 
 gloriósum effécit.
\switchcolumn
\selectlanguage{english}
At Alexandria, the passion of St. 
 Basilides, under Emperor Severus. He protected the saintly virgin 
 Potamioena from the insults of shameless men when he was leading her to 
 execution. He was rewarded for his considerate action, for at the end 
 of three days she appeared to him, placed a crown on his head, not only 
 converting him to Christ, but by her prayers making him, after a short 
 combat, a glorious martyr.
\switchcolumn*
\selectlanguage{latin}
Lemóvicis, in Aquitánia, 
 sancti Martiális Epíscopi, cum duóbus Presbyteris Alpiniáno et 
 Austricliniáno; quorum vita signis miraculórum ádmodum effúlsit.
\switchcolumn
\selectlanguage{english}
At Limoges in France, St. Martial, 
 bishop, and two priests Alpinian and Austriclinian, whose lives were 
 distinguished for miracles.
\switchcolumn*
\selectlanguage{latin}
In território 
 Vivariénsi, in Gálliis, sancti Ostiáni, Presbyteri et Confessóris.
\switchcolumn
\selectlanguage{english}
In the territory of Vivers, St. 
 Ostian, priest and confessor.
\switchcolumn*
\selectlanguage{latin}
Salánicæ, in território 
 Vicentíno, sancti Theobáldi, Presbyteri et Eremítæ, ex Campániæ Gállicæ 
 Comítibus; quem Alexánder Papa Tértius, ob sanctitátis et miraculórum famam, 
 Sanctórum cánoni adjúnxit.
\switchcolumn
\selectlanguage{english}
At Salanica, in the district of 
 Vicenza, St. Theobald, priest and hermit, one of the counts of Champagne. 
 He was added to the number of the saints by Alexander III because of his 
 holiness and miracles.
\switchcolumn*
\selectlanguage{latin}
\end{paracol}

\setrunningtitles{Julius}{July}

% ---- martyrology/mart07/mart0701.htm
\needspace{10\baselineskip}
\begin{paracol}{2}
\selectlanguage{latin}
\begin{center}{\color{gregoriocolor} Kaléndis Júlii. 
 Luna\dots\ }\end{center}
\switchcolumn
\selectlanguage{english}
\begin{center}{\color{gregoriocolor} The 
 First Day of 
 July. The\dots\ Day of the Moon.}\end{center}
\end{paracol}

\noindent\begin{tabularx}{\linewidth}{*{19}{>{\centering\arraybackslash}X}}
 \textcolor{gregoriocolor}{a} & \textcolor{gregoriocolor}{b} & \textcolor{gregoriocolor}{c} & \textcolor{gregoriocolor}{d} & \textcolor{gregoriocolor}{e} & \textcolor{gregoriocolor}{f} & \textcolor{gregoriocolor}{g} & \textcolor{gregoriocolor}{h} & \textcolor{gregoriocolor}{i} & \textcolor{gregoriocolor}{k} & \textcolor{gregoriocolor}{l} & \textcolor{gregoriocolor}{m} & \textcolor{gregoriocolor}{n} & \textcolor{gregoriocolor}{p} & \textcolor{gregoriocolor}{q} & \textcolor{gregoriocolor}{r} & \textcolor{gregoriocolor}{s} & \textcolor{gregoriocolor}{t} & \textcolor{gregoriocolor}{u} \\
 6 & 7 & 8 & 9 & 10 & 11 & 12 & 13 & 14 & 15 & 16 & 17 & 18 & 19 & 20 & 21 & 22 & 23 & 24 \\
\end{tabularx}
\vspace{0.5\baselineskip}
\noindent\begin{tabularx}{\linewidth}{*{12}{>{\centering\arraybackslash}X}}
 \textcolor{gregoriocolor}{A} & \textcolor{gregoriocolor}{B} & \textcolor{gregoriocolor}{C} & \textcolor{gregoriocolor}{D} & \textcolor{gregoriocolor}{E} & F & \textcolor{gregoriocolor}{F} & \textcolor{gregoriocolor}{G} & \textcolor{gregoriocolor}{H} & \textcolor{gregoriocolor}{M} & \textcolor{gregoriocolor}{N} & \textcolor{gregoriocolor}{P} \\
 25 & 26 & 27 & 28 & 29 & 30 & 29 & 1 & 2 & 3 & 4 & 5 \\
\end{tabularx}

\begin{paracol}{2}
\selectlanguage{latin}
\lettrine[lines=1]{O}{ctáva} Nativitátis sancti Joánnis Baptístæ.
\switchcolumn
\selectlanguage{english}
\lettrine[lines=1]{T}{he} Octave of the Nativity of St. John the Baptist.
\switchcolumn*
\selectlanguage{latin}
Festum pretiosíssimi 
 Sánguinis Dómini nostri Jesu Christi.
\switchcolumn
\selectlanguage{english}
The feast of the most Precious Blood 
 of our Lord Jesus Christ.
\switchcolumn*
\selectlanguage{latin}
In monte Hor deposítio 
 sancti Aaron, primi ex órdine Levítico Sacerdótis.
\switchcolumn
\selectlanguage{english}
On Mt. Hor, the death of St. Aaron, 
 the first priest of the Levitical order.
\switchcolumn*
\selectlanguage{latin}
Viénnæ, in Gállia, 
 sancti Martíni Epíscopi, Apostolórum discípuli.
\switchcolumn
\selectlanguage{english}
At Vienne in France, St. Martin, a 
 bishop who was a disciple of the apostles.
\switchcolumn*
\selectlanguage{latin}
Sinuéssæ, in Campánia, 
 sanctórum Mártyrum Casti et Secundíni Episcopórum.
\switchcolumn
\selectlanguage{english}
At Sinuessa in Campánia the holy 
 martyrs Castus and Secundinus, bishops.
\switchcolumn*
\selectlanguage{latin}
In Británnia sanctórum 
 Mártyrum Júlii et Aaron, qui post sanctum Albánum, in persecutióne 
 Diocletiáni Imperatóris, passi sunt; quo témpore ibídem quamplúrimi, 
 divérsis cruciátibus torti et sævíssime laceráti, ad supérnæ civitátis 
 gáudia, consummáto agóne, pervenérunt.
\switchcolumn
\selectlanguage{english}
In England, the holy martyrs Julius 
 and Aaron, who suffered after St. Alban in the persecution of Diocletian. 
 In the same country a great number were tortured at that time in different 
 ways and barbarously lacerated, ended their combat, and attained to the joys 
 of the heavenly city.
\switchcolumn*
\selectlanguage{latin}
Arvérnis, in Gállia, 
 sancti Galli Epíscopi.
\switchcolumn
\selectlanguage{english}
In Auvergne in France, St. Gall, 
 bishop.
\switchcolumn*
\selectlanguage{latin}
In território 
 Lugdunénsi deposítio sancti Domitiáni Abbátis, qui primus illic eremitícam 
 vitam exércuit, et, cum plúrimos ibi in Dei servítio congregásset, tandem, 
 magnis virtútibus et gloriósis miráculis valde clarus, ad patres, in 
 senectúte bona, colléctus est.
\switchcolumn
\selectlanguage{english}
In the diocese of Lyons, the death 
 of St. Domitian, abbot, who was first to lead the life of a monk in that 
 district. After having called together many servants of God to that 
 place, and having gained great renown for virtues and miracles, he was 
 summoned to his fathers at an advanced age.
\switchcolumn*
\selectlanguage{latin}
Engolísmæ, in Gállia, 
 sancti Epárchii Abbátis.
\switchcolumn
\selectlanguage{english}
At Angoulême, St. Eparchius, abbot.
\switchcolumn*
\selectlanguage{latin}
In território Rheménsi 
 sancti Theodoríci Presbyteri, qui fuit discípulus beáti Remígii Epíscopi.
\switchcolumn
\selectlanguage{english}
In the diocese of Rheims, St. 
 Theodoric, priest and disciple of the blessed Bishop Remigius.
\switchcolumn*
\selectlanguage{latin}
Apud Eméssam, in 
 Phœnícia, sancti Simeónis Confessóris, cognoménto Sali, qui stultus propter 
 Christum factus est; sed altam ejus sapiéntiam Deus magnis miráculis 
 declarávit.
\switchcolumn
\selectlanguage{english}
At Emesa, St. Simeon, surnamed Salus, 
 confessor. He had feigned to be an idiot for the sake of Christ, but 
 God manifested his high wisdom by great miracles.
\switchcolumn*
\selectlanguage{latin}
\end{paracol}


% ---- martyrology/mart07/mart0702.htm
\needspace{10\baselineskip}
\begin{paracol}{2}
\selectlanguage{latin}
\begin{center}{\color{gregoriocolor} Sexto Nonas Júlii. 
 Luna\dots\ }\end{center}
\switchcolumn
\selectlanguage{english}
\begin{center}{\color{gregoriocolor} The Second Day of July. The\dots\ Day of the 
 Moon.}\end{center}
\end{paracol}

\noindent\begin{tabularx}{\linewidth}{*{19}{>{\centering\arraybackslash}X}}
 \textcolor{gregoriocolor}{a} & \textcolor{gregoriocolor}{b} & \textcolor{gregoriocolor}{c} & \textcolor{gregoriocolor}{d} & \textcolor{gregoriocolor}{e} & \textcolor{gregoriocolor}{f} & \textcolor{gregoriocolor}{g} & \textcolor{gregoriocolor}{h} & \textcolor{gregoriocolor}{i} & \textcolor{gregoriocolor}{k} & \textcolor{gregoriocolor}{l} & \textcolor{gregoriocolor}{m} & \textcolor{gregoriocolor}{n} & \textcolor{gregoriocolor}{p} & \textcolor{gregoriocolor}{q} & \textcolor{gregoriocolor}{r} & \textcolor{gregoriocolor}{s} & \textcolor{gregoriocolor}{t} & \textcolor{gregoriocolor}{u} \\
 7 & 8 & 9 & 10 & 11 & 12 & 13 & 14 & 15 & 16 & 17 & 18 & 19 & 20 & 21 & 22 & 23 & 24 & 25 \\
\end{tabularx}
\vspace{0.5\baselineskip}
\noindent\begin{tabularx}{\linewidth}{*{12}{>{\centering\arraybackslash}X}}
 \textcolor{gregoriocolor}{A} & \textcolor{gregoriocolor}{B} & \textcolor{gregoriocolor}{C} & \textcolor{gregoriocolor}{D} & \textcolor{gregoriocolor}{E} & F & \textcolor{gregoriocolor}{F} & \textcolor{gregoriocolor}{G} & \textcolor{gregoriocolor}{H} & \textcolor{gregoriocolor}{M} & \textcolor{gregoriocolor}{N} & \textcolor{gregoriocolor}{P} \\
 26 & 27 & 28 & 29 & 30 & 1 & 1 & 2 & 3 & 4 & 5 & 6 \\
\end{tabularx}

\begin{paracol}{2}
\selectlanguage{latin}
\lettrine[lines=2]{V}{isitátio} beátæ Maríæ 
 Vírginis ad Elísabeth.
\switchcolumn
\selectlanguage{english}
\lettrine[lines=2]{T}{he} Visitation of the Blessed Virgin 
 Mary to Elizabeth.
\switchcolumn*
\selectlanguage{latin}
Romæ, via Aurélia, natális sanctórum Mártyrum Procéssi et Martiniáni, qui, a beáto Petro 
 Apóstolo in Mamertíni custódia baptizáti, et, sub Neróne, oris contusiónem, 
 equúleum, nervos, fustes, flammas scorpionésque perpéssi, novíssime, gládio 
 percússi, martyrio coronáti sunt.
\switchcolumn
\selectlanguage{english}
At Rome, on the Aurelian Way, the 
 birthday of the holy martyrs Processus and Martinian, who were baptized by 
 the blessed apostle Peter in the Mamertine Prison. After being struck 
 on the mouth, racked, scourged with knotted ropes and whips strung with 
 pieces of metal; after being beaten with rods and exposed to the flames, 
 they were beheaded in the days of Nero, thus obtaining the crown of 
 martyrdom.
\switchcolumn*
\selectlanguage{latin}
Item Romæ pássio 
 sanctórum trium mílitum, qui, ad Christum convérsi in martyrio beáti Pauli 
 Apóstoli, cum eo partícipes fíeri cæléstis glóriæ meruérunt.
\switchcolumn
\selectlanguage{english}
Also at Rome, three holy soldiers, 
 who were converted to Christ by the martyrdom of the blessed apostle Paul, 
 and with him merited to be made partakers of the heavenly glory.
\switchcolumn*
\selectlanguage{latin}
Eódem die sanctórum 
 Mártyrum Aristónis, Crescentiáni, Eutychiáni, Urbáni, Vitális, Justi, 
 Felicíssimi, Felícis, Márciæ et Symphorósæ; qui omnes apud Campániam, cum 
 Diocletiáni Imperatóris persecútio desævíret, martyrio coronáti sunt.
\switchcolumn
\selectlanguage{english}
The same day, the holy martyrs 
 Ariston, Crescentian, Eutychian, Urbanus, Vitalis, Justus, Felicissimus, 
 Felix, Marcia, and Symphorosa, all of whom were crowned with martyrdom when 
 the persecution of Emperor Diocletian was raging.
\switchcolumn*
\selectlanguage{latin}
Wintóniæ, in Anglia, 
 sancti Swithúni Epíscopi, cujus sánctitas miráculis effúlsit.
\switchcolumn
\selectlanguage{english}
At Winchester in England, St. 
 Swithin, bishop, whose sanctity was verified by the gift of miracles.
\switchcolumn*
\selectlanguage{latin}
Bambérgæ sancti Othónis 
 Epíscopi, qui Pomeránis Evangélium prædicávit, et eos ad fidem convértit.
\switchcolumn
\selectlanguage{english}
At Bamberg, the holy bishop Otho, 
 who preached the Gospel to the people of Pomerania, and converted them to 
 the faith.
\switchcolumn*
\selectlanguage{latin}
Lyciis, in Apúlia, sancti Bernardíni Realíno, Confessóris, qui magistrátus múnere egrégie 
 functus, Societátem Jesu ingréssus et sacerdótio auctus, caritáte ac 
 miráculis incláruit.
\switchcolumn
\selectlanguage{english}
At Lecce in Apulia, St. Bernardino 
 Realino, confessor, who after practising the legal profession as a judge, 
 entered the Society of Jesus, was ordained to the priesthood, and was 
 renowned for his charity and miracles.
\switchcolumn*
\selectlanguage{latin}
Turónis, in Gállia, 
 deposítio sanctæ Monegúndis, religiósæ féminæ.
\switchcolumn
\selectlanguage{english}
At Tours, the death of St. 
 Monegundes, a pious woman.
\switchcolumn*
\selectlanguage{latin}
\end{paracol}


% ---- martyrology/mart07/mart0703.htm
\needspace{10\baselineskip}
\begin{paracol}{2}
\selectlanguage{latin}
\begin{center}{\color{gregoriocolor} Quinto Nonas Júlii. 
 Luna\dots\ }\end{center}
\switchcolumn
\selectlanguage{english}
\begin{center}{\color{gregoriocolor} The 
 Third Day of 
 July. The\dots\ Day of the Moon.}\end{center}
\end{paracol}

\noindent\begin{tabularx}{\linewidth}{*{19}{>{\centering\arraybackslash}X}}
 \textcolor{gregoriocolor}{a} & \textcolor{gregoriocolor}{b} & \textcolor{gregoriocolor}{c} & \textcolor{gregoriocolor}{d} & \textcolor{gregoriocolor}{e} & \textcolor{gregoriocolor}{f} & \textcolor{gregoriocolor}{g} & \textcolor{gregoriocolor}{h} & \textcolor{gregoriocolor}{i} & \textcolor{gregoriocolor}{k} & \textcolor{gregoriocolor}{l} & \textcolor{gregoriocolor}{m} & \textcolor{gregoriocolor}{n} & \textcolor{gregoriocolor}{p} & \textcolor{gregoriocolor}{q} & \textcolor{gregoriocolor}{r} & \textcolor{gregoriocolor}{s} & \textcolor{gregoriocolor}{t} & \textcolor{gregoriocolor}{u} \\
 8 & 9 & 10 & 11 & 12 & 13 & 14 & 15 & 16 & 17 & 18 & 19 & 20 & 21 & 22 & 23 & 24 & 25 & 26 \\
\end{tabularx}
\vspace{0.5\baselineskip}
\noindent\begin{tabularx}{\linewidth}{*{12}{>{\centering\arraybackslash}X}}
 \textcolor{gregoriocolor}{A} & \textcolor{gregoriocolor}{B} & \textcolor{gregoriocolor}{C} & \textcolor{gregoriocolor}{D} & \textcolor{gregoriocolor}{E} & F & \textcolor{gregoriocolor}{F} & \textcolor{gregoriocolor}{G} & \textcolor{gregoriocolor}{H} & \textcolor{gregoriocolor}{M} & \textcolor{gregoriocolor}{N} & \textcolor{gregoriocolor}{P} \\
 27 & 28 & 29 & 30 & 1 & 2 & 2 & 3 & 4 & 5 & 6 & 7 \\
\end{tabularx}

\begin{paracol}{2}
\selectlanguage{latin}
\lettrine[lines=2]{R}{omæ} natális sancti 
 Leónis Secúndi, Papæ et Confessóris, qui primo sui Pontificátus anno, plenus 
 méritis, migrávit in cælum.
\switchcolumn
\selectlanguage{english}
\lettrine[lines=2]{A}{t} Rome, the birthday of Pope St. 
 Leo II, confessor, who passed to heaven filled with merits during the first 
 year of his pontificate.
\switchcolumn*
\selectlanguage{latin}
Clúsii, in Etrúria, 
 sanctórum Mártyrum Irenæi Diáconi, et Mustíolæ matrónæ; qui sub Aureliáno 
 Imperatóre, divérsis atrocibúsque supplíciis cruciáti, corónam martyrii 
 meruérunt.
\switchcolumn
\selectlanguage{english}
At Chiusi in Tuscany, in the reign 
 of Emperor Aurelian, the holy martyrs Irenaeus, a deacon, and Mustiola, a 
 matron, who were subjected to various atrocious tortures and merited the 
 crown of martyrdom.
\switchcolumn*
\selectlanguage{latin}
Alexandríæ sanctórum 
 Mártyrum Tryphónis et aliórum duódecim.
\switchcolumn
\selectlanguage{english}
At Alexandria, the holy martyrs 
 Trypho and twelve others.
\switchcolumn*
\selectlanguage{latin}
Constantinópoli 
 sanctórum Eulógii et Sociórum Mártyrum.
\switchcolumn
\selectlanguage{english}
At Constantinople, the holy martyrs 
 Eulogius and his companions.
\switchcolumn*
\selectlanguage{latin}
Cæsaréæ, in Cappadócia, 
 sancti Hyacínthi, qui Trajáni Imperatóris cubiculárius fuit; et, accusátus 
 quod Christiánus esset, plagis várie afflíctus est, atque, in cárcerem 
 trusus, ibi fame consúmptus exspirávit.
\switchcolumn
\selectlanguage{english}
At Caesarea in Cappadocia, St. 
 Hyacinth, chamberlain of the emperor Trajan. Accused of being a 
 Christian, he was scourged and thrown into prison where he died of hunger.
\switchcolumn*
\selectlanguage{latin}
Eódem die sanctórum 
 Mártyrum Marci et Muciáni, qui pro Christo sunt gládio cæsi. Hos cum 
 puer párvulus, ne immolárent idólis, alta voce monéret, verbéribus jussus 
 est cædi; cumque veheméntius Christum confiterétur, tandem, una cum quodam 
 Paulo, Mártyres exhortánte, necátus est.
\switchcolumn
\selectlanguage{english}
The same day, the holy martyrs Mark 
 and Mucian, who were put to the sword for Christ. A small boy who 
 cried out to them not to sacrifice to idols was then whipped, but confessing 
 Christ still more vehemently, he was put to death with a man named Paul, who 
 had also exhorted the martyrs.
\switchcolumn*
\selectlanguage{latin}
Laodicéæ, in Syria, 
 sancti Anatólii Epíscopi, qui non solum religiósis viris, sed et philósophis 
 admiránda scripta relíquit.
\switchcolumn
\selectlanguage{english}
At Laodicea in Syria, St. Anatolius, 
 a bishop whose writings were admired not only by religious men, but by 
 philosophers.
\switchcolumn*
\selectlanguage{latin}
Altíni, in Venetórum 
 fínibus, sancti Heliodóri Epíscopi, doctrína et sanctitáte insígnis.
\switchcolumn
\selectlanguage{english}
At Altino, St. Heliodorus, a bishop 
 distinguished for holiness and learning.
\switchcolumn*
\selectlanguage{latin}
Ravénnæ sancti Dathi, 
 Epíscopi et Confessóris.
\switchcolumn
\selectlanguage{english}
At Ravenna, St. Dathus, bishop and 
 confessor.
\switchcolumn*
\selectlanguage{latin}
Edéssæ, in Mesopotámia, Translátio sancti Thomæ Apóstoli ex India; cujus relíquiæ Ortónam, apud 
 Frentános, póstea translátæ sunt.
\switchcolumn
\selectlanguage{english}
At Edessa in Mesopotamia, the 
 translation of the apostle St. Thomas from India. His relics were 
 afterwards taken to Ortona.
\switchcolumn*
\selectlanguage{latin}
\end{paracol}


% ---- martyrology/mart07/mart0704.htm
\needspace{10\baselineskip}
\begin{paracol}{2}
\selectlanguage{latin}
\begin{center}{\color{gregoriocolor} Quarto Nonas Júlii. 
 Luna\dots\ }\end{center}
\switchcolumn
\selectlanguage{english}
\begin{center}{\color{gregoriocolor} The 
 Fourth Day of 
 July. The\dots\ Day of the Moon.}\end{center}
\end{paracol}

\noindent\begin{tabularx}{\linewidth}{*{19}{>{\centering\arraybackslash}X}}
 \textcolor{gregoriocolor}{a} & \textcolor{gregoriocolor}{b} & \textcolor{gregoriocolor}{c} & \textcolor{gregoriocolor}{d} & \textcolor{gregoriocolor}{e} & \textcolor{gregoriocolor}{f} & \textcolor{gregoriocolor}{g} & \textcolor{gregoriocolor}{h} & \textcolor{gregoriocolor}{i} & \textcolor{gregoriocolor}{k} & \textcolor{gregoriocolor}{l} & \textcolor{gregoriocolor}{m} & \textcolor{gregoriocolor}{n} & \textcolor{gregoriocolor}{p} & \textcolor{gregoriocolor}{q} & \textcolor{gregoriocolor}{r} & \textcolor{gregoriocolor}{s} & \textcolor{gregoriocolor}{t} & \textcolor{gregoriocolor}{u} \\
 9 & 10 & 11 & 12 & 13 & 14 & 15 & 16 & 17 & 18 & 19 & 20 & 21 & 22 & 23 & 24 & 25 & 26 & 27 \\
\end{tabularx}
\vspace{0.5\baselineskip}
\noindent\begin{tabularx}{\linewidth}{*{12}{>{\centering\arraybackslash}X}}
 \textcolor{gregoriocolor}{A} & \textcolor{gregoriocolor}{B} & \textcolor{gregoriocolor}{C} & \textcolor{gregoriocolor}{D} & \textcolor{gregoriocolor}{E} & F & \textcolor{gregoriocolor}{F} & \textcolor{gregoriocolor}{G} & \textcolor{gregoriocolor}{H} & \textcolor{gregoriocolor}{M} & \textcolor{gregoriocolor}{N} & \textcolor{gregoriocolor}{P} \\
 28 & 29 & 30 & 1 & 2 & 3 & 3 & 4 & 5 & 6 & 7 & 8 \\
\end{tabularx}

\begin{paracol}{2}
\selectlanguage{latin}
\lettrine[lines=2]{S}{tremótii,} in Lusitánia, 
 natális sanctæ Elísabeth Víduæ, Lusitanórum Regínæ, quam, virtútibus et 
 miráculis claram, Urbánus Octávus, Póntifex Máximus, in Sanctórum númerum 
 rétulit. Ejus tamen celébritas octávo Idus mensis hujus recólitur, ex 
 dispositióne Innocéntii Papæ Duodécimi.
\switchcolumn
\selectlanguage{english}
\lettrine[lines=2]{A}{t} Estremos in Portugal, the 
 birthday of St. Elizabeth the Widow, queen of Portugal, whom Pope Urban 
 VIII, mindful of her virtues and miracles, placed among the number of the 
 saints. Pope Innocent XII ordered her feast to be kept on the 8th of 
 July.
\switchcolumn*
\selectlanguage{latin}
Sanctórum Osée et Aggǽi 
 Prophetárum.
\switchcolumn
\selectlanguage{english}
The holy prophets Osee and Aggaeus.
\switchcolumn*
\selectlanguage{latin}
In território 
 Bituricénsi sancti Lauriáni, Epíscopi Hispalénsis et Mártyris, cujus caput 
 Híspalim, in Hispánia, delátum est.
\switchcolumn
\selectlanguage{english}
In the diocese of Bourges, St. 
 Laurian, bishop of Seville and martyr, whose head was taken to Seville in 
 Spain.
\switchcolumn*
\selectlanguage{latin}
In Africa natális 
 sancti Jucundiáni Mártyris, pro Christo in mare demérsi.
\switchcolumn
\selectlanguage{english}
In Africa, the birthday of St. 
 Jucundian, a martyr who was drowned in the sea for Christ.
\switchcolumn*
\selectlanguage{latin}
Sírmii sanctórum 
 Mártyrum Innocéntii et Sebástiæ, cum áliis trigínta.
\switchcolumn
\selectlanguage{english}
At Sirmium, Saints Innocent and 
 Sebastia, with thirty other martyrs.
\switchcolumn*
\selectlanguage{latin}
Madáuri, in Africa, 
 sancti Namphaniónis Mártyris et Sociórum, quos ille roborávit ad pugnam, et 
 ad corónam provéxit.
\switchcolumn
\selectlanguage{english}
At Madaurus in Africa, the martyr 
 Namphanion and his companions, whom he strengthened for the combat and led 
 to the crown of martyrdom.
\switchcolumn*
\selectlanguage{latin}
Cyréne, in Líbya, sancti Theodóri Epíscopi, qui, in persecutióne Diocletiáni, sub Digniáno 
 Prǽside, plumbátis cæsus et lingua abscíssus est; atque in pace tandem 
 Conféssor occúbuit.
\switchcolumn
\selectlanguage{english}
At Cyrene in Libya, the holy bishop 
 Theodore. In the persecution of Diocletian, under the governor Dignian, 
 he was scourged with leaded whips and had his tongue cut out. Finally, 
 however, he died a confessor.
\switchcolumn*
\selectlanguage{latin}
Augústæ, in Rhǽtia, 
 sancti Uldaríci Epíscopi, miræ abstinéntiæ, largitátis et vigilántiæ virtúte, 
 ac miraculórum dono illústris.
\switchcolumn
\selectlanguage{english}
At Augsburg in Germany, St. Uldaric, 
 a bishop illustrious for extraordinary abstinence, liberality, vigilance, 
 and the gift of miracles.
\switchcolumn*
\selectlanguage{latin}
Turónis, in Gállia, 
 Translátio sancti Martíni, Epíscopi et Confessóris; et Dedicátio Basílicæ 
 suæ hoc ipso die, quo étiam is ante áliquot annos in Epíscopum fúerat 
 ordinátus.
\switchcolumn
\selectlanguage{english}
At Tours in France, the translation 
 of St. Martin, bishop and confessor, and the dedication of his basilica, 
 consecrated on the same day that he had been raised to the episcopate some 
 years previously.
\switchcolumn*
\selectlanguage{latin}
\end{paracol}


% ---- martyrology/mart07/mart0705.htm
\needspace{10\baselineskip}
\begin{paracol}{2}
\selectlanguage{latin}
\begin{center}{\color{gregoriocolor} Tértio Nonas Júlii. 
 Luna\dots\ }\end{center}
\switchcolumn
\selectlanguage{english}
\begin{center}{\color{gregoriocolor} The 
 Fifth Day of 
 July. The\dots\ Day of the Moon.}\end{center}
\end{paracol}

\noindent\begin{tabularx}{\linewidth}{*{19}{>{\centering\arraybackslash}X}}
 \textcolor{gregoriocolor}{a} & \textcolor{gregoriocolor}{b} & \textcolor{gregoriocolor}{c} & \textcolor{gregoriocolor}{d} & \textcolor{gregoriocolor}{e} & \textcolor{gregoriocolor}{f} & \textcolor{gregoriocolor}{g} & \textcolor{gregoriocolor}{h} & \textcolor{gregoriocolor}{i} & \textcolor{gregoriocolor}{k} & \textcolor{gregoriocolor}{l} & \textcolor{gregoriocolor}{m} & \textcolor{gregoriocolor}{n} & \textcolor{gregoriocolor}{p} & \textcolor{gregoriocolor}{q} & \textcolor{gregoriocolor}{r} & \textcolor{gregoriocolor}{s} & \textcolor{gregoriocolor}{t} & \textcolor{gregoriocolor}{u} \\
 10 & 11 & 12 & 13 & 14 & 15 & 16 & 17 & 18 & 19 & 20 & 21 & 22 & 23 & 24 & 25 & 26 & 27 & 28 \\
\end{tabularx}
\vspace{0.5\baselineskip}
\noindent\begin{tabularx}{\linewidth}{*{12}{>{\centering\arraybackslash}X}}
 \textcolor{gregoriocolor}{A} & \textcolor{gregoriocolor}{B} & \textcolor{gregoriocolor}{C} & \textcolor{gregoriocolor}{D} & \textcolor{gregoriocolor}{E} & F & \textcolor{gregoriocolor}{F} & \textcolor{gregoriocolor}{G} & \textcolor{gregoriocolor}{H} & \textcolor{gregoriocolor}{M} & \textcolor{gregoriocolor}{N} & \textcolor{gregoriocolor}{P} \\
 29 & 30 & 1 & 2 & 3 & 4 & 4 & 5 & 6 & 7 & 8 & 9 \\
\end{tabularx}

\begin{paracol}{2}
\selectlanguage{latin}
\lettrine[lines=2]{C}{remónæ,} in Insúbria, 
 sancti Antónii-Maríæ Zaccaría, Confessóris, qui Clericórum Regulárium sancti 
 Pauli et Angelicárum Vírginum fuit Institútor; atque, virtútibus ómnibus et 
 miráculis insígnis, a Leóne Papa Décimo tértio inter Sanctos adscríptus est. 
 Ejus corpus Medioláni, in Ecclésia sancti Bárnabæ cólitur.
\switchcolumn
\selectlanguage{english}
\lettrine[lines=2]{A}{t} Cremona in Italy, St. 
 Anthony-Mary Zacharias, confessor, founder of the Clerks Regular of St. Paul 
 and also of the Angelic Virgins. Distinguished for all the virtues and 
 for miracles, he was placed among the saints by Pope Leo XIII. His 
 body is venerated in the Church of St. Barnabas at Milan.
\switchcolumn*
\selectlanguage{latin}
Romæ sanctæ Zoæ 
 Mártyris, uxóris beáti Nicostráti Mártyris, quæ, sub Diocletiáno Imperatóre, 
 dum ad confessiónem beáti Petri Apóstoli oráret, a persecutóribus est 
 arctáta et in custódiam obscuríssimam trusa; deínde collo et capíllis in 
 árbore, adhíbito subter horríbili fumo, suspénsa est, et ita, in confessióne 
 Dómini, emísit spíritum.
\switchcolumn
\selectlanguage{english}
At Rome, St. Zoe, martyr, wife of 
 the blessed martyr Nicostratus. While praying at the tomb of the 
 apostle St. Peter, during the time of Diocletian, she was seized by the 
 persecutors, cast into a dark dungeon, then hanged on a tree by her neck and 
 hair, and suffocated by a loathsome smoke, finally yielding up her soul in 
 the confession of the Lord.
\switchcolumn*
\selectlanguage{latin}
Hierosólymis sancti 
 Athanásii Diáconi, qui, pro sancta Synodo Chalcedonénsi, ab hæréticis 
 comprehénsus et ómnium cruciátum expértus génera, ferro tandem necátus est.
\switchcolumn
\selectlanguage{english}
At Jerusalem, St. Athanasius, a 
 deacon, who was apprehended by the heretics for defending the Council of 
 Chalcedon, and after experiencing all kinds of torments, was finally put to 
 the sword.
\switchcolumn*
\selectlanguage{latin}
In Syria natális 
 sancti Domítii Mártyris, qui virtútibus suis multa íncolis præstat benefícia.
\switchcolumn
\selectlanguage{english}
In Syria, the birthday of St. 
 Domitius, martyr, who confers many favours on the people of that country by 
 his miracles.
\switchcolumn*
\selectlanguage{latin}
In Sicília sanctórum 
 Mártyrum Agathónis et Triphínæ.
\switchcolumn
\selectlanguage{english}
In Sicily, the holy martyrs Agatho 
 and Triphina.
\switchcolumn*
\selectlanguage{latin}
Tomis, in Scythia, 
 sanctórum Mártyrum Maríni, Theódoti et Sédophæ.
\switchcolumn
\selectlanguage{english}
At Tomis in Scythia, the holy 
 martyrs Marinus, Theodotus, and Sedopha.
\switchcolumn*
\selectlanguage{latin}
Cyréne, in Libya, 
 sanctæ Cyríllæ Mártyris, quæ, in persecutióne Diocletiáni, ardéntes carbónes 
 cum thure super manum pósitos diu ténuit, ne, prunas excutiéndo, thus 
 obtulísse viderétur; deínde, sævíssime laniáta, próprio decoráta sánguine 
 migrávit ad Sponsum.
\switchcolumn
\selectlanguage{english}
At Cyrene in Libya, St. Cyrilla, 
 martyr, in the persecution of Diocletian. She held burning coals with 
 incense on her hand for a long time, lest by shaking off the coals she 
 should seem to offer incense to the idols. She was afterwards cruelly 
 scourged, and went to her Spouse adorned with her own blood.
\switchcolumn*
\selectlanguage{latin}
Tréviris sancti 
 Numeriáni, Epíscopi et Confessóris.
\switchcolumn
\selectlanguage{english}
At Treves, St. Numerian, bishop and 
 confessor.
\switchcolumn*
\selectlanguage{latin}
Apud Septempedános, in 
 Picéno, sanctæ Philoménæ Vírginis.
\switchcolumn
\selectlanguage{english}
At San Severino in Piceno, St. 
 Philomena, virgin.
\switchcolumn*
\selectlanguage{latin}
\end{paracol}


% ---- martyrology/mart07/mart0706.htm
\needspace{10\baselineskip}
\begin{paracol}{2}
\selectlanguage{latin}
\begin{center}{\color{gregoriocolor} Prídie Nonas Júlii. 
 Luna\dots\ }\end{center}
\switchcolumn
\selectlanguage{english}
\begin{center}{\color{gregoriocolor} The 
 Sixth Day of 
 July. The\dots\ Day of the Moon.}\end{center}
\end{paracol}

\noindent\begin{tabularx}{\linewidth}{*{19}{>{\centering\arraybackslash}X}}
 \textcolor{gregoriocolor}{a} & \textcolor{gregoriocolor}{b} & \textcolor{gregoriocolor}{c} & \textcolor{gregoriocolor}{d} & \textcolor{gregoriocolor}{e} & \textcolor{gregoriocolor}{f} & \textcolor{gregoriocolor}{g} & \textcolor{gregoriocolor}{h} & \textcolor{gregoriocolor}{i} & \textcolor{gregoriocolor}{k} & \textcolor{gregoriocolor}{l} & \textcolor{gregoriocolor}{m} & \textcolor{gregoriocolor}{n} & \textcolor{gregoriocolor}{p} & \textcolor{gregoriocolor}{q} & \textcolor{gregoriocolor}{r} & \textcolor{gregoriocolor}{s} & \textcolor{gregoriocolor}{t} & \textcolor{gregoriocolor}{u} \\
 11 & 12 & 13 & 14 & 15 & 16 & 17 & 18 & 19 & 20 & 21 & 22 & 23 & 24 & 25 & 26 & 27 & 28 & 29 \\
\end{tabularx}
\vspace{0.5\baselineskip}
\noindent\begin{tabularx}{\linewidth}{*{12}{>{\centering\arraybackslash}X}}
 \textcolor{gregoriocolor}{A} & \textcolor{gregoriocolor}{B} & \textcolor{gregoriocolor}{C} & \textcolor{gregoriocolor}{D} & \textcolor{gregoriocolor}{E} & F & \textcolor{gregoriocolor}{F} & \textcolor{gregoriocolor}{G} & \textcolor{gregoriocolor}{H} & \textcolor{gregoriocolor}{M} & \textcolor{gregoriocolor}{N} & \textcolor{gregoriocolor}{P} \\
 30 & 1 & 2 & 3 & 4 & 5 & 5 & 6 & 7 & 8 & 9 & 10 \\
\end{tabularx}

\begin{paracol}{2}
\selectlanguage{latin}
\lettrine[lines=1]{O}{ctava} sanctórum Apostolórum Petri et Pauli.
\switchcolumn
\selectlanguage{english}
\lettrine[lines=1]{T}{he} Octave of the holy apostles Peter and Paul.
\switchcolumn*
\selectlanguage{latin}
Hierosólymis sancti 
 Isaiæ Prophétæ, qui, sub Manásse Rege, in duas sectus partes occúbuit, 
 sepultúsque est sub quercu Rogel, juxta tránsitum aquárum.
\switchcolumn
\selectlanguage{english}
In Jerusalem, the holy prophet 
 Isaias. During the reign of King Manasses he was put to death by being 
 sawn in two and was buried beneath the oak of Rogel, near a running stream.
\switchcolumn*
\selectlanguage{latin}
Fæsulis, in Túscia, 
 sancti Romuli, Epíscopi et Mártyris, qui fuit beáti Petri Apóstoli 
 discípulus. Hic, ab eódem Apóstolo missus ad prædicándum Evangélium, 
 in multis Itáliæ locis Christum annuntiávit; ac tandem, Fæsulas regréssus, 
 ibi, sub Domitiáno Príncipe, martyrio coronátus est cum áliis Sóciis.
\switchcolumn
\selectlanguage{english}
At Fiesole in Tuscany, St. Romulus, 
 bishop and martyr, disciple of the blessed apostle Peter, who commissioned 
 him to preach the Gospel. After announcing Christ in many parts of 
 Italy, he returned to Fiesole, and was crowned with martyrdom along with 
 other Christians in the reign of Domitian.
\switchcolumn*
\selectlanguage{latin}
Romæ natális sancti 
 Tranquillíni Mártyris, patris sanctórum Marci et Marcelliáni, qui, ad 
 prædicatiónem sancti Sebastiáni Mártyris convérsus ad Christum, a beáto 
 Polycárpo Presbytero baptizátus est, et a sancto Cajo Papa Présbyter ordinátus. Ipse, die Octavárum Apostolórum, cum ad confessiónem beáti 
 Pauli oráret, ibídem, sub Diocletiáno Imperatóre, a Pagánis tentus est, et, 
 ab eis lapidátus, martyrium consummávit.
\switchcolumn
\selectlanguage{english}
At Rome, the birthday of St. 
 Tranquillinus, martyr, father of Saints Mark and Marcellianus, who had been 
 converted to Christ by the preaching of the martyr St. Sebastian. 
 Baptized by the blessed priest Polycarp, he was ordained priest by Pope St. 
 Caius. As he prayed at the tomb of blessed Paul on the octave of the 
 apostles, he was arrested and stoned to death by the pagans, and thus 
 completed his martyrdom.
\switchcolumn*
\selectlanguage{latin}
Londíni, in Anglia, 
 sancti Thomæ More, regni Cancellárii, qui, pro fide cathólica ac beáti Petri 
 primátu, jubénte Henríco Octávo Rege, decollátus est.
\switchcolumn
\selectlanguage{english}
At London in England, on Tower Hill, 
 St. Thomas More, chancellor of the entire realm, who was beheaded by order 
 of King Henry VIII for the defence of the Catholic faith and the primacy of 
 blessed Peter.
\switchcolumn*
\selectlanguage{latin}
In Campánia sanctæ 
 Domínicæ, Vírginis et Mártyris, quæ, sub Diocletiáno Imperatóre, cum 
 fregísset idóla, hinc ad béstias damnáta, sed ab illis nil læsa, demum, 
 cápite obtruncáta, migrávit ad Dóminum. Ipsíus vero corpus Tropéæ, in 
 Calábria, summa veneratióne asservátur.
\switchcolumn
\selectlanguage{english}
In Campania, St. Dominica, virgin 
 and martyr, in the time of Emperor Diocletian. For having destroyed 
 idols, she was condemned to the beasts, but being left uninjured by them, 
 she was beheaded and departed for heaven. Her body is kept with great 
 veneration at Tropea in Calabria.
\switchcolumn*
\selectlanguage{latin}
Eódem die sanctæ Lúciæ 
 Mártyris, quæ, natióne Campána, a Ríxio Varo Vicário tenta et ácriter 
 cruciáta, eúndem convértit ad Christum; quibus adjúncti sunt Antonínus, 
 Severínus, Diodórus, Dion, et álii decem et septem, qui in passióne collégæ 
 et in coróna fuére consórtes.
\switchcolumn
\selectlanguage{english}
The same day, St. Lucia, martyr, a 
 native of Campania. Being arrested and severely tortured by the 
 lieutenant-governor Rictiovarus, she converted him to Christ. To them 
 were added Antoninus, Severinus, Diodorus, Dion, and seventeen others who 
 shared their sufferings and their crowns.
\switchcolumn*
\selectlanguage{latin}
Neptúni, in Látio, 
 sanctæ Maríæ Gorétti, piíssimæ adolescéntis, in defendénda virginitáte 
 crudelíssime necátæ, quam Pius Papa Duodécimus sanctárum Mártyrum catálogo 
 solémniter accénsuit.
\switchcolumn
\selectlanguage{english}
At Nettuno in Lazio, St. Maria 
 Goretti, a most devout young girl, who was savagely murdered for the defence 
 of her virginity, and whom Pope Pius XII solemnly added to the 
 catalogue of holy martyrs.
\switchcolumn*
\selectlanguage{latin}
In pago Trevirénsi 
 sancti Góaris, Presbyteri et Confessóris.
\switchcolumn
\selectlanguage{english}
In the vicinity of Treves, St. Goar, 
 priest and confessor.
\switchcolumn*
\selectlanguage{latin}
\end{paracol}


% ---- martyrology/mart07/mart0707.htm
\needspace{10\baselineskip}
\begin{paracol}{2}
\selectlanguage{latin}
\begin{center}{\color{gregoriocolor} Nonis Júlii. 
 Luna\dots\ }\end{center}
\switchcolumn
\selectlanguage{english}
\begin{center}{\color{gregoriocolor} The 
 Seventh Day of 
 July. The\dots\ Day of the Moon.}\end{center}
\end{paracol}

\noindent\begin{tabularx}{\linewidth}{*{19}{>{\centering\arraybackslash}X}}
 \textcolor{gregoriocolor}{a} & \textcolor{gregoriocolor}{b} & \textcolor{gregoriocolor}{c} & \textcolor{gregoriocolor}{d} & \textcolor{gregoriocolor}{e} & \textcolor{gregoriocolor}{f} & \textcolor{gregoriocolor}{g} & \textcolor{gregoriocolor}{h} & \textcolor{gregoriocolor}{i} & \textcolor{gregoriocolor}{k} & \textcolor{gregoriocolor}{l} & \textcolor{gregoriocolor}{m} & \textcolor{gregoriocolor}{n} & \textcolor{gregoriocolor}{p} & \textcolor{gregoriocolor}{q} & \textcolor{gregoriocolor}{r} & \textcolor{gregoriocolor}{s} & \textcolor{gregoriocolor}{t} & \textcolor{gregoriocolor}{u} \\
 12 & 13 & 14 & 15 & 16 & 17 & 18 & 19 & 20 & 21 & 22 & 23 & 24 & 25 & 26 & 27 & 28 & 29 & 30 \\
\end{tabularx}
\vspace{0.5\baselineskip}
\noindent\begin{tabularx}{\linewidth}{*{12}{>{\centering\arraybackslash}X}}
 \textcolor{gregoriocolor}{A} & \textcolor{gregoriocolor}{B} & \textcolor{gregoriocolor}{C} & \textcolor{gregoriocolor}{D} & \textcolor{gregoriocolor}{E} & F & \textcolor{gregoriocolor}{F} & \textcolor{gregoriocolor}{G} & \textcolor{gregoriocolor}{H} & \textcolor{gregoriocolor}{M} & \textcolor{gregoriocolor}{N} & \textcolor{gregoriocolor}{P} \\
 1 & 2 & 3 & 4 & 5 & 6 & 6 & 7 & 8 & 9 & 10 & 11 \\
\end{tabularx}

\begin{paracol}{2}
\selectlanguage{latin}
\lettrine[lines=2]{S}{anctórum} Episcopórum 
 et Confessórum Cyrílli et Methódii fratrum, quorum natális respectíve ágitur 
 sextodécimo Kaléndas Mártii et octávo Idus Aprílis.
\switchcolumn
\selectlanguage{english}
\lettrine[lines=2]{T}{he} holy bishops Cyril and Methodius, 
 whose respective birthdays are on the 14th of February and the 6th of 
 April.
\switchcolumn*
\selectlanguage{latin}
Romæ sanctórum Mártyrum 
 Cláudii Commentariénsis, Nicóstrati primiscrínii, qui fuit vir beátæ 
 Mártyris Zoæ, Castórii, Victoríni et Symphoriáni; quos omnes sanctus 
 Sebastiánus ad fidem Christi perdúxit, et beátus Polycárpus Presbyter 
 baptizávit. Eósdem, in perquiréndis sanctórum Mártyrum corpóribus 
 occupátos, Fabiánus Judex comprehéndi jussit, et, cum eos, per decem dies 
 minis et blandítiis exágitans, in nullo pénitus posset commovére, jussit 
 tértio torquéri, ac póstea in mare præcípites dari.
\switchcolumn
\selectlanguage{english}
At Rome, the holy martyrs Claudius, 
 a notary, Nicostratus, an assistant prefect who had been the husband of the 
 blessed Martyr Zoa, Castorius, Victorinus, and 
 Symphorian. All these had been brought to the faith of Christ by St. Sebastian, 
 and baptized by the blessed priest Polycarp. While they were engaged 
 in searching for the bodies of the holy martyrs, the judge Fabian had them 
 arrested, and for ten days he tried to shake their constancy by threats and 
 flatteries, but being utterly unable to succeed, he ordered them to be 
 thrice tortured, then thrown into the sea.
\switchcolumn*
\selectlanguage{latin}
Dyrráchii, in 
 Macedónia, sanctórum Mártyrum Peregríni, Luciáni, Pompéji, Hesychii, Pápii, 
 Saturníni et Germáni; qui, natióne Itali, in persecutióne Trajáni, cum ad 
 eam urbem confugíssent, et ibi sanctum Astium Epíscopum, pro fide Christi, 
 in cruce pendéntem vidérent, ac se páriter Christiános esse palam 
 confiteréntur, ideo, Præsidis jussu, tenti sunt atque in mare demérsi.
\switchcolumn
\selectlanguage{english}
At Durazzo in Macedonia, the holy 
 martyrs Peregrínus, Lucian, Pompeius, Hesychius, Papius, Saturninus, and 
 Germanus, all natives of Italy. In the persecution of Trajan they took 
 refuge in the town of Durazzo where they saw the saintly bishop Astius 
 hanging on a cross for the faith of Christ. They then publicly 
 declared themselves to be Christians, when, by order of the governor, they 
 were arrested and cast into the sea.
\switchcolumn*
\selectlanguage{latin}
Bríxiæ sancti Apollónii, 
 Epíscopi et Confessóris.
\switchcolumn
\selectlanguage{english}
At Brescia, St. Apollonius, bishop 
 and confessor.
\switchcolumn*
\selectlanguage{latin}
Eystádii, in Germánia, sancti Willebáldi, ejúsdem urbis primi Epíscopi, qui fílius 
 éxstitit sancti 
 Richárdi, Anglórum Regis, et frater sanctæ Walbúrgæ Vírginis; atque, una cum 
 sancto Bonifátio, labórans in Evangélio, multas gentes convértit ad Christum.
\switchcolumn
\selectlanguage{english}
At Eichstadt in Germany, St. 
 Willibald, the first bishop of that city. He was the son of St. 
 Richard, king of England, and brother of St. Walburga, virgin. He 
 laboured with St. Boniface in preaching the Gospel and converted many 
 nations to Christ.
\switchcolumn*
\selectlanguage{latin}
Arvérnis, in Gállia, 
 sancti Illídii Epíscopi.
\switchcolumn
\selectlanguage{english}
In Auvergne, St. Illidius, bishop.
\switchcolumn*
\selectlanguage{latin}
Urgéllæ, in Hispánia 
 Tarraconénsi, sancti Odónis Epíscopi.
\switchcolumn
\selectlanguage{english}
At Urgal in Spain, St. Odo, bishop.
\switchcolumn*
\selectlanguage{latin}
In Anglia sancti Heddæ, 
 Epíscopi Sáxonum occidentálium.
\switchcolumn
\selectlanguage{english}
In England, St. Hedda, bishop of the 
 West Saxons.
\switchcolumn*
\selectlanguage{latin}
Alexandríæ natális 
 sancti Pantæni, viri Apostólici et omni sapiéntia adornáti, cui tantum 
 stúdii et amóris erga verbum Dei quæ in Oriéntis últimis secéssibus 
 recondúntur, fídei et devotiónis calóre succénsus, proféctus sit; ac demum, 
 Alexandríam revérsus, ibi, sub Antoníno Caracálla, in pace quiévit.
\switchcolumn
\selectlanguage{english}
At Alexandria, the birthday of St. 
 Pantaenus, a man of apostolic manner, filled with wisdom. He had such 
 an affection and love for the word of God, and was so inflamed with the 
 ardour of faith and devotion, that he set out to preach the Gospel of Christ 
 to the nations living in the farthest districts of the East. Returning 
 at last to Alexandria, he rested in peace, in the time of Antoninus 
 Caracalla.
\switchcolumn*
\selectlanguage{latin}
Eboríaci, in território Meldénsi, sanctæ Edilbúrgæ, Abbatíssæ et Vírginis, Anglórum Regis fíliæ.
\switchcolumn
\selectlanguage{english}
At Faremoutier, in the neighbourhood 
 of Meaux, St. Ethelburga, virgin, daughter of the English king.
\switchcolumn*
\selectlanguage{latin}
Perúsiæ Beáti Benedícti 
 Papæ Undécimi, Tarvisíni, ex Ordine Prædicatórum, Confessóris; qui, brevi 
 Pontificátus sui spátio, Ecclésiæ pacem, disciplínæ restauratiónem, 
 religiónis increméntum mirífice promóvit.
\switchcolumn
\selectlanguage{english}
At Perugia, blessed Pope Benedict 
 XI, a native of Treviso, of the Order of Preachers, who in the brief space 
 of his pontificate greatly promoted the peace of the Church, the restoration 
 of discipline, and the spread of religion.
\switchcolumn*
\selectlanguage{latin}
\end{paracol}


% ---- martyrology/mart07/mart0708.htm
\needspace{10\baselineskip}
\begin{paracol}{2}
\selectlanguage{latin}
\begin{center}{\color{gregoriocolor} Octávo Idus Júlii. 
 Luna\dots\ }\end{center}
\switchcolumn
\selectlanguage{english}
\begin{center}{\color{gregoriocolor} The 
 Eighth Day of 
 July. The\dots\ Day of the Moon.}\end{center}
\end{paracol}

\noindent\begin{tabularx}{\linewidth}{*{19}{>{\centering\arraybackslash}X}}
 \textcolor{gregoriocolor}{a} & \textcolor{gregoriocolor}{b} & \textcolor{gregoriocolor}{c} & \textcolor{gregoriocolor}{d} & \textcolor{gregoriocolor}{e} & \textcolor{gregoriocolor}{f} & \textcolor{gregoriocolor}{g} & \textcolor{gregoriocolor}{h} & \textcolor{gregoriocolor}{i} & \textcolor{gregoriocolor}{k} & \textcolor{gregoriocolor}{l} & \textcolor{gregoriocolor}{m} & \textcolor{gregoriocolor}{n} & \textcolor{gregoriocolor}{p} & \textcolor{gregoriocolor}{q} & \textcolor{gregoriocolor}{r} & \textcolor{gregoriocolor}{s} & \textcolor{gregoriocolor}{t} & \textcolor{gregoriocolor}{u} \\
 13 & 14 & 15 & 16 & 17 & 18 & 19 & 20 & 21 & 22 & 23 & 24 & 25 & 26 & 27 & 28 & 29 & 30 & 1 \\
\end{tabularx}
\vspace{0.5\baselineskip}
\noindent\begin{tabularx}{\linewidth}{*{12}{>{\centering\arraybackslash}X}}
 \textcolor{gregoriocolor}{A} & \textcolor{gregoriocolor}{B} & \textcolor{gregoriocolor}{C} & \textcolor{gregoriocolor}{D} & \textcolor{gregoriocolor}{E} & F & \textcolor{gregoriocolor}{F} & \textcolor{gregoriocolor}{G} & \textcolor{gregoriocolor}{H} & \textcolor{gregoriocolor}{M} & \textcolor{gregoriocolor}{N} & \textcolor{gregoriocolor}{P} \\
 2 & 3 & 4 & 5 & 6 & 7 & 7 & 8 & 9 & 10 & 11 & 12 \\
\end{tabularx}

\begin{paracol}{2}
\selectlanguage{latin}
\lettrine[lines=2]{S}{anctæ} Elísabeth Víduæ, 
 Lusitanórum Regínæ, quæ ad regnum cæléste quarto Nonas hujus mensis 
 transívit.
\switchcolumn
\selectlanguage{english}
\lettrine[lines=2]{S}{t.} Elisabeth, widow, queen of 
 Portugal, whose birthday is observed on the 4th of July.
\switchcolumn*
\selectlanguage{latin}
In Asia minóre 
 sanctórum Aquilæ et Priscíllæ uxóris, de quibus in Actibus Apostolórum 
 scríbitur.
\switchcolumn
\selectlanguage{english}
In Asia Minor, the Saints Aquilla 
 and his wife Priscilla, of whom mention is made in the Acts of the Apostles.
\switchcolumn*
\selectlanguage{latin}
Herbípoli, in Germánia, 
 sancti Chiliáni Epíscopi, qui, a Románo Pontífice ad prædicándum Evangélium 
 missus, ibi, cum multos ad Christum perduxísset, una cum Sóciis Colománno 
 Presbytero et Totnáno Diácono, trucidátus est.
\switchcolumn
\selectlanguage{english}
At Wurtzburg in Germany, St. Kilian, 
 bishop, who was commissioned by the Roman Pontiff to preach the Gospel. 
 After having converted many to Christ, he was put to death along with his 
 companions Colman, a priest, and Totnan, a deacon.
\switchcolumn*
\selectlanguage{latin}
In Portu Románo 
 sanctórum quinquagínta mílitum Mártyrum, qui, in sanctæ Bonósæ confessióne 
 ad fidem addúcti, et a beáto Felíce Papa Primo baptizáti, in Aureliáno 
 Imperatóris persecutióne occísi sunt.
\switchcolumn
\selectlanguage{english}
At Porto, fifty holy martyrs, all 
 soldiers, who were led to the faith by the martyrdom of St. Bonosa, and 
 baptized by the blessed Pope Felix. They were put to death in the 
 persecution of Aurelian.
\switchcolumn*
\selectlanguage{latin}
Cæsaréæ, in Palæstína, 
 sancti Procópii Mártyris, qui, sub Diocletiáno Imperatóre, a Scythópoli 
 ductus Cæsaréam, illic, ad primam responsiónum ejus confidéntiam, a Júdice 
 Fabiáno est cápite cæsus.
\switchcolumn
\selectlanguage{english}
In Palestine, in the reign of 
 Diocletian, St. Procopius, martyr, who was brought from Scythopolis to 
 Caesarea, and upon his first resolute answer was beheaded by the judge 
 Fabian.
\switchcolumn*
\selectlanguage{latin}
Constantinópoli pássio 
 sanctórum Monachórum Abrahamitárum, qui ob cultum sanctárum Imáginum, 
 resisténtes Theóphilo Imperatóri, martyrium consummárunt.
\switchcolumn
\selectlanguage{english}
At Constantinople, the holy 
 Abrahamite monks, who resisted Emperor Theophilus by defending the 
 veneration of sacred images, and suffered martyrdom.
\switchcolumn*
\selectlanguage{latin}
Spinæ Lambérti, in 
 Æmília, sancti Hadriáni Papæ Tértii, stúdio conciliándi Románæ Ecclésias 
 Orientáles insígnis, ac miráculis clari; cujus corpus in monastérium 
 Nonantulénse fuit delátum, et in æde sancti Silvéstri honorífice cónditum.
\switchcolumn
\selectlanguage{english}
At Spina Lamberti in Emília, Pope 
 St. Adrian III, famous for his zeal in reconciling the Eastern to the Roman 
 Church, and renowned for his miracles. His body was taken to the 
 monastery of Nonantola and buried with honours in the Church of St. 
 Sylvester.
\switchcolumn*
\selectlanguage{latin}
Tréviris sancti 
 Auspícii, Epíscopi et Confessóris.
\switchcolumn
\selectlanguage{english}
At Treves, St. Auspicius, bishop and 
 confessor.
\switchcolumn*
\selectlanguage{latin}
Romæ beáti Eugénii Papæ 
 Tértii, qui, postquam cœnóbium sanctórum Vincéntii et Anastásii ad Aquas 
 Sálvias magna sanctimóniæ ac prudéntiæ laude rexísset, Ecclésiam univérsam, 
 Póntifex Máximus renuntiátus, sanctíssime gubernávit. Cultum autem, ab 
 immemorábili témpore ipsi exhíbitum, Pius Papa Nonus ratum hábuit et 
 confirmávit.
\switchcolumn
\selectlanguage{english}
At Rome, blessed Eugene III, pope. 
 Having gained a great reputation for sanctity and prudence in his government 
 of the monastery of Saints Vincent and Anastasius, he was raised to the 
 Sovereign Pontificate and ruled the universal Church in much holiness. 
 Pope Pius IX approved and confirmed the veneration paid to him.
\switchcolumn*
\selectlanguage{latin}
\end{paracol}


% ---- martyrology/mart07/mart0709.htm
\needspace{10\baselineskip}
\begin{paracol}{2}
\selectlanguage{latin}
\begin{center}{\color{gregoriocolor} Séptimo Idus Júlii. 
 Luna\dots\ }\end{center}
\switchcolumn
\selectlanguage{english}
\begin{center}{\color{gregoriocolor} The 
 Ninth Day of 
 July. The\dots\ Day of the Moon.}\end{center}
\end{paracol}

\noindent\begin{tabularx}{\linewidth}{*{19}{>{\centering\arraybackslash}X}}
 \textcolor{gregoriocolor}{a} & \textcolor{gregoriocolor}{b} & \textcolor{gregoriocolor}{c} & \textcolor{gregoriocolor}{d} & \textcolor{gregoriocolor}{e} & \textcolor{gregoriocolor}{f} & \textcolor{gregoriocolor}{g} & \textcolor{gregoriocolor}{h} & \textcolor{gregoriocolor}{i} & \textcolor{gregoriocolor}{k} & \textcolor{gregoriocolor}{l} & \textcolor{gregoriocolor}{m} & \textcolor{gregoriocolor}{n} & \textcolor{gregoriocolor}{p} & \textcolor{gregoriocolor}{q} & \textcolor{gregoriocolor}{r} & \textcolor{gregoriocolor}{s} & \textcolor{gregoriocolor}{t} & \textcolor{gregoriocolor}{u} \\
 14 & 15 & 16 & 17 & 18 & 19 & 20 & 21 & 22 & 23 & 24 & 25 & 26 & 27 & 28 & 29 & 30 & 1 & 2 \\
\end{tabularx}
\vspace{0.5\baselineskip}
\noindent\begin{tabularx}{\linewidth}{*{12}{>{\centering\arraybackslash}X}}
 \textcolor{gregoriocolor}{A} & \textcolor{gregoriocolor}{B} & \textcolor{gregoriocolor}{C} & \textcolor{gregoriocolor}{D} & \textcolor{gregoriocolor}{E} & F & \textcolor{gregoriocolor}{F} & \textcolor{gregoriocolor}{G} & \textcolor{gregoriocolor}{H} & \textcolor{gregoriocolor}{M} & \textcolor{gregoriocolor}{N} & \textcolor{gregoriocolor}{P} \\
 3 & 4 & 5 & 6 & 7 & 8 & 8 & 9 & 10 & 11 & 12 & 13 \\
\end{tabularx}

\begin{paracol}{2}
\selectlanguage{latin}
\lettrine[lines=2]{R}{omæ} ad Guttam júgiter 
 manéntem, natális sanctórum Mártyrum Zenónis, et aliórum decem míllium ac 
 ducentórum trium.
\switchcolumn
\selectlanguage{english}
\lettrine[lines=2]{A}{t} Rome, at the Ever-flowing Spring, 
 the birthday of St. Zeno and ten thousand two hundred and three other 
 martyrs.
\switchcolumn*
\selectlanguage{latin}
Gortynæ, in Creta, 
 sancti Cyrílli Epíscopi, qui, in persecutióne Décii, sub Lúcio Præside, 
 flammis est injéctus, et, cum ab igne, incénsis vínculis, illæsus evasísset, 
 ac stupóre tanti miráculi a Júdice dimíssus esset, rursus ab eódem, pro 
 instánti et álacri fídei prædicatióne facta de Christo, comprehénsus et 
 cápite plexus est.
\switchcolumn
\selectlanguage{english}
At Gortyna in Crete, in the 
 persecution of Decius, under the governor Lucius, Bishop St. Cyril. 
 When he was thrown into the flames his bonds were burned, but he was not 
 injured. The judge, struck with awe at so great a miracle, set him at 
 liberty, but as the saint began again immediately to preach with zeal the 
 faith of Christ, he was beheaded.
\switchcolumn*
\selectlanguage{latin}
Brilæ, in Hollándia, 
 pássio novémdecim Mártyrum, Gorcomiénsium nuncupatórum; quorum ex número 
 novem Sacerdótes ac duo Láici erant Fratres Minóres, quátuor Presbyteri 
 sæculáres, duo Præmonstraténses, unus Reguláris Canónicus sancti Augústíni, 
 et unus Dominicánus. Hi omnes, ob tuéndam Ecclésiæ Románæ auctoritátem 
 et reálem Christi in Eucharístia præséntiam, a Calviniánis hæréticis varia 
 ludíbria et torménta perpéssi, tandem, in trabem acti, agónem suum 
 adstríctis láqueo fáucibus, consummárunt; et a Pio Nono, Pontífice Máximo, 
 inter sanctos Mártyres reláti sunt.
\switchcolumn
\selectlanguage{english}
At Briel in Holland, the passion of 
 the nineteen martyrs of Gorcum. Of these, nine priests and two lay 
 brothers were of the Order of Friars Minor, four were secular priests, two 
 Premonstratensians, one Canon Regular of St. Augustine, and one Dominican. 
 For vindicating the authority of the Roman Church and the real presence of 
 Christ in the Eucharist, they endured various insults and torments from the 
 Calvinist heretics, and their great suffering was ended by all of them being 
 hanged. Pope Pius IX included them in the number of holy martyrs.
\switchcolumn*
\selectlanguage{latin}
In civitáte Thora, apud 
 lacum Velínum, item pássio sanctórum Anatóliæ et Audácis, sub Décio 
 Imperatóre. Ex his Anatóliæ, Christi Virgo, postquam plúrimos per 
 totam Picéni provínciam váriis languóribus afféctos curásset et in Christum 
 credéntes fecísset, Júdicis Faustiniáni jussu divérsis pœnárum genéribus est 
 vexáta, et, cum ab immísso serpénte líbera evádens Audácem convertísset ad 
 fidem, novíssime, exténsis mánibus orans, gládio transverberáta est; Audax 
 quoque, in custódiam tráditus, sine mora senténtia capitáli coronátur.
\switchcolumn
\selectlanguage{english}
In the town of Thora, on Lake Velino 
 in Italy, the martyrdom of the Saints Anatolia and Audax, under Emperor 
 Decius. Anatolia, a virgin consecrated to Christ, cured many persons 
 afflicted with various infirmities throughout the province of Piceno, and 
 made them believe in Christ. By order of the judge Faustinian she was 
 condemned to different kinds of punishment. She was cured of the sting 
 of a serpent to which she had been exposed, a miracle that converted Audax 
 to the faith. At last, praying with outstretched hands, she was 
 pierced with a sword. Audax was sent to prison, and without delay 
 sentenced to capital punishment, thus obtaining the crown of martyrdom.
\switchcolumn*
\selectlanguage{latin}
Alexandríæ sanctórum 
 Mártyrum Patermúthii, Coprétis et Alexándri; qui sub Juliáno Apóstata cæsi 
 sunt.
\switchcolumn
\selectlanguage{english}
At Alexandria, the holy martyrs 
 Patermuthius, Copres, and Alexander, who were put to death under Julian the 
 Apostate.
\switchcolumn*
\selectlanguage{latin}
Mártulæ, in Umbria, 
 sancti Bríctii Epíscopi, qui, sub Marciáno Júdice, ob confessiónem Dómini, 
 multa passus est; ac tandem, cum magnam pópuli multitúdinem ad Christum 
 convertísset, Conféssor in pace quiévit.
\switchcolumn
\selectlanguage{english}
At Martula in Umbria, St. Brictius, 
 bishop. Under the judge Marcian, after having suffered much for the 
 confession of our Lord, and having converted to Christ a great multitude of 
 people, he rested in peace, a confessor.
\switchcolumn*
\selectlanguage{latin}
Tiférni, in Umbria, 
 sanctæ Verónicæ de Juliánis, Vírginis, in Urbaniénsis diœcésis óppido 
 Mercatéllo natæ, Moniális e secúndo sancti Francísci Ordine ac Tifernátis 
 ascetérii Abbatíssæ; quam, insígni patiéndi stúdio, ceterísque virtútibus et 
 cæléstibus charismátibus illústrem, Gregórius Papa Décimus sextus in 
 sanctárum Vírginum collégium adscrípsit.
\switchcolumn
\selectlanguage{english}
At Tiferno in Umbria, St. Veronica 
 Giuliani, a nun of the second Order of St. Francis and abbess of the 
 monastery in that town. Born at Mercatello in the diocese of Urbania, 
 she became illustrious by her great love for suffering and other virtues, 
 and by her heavenly gifts. She was inscribed among the holy virgins by 
 Pope Gregory XVI.
\switchcolumn*
\selectlanguage{latin}
\end{paracol}


% ---- martyrology/mart07/mart0710.htm
\needspace{10\baselineskip}
\begin{paracol}{2}
\selectlanguage{latin}
\begin{center}{\color{gregoriocolor} Sexto Idus Júlii. 
 Luna\dots\ }\end{center}
\switchcolumn
\selectlanguage{english}
\begin{center}{\color{gregoriocolor} The 
 Tenth Day of 
 July. The\dots\ Day of the Moon.}\end{center}
\end{paracol}

\noindent\begin{tabularx}{\linewidth}{*{19}{>{\centering\arraybackslash}X}}
 \textcolor{gregoriocolor}{a} & \textcolor{gregoriocolor}{b} & \textcolor{gregoriocolor}{c} & \textcolor{gregoriocolor}{d} & \textcolor{gregoriocolor}{e} & \textcolor{gregoriocolor}{f} & \textcolor{gregoriocolor}{g} & \textcolor{gregoriocolor}{h} & \textcolor{gregoriocolor}{i} & \textcolor{gregoriocolor}{k} & \textcolor{gregoriocolor}{l} & \textcolor{gregoriocolor}{m} & \textcolor{gregoriocolor}{n} & \textcolor{gregoriocolor}{p} & \textcolor{gregoriocolor}{q} & \textcolor{gregoriocolor}{r} & \textcolor{gregoriocolor}{s} & \textcolor{gregoriocolor}{t} & \textcolor{gregoriocolor}{u} \\
 15 & 16 & 17 & 18 & 19 & 20 & 21 & 22 & 23 & 24 & 25 & 26 & 27 & 28 & 29 & 30 & 1 & 2 & 3 \\
\end{tabularx}
\vspace{0.5\baselineskip}
\noindent\begin{tabularx}{\linewidth}{*{12}{>{\centering\arraybackslash}X}}
 \textcolor{gregoriocolor}{A} & \textcolor{gregoriocolor}{B} & \textcolor{gregoriocolor}{C} & \textcolor{gregoriocolor}{D} & \textcolor{gregoriocolor}{E} & F & \textcolor{gregoriocolor}{F} & \textcolor{gregoriocolor}{G} & \textcolor{gregoriocolor}{H} & \textcolor{gregoriocolor}{M} & \textcolor{gregoriocolor}{N} & \textcolor{gregoriocolor}{P} \\
 4 & 5 & 6 & 7 & 8 & 9 & 9 & 10 & 11 & 12 & 13 & 14 \\
\end{tabularx}

\begin{paracol}{2}
\selectlanguage{latin}
\lettrine[lines=2]{R}{omæ} pássio sanctórum 
 septem Mártyrum fratrum, filiórum sanctæ Felicitátis Mártyris, id est 
 Januárii, Felícis, Philíppi, Silváni, Alexándri, Vitális et Martiális, 
 témpore Antoníni Imperatóris, sub Præfécto Urbis Públio. Ex ipsis vero 
 Januárius, post virgárum vérbera et cárceris maceratiónem, plumbátis occísus; 
 Felix et Philíppus fústibus mactáti; Silvánus præcipítio interémptus; 
 Alexánder, Vitális et Martiális capitáli senténtia puníti sunt.
\switchcolumn
\selectlanguage{english}
\lettrine[lines=2]{A}{t} Rome, the martyrdom of the seven 
 holy brothers, sons of the saintly martyr Felicitas. They are 
 Januarius, Felix, Philip, Sylvanus, Alexander, Vitalis, and Martial. 
 They died in the time of Emperor Antoninus, under Publius, prefect of the 
 city. Januarius, after being scourged with rods and detained in 
 prison, died from the blows inflicted with leaded whips. Felix and 
 Philip were scourged to death. Sylvanus was thrown headlong from a 
 great height. Alexander, Vitalis, and Martial were beheaded.
\switchcolumn*
\selectlanguage{latin}
Item Romæ sanctárum 
 Vírginum et Mártyrum Rufínæ et Secúndæ sorórum, quæ, in persecutióne 
 Valeriáni et Galliéni, torméntis subáctæ sunt; atque ad últimum, cum álteri 
 caput illísum esset gládio, et álteri cervix cæsa, migrárunt in cælum. 
 Ipsárum vero córpora in Basílica Lateranénsi, prope Baptistérium, débito 
 honóre servántur.
\switchcolumn
\selectlanguage{english}
Also at Rome, in the persecution of 
 Valerian and Gallienus, the holy virgins and martyrs Rufina and Secunda, 
 sisters. After being subjected to torments, and one having her head 
 crushed with a sword, the other beheaded, they departed for heaven. 
 Their bodies are kept with due honour in the Lateran basilica, near the baptistry.
\switchcolumn*
\selectlanguage{latin}
In Africa sanctórum 
 Mártyrum Januárii, Maríni, Nabóris et Felícis, qui gládio decolláti sunt.
\switchcolumn
\selectlanguage{english}
In Africa, the holy martyrs 
 Januarius, Marinus, Nabor and Felix, all of whom were beheaded.
\switchcolumn*
\selectlanguage{latin}
Nicópoli, in Arménia, sanctórum Mártyrum Leóntii, Maurítii, Daniélis, et Sociórum; qui, sub 
 Licínio Imperatóre et Lysia Præside, várie excruciáti, tandem, in ignem 
 conjécti, martyrii cursum confecérunt.
\switchcolumn
\selectlanguage{english}
At Nicopolis in Armenia, the holy 
 martyrs Leontius, Mauritius, Daniel, and their companions, who were tortured 
 in different ways, and being lastly cast into the fire, ended their long 
 martyrdom in the time of Emperor Licinius and the governor Lysias.
\switchcolumn*
\selectlanguage{latin}
In Pisídia sanctórum 
 Mártyrum Biánoris et Silváni, qui, sævíssima pro Christi nómine passi, demum, 
 cervícibus abscíssis, coronántur.
\switchcolumn
\selectlanguage{english}
In Pisidia, the holy martyrs Bianor 
 and Silvanus, who were merited an immortal crown by being beheaded, after 
 enduring most bitter torments for the name of Christ.
\switchcolumn*
\selectlanguage{latin}
Icónii, in Lycaónia, 
 sancti Apollónii Mártyris, qui per crucem insígne martyrium consummávit.
\switchcolumn
\selectlanguage{english}
At Iconium, St. Apollonius, martyr, 
 whose glorious martyrdom was fulfilled by death on the cross.
\switchcolumn*
\selectlanguage{latin}
Apud Gandávum, in 
 Flándria, sanctæ Amelbérgæ Vírginis.
\switchcolumn
\selectlanguage{english}
At Ghent in Flanders, St. Amelberga, 
 virgin.
\switchcolumn*
\selectlanguage{latin}
\end{paracol}


% ---- martyrology/mart07/mart0711.htm
\needspace{10\baselineskip}
\begin{paracol}{2}
\selectlanguage{latin}
\begin{center}{\color{gregoriocolor} Quinto Idus Júlii. 
 Luna\dots\ }\end{center}
\switchcolumn
\selectlanguage{english}
\begin{center}{\color{gregoriocolor} The 
 Eleventh Day of 
 July. The\dots\ Day of the Moon.}\end{center}
\end{paracol}

\noindent\begin{tabularx}{\linewidth}{*{19}{>{\centering\arraybackslash}X}}
 \textcolor{gregoriocolor}{a} & \textcolor{gregoriocolor}{b} & \textcolor{gregoriocolor}{c} & \textcolor{gregoriocolor}{d} & \textcolor{gregoriocolor}{e} & \textcolor{gregoriocolor}{f} & \textcolor{gregoriocolor}{g} & \textcolor{gregoriocolor}{h} & \textcolor{gregoriocolor}{i} & \textcolor{gregoriocolor}{k} & \textcolor{gregoriocolor}{l} & \textcolor{gregoriocolor}{m} & \textcolor{gregoriocolor}{n} & \textcolor{gregoriocolor}{p} & \textcolor{gregoriocolor}{q} & \textcolor{gregoriocolor}{r} & \textcolor{gregoriocolor}{s} & \textcolor{gregoriocolor}{t} & \textcolor{gregoriocolor}{u} \\
 16 & 17 & 18 & 19 & 20 & 21 & 22 & 23 & 24 & 25 & 26 & 27 & 28 & 29 & 30 & 1 & 2 & 3 & 4 \\
\end{tabularx}
\vspace{0.5\baselineskip}
\noindent\begin{tabularx}{\linewidth}{*{12}{>{\centering\arraybackslash}X}}
 \textcolor{gregoriocolor}{A} & \textcolor{gregoriocolor}{B} & \textcolor{gregoriocolor}{C} & \textcolor{gregoriocolor}{D} & \textcolor{gregoriocolor}{E} & F & \textcolor{gregoriocolor}{F} & \textcolor{gregoriocolor}{G} & \textcolor{gregoriocolor}{H} & \textcolor{gregoriocolor}{M} & \textcolor{gregoriocolor}{N} & \textcolor{gregoriocolor}{P} \\
 5 & 6 & 7 & 8 & 9 & 10 & 10 & 11 & 12 & 13 & 14 & 15 \\
\end{tabularx}

\begin{paracol}{2}
\selectlanguage{latin}
\lettrine[lines=2]{R}{omæ} sancti Pii Primi, 
 Papæ et Mártyris; qui martyrio coronátus est in persecutióne Marci Aurélii 
 Antoníni.
\switchcolumn
\selectlanguage{english}
\lettrine[lines=2]{A}{t} Rome, Pope Pius I, who was 
 crowned with martyrdom in the persecution of Marcus Aurelius Antoninus.
\switchcolumn*
\selectlanguage{latin}
Bérgomi sancti Joánnis 
 Epíscopi, qui, ob tuéndam cathólicam fidem, ab Ariánis occísus est.
\switchcolumn
\selectlanguage{english}
At Bergamo, St. John, a bishop, who 
 was killed by the Arians for defending the Catholic faith.
\switchcolumn*
\selectlanguage{latin}
Sidæ, in Pamphylia, 
 sancti Cindéi Presbyteri, qui, sub Diocletiáno Imperatóre et Stratoníco 
 Præside, post multa torménta, injéctus in ignem et nil læsus, demum in 
 oratióne réddidit spíritum.
\switchcolumn
\selectlanguage{english}
At Sida in Pamphylia, St. Cindeus, 
 priest, in the time of Emperor Diocletian and the governor Stratonicus. 
 After suffering many torments, he was thrown into the fire, but was not 
 injured by it. He later yielded up his soul in prayer.
\switchcolumn*
\selectlanguage{latin}
Córdubæ, in Hispánia, 
 sancti Abúndii Presbyteri, qui, in persecutióne Arábica, cum in Mahumétis 
 sectam inveherétur, martyrio coronátus est.
\switchcolumn
\selectlanguage{english}
At Cordova in Spain, St. Abundius, a 
 priest, crowned with martyrdom while preaching against the sect of Mohammed.
\switchcolumn*
\selectlanguage{latin}
Nicópoli, in Arménia, natális sanctórum Mártyrum Januárii et Pelágiæ, qui, equúleo, 
 úngulis et 
 testárum fragméntis per dies quátuor cruciáti, martyrium implevérunt.
\switchcolumn
\selectlanguage{english}
At Nicopolis in Armenia, the 
 birthday of the holy martyrs Januarius and Pelagia, who for four days were 
 racked, torn with iron claws and pieces of earthenware, and thus achieved 
 their martyrdom.
\switchcolumn*
\selectlanguage{latin}
In território Senonénsi 
 sancti Sidrónii Mártyris.
\switchcolumn
\selectlanguage{english}
In the territory of Sens, St. 
 Sidronius, martyr.
\switchcolumn*
\selectlanguage{latin}
Icónii, in Lycaónia, 
 sancti Marciáni Mártyris, qui, sub Perénnio Præside, per multa torménta 
 pervénit ad palmam.
\switchcolumn
\selectlanguage{english}
At Iconium in Lycaonia, St. Marcian, 
 martyr. He obtained the palm of martyrdom by many torments, under the 
 governor Perennius.
\switchcolumn*
\selectlanguage{latin}
Bríxiæ sanctórum 
 Mártyrum Savíni et Cypriáni.
\switchcolumn
\selectlanguage{english}
At Brescia, the holy martyrs Savinus 
 and Cyprian.
\switchcolumn*
\selectlanguage{latin}
In território 
 Pictaviénsi sancti Sabíni Confessóris.
\switchcolumn
\selectlanguage{english}
In the territory of Poitiers, St. 
 Sabinus, confessor.
\switchcolumn*
\selectlanguage{latin}
\end{paracol}


% ---- martyrology/mart07/mart0712.htm
\needspace{10\baselineskip}
\begin{paracol}{2}
\selectlanguage{latin}
\begin{center}{\color{gregoriocolor} Quarto Idus Júlii. 
 Luna\dots\ }\end{center}
\switchcolumn
\selectlanguage{english}
\begin{center}{\color{gregoriocolor} The 
 Twelfth Day of 
 July. The\dots\ Day of the Moon.}\end{center}
\end{paracol}

\noindent\begin{tabularx}{\linewidth}{*{19}{>{\centering\arraybackslash}X}}
 \textcolor{gregoriocolor}{a} & \textcolor{gregoriocolor}{b} & \textcolor{gregoriocolor}{c} & \textcolor{gregoriocolor}{d} & \textcolor{gregoriocolor}{e} & \textcolor{gregoriocolor}{f} & \textcolor{gregoriocolor}{g} & \textcolor{gregoriocolor}{h} & \textcolor{gregoriocolor}{i} & \textcolor{gregoriocolor}{k} & \textcolor{gregoriocolor}{l} & \textcolor{gregoriocolor}{m} & \textcolor{gregoriocolor}{n} & \textcolor{gregoriocolor}{p} & \textcolor{gregoriocolor}{q} & \textcolor{gregoriocolor}{r} & \textcolor{gregoriocolor}{s} & \textcolor{gregoriocolor}{t} & \textcolor{gregoriocolor}{u} \\
 17 & 18 & 19 & 20 & 21 & 22 & 23 & 24 & 25 & 26 & 27 & 28 & 29 & 30 & 1 & 2 & 3 & 4 & 5 \\
\end{tabularx}
\vspace{0.5\baselineskip}
\noindent\begin{tabularx}{\linewidth}{*{12}{>{\centering\arraybackslash}X}}
 \textcolor{gregoriocolor}{A} & \textcolor{gregoriocolor}{B} & \textcolor{gregoriocolor}{C} & \textcolor{gregoriocolor}{D} & \textcolor{gregoriocolor}{E} & F & \textcolor{gregoriocolor}{F} & \textcolor{gregoriocolor}{G} & \textcolor{gregoriocolor}{H} & \textcolor{gregoriocolor}{M} & \textcolor{gregoriocolor}{N} & \textcolor{gregoriocolor}{P} \\
 6 & 7 & 8 & 9 & 10 & 11 & 11 & 12 & 13 & 14 & 15 & 16 \\
\end{tabularx}

\begin{paracol}{2}
\selectlanguage{latin}
\lettrine[lines=2]{I}{n} monastério 
 Passiniáno, prope Floréntiam, sancti Joánnis Gualbérti Abbátis, qui fuit 
 Institútor Ordinis Vallis Umbrósæ.
\switchcolumn
\selectlanguage{english}
\lettrine[lines=2]{I}{n} the monastery of Passignano, near 
 Florence, Abbot St. John Gualbert, founder of the Order of Vallombrosa.
\switchcolumn*
\selectlanguage{latin}
Laudæ, in Insúbria, 
 sanctórum Mártyrum Náboris et Felícis, qui, in persecutióne Maximiáni, post 
 vária torménta, cápitis decollatióne martyrium complevérunt; eorúmque 
 córpora a beáta Savína Mediolánum advécta sunt, ibíque honorífice sepúlta.
\switchcolumn
\selectlanguage{english}
At Milan, the holy martyrs Nabor and 
 Felix, who suffered in the persecution of Maximian. Their bodies were 
 brought into the city by blessed Savina, and were honourably buried there.
\switchcolumn*
\selectlanguage{latin}
In Cypro beáti Jasónis, 
 qui fuit unus ex antíquis Christi discípulis.
\switchcolumn
\selectlanguage{english}
In Cyprus, St. Jason, one of the 
 first disciples of Christ.
\switchcolumn*
\selectlanguage{latin}
Lucæ, in Túscia, beáti 
 Paulíni, qui, a sancto Petro Apóstolo primus ejúsdem civitátis Epíscopus est 
 ordinátus; et, sub Neróne, ad radíces montis Pisáni, post multos agónes, 
 martyrium suum cum áliis Sóciis consummávit.
\switchcolumn
\selectlanguage{english}
At Lucca in Tuscany, blessed 
 Paulinus, who was consecrated first bishop of that city by St. Peter. 
 Under Nero he completed his martyrdom along with many others at the foot of 
 Mt. Pisa, but only after many trials.
\switchcolumn*
\selectlanguage{latin}
Aquiléjæ natális sancti 
 Hermágoræ, qui éxstitit discípulus beáti Marci Evangelístæ, et primus 
 ejúsdem civitátis Epíscopus. Hic, inter mirácula sanitátum, et 
 prædicatiónis instántiam, ac populórum conversiónem, plúrima pœnárum génera 
 expértus, tandem, una cum Diácono suo Fortunáto, capitáli supplício 
 perpétuum méruit obtinére triúmphum.
\switchcolumn
\selectlanguage{english}
At Aquileia, the birthday of St. 
 Hermagoras, disciple of the blessed evangelist Mark, and first bishop of 
 that city. When performing miraculous cures, or while preaching, 
 frequently bringing souls to repentance, he suffered many torments. 
 Finally by capital punishment her merited an immortal triumph along with his 
 deacon Fortunatus.
\switchcolumn*
\selectlanguage{latin}
Eódem die pássio 
 sanctórum Proci et Hilariónis, qui, sub Trajáno Imperatóre et Máximo Præside, 
 per acerbíssima torménta ad palmam martyrii pervenérunt.
\switchcolumn
\selectlanguage{english}
The same day, the Saints Proclus and 
 Hilarion, who won the palm of martyrdom after most bitter torments, in the 
 time of Emperor Trajan and the governor Maximus.
\switchcolumn*
\selectlanguage{latin}
Toléti, in Hispánia, 
 sanctæ Marciánæ, Vírginis et Mártyris; quæ, pro fide Christi, objécta 
 béstiis atque a tauro discérpta, martyrio coronátur.
\switchcolumn
\selectlanguage{english}
At Toledo in Spain, St. Marciana, 
 virgin and martyr. For the faith of Christ, she was cast to the 
 beasts, torn to pieces by a bull, and was thus crowned with martyrdom.
\switchcolumn*
\selectlanguage{latin}
Apud Leontínos, in 
 Sicília, sanctæ Epíphanæ, quæ sub Diocletiáno Imperatóre et Tertyllo Præside, 
 ubéribus præcísis, réddidit spíritum.
\switchcolumn
\selectlanguage{english}
At Lentini, St. Epiphana, who, after 
 her breasts were cut away, died in the time of Emperor Diocletian and the 
 governor Tertillus.
\switchcolumn*
\selectlanguage{latin}
Lugdúni, in Gállia, 
 sancti Viventíoli Epíscopi.
\switchcolumn
\selectlanguage{english}
At Lyons in France, St. Viventiolus,bishop.
\switchcolumn*
\selectlanguage{latin}
Bonóniæ sancti 
 Paterniáni Epíscopi.
\switchcolumn
\selectlanguage{english}
At Bologna, St. Paternian, bishop.
\switchcolumn*
\selectlanguage{latin}
\end{paracol}


% ---- martyrology/mart07/mart0713.htm
\needspace{10\baselineskip}
\begin{paracol}{2}
\selectlanguage{latin}
\begin{center}{\color{gregoriocolor} Tértio Idus Júlii. 
 Luna\dots\ }\end{center}
\switchcolumn
\selectlanguage{english}
\begin{center}{\color{gregoriocolor} The 
 Thirteenth Day of 
 July. The\dots\ Day of the Moon.}\end{center}
\end{paracol}

\noindent\begin{tabularx}{\linewidth}{*{19}{>{\centering\arraybackslash}X}}
 \textcolor{gregoriocolor}{a} & \textcolor{gregoriocolor}{b} & \textcolor{gregoriocolor}{c} & \textcolor{gregoriocolor}{d} & \textcolor{gregoriocolor}{e} & \textcolor{gregoriocolor}{f} & \textcolor{gregoriocolor}{g} & \textcolor{gregoriocolor}{h} & \textcolor{gregoriocolor}{i} & \textcolor{gregoriocolor}{k} & \textcolor{gregoriocolor}{l} & \textcolor{gregoriocolor}{m} & \textcolor{gregoriocolor}{n} & \textcolor{gregoriocolor}{p} & \textcolor{gregoriocolor}{q} & \textcolor{gregoriocolor}{r} & \textcolor{gregoriocolor}{s} & \textcolor{gregoriocolor}{t} & \textcolor{gregoriocolor}{u} \\
 18 & 19 & 20 & 21 & 22 & 23 & 24 & 25 & 26 & 27 & 28 & 29 & 30 & 1 & 2 & 3 & 4 & 5 & 6 \\
\end{tabularx}
\vspace{0.5\baselineskip}
\noindent\begin{tabularx}{\linewidth}{*{12}{>{\centering\arraybackslash}X}}
 \textcolor{gregoriocolor}{A} & \textcolor{gregoriocolor}{B} & \textcolor{gregoriocolor}{C} & \textcolor{gregoriocolor}{D} & \textcolor{gregoriocolor}{E} & F & \textcolor{gregoriocolor}{F} & \textcolor{gregoriocolor}{G} & \textcolor{gregoriocolor}{H} & \textcolor{gregoriocolor}{M} & \textcolor{gregoriocolor}{N} & \textcolor{gregoriocolor}{P} \\
 7 & 8 & 9 & 10 & 11 & 12 & 12 & 13 & 14 & 15 & 16 & 17 \\
\end{tabularx}

\begin{paracol}{2}
\selectlanguage{latin}
\lettrine[lines=2]{R}{omæ} sancti Anacléti, 
 Papæ et Mártyris, qui, post sanctum Cleméntem Ecclésiam Dei regens, eam 
 glorióso martyrio decorávit.
\switchcolumn
\selectlanguage{english}
\lettrine[lines=2]{A}{t} Rome, St. Anacletus, pope and 
 martyr, who governed the Church of God after St. Clement, and shed lustre 
 upon it by a glorious martyrdom.
\switchcolumn*
\selectlanguage{latin}
Bambérgæ natális sancti 
 Henríci Primi, Imperatóris Romanórum et Confessóris, qui cum sancta 
 Cunegúnde, uxóre sua, perpétuam virginitátem servávit, et sanctum Stéphanum, 
 Hungarórum Regem, cum univérso fere ipsíus regno, ad fidem Christi 
 suscipiéndam perdúxit. Ejus autem festívitas Idibus mensis hujus 
 celebrátur.
\switchcolumn
\selectlanguage{english}
At Bamberg, the birthday of the 
 Roman emperor St. Henry I, confessor. He led a life of perpetual 
 virginity with his wife St. Cunegunde, and converted St. Stephen, king of 
 Hungary,and almost all his people to the faith of Christ. His festival 
 is celebrated on the 15th of July.
\switchcolumn*
\selectlanguage{latin}
In Palæstína sanctórum 
 Joélis et Esdræ Prophetárum.
\switchcolumn
\selectlanguage{english}
In Palestine the holy prophets Joel 
 and Esdras.
\switchcolumn*
\selectlanguage{latin}
In Macedónia beáti Silæ, 
 qui, cum esset unus de primis frátribus et ab Apóstolis ad Ecclésias Géntium, 
 una cum Paulo et Bárnaba, destinátus, prædicatiónis offícium, grátia Dei 
 plenus, instánter consummávit, atque, in passiónibus suis Christum 
 claríficans, póstmodum requiévit.
\switchcolumn
\selectlanguage{english}
In Macedonia, blessed Silas, one of 
 the first Christians. The apostles sent him with Paul and Barnabas to 
 the churches of the gentiles. Filled with the grace of God, he 
 zealously discharged the office of preaching, and after glorifying Christ by 
 his sufferings, rested in peace.
\switchcolumn*
\selectlanguage{latin}
Item sancti Serapiónis 
 Mártyris, qui, sub Sevéro Imperatóre et Aquila Præside, per ignem pervénit 
 ad corónam martyrii.
\switchcolumn
\selectlanguage{english}
Also, St. Serapion, martyr, who 
 obtained the crown of martyrdom by fire, in the time of Emperor Severus and 
 the governor Aquila.
\switchcolumn*
\selectlanguage{latin}
In ínsula Chio sanctæ 
 Myrópis Mártyris, quæ sub Décio Imperatóre et Numeriáno Præside, véctibus 
 contúsa, migrávit ad Dóminum.
\switchcolumn
\selectlanguage{english}
In the island of Chio, in the time 
 of Emperor Decius and the governor Numerian, the martyr St. Myrope. 
 She went to the Lord after being beaten with clubs.
\switchcolumn*
\selectlanguage{latin}
In Africa sanctórum 
 Confessórum Eugénii, Carthaginénsis Epíscopi, fide ac virtútibus gloriósi, 
 et univérsi cleri ejúsdem Ecclésiæ, qui fere quingénti, vel eo ámplius, in 
 persecutióne Wandálica, sub Ariáno Rege Hunneríco, cæde inediáque maceráti 
 (inter quos plúrimi erant Lectóres infántuli), gaudéntes in Dómino, exsílio 
 crudéli procul extrúsi sunt. Erant étiam nobilíssimi in eis 
 Archidiáconus, nómine Salutáris, et Murítta, secúndus in offício ministrórum; 
 qui, tértio Confessóres effécti, ambo glorióse in Christo, perseverántiæ 
 título, illustráti sunt.
\switchcolumn
\selectlanguage{english}
In Africa, the holy confessors 
 Eugene, the faithful and virtuous bishop of Carthage, and all the clergy of 
 that Church, to the number of about five hundred or more, among whom were 
 many small children who performed the office of lector. In the 
 persecution of the Vandals, under the Arian king Hunneric, they were 
 subjected to scourging and starvation, and driven into a most painful 
 banishment which they bore with joy for God's sake. In their number 
 were also two distinguished persons, the archdeacon Salutaris, and Muritta, 
 occupying the second rank among the ministers of the Church. Both had 
 three times confessed the faith, and were illustrious by their sturdy 
 perseverance in Christianity.
\switchcolumn*
\selectlanguage{latin}
In Británnia minóre 
 sancti Turiávi, Epíscopi et Confessóris, miræ simplicitátis et innocéntiæ 
 viri.
\switchcolumn
\selectlanguage{english}
In Brittany, St. Turian, bishop and 
 confessor, a man of admirable simplicity and innocence.
\switchcolumn*
\selectlanguage{latin}
\end{paracol}


% ---- martyrology/mart07/mart0714.htm
\needspace{10\baselineskip}
\begin{paracol}{2}
\selectlanguage{latin}
\begin{center}{\color{gregoriocolor} Prídie Idus Júlii. 
 Luna\dots\ }\end{center}
\switchcolumn
\selectlanguage{english}
\begin{center}{\color{gregoriocolor} The 
 Fourteenth Day of 
 July. The\dots\ Day of the Moon.}\end{center}
\end{paracol}

\noindent\begin{tabularx}{\linewidth}{*{19}{>{\centering\arraybackslash}X}}
 \textcolor{gregoriocolor}{a} & \textcolor{gregoriocolor}{b} & \textcolor{gregoriocolor}{c} & \textcolor{gregoriocolor}{d} & \textcolor{gregoriocolor}{e} & \textcolor{gregoriocolor}{f} & \textcolor{gregoriocolor}{g} & \textcolor{gregoriocolor}{h} & \textcolor{gregoriocolor}{i} & \textcolor{gregoriocolor}{k} & \textcolor{gregoriocolor}{l} & \textcolor{gregoriocolor}{m} & \textcolor{gregoriocolor}{n} & \textcolor{gregoriocolor}{p} & \textcolor{gregoriocolor}{q} & \textcolor{gregoriocolor}{r} & \textcolor{gregoriocolor}{s} & \textcolor{gregoriocolor}{t} & \textcolor{gregoriocolor}{u} \\
 19 & 20 & 21 & 22 & 23 & 24 & 25 & 26 & 27 & 28 & 29 & 30 & 1 & 2 & 3 & 4 & 5 & 6 & 7 \\
\end{tabularx}
\vspace{0.5\baselineskip}
\noindent\begin{tabularx}{\linewidth}{*{12}{>{\centering\arraybackslash}X}}
 \textcolor{gregoriocolor}{A} & \textcolor{gregoriocolor}{B} & \textcolor{gregoriocolor}{C} & \textcolor{gregoriocolor}{D} & \textcolor{gregoriocolor}{E} & F & \textcolor{gregoriocolor}{F} & \textcolor{gregoriocolor}{G} & \textcolor{gregoriocolor}{H} & \textcolor{gregoriocolor}{M} & \textcolor{gregoriocolor}{N} & \textcolor{gregoriocolor}{P} \\
 8 & 9 & 10 & 11 & 12 & 13 & 13 & 14 & 15 & 16 & 17 & 18 \\
\end{tabularx}

\begin{paracol}{2}
\selectlanguage{latin}
\lettrine[lines=2]{S}{ancti} Bonaventúræ, ex 
 Ordine Minórum, Cardinális et Epíscopi Albanénsis, Confessóris et Ecclésiæ 
 Doctóris; qui sequénti die migrávit ad Dóminum.
\switchcolumn
\selectlanguage{english}
\lettrine[lines=2]{S}{t.} Bonaventure of the Order of 
 Friars Minor, cardinal and bishop of Albano, confessor and doctor of the 
 Church, who passed to the Lord on the day following this.
\switchcolumn*
\selectlanguage{latin}
Romæ natális sancti 
 Camílli de Lellis, Presbyteri et Confessóris, Clericórum Regulárium infírmis 
 ministrántium Institutóris; quem virtútibus et miráculis clarum, Summus 
 Póntifex Benedíctus Décimus quartus in Sanctórum número recénsuit, et Leo 
 Décimus tértius cæléstem hospitálium et infirmórum Patrónum renuntiávit. 
 Ipsíus tamen festívitas quintodécimo Kaléndas Augústi recólitur.
\switchcolumn
\selectlanguage{english}
At Rome, the birthday of St. 
 Camillus de Lellis, priest and confessor, founder of the Clerks Regular for 
 Ministering to the Sick. Pope Benedict XIV numbered him among the 
 saints because of the fame of his miracles and virtues; Pope Leo XIII 
 appointed him heavenly protector of hospitals and of the sick. His 
 feast is observed on the 18th of July.
\switchcolumn*
\selectlanguage{latin}
Item Romæ sancti Justi 
 mílitis, qui, sub Cláudio Tribúno, apparénte sibi divínitus Cruce, crédidit 
 in Christum, et, mox baptizátus, ómnia sua paupéribus erogávit; tentus deínde 
 a Magnétio Præfécto, atque verberári nervis, gálea igníta cóntegi et in 
 rogum immítti jussus ac ne in capíllo quidem læsus, in Dómini confessióne 
 réddidit spíritum.
\switchcolumn
\selectlanguage{english}
Also at Rome, St. Justus, a soldier 
 under the tribune Claudius. When a miraculous cross appeared to him he 
 believed in Christ, was baptized, and gave away his goods to the poor. 
 Afterwards arrested by the prefect Magnetius, he was scourged with rods, had 
 a heated helmet put on his head, and was thrown into the fire, but received 
 no injury, not even to a hair of his head. In the end he yielded up 
 his soul confessing the Lord.
\switchcolumn*
\selectlanguage{latin}
Synópe, in Ponto, 
 sancti Phocæ Mártyris, ejúsdem civitátis Epíscopi, qui, sub Trajáno 
 Imperatóre, cárcerem, víncula, ferrum ignémque pro Christo súperans, 
 evolávit in cælum. Ejus relíquiæ póstea Viénnam, in Gállia, sunt 
 delátæ, et in Basílica sanctórum Apostolórum cónditæ.
\switchcolumn
\selectlanguage{english}
At Sinope in Pontus, the martyr St. 
 Phocas, bishop of the city. Under Emperor Trajan, after having been 
 imprisoned, bound, struck with the sword, and exposed to the fire for 
 Christ, he departed to heaven. His remains were brought to Vienne in 
 France, and deposited in the Church of the Holy Apostles.
\switchcolumn*
\selectlanguage{latin}
Alexandríæ sancti 
 Héraclæ Antístitis, ob cujus celebérrimam opiniónem Africánus 
 historiógraphus mémorat se Alexandriam, ad eum viséndum, properásse.
\switchcolumn
\selectlanguage{english}
At Alexandria, St. Heracles, bishop, 
 whose fame was so great that the historian Africanus testifies that he 
 journeyed to Alexandria to see him.
\switchcolumn*
\selectlanguage{latin}
Carthágine sancti Cyri 
 Epíscopi, in cujus festivitáte sanctus Augustínus de eo sermónem ad pópulum 
 hábuit.
\switchcolumn
\selectlanguage{english}
At Carthage, St. Cyrus, bishop, on 
 whose festival St. Augustine spoke of him to his people.
\switchcolumn*
\selectlanguage{latin}
Novocómi sancti Felícis, 
 qui fuit primus ejúsdem civitátis Epíscopus.
\switchcolumn
\selectlanguage{english}
At Como, St. Felix, first bishop of 
 that city.
\switchcolumn*
\selectlanguage{latin}
Bríxiæ sancti Optatiáni 
 Epíscopi.
\switchcolumn
\selectlanguage{english}
At Brescia, St. Optatian, bishop.
\switchcolumn*
\selectlanguage{latin}
Davéntriæ, in Belgis, 
 sancti Marcellíni, Presbyteri et Confessóris.
\switchcolumn
\selectlanguage{english}
At Deventer in Belgium, St. 
 Marcellinus, priest and confessor.
\switchcolumn*
\selectlanguage{latin}
Limæ, in Perúvia, 
 sancti Francísci Soláni, Sacerdótis ex Ordine Minórum et Confessóris; qui, 
 prædicatióne, signis et virtútibus apud Indos Occidentáles illústris, 
 migrávit ad Dóminum, atque a Benedícto Décimo tértio, Pontífice Máximo, 
 Sanctórum fastis adscríptus est.
\switchcolumn
\selectlanguage{english}
At Lima in Peru, St. Francis Solano, 
 a priest and confessor of the Order of Friars Minor. He passed to the 
 Lord in the West Indies, renowned for his preaching, miracles and virtues. 
 Pope Benedict XIII placed him on the canon of the saints.
\switchcolumn*
\selectlanguage{latin}
\end{paracol}


% ---- martyrology/mart07/mart0715.htm
\needspace{10\baselineskip}
\begin{paracol}{2}
\selectlanguage{latin}
\begin{center}{\color{gregoriocolor} Idibus Júlii. 
 Luna\dots\ }\end{center}
\switchcolumn
\selectlanguage{english}
\begin{center}{\color{gregoriocolor} The 
 Fifteenth Day of 
 July. The\dots\ Day of the Moon.}\end{center}
\end{paracol}

\noindent\begin{tabularx}{\linewidth}{*{19}{>{\centering\arraybackslash}X}}
 \textcolor{gregoriocolor}{a} & \textcolor{gregoriocolor}{b} & \textcolor{gregoriocolor}{c} & \textcolor{gregoriocolor}{d} & \textcolor{gregoriocolor}{e} & \textcolor{gregoriocolor}{f} & \textcolor{gregoriocolor}{g} & \textcolor{gregoriocolor}{h} & \textcolor{gregoriocolor}{i} & \textcolor{gregoriocolor}{k} & \textcolor{gregoriocolor}{l} & \textcolor{gregoriocolor}{m} & \textcolor{gregoriocolor}{n} & \textcolor{gregoriocolor}{p} & \textcolor{gregoriocolor}{q} & \textcolor{gregoriocolor}{r} & \textcolor{gregoriocolor}{s} & \textcolor{gregoriocolor}{t} & \textcolor{gregoriocolor}{u} \\
 20 & 21 & 22 & 23 & 24 & 25 & 26 & 27 & 28 & 29 & 30 & 1 & 2 & 3 & 4 & 5 & 6 & 7 & 8 \\
\end{tabularx}
\vspace{0.5\baselineskip}
\noindent\begin{tabularx}{\linewidth}{*{12}{>{\centering\arraybackslash}X}}
 \textcolor{gregoriocolor}{A} & \textcolor{gregoriocolor}{B} & \textcolor{gregoriocolor}{C} & \textcolor{gregoriocolor}{D} & \textcolor{gregoriocolor}{E} & F & \textcolor{gregoriocolor}{F} & \textcolor{gregoriocolor}{G} & \textcolor{gregoriocolor}{H} & \textcolor{gregoriocolor}{M} & \textcolor{gregoriocolor}{N} & \textcolor{gregoriocolor}{P} \\
 9 & 10 & 11 & 12 & 13 & 14 & 14 & 15 & 16 & 17 & 18 & 19 \\
\end{tabularx}

\begin{paracol}{2}
\selectlanguage{latin}
\lettrine[lines=2]{S}{ancti} Henríci Primi, 
 Imperatóris Romanórum et Confessóris, cujus dies natális tértio Idus mensis 
 hujus recensétur.
\switchcolumn
\selectlanguage{english}
\lettrine[lines=2]{S}{t.} Henry I, Roman emperor and 
 confessor, whose birthday was noted on the 13th of this month.
\switchcolumn*
\selectlanguage{latin}
Lugdúni, in Gállia, 
 deposítio sancti Bonaventúræ, Cardinális et Epíscopi Albanénsis, Confessóris 
 et Ecclésiæ Doctóris, ex Ordine Minórum, doctrína et vitæ sanctitáte 
 celebérrimi. Ipsíus tamen festívitas prídie hujus diéi recólitur.
\switchcolumn
\selectlanguage{english}
At Lyons in France, the death of St. 
 Bonaventure, cardinal and bishop of Albano, confessor and doctor of the 
 Church, of the Order of Friars Minor, who is famed for his learning and the 
 sanctity of his life. His feast is celebrated on the previous day.
\switchcolumn*
\selectlanguage{latin}
Papíæ sancti Felícis, 
 Epíscopi et Mártyris.
\switchcolumn
\selectlanguage{english}
At Pavia, St. Felix, bishop and 
 martyr.
\switchcolumn*
\selectlanguage{latin}
In Portu Románo natális 
 sanctórum Mártyrum Eutrópii, atque Zósimæ et Bonósæ sorórum.
\switchcolumn
\selectlanguage{english}
At Porto, the birthday of the holy 
 martyrs Eutropius, and the sisters Zosima and Bonosa.
\switchcolumn*
\selectlanguage{latin}
Carthágine beáti 
 Catulíni Diáconi, de cujus láudibus sanctus Augustínus sermónem ad pópulum 
 hábuit, et sanctórum Januárii, Floréntii, Júliæ et Justæ Mártyrum; qui 
 pósiti sunt in Basílica Fausti.
\switchcolumn
\selectlanguage{english}
At Carthage, blessed Catulinus, 
 deacon, whose glories were proclaimed by St. Augustine in a sermon to his 
 people. Also the saints Januarius, Florentius, Julia, and Justa, 
 martyrs, who were entombed in the Church of St. Faustus.
\switchcolumn*
\selectlanguage{latin}
Alexandríæ sanctórum 
 Mártyrum Philíppi, Zenónis, Narséi et decem infántum.
\switchcolumn
\selectlanguage{english}
At Alexandria, the holy martyrs 
 Philip, Zeno, Narseus, and ten children.
\switchcolumn*
\selectlanguage{latin}
In ínsula Ténedo sancti 
 Abudémii Mártyris, qui sub Diocletiáno passus est.
\switchcolumn
\selectlanguage{english}
In the island of Tenedos, St. 
 Abudemius, martyr, who suffered under Diocletian.
\switchcolumn*
\selectlanguage{latin}
Sebáste, in Arménia, sancti Antíochi médici, qui, sub Hadriáno Præside, cápite obtruncátus est; cumque ex eo lac pro sánguine manáret, Cyríacus cárnifex, convérsus ad 
 Christum, et ipse martyrium súbiit.
\switchcolumn
\selectlanguage{english}
At Sebaste in Armenia, St. 
 Antiochus, a physician, who was beheaded under the governor Adrian. On 
 seeing milk flowing from his wounds instead of blood, Cyriacus, his 
 executioner, was converted to Christ and endured martyrdom.
\switchcolumn*
\selectlanguage{latin}
Nísibi, in Mesopotámia, natális sancti Jacóbi, ejúsdem urbis Epíscopi, magnæ sanctitátis viri. 
 Hic, miráculis et eruditióne clarus, unus fuit, sub persecutióne Galérii 
 Maximiáni, ex número Confessórum, qui in Nicæna deínde Synodo perversitátem 
 Aríi, Homoúsii oppositióne, damnárunt; cujus et sancti Alexándri Epíscopi 
 oratióne ipse Aríus condígnam suæ iniquitátis mercédem, effúsis viscéribus, 
 Constantinópoli recépit.
\switchcolumn
\selectlanguage{english}
At Nisibis in Mesopotamia, the 
 birthday of St. James, bishop of that city, a man celebrated for great 
 holiness, miracles and learning. He was one of those who confessed the 
 faith during the persecution of Galerius Maximian, and later condemned the 
 perverse heresy of Arius in the Council of Nicaea by opposing to the 
 doctrine of consubstantiality. It was also owing to his prayers, and 
 those of the bishop Alexander, that Arius received at Constantinople the 
 suitable punishment of his iniquity, his bowels gushing out.
\switchcolumn*
\selectlanguage{latin}
Neápoli, in Campánia, 
 sancti Athanásii, ejúsdem civitátis Epíscopi, qui, ab ímpio nepóte Sérgio 
 multa passus ac sede pulsus, tandem Vérulis, in Hérnicis, conféctus ærúmnis, 
 migrávit in cælum, témpore Cároli Calvi.
\switchcolumn
\selectlanguage{english}
At Naples in Campania, St. 
 Athanasius, bishop of that city, who suffered a great deal from his wicked 
 nephew Sergius, by whom he was driven from his diocese. Overcome with 
 afflictions, he departed for heaven at Veroli, in the time of Charles the 
 Bald.
\switchcolumn*
\selectlanguage{latin}
Campi Salentinórum, in 
 Apúlia, sancti Pompílii Maríæ Pirrótti, Confessóris, ex Ordine Clericórum 
 Páuperum Matris Dei Scholárum Piárum, vita apostólica præclári, a Pio 
 Undécimo, Pontífice Máximo, inter Sanctos reláti.
\switchcolumn
\selectlanguage{english}
At Campo in Italy, the birthday of 
 St. Pompilio Maria Pirrotti of St. Nicholas, confessor, a member of the 
 Congregation of Poor Clerks Regular of Pious Schools of the Mother of God, 
 who spent his entire life in safeguarding the salvation of souls. He 
 was registered among the saints by Pope Pius XI.
\switchcolumn*
\selectlanguage{latin}
Panórmi Invéntio 
 córporis sanctæ Rosáliæ, Vírginis Panormitánæ; quod, Urbáno Octávo Pontífice 
 Máximo, repértum divínitus, Jubilæi anno Sicíliam a peste liberávit.
\switchcolumn
\selectlanguage{english}
At Palermo, the finding of the body 
 of St. Rosalia, virgin of that city. Miraculously discovered in the 
 time of Pope Urban VIII, it delivered Sicily from the plague in the year of 
 the Jubilee.
\switchcolumn*
\selectlanguage{latin}
\end{paracol}


% ---- martyrology/mart07/mart0716.htm
\needspace{10\baselineskip}
\begin{paracol}{2}
\selectlanguage{latin}
\begin{center}{\color{gregoriocolor} Décimo séptimo Kaléndas Augústi. 
 Luna\dots\ }\end{center}
\switchcolumn
\selectlanguage{english}
\begin{center}{\color{gregoriocolor} The 
 Sixteenth Day of 
 July. The\dots\ Day of the Moon.}\end{center}
\end{paracol}

\noindent\begin{tabularx}{\linewidth}{*{19}{>{\centering\arraybackslash}X}}
 \textcolor{gregoriocolor}{a} & \textcolor{gregoriocolor}{b} & \textcolor{gregoriocolor}{c} & \textcolor{gregoriocolor}{d} & \textcolor{gregoriocolor}{e} & \textcolor{gregoriocolor}{f} & \textcolor{gregoriocolor}{g} & \textcolor{gregoriocolor}{h} & \textcolor{gregoriocolor}{i} & \textcolor{gregoriocolor}{k} & \textcolor{gregoriocolor}{l} & \textcolor{gregoriocolor}{m} & \textcolor{gregoriocolor}{n} & \textcolor{gregoriocolor}{p} & \textcolor{gregoriocolor}{q} & \textcolor{gregoriocolor}{r} & \textcolor{gregoriocolor}{s} & \textcolor{gregoriocolor}{t} & \textcolor{gregoriocolor}{u} \\
 21 & 22 & 23 & 24 & 25 & 26 & 27 & 28 & 29 & 30 & 1 & 2 & 3 & 4 & 5 & 6 & 7 & 8 & 9 \\
\end{tabularx}
\vspace{0.5\baselineskip}
\noindent\begin{tabularx}{\linewidth}{*{12}{>{\centering\arraybackslash}X}}
 \textcolor{gregoriocolor}{A} & \textcolor{gregoriocolor}{B} & \textcolor{gregoriocolor}{C} & \textcolor{gregoriocolor}{D} & \textcolor{gregoriocolor}{E} & F & \textcolor{gregoriocolor}{F} & \textcolor{gregoriocolor}{G} & \textcolor{gregoriocolor}{H} & \textcolor{gregoriocolor}{M} & \textcolor{gregoriocolor}{N} & \textcolor{gregoriocolor}{P} \\
 10 & 11 & 12 & 13 & 14 & 15 & 15 & 16 & 17 & 18 & 19 & 20 \\
\end{tabularx}

\begin{paracol}{2}
\selectlanguage{latin}
\lettrine[lines=2]{F}{estum} beátæ Maríæ 
 Vírginis de monte Carmélo.
\switchcolumn
\selectlanguage{english}
\lettrine[lines=2]{T}{he} feast of the Blessed Virgin Mary 
 of Mount Carmel.
\switchcolumn*
\selectlanguage{latin}
Sebáste in Arménia, sanctórum Mártyrum Athenógenis Epíscopi, et decem discipulórum ejus, sub 
 Diocletiáno Imperatóre.
\switchcolumn
\selectlanguage{english}
At Sebaste in Armenia, the holy 
 martyrs Athenogenes, bishop, and ten of his disciples, in the time of 
 Emperor Diocletian.
\switchcolumn*
\selectlanguage{latin}
Tréviris sancti 
 Valentíni, Epíscopi et Mártyris.
\switchcolumn
\selectlanguage{english}
At Treves, St. Valentine, bishop and 
 martyr.
\switchcolumn*
\selectlanguage{latin}
Córdubæ, in Hispánia, 
 sancti Sisenándi, Levítæ et Mártyris; qui a Saracénis, pro Christi fide, 
 jugulátus est.
\switchcolumn
\selectlanguage{english}
At Cordova in Spain, St. Sisenand, 
 cleric and martyr, who was strangled by the Saracens for the faith of 
 Christ.
\switchcolumn*
\selectlanguage{latin}
Eódem die natális 
 sancti Fausti Mártyris, qui, sub Décio Imperatóre, cruci affíxus, quinque 
 diébus in ea vixit, ac demum, sagíttis confóssus, migrávit in cælum.
\switchcolumn
\selectlanguage{english}
The same day, the birthday of St. 
 Faustus, martyr, under Decius. He lived five days fastened to a cross, 
 and being then pierced with arrows, he went to heaven.
\switchcolumn*
\selectlanguage{latin}
Sántis, in Gállia, 
 sanctórum Mártyrum Rainéldis Vírginis, et Sociórum ejus, qui a bárbaris, pro 
 Christe fide, cæsi sunt.
\switchcolumn
\selectlanguage{english}
At Saintes in France, the holy 
 martyrs Raineld, virgin, and her companions who were slain by barbarians for 
 the Christian faith.
\switchcolumn*
\selectlanguage{latin}
Bérgomi sancti 
 Domniónis Mártyris.
\switchcolumn
\selectlanguage{english}
At Bergamo, St. Domnio, martyr.
\switchcolumn*
\selectlanguage{latin}
Antiochíæ natális beáti 
 Eustáchii, Epíscopi et Confessóris, doctrína et sanctitáte célebris; qui, 
 sub Ariáno Imperatóre Constántio, ob cathólicæ fídei defensiónem, exsílio 
 Trajanópolim Thráciæ pulsus, ibídem in Dómino requiévit.
\switchcolumn
\selectlanguage{english}
At Antioch, the birthday of blessed 
 Eustace, bishop and confessor, celebrated for learning and sanctity. 
 Under the Arian emperor Constantius, for the defence of the Catholic faith, 
 he was banished to Trajanopolis in Thrace, where he rested in the Lord.
\switchcolumn*
\selectlanguage{latin}
Cápuæ sancti Vitaliáni, 
 Epíscopi et Confessóris.
\switchcolumn
\selectlanguage{english}
At Capua, St. Vitalian, bishop and 
 confessor.
\switchcolumn*
\selectlanguage{latin}
Apud Abbatíam 
 Sanctíssimi Salvatóris, diœcésis Constantiénsis, in Gállia, sanctæ 
 Maríæ-Magdalénæ Postel, Fundatrícis Institúti Sorórum Scholárum 
 Christianárum a Misericórdia, a Pio Papa Undécimo in sanctárum Vírginum 
 album relátæ.
\switchcolumn
\selectlanguage{english}
At the abbey of our Most Holy 
 Redeemer, in the diocese of Coutances in France, St. Mary Magdalene Postel, 
 foundress of the Sisters of Mercy of the Christian Schools, who was added to 
 the list of the holy virgins by Pope Pius XI.
\switchcolumn*
\selectlanguage{latin}
Apud Ostia Tiberína 
 Translátio córporis sancti Hilaríni Mónachi, qui, una cum sancto Donáto, in 
 persecutióne Juliáni Apóstatæ, comprehénsus est, et, cum nollet sacrificáre, 
 demum, fústibus cæsus, Arétii, in Túscia, martyrium sumpsit séptimo Idus 
 Augústi.
\switchcolumn
\selectlanguage{english}
The translation of St. Hilarinus, a 
 monk, to Ostia. He was arrested with St. Donatus in the persecution of 
 Julian. Because he refused to sacrifice to idols, he was finally 
 scourged at Arezzo in Tuscany, and underwent martyrdom on the 7th of August.
\switchcolumn*
\selectlanguage{latin}
\end{paracol}


% ---- martyrology/mart07/mart0717.htm
\needspace{10\baselineskip}
\begin{paracol}{2}
\selectlanguage{latin}
\begin{center}{\color{gregoriocolor} Sextodécimo Kaléndas Augústi. 
 Luna\dots\ }\end{center}
\switchcolumn
\selectlanguage{english}
\begin{center}{\color{gregoriocolor} The 
 Seventeenth Day of 
 July. The\dots\ Day of the Moon.}\end{center}
\end{paracol}

\noindent\begin{tabularx}{\linewidth}{*{19}{>{\centering\arraybackslash}X}}
 \textcolor{gregoriocolor}{a} & \textcolor{gregoriocolor}{b} & \textcolor{gregoriocolor}{c} & \textcolor{gregoriocolor}{d} & \textcolor{gregoriocolor}{e} & \textcolor{gregoriocolor}{f} & \textcolor{gregoriocolor}{g} & \textcolor{gregoriocolor}{h} & \textcolor{gregoriocolor}{i} & \textcolor{gregoriocolor}{k} & \textcolor{gregoriocolor}{l} & \textcolor{gregoriocolor}{m} & \textcolor{gregoriocolor}{n} & \textcolor{gregoriocolor}{p} & \textcolor{gregoriocolor}{q} & \textcolor{gregoriocolor}{r} & \textcolor{gregoriocolor}{s} & \textcolor{gregoriocolor}{t} & \textcolor{gregoriocolor}{u} \\
 22 & 23 & 24 & 25 & 26 & 27 & 28 & 29 & 30 & 1 & 2 & 3 & 4 & 5 & 6 & 7 & 8 & 9 & 10 \\
\end{tabularx}
\vspace{0.5\baselineskip}
\noindent\begin{tabularx}{\linewidth}{*{12}{>{\centering\arraybackslash}X}}
 \textcolor{gregoriocolor}{A} & \textcolor{gregoriocolor}{B} & \textcolor{gregoriocolor}{C} & \textcolor{gregoriocolor}{D} & \textcolor{gregoriocolor}{E} & F & \textcolor{gregoriocolor}{F} & \textcolor{gregoriocolor}{G} & \textcolor{gregoriocolor}{H} & \textcolor{gregoriocolor}{M} & \textcolor{gregoriocolor}{N} & \textcolor{gregoriocolor}{P} \\
 11 & 12 & 13 & 14 & 15 & 16 & 16 & 17 & 18 & 19 & 20 & 21 \\
\end{tabularx}

\begin{paracol}{2}
\selectlanguage{latin}
\lettrine[lines=2]{R}{omæ} sancti Aléxii 
 Confessóris, ex Euphemiáno Senatóre progéniti. Hic, prima nocte 
 nuptiárum, sponsa intácta, e domo sua abscédens, ac, post longam 
 peregrinatiónem, ad Urbem rédiens, decem et septem annos tamquam egénus in 
 domo patérna recéptus hospítio, nova mundum arte delúdens, incógnitus mansit; 
 sed post óbitum, et voce per Urbis Ecclésias audíta et scripto suo ágnitus, 
 Innocéntio Primo Pontífice Máximo, ad sancti Bonifátii Ecclésiam summo 
 honóre delátus est, ibíque multis miráculis cláruit.
\switchcolumn
\selectlanguage{english}
\lettrine[lines=2]{A}{t} Rome, St. Alexius, confessor, son 
 of the senator Euphemian. Leaving his spouse before the night of 
 marriage, he withdrew from his house, and after a long pilgrimage, returned 
 to Rome where he was for seventeen years harboured in his father's house as 
 an unknown pauper, thus deluding the world in this strange way. After 
 his death, however, becoming known through a voice heard in the churches of 
 the city, and by his own writings, he was, under the sovereign Pontiff 
 Innocent I, translated to the Church of St. Boniface, where he wrought many 
 miracles.
\switchcolumn*
\selectlanguage{latin}
Carthágine natális 
 sanctórum Mártyrum Scillitanórum, id est Speráti, Narzális, Cythíni, Vetúrii, 
 Felícis, Acyllíni, Lætántii, Januáriæ, Generósæ, Vestínæ, Donátæ et Secúndæ; 
 qui, jussu Saturníni Præfécti, post primam Christi confessiónem, in cárcerem 
 trusi et in ligno confíxi, deínde gládio decolláti sunt. Speráti autem 
 relíquiæ, cum óssibus beáti Cypriáni et cápite sancti Pantaleónis Mártyris, 
 ex Africa in Gállias translátæ, Lugdúni, in Basílica sancti Joánnis Baptístæ, 
 religióse cónditæ fuérunt.
\switchcolumn
\selectlanguage{english}
At Carthage, the birthday of the 
 holy Scillitan martyrs Speratus, Narzales, Cythinus, Venturius, Felix, 
 Acyllinus, Laetantius, Januaria, Generosa, Vestina, Donata, and Secunda. 
 By order of the prefect Saturninus, after their first confession of the 
 faith, they were sent to prison, nailed to a cross, and finally beheaded. 
 The relics of Speratus, with the bones of blessed Cyprian and the head of 
 the martyr, St. Pantaleon, were carried from Africa into France and 
 honourably buried in the basilica of St. John the Baptist at Lyons.
\switchcolumn*
\selectlanguage{latin}
Amástride, in 
 Paphlagónia, sancti Hyacínthi Mártyris, qui, sub Castrítio Præside, multa 
 passus, quiévit in cárcere.
\switchcolumn
\selectlanguage{english}
At Amastris in Paphlagonia, St. 
 Hyacinth, martyr, who died in prison after much suffering, under the prefect 
 Castritus.
\switchcolumn*
\selectlanguage{latin}
Tíbure sancti Generósi 
 Mártyris.
\switchcolumn
\selectlanguage{english}
At Tivoli, St. Generosus, martyr.
\switchcolumn*
\selectlanguage{latin}
Constantinópoli sanctæ 
 Theódotæ Mártyris, sub Leóne Iconoclásta.
\switchcolumn
\selectlanguage{english}
At Constantinople, St. Theodota, 
 martyr, under Leo the Iconoclast.
\switchcolumn*
\selectlanguage{latin}
Romæ deposítio sancti 
 Leónis Papæ Quarti.
\switchcolumn
\selectlanguage{english}
At Rome, the death of Pope St. Leo 
 IV.
\switchcolumn*
\selectlanguage{latin}
Papíæ sancti Ennódii, 
 Epíscopi et Confessóris.
\switchcolumn
\selectlanguage{english}
At Pavia, St. Ennodius, bishop and 
 confessor.
\switchcolumn*
\selectlanguage{latin}
Antisiodóri sancti 
 Theodósii Epíscopi.
\switchcolumn
\selectlanguage{english}
At Auxerre, St. Theodosius, bishop.
\switchcolumn*
\selectlanguage{latin}
Medioláni sanctæ 
 Marcellínæ Vírginis, soróris beáti Ambrósii Epíscopi, quæ Romæ, in Basílica 
 sancti Petri, a Libério Papa velum consecratiónis accépit; cujus étiam 
 sanctitátem idem beátus Ambrósius scriptis suis testátam facit.
\switchcolumn
\selectlanguage{english}
At Milan, the virgin saint 
 Marcellina, sister of the blessed bishop Ambrose, who received the religious 
 veil from Pope Liberius, in the basilica of St. Peter at Rome. Her 
 sanctity is attested to by St. Ambrose in his writings.
\switchcolumn*
\selectlanguage{latin}
Venétiis Translátio 
 sanctæ Marínæ Vírginis.
\switchcolumn
\selectlanguage{english}
At Venice, the translation of St. 
 Marina, virgin.
\switchcolumn*
\selectlanguage{latin}
\end{paracol}


% ---- martyrology/mart07/mart0718.htm
\needspace{10\baselineskip}
\begin{paracol}{2}
\selectlanguage{latin}
\begin{center}{\color{gregoriocolor} Quintodécimo Kaléndas Augústi. 
 Luna\dots\ }\end{center}
\switchcolumn
\selectlanguage{english}
\begin{center}{\color{gregoriocolor} The 
 Eighteenth Day of 
 July. The\dots\ Day of the Moon.}\end{center}
\end{paracol}

\noindent\begin{tabularx}{\linewidth}{*{19}{>{\centering\arraybackslash}X}}
 \textcolor{gregoriocolor}{a} & \textcolor{gregoriocolor}{b} & \textcolor{gregoriocolor}{c} & \textcolor{gregoriocolor}{d} & \textcolor{gregoriocolor}{e} & \textcolor{gregoriocolor}{f} & \textcolor{gregoriocolor}{g} & \textcolor{gregoriocolor}{h} & \textcolor{gregoriocolor}{i} & \textcolor{gregoriocolor}{k} & \textcolor{gregoriocolor}{l} & \textcolor{gregoriocolor}{m} & \textcolor{gregoriocolor}{n} & \textcolor{gregoriocolor}{p} & \textcolor{gregoriocolor}{q} & \textcolor{gregoriocolor}{r} & \textcolor{gregoriocolor}{s} & \textcolor{gregoriocolor}{t} & \textcolor{gregoriocolor}{u} \\
 23 & 24 & 25 & 26 & 27 & 28 & 29 & 30 & 1 & 2 & 3 & 4 & 5 & 6 & 7 & 8 & 9 & 10 & 11 \\
\end{tabularx}
\vspace{0.5\baselineskip}
\noindent\begin{tabularx}{\linewidth}{*{12}{>{\centering\arraybackslash}X}}
 \textcolor{gregoriocolor}{A} & \textcolor{gregoriocolor}{B} & \textcolor{gregoriocolor}{C} & \textcolor{gregoriocolor}{D} & \textcolor{gregoriocolor}{E} & F & \textcolor{gregoriocolor}{F} & \textcolor{gregoriocolor}{G} & \textcolor{gregoriocolor}{H} & \textcolor{gregoriocolor}{M} & \textcolor{gregoriocolor}{N} & \textcolor{gregoriocolor}{P} \\
 12 & 13 & 14 & 15 & 16 & 17 & 17 & 18 & 19 & 20 & 21 & 22 \\
\end{tabularx}

\begin{paracol}{2}
\selectlanguage{latin}
\lettrine[lines=2]{S}{ancti} Camílli de 
 Lellis, Presbyteri et Confessóris, Clericórum Regulárium infírmis 
 ministrántium Institutóris, cæléstis hospitálium et infirmórum Patróni; 
 cujus dies natális prídie Idus Júlii recólitur.
\switchcolumn
\selectlanguage{english}
\lettrine[lines=2]{S}{t.} Camillus de Lellis, priest and 
 confessor, founder of the Clerks Regular Ministering to the Sick, the 
 heavenly patron of hospitals and of the sick, whose birthday is the 14th day 
 of July.
\switchcolumn*
\selectlanguage{latin}
Tíbure sanctæ 
 Symphorósæ, uxóris sancti Getúlii Mártyris, cum septem fíliis suis, scílicet 
 Crescénte, Juliáno, Nemésio, Primitívo, Justíno, Stácteo et Eugénio. 
 Horum mater, sub Hadriáno Príncipe, ob insuperábilem constántiam, primo cæsa 
 diu palmis, deínde crínibus suspénsa, novíssime saxo alligáta, in flumen 
 præcipitáta est; fílii autem, stipítibus ad tróchleas exténsi, divérso 
 mortis éxitu martyrium complevérunt. Eorúndem córpora póstea 
 Romam transláta, et, Pio Quarto Summo Pontífice, in Diacónia sancti Angeli 
 in Piscína fuérunt invénta.
\switchcolumn
\selectlanguage{english}
At Tivoli, in the time of Emperor 
 Hadrian, St. Symphorosa, wife of the martyr St. Getulius, with her seven 
 sons, Crescens, Julian, Nemesius, Primitivus, Justin, Stacteus, and Eugene. 
 The mother, because of her invincible constancy, was first beaten a long 
 time, then suspended by her hair, and lastly thrown into the river with a 
 stone tied to her body. Her sons were stretched by pulleys attached to 
 stakes, and completed their martyrdom in divers ways. Afterwards, 
 their bodies were taken to Rome, and in the pontificate of Pius IV, were 
 found in the sacristy of St. Angelo in Piscina.
\switchcolumn*
\selectlanguage{latin}
Trajécti sancti 
 Frideríci, Epíscopi et Mártyris.
\switchcolumn
\selectlanguage{english}
At Utrecht, St. Frederick, bishop 
 and martyr.
\switchcolumn*
\selectlanguage{latin}
Doróstori, in Mysia 
 inferióre, sancti Æmiliáni Mártyris, qui, témpore Juliáni Apóstatæ, sub 
 Capitolíno Præside, in fornácem injéctus, martyrii palmam accépit.
\switchcolumn
\selectlanguage{english}
At Silistria in Bulgaria, St. 
 Emilian, martyr, who was cast into a furnace, in the time of Julian the 
 Apostate, under the governor Capitolinus, and received the palm of 
 martyrdom.
\switchcolumn*
\selectlanguage{latin}
Carthágine sanctæ 
 Gundénis Vírginis, quæ, jussu Rufíni Procónsulis, ob Christi confessiónem, 
 quater divérsis tempóribus per extensiónem in equúleo torta, et úngulis 
 horrénde lacerántibus cruciáta, cárceris squalóre diu afflícta, novíssime 
 gládio cæsa est.
\switchcolumn
\selectlanguage{english}
At Carthage, St. Gundenes, virgin. 
 By order of the proconsul Ruffinus, she was at four different times 
 stretched on the rack for the faith of Christ, horribly lacerated with iron 
 hooks, confined for a long time in a filthy prison, and finally put to the 
 sword.
\switchcolumn*
\selectlanguage{latin}
Gallǽciæ, in Hispánia, 
 sanctæ Marínæ, Vírginis et Mártyris.
\switchcolumn
\selectlanguage{english}
In Spanish Galicia, St. Marina, 
 virgin and martyr.
\switchcolumn*
\selectlanguage{latin}
Medioláni sancti 
 Matérni Epíscopi, qui, sub Maximiáni Imperatóre, pro fide Christi et pro 
 Ecclésia sibi commíssa, in cárcerem detrúsus et sæpe verbéribus cæsus est; 
 ac tandem, multis confessiónibus clarus, obdormívit in Dómino.
\switchcolumn
\selectlanguage{english}
At Milan, in the reign of Maximian, 
 the holy bishop Maternus. For the faith of Christ and the Church 
 entrusted to him, he went to his rest in the Lord with a great renown for 
 his repeated confession of the faith.
\switchcolumn*
\selectlanguage{latin}
Bríxiæ natális sancti 
 Philástrii, qui fuit ejúsdem civitátis Epíscopus. Hic advérsus 
 hæréticos, præsértim Ariános, a quibus multa passus est, plúrimum verbis 
 scriptísque pugnávit; demum, clarus miráculis, Conféssor in pace quiévit.
\switchcolumn
\selectlanguage{english}
At Brescia, the birthday of St. 
 Philastrius, bishop of that city, who both by word and writing opposed the 
 heretics, especially the Arians, from whom he suffered greatly. 
 Finally he died in peace, a confessor renowned for miracles.
\switchcolumn*
\selectlanguage{latin}
Metis, in Gállia, 
 sancti Arnúlfi Epíscopi, qui, sanctitáte et miráculis illústris, eremíticam 
 delégit vitam, et beáto fine quiévit.
\switchcolumn
\selectlanguage{english}
At Metz in France, St. Arnulf, a 
 bishop illustrious for holiness and miracles. He chose the life of a 
 hermit and ended his blessed career in peace.
\switchcolumn*
\selectlanguage{latin}
Sígniæ sancti Brunónis, 
 Epíscopi et Confessóris.
\switchcolumn
\selectlanguage{english}
At Segni, St. Bruno, bishop and 
 confessor.
\switchcolumn*
\selectlanguage{latin}
Foro Pompílii, in 
 Æmília, sancti Ruffilli, ejúsdem civitátis Epíscopi.
\switchcolumn
\selectlanguage{english}
At Forlimpopoli in Emilia, St. 
 Ruffillus, bishop of that city.
\switchcolumn*
\selectlanguage{latin}
\end{paracol}


% ---- martyrology/mart07/mart0719.htm
\needspace{10\baselineskip}
\begin{paracol}{2}
\selectlanguage{latin}
\begin{center}{\color{gregoriocolor} Quartodécimo Kaléndas Augústi. Luna\dots\ }\end{center}
\switchcolumn
\selectlanguage{english}
\begin{center}{\color{gregoriocolor} The 
 Nineteenth Day of 
 July. The\dots\ Day of the Moon.}\end{center}
\end{paracol}

\noindent\begin{tabularx}{\linewidth}{*{19}{>{\centering\arraybackslash}X}}
 \textcolor{gregoriocolor}{a} & \textcolor{gregoriocolor}{b} & \textcolor{gregoriocolor}{c} & \textcolor{gregoriocolor}{d} & \textcolor{gregoriocolor}{e} & \textcolor{gregoriocolor}{f} & \textcolor{gregoriocolor}{g} & \textcolor{gregoriocolor}{h} & \textcolor{gregoriocolor}{i} & \textcolor{gregoriocolor}{k} & \textcolor{gregoriocolor}{l} & \textcolor{gregoriocolor}{m} & \textcolor{gregoriocolor}{n} & \textcolor{gregoriocolor}{p} & \textcolor{gregoriocolor}{q} & \textcolor{gregoriocolor}{r} & \textcolor{gregoriocolor}{s} & \textcolor{gregoriocolor}{t} & \textcolor{gregoriocolor}{u} \\
 24 & 25 & 26 & 27 & 28 & 29 & 30 & 1 & 2 & 3 & 4 & 5 & 6 & 7 & 8 & 9 & 10 & 11 & 12 \\
\end{tabularx}
\vspace{0.5\baselineskip}
\noindent\begin{tabularx}{\linewidth}{*{12}{>{\centering\arraybackslash}X}}
 \textcolor{gregoriocolor}{A} & \textcolor{gregoriocolor}{B} & \textcolor{gregoriocolor}{C} & \textcolor{gregoriocolor}{D} & \textcolor{gregoriocolor}{E} & F & \textcolor{gregoriocolor}{F} & \textcolor{gregoriocolor}{G} & \textcolor{gregoriocolor}{H} & \textcolor{gregoriocolor}{M} & \textcolor{gregoriocolor}{N} & \textcolor{gregoriocolor}{P} \\
 13 & 14 & 15 & 16 & 17 & 18 & 18 & 19 & 20 & 21 & 22 & 23 \\
\end{tabularx}

\begin{paracol}{2}
\selectlanguage{latin}
\lettrine[lines=2]{S}{ancti} Vincénti a 
 Paulo, Presbyteri et Confessóris, Congregatiónis Presbyterórum Missiónis et 
 Puellárum Caritátis Fundatóris, cæléstis ómnium caritátis Societátum Patróni; 
 qui in Dómino obdormívit quinto Kaléndas Octóbris.
\switchcolumn
\selectlanguage{english}
\lettrine[lines=2]{S}{t.} Vincent de Paul, priest and 
 confessor, founder of the priests of the Congregation of the Mission and the 
 Sisters of Charity, the heavenly patron of all charitable organizations. 
 He fell asleep in the Lord on the 27th of September.
\switchcolumn*
\selectlanguage{latin}
Colóssis, in Phrygia, 
 natális sancti Epaphræ, quem sanctus Paulus Apóstolus concaptívum appéllat. 
 Hic, ab eódem Apóstolo Colóssis Epíscopus ordinátus, ibídem, clarus 
 virtútibus, martyrii palmam, pro óvibus sibi commendátis, viríli agóne 
 percépit; cujus corpus Romæ, in Basílica sanctæ Maríæ Majóris, cónditum est.
\switchcolumn
\selectlanguage{english}
At Colossae in Phrygia, the birthday 
 of St. Epaphras, whom the apostle St. Paul calls his fellow-prisoner. 
 By the same apostle he was consecrated bishop of Colossae, where, becoming 
 renowned for his virtues, he received the palm of martyrdom for defending 
 courageously the flock committed to his charge. His body lies at Rome 
 in the basilica of St. Mary Major.
\switchcolumn*
\selectlanguage{latin}
Tréviris sancti 
 Martíni, Epíscopi et Mártyris.
\switchcolumn
\selectlanguage{english}
At Treves, St. Martin, bishop and 
 martyr.
\switchcolumn*
\selectlanguage{latin}
Híspali, in Hispánia, 
 pássio sanctárum Vírginum Justæ et Rufínæ, quæ, a Præside Diogeniáno 
 comprehénsæ, primo equúlei extensióne et ungulárum laniatióne vexátæ, póstea cárcere, inédia et váriis torsiónibus sunt afflíctæ; tandem Justa in cárcere 
 spíritum exhalávit, Rufínæ vero cervix, in confessióne Dómini, confracta 
 est.
\switchcolumn
\selectlanguage{english}
At Seville in Spain, the martyrdom 
 of the holy virgins Justa and Rufina. Arrested by the governor 
 Diogenian, they were stretched on the rack and lacerated with iron claws, 
 then imprisoned and subjected to starvation and various tortures. 
 Justa died in prison, but Rufina's neck was broken for the confession of the 
 Lord.
\switchcolumn*
\selectlanguage{latin}
Córdubæ, in Hispánia, 
 sanctæ Aureæ Vírginis, beatórum Adúlfi et Joánnis Mártyrum soróris; quæ 
 aliquándo in apostasíæ crimen a Mahumetáno Júdice indúcta est, sed mox, 
 facti pænitens, iteráto certámine, hostem effúso sánguine superávit.
\switchcolumn
\selectlanguage{english}
At Cordova in Spain, St. Aura, 
 virgin, the sister of the holy martyrs Adulphus and John. A Mohammedan 
 judge had persuaded her to apostatize for a while, but quickly repenting of 
 what she had done, in the second trial overcame the enemy by the shedding of 
 her blood.
\switchcolumn*
\selectlanguage{latin}
Romæ sancti Symmachi 
 Papæ, qui, a schismaticórum factióne diútius fatigátus, demum, sanctitáte 
 conspícuus, migrávit ad Dóminum.
\switchcolumn
\selectlanguage{english}
At Rome, Pope St. Symmachus, who for 
 a long time had much to bear, from a faction of schismatics. At last, 
 distinguished by holiness, he went to God.
\switchcolumn*
\selectlanguage{latin}
Verónæ sancti Felícis 
 Epíscopi.
\switchcolumn
\selectlanguage{english}
At Verona, St. Felix, bishop.
\switchcolumn*
\selectlanguage{latin}
Apud Scetim, Ægypti 
 montem, sancti Arsénii, Románæ Ecclésiæ Diáconi; qui, Theodósii témpore, in 
 solitúdinem secéssit, ibíque, virtútibus ómnibus consummátus et jugi 
 lacrimárum imbre perfúsus, spíritum Deo réddidit.
\switchcolumn
\selectlanguage{english}
At Scete, a mountain in Egypt, St. 
 Arsenius, a deacon of the Roman Church. In the time of Theodosius he 
 retired into a desert where, endowed with every virtue and shedding 
 continual tears, he yielded his soul unto God.
\switchcolumn*
\selectlanguage{latin}
In Cappadócia sanctæ 
 Macrínæ Vírginis, fíliæ sanctórum Basilíi et Emméliæ, atque soróris item 
 sanctórum Episcopórum Basilíi Magni, Gregórii Nysséni et Petri Sebastiénsis.
\switchcolumn
\selectlanguage{english}
In Cappadocia, St. Macrina, virgin. 
 She was the daughter of Saints Basil and Emmelia, and the sister of the holy 
 bishops, St. Basil the Great, St. Gregory of Nyssa, and St. Peter of Sebaste.
\switchcolumn*
\selectlanguage{latin}
\end{paracol}


% ---- martyrology/mart07/mart0720.htm
\needspace{10\baselineskip}
\begin{paracol}{2}
\selectlanguage{latin}
\begin{center}{\color{gregoriocolor} Tertiodécimo Kaléndas Augústi. 
 Luna\dots\ }\end{center}
\switchcolumn
\selectlanguage{english}
\begin{center}{\color{gregoriocolor} The 
 Twentieth Day of 
 July. The\dots\ Day of the Moon.}\end{center}
\end{paracol}

\noindent\begin{tabularx}{\linewidth}{*{19}{>{\centering\arraybackslash}X}}
 \textcolor{gregoriocolor}{a} & \textcolor{gregoriocolor}{b} & \textcolor{gregoriocolor}{c} & \textcolor{gregoriocolor}{d} & \textcolor{gregoriocolor}{e} & \textcolor{gregoriocolor}{f} & \textcolor{gregoriocolor}{g} & \textcolor{gregoriocolor}{h} & \textcolor{gregoriocolor}{i} & \textcolor{gregoriocolor}{k} & \textcolor{gregoriocolor}{l} & \textcolor{gregoriocolor}{m} & \textcolor{gregoriocolor}{n} & \textcolor{gregoriocolor}{p} & \textcolor{gregoriocolor}{q} & \textcolor{gregoriocolor}{r} & \textcolor{gregoriocolor}{s} & \textcolor{gregoriocolor}{t} & \textcolor{gregoriocolor}{u} \\
 25 & 26 & 27 & 28 & 29 & 30 & 1 & 2 & 3 & 4 & 5 & 6 & 7 & 8 & 9 & 10 & 11 & 12 & 13 \\
\end{tabularx}
\vspace{0.5\baselineskip}
\noindent\begin{tabularx}{\linewidth}{*{12}{>{\centering\arraybackslash}X}}
 \textcolor{gregoriocolor}{A} & \textcolor{gregoriocolor}{B} & \textcolor{gregoriocolor}{C} & \textcolor{gregoriocolor}{D} & \textcolor{gregoriocolor}{E} & F & \textcolor{gregoriocolor}{F} & \textcolor{gregoriocolor}{G} & \textcolor{gregoriocolor}{H} & \textcolor{gregoriocolor}{M} & \textcolor{gregoriocolor}{N} & \textcolor{gregoriocolor}{P} \\
 14 & 15 & 16 & 17 & 18 & 19 & 19 & 20 & 21 & 22 & 23 & 24 \\
\end{tabularx}

\begin{paracol}{2}
\selectlanguage{latin}
\lettrine[lines=2]{S}{ancti} Hierónymi 
 Æmiliáni Confessóris, Congregatiónis Somáschæ Institutóris, cæléstis ómnium 
 orphanórum ac derelíctæ juventútis Patróni; qui sexto Idus Februárii 
 obdormívit in Dómino.
\switchcolumn
\selectlanguage{english}
\lettrine[lines=2]{S}{t.} Jerome Emiliani, confessor, 
 founder of the Congregation of Somascha, the heavenly patron of all orphans 
 and destitute children. He fell asleep in the Lord on the 8th of 
 February.
\switchcolumn*
\selectlanguage{latin}
Antiochíæ pássio sanctæ 
 Margarítæ, Vírginis et Mártyris.
\switchcolumn
\selectlanguage{english}
At Antioch, the passion of St. 
 Margaret, virgin and martyr.
\switchcolumn*
\selectlanguage{latin}
In monte Carmélo sancti 
 Elíæ Prophétæ.
\switchcolumn
\selectlanguage{english}
On Mount Carmel, the holy prophet 
 Elijah.
\switchcolumn*
\selectlanguage{latin}
In Judæa natális beáti 
 Joseph, qui cognominátus est Justus, quem Apóstoli cum beáto Matthía 
 statuérunt ut locum apostolátus Judæ proditóris impléret; sed, cum sors 
 cecidísset, super Matthíam, ipse, nihilóminus prædicatiónis et sanctitátis 
 offício insérviens, multámque pro fide Christi a Judæis persecutiónem 
 sústinens, victorióso fine quiévit. De quo étiam refértur quod venénum 
 bíberit, et nihil ex hoc triste propter Dómini fidem pertúlerit.
\switchcolumn
\selectlanguage{english}
In Judea, the birthday of blessed 
 Joseph, surnamed the Just, whom the apostles selected with blessed Matthias 
 for the apostleship to replace the traitor Judas. The lot having 
 fallen upon Matthias, Joseph, notwithstanding, continued to preach and to 
 advance in virtue, and after having sustained from the Jews many 
 persecutions for the faith of Christ, he happily completed his life. 
 It is related of him that having drunk poison, he received no injury from 
 it, because of his confidence in the Lord.
\switchcolumn*
\selectlanguage{latin}
Córdubæ, in Hispánia, 
 sancti Pauli, Diáconi et Mártyris; qui, cum infidéles príncipes argúeret 
 Mahuméticæ impietátis ac sævítiæ, et Christum constantíssime prædicáret, 
 idcírco, eórum jussu necátus, ad præmia evolávit in cælum.
\switchcolumn
\selectlanguage{english}
At Cordova in Spain, St. Paul, 
 deacon and martyr. For rebuking Mohammedan princes for their impiety 
 and cruelty, and preaching Christ with constancy, he was put to death and 
 went to his reward in heaven.
\switchcolumn*
\selectlanguage{latin}
Damásci sanctórum 
 Mártyrum Sabíni, Juliáni, Máximi, Macróbii, Cássiæ et Paulæ, cum áliis decem.
\switchcolumn
\selectlanguage{english}
At Damascus, the holy martyrs 
 Sabinus, Julian, Maximus, Macrobius, Cassia, and Paul, with ten others.
\switchcolumn*
\selectlanguage{latin}
In Lusitánia sanctæ 
 Wilgefórtis, Vírginis et Mártyris; quæ, pro Christiána fide ac pudicítia 
 decértans, in cruce méruit gloriósum obtinére triúmphum.
\switchcolumn
\selectlanguage{english}
In Portugal, St. Wilgefortis, virgin 
 and martyr, who merited the crown of martyrdom on a cross in defence of the 
 faith and her chastity.
\switchcolumn*
\selectlanguage{latin}
Eódem die natális 
 sanctórum Flaviáni Secúndi, Epíscopi Antiochéni, et Elíæ, Epíscopi 
 Hierosolymitáni; qui, pro Synodo Chalcedonénsi ab Anastásio Imperatóre ambo 
 in exsílium acti, victóres migrárunt ad Dóminum.
\switchcolumn
\selectlanguage{english}
The same day, the birthday of St. 
 Flavian II, bishop of Antioch, and St. Elias, bishop of Jerusalem. 
 They were driven into exile by Emperor Anastasius for their defence of the 
 Council of Chalcedon, and there they went victoriously to the Lord.
\switchcolumn*
\selectlanguage{latin}
In pago Bononiénsi, in 
 Gállia, sancti Vulmári Abbátis, admirándæ sanctitátis viri.
\switchcolumn
\selectlanguage{english}
At Boulogne in France, the abbot St. 
 Wulmar, a man of admirable sanctity.
\switchcolumn*
\selectlanguage{latin}
Tréviris sanctæ Sevéræ 
 Vírginis.
\switchcolumn
\selectlanguage{english}
At Treves, St. Severa, virgin.
\switchcolumn*
\selectlanguage{latin}
\end{paracol}


% ---- martyrology/mart07/mart0721.htm
\needspace{10\baselineskip}
\begin{paracol}{2}
\selectlanguage{latin}
\begin{center}{\color{gregoriocolor} Duodécimo Kaléndas Augústi. 
 Luna\dots\ }\end{center}
\switchcolumn
\selectlanguage{english}
\begin{center}{\color{gregoriocolor} The 
 Twenty-First Day of 
 July. The\dots\ Day of the Moon.}\end{center}
\end{paracol}

\noindent\begin{tabularx}{\linewidth}{*{19}{>{\centering\arraybackslash}X}}
 \textcolor{gregoriocolor}{a} & \textcolor{gregoriocolor}{b} & \textcolor{gregoriocolor}{c} & \textcolor{gregoriocolor}{d} & \textcolor{gregoriocolor}{e} & \textcolor{gregoriocolor}{f} & \textcolor{gregoriocolor}{g} & \textcolor{gregoriocolor}{h} & \textcolor{gregoriocolor}{i} & \textcolor{gregoriocolor}{k} & \textcolor{gregoriocolor}{l} & \textcolor{gregoriocolor}{m} & \textcolor{gregoriocolor}{n} & \textcolor{gregoriocolor}{p} & \textcolor{gregoriocolor}{q} & \textcolor{gregoriocolor}{r} & \textcolor{gregoriocolor}{s} & \textcolor{gregoriocolor}{t} & \textcolor{gregoriocolor}{u} \\
 26 & 27 & 28 & 29 & 30 & 1 & 2 & 3 & 4 & 5 & 6 & 7 & 8 & 9 & 10 & 11 & 12 & 13 & 14 \\
\end{tabularx}
\vspace{0.5\baselineskip}
\noindent\begin{tabularx}{\linewidth}{*{12}{>{\centering\arraybackslash}X}}
 \textcolor{gregoriocolor}{A} & \textcolor{gregoriocolor}{B} & \textcolor{gregoriocolor}{C} & \textcolor{gregoriocolor}{D} & \textcolor{gregoriocolor}{E} & F & \textcolor{gregoriocolor}{F} & \textcolor{gregoriocolor}{G} & \textcolor{gregoriocolor}{H} & \textcolor{gregoriocolor}{M} & \textcolor{gregoriocolor}{N} & \textcolor{gregoriocolor}{P} \\
 15 & 16 & 17 & 18 & 19 & 20 & 20 & 21 & 22 & 23 & 24 & 25 \\
\end{tabularx}

\begin{paracol}{2}
\selectlanguage{latin}
\lettrine[lines=2]{R}{omæ} sanctæ Praxédis 
 Vírginis, quæ, in omni castitáte et lege divína cum esset erudíta, vigíliis 
 et oratiónibus atque jejúniis assídue vacans, quiévit in Christo, sepúltaque est, juxta sorórem suam Pudentiánam, via Salária.
\switchcolumn
\selectlanguage{english}
\lettrine[lines=2]{A}{t} Rome, the holy virgin Praxedes, 
 who was brought up in all chastity and in the knowledge of the divine law. 
 Diligently attending to watching, prayer, and fasting, she rested in Christ, 
 and was buried near her sister Pudentiana on the Salarian Way.
\switchcolumn*
\selectlanguage{latin}
Babylóne sancti 
 Daniélis Prophétæ.
\switchcolumn
\selectlanguage{english}
At Babylon, the holy prophet Daniel.
\switchcolumn*
\selectlanguage{latin}
Cománæ, in Arménia, sancti Zótici, Epíscopi et Mártyris; qui sub Sevéro coronátus est.
\switchcolumn
\selectlanguage{english}
At Comana in Armenia, the holy 
 bishop and martyr Zoticus, who was crowned under Severus.
\switchcolumn*
\selectlanguage{latin}
Massíliæ, in Gállia, 
 natális sancti Victóris, qui, cum esset miles, et nec militáre neque idólis 
 sacrificáre vellet, hinc, primo in cárcerem trusus ibíque ab Angelo 
 visitátus, deínde váriis cruciátibus punítus, novíssime, contrítus in mola 
 pistória, martyrium consummávit. Passi sunt cum ipso et tres mílites, 
 id est Alexánder, Feliciánus et Longínus.
\switchcolumn
\selectlanguage{english}
At Marseilles in France, the 
 birthday of St. Victor, a soldier. Because he refused to serve in the 
 army and sacrifice to idols, he was thrust into prison, where he was visited 
 by an angel. He was subjected to various torments, and finally being 
 crushed under a millstone, he ended his martyrdom. With him also 
 suffered three soldiers, Alexander, Felician, and Longinus.
\switchcolumn*
\selectlanguage{latin}
Trecis, in Gállia, 
 pássio sanctórum Cláudii, Justi, Jucundíni et Sociórum quinque, sub 
 Aureliáno Imperatóre.
\switchcolumn
\selectlanguage{english}
At Troyes in France, the martyrdom 
 of the saints Claudius, Justus, Jucundinus, and five companions, in the time 
 of Emperor Aurelian.
\switchcolumn*
\selectlanguage{latin}
Ibídem sanctæ Júliæ, 
 Vírginis et Mártyris.
\switchcolumn
\selectlanguage{english}
In the same place, St. Julia, virgin 
 and martyr.
\switchcolumn*
\selectlanguage{latin}
Argentoráti sancti 
 Arbogásti Epíscopi, miráculis clari.
\switchcolumn
\selectlanguage{english}
At Strasbourg, St. Arbogast, a 
 bishop, renowned for miracles.
\switchcolumn*
\selectlanguage{latin}
In Syria sancti Joánnis 
 Mónachi, qui éxstitit colléga sancti Simeónis.
\switchcolumn
\selectlanguage{english}
In Syria, the holy monk John, a 
 companion of St. Simeon.
\switchcolumn*
\selectlanguage{latin}
\end{paracol}


% ---- martyrology/mart07/mart0722.htm
\needspace{10\baselineskip}
\begin{paracol}{2}
\selectlanguage{latin}
\begin{center}{\color{gregoriocolor} Undécimo Kaléndas Augústi. 
 Luna\dots\ }\end{center}
\switchcolumn
\selectlanguage{english}
\begin{center}{\color{gregoriocolor} The 
 Twenty-Second Day of 
 July. The\dots\ Day of the Moon.}\end{center}
\end{paracol}

\noindent\begin{tabularx}{\linewidth}{*{19}{>{\centering\arraybackslash}X}}
 \textcolor{gregoriocolor}{a} & \textcolor{gregoriocolor}{b} & \textcolor{gregoriocolor}{c} & \textcolor{gregoriocolor}{d} & \textcolor{gregoriocolor}{e} & \textcolor{gregoriocolor}{f} & \textcolor{gregoriocolor}{g} & \textcolor{gregoriocolor}{h} & \textcolor{gregoriocolor}{i} & \textcolor{gregoriocolor}{k} & \textcolor{gregoriocolor}{l} & \textcolor{gregoriocolor}{m} & \textcolor{gregoriocolor}{n} & \textcolor{gregoriocolor}{p} & \textcolor{gregoriocolor}{q} & \textcolor{gregoriocolor}{r} & \textcolor{gregoriocolor}{s} & \textcolor{gregoriocolor}{t} & \textcolor{gregoriocolor}{u} \\
 27 & 28 & 29 & 30 & 1 & 2 & 3 & 4 & 5 & 6 & 7 & 8 & 9 & 10 & 11 & 12 & 13 & 14 & 15 \\
\end{tabularx}
\vspace{0.5\baselineskip}
\noindent\begin{tabularx}{\linewidth}{*{12}{>{\centering\arraybackslash}X}}
 \textcolor{gregoriocolor}{A} & \textcolor{gregoriocolor}{B} & \textcolor{gregoriocolor}{C} & \textcolor{gregoriocolor}{D} & \textcolor{gregoriocolor}{E} & F & \textcolor{gregoriocolor}{F} & \textcolor{gregoriocolor}{G} & \textcolor{gregoriocolor}{H} & \textcolor{gregoriocolor}{M} & \textcolor{gregoriocolor}{N} & \textcolor{gregoriocolor}{P} \\
 16 & 17 & 18 & 19 & 20 & 21 & 21 & 22 & 23 & 24 & 25 & 26 \\
\end{tabularx}

\begin{paracol}{2}
\selectlanguage{latin}
\lettrine[lines=2]{A}{pud} Massíliam, in 
 Gállia, natális sanctæ Maríæ Magdalénæ, de qua Dóminus ejécit septem dæmónia, 
 et quæ ipsum Salvatórem a mórtuis resurgéntem prima vidére méruit.
\switchcolumn
\selectlanguage{english}
\lettrine[lines=2]{A}{t} Marseilles in France, the 
 birthday of St. Mary Magdalene, out of whom our Lord expelled seven demons, 
 and who deserved to be the first to see the Saviour after he had risen from 
 the dead.
\switchcolumn*
\selectlanguage{latin}
Philíppis, in 
 Macedónia, sanctæ Syntyches, cujus méminit beátus Paulus Apóstolus.
\switchcolumn
\selectlanguage{english}
At Philippi in Macedonia, St. 
 Syntyche, mentioned by the blessed apostle Paul.
\switchcolumn*
\selectlanguage{latin}
Ancyræ, in Galátia, natális sancti Platónis Mártyris, qui, sub Agrippíno Vicário, verbéribus 
 cæsus, uncis laniátus férreis, aliísque immaníssimis tormentórum genéribus 
 cruciátus, demum, abscísso cápite, invíctam ánimam Deo réddidit. 
 Ipsíus vero mirácula in subveniéndis captívis, Acta secúndæ Synodi Nicǽnæ 
 testántur.
\switchcolumn
\selectlanguage{english}
At Ancyra in Galatia, the birthday 
 of the martyr St. Plato. Under the lieutenant-governor Agrippinus, he 
 was scourged, lacerated with iron hooks, and subjected to the most atrocious 
 torments, and finally being beheaded, he rendered his invincible soul to 
 God. The Acts of the Second Council of Nicaea bear witness to his 
 miracles in helping captives.
\switchcolumn*
\selectlanguage{latin}
In Cypro sancti 
 Theóphili Prætóris, qui ab Arábibus tentus, et, cum nec donis nec minis 
 flecti posset ut Christum negáret, gládio cæsus est.
\switchcolumn
\selectlanguage{english}
In Cyprus, St. Theophilus, a 
 praetor, who was apprehended by the Arabs, and as he could not be induced 
 either by gifts or by threats to deny Christ, was put to the sword.
\switchcolumn*
\selectlanguage{latin}
Antiochíæ sancti 
 Cyrílli Epíscopi, doctrína et sanctitáte conspícui.
\switchcolumn
\selectlanguage{english}
At Antioch, the holy bishop Cyril, 
 who was distinguished for learning and holiness.
\switchcolumn*
\selectlanguage{latin}
Menáti, in território 
 Arvernénsi, sancti Meneléi Abbátis.
\switchcolumn
\selectlanguage{english}
At Menat, in the territory of 
 Auvergne, St. Meneleus, abbot.
\switchcolumn*
\selectlanguage{latin}
In monastério 
 Fontanéllæ, in Gállia, sancti Wandregísili Abbátis, miráculis clari; cujus 
 corpus ad Blandínum monastérium, in Flándria, póstea delátum fuit.
\switchcolumn
\selectlanguage{english}
In the monastery of Fontanelle in 
 France, Abbot St. Wandrille, famous for his miracles. His body was 
 afterwards translated to the monastery of Blandin, in Flanders.
\switchcolumn*
\selectlanguage{latin}
Ulyssipóne, in 
 Lusitánia, sancti Lauréntii a Brundísio, Sacerdótis et Confessóris; qui 
 Ordinis Minórum sancti Francísci Capuccinórum Miníster éxstitit Generális, 
 atque, divíni verbi prædicatióne et árduis pro Dei glória gestis præclárus, 
 a Leóne Décimo tértio, Summo Pontífice, Sanctórum fastis adscríptus est.
\switchcolumn
\selectlanguage{english}
At Lisbon in Portugal, St. Lawrence 
 of Brindisi, priest and confessor, superior general of the Order of Friars 
 Minor Capuchin of St. Francis. Illustrious for his preaching and his 
 arduous labour for the glory of God, he was canonized by Pope Leo XIII.
\switchcolumn*
\selectlanguage{latin}
Scythópoli, in 
 Palæstina, sancti Joséphi Cómitis.
\switchcolumn
\selectlanguage{english}
At Scythopolis in Palestine, St. 
 Joseph, a count.
\switchcolumn*
\selectlanguage{latin}
\end{paracol}


% ---- martyrology/mart07/mart0723.htm
\needspace{10\baselineskip}
\begin{paracol}{2}
\selectlanguage{latin}
\begin{center}{\color{gregoriocolor} Décimo Kaléndas Augústi. 
 Luna\dots\ }\end{center}
\switchcolumn
\selectlanguage{english}
\begin{center}{\color{gregoriocolor} The 
 Twenty-Third Day of 
 July. The\dots\ Day of the Moon.}\end{center}
\end{paracol}

\noindent\begin{tabularx}{\linewidth}{*{19}{>{\centering\arraybackslash}X}}
 \textcolor{gregoriocolor}{a} & \textcolor{gregoriocolor}{b} & \textcolor{gregoriocolor}{c} & \textcolor{gregoriocolor}{d} & \textcolor{gregoriocolor}{e} & \textcolor{gregoriocolor}{f} & \textcolor{gregoriocolor}{g} & \textcolor{gregoriocolor}{h} & \textcolor{gregoriocolor}{i} & \textcolor{gregoriocolor}{k} & \textcolor{gregoriocolor}{l} & \textcolor{gregoriocolor}{m} & \textcolor{gregoriocolor}{n} & \textcolor{gregoriocolor}{p} & \textcolor{gregoriocolor}{q} & \textcolor{gregoriocolor}{r} & \textcolor{gregoriocolor}{s} & \textcolor{gregoriocolor}{t} & \textcolor{gregoriocolor}{u} \\
 28 & 29 & 30 & 1 & 2 & 3 & 4 & 5 & 6 & 7 & 8 & 9 & 10 & 11 & 12 & 13 & 14 & 15 & 16 \\
\end{tabularx}
\vspace{0.5\baselineskip}
\noindent\begin{tabularx}{\linewidth}{*{12}{>{\centering\arraybackslash}X}}
 \textcolor{gregoriocolor}{A} & \textcolor{gregoriocolor}{B} & \textcolor{gregoriocolor}{C} & \textcolor{gregoriocolor}{D} & \textcolor{gregoriocolor}{E} & F & \textcolor{gregoriocolor}{F} & \textcolor{gregoriocolor}{G} & \textcolor{gregoriocolor}{H} & \textcolor{gregoriocolor}{M} & \textcolor{gregoriocolor}{N} & \textcolor{gregoriocolor}{P} \\
 17 & 18 & 19 & 20 & 21 & 22 & 22 & 23 & 24 & 25 & 26 & 27 \\
\end{tabularx}

\begin{paracol}{2}
\selectlanguage{latin}
\lettrine[lines=2]{R}{avénnæ} natális sancti 
 Apollináris Epíscopi, qui, ab Apóstolo Petro Romæ ordinátus et Ravénnam 
 missus, pro fide Christi divérsas et multíplices pœnas perpéssus est; póstea, 
 Evangélium in Æmília prædicans, plúrimos ab idolórum cultu revocávit; 
 tandem, Ravénnam revérsus, gloriósum martyrium, sub Vespasiáno Cæsare, 
 complévit.
\switchcolumn
\selectlanguage{english}
\lettrine[lines=2]{A}{t} Ravenna, the birthday of the holy 
 bishop Apollinaris, who was consecrated at Rome by the Apostle Peter, and 
 sent to Ravenna, where he endured many different tribulations for the faith 
 of Christ. He afterwards preached the Gospel in Emilia, where he 
 converted many from the worship of idols. Finally, returning to 
 Ravenna, he completed his confession of Christ by a glorious martyrdom under 
 Vespasian Caesar.
\switchcolumn*
\selectlanguage{latin}
Cenómanis, in Gállia, 
 sancti Libórii, Epíscopi et Confessóris.
\switchcolumn
\selectlanguage{english}
At Le Mans in France, St. Liborius, 
 bishop and confessor.
\switchcolumn*
\selectlanguage{latin}
Romæ natális sanctæ 
 Birgíttæ Víduæ, quæ, post multas sanctórum locórum peregrinatiónes, divíno 
 affláta Spíritu, quiévit. Ipsíus autem festívitas octávo Idus Octóbris 
 celebrátur.
\switchcolumn
\selectlanguage{english}
At Rome, St. Bridget, widow, who, 
 after many pilgrimages to the holy places, fell asleep filled with the 
 Spirit of God. Her feast is observed on the 8th of October.
\switchcolumn*
\selectlanguage{latin}
Ibídem sancti Rásyphi 
 Mártyris.
\switchcolumn
\selectlanguage{english}
Also, St. Rasyphus, martyr.
\switchcolumn*
\selectlanguage{latin}
Item Romæ pássio sanctæ 
 Primitívæ, Vírginis et Mártyris.
\switchcolumn
\selectlanguage{english}
In the same city, the martyrdom of 
 St. Primitiva, virgin and martyr.
\switchcolumn*
\selectlanguage{latin}
Item sanctórum Mártyrum 
 Apollónii et Eugénii.
\switchcolumn
\selectlanguage{english}
Also the holy martyrs Apollonius and 
 Eugene.
\switchcolumn*
\selectlanguage{latin}
Eódem die natális 
 sanctórum Mártyrum Tróphimi et Theóphili, qui, sub Diocletiáno Imperatóre, 
 cæsi lapídibus et igne incénsi, demum, gládio percússi, martyrio coronántur.
\switchcolumn
\selectlanguage{english}
The same day, the birthday of the 
 holy martyrs Trophimus and Theophilus, who received their crown of martyrdom 
 by being beaten with stones, scorched with fire, and finally struck with the 
 sword, in the time of Emperor Diocletian.
\switchcolumn*
\selectlanguage{latin}
In Bulgária sanctórum 
 plurimórum Mártyrum, quos ímpius Imperátor Nicéphorus, Ecclésias Dei 
 devástans, divérso mortis génere, nimírum ense, láqueo, sagíttis, diútino cárcere et fame necári fecit.
\switchcolumn
\selectlanguage{english}
In Bulgaria, many holy martyrs, whom 
 the impious Emperor Nicephorus, while devastating the churches of God, put 
 to death in various ways: by the sword, by hanging, arrows, long 
 imprisonment, and by starvation.
\switchcolumn*
\selectlanguage{latin}
Romæ sanctárum Vírginum 
 Rómulæ, Redémptæ et Herúndinis, de quibus scribit sanctus Gregórius Papa.
\switchcolumn
\selectlanguage{english}
At Rome, the saintly virgins Romula, 
 Redempta, and Herundo, mentioned by Pope St. Gregory in his writings.
\switchcolumn*
\selectlanguage{latin}
\end{paracol}


% ---- martyrology/mart07/mart0724.htm
\needspace{10\baselineskip}
\begin{paracol}{2}
\selectlanguage{latin}
\begin{center}{\color{gregoriocolor} Nono Kaléndas Augústi. 
 Luna\dots\ }\end{center}
\switchcolumn
\selectlanguage{english}
\begin{center}{\color{gregoriocolor} The 
 Twenty-Fourth Day of 
 July. The\dots\ Day of the Moon.}\end{center}
\end{paracol}

\noindent\begin{tabularx}{\linewidth}{*{19}{>{\centering\arraybackslash}X}}
 \textcolor{gregoriocolor}{a} & \textcolor{gregoriocolor}{b} & \textcolor{gregoriocolor}{c} & \textcolor{gregoriocolor}{d} & \textcolor{gregoriocolor}{e} & \textcolor{gregoriocolor}{f} & \textcolor{gregoriocolor}{g} & \textcolor{gregoriocolor}{h} & \textcolor{gregoriocolor}{i} & \textcolor{gregoriocolor}{k} & \textcolor{gregoriocolor}{l} & \textcolor{gregoriocolor}{m} & \textcolor{gregoriocolor}{n} & \textcolor{gregoriocolor}{p} & \textcolor{gregoriocolor}{q} & \textcolor{gregoriocolor}{r} & \textcolor{gregoriocolor}{s} & \textcolor{gregoriocolor}{t} & \textcolor{gregoriocolor}{u} \\
 29 & 30 & 1 & 2 & 3 & 4 & 5 & 6 & 7 & 8 & 9 & 10 & 11 & 12 & 13 & 14 & 15 & 16 & 17 \\
\end{tabularx}
\vspace{0.5\baselineskip}
\noindent\begin{tabularx}{\linewidth}{*{12}{>{\centering\arraybackslash}X}}
 \textcolor{gregoriocolor}{A} & \textcolor{gregoriocolor}{B} & \textcolor{gregoriocolor}{C} & \textcolor{gregoriocolor}{D} & \textcolor{gregoriocolor}{E} & F & \textcolor{gregoriocolor}{F} & \textcolor{gregoriocolor}{G} & \textcolor{gregoriocolor}{H} & \textcolor{gregoriocolor}{M} & \textcolor{gregoriocolor}{N} & \textcolor{gregoriocolor}{P} \\
 18 & 19 & 20 & 21 & 22 & 23 & 23 & 24 & 25 & 26 & 27 & 28 \\
\end{tabularx}

\begin{paracol}{2}
\selectlanguage{latin}
\lettrine[lines=1]{V}{igília} sancti Jacóbi Apóstoli.
\switchcolumn
\selectlanguage{english}
\lettrine[lines=1]{T}{he} Vigil of St. James the Apostle.
\switchcolumn*
\selectlanguage{latin}
Tyri, apud lacum 
 Vulsínium, in Túscia, sanctæ Christínæ, Vírginis et Mártyris. Hæc 
 Virgo, cum patris idóla áurea et argéntea, in Christum credens, comminuísset 
 atque illórum frágmina paupéribus erogásset, ejúsdem patris jussu verbéribus 
 dilaniáta est, aliísque supplíciis diríssime cruciáta, et cum magno saxi 
 póndere in lacum projécta, sed ab Angelo liberáta; deínde, sub álio Júdice, 
 patris sui successóre, acerbióra torménta constánter pértulit; novíssime, 
 sub Juliáno Præside, post fornácem ardéntem, ubi quinque diébus illæsa 
 permánsit, post serpéntes virtúte Christi superátos, martyrii sui cursum 
 abscissióne linguæ et sagittárum infixióne complévit.
\switchcolumn
\selectlanguage{english}
At Tiro in Tuscany, on Lake Bolsena, 
 St. Christina, virgin and martyr. Because she believed in Christ, and 
 broke up her father's gold and silver idols to give them to the poor, she 
 was cruelly scourged at his command, subjected to other most severe 
 torments, and thrown with a heavy stone into the lake from which she was 
 drawn out by an angel. Then under another judge, who succeeded her 
 father, she bore courageously still more bitter tortures. Finally, 
 after she had been shut up by the governor Julian in a burning furnace for 
 five days without any injury, after being cured of the sting of serpents, 
 she ended her martyrdom by having her tongue cut out, and being pierced with 
 arrows.
\switchcolumn*
\selectlanguage{latin}
Romæ, via Tiburtína, 
 sancti Vincéntii Mártyris.
\switchcolumn
\selectlanguage{english}
At Rome, on the Tiburtine Way, St. 
 Vincent, martyr.
\switchcolumn*
\selectlanguage{latin}
Amitérni, in Vestínis, 
 pássio sanctórum mílitum octogínta trium.
\switchcolumn
\selectlanguage{english}
At Amiterno in Abruzzi, the 
 martyrdom of eighty-three holy soldiers.
\switchcolumn*
\selectlanguage{latin}
Eméritæ, in Hispánia, 
 sancti Victóris, viri militáris, qui, cum frátribus Stercátio et Antinógene, 
 in persecutióne Diocletiáni, per divérsa supplícia martyrium consummávit.
\switchcolumn
\selectlanguage{english}
At Merida in Spain, St. Victor, a 
 soldier who, with his two brothers, Stercatius and Antinogenes, by divers 
 torments fulfilled his martyrdom in the persecution of Diocletian.
\switchcolumn*
\selectlanguage{latin}
Item sanctórum Mártyrum 
 Menéi et Capitónis.
\switchcolumn
\selectlanguage{english}
Also, the holy martyrs Meneus and 
 Capito.
\switchcolumn*
\selectlanguage{latin}
In Lycia sanctárum 
 Mártyrum Nicétæ et Aquilínæ, quæ, beáti Christóphori Mártyris prædicatióne 
 ad Christum convérsæ, martyrii palmam obtruncatióne cápitis sumpsérunt.
\switchcolumn
\selectlanguage{english}
In Lycia, the holy martyrs Niceta 
 and Aquilina, who were converted to Christ by the preaching of the blessed 
 martyr Christopher, and gained the palm of martyrdom by being beheaded.
\switchcolumn*
\selectlanguage{latin}
Apud Sénonas sancti 
 Ursicíni, Epíscopi et Confessóris.
\switchcolumn
\selectlanguage{english}
At Sens, St. Ursicinus, bishop and 
 confessor.
\switchcolumn*
\selectlanguage{latin}
\end{paracol}


% ---- martyrology/mart07/mart0725.htm
\needspace{10\baselineskip}
\begin{paracol}{2}
\selectlanguage{latin}
\begin{center}{\color{gregoriocolor} Octávo Kaléndas Augústi. 
 Luna\dots\ }\end{center}
\switchcolumn
\selectlanguage{english}
\begin{center}{\color{gregoriocolor} The 
 Twenty-Fifth Day of 
 July. The\dots\ Day of the Moon.}\end{center}
\end{paracol}

\noindent\begin{tabularx}{\linewidth}{*{19}{>{\centering\arraybackslash}X}}
 \textcolor{gregoriocolor}{a} & \textcolor{gregoriocolor}{b} & \textcolor{gregoriocolor}{c} & \textcolor{gregoriocolor}{d} & \textcolor{gregoriocolor}{e} & \textcolor{gregoriocolor}{f} & \textcolor{gregoriocolor}{g} & \textcolor{gregoriocolor}{h} & \textcolor{gregoriocolor}{i} & \textcolor{gregoriocolor}{k} & \textcolor{gregoriocolor}{l} & \textcolor{gregoriocolor}{m} & \textcolor{gregoriocolor}{n} & \textcolor{gregoriocolor}{p} & \textcolor{gregoriocolor}{q} & \textcolor{gregoriocolor}{r} & \textcolor{gregoriocolor}{s} & \textcolor{gregoriocolor}{t} & \textcolor{gregoriocolor}{u} \\
 30 & 1 & 2 & 3 & 4 & 5 & 6 & 7 & 8 & 9 & 10 & 11 & 12 & 13 & 14 & 15 & 16 & 17 & 18 \\
\end{tabularx}
\vspace{0.5\baselineskip}
\noindent\begin{tabularx}{\linewidth}{*{12}{>{\centering\arraybackslash}X}}
 \textcolor{gregoriocolor}{A} & \textcolor{gregoriocolor}{B} & \textcolor{gregoriocolor}{C} & \textcolor{gregoriocolor}{D} & \textcolor{gregoriocolor}{E} & F & \textcolor{gregoriocolor}{F} & \textcolor{gregoriocolor}{G} & \textcolor{gregoriocolor}{H} & \textcolor{gregoriocolor}{M} & \textcolor{gregoriocolor}{N} & \textcolor{gregoriocolor}{P} \\
 19 & 20 & 21 & 22 & 23 & 24 & 24 & 25 & 26 & 27 & 28 & 29 \\
\end{tabularx}

\begin{paracol}{2}
\selectlanguage{latin}
\lettrine[lines=2]{S}{ancti} Jacóbi Apóstoli, 
 qui éxstitit beáti Joánnis Evangelístæ frater; et, prope festum Paschæ ab 
 Heróde Agríppa decollátus, primus ex Apóstolis corónam martyrii percépit. 
 Ejus sacra ossa, ab Hierosólymis ad Hispánias hoc die transláta, et in 
 últimis eárum fínibus apud Gallæciam recóndita, celebérrima illárum géntium 
 veneratióne, et frequénti Christianórum concúrsu, religiónis et voti causa 
 illuc adeúntium, pie colúntur.
\switchcolumn
\selectlanguage{english}
\lettrine[lines=2]{S}{t.} James the Apostle, brother of 
 the blessed evangelist John, who was beheaded by Herod Agrippa at about the 
 feast of Easter. He was the first of the apostles to receive the crown 
 of martyrdom. His sacred bones were on this day carried from Jerusalem 
 to Spain, and placed in the remote province of Galicia, where they are 
 devoutly honoured by the far-famed piety of the inhabitants, and the 
 frequent concourse of Christians, who visit them through piety and in fulfillment of vows.
\switchcolumn*
\selectlanguage{latin}
In Lycia sancti 
 Christóphori Mártyris, qui, sub Décio, virgis férreis attrítus, et a flammæ 
 æstuántis incéndio supérna Christi virtúte servátus, ad últimum, sagittárum 
 íctibus confóssus, cápitis obtruncatióne martyrium complévit.
\switchcolumn
\selectlanguage{english}
In Lycia, in the time of Decius, St. 
 Christopher, martyr. Being scourged with iron rods, cast into the 
 flames, from which he was saved by the power of Christ, and finally 
 transfixed with arrows and beheaded, he completed his martyrdom.
\switchcolumn*
\selectlanguage{latin}
Barcinóne, in Hispánia, 
 natális beáti Cucuphátis Mártyris, qui, in persecutióne Diocletiáni, sub 
 Daciáno Præside, plúrimis torméntis superátis, tandem, percússus gládio, 
 victor migrávit in cælum.
\switchcolumn
\selectlanguage{english}
At Barcelona in Spain, during the 
 persecution of Diocletian and under the governor Dacian, the birthday of the 
 holy martyr Cucuphas. After overcoming many torments, he was struck 
 with the sword, and thus went triumphantly to heaven.
\switchcolumn*
\selectlanguage{latin}
In Palæstína sancti 
 Pauli Mártyris, qui, in persecutióne Maximiáni Galérii, sub Firmiliáno 
 Præside, cápitis damnátus, et, exíguo orándi spátio impetráto, primum pro 
 contribúlibus suis, deínde pro Judæis qui ipsum damnáverat, et pro carnífice 
 a quo feriéndus erat, Deum toto corde precátus, martyrii corónam, abscíssis 
 cervícibus, accépit.
\switchcolumn
\selectlanguage{english}
In Palestine, St. Paul, a martyr in 
 the persecution of Maximian Galerius, under the governor Firmilian. He 
 was condemned to death, but having obtained a short period for prayer, he 
 besought God with all his heart, first for his own countrymen, then for the 
 Jews and the Gentiles, that they might embrace the true faith, next for the 
 multitude of spectators, and finally for the judge who had condemned him and 
 the executioner who was to strike him; after which he received the crown of 
 martyrdom by beheading.
\switchcolumn*
\selectlanguage{latin}
Furcónii, in Vestínis, 
 sanctórum Mártyrum Sipontinórum Floréntii et Felícis.
\switchcolumn
\selectlanguage{english}
At Forcono in Abruzzi, the holy 
 martyrs Florentius and Felix, natives of Siponte.
\switchcolumn*
\selectlanguage{latin}
Córdubæ, in Hispánia, 
 sancti Theodemíri, Mónachi et Mártyris.
\switchcolumn
\selectlanguage{english}
At Cordova, St. Theodemir, monk and 
 martyr.
\switchcolumn*
\selectlanguage{latin}
In Palæstína sanctæ 
 Valentínæ Vírginis, quæ, cum ad aram, ut immoláret, addúcta esset, eámque 
 cálcibus evertísset, prius diríssime cruciáta est; deínde, una cum sócia 
 Vírgine, in ignem conjécta cucúrrit ad Sponsum.
\switchcolumn
\selectlanguage{english}
In Palestine, St. Valentina, a 
 virgin, who was led to an altar to offer sacrifice, but overturning it with 
 her foot, she was cruelly tortured, and being cast into the fire with 
 another virgin, her companion, she went to her Spouse.
\switchcolumn*
\selectlanguage{latin}
Tréviris sancti 
 Magneríci, Epíscopi et Confessóris.
\switchcolumn
\selectlanguage{english}
At Treves, St. Magnericus, bishop 
 and confessor.
\switchcolumn*
\selectlanguage{latin}
\end{paracol}


% ---- martyrology/mart07/mart0726.htm
\needspace{10\baselineskip}
\begin{paracol}{2}
\selectlanguage{latin}
\begin{center}{\color{gregoriocolor} Séptimo Kaléndas Augústi. 
 Luna\dots\ }\end{center}
\switchcolumn
\selectlanguage{english}
\begin{center}{\color{gregoriocolor} The 
 Twenty-Sixth Day of 
 July. The\dots\ Day of the Moon.}\end{center}
\end{paracol}

\noindent\begin{tabularx}{\linewidth}{*{19}{>{\centering\arraybackslash}X}}
 \textcolor{gregoriocolor}{a} & \textcolor{gregoriocolor}{b} & \textcolor{gregoriocolor}{c} & \textcolor{gregoriocolor}{d} & \textcolor{gregoriocolor}{e} & \textcolor{gregoriocolor}{f} & \textcolor{gregoriocolor}{g} & \textcolor{gregoriocolor}{h} & \textcolor{gregoriocolor}{i} & \textcolor{gregoriocolor}{k} & \textcolor{gregoriocolor}{l} & \textcolor{gregoriocolor}{m} & \textcolor{gregoriocolor}{n} & \textcolor{gregoriocolor}{p} & \textcolor{gregoriocolor}{q} & \textcolor{gregoriocolor}{r} & \textcolor{gregoriocolor}{s} & \textcolor{gregoriocolor}{t} & \textcolor{gregoriocolor}{u} \\
 1 & 2 & 3 & 4 & 5 & 6 & 7 & 8 & 9 & 10 & 11 & 12 & 13 & 14 & 15 & 16 & 17 & 18 & 19 \\
\end{tabularx}
\vspace{0.5\baselineskip}
\noindent\begin{tabularx}{\linewidth}{*{12}{>{\centering\arraybackslash}X}}
 \textcolor{gregoriocolor}{A} & \textcolor{gregoriocolor}{B} & \textcolor{gregoriocolor}{C} & \textcolor{gregoriocolor}{D} & \textcolor{gregoriocolor}{E} & F & \textcolor{gregoriocolor}{F} & \textcolor{gregoriocolor}{G} & \textcolor{gregoriocolor}{H} & \textcolor{gregoriocolor}{M} & \textcolor{gregoriocolor}{N} & \textcolor{gregoriocolor}{P} \\
 20 & 21 & 22 & 23 & 24 & 25 & 25 & 26 & 27 & 28 & 29 & 30 \\
\end{tabularx}

\begin{paracol}{2}
\selectlanguage{latin}
\lettrine[lines=2]{D}{ormítio} sanctæ Annæ, 
 quæ mater éxstitit Immaculátæ Vírginis Genitrícis Dei Maríæ.
\switchcolumn
\selectlanguage{english}
\lettrine[lines=2]{T}{he} departure from this life of St. 
 Anne, mother of the Immaculate Virgin Mary, the Mother of God.
\switchcolumn*
\selectlanguage{latin}
Philíppis, in 
 Macedónia, natális sancti Erásti, qui, illic a beáto Paulo Apóstolo relíctus 
 Epíscopus, ibídem martyrio coronátus est.
\switchcolumn
\selectlanguage{english}
At Philippi in Macedonia, the 
 birthday of St. Erastus, who was appointed bishop of that place by the 
 blessed apostle Paul, and was there crowned with martyrdom.
\switchcolumn*
\selectlanguage{latin}
Romæ, via Latína, sanctórum Mártyrum Symphrónii, Olympii, Theodúli et Exsupériæ; qui (ut in 
 gestis sancti Stéphani Papæ légitur), ígnibus combústi, martyrii palmam 
 adépti sunt.
\switchcolumn
\selectlanguage{english}
At Rome, on the Latin Way, the holy 
 martyrs Symphronius, Olympius, Theodulus, and Exuperia, who (as we read in 
 the Acts of Pope St. Stephen) were burned alive, and thus obtained the palm 
 of martyrdom.
\switchcolumn*
\selectlanguage{latin}
In Portu Románo sancti 
 Hyacínthi Mártyris, qui, primo in ignem injéctus, deínde in profluéntem 
 præcipitátus, illæsus evásit; post hæc, sub Trajáno Imperatóre, a Leóntio 
 Consulári percússus gládio, vitam finívit. Ipsíus corpus Júlia Matróna 
 in prædio suo, juxta Urbem, sepelívit.
\switchcolumn
\selectlanguage{english}
At Porto, St. Hyacinth, martyr, who 
 was first thrown into the fire, and then cast into a stream without being 
 injured. Afterwards, under Emperor Trajan, being struck with the sword 
 by the exconsul Leontius, his martyrdom was fulfilled. His body was 
 buried by the matron Julia on her own estate near Rome.
\switchcolumn*
\selectlanguage{latin}
Verónæ sancti Valéntis, 
 Epíscopi et Confessóris.
\switchcolumn
\selectlanguage{english}
At Verona, St. Valens, bishop and 
 confessor.
\switchcolumn*
\selectlanguage{latin}
Romæ sancti Pastóris 
 Presbyteri, cujus nómine Títulus exstat in Vimináli, apud sanctam 
 Pudentiánam.
\switchcolumn
\selectlanguage{english}
At Rome, St. Pastor, a priest in 
 whose name a title exists in the church of St. Pudentiana, on the Viminal 
 Hill.
\switchcolumn*
\selectlanguage{latin}
In monastério sancti 
 Benedícti, in agro Mantuáno, sancti Simeónis, Mónachi et Eremítæ, qui, 
 multis miráculis clarus, in senectúte bona quiévit.
\switchcolumn
\selectlanguage{english}
In the monastery of St. Benedict, 
 near Mantua, St. Simeon, monk and hermit, who was renowned for many 
 miracles, and at an advanced age rested in the Lord.
\switchcolumn*
\selectlanguage{latin}
Lúere, in diœcési 
 Brixiénsi, sanctæ Bartholomǽæ Capitánio, Vírginis, Sorórum a Caritáte 
 Fundatrícis, puéllis instituéndis præcláræ, quam Pius Papa Duodécimus albo 
 sanctárum Vírginum adscrípsit.
\switchcolumn
\selectlanguage{english}
At Lovere, in the diocese of Brescia, 
 St. Bartholemea Capitanio, virgin, who founded the Sisters of Charity, 
 dedicated to teaching the young. Pope Pius XII added her name to the 
 catalogue of holy virgins.
\switchcolumn*
\selectlanguage{latin}
\end{paracol}


% ---- martyrology/mart07/mart0727.htm
\needspace{10\baselineskip}
\begin{paracol}{2}
\selectlanguage{latin}
\begin{center}{\color{gregoriocolor} Sexto Kaléndas Augústi. 
 Luna\dots\ }\end{center}
\switchcolumn
\selectlanguage{english}
\begin{center}{\color{gregoriocolor} The 
 Twenty-Seventh Day of 
 July. The\dots\ Day of the Moon.}\end{center}
\end{paracol}

\noindent\begin{tabularx}{\linewidth}{*{19}{>{\centering\arraybackslash}X}}
 \textcolor{gregoriocolor}{a} & \textcolor{gregoriocolor}{b} & \textcolor{gregoriocolor}{c} & \textcolor{gregoriocolor}{d} & \textcolor{gregoriocolor}{e} & \textcolor{gregoriocolor}{f} & \textcolor{gregoriocolor}{g} & \textcolor{gregoriocolor}{h} & \textcolor{gregoriocolor}{i} & \textcolor{gregoriocolor}{k} & \textcolor{gregoriocolor}{l} & \textcolor{gregoriocolor}{m} & \textcolor{gregoriocolor}{n} & \textcolor{gregoriocolor}{p} & \textcolor{gregoriocolor}{q} & \textcolor{gregoriocolor}{r} & \textcolor{gregoriocolor}{s} & \textcolor{gregoriocolor}{t} & \textcolor{gregoriocolor}{u} \\
 2 & 3 & 4 & 5 & 6 & 7 & 8 & 9 & 10 & 11 & 12 & 13 & 14 & 15 & 16 & 17 & 18 & 19 & 20 \\
\end{tabularx}
\vspace{0.5\baselineskip}
\noindent\begin{tabularx}{\linewidth}{*{12}{>{\centering\arraybackslash}X}}
 \textcolor{gregoriocolor}{A} & \textcolor{gregoriocolor}{B} & \textcolor{gregoriocolor}{C} & \textcolor{gregoriocolor}{D} & \textcolor{gregoriocolor}{E} & F & \textcolor{gregoriocolor}{F} & \textcolor{gregoriocolor}{G} & \textcolor{gregoriocolor}{H} & \textcolor{gregoriocolor}{M} & \textcolor{gregoriocolor}{N} & \textcolor{gregoriocolor}{P} \\
 21 & 22 & 23 & 24 & 25 & 26 & 26 & 27 & 28 & 29 & 30 & 1 \\
\end{tabularx}

\begin{paracol}{2}
\selectlanguage{latin}
\lettrine[lines=2]{N}{icomedíæ} pássio sancti 
 Pantaleónis médici, qui, pro fide Christi, a Maximiáno Imperatóre tentus, et 
 equúlei pœna ac lampadárum exustióne afflíctus, sed inter hæc, Dómino sibi 
 apparénte, refrigerátus, gládii tandem ictu martyrium consummávit.
\switchcolumn
\selectlanguage{english}
\lettrine[lines=2]{A}{t} Nicomedia, the martyrdom of St. 
 Pantaleon, a physician. For the faith of Christ he was apprehended by 
 Emperor Maximian, subjected to the torture and burned with torches, during 
 which torments he was comforted by an apparition of our Lord. He ended 
 his martyrdom by a stroke of the sword.
\switchcolumn*
\selectlanguage{latin}
Vigíliis, in Apúlia, sanctórum Mártyrum Mauri Epíscopi, Pantaleémonis et Sérgii; qui passi sunt 
 sub Trajáno.
\switchcolumn
\selectlanguage{english}
At Bisceglia in Apulia, the holy 
 martyrs Maur, a bishop, Pantaleon, and Sergius, who suffered under Trajan.
\switchcolumn*
\selectlanguage{latin}
Nicomedíæ sancti 
 Hermolái Presbyteri, cujus doctrína beátus Pantáleon ad fidem convérsus est; 
 itémque sanctórum Hermíppi et Hermócratis fratrum, qui, post multas pœnas 
 sibi illátas, capitáli senténtia, ob confessiónem Christi, a Maximiáno 
 Imperatóre puníti sunt.
\switchcolumn
\selectlanguage{english}
At Nicomedia, St. Hermolaus, priest, 
 by whose instructions blessed Pantaleon was converted to the faith. 
 Also, the Saints Hermippus and Hermocrates, brothers. After many 
 sufferings borne for the confession of Christ, they were condemned to death 
 by the same Maximian.
\switchcolumn*
\selectlanguage{latin}
Córdubæ, in Hispánia, 
 sanctórum Mártyrum Geórgii Diáconi, Aurélii et uxóris Natáliæ, Felícis et 
 uxóris Liliósæ, in persecutióne Arábica.
\switchcolumn
\selectlanguage{english}
At Cordova in Spain, during the Arab 
 persecution, the holy martyrs George, a deacon, Aurelius and his wife 
 Natalia, Felix and his wife Liliosa.
\switchcolumn*
\selectlanguage{latin}
Nolæ, in Campánia, sanctórum Mártyrum Felícis, Júliæ et Jucúndæ.
\switchcolumn
\selectlanguage{english}
At Nola in Campania, the holy 
 martyrs Felix, Julia, and Jucunda.
\switchcolumn*
\selectlanguage{latin}
Apud Homerítas, in 
 Arábia, commemorátio sanctórum Mártyrum, qui, sub Dúnaan tyránno, ob Christi 
 fidem, incéndio tráditi sunt.
\switchcolumn
\selectlanguage{english}
In the country of the Homerites in 
 Arabia, the commemoration of the holy martyrs, who were delivered to the 
 flames for the faith of Christ under the tyrant Dunaan.
\switchcolumn*
\selectlanguage{latin}
Ephesi natális 
 sanctórum septem Dormiéntium, scílicet Maximiáni, Malchi, Martiniáni, 
 Dionysii, Joánnis, Serapiónis et Constantíni.
\switchcolumn
\selectlanguage{english}
At Ephesus, the birthday of the 
 Seven Holy Sleepers, Maximian, Malchus, Martinian, Denis, John, Serapion, 
 and Constantine.
\switchcolumn*
\selectlanguage{latin}
Romæ sancti Cælestíni 
 Papæ Primi, qui damnávit Constantinopolitánum Epíscopum Nestórium, 
 Pelagiúmque fugávit; cujus étiam auctoritáte universális sancta Synodus 
 Ephesína advérsus eúndem Nestórium celebráta est.
\switchcolumn
\selectlanguage{english}
At Rome, Pope St. Celestine I, who 
 had condemned Nestorius, bishop of Constantinople, and put Pelagius to 
 flight. By his command the holy universal Council of Ephesus was also 
 held against the same Nestorius.
\switchcolumn*
\selectlanguage{latin}
Antisiodóri deposítio 
 beáti Æthérii, Epíscopi et Confessóris.
\switchcolumn
\selectlanguage{english}
At Auxerre, the death of blessed 
 Aetherius, bishop and confessor.
\switchcolumn*
\selectlanguage{latin}
Constantinópoli beátæ 
 Anthúsæ Vírginis, quæ, sub Constantíno Coprónymo, ob cultum sanctárum 
 Imáginum, verbéribus cæsa et exsílio relegáta, quiévit in Dómino.
\switchcolumn
\selectlanguage{english}
At Constantinople, blessed Anthusa, 
 virgin. After being scourged and banished by Constantine Copronymus 
 for the veneration of holy images, she rested in the Lord.
\switchcolumn*
\selectlanguage{latin}
\end{paracol}


% ---- martyrology/mart07/mart0728.htm
\needspace{10\baselineskip}
\begin{paracol}{2}
\selectlanguage{latin}
\begin{center}{\color{gregoriocolor} Quinto Kaléndas Augústi. 
 Luna\dots\ }\end{center}
\switchcolumn
\selectlanguage{english}
\begin{center}{\color{gregoriocolor} The 
 Twenty-Eighth Day of 
 July. The\dots\ Day of the Moon.}\end{center}
\end{paracol}

\noindent\begin{tabularx}{\linewidth}{*{19}{>{\centering\arraybackslash}X}}
 \textcolor{gregoriocolor}{a} & \textcolor{gregoriocolor}{b} & \textcolor{gregoriocolor}{c} & \textcolor{gregoriocolor}{d} & \textcolor{gregoriocolor}{e} & \textcolor{gregoriocolor}{f} & \textcolor{gregoriocolor}{g} & \textcolor{gregoriocolor}{h} & \textcolor{gregoriocolor}{i} & \textcolor{gregoriocolor}{k} & \textcolor{gregoriocolor}{l} & \textcolor{gregoriocolor}{m} & \textcolor{gregoriocolor}{n} & \textcolor{gregoriocolor}{p} & \textcolor{gregoriocolor}{q} & \textcolor{gregoriocolor}{r} & \textcolor{gregoriocolor}{s} & \textcolor{gregoriocolor}{t} & \textcolor{gregoriocolor}{u} \\
 3 & 4 & 5 & 6 & 7 & 8 & 9 & 10 & 11 & 12 & 13 & 14 & 15 & 16 & 17 & 18 & 19 & 20 & 21 \\
\end{tabularx}
\vspace{0.5\baselineskip}
\noindent\begin{tabularx}{\linewidth}{*{12}{>{\centering\arraybackslash}X}}
 \textcolor{gregoriocolor}{A} & \textcolor{gregoriocolor}{B} & \textcolor{gregoriocolor}{C} & \textcolor{gregoriocolor}{D} & \textcolor{gregoriocolor}{E} & F & \textcolor{gregoriocolor}{F} & \textcolor{gregoriocolor}{G} & \textcolor{gregoriocolor}{H} & \textcolor{gregoriocolor}{M} & \textcolor{gregoriocolor}{N} & \textcolor{gregoriocolor}{P} \\
 22 & 23 & 24 & 25 & 26 & 27 & 27 & 28 & 29 & 30 & 1 & 2 \\
\end{tabularx}

\begin{paracol}{2}
\selectlanguage{latin}
\lettrine[lines=2]{M}{edioláni} natális 
 sanctórum Mártyrum Nazárii, et Celsi púeri, quos Anolínus, sub rábie 
 persecutiónis quæ per Nerónem excitáta est, diu macerátos et afflíctos in cárcere, gládio feríri jussit.
\switchcolumn
\selectlanguage{english}
\lettrine[lines=2]{A}{t} Milan, the birthday of the holy 
 martyrs Nazarius and a boy named Celsus. While the persecution excited 
 by Nero was raging, they were beheaded by Anolinus, after long sufferings 
 and afflictions endured in prison.
\switchcolumn*
\selectlanguage{latin}
Romæ pássio sancti 
 Victóris Primi, Papæ et Mártyris.
\switchcolumn
\selectlanguage{english}
At Rome, the martyrdom of St. 
 Victor, pope and martyr.
\switchcolumn*
\selectlanguage{latin}
Item Romæ sancti 
 Innocéntii Primi, Papæ et Confessóris, qui ad Dóminum migrávit quarto Idus 
 Mártii.
\switchcolumn
\selectlanguage{english}
Also at Rome, St. Innocent, pope and 
 confessor, who passed to the Lord on the 12th of March.
\switchcolumn*
\selectlanguage{latin}
Thebáide, in Ægypto, 
 commemorátio plurimórum sanctórum Mártyrum, qui in Décii et Valeriáni 
 persecutióne passi sunt, quando Christiánis, optántibus pro Christi nómine 
 gládio pércuti, cállidus hostis, tarda ad montem supplícia conquírens, 
 ánimas cupiébat juguláre, non córpora. Ex eórum número unus, post 
 equúleos et láminas ac sartágines superátas, melle perúnctus, ligátis 
 mánibus post tergum, sub ardentíssimo sole, fucórum ac muscárum acúleis 
 expósitus fuit; álius, inter flores mólliter vinctus, cum ad eum, ut 
 concitáret in libídinem, impudicíssimum scortum venísset, præcísam morsu 
 linguam in blandiéntis fáciem éxspuit.
\switchcolumn
\selectlanguage{english}
In Thebais in Egypt, the 
 commemoration of many holy martyrs who suffered in the persecution of Decius 
 and Valerian. At this time, when Christians sought death by the sword 
 for the name of Christ, the crafty enemy devised certain slow torments to put 
 them to death, wishing to kill their souls much more than their bodies. 
 One of these Christians, after suffering the tortured of the rack, of hot 
 metal plates and of seething oil, was smeared with honey and exposed, in the 
 broiling heat of the sun, with his hands tied behind him, to the sting of 
 wasps and flies. Another, bound and placed among flowers, being 
 approached by a shameless woman for the purpose of exciting his passions, 
 bit through his tongue and spat it in her face.
\switchcolumn*
\selectlanguage{latin}
Ancyræ, in Galátia, sancti Eustáthii Mártyris, qui, váriis tormentórum genéribus excruciátibus, 
 atque demérsus in flumen, sed inde ab Angelo eréptus, demum, colúmba e cælis 
 adveniénte, ad præmia ætérna vocátus est.
\switchcolumn
\selectlanguage{english}
At Ancyra in Galatia, the holy 
 martyr Eustathius. After various torments he was plunged into a river, 
 but being delivered by an angel, was finally called to his eternal reward by 
 a dove coming from heaven.
\switchcolumn*
\selectlanguage{latin}
Miléti, in Cária, sancti Acátii Mártyris, qui, sub Licínio Imperatóre, post divérsas pœnas, 
 injéctus in fornácem ac Dei ope servátus illæsus, cápitis abscissióne 
 martyrium complévit.
\switchcolumn
\selectlanguage{english}
At Miletus, in the time of Emperor 
 Licinius, the holy martyr Acatius, who completed his martyrdom by having his 
 head struck off, after having undergone different torments and having been 
 thrown into a furnace, from which through the assistance of God he came 
 out uninjured.
\switchcolumn*
\selectlanguage{latin}
In Británnia minóre 
 sancti Sampsónis, Epíscopi et Confessóris.
\switchcolumn
\selectlanguage{english}
In Brittany, St. Sampson, bishop and 
 confessor.
\switchcolumn*
\selectlanguage{latin}
Lugdúni, in Gállia, 
 sancti Peregríni Presbyteri, cujus beatitúdinem glória miraculórum testátur.
\switchcolumn
\selectlanguage{english}
At Lyons in France, St. Peregrinus, 
 priest, whose happiness in heaven is testified by glorious miracles.
\switchcolumn*
\selectlanguage{latin}
\end{paracol}


% ---- martyrology/mart07/mart0729.htm
\needspace{10\baselineskip}
\begin{paracol}{2}
\selectlanguage{latin}
\begin{center}{\color{gregoriocolor} Quarto Kaléndas Augústi. 
 Luna\dots\ }\end{center}
\switchcolumn
\selectlanguage{english}
\begin{center}{\color{gregoriocolor} The 
 Twenty-Ninth Day of 
 July. The\dots\ Day of the Moon.}\end{center}
\end{paracol}

\noindent\begin{tabularx}{\linewidth}{*{19}{>{\centering\arraybackslash}X}}
 \textcolor{gregoriocolor}{a} & \textcolor{gregoriocolor}{b} & \textcolor{gregoriocolor}{c} & \textcolor{gregoriocolor}{d} & \textcolor{gregoriocolor}{e} & \textcolor{gregoriocolor}{f} & \textcolor{gregoriocolor}{g} & \textcolor{gregoriocolor}{h} & \textcolor{gregoriocolor}{i} & \textcolor{gregoriocolor}{k} & \textcolor{gregoriocolor}{l} & \textcolor{gregoriocolor}{m} & \textcolor{gregoriocolor}{n} & \textcolor{gregoriocolor}{p} & \textcolor{gregoriocolor}{q} & \textcolor{gregoriocolor}{r} & \textcolor{gregoriocolor}{s} & \textcolor{gregoriocolor}{t} & \textcolor{gregoriocolor}{u} \\
 4 & 5 & 6 & 7 & 8 & 9 & 10 & 11 & 12 & 13 & 14 & 15 & 16 & 17 & 18 & 19 & 20 & 21 & 22 \\
\end{tabularx}
\vspace{0.5\baselineskip}
\noindent\begin{tabularx}{\linewidth}{*{12}{>{\centering\arraybackslash}X}}
 \textcolor{gregoriocolor}{A} & \textcolor{gregoriocolor}{B} & \textcolor{gregoriocolor}{C} & \textcolor{gregoriocolor}{D} & \textcolor{gregoriocolor}{E} & F & \textcolor{gregoriocolor}{F} & \textcolor{gregoriocolor}{G} & \textcolor{gregoriocolor}{H} & \textcolor{gregoriocolor}{M} & \textcolor{gregoriocolor}{N} & \textcolor{gregoriocolor}{P} \\
 23 & 24 & 25 & 26 & 27 & 28 & 28 & 29 & 30 & 1 & 2 & 3 \\
\end{tabularx}

\begin{paracol}{2}
\selectlanguage{latin}
\lettrine[lines=2]{T}{arásci,} in Gállia 
 Narbonénsi, sanctæ Marthæ Vírginis, hóspitæ Salvatóris nostri ac soróris 
 beatórum Maríæ Magdalénæ et Lázari.
\switchcolumn
\selectlanguage{english}
\lettrine[lines=2]{A}{t} Tarascon, in the province of 
 Narbonne in France, St. Martha, virgin, the hostess of our Saviour and 
 sister of blessed Mary Magdalene and St. Lazarus.
\switchcolumn*
\selectlanguage{latin}
Romæ, via Aurélia, sancti Felícis Secúndi, Papæ et Mártyris; qui, ab Ariáno Imperatóre 
 Constántio, ob cathólicæ fídei defensiónem, e sede sua dejéctus, et 
 Ceræ, in Túscia, occúlte gládio necátus, glorióse occúbuit. Ejus 
 corpus, inde a Cléricis raptum, eádem via sepúltum fuit; póstea vero, ad 
 Ecclésiam sanctórum Cosmæ et Damiáni delátum, ibídem, Gregório Décimo tértio 
 Summo Pontífice, repértum est sub altári, una cum relíquiis sanctórum 
 Mártyrum Marci, Marcelliáni et Tranquillíni, atque in eódem loco, prídie 
 Kaléndas Augústi, simul cum eis recónditum. In quo étiam altári 
 invénta fuérunt córpora sanctórum Mártyrum Abúndii Presbyteri, et Abundántii 
 Diáconi; quæ, non multo post, ad Ecclésiam Societátis Jesu solémniter 
 transláta sunt prídie natális eórum.
\switchcolumn
\selectlanguage{english}
At Rome, on the Aurelian Way, St. 
 Felix II, pope and martyr. Being expelled from his See by the Arian 
 emperor Constantius for defending the Catholic faith, and being put to the 
 sword privately at Cera in Tuscany, he died gloriously. His body was 
 taken away from that place by clerics, and buried on the Aurelian Way. 
 It was afterwards brought to the Church of the Saints Cosmas and Damian, 
 where, under the Sovereign Pontiff Gregory XIII, it was found beneath the 
 altar with the relics of the holy martyrs Mark, Marcellian, and 
 Tranquillinus, and with the latter was put back in the same place on the 
 31st of July. In the same altar were also found the bodies of the holy 
 martyrs Abundius, a priest, and Abundantius, a deacon, which were shortly 
 after solemnly transferred to the church of the Society of Jesus, on the eve 
 of their feast.
\switchcolumn*
\selectlanguage{latin}
Item Romæ, via 
 Portuénsi, sanctórum Mártyrum Simplícii, Faustíni et Beatrícis, tempóribus 
 Diocletiáni Imperatóris. Horum duo primi, post multa et divérsa 
 supplícia, jussi sunt capitálem subíre senténtiam; Beátrix vero, eórum soror, 
 in confessióne Christi est in cárcere præfocáta.
\switchcolumn
\selectlanguage{english}
Also at Rome, on the Via Portuensis, 
 the holy martyrs Simplicius, Faustinus, and Beatrice, in the time of Emperor 
 Diocletian. The first two, after being subjected to many different 
 torments, were condemned to suffer death; Beatrice, their sister, was 
 smothered in prison for the confession of Christ.
\switchcolumn*
\selectlanguage{latin}
Romæ prætérea sanctórum 
 Mártyrum Lucíllæ et Floræ Vírginum, Eugénii, Antoníni, Theodóri, et Sociórum 
 decem et octo; qui sub Galliéno Imperatóre martyrium obiérunt.
\switchcolumn
\selectlanguage{english}
At Rome, likewise the holy martyrs 
 Lucilla and Flora, virgins, Eugenius, Antoninus, Theodore, and eighteen 
 companions, who underwent martyrdom in the reign of Emperor Gallienus.
\switchcolumn*
\selectlanguage{latin}
Item Romæ sanctæ 
 Serápiæ Vírginis, quæ, sub Hadriáno Príncipe, cum esset trádita duóbus 
 lascívis juvénibus et mínime potuísset illúdi, nec póstmodum ardéntibus 
 fácibus inflammári. Derílli Júdicis jussu fústibus cæsa est, dehinc 
 gládio decolláta. Ejus corpus a beáta Sabína in suo monuménto, juxta áream Vindiciáni, sepúltum est; sed memória ipsíus martyrii tértio Nonas 
 Septémbris celébrior habétur, quo die ambárum sarcóphagum ibi compósitum et 
 ornátum fuit, et locus oratiónis condígne dicátus.
\switchcolumn
\selectlanguage{english}
Again at Rome, St. Serapia, virgin. 
 Under Emperor Hadrian, she was delivered to two lustful young men, and as 
 she could not be corrupted, nor afterwards burned with lighted torches, she 
 was beaten with rods, and finally beheaded by order of the judge Derillus. 
 She was buried by blessed Sabina in her own tomb, near the field of 
 Vindician. But the commemoration of her martyrdom is celebrated more 
 solemnly on the 3rd of September, when their common tomb was finished and 
 adorned, and dedicated as a place of prayer.
\switchcolumn*
\selectlanguage{latin}
Gangris in Paphlagónia, 
 sancti Calliníci Mártyris, qui, virgis férreis verberátus aliísque 
 supplíciis afflíctus, tandem, in fornácem injéctus, spíritum Deo réddidit.
\switchcolumn
\selectlanguage{english}
At Gangra in Paphlagonia, St. 
 Callinicus, martyr, who was scourged with iron rods, and given over to other 
 torments. Being finally cast into a furnace, he gave up his soul to 
 God.
\switchcolumn*
\selectlanguage{latin}
In Norvégia sancti 
 Olávi, Regis et Mártyris.
\switchcolumn
\selectlanguage{english}
In Norway, St. Olaf, king and 
 martyr.
\switchcolumn*
\selectlanguage{latin}
Trecis, in Gállia, 
 sancti Lupi, Epíscopi et Confessóris; qui, cum beáto Germáno, ad expugnándam 
 Pelagianórum hæresim, in Británnia perréxit, urbémque Trecas a furóre Attilæ, 
 Gálliam omnem devastántis, assídua oratióne deféndit; demum, quinquagínta 
 duóbus annis sacerdótio venerabíliter functus, in pace quiévit.
\switchcolumn
\selectlanguage{english}
At Troyes in France, St. Lupus, 
 bishop and confessor, who went with blessed Germanus to England to 
 exterminate the Pelagian heresy, and by diligent prayer defended the city of 
 Troyes from the wrath of Attila, who was devastating all of France. At 
 length, having religiously discharged the functions of the priesthood for 
 fifty-two years, he rested in peace.
\switchcolumn*
\selectlanguage{latin}
In civitáte Briocénsi, 
 in Gállia, sancti Guliélmi, Epíscopi et Confessóris.
\switchcolumn
\selectlanguage{english}
At St. Brieuc in France, St. 
 William, bishop and confessor.
\switchcolumn*
\selectlanguage{latin}
Item deposítio beáti 
 Prósperi, Aurelianénsis Epíscopi.
\switchcolumn
\selectlanguage{english}
Also, the death of blessed Prosper, 
 bishop of Orleans.
\switchcolumn*
\selectlanguage{latin}
Apud Tudértum, in 
 Umbria, sancti Faustíni Confessóris.
\switchcolumn
\selectlanguage{english}
At Todi in Umbria, St. Faustinus, 
 confessor.
\switchcolumn*
\selectlanguage{latin}
In civitáte Mamiénsi 
 sanctæ Seraphínæ.
\switchcolumn
\selectlanguage{english}
At Mamia, St. Serafina.
\switchcolumn*
\selectlanguage{latin}
Romæ beáti Urbáni Papæ 
 Secúndi, qui, sancti Gregórii Séptimi vestígia secútus, doctrínæ et 
 religiónis stúdio enítuit, et fidéles Cruce signátos ad sacra Palæstínæ loca 
 ex infidélium domínio rediménda excitávit. Cultum autem, ab 
 immemorábili témpore eídem exhíbitum, Leo Décimus tértius, Póntifex Máximus, 
 ratum hábuit et confirmávit.
\switchcolumn
\selectlanguage{english}
At Rome, blessed Pope Urban II who 
 followed in the path of St. Gregory VII. He was resplendent for his 
 zeal for learning and religion, and aroused the faithful, signed with the 
 sign of the cross, to recover the holy places of Palestine from the power of 
 the infidels. Pope Leo XIII ratified and confirmed the veneration 
 shewn him from time immemorial.
\switchcolumn*
\selectlanguage{latin}
\end{paracol}


% ---- martyrology/mart07/mart0730.htm
\needspace{10\baselineskip}
\begin{paracol}{2}
\selectlanguage{latin}
\begin{center}{\color{gregoriocolor} Tértio Kaléndas Augústi. 
 Luna\dots\ }\end{center}
\switchcolumn
\selectlanguage{english}
\begin{center}{\color{gregoriocolor} The 
 Thirtieth Day of 
 July. The\dots\ Day of the Moon.}\end{center}
\end{paracol}

\noindent\begin{tabularx}{\linewidth}{*{19}{>{\centering\arraybackslash}X}}
 \textcolor{gregoriocolor}{a} & \textcolor{gregoriocolor}{b} & \textcolor{gregoriocolor}{c} & \textcolor{gregoriocolor}{d} & \textcolor{gregoriocolor}{e} & \textcolor{gregoriocolor}{f} & \textcolor{gregoriocolor}{g} & \textcolor{gregoriocolor}{h} & \textcolor{gregoriocolor}{i} & \textcolor{gregoriocolor}{k} & \textcolor{gregoriocolor}{l} & \textcolor{gregoriocolor}{m} & \textcolor{gregoriocolor}{n} & \textcolor{gregoriocolor}{p} & \textcolor{gregoriocolor}{q} & \textcolor{gregoriocolor}{r} & \textcolor{gregoriocolor}{s} & \textcolor{gregoriocolor}{t} & \textcolor{gregoriocolor}{u} \\
 5 & 6 & 7 & 8 & 9 & 10 & 11 & 12 & 13 & 14 & 15 & 16 & 17 & 18 & 19 & 20 & 21 & 22 & 23 \\
\end{tabularx}
\vspace{0.5\baselineskip}
\noindent\begin{tabularx}{\linewidth}{*{12}{>{\centering\arraybackslash}X}}
 \textcolor{gregoriocolor}{A} & \textcolor{gregoriocolor}{B} & \textcolor{gregoriocolor}{C} & \textcolor{gregoriocolor}{D} & \textcolor{gregoriocolor}{E} & F & \textcolor{gregoriocolor}{F} & \textcolor{gregoriocolor}{G} & \textcolor{gregoriocolor}{H} & \textcolor{gregoriocolor}{M} & \textcolor{gregoriocolor}{N} & \textcolor{gregoriocolor}{P} \\
 24 & 25 & 26 & 27 & 28 & 29 & 29 & 30 & 1 & 2 & 3 & 4 \\
\end{tabularx}

\begin{paracol}{2}
\selectlanguage{latin}
\lettrine[lines=2]{R}{omæ} sanctórum Mártyrum 
 Abdon et Sennen Persárum, qui, sub Décio, caténis alligáti et Romam addúcti, 
 pro Christi fide prius plumbátis cæsi sunt, deínde gládio interfécti.
\switchcolumn
\selectlanguage{english}
\lettrine[lines=2]{A}{t} Rome, in the reign of Decius, the 
 holy Persian martyrs Abdon and Sennen, who were bound with chains, brought 
 to Rome, scourged with leaded whips for the faith of Christ, and then put to 
 the sword.
\switchcolumn*
\selectlanguage{latin}
Assísii, in Umbria, 
 sancti Rufíni Mártyris.
\switchcolumn
\selectlanguage{english}
At Assisi in Umbria, St. Rufinus, 
 martyr.
\switchcolumn*
\selectlanguage{latin}
Tubúrbi Lucernáriæ, in 
 Africa, pássio sanctárum Vírginum Máximæ, Donatíllæ et Secúndæ; quarum duæ 
 primæ, in persecutióne Valeriáni et Galliéni, acéto et felle potátæ, deínde 
 plagis acérrimis cæsæ et equúlei extensióne cruciátæ, cratículus étiam 
 exústæ et calce perfricátæ, póstmodum cum Secúnda, Vírgine duódecim annórum, 
 ad béstias simul projéctæ, sed ab his intáctæ, novíssime gládio jugulátæ 
 sunt.
\switchcolumn
\selectlanguage{english}
At Tuberbum Lucernarium in Africa, 
 the holy virgins and martyrs Maxima, Donatilla, and Secunda. The first 
 two, in the persecution of Valerian and Gallienus, were forced to drink 
 vinegar and gall, then scourged most severely, stretched on the rack, 
 burned on the gridiron, rubbed over with lime, and afterwards exposed to the 
 beasts with the virgin Secunda, twelve years old. But being untouched 
 by them, they were finally beheaded.
\switchcolumn*
\selectlanguage{latin}
Cæsaréæ, in Cappadócia, sanctæ Julíttæ Mártyris, quæ, cum bona sua, a quodam poténte sibi usurpáta, 
 in judícia repéteret, atque ille exceptiónem daret quod ut Christiáno non 
 debéret audíri, mox a Júdice jussa est thus idólis offérre, ut posset audíri. 
 Quod illa constánter recúsans, in ignem conjécta est, sicque spíritum Deo 
 réddidit; corpus autem a flamma remánsit illæsum. Ejus præclárus 
 laudes sanctus Basilíus Magnus egrégio encómio celebrávit.
\switchcolumn
\selectlanguage{english}
At Caesarea in Cappadocia, St. 
 Julitta, martyr. As she sought through the courts the restitution of 
 goods seized by a man of influence, the latter objected that, being a 
 Christian, her cause could not be pleaded. The judge commanded her to 
 offer sacrifice to the idols, that she might be heard. She refused 
 with great constancy, and being thrown into the fire, yielded her soul unto 
 God. Her body remained uninjured by the flames. St. Basil the 
 Great has proclaimed her praise in an excellent eulogy.
\switchcolumn*
\selectlanguage{latin}
Antisiodóri sancti Ursi, 
 Epíscopi et Confessóris.
\switchcolumn
\selectlanguage{english}
At Auxerre, St. Ursus, bishop and 
 confessor.
\switchcolumn*
\selectlanguage{latin}
\end{paracol}


% ---- martyrology/mart07/mart0731.htm
\needspace{10\baselineskip}
\begin{paracol}{2}
\selectlanguage{latin}
\begin{center}{\color{gregoriocolor} Prídie Kaléndas Augústi. 
 Luna\dots\ }\end{center}
\switchcolumn
\selectlanguage{english}
\begin{center}{\color{gregoriocolor} The 
 Thirty-First Day of 
 July. The\dots\ Day of the Moon.}\end{center}
\end{paracol}

\noindent\begin{tabularx}{\linewidth}{*{19}{>{\centering\arraybackslash}X}}
 \textcolor{gregoriocolor}{a} & \textcolor{gregoriocolor}{b} & \textcolor{gregoriocolor}{c} & \textcolor{gregoriocolor}{d} & \textcolor{gregoriocolor}{e} & \textcolor{gregoriocolor}{f} & \textcolor{gregoriocolor}{g} & \textcolor{gregoriocolor}{h} & \textcolor{gregoriocolor}{i} & \textcolor{gregoriocolor}{k} & \textcolor{gregoriocolor}{l} & \textcolor{gregoriocolor}{m} & \textcolor{gregoriocolor}{n} & \textcolor{gregoriocolor}{p} & \textcolor{gregoriocolor}{q} & \textcolor{gregoriocolor}{r} & \textcolor{gregoriocolor}{s} & \textcolor{gregoriocolor}{t} & \textcolor{gregoriocolor}{u} \\
 6 & 7 & 8 & 9 & 10 & 11 & 12 & 13 & 14 & 15 & 16 & 17 & 18 & 19 & 20 & 21 & 22 & 23 & 24 \\
\end{tabularx}
\vspace{0.5\baselineskip}
\noindent\begin{tabularx}{\linewidth}{*{12}{>{\centering\arraybackslash}X}}
 \textcolor{gregoriocolor}{A} & \textcolor{gregoriocolor}{B} & \textcolor{gregoriocolor}{C} & \textcolor{gregoriocolor}{D} & \textcolor{gregoriocolor}{E} & F & \textcolor{gregoriocolor}{F} & \textcolor{gregoriocolor}{G} & \textcolor{gregoriocolor}{H} & \textcolor{gregoriocolor}{M} & \textcolor{gregoriocolor}{N} & \textcolor{gregoriocolor}{P} \\
 25 & 26 & 27 & 28 & 29 & 1 & 30 & 1 & 2 & 3 & 4 & 5 \\
\end{tabularx}

\begin{paracol}{2}
\selectlanguage{latin}
\lettrine[lines=2]{R}{omæ} natális sancti 
 Ignátii, Presbyteri et Confessóris, qui Fundátor éxstitit Societátis Jesu, 
 atque vir fuit sanctitáte et miráculis clarus, ac religiónis cathólicæ 
 ubíque dilatándæ studiosíssimus; quem Pius Undécimus, Póntifex Máximus, 
 cæléstem ómnium Exercitiórum spirituálium Patrónum constítuit.
\switchcolumn
\selectlanguage{english}
\lettrine[lines=2]{A}{t} Rome, the birthday of St. 
 Ignatius, priest and confessor, founder of the Society of Jesus, renowned 
 for sanctity and miracles, and most zealous for propagating the Catholic 
 religion in all parts of the world. Pope Pius XI declared him to be 
 the heavenly patron of all spiritual retreats.
\switchcolumn*
\selectlanguage{latin}
Medioláni sancti 
 Calimérii, Epíscopi et Mártyris; qui, in Antoníni persecutióne, comprehénsus, 
 vulnéribus confóssus, cervicibúsque gládio transverberátis, præceps in 
 púteum dejéctus, martyrii cursum confécit.
\switchcolumn
\selectlanguage{english}
At Milan, during the persecution of 
 Antoninus, St. Calimerius, bishop and martyr, who was arrested, covered with 
 wounds, and his throat transfixed with a sword. He completed his 
 martyrdom by being cast into a well.
\switchcolumn*
\selectlanguage{latin}
Cæsaréæ, in Mauritánia, pássio beáti Fábii Mártyris, qui, cum ferre vexílla præsidália recusáret, 
 primo in cárcerem trusus est, ibíque diébus áliquot deténtus; deínde, cum in 
 confessióne Christi, interrogátus semel et íterum, immóbilis perduráret, 
 senténtia capitáli a Júdice condemnátur.
\switchcolumn
\selectlanguage{english}
At Caesarea in Mauretania, the 
 martyrdom of the blessed martyr Fabius. Because he refused to carry 
 the banners of the governor of the province, he was thrown into prison for 
 some days, and as he persisted twice in confessing Christ when brought 
 before the judge, he was condemned to death.
\switchcolumn*
\selectlanguage{latin}
Synnadæ, in Phrygia 
 Pacatiána, sanctórum Mártyrum Demócriti, Secúndi et Dionysii.
\switchcolumn
\selectlanguage{english}
At Synnada in Phrygia Pacatiana, the 
 holy martyrs Democritus, Secundus, and Denis.
\switchcolumn*
\selectlanguage{latin}
In Syria sanctórum 
 trecentórum quinquagínta Monachórum Mártyrum, qui, ob defensiónem Synodi 
 Chalcedonénsis, ab hæréticis sunt occísi.
\switchcolumn
\selectlanguage{english}
In Syria, three hundred and fifty 
 monks, who became martyrs by being slain by the heretics for defending the 
 Council of Chalcedon.
\switchcolumn*
\selectlanguage{latin}
Ravénnæ tránsitus 
 sancti Germáni, Antisiodorénsis Epíscopi, génere, fide, doctrína et 
 miraculórum glória claríssimi; qui Británniam a Pelagianórum hærésibus 
 pénitus liberávit.
\switchcolumn
\selectlanguage{english}
At Ravenna, the death of St. 
 German, bishop of Auxerre, a man most renowned for his birth, faith, 
 learning, and glorious miracles, who freed England completely from the 
 heretical doctrines of the Pelagians.
\switchcolumn*
\selectlanguage{latin}
Tagáste, in Africa, 
 sancti Firmi Epíscopi, confessiónis glória conspícui.
\switchcolumn
\selectlanguage{english}
At Tagaste in Africa, St. Firmus, 
 bishop, illustrious by a glorious confession of the faith.
\switchcolumn*
\selectlanguage{latin}
Senis, in Túscia, 
 natális beáti Joánnis Columbíni, qui fuit Institútor Ordinis Jesuatórum, et 
 sanctitáte ac miráculis cláruit.
\switchcolumn
\selectlanguage{english}
At Siena in Tuscany, the birthday of 
 blessed John Colombini, founder of the Order of Gesuati, renowned for 
 sanctity and miracles.
\switchcolumn*
\selectlanguage{latin}
\end{paracol}

\setrunningtitles{Augustus}{August}

% ---- martyrology/mart08/mart0801.htm
\needspace{10\baselineskip}
\begin{paracol}{2}
\selectlanguage{latin}
\begin{center}{\color{gregoriocolor} Kaléndis Augústi. 
 Luna\dots\ }\end{center}
\switchcolumn
\selectlanguage{english}
\begin{center}{\color{gregoriocolor} The   First Day of 
 August. The\dots\ Day of the Moon.}\end{center}
\end{paracol}

\noindent\begin{tabularx}{\linewidth}{*{19}{>{\centering\arraybackslash}X}}
 \textcolor{gregoriocolor}{a} & \textcolor{gregoriocolor}{b} & \textcolor{gregoriocolor}{c} & \textcolor{gregoriocolor}{d} & \textcolor{gregoriocolor}{e} & \textcolor{gregoriocolor}{f} & \textcolor{gregoriocolor}{g} & \textcolor{gregoriocolor}{h} & \textcolor{gregoriocolor}{i} & \textcolor{gregoriocolor}{k} & \textcolor{gregoriocolor}{l} & \textcolor{gregoriocolor}{m} & \textcolor{gregoriocolor}{n} & \textcolor{gregoriocolor}{p} & \textcolor{gregoriocolor}{q} & \textcolor{gregoriocolor}{r} & \textcolor{gregoriocolor}{s} & \textcolor{gregoriocolor}{t} & \textcolor{gregoriocolor}{u} \\
 7 & 8 & 9 & 10 & 11 & 12 & 13 & 14 & 15 & 16 & 17 & 18 & 19 & 20 & 21 & 22 & 23 & 24 & 25 \\
\end{tabularx}
\vspace{0.5\baselineskip}
\noindent\begin{tabularx}{\linewidth}{*{12}{>{\centering\arraybackslash}X}}
 \textcolor{gregoriocolor}{A} & \textcolor{gregoriocolor}{B} & \textcolor{gregoriocolor}{C} & \textcolor{gregoriocolor}{D} & \textcolor{gregoriocolor}{E} & F & \textcolor{gregoriocolor}{F} & \textcolor{gregoriocolor}{G} & \textcolor{gregoriocolor}{H} & \textcolor{gregoriocolor}{M} & \textcolor{gregoriocolor}{N} & \textcolor{gregoriocolor}{P} \\
 26 & 27 & 28 & 29 & 1 & 2 & 1 & 2 & 3 & 4 & 5 & 6 \\
\end{tabularx}

\begin{paracol}{2}
\selectlanguage{latin}
\lettrine[lines=2]{R}{omæ,} in Exquíliis, 
 Dedicátio sancti Petri Apóstoli ad Víncula.
\switchcolumn
\selectlanguage{english}
\lettrine[lines=2]{A}{t} Rome, on the Esquiline, the 
 Dedication of the Church of St. Peter in Chains.
\switchcolumn*
\selectlanguage{latin}
Antiochíæ pássio 
 sanctórum septem fratrum Machabæórum Mártyrum, qui, cum matre sua, passi 
 sunt sub Antíocho Epíphane Rege. Eórum relíquiæ, Romam translátæ, in 
 eádem Ecclésia sancti Petri ad Víncula cónditæ fuérunt.
\switchcolumn
\selectlanguage{english}
At Antioch, the martyrdom of the 
 seven brothers, the holy Machabees, martyrs, and their mother, who suffered 
 under King Antiochus Epiphanes. Their relics were transferred to Rome, 
 and placed in the church or St. Peter in Chains.
\switchcolumn*
\selectlanguage{latin}
Vercéllis natális 
 sancti Eusébii, Epíscopi et Mártyris, qui, ob confessiónem fídei cathólicæ, 
 a Constántio Príncipe Scythópolim, in Palæstína, et inde in Cappadóciam 
 relegátus, póstmodum ad Ecclésiam suam revérsus, martyrium, persequéntibus 
 Ariánis, passus est. Ipsíus autem memória étiam décimo octávo Kaléndas 
 Januárii solémniter habétur, quo die fuit Epíscopus ordinátus; ejúsque 
 festum recólitur décimo séptimo Kaléndas Januárii.
\switchcolumn
\selectlanguage{english}
At Vercelli, St. Eusebius, bishop 
 and martyr, who, for the confession of the Catholic faith was banished to 
 Scythopolis in Palestine, and thence to Cappadocia, by Emperor Constantine. 
 Afterwards, returning to his church, he suffered martyrdom in the 
 persecution of the Arians. His memory is more especially honoured on 
 the 15th of December, when he was consecrated bishop, and his feast is kept 
 on the 16th of December.
\switchcolumn*
\selectlanguage{latin}
Nucériæ Paganórum, in 
 Campánia, item natális sancti Alfónsi-Maríæ de Ligório, Fundatóris 
 Congregatiónis a sanctíssimo Redemptóre nuncupátæ, Epíscopi sanctæ Agathæ 
 Gothórum et Confessóris, zelo animárum, scriptis, verbo et exémplo insígnis; 
 quem Summus Póntifex Gregórius Décimus sextus albo Sanctórum adscrípsit, et 
 Pius Nonus Doctórem universális Ecclésiæ declarávit, et Pius Duodécimus 
 ómnium Confessariórum ac Moralistárum cæléstem apud Deum Patrónum constítuit. 
 Ipsíus vero festívitas sequénti die celebrátur.
\switchcolumn
\selectlanguage{english}
At Nocera dei Pagani in Campani, the 
 birthday also of St. Alphonsus Maria Liguori, founder of the Congregation of 
 our most Holy Redeemer, bishop of Santa Agata dei Goti, and confessor. 
 Noted for his zeal for souls, his writings, and his example, Pope Gregory 
 XVI added him to the canon of saints, and Pius IX declared him to be a 
 doctor of the Universal Church. Pius XII established him as heavenly 
 patron of all moral theologians and of those who hear Confession. His 
 feast, however, is observed on the day following.
\switchcolumn*
\selectlanguage{latin}
Romæ, via Latína, sanctórum Mártyrum Boni Presbyteri, Fausti et Mauri, cum áliis novem; qui in 
 Actis sancti Stéphani Papæ describúntur.
\switchcolumn
\selectlanguage{english}
At Rome, on the Latin Way, the holy 
 martyrs Bonus, a priest, Faustus and Maur, with nine others, mentioned in 
 the Acts of Pope St. Stephen.
\switchcolumn*
\selectlanguage{latin}
Item Romæ pássio 
 sanctárum Vírginum Fídei, Spei et Caritátis, e sancta Sophía matre 
 progenitárum, quæ, sub Hadriáno Príncipe, martyrii corónam adéptæ sunt.
\switchcolumn
\selectlanguage{english}
Also at Rome, the holy virgins 
 Faith, Hope, and Charity, children of St. Sophia, who won the crown of 
 martyrdom under Emperor Hadrian.
\switchcolumn*
\selectlanguage{latin}
Philadelphíæ, in 
 Arábia, sanctórum Mártyrum Cyrílli, Aquilæ, Petri, Domitiáni, Rufi et 
 Menándri, una die coronatórum.
\switchcolumn
\selectlanguage{english}
At Philadelphia in Arabia, the holy 
 martyrs Cyril, Aquila, Peter, Domitian, Rufus, and Menander, crowned on the 
 same day.
\switchcolumn*
\selectlanguage{latin}
Perge, in Pamphylia, 
 sanctórum Mártyrum Leóntii, Attii, Alexándri, et aliórum sex agricolárum; 
 qui, in persecutióne Diocletiáni, sub Flaviáno Præside, decolláti sunt.
\switchcolumn
\selectlanguage{english}
At Perge in Pamphylia, the holy 
 martyrs Leontius, Attius, Alexander, and six peasants, who were beheaded in 
 the persecution of Diocletian, under the governor Flavian.
\switchcolumn*
\selectlanguage{latin}
Gerúndæ, in Hispánia, 
 natális sancti Felícis Mártyris, qui, post divérsa tormentórum génera, a Daciáno támdiu jussus est laniári, donec invíctum Christo spíritum rédderet.
\switchcolumn
\selectlanguage{english}
At Gerona in Spain, the birthday of 
 the holy martyr Felix. After enduring various torments, by order of 
 Dacian he was cut with knives until he gave his undaunted soul to Christ.
\switchcolumn*
\selectlanguage{latin}
In território Parisiénsi sancti Justíni Mártyris.
\switchcolumn
\selectlanguage{english}
In the diocese of Paris, St. Justin, 
 martyr.
\switchcolumn*
\selectlanguage{latin}
Viénnæ, in Gállia, 
 sancti Veri Epíscopi.
\switchcolumn
\selectlanguage{english}
At Vienne in France, St. Verus, 
 bishop.
\switchcolumn*
\selectlanguage{latin}
Wintóniæ, in Anglia, 
 sancti Ethelwóldi Epíscopi.
\switchcolumn
\selectlanguage{english}
At Winchester in England, St. 
 Ethelwold, bishop.
\switchcolumn*
\selectlanguage{latin}
In pago Lisuíno, in 
 Gállia, sancti Nemésii Confessóris.
\switchcolumn
\selectlanguage{english}
In the country of Lisieux, St. 
 Nemesius, confessor.
\switchcolumn*
\selectlanguage{latin}
\end{paracol}


% ---- martyrology/mart08/mart0802.htm
\needspace{10\baselineskip}
\begin{paracol}{2}
\selectlanguage{latin}
\begin{center}{\color{gregoriocolor} Quarto Nonas Augústi. 
 Luna\dots\ }\end{center}
\switchcolumn
\selectlanguage{english}
\begin{center}{\color{gregoriocolor} The   Second Day of 
 August. The\dots\ Day of the Moon.}\end{center}
\end{paracol}

\noindent\begin{tabularx}{\linewidth}{*{19}{>{\centering\arraybackslash}X}}
 \textcolor{gregoriocolor}{a} & \textcolor{gregoriocolor}{b} & \textcolor{gregoriocolor}{c} & \textcolor{gregoriocolor}{d} & \textcolor{gregoriocolor}{e} & \textcolor{gregoriocolor}{f} & \textcolor{gregoriocolor}{g} & \textcolor{gregoriocolor}{h} & \textcolor{gregoriocolor}{i} & \textcolor{gregoriocolor}{k} & \textcolor{gregoriocolor}{l} & \textcolor{gregoriocolor}{m} & \textcolor{gregoriocolor}{n} & \textcolor{gregoriocolor}{p} & \textcolor{gregoriocolor}{q} & \textcolor{gregoriocolor}{r} & \textcolor{gregoriocolor}{s} & \textcolor{gregoriocolor}{t} & \textcolor{gregoriocolor}{u} \\
 8 & 9 & 10 & 11 & 12 & 13 & 14 & 15 & 16 & 17 & 18 & 19 & 20 & 21 & 22 & 23 & 24 & 25 & 26 \\
\end{tabularx}
\vspace{0.5\baselineskip}
\noindent\begin{tabularx}{\linewidth}{*{12}{>{\centering\arraybackslash}X}}
 \textcolor{gregoriocolor}{A} & \textcolor{gregoriocolor}{B} & \textcolor{gregoriocolor}{C} & \textcolor{gregoriocolor}{D} & \textcolor{gregoriocolor}{E} & F & \textcolor{gregoriocolor}{F} & \textcolor{gregoriocolor}{G} & \textcolor{gregoriocolor}{H} & \textcolor{gregoriocolor}{M} & \textcolor{gregoriocolor}{N} & \textcolor{gregoriocolor}{P} \\
 27 & 28 & 29 & 1 & 2 & 3 & 2 & 3 & 4 & 5 & 6 & 7 \\
\end{tabularx}

\begin{paracol}{2}
\selectlanguage{latin}
\lettrine[lines=2]{S}{ancti} Alfónsi-Maríæ de 
 Ligório, Fundatóris Congregatiónis a sanctíssimo Redemptóre nuncupátæ, 
 Epíscopi sanctæ Agathæ Gothórum, Confessóris et Ecclésiæ Doctóris, qui 
 requiévit in Dómino prídie hujus diéi.
\switchcolumn
\selectlanguage{english}
\lettrine[lines=2]{S}{t.} Alphonsus Maria Liguori, founder 
 of the Congregation of our most Holy Redeemer, bishop of Santa Agata dei 
 Goti, confessor and doctor of the Church, who fell asleep in the Lord on the 
 previous day.
\switchcolumn*
\selectlanguage{latin}
Romæ, in cœmetério 
 Callísti, natális sancti Stéphani Primi, Papæ et Mártyris; qui, in 
 persecutióne Valeriáni, dum Missæ Sacrum perágeret, et, superveniéntibus 
 milítibus, ante altáre intrépidus et immóbilis cœpta mystéria perfíceret, in 
 sede sua decollátus est.
\switchcolumn
\selectlanguage{english}
At Rome, in the cemetery of 
 Callistus, the birthday of St. Stephen I, pope and martyr. In the 
 persecution of Valerian, the soldiers suddenly entered while he was saying 
 Mass, but remaining before the altar, fearless and unmoved, he concluded the 
 sacred mysteries, and was beheaded on his throne.
\switchcolumn*
\selectlanguage{latin}
Nicææ, in Bithynia, 
 pássio sanctæ Theódotæ, cum tribus fíliis suis. Ex his primogénitum, 
 nómine Evódium, cum Christum fiduciáliter confiterétur, fecit primo Nicétius, 
 Consuláris Bithyniæ, fústibus cædi; deínde matrem, cum ómnibus fíliis, igne 
 consúmi.
\switchcolumn
\selectlanguage{english}
At Nicaea in Bithynia, the martyrdom 
 of St. Theodota with her three sons. The eldest named Evodius, 
 confessing Christ with confidence, was first beaten with rods by order of 
 Nicetius, exconsul of Bithynia, and then the mother with all her sons, was 
 consumed by fire.
\switchcolumn*
\selectlanguage{latin}
In Africa sancti 
 Rutílii Mártyris, qui cum sæpius, de loco in locum fúgiens, persecutiónem 
 declinásset, et perículum intérdum étiam pecúnia redemísset, ex inopináto 
 aliquándo comprehénsus est, et, Præsidi oblátus, torméntis cruciátur 
 plúrimis; demum, ígnibus tráditus, egrégio martyrio coronátur.
\switchcolumn
\selectlanguage{english}
In Africa, St. Rutilius, martyr. 
 He had frequently secured safety from the perils of persecution by flight, 
 and sometimes even by means of money, but at last, being unexpectedly 
 apprehended, he was led to the governor and subjected to many tortures. 
 Afterwards he was cast into the fire, and thus merited the glorious crown of 
 martyrdom.
\switchcolumn*
\selectlanguage{latin}
Patávii sancti Máximi, 
 ejúsdem civitátis Epíscopi, qui, miráculis clarus, beáto fine quiévit.
\switchcolumn
\selectlanguage{english}
At Padua, St. Maximus, bishop of 
 that city, who ended his blessed life in peace, with a reputation for 
 miracles.
\switchcolumn*
\selectlanguage{latin}
\end{paracol}


% ---- martyrology/mart08/mart0803.htm
\needspace{10\baselineskip}
\begin{paracol}{2}
\selectlanguage{latin}
\begin{center}{\color{gregoriocolor} Tértio Nonas Augústi. 
 Luna\dots\ }\end{center}
\switchcolumn
\selectlanguage{english}
\begin{center}{\color{gregoriocolor} The   Third Day of 
 August. The\dots\ Day of the Moon.}\end{center}
\end{paracol}

\noindent\begin{tabularx}{\linewidth}{*{19}{>{\centering\arraybackslash}X}}
 \textcolor{gregoriocolor}{a} & \textcolor{gregoriocolor}{b} & \textcolor{gregoriocolor}{c} & \textcolor{gregoriocolor}{d} & \textcolor{gregoriocolor}{e} & \textcolor{gregoriocolor}{f} & \textcolor{gregoriocolor}{g} & \textcolor{gregoriocolor}{h} & \textcolor{gregoriocolor}{i} & \textcolor{gregoriocolor}{k} & \textcolor{gregoriocolor}{l} & \textcolor{gregoriocolor}{m} & \textcolor{gregoriocolor}{n} & \textcolor{gregoriocolor}{p} & \textcolor{gregoriocolor}{q} & \textcolor{gregoriocolor}{r} & \textcolor{gregoriocolor}{s} & \textcolor{gregoriocolor}{t} & \textcolor{gregoriocolor}{u} \\
 9 & 10 & 11 & 12 & 13 & 14 & 15 & 16 & 17 & 18 & 19 & 20 & 21 & 22 & 23 & 24 & 25 & 26 & 27 \\
\end{tabularx}
\vspace{0.5\baselineskip}
\noindent\begin{tabularx}{\linewidth}{*{12}{>{\centering\arraybackslash}X}}
 \textcolor{gregoriocolor}{A} & \textcolor{gregoriocolor}{B} & \textcolor{gregoriocolor}{C} & \textcolor{gregoriocolor}{D} & \textcolor{gregoriocolor}{E} & F & \textcolor{gregoriocolor}{F} & \textcolor{gregoriocolor}{G} & \textcolor{gregoriocolor}{H} & \textcolor{gregoriocolor}{M} & \textcolor{gregoriocolor}{N} & \textcolor{gregoriocolor}{P} \\
 28 & 29 & 1 & 2 & 3 & 4 & 3 & 4 & 5 & 6 & 7 & 8 \\
\end{tabularx}

\begin{paracol}{2}
\selectlanguage{latin}
\lettrine[lines=2]{H}{ierosólymis} Invéntio 
 beatíssimi Stéphani Protomártyris, et sanctórum Gamaliélis, Nicodémi et 
 Abibónis, sicut Luciáno Presbytero divínitus revelátum est, Honórii 
 Príncipis témpore.
\switchcolumn
\selectlanguage{english}
\lettrine[lines=2]{A}{t} Jerusalem, the finding of the 
 body of blessed Stephen, protomartyr, and of the Saints Gamaliel, Nicodemus, 
 and Abibo, through a divine revelation made to the priest Lucian, in the 
 time of Emperor Honorius.
\switchcolumn*
\selectlanguage{latin}
Philíppis, in 
 Macedónia, sanctæ Lydiæ purpuráriæ, quæ, prædicánte ibídem sancto Paulo 
 Apóstolo, ut beátus Lucas in Actibus Apostolórum refert, ómnium prima 
 crédidit Evangélio.
\switchcolumn
\selectlanguage{english}
At Philippi in Macedonia, St. Lydia, a dealer in purple, 
 who was the first to believe in the Gospel when the apostle St. Paul 
 preached in that city, as is related by St. Luke in the Acts of the Apostles
\switchcolumn*
\selectlanguage{latin}
Neápoli, in Campánia, sancti Aspréni Epíscopi, qui a sancto Petro Apóstolo, curátus ab infirmitáte 
 ac deínde baptizátus, ejúsdem civitátis Epíscopus ordinátus fuit.
\switchcolumn
\selectlanguage{english}
At Naples in Campania, St. Aspren, 
 bishop, who was cured of a sickness by the apostle St. Peter, and after 
 being baptized, was made bishop of that city.
\switchcolumn*
\selectlanguage{latin}
Constantinópoli natális 
 sancti Hermélli Mártyris.
\switchcolumn
\selectlanguage{english}
At Constantinople, the birthday of 
 St. Hermellus, martyr.
\switchcolumn*
\selectlanguage{latin}
Apud Indos, Persis 
 finítimos, pássio sanctórum Monachórum, et aliórum fidélium, quos Abénner 
 Rex, pérsequens Ecclésiam Dei, divérsis afflíctos supplíciis, cædi jussit.
\switchcolumn
\selectlanguage{english}
Among the Indians, bordering on 
 Persia, the martyrdom of holy monks and other Christians who were put to 
 death after suffering diverse torments, during the persecution of the Church 
 of God by King Abenner.
\switchcolumn*
\selectlanguage{latin}
Augustodúni deposítio 
 sancti Euphrónii, Epíscopi et Confessóris.
\switchcolumn
\selectlanguage{english}
At Autun, the death of St. 
 Euphronius, bishop and confessor.
\switchcolumn*
\selectlanguage{latin}
Anágniæ sancti Petri 
 Epíscopi, qui, monástica primum observántia, deínde pastoráli vigilántia 
 clarus, quiévit in Dómino.
\switchcolumn
\selectlanguage{english}
At Anagni, St. Peter, who rested in 
 the Lord after gaining great renown for monastical observance and for 
 pastoral vigilance.
\switchcolumn*
\selectlanguage{latin}
BerϾ, in Syria, 
 sanctárum mulíerum Maránæ et Cyræ.
\switchcolumn
\selectlanguage{english}
At Beroea in Syria, the holy women 
 Marana and Cyra.
\switchcolumn*
\selectlanguage{latin}
\end{paracol}


% ---- martyrology/mart08/mart0804.htm
\needspace{10\baselineskip}
\begin{paracol}{2}
\selectlanguage{latin}
\begin{center}{\color{gregoriocolor} Prídie Nonas Augústi. 
 Luna\dots\ }\end{center}
\switchcolumn
\selectlanguage{english}
\begin{center}{\color{gregoriocolor} The   Fourth Day of 
 August. The\dots\ Day of the Moon.}\end{center}
\end{paracol}

\noindent\begin{tabularx}{\linewidth}{*{19}{>{\centering\arraybackslash}X}}
 \textcolor{gregoriocolor}{a} & \textcolor{gregoriocolor}{b} & \textcolor{gregoriocolor}{c} & \textcolor{gregoriocolor}{d} & \textcolor{gregoriocolor}{e} & \textcolor{gregoriocolor}{f} & \textcolor{gregoriocolor}{g} & \textcolor{gregoriocolor}{h} & \textcolor{gregoriocolor}{i} & \textcolor{gregoriocolor}{k} & \textcolor{gregoriocolor}{l} & \textcolor{gregoriocolor}{m} & \textcolor{gregoriocolor}{n} & \textcolor{gregoriocolor}{p} & \textcolor{gregoriocolor}{q} & \textcolor{gregoriocolor}{r} & \textcolor{gregoriocolor}{s} & \textcolor{gregoriocolor}{t} & \textcolor{gregoriocolor}{u} \\
 10 & 11 & 12 & 13 & 14 & 15 & 16 & 17 & 18 & 19 & 20 & 21 & 22 & 23 & 24 & 25 & 26 & 27 & 28 \\
\end{tabularx}
\vspace{0.5\baselineskip}
\noindent\begin{tabularx}{\linewidth}{*{12}{>{\centering\arraybackslash}X}}
 \textcolor{gregoriocolor}{A} & \textcolor{gregoriocolor}{B} & \textcolor{gregoriocolor}{C} & \textcolor{gregoriocolor}{D} & \textcolor{gregoriocolor}{E} & F & \textcolor{gregoriocolor}{F} & \textcolor{gregoriocolor}{G} & \textcolor{gregoriocolor}{H} & \textcolor{gregoriocolor}{M} & \textcolor{gregoriocolor}{N} & \textcolor{gregoriocolor}{P} \\
 29 & 1 & 2 & 3 & 4 & 5 & 4 & 5 & 6 & 7 & 8 & 9 \\
\end{tabularx}

\begin{paracol}{2}
\selectlanguage{latin}
\lettrine[lines=2]{S}{ancti} Domínici 
 Confessóris, qui Ordinis Fratrum Prædicatórum Fundátor fuit, atque octávo 
 Idus mensis hujus in pace quiévit.
\switchcolumn
\selectlanguage{english}
\lettrine[lines=2]{S}{t.} Dominic, confessor, founder of 
 the Order of Friars Preachers, who on the sixth day of this month rested in 
 peace.
\switchcolumn*
\selectlanguage{latin}
In vico Ars, diœcésis 
 Bellicénsis, in Gállia, natális sancti Joánnis Baptístæ-Maríæ Vianney, 
 Presbyteri et Confessóris, in parochiáli múnere obeúndo insígnis, quem Pius 
 Papa Undécimus in Sanctórum númerum rétulit; ipsíus festum quinto Idus mensis hujus agéndum indíxit, eúmque ómnium parochórum cæléstem Patrónum 
 constítuit.
\switchcolumn
\selectlanguage{english}
In the village of Ars, in the 
 diocese of Belley, France, the birthday of St. John Baptist-Mary Vianney, priest and 
 confessor, renowned for his devotion as a parish priest. Pope Pius XI 
 placed him in the number of the saints, ordered that his feast should be 
 observed on the 9th day of this month, and appointed him as the heavenly 
 patron of all parish priests.
\switchcolumn*
\selectlanguage{latin}
Thessalonícæ item 
 natális beáti Aristárchi, qui discípulus et comes indivíduus fuit sancti 
 Pauli Apóstoli, de quo ipse Paulus ad Colossénses scribit: « Salútat vos 
 Aristárchus, concaptívus meus ». Is, ab eódem Apóstolo Epíscopus 
 Thessalonicénsium ordinátus, tandem, post longos agónes, sub Neróne, 
 coronátus a Christo, quiévit.
\switchcolumn
\selectlanguage{english}
At Thessalonica, the birthday of 
 blessed Aristarchus, disciple and inseparable companion of the apostle St. 
 Paul, who writes to the Colossians: ``Aristarchus my fellow-prisoner saluteth 
 you.'' He was consecrated bishop of the Thessalonians by the same 
 apostle, and after long sufferings under Nero, crowned by Christ, rested in 
 peace.
\switchcolumn*
\selectlanguage{latin}
Romæ sanctæ Perpétuæ, 
 quæ, a beáto Petro Apóstolo baptizáta, Nazárium fílium et Africánum virum ad 
 Christi fidem perdúxit, et multa sanctórum Mártyrum córpora sepelívit; ac 
 tandem, bonórum óperum méritis cumuláta, migrávit ad Dóminum.
\switchcolumn
\selectlanguage{english}
At Rome, St. Perpetua, who was 
 baptized by the blessed apostle Peter. She converted to the faith her 
 son Nazarius and her husband Africanus, buried the remains of many holy 
 martyrs, and finally went to our Lord endowed with an abundance of merit.
\switchcolumn*
\selectlanguage{latin}
Item Romæ, via Latína, pássio beáti Tertullíni, Presbyteri et Mártyris; qui, sub Valeriáno 
 Imperatóre, post ímpiam fústium mactatiónem, ígnium circa látera exustiónem, 
 oris quassatiónem, atque in equúleo extensiónem nervorúmque cæsiónem, data 
 senténtia, cápitis amputatióne martyrium consummávit.
\switchcolumn
\selectlanguage{english}
At Rome, on the Latin Way, the 
 martyrdom of blessed Tertullinus, priest and martyr, in the time of Emperor 
 Valerian. After being cruelly beaten with rods, after having his sides 
 burned, his mouth shattered; after being stretched on the rack and his limbs 
 crushed, he completed his martyrdom by being beheaded.
\switchcolumn*
\selectlanguage{latin}
Constantinópoli sancti 
 Eleuthérii Mártyris, ex órdine Senatório viri, qui pro Christo, in 
 persecutióne Maximiáni, gládio cæsus est.
\switchcolumn
\selectlanguage{english}
At Constantinople, the holy martyr 
 Eleutherius, of the senatorial rank, who was put to the sword for Christ in 
 the persecution of Maximian.
\switchcolumn*
\selectlanguage{latin}
In Pérside sanctárum 
 Mártyrum Iæ et Sociárum; quæ, cum novem míllibus Christiánis captívis, sub 
 Sápore Rege, divérsis pœnis afflíctæ, martyrium subiérunt.
\switchcolumn
\selectlanguage{english}
In Persia, in the time of King Sapor, 
 the holy martyr Ia and her companions, who, with nine thousand Christian 
 captives, underwent martyrdom after having been subjected to various 
 torments.
\switchcolumn*
\selectlanguage{latin}
Verónæ sancti Agábii, 
 Epíscopi et Confessóris.
\switchcolumn
\selectlanguage{english}
At Verona, St. Agabius, bishop and 
 confessor.
\switchcolumn*
\selectlanguage{latin}
Turónis, in Gállia, 
 sancti Euphrónii Epíscopi.
\switchcolumn
\selectlanguage{english}
At Tours in France, St. Euphronius, bishop.
\switchcolumn*
\selectlanguage{latin}
Colóniæ Agrippínæ 
 commemorátio sancti Protásii Mártyris; qui Medioláni, una cum Gervásio 
 fratre, passus est tertiodécimo Kaléndas Júlii.
\switchcolumn
\selectlanguage{english}
At Cologne, the commemoration of St. 
 Protase, martyr. In company with his brother Gervase, he suffered at 
 Milan on the 19th of June.
\switchcolumn*
\selectlanguage{latin}
\end{paracol}


% ---- martyrology/mart08/mart0805.htm
\needspace{10\baselineskip}
\begin{paracol}{2}
\selectlanguage{latin}
\begin{center}{\color{gregoriocolor} Nonis Augústi. 
 Luna\dots\ }\end{center}
\switchcolumn
\selectlanguage{english}
\begin{center}{\color{gregoriocolor} The   Fifth Day of 
 August. The\dots\ Day of the Moon.}\end{center}
\end{paracol}

\noindent\begin{tabularx}{\linewidth}{*{19}{>{\centering\arraybackslash}X}}
 \textcolor{gregoriocolor}{a} & \textcolor{gregoriocolor}{b} & \textcolor{gregoriocolor}{c} & \textcolor{gregoriocolor}{d} & \textcolor{gregoriocolor}{e} & \textcolor{gregoriocolor}{f} & \textcolor{gregoriocolor}{g} & \textcolor{gregoriocolor}{h} & \textcolor{gregoriocolor}{i} & \textcolor{gregoriocolor}{k} & \textcolor{gregoriocolor}{l} & \textcolor{gregoriocolor}{m} & \textcolor{gregoriocolor}{n} & \textcolor{gregoriocolor}{p} & \textcolor{gregoriocolor}{q} & \textcolor{gregoriocolor}{r} & \textcolor{gregoriocolor}{s} & \textcolor{gregoriocolor}{t} & \textcolor{gregoriocolor}{u} \\
 11 & 12 & 13 & 14 & 15 & 16 & 17 & 18 & 19 & 20 & 21 & 22 & 23 & 24 & 25 & 26 & 27 & 28 & 29 \\
\end{tabularx}
\vspace{0.5\baselineskip}
\noindent\begin{tabularx}{\linewidth}{*{12}{>{\centering\arraybackslash}X}}
 \textcolor{gregoriocolor}{A} & \textcolor{gregoriocolor}{B} & \textcolor{gregoriocolor}{C} & \textcolor{gregoriocolor}{D} & \textcolor{gregoriocolor}{E} & F & \textcolor{gregoriocolor}{F} & \textcolor{gregoriocolor}{G} & \textcolor{gregoriocolor}{H} & \textcolor{gregoriocolor}{M} & \textcolor{gregoriocolor}{N} & \textcolor{gregoriocolor}{P} \\
 1 & 2 & 3 & 4 & 5 & 6 & 5 & 6 & 7 & 8 & 9 & 10 \\
\end{tabularx}

\begin{paracol}{2}
\selectlanguage{latin}
\lettrine[lines=2]{R}{omæ,} in Exquíliis, 
 Dedicátio Basílicæ sanctæ Maríæ ad Nives.
\switchcolumn
\selectlanguage{english}
\lettrine[lines=2]{A}{t} Rome, on the Esquiline, the 
 Dedication of the Basilica of St. Mary of the Snows.
\switchcolumn*
\selectlanguage{latin}
Cataláuni, in Gállia, 
 sancti Mémmii, civis Románi, qui, a sancto Petro Apóstolo consecrátus illíus 
 civitátis Epíscopus, pópulum sibi commíssum ad Evangélii veritátem perdúxit.
\switchcolumn
\selectlanguage{english}
At Chalons in France, St. Memmius, a 
 Roman citizen, who was consecrated bishop of that city by St. Peter the 
 Apostle, and brought to the truth of the Gospel the people committed to his 
 care.
\switchcolumn*
\selectlanguage{latin}
Romæ pássio sanctórum 
 Mártyrum vigínti trium, qui, in persecutióne Diocletiáni, via Salária véteri, 
 cápite obtruncáti sunt, et, ad clivum Cucúmeris, ibídem sepúlti.
\switchcolumn
\selectlanguage{english}
At Rome, during the persecution of 
 Diocletian, the martyrdom of twenty-three holy martyrs, who were beheaded on 
 the Salarian Way, and buried at the foot of Cucumer Hill.
\switchcolumn*
\selectlanguage{latin}
Asculi, in Picéno, sancti Emygdii, Epíscopi et Mártyris; qui, a sancto Marcéllo Papa Epíscopus 
 ordinátus et illuc ad prædicándum Evangélium missus, ibídem, in confessióne 
 Christi, sub Diocletiáno Imperatóre, martyrii corónam accépit.
\switchcolumn
\selectlanguage{english}
At Ascoli in Piceno, St. Emygdius, 
 bishop and martyr, who was consecrated bishop by Pope St. Marcellus, and 
 sent thither to preach the Gospel. He received the crown of martyrdom 
 for the confession of Christ under Emperor Diocletian.
\switchcolumn*
\selectlanguage{latin}
Antiochíæ sancti 
 Eusígnii mílitis, qui, annum agens centésimum décimum, cum Constantíni Magni 
 fidem, sub quo militáverat, Juliáno Apóstatæ exprobráret, eúmque ut pátriæ 
 pietátis desertórem redargúeret, ab eódem jussus est cápite cædi.
\switchcolumn
\selectlanguage{english}
At Antioch, St. Eusignius, a 
 soldier, who, at the age of one hundred and ten years, because he reproached 
 Julian the Apostate for forsaking the faith of Constantine the Great, under 
 whom he had served, and for having degenerated from his ancestor's piety, 
 was beheaded at his command.
\switchcolumn*
\selectlanguage{latin}
Item sanctórum Mártyrum 
 Ægyptiórum Cantídii, Cantidiáni et Sobélis.
\switchcolumn
\selectlanguage{english}
Also the holy martyrs Cantidius, 
 Cantidian, and Sobel, Egyptians.
\switchcolumn*
\selectlanguage{latin}
Augústæ Vindelicórum 
 natális sanctæ Afræ Mártyris, quæ, cum esset pagána, per doctrínam sancti 
 Narcíssi Epíscopi ad Christum est convérsa, et, cum ómnibus domus suæ 
 membris, ab eódem Epíscopo baptizáta; póstmodum vero, ob Christi 
 confessiónem igni trádita, martyrium suum, septem diébus ántequam beáta 
 Hilária mater ac tres ancíllæ eódem cruciátus génere coronaréntur, felíciter 
 explévit.
\switchcolumn
\selectlanguage{english}
At Augsburg, the birthday of St. 
 Afra, martyr, who being a pagan, was converted to Christ by the teaching of 
 St. Narcissus the bishop, and being baptized with all her household, was 
 given over to the flames for the sake of Christ. Seven days later her 
 mother Hilaria and three handmaids were also crowned by enduring the same 
 kind of torment.
\switchcolumn*
\selectlanguage{latin}
Augustodúni beáti 
 Cassiáni Epíscopi.
\switchcolumn
\selectlanguage{english}
At Autun, blessed Cassian, bishop.
\switchcolumn*
\selectlanguage{latin}
Apud Theánum, in 
 Campánia, sancti Páridis Epíscopi.
\switchcolumn
\selectlanguage{english}
At Teano in Campania, St. Paris, 
 bishop.
\switchcolumn*
\selectlanguage{latin}
In Anglia sancti 
 Oswáldi Regis, cujus gesta sanctus Beda Venerábilis commémorat.
\switchcolumn
\selectlanguage{english}
In England, St. Oswald, king, whose 
 life is related by St. Venerable Bede.
\switchcolumn*
\selectlanguage{latin}
Eódem die sanctæ Nonnæ, 
 quæ fuit mater beatórum Gregórii Nazianzéni, Cæsárii et Gorgoniæ.
\switchcolumn
\selectlanguage{english}
On the same day, St. Nonna, mother 
 of Saints Gregory Nazianzen, Caesarius, and Gorgonia.
\switchcolumn*
\selectlanguage{latin}
\end{paracol}


% ---- martyrology/mart08/mart0806.htm
\needspace{10\baselineskip}
\begin{paracol}{2}
\selectlanguage{latin}
\begin{center}{\color{gregoriocolor} Octávo Idus Augústi. 
 Luna\dots\ }\end{center}
\switchcolumn
\selectlanguage{english}
\begin{center}{\color{gregoriocolor} The   Sixth Day of 
 August. The\dots\ Day of the Moon.}\end{center}
\end{paracol}

\noindent\begin{tabularx}{\linewidth}{*{19}{>{\centering\arraybackslash}X}}
 \textcolor{gregoriocolor}{a} & \textcolor{gregoriocolor}{b} & \textcolor{gregoriocolor}{c} & \textcolor{gregoriocolor}{d} & \textcolor{gregoriocolor}{e} & \textcolor{gregoriocolor}{f} & \textcolor{gregoriocolor}{g} & \textcolor{gregoriocolor}{h} & \textcolor{gregoriocolor}{i} & \textcolor{gregoriocolor}{k} & \textcolor{gregoriocolor}{l} & \textcolor{gregoriocolor}{m} & \textcolor{gregoriocolor}{n} & \textcolor{gregoriocolor}{p} & \textcolor{gregoriocolor}{q} & \textcolor{gregoriocolor}{r} & \textcolor{gregoriocolor}{s} & \textcolor{gregoriocolor}{t} & \textcolor{gregoriocolor}{u} \\
 12 & 13 & 14 & 15 & 16 & 17 & 18 & 19 & 20 & 21 & 22 & 23 & 24 & 25 & 26 & 27 & 28 & 29 & 1 \\
\end{tabularx}
\vspace{0.5\baselineskip}
\noindent\begin{tabularx}{\linewidth}{*{12}{>{\centering\arraybackslash}X}}
 \textcolor{gregoriocolor}{A} & \textcolor{gregoriocolor}{B} & \textcolor{gregoriocolor}{C} & \textcolor{gregoriocolor}{D} & \textcolor{gregoriocolor}{E} & F & \textcolor{gregoriocolor}{F} & \textcolor{gregoriocolor}{G} & \textcolor{gregoriocolor}{H} & \textcolor{gregoriocolor}{M} & \textcolor{gregoriocolor}{N} & \textcolor{gregoriocolor}{P} \\
 2 & 3 & 4 & 5 & 6 & 7 & 6 & 7 & 8 & 9 & 10 & 11 \\
\end{tabularx}

\begin{paracol}{2}
\selectlanguage{latin}
\lettrine[lines=2]{I}{n} monte Thabor 
 Transfigurátio Dómini nostri Jesu Christi.
\switchcolumn
\selectlanguage{english}
\lettrine[lines=2]{O}{n} Mount Tabor, the Transfiguration 
 of our Lord Jesus Christ.
\switchcolumn*
\selectlanguage{latin}
Romæ, via Appia, in 
 cœmetério Callísti, natális beáti Xysti Secúndi, Papæ et Mártyris, qui in 
 persecutióne Valeriáni, gládio animadvérsus, martyrii corónam accépit.
\switchcolumn
\selectlanguage{english}
At Rome, on the Appian Way, in the 
 cemetery of Callistus, the birthday of blessed Sixtus II, pope and martyr, 
 who received the crown of martyrdom in the persecution of Valerian by being 
 put to the sword.
\switchcolumn*
\selectlanguage{latin}
Item Romæ sanctórum 
 Mártyrum Felicíssimi et Agapíti, ipsíus beáti Xysti Diaconórum; Januárii, 
 Magni, Vincéntii et Stéphani, Subdiaconórum. Hi omnes, una cum eódem 
 Pontífice páriter decolláti sunt, atque in Prætextáti cœmetério sepúlti. 
 Passus est étiam cum eis beátus Quartus, ut scribit sanctus Cypriánus.
\switchcolumn
\selectlanguage{english}
Also, the holy martyrs Felicissimus 
 and Agapitus, deacons of blessed Sixtus; Januarius, Magnus, Vincent, and 
 Stephen, subdeacons, all of whom were beheaded with him and buried in the 
 cemetery of Praetextatus. With them suffered also blessed Quartus, as 
 is related by St. Cyprian.
\switchcolumn*
\selectlanguage{latin}
Bonóniæ natális sancti 
 Domínici Confessóris, qui Ordinis Fratrum Prædicatórum Fundátor éxstitit. 
 Hic vir, sanctitáte et doctrína claríssimus, virginitátem perpétuo illibátam 
 custodívit, et, ob singulárem meritórum grátiam, tres mórtuos suscitávit; cumque prædicatióne sua compressísset hæreses, ac plúrimos ad religiósam et 
 piam vitam instituísset, in pace quiévit. Ejus autem festívitas prídie 
 Nonas mensis hujus celebrátur, ex constitutióne Pauli Papæ Quarti.
\switchcolumn
\selectlanguage{english}
At Bologna, the birthday of St. 
 Dominic, confessor, founder of the Order of Friars Preachers, most renowned 
 for sanctity and learning. He preserved his chastity unsullied to the 
 end of his life, and by his great merits raised three persons from the dead. 
 After having repressed heresies by his preaching, and instructed many in the 
 religious and godly life, he rested in peace. His feast is celebrated 
 on the 4th of August by decree of Pope Paul IV.
\switchcolumn*
\selectlanguage{latin}
In monastério sancti 
 Petri de Cardégna, Ordinis sancti Benedícti, apud Burgos, in Hispánia, 
 pássio ducentórum Monachórum cum Stéphano Abbáte, qui a Saracénis pro Jesu 
 Christi fide interfécti sunt, atque ibídem in claustro a Christiánis sepúlti.
\switchcolumn
\selectlanguage{english}
At Burgos in Spain, in the monastery 
 of St. Peter of Cardegna, of the Order of St. Benedict, two hundred monks, 
 with their abbot Stephen, who were put to death for the faith of Christ by 
 the Saracens, and buried in the monastery by the Christians.
\switchcolumn*
\selectlanguage{latin}
Complúti, in Hispánia, 
 sanctórum Mártyrum Justi et Pastóris fratrum, qui, cum adhuc púeri lítteris 
 imbueréntur, sponte ad martyrium, projéctis in schola tábulis, cucurrérunt; 
 et mox, a Daciáno Præside tenéri jussi et fústibus cædi, ambo, extra 
 civitátem perdúcti sunt, et ibi a carnífice juguláti.
\switchcolumn
\selectlanguage{english}
At Alcala in Spain, the holy martyrs 
 Justus and Pastor, brothers. While they were yet schoolboys, they 
 threw aside their books in school, and spontaneously ran to martyrdom. 
 By order of the governor Dacian, they were arrested, beaten with rods, and 
 as they exhorted each other to constancy, were led out of the city, and had 
 their throats cut by the executioner.
\switchcolumn*
\selectlanguage{latin}
Romæ sancti Hormísdæ, 
 Papæ et Confessóris.
\switchcolumn
\selectlanguage{english}
At Rome, St. Hormisdas, pope and 
 confessor.
\switchcolumn*
\selectlanguage{latin}
Amidæ, in Mesopotámia, 
 sancti Jacóbi Eremítæ, miráculis clari.
\switchcolumn
\selectlanguage{english}
At Amida in Mesopotamia, St. James, a hermit 
 renowned for miracles.
\switchcolumn*
\selectlanguage{latin}
\end{paracol}


% ---- martyrology/mart08/mart0807.htm
\needspace{10\baselineskip}
\begin{paracol}{2}
\selectlanguage{latin}
\begin{center}{\color{gregoriocolor} Séptimo Idus Augústi. 
 Luna\dots\ }\end{center}
\switchcolumn
\selectlanguage{english}
\begin{center}{\color{gregoriocolor} The   Seventh Day of 
 August. The\dots\ Day of the Moon.}\end{center}
\end{paracol}

\noindent\begin{tabularx}{\linewidth}{*{19}{>{\centering\arraybackslash}X}}
 \textcolor{gregoriocolor}{a} & \textcolor{gregoriocolor}{b} & \textcolor{gregoriocolor}{c} & \textcolor{gregoriocolor}{d} & \textcolor{gregoriocolor}{e} & \textcolor{gregoriocolor}{f} & \textcolor{gregoriocolor}{g} & \textcolor{gregoriocolor}{h} & \textcolor{gregoriocolor}{i} & \textcolor{gregoriocolor}{k} & \textcolor{gregoriocolor}{l} & \textcolor{gregoriocolor}{m} & \textcolor{gregoriocolor}{n} & \textcolor{gregoriocolor}{p} & \textcolor{gregoriocolor}{q} & \textcolor{gregoriocolor}{r} & \textcolor{gregoriocolor}{s} & \textcolor{gregoriocolor}{t} & \textcolor{gregoriocolor}{u} \\
 13 & 14 & 15 & 16 & 17 & 18 & 19 & 20 & 21 & 22 & 23 & 24 & 25 & 26 & 27 & 28 & 29 & 1 & 2 \\
\end{tabularx}
\vspace{0.5\baselineskip}
\noindent\begin{tabularx}{\linewidth}{*{12}{>{\centering\arraybackslash}X}}
 \textcolor{gregoriocolor}{A} & \textcolor{gregoriocolor}{B} & \textcolor{gregoriocolor}{C} & \textcolor{gregoriocolor}{D} & \textcolor{gregoriocolor}{E} & F & \textcolor{gregoriocolor}{F} & \textcolor{gregoriocolor}{G} & \textcolor{gregoriocolor}{H} & \textcolor{gregoriocolor}{M} & \textcolor{gregoriocolor}{N} & \textcolor{gregoriocolor}{P} \\
 3 & 4 & 5 & 6 & 7 & 8 & 7 & 8 & 9 & 10 & 11 & 12 \\
\end{tabularx}

\begin{paracol}{2}
\selectlanguage{latin}
\lettrine[lines=2]{N}{eápoli,} in Campánia, sancti Cajetáni Thienǽi Confessóris, Clericórum Regulárium Fundatóris, qui, 
 singulári in Deum fidúcia, prístinam Apostólicam vivéndi formam suis 
 coléndam trádidit, et, miráculis clarus, a Cleménte Papa Décimo inter 
 Sanctos relátus est.
\switchcolumn
\selectlanguage{english}
\lettrine[lines=2]{A}{t} Naples in Campania, St. Cajetan 
 the Theatine, confessor, founder of the Clerics Regular, who, through 
 singular confidence in God, made his disciples practise the primitive mode 
 of life of the apostles. Being renowned for miracles, he was ranked 
 among the saints by Clement X.
\switchcolumn*
\selectlanguage{latin}
Arétii, in Túscia, 
 natális sancti Donáti, Epíscopi et Mártyris; qui, inter cétera virtútis 
 ópera (ut scribit beátus Gregórius Papa), cálicem sanctum, a Pagánis fractum, 
 orándo instaurávit. Is, in persecutióne Juliáni Apóstatæ, a 
 Quadratiáno Augustáli comprehénsus, et, cum sacrificáre idólis renuísset, 
 gládio percússus, martyrium consummávit. Passus est étiam cum eo 
 beátus Hilarínus Mónachus; cujus memória décimo séptimo Kaléndas Augústi 
 recólitur, quo die sacrum ipsíus corpus ad Ostia Tiberína translátum fuit.
\switchcolumn
\selectlanguage{english}
At Arezzo in Tuscany, the birthday 
 of St. Donatus, bishop and martyr, who among other miraculous deeds by his 
 prayers (as is related by blessed Pope Gregory) made whole again a sacred 
 chalice which had been broken by pagans. Being apprehended by the 
 imperial officer Quadratian, during the persecution of Julian the Apostate, 
 and refusing to sacrifice to idols, he was struck with the sword, and thus 
 fulfilled his martyrdom. With him suffered also the blessed monk 
 Hilarinus, whose feast is celebrated on the 16th of July, at which time his 
 body was taken to Ostia.
\switchcolumn*
\selectlanguage{latin}
Romæ sanctórum Mártyrum 
 Petri et Juliáni, cum áliis decem et octo.
\switchcolumn
\selectlanguage{english}
At Rome, the holy martyrs Peter and 
 Julian, with eighteen others.
\switchcolumn*
\selectlanguage{latin}
Medioláni sancti Fausti 
 mílitis, qui, sub Aurélio Cómmodo, post multa certámina, martyrii palmam 
 adéptus est.
\switchcolumn
\selectlanguage{english}
At Milan, St. Faustus, a soldier, 
 who obtained the palm of martyrdom after many trials in the time of Aurélius 
 Commodus.
\switchcolumn*
\selectlanguage{latin}
Novocómi pássio 
 sanctórum Mártyrum Carpóphori, Exánthi, Cássii, Severíni, Secúndi et Licínii; 
 qui, in confessióne Christi, cápite truncáti sunt.
\switchcolumn
\selectlanguage{english}
At Como, the passion of the holy 
 martyrs Carpophorus, Exanthus, Cassius, Severinus, Secundus, and Licinius, 
 who were beheaded for the confession of Christ.
\switchcolumn*
\selectlanguage{latin}
Nísibi, in Mesopotámia, sancti Dométii, Mónachi Persæ, qui cum duóbus discípulis, sub Juliáno Apóstata, lapidátus est.
\switchcolumn
\selectlanguage{english}
At Nisibis in Mesopotamia, St. 
 Dometius, a Persian monk, who was stoned to death with two of his disciples 
 at the time of Julian the Apostate.
\switchcolumn*
\selectlanguage{latin}
Rotómagi sancti 
 Victrícii Epíscopi, qui adhuc miles, sub eódem Juliáno, abjíciens pro 
 Christo cíngulum, a Tribúno multis torméntis affícitur, et cápitis damnátur; 
 sed, carnífice, qui ad eum cædéndum missus fúerat, cæcitáte percússo, ipse, 
 vínculis solútis, liber evásit. Póstea, Epíscopus factus, indómitas 
 Morinórum et Nerviórum gentes divíni prædicatióne verbi ad Christi fidem 
 perdúxit, et demum Conféssor in pace quiévit.
\switchcolumn
\selectlanguage{english}
At Rouen, the holy bishop St. 
 Victricius. While he was yet a soldier under Julian, he threw away his 
 military belt for Christ, and after being subjected by the tribune to many 
 torments, was condemned to death. But the executioner sent to slay him 
 being struck blind, and the confessor's chains being loosened, he made his 
 escape. Afterwards being made bishop, by preaching the word of God, he 
 brought to the faith of Christ the barbarous people of Belgic Gaul, and 
 finally died in peace, a confessor.
\switchcolumn*
\selectlanguage{latin}
Cataláuni, in Gállia, 
 sancti Donatiáni Epíscopi.
\switchcolumn
\selectlanguage{english}
At Chalons in France, St. Donatian, 
 bishop.
\switchcolumn*
\selectlanguage{latin}
Messánæ, in Sicília, 
 sancti Albérti Confessóris, ex Ordine Carmelitárum, miráculis clari.
\switchcolumn
\selectlanguage{english}
At Messina in Sicily, St. Albert, 
 confessor of the Carmelite Order, renowned for miracles.
\switchcolumn*
\selectlanguage{latin}
\end{paracol}


% ---- martyrology/mart08/mart0808.htm
\needspace{10\baselineskip}
\begin{paracol}{2}
\selectlanguage{latin}
\begin{center}{\color{gregoriocolor} Sexto Idus Augústi. 
 Luna\dots\ }\end{center}
\switchcolumn
\selectlanguage{english}
\begin{center}{\color{gregoriocolor} The Eighth Day of 
 August. The\dots\ Day of the Moon.}\end{center}
\end{paracol}

\noindent\begin{tabularx}{\linewidth}{*{19}{>{\centering\arraybackslash}X}}
 \textcolor{gregoriocolor}{a} & \textcolor{gregoriocolor}{b} & \textcolor{gregoriocolor}{c} & \textcolor{gregoriocolor}{d} & \textcolor{gregoriocolor}{e} & \textcolor{gregoriocolor}{f} & \textcolor{gregoriocolor}{g} & \textcolor{gregoriocolor}{h} & \textcolor{gregoriocolor}{i} & \textcolor{gregoriocolor}{k} & \textcolor{gregoriocolor}{l} & \textcolor{gregoriocolor}{m} & \textcolor{gregoriocolor}{n} & \textcolor{gregoriocolor}{p} & \textcolor{gregoriocolor}{q} & \textcolor{gregoriocolor}{r} & \textcolor{gregoriocolor}{s} & \textcolor{gregoriocolor}{t} & \textcolor{gregoriocolor}{u} \\
 14 & 15 & 16 & 17 & 18 & 19 & 20 & 21 & 22 & 23 & 24 & 25 & 26 & 27 & 28 & 29 & 1 & 2 & 3 \\
\end{tabularx}
\vspace{0.5\baselineskip}
\noindent\begin{tabularx}{\linewidth}{*{12}{>{\centering\arraybackslash}X}}
 \textcolor{gregoriocolor}{A} & \textcolor{gregoriocolor}{B} & \textcolor{gregoriocolor}{C} & \textcolor{gregoriocolor}{D} & \textcolor{gregoriocolor}{E} & F & \textcolor{gregoriocolor}{F} & \textcolor{gregoriocolor}{G} & \textcolor{gregoriocolor}{H} & \textcolor{gregoriocolor}{M} & \textcolor{gregoriocolor}{N} & \textcolor{gregoriocolor}{P} \\
 4 & 5 & 6 & 7 & 8 & 9 & 8 & 9 & 10 & 11 & 12 & 13 \\
\end{tabularx}

\begin{paracol}{2}
\selectlanguage{latin}
\lettrine[lines=2]{S}{anctórum} Mártyrum 
 Cyríaci Diáconi, Largi et Smarágdi, qui, cum áliis vigínti Sóciis, passi 
 sunt décimo séptimo Kaléndas Aprílis. Eórum córpora, via Salária a 
 Joánne Presbytero sepúlta, sanctus Marcéllus Papa in prædium Lucínæ, via 
 Ostiénsi, hoc die tránstulit; quæ póstea, in Urbem deláta, in Diaconía 
 sanctæ Maríæ in via Lata fuérunt recóndita.
\switchcolumn
\selectlanguage{english}
\lettrine[lines=2]{T}{he} holy martyrs Cyriacus, deacon, 
 Largus, and Smaragdus, with twenty others who suffered on the 16th of March, 
 during the persecution of Diocletian and Maximian. Their bodies were 
 buried on the Salarian Way by the priest John, but were on this day 
 translated by Pope St. Marcellus to the estate of Lucina, on the Ostian Way. 
 Afterwards they were brought to the city and placed in the church of St. 
 Mary in Via Lata.
\switchcolumn*
\selectlanguage{latin}
Anazárbi, in Cilícia, sancti Maríni senis, qui, sub Diocletiáno Imperatóre et Lysia Præside, cæsus 
 flagris, in ligno suspénsus ac laniátus, feris tandem objéctus intériit.
\switchcolumn
\selectlanguage{english}
At Anzarba in Cilicia, St. Marinus, 
 an old man who was scourged, racked, and lacerated, and who died by being 
 exposed to wild beasts, in the time of Emperor Diocletian and the governor 
 Lysias.
\switchcolumn*
\selectlanguage{latin}
Item sanctórum Mártyrum 
 Eleuthérii et Leónidæ, qui per ignem martyrium consummárunt.
\switchcolumn
\selectlanguage{english}
Also, the holy martyrs Eleutherius 
 and Leonides, who underwent martyrdom by fire.
\switchcolumn*
\selectlanguage{latin}
In Pérside sancti 
 Hormísdæ Mártyris, sub Sápore Rege.
\switchcolumn
\selectlanguage{english}
In Persia, St. Hormisdas, a martyr 
 under King Sapor.
\switchcolumn*
\selectlanguage{latin}
Cyzici, in Hellespónto, 
 sancti Æmiliáni Epíscopi, qui, pro sacrárum Imáginum cultu a Leóne Imperatóre multa passus, demum in exsílio vitam finívit.
\switchcolumn
\selectlanguage{english}
At Cyzicum, on the Hellespont, St. 
 Aemilian, bishop, who ended his life in exile after having suffered much 
 from Emperor Leo for the veneration of holy images.
\switchcolumn*
\selectlanguage{latin}
In Creta sancti Myrónis 
 Epíscopi, miráculis clari.
\switchcolumn
\selectlanguage{english}
In Crete, St. Myron, a bishop 
 renowned for miracles.
\switchcolumn*
\selectlanguage{latin}
Viénnæ, in Gállia, 
 sancti Sevéri, Presbyteri et Confessóris; qui ex India, Evangélii prædicándi 
 causa, laboriósam peregrinatiónem suscépit, et, cum ad præfátam urbem 
 devenísset, ingéntem Paganórum multitúdinem verbo et miráculis ad Christi 
 fidem convértit.
\switchcolumn
\selectlanguage{english}
At Vienne in France, St. Severus, 
 priest and confessor, who undertook a painful journey from India in order to 
 preach the Gospel in that city, and converted a great number of pagans to 
 the faith of Christ by his works and miracles.
\switchcolumn*
\selectlanguage{latin}
\end{paracol}


% ---- martyrology/mart08/mart0809.htm
\needspace{10\baselineskip}
\begin{paracol}{2}
\selectlanguage{latin}
\begin{center}{\color{gregoriocolor} Quinto Idus Augústi. 
 Luna\dots\ }\end{center}
\switchcolumn
\selectlanguage{english}
\begin{center}{\color{gregoriocolor} The Ninth Day of 
 August. The\dots\ Day of the Moon.}\end{center}
\end{paracol}

\noindent\begin{tabularx}{\linewidth}{*{19}{>{\centering\arraybackslash}X}}
 \textcolor{gregoriocolor}{a} & \textcolor{gregoriocolor}{b} & \textcolor{gregoriocolor}{c} & \textcolor{gregoriocolor}{d} & \textcolor{gregoriocolor}{e} & \textcolor{gregoriocolor}{f} & \textcolor{gregoriocolor}{g} & \textcolor{gregoriocolor}{h} & \textcolor{gregoriocolor}{i} & \textcolor{gregoriocolor}{k} & \textcolor{gregoriocolor}{l} & \textcolor{gregoriocolor}{m} & \textcolor{gregoriocolor}{n} & \textcolor{gregoriocolor}{p} & \textcolor{gregoriocolor}{q} & \textcolor{gregoriocolor}{r} & \textcolor{gregoriocolor}{s} & \textcolor{gregoriocolor}{t} & \textcolor{gregoriocolor}{u} \\
 15 & 16 & 17 & 18 & 19 & 20 & 21 & 22 & 23 & 24 & 25 & 26 & 27 & 28 & 29 & 1 & 2 & 3 & 4 \\
\end{tabularx}
\vspace{0.5\baselineskip}
\noindent\begin{tabularx}{\linewidth}{*{12}{>{\centering\arraybackslash}X}}
 \textcolor{gregoriocolor}{A} & \textcolor{gregoriocolor}{B} & \textcolor{gregoriocolor}{C} & \textcolor{gregoriocolor}{D} & \textcolor{gregoriocolor}{E} & F & \textcolor{gregoriocolor}{F} & \textcolor{gregoriocolor}{G} & \textcolor{gregoriocolor}{H} & \textcolor{gregoriocolor}{M} & \textcolor{gregoriocolor}{N} & \textcolor{gregoriocolor}{P} \\
 5 & 6 & 7 & 8 & 9 & 10 & 9 & 10 & 11 & 12 & 13 & 14 \\
\end{tabularx}

\begin{paracol}{2}
\selectlanguage{latin}
\lettrine[lines=1]{V}{igília} sancti Lauréntii Mártyris.
\switchcolumn
\selectlanguage{english}
\lettrine[lines=1]{T}{he} vigil of St. Lawrence, martyr.
\switchcolumn*
\selectlanguage{latin}
Sancti Joánnis 
 Baptístæ-Maríæ Vianney, Presbyteri et Confessóris, cæléstis ómnium 
 parochórum Patróni; cujus dies natális prídie Nonas mensis hujus recensétur.
\switchcolumn
\selectlanguage{english}
St. John Baptist-Mary Vianney, 
 priest and confessor, and heavenly patron of all parish priests, whose 
 birthday is remembered on the 4th day of this month.
\switchcolumn*
\selectlanguage{latin}
Romæ sancti Románi, 
 mílitis et Mártyris; qui, confessióne beáti Lauréntii compúnctus, pétiit ab 
 eo baptizári, et, mox exhíbitus ac fústibus cæsus, ad últimum decollátus 
 est.
\switchcolumn
\selectlanguage{english}
At Rome, St. Romanus, a soldier, who 
 was moved by the torments of blessed Lawrence to ask for baptism from him. 
 He was immediately prosecuted, scourged, and finally beheaded.
\switchcolumn*
\selectlanguage{latin}
In Túscia natális 
 sanctórum Mártyrum Secundiáni, Marcelliáni et Veriáni; qui, témpore Décii, a 
 Promóto Consulári primum cæsi sunt, deínde in equúleo suspénsi, et abrási úngulis, atque latéribus appósito assáti, ac tandem triumphálem martyrii 
 palmam, cápite cæsi, meruérunt.
\switchcolumn
\selectlanguage{english}
In Tuscany, the birthday of the holy 
 martyrs Secundian, Marcellian, and Verian. In the time of Decius, they 
 were scourged by the exconsul Promotus, then racked and torn with iron 
 hooks. Being burned with fie applied to their sides, they merited the 
 triumphant palm of martyrdom by being beheaded.
\switchcolumn*
\selectlanguage{latin}
Verónæ sanctórum 
 Mártyrum Firmi et Rústici, qui, témpore Maximiáni Imperatóris, sub Anolíno 
 Júdice, cum sacrificáre idólis renuérunt et constánter in Christi fide persísterent, ambo jussi sunt, post ália superáta supplícia, fústibus cædi 
 et cápite amputári.
\switchcolumn
\selectlanguage{english}
At Verona, the holy martyrs Firmus 
 and Rusticus. When they refused to sacrifice to idols and remained 
 constant in confessing Christ, after they had overcome many other torments, 
 they were condemned to be scourged and beheaded by Anolinus, a judge, during 
 the reign of Emperor Maximian.
\switchcolumn*
\selectlanguage{latin}
In Africa commemorátio 
 plurimórum sanctórum Mártyrum, qui, in persecutióne Valeriáni, hortánte eos 
 ad constántiam sancto Numídico, in ignem conjécti, martyrii palmam adépti 
 sunt. Ipse autem Numídicus, licet cum áliis in rogum injéctus et 
 lapídibus óbrutus fuísset, a fília tamen, effóssus et semivívus repértus, 
 curátus est; ac póstea, ob ejus virtútem, in Ecclésiæ Carthaginénsis 
 Presbyterum a beáto Cypriáno méruit cooptári.
\switchcolumn
\selectlanguage{english}
In Africa, the commemoration of many 
 holy martyrs during the persecution of Valerian. Being exhorted by St. 
 Numidicus, they obtained the palm of martyrdom by being cast into the fire, 
 but Numidicus, although thrown into the flames with the others and 
 overwhelmed with stones, was nevertheless taken out by his daughter. 
 Found half dead, he was restored and deserved afterwards by his virtue to be 
 made priest of the Church of Carthage by blessed Cyprian.
\switchcolumn*
\selectlanguage{latin}
Constantinópoli 
 sanctórum Mártyrum Juliáni, Marciáni et aliórum octo; qui, ob Salvatóris 
 imáginem, quam in porta ǽnea constitúerant, omnes, ímpii Leónis Imperatóris 
 jussu, post multa torménta, gládio necáti sunt.
\switchcolumn
\selectlanguage{english}
At Constantinople, the holy martyrs 
 Julian, Marcian, and eight others. For having set up the image of our 
 Saviour on the brass gate, they were exposed to many torments, and then 
 beheaded by order of the impious emperor Leo.
\switchcolumn*
\selectlanguage{latin}
Cataláuni, in Gállia, 
 sancti Domitiáni, Epíscopi et Confessóris.
\switchcolumn
\selectlanguage{english}
At Chalons in France, St. Domitian, 
 bishop and confessor.
\switchcolumn*
\selectlanguage{latin}
\end{paracol}


% ---- martyrology/mart08/mart0810.htm
\needspace{10\baselineskip}
\begin{paracol}{2}
\selectlanguage{latin}
\begin{center}{\color{gregoriocolor} Quarto Idus Augústi. 
 Luna\dots\ }\end{center}
\switchcolumn
\selectlanguage{english}
\begin{center}{\color{gregoriocolor} The Tenth Day of 
 August. The\dots\ Day of the Moon.}\end{center}
\end{paracol}

\noindent\begin{tabularx}{\linewidth}{*{19}{>{\centering\arraybackslash}X}}
 \textcolor{gregoriocolor}{a} & \textcolor{gregoriocolor}{b} & \textcolor{gregoriocolor}{c} & \textcolor{gregoriocolor}{d} & \textcolor{gregoriocolor}{e} & \textcolor{gregoriocolor}{f} & \textcolor{gregoriocolor}{g} & \textcolor{gregoriocolor}{h} & \textcolor{gregoriocolor}{i} & \textcolor{gregoriocolor}{k} & \textcolor{gregoriocolor}{l} & \textcolor{gregoriocolor}{m} & \textcolor{gregoriocolor}{n} & \textcolor{gregoriocolor}{p} & \textcolor{gregoriocolor}{q} & \textcolor{gregoriocolor}{r} & \textcolor{gregoriocolor}{s} & \textcolor{gregoriocolor}{t} & \textcolor{gregoriocolor}{u} \\
 16 & 17 & 18 & 19 & 20 & 21 & 22 & 23 & 24 & 25 & 26 & 27 & 28 & 29 & 1 & 2 & 3 & 4 & 5 \\
\end{tabularx}
\vspace{0.5\baselineskip}
\noindent\begin{tabularx}{\linewidth}{*{12}{>{\centering\arraybackslash}X}}
 \textcolor{gregoriocolor}{A} & \textcolor{gregoriocolor}{B} & \textcolor{gregoriocolor}{C} & \textcolor{gregoriocolor}{D} & \textcolor{gregoriocolor}{E} & F & \textcolor{gregoriocolor}{F} & \textcolor{gregoriocolor}{G} & \textcolor{gregoriocolor}{H} & \textcolor{gregoriocolor}{M} & \textcolor{gregoriocolor}{N} & \textcolor{gregoriocolor}{P} \\
 6 & 7 & 8 & 9 & 10 & 11 & 10 & 11 & 12 & 13 & 14 & 15 \\
\end{tabularx}

\begin{paracol}{2}
\selectlanguage{latin}
\lettrine[lines=2]{R}{omæ,} via Tiburtína, 
 natális beáti Lauréntii Archidiáconi, qui, in persecutióne Valeriáni, post 
 plúrima torménta cárceris, vérberum diversórum, fústium, ac plumbatárum et 
 laminárum ardéntium, ad últimum, in cratícula férrea assátus, martyrium 
 complévit; ejúsque corpus a beáto Hippólyto et Justíno Presbytero sepúltum 
 fuit in cœmetério Cyríacæ, in agro Veráno.
\switchcolumn
\selectlanguage{english}
\lettrine[lines=2]{A}{t} Rome, on the Tiburtine Way, the 
 birthday of the blessed archdeacon Lawrence, martyred during the persecution 
 of Valerian. After much suffering from imprisonment, from scourging 
 with whips set with iron or lead, from hot metal plates, he at last 
 completed his martyrdom by being slowly consumed on an iron instrument made 
 in the form of a gridiron. His body was buried by blessed Hippolytus 
 and the priest Justin in the cemetery of Cyriaca, in the Agro Verano.
\switchcolumn*
\selectlanguage{latin}
In Hispánia Apparítio 
 beátæ Maríæ Vírginis de Mercéde nuncupátæ, quæ Ordinis redemptiónis 
 captivórum sub ejus nómine Institútrix fuit. Ipsíus autem festívitas 
 octávo Kaléndas Octóbris recólitur.
\switchcolumn
\selectlanguage{english}
In Spain, the apparition of the 
 Blessed Virgin Mary under the title of our Lady of Ransom, foundress of the 
 Order for the Redemption of Captives. Her feast is celebrated on the 
 24th of September.
\switchcolumn*
\selectlanguage{latin}
Romæ pássio sanctórum 
 centum sexagínta quinque mílitum Mártyrum, sub Aureliáno Imperatóre.
\switchcolumn
\selectlanguage{english}
At Rome, the passion of one hundred 
 and sixty-five holy martyrs, who were soldiers under Emperor Aurelian.
\switchcolumn*
\selectlanguage{latin}
Alexandríæ commemorátio 
 sanctórum Mártyrum, qui, in persecutióne Valeriáni, sub Præside Æmiliáno, 
 divérsis exquisitísque torméntis diútius cruciáti, vário mortis génere 
 corónam martyrii sunt adépti.
\switchcolumn
\selectlanguage{english}
At Alexandria, the commemoration of 
 the holy martyrs who in the persecution of Valerian, under the governor 
 Emilian, were long tormented with diverse and sharp tortures, and obtained 
 the crown of martyrdom by various kinds of deaths.
\switchcolumn*
\selectlanguage{latin}
Bérgomi sanctæ Astériæ, 
 Vírginis et Mártyris, in persecutióne Diocletiáni et Maximiáni Imperatórum.
\switchcolumn
\selectlanguage{english}
At Bergamo, St. Asteria, virgin and 
 martyr, in the persecution of Emperors Diocletian and Maximian.
\switchcolumn*
\selectlanguage{latin}
Carthágine sanctárum 
 Vírginum et Mártyrum Bassæ, Paulæ et Agathonícæ.
\switchcolumn
\selectlanguage{english}
At Carthage, the holy virgins and 
 martyrs Bassa, Paula, and Agathonica.
\switchcolumn*
\selectlanguage{latin}
Romæ sancti Deúsdedit 
 Confessóris, qui quod in hebdómada mánibus suis operándo lucrabátur, die 
 sábbati paupéribus erogábat.
\switchcolumn
\selectlanguage{english}
At Rome, the holy confessor 
 Deusdedit, a labouring man who gave to the poor every Saturday what he had 
 earned during the week.
\switchcolumn*
\selectlanguage{latin}
\end{paracol}


% ---- martyrology/mart08/mart0811.htm
\needspace{10\baselineskip}
\begin{paracol}{2}
\selectlanguage{latin}
\begin{center}{\color{gregoriocolor} Tértio Idus Augústi. 
 Luna\dots\ }\end{center}
\switchcolumn
\selectlanguage{english}
\begin{center}{\color{gregoriocolor} The Eleventh Day of 
 August. The\dots\ Day of the Moon.}\end{center}
\end{paracol}

\noindent\begin{tabularx}{\linewidth}{*{19}{>{\centering\arraybackslash}X}}
 \textcolor{gregoriocolor}{a} & \textcolor{gregoriocolor}{b} & \textcolor{gregoriocolor}{c} & \textcolor{gregoriocolor}{d} & \textcolor{gregoriocolor}{e} & \textcolor{gregoriocolor}{f} & \textcolor{gregoriocolor}{g} & \textcolor{gregoriocolor}{h} & \textcolor{gregoriocolor}{i} & \textcolor{gregoriocolor}{k} & \textcolor{gregoriocolor}{l} & \textcolor{gregoriocolor}{m} & \textcolor{gregoriocolor}{n} & \textcolor{gregoriocolor}{p} & \textcolor{gregoriocolor}{q} & \textcolor{gregoriocolor}{r} & \textcolor{gregoriocolor}{s} & \textcolor{gregoriocolor}{t} & \textcolor{gregoriocolor}{u} \\
 17 & 18 & 19 & 20 & 21 & 22 & 23 & 24 & 25 & 26 & 27 & 28 & 29 & 1 & 2 & 3 & 4 & 5 & 6 \\
\end{tabularx}
\vspace{0.5\baselineskip}
\noindent\begin{tabularx}{\linewidth}{*{12}{>{\centering\arraybackslash}X}}
 \textcolor{gregoriocolor}{A} & \textcolor{gregoriocolor}{B} & \textcolor{gregoriocolor}{C} & \textcolor{gregoriocolor}{D} & \textcolor{gregoriocolor}{E} & F & \textcolor{gregoriocolor}{F} & \textcolor{gregoriocolor}{G} & \textcolor{gregoriocolor}{H} & \textcolor{gregoriocolor}{M} & \textcolor{gregoriocolor}{N} & \textcolor{gregoriocolor}{P} \\
 7 & 8 & 9 & 10 & 11 & 12 & 11 & 12 & 13 & 14 & 15 & 16 \\
\end{tabularx}

\begin{paracol}{2}
\selectlanguage{latin}
\lettrine[lines=2]{R}{omæ,} inter duas Lauros, 
 natális sancti Tibúrtii Mártyris, qui, sub Júdice Fabiáno, in persecutióne 
 Diocletiáni, cum Christum, nudis plántis super carbónes ardéntes ingréssus, 
 majóri confiterétur constántia, duci in tértium ab Urbe milliárium atque 
 ibídem gládio animadvérti jubétur.
\switchcolumn
\selectlanguage{english}
\lettrine[lines=2]{A}{t} Rome, between the two laurels 
 situation about three miles from the city, the birthday of St. Tiburtius, 
 martyr, under the judge Fabian, in the persecution of Diocletian. 
 After he had walked barefooted on burning coals and confessed Christ with 
 increased constancy, he was put to the sword.
\switchcolumn*
\selectlanguage{latin}
Item Romæ sanctæ 
 Susánnæ Vírginis, quæ, cum ex nóbili prosápia esset orta et beáti Caji 
 Pontíficis neptis, martyrii palmam, témpore Diocletiáni, cápitis 
 obtruncatióne proméruit.
\switchcolumn
\selectlanguage{english}
Also at Rome, the holy virgin 
 Susanna, a woman of noble race, and niece of the blessed Pontiff Caius. 
 She merited the palm of martyrdom by being beheaded in the time of 
 Diocletian.
\switchcolumn*
\selectlanguage{latin}
Assísii in Umbria, 
 natális sanctæ Claræ Vírginis, primæ plantæ Páuperum Dominárum Ordinis 
 Minórum; quam, vita et miráculis célebrem, Alexánder Papa Quartus in númerum 
 sanctárum Vírginum rétulit. Ipsíus tamen festum sequénti die 
 celebrátur.
\switchcolumn
\selectlanguage{english}
At Assisi in Umbria, the birthday of 
 St. Clare, virgin, the first of the Poor Ladies of the Order of Friars 
 Minor. Being celebrated for holiness of life and miracles, she was 
 placed among the holy virgins by Pope Alexander IV. Her feast, 
 however, is observed on the day following.
\switchcolumn*
\selectlanguage{latin}
Cománæ, in Ponto, 
 sancti Alexándri Epíscopi, cognoménto Carbonárii, qui, ex philósopho disertíssimo eminéntem Christiánæ humilitátis sciéntiam adéptus, et a sancto 
 Gregório Thaumatúrgo in thronum illíus Ecclésiæ sublimátus, non solum 
 prædicatióne, sed étiam consummáto per ignem martyrio fuit illústris.
\switchcolumn
\selectlanguage{english}
At Comana in Pontus, St. Alexander, 
 bishop, surnamed Carbonarius, who added to a masterful knowledge of 
 philosophy an eminent degree of Christian humility. He was promoted to 
 the See of that church by St. Gregory Thaumaturgus, and became illustrious, 
 not only by preaching, but also by suffering martyrdom by fire.
\switchcolumn*
\selectlanguage{latin}
Eódem die pássio 
 sanctórum Rufíni, Marsórum Epíscopi, et Sociórum ejus, sub Maximiáno 
 Imperatóre.
\switchcolumn
\selectlanguage{english}
The same day, the martyrdom of St. 
 Rufinus, Bishop of the Marsi, and his companions, under Emperor Maximinus.
\switchcolumn*
\selectlanguage{latin}
Apud Ebroicénses, in 
 Gállia, sancti Tauríni Epíscopi, qui, a beáto Cleménte Papa ordinátus illíus 
 civitátis Epíscopus, Evangélii prædicatióne Christiánam fidem propagávit, 
 ac, multis pro ea suscéptis labóribus, miraculórum glória conspícuus 
 obdormívit in Dómino.
\switchcolumn
\selectlanguage{english}
At Evreux in France, St. Thaurinus, 
 bishop. Being made bishop of that city by blessed Pope Clement, he 
 propagated the Christian faith by the preaching of the Gospel, and the many 
 labours he sustained for it. Celebrated for glorious miracles, he fell 
 asleep in the Lord.
\switchcolumn*
\selectlanguage{latin}
Cameráci, in Gállia, 
 sancti Gaugeríci, Epíscopi et Confessóris.
\switchcolumn
\selectlanguage{english}
At Cambrai in France, St. Gaugericus, 
 bishop and confessor.
\switchcolumn*
\selectlanguage{latin}
In província Valériæ 
 sancti Equítii Abbátis, cujus sánctitas testimónio beáti Gregórii Papæ 
 comprobátur.
\switchcolumn
\selectlanguage{english}
In the province of Valeria, St. 
 Equitius, abbot, whose sanctity is attested by blessed Pope Gregory.
\switchcolumn*
\selectlanguage{latin}
Tudérti, in Umbria, 
 sanctæ Dignæ Vírginis.
\switchcolumn
\selectlanguage{english}
At Todi in Umbria, St. Digna, 
 virgin.
\switchcolumn*
\selectlanguage{latin}
\end{paracol}


% ---- martyrology/mart08/mart0812.htm
\needspace{10\baselineskip}
\begin{paracol}{2}
\selectlanguage{latin}
\begin{center}{\color{gregoriocolor} Prídie Idus Augústi. 
 Luna\dots\ }\end{center}
\switchcolumn
\selectlanguage{english}
\begin{center}{\color{gregoriocolor} The Twelfth Day of 
 August. The\dots\ Day of the Moon.}\end{center}
\end{paracol}

\noindent\begin{tabularx}{\linewidth}{*{19}{>{\centering\arraybackslash}X}}
 \textcolor{gregoriocolor}{a} & \textcolor{gregoriocolor}{b} & \textcolor{gregoriocolor}{c} & \textcolor{gregoriocolor}{d} & \textcolor{gregoriocolor}{e} & \textcolor{gregoriocolor}{f} & \textcolor{gregoriocolor}{g} & \textcolor{gregoriocolor}{h} & \textcolor{gregoriocolor}{i} & \textcolor{gregoriocolor}{k} & \textcolor{gregoriocolor}{l} & \textcolor{gregoriocolor}{m} & \textcolor{gregoriocolor}{n} & \textcolor{gregoriocolor}{p} & \textcolor{gregoriocolor}{q} & \textcolor{gregoriocolor}{r} & \textcolor{gregoriocolor}{s} & \textcolor{gregoriocolor}{t} & \textcolor{gregoriocolor}{u} \\
 18 & 19 & 20 & 21 & 22 & 23 & 24 & 25 & 26 & 27 & 28 & 29 & 1 & 2 & 3 & 4 & 5 & 6 & 7 \\
\end{tabularx}
\vspace{0.5\baselineskip}
\noindent\begin{tabularx}{\linewidth}{*{12}{>{\centering\arraybackslash}X}}
 \textcolor{gregoriocolor}{A} & \textcolor{gregoriocolor}{B} & \textcolor{gregoriocolor}{C} & \textcolor{gregoriocolor}{D} & \textcolor{gregoriocolor}{E} & F & \textcolor{gregoriocolor}{F} & \textcolor{gregoriocolor}{G} & \textcolor{gregoriocolor}{H} & \textcolor{gregoriocolor}{M} & \textcolor{gregoriocolor}{N} & \textcolor{gregoriocolor}{P} \\
 8 & 9 & 10 & 11 & 12 & 13 & 12 & 13 & 14 & 15 & 16 & 17 \\
\end{tabularx}

\begin{paracol}{2}
\selectlanguage{latin}
\lettrine[lines=2]{S}{anctæ} Claræ Vírginis, 
 primæ plantæ Páuperum Dominárum Ordinis Minórum; quæ ad ætérnas Agni núptias 
 evocáta est prídie hujus diéi.
\switchcolumn
\selectlanguage{english}
\lettrine[lines=2]{S}{t.} Clare, virgin, the first fruits 
 of the Poor Ladies of the Order of Friars Minor, who was called to the 
 everlasting nuptials of the Lamb on the day previous.
\switchcolumn*
\selectlanguage{latin}
Eódem die sanctórum 
 Mártyrum Porcárii, Abbátis monastérii Lirinénsis, et Sociórum ejus 
 quingentórum Monachórum; qui pro fide cathólica, a bárbaris cæsi, martyrio 
 coronáti sunt.
\switchcolumn
\selectlanguage{english}
The same day, the holy martyrs 
 Porcarius, abbot of the monastery of Lerins, and five hundred monks, who 
 were slain for the Catholic faith by barbarians, and were thus crowned with 
 martyrdom.
\switchcolumn*
\selectlanguage{latin}
Catánæ, in Sicília, 
 natális sancti Euplii Diáconi, sub Diocletiáno et Maximiáno Augústis; qui, 
 cum diutíssime pro confessióne Dómini tortus esset, tandem martyrii palmam, 
 gládio cædénte, percépit.
\switchcolumn
\selectlanguage{english}
At Catania in Sicily, the birthday 
 of St. Euplius, deacon, under Emperors Diocletian and Maximian. He was 
 long tortured for the confession of the Lord, and finally obtained the palm 
 of martyrdom by being put to the sword.
\switchcolumn*
\selectlanguage{latin}
Augústæ Vindelicórum 
 sanctæ Hiláriæ, quæ, cum esset beátæ Afræ Mártyris mater et ad sepúlcrum illíus excubáret, ibídem, pro fide Christi, a persecutóribus igni trádita 
 est cum Digna, Euprépia et Eunómia, ancíllis suis. Passi sunt étiam 
 eódem die, in præfáta urbe, Quiríacus, Lárgio, Crescentiánus, Nímmia et 
 Juliána, cum áliis vigínti.
\switchcolumn
\selectlanguage{english}
At Augsburg, St. Hilaria, mother of 
 the blessed martyr Afra. Because she watched at the tomb of her 
 daughter she was cast into the fire for the faith of Christ, together with 
 her maidservants Digna, Euprepia, and Eunomia. On the same day there 
 suffered also in that city Quiriacus, Largius, Crescentian, Nimmia, and 
 Juliana, with twenty others.
\switchcolumn*
\selectlanguage{latin}
In Syria sanctórum 
 Mártyrum Macárii et Juliáni.
\switchcolumn
\selectlanguage{english}
In Syria, the holy martyrs Marcarius 
 and Julian.
\switchcolumn*
\selectlanguage{latin}
Nicomedíæ sanctórum 
 Mártyrum Anicéti Cómitis, et Photíni fratris, cum áliis plúribus, sub 
 Diocletiáno Imperatóre.
\switchcolumn
\selectlanguage{english}
At Nicomedia, the holy martyrs Count 
 Anicetus and his brother Photinus, along with many others, under Emperor 
 Diocletian.
\switchcolumn*
\selectlanguage{latin}
Falériæ, in Túscia, 
 pássio sanctórum Graciliáni, et Felicíssimæ Vírginis, quorum pro fídei 
 confessióne ora lapídibus primo contúsa sunt; dehinc ambo, gládio percússi, 
 optátam martyrii palmam suscepérunt.
\switchcolumn
\selectlanguage{english}
At Faleria in Tuscany, the Saints 
 Gracilian, and Felicíssima, virgin, who, for the confession of the faith, 
 first had their mouths bruised with stones, and being afterwards struck with 
 the sword, received the palm of martyrdom.
\switchcolumn*
\selectlanguage{latin}
Medioláni deposítio 
 sancti Eusébii, Epíscopi et Confessóris.
\switchcolumn
\selectlanguage{english}
At Milan, the death of St. Eusebius, bishop and 
 confessor.
\switchcolumn*
\selectlanguage{latin}
Bríxiæ sancti Herculáni 
 Epíscopi.
\switchcolumn
\selectlanguage{english}
At Brescia, St. Herculanus, bishop.
\switchcolumn*
\selectlanguage{latin}
\end{paracol}


% ---- martyrology/mart08/mart0813.htm
\needspace{10\baselineskip}
\begin{paracol}{2}
\selectlanguage{latin}
\begin{center}{\color{gregoriocolor} Idibus Augústi. 
 Luna\dots\ }\end{center}
\switchcolumn
\selectlanguage{english}
\begin{center}{\color{gregoriocolor} The Thirteenth Day of 
 August. The\dots\ Day of the Moon.}\end{center}
\end{paracol}

\noindent\begin{tabularx}{\linewidth}{*{19}{>{\centering\arraybackslash}X}}
 \textcolor{gregoriocolor}{a} & \textcolor{gregoriocolor}{b} & \textcolor{gregoriocolor}{c} & \textcolor{gregoriocolor}{d} & \textcolor{gregoriocolor}{e} & \textcolor{gregoriocolor}{f} & \textcolor{gregoriocolor}{g} & \textcolor{gregoriocolor}{h} & \textcolor{gregoriocolor}{i} & \textcolor{gregoriocolor}{k} & \textcolor{gregoriocolor}{l} & \textcolor{gregoriocolor}{m} & \textcolor{gregoriocolor}{n} & \textcolor{gregoriocolor}{p} & \textcolor{gregoriocolor}{q} & \textcolor{gregoriocolor}{r} & \textcolor{gregoriocolor}{s} & \textcolor{gregoriocolor}{t} & \textcolor{gregoriocolor}{u} \\
 19 & 20 & 21 & 22 & 23 & 24 & 25 & 26 & 27 & 28 & 29 & 1 & 2 & 3 & 4 & 5 & 6 & 7 & 8 \\
\end{tabularx}
\vspace{0.5\baselineskip}
\noindent\begin{tabularx}{\linewidth}{*{12}{>{\centering\arraybackslash}X}}
 \textcolor{gregoriocolor}{A} & \textcolor{gregoriocolor}{B} & \textcolor{gregoriocolor}{C} & \textcolor{gregoriocolor}{D} & \textcolor{gregoriocolor}{E} & F & \textcolor{gregoriocolor}{F} & \textcolor{gregoriocolor}{G} & \textcolor{gregoriocolor}{H} & \textcolor{gregoriocolor}{M} & \textcolor{gregoriocolor}{N} & \textcolor{gregoriocolor}{P} \\
 9 & 10 & 11 & 12 & 13 & 14 & 13 & 14 & 15 & 16 & 17 & 18 \\
\end{tabularx}

\begin{paracol}{2}
\selectlanguage{latin}
\lettrine[lines=2]{R}{omæ} beáti Hippólyti 
 Mártyris, qui pro confessióne glória, sub Valeriáno Imperatóre, post ália 
 torménta, ligátis pédibus ad colla indomitórum equórum, per carduétum et 
 tríbulos crudéliter tractus est, ac, toto córpore laceráto, emísit spíritum. 
 Passi sunt étiam eódem die beáta Concórdia, ejus nutrix, quæ ante ipsum, 
 plumbátis cæsa, migrávit ad Dóminum; et álii decem et novem de domo sua, qui 
 extra portam Tiburtínam decolláti sunt, et, una cum ipso, in agro Veráno 
 sepúlti.
\switchcolumn
\selectlanguage{english}
\lettrine[lines=2]{A}{t} Rome, the blessed Hippolytus, 
 martyr, who gloriously confessed the faith, under Emperor Valerian. 
 After enduring other torments, he was tied by the feet to the necks of wild 
 horses, and being cruelly dragged through briars and brambles, and having 
 all his body lacerated, he yielded up his spirit. On the same day 
 suffered also blessed Concordia, his nurse, who being scourged in his 
 presence with leaded whips, went to our Lord, and nineteen others of his 
 household, who were beheaded beyond the Tiburtine Gate, and buried with him 
 in the Agro Verano.
\switchcolumn*
\selectlanguage{latin}
Apud Forum Sillæ 
 natális sancti Cassiáni Mártyris, qui cum adoráre idóla noluísset, ídeo, 
 vocátis a persecutóre púeris, quibus exósus docéndo factus fúerat, data est 
 eis facúltas eum periméndi; quorum quanto infírmior erat manus, tanto 
 graviórem martyrii pœnam, diláta morte, faciébat.
\switchcolumn
\selectlanguage{english}
At Imola, the birthday of St. 
 Cassian, martyr. As he refused to worship idols, the persecutor called 
 the boys whom the saint had taught and who hated him, giving them leave to 
 kill him. The torment suffered by the martyr was the more grievous, as 
 the hands which inflicted it, by reason of weakness, rendered death long 
 drawn-out.
\switchcolumn*
\selectlanguage{latin}
Tudérti, in Umbria, 
 sancti Cassiáni, Epíscopi et Mártyris, sub Diocletiáno Imperatóre.
\switchcolumn
\selectlanguage{english}
At Todi in Umbria, St. Cassian, 
 bishop and martyr, under Emperor Diocletian.
\switchcolumn*
\selectlanguage{latin}
Burgis, in Hispánia, 
 sanctárum Centóllæ et Hélenæ Mártyrum.
\switchcolumn
\selectlanguage{english}
At Burgos in Spain, Saints Centolla 
 and Helena, martyrs.
\switchcolumn*
\selectlanguage{latin}
Constantinópoli sancti 
 Máximi Abbátis, doctrína et cathólicæ veritátis zelo insígnis; qui, cum 
 advérsus Monothelítas strénue decertáret, ab hærético Imperatóre Constánte, 
 præcísis mánibus ac lingua, in Chersonésum relegátus est, ibíque, glória 
 confessiónis clarus, spíritum Deo réddidit. Tunc étiam duo Anastásii, 
 qui ejus erant discípuli, aliíque plures divérsa torménta et dura exsília 
 sunt expérti.
\switchcolumn
\selectlanguage{english}
At Constantinople, St. Maximus, a 
 monk distinguished for learning and for zeal for Catholic truth. 
 Valiantly disputing the Monothelites, he had his hands and tongue torn from 
 him by the heretical emperor Constans, and was banished to Chersonesus, 
 where he breathed his last. At this time, two of his disciples, both 
 named Anastasius, and many others endured divers torments and the hardships 
 of exile.
\switchcolumn*
\selectlanguage{latin}
Fritesláriæ, in 
 Germánia, sancti Wigbérti, Presbyteri et Confessóris.
\switchcolumn
\selectlanguage{english}
At Fritzlar in Germany, St. Wigbert, 
 priest and confessor.
\switchcolumn*
\selectlanguage{latin}
Romæ natális sancti 
 Joánnis Berchmans, scholástici e Societáte Jesu et Confessóris, vitæ 
 innocéntia et religiósæ disciplínæ custódia præclári; cui Leo Décimus 
 tértius, Póntifex Máximus, cælitum Sanctórum honóres decrévit.
\switchcolumn
\selectlanguage{english}
At Rome, the birthday of St. John 
 Berchmans, a scholastic of the Society of Jesus, illustrious for his 
 innocence and for his fidelity to the rules of the religious life. He 
 was canonized by Pope Leo XIII.
\switchcolumn*
\selectlanguage{latin}
Pictávis, in Gállia, 
 sanctæ Radegúndis Regínæ, cujus vita miráculis et virtútibus cláruit.
\switchcolumn
\selectlanguage{english}
At Poitiers in France, St. Radegund, 
 queen, whose life was renowned for miracles and virtues.
\switchcolumn*
\selectlanguage{latin}
\end{paracol}


% ---- martyrology/mart08/mart0814.htm
\needspace{10\baselineskip}
\begin{paracol}{2}
\selectlanguage{latin}
\begin{center}{\color{gregoriocolor} Décimo nono Kaléndas Septémbris. 
 Luna\dots\ }\end{center}
\switchcolumn
\selectlanguage{english}
\begin{center}{\color{gregoriocolor} The 
 Fourteenth Day of 
 August. The\dots\ Day of the Moon.}\end{center}
\end{paracol}

\noindent\begin{tabularx}{\linewidth}{*{19}{>{\centering\arraybackslash}X}}
 \textcolor{gregoriocolor}{a} & \textcolor{gregoriocolor}{b} & \textcolor{gregoriocolor}{c} & \textcolor{gregoriocolor}{d} & \textcolor{gregoriocolor}{e} & \textcolor{gregoriocolor}{f} & \textcolor{gregoriocolor}{g} & \textcolor{gregoriocolor}{h} & \textcolor{gregoriocolor}{i} & \textcolor{gregoriocolor}{k} & \textcolor{gregoriocolor}{l} & \textcolor{gregoriocolor}{m} & \textcolor{gregoriocolor}{n} & \textcolor{gregoriocolor}{p} & \textcolor{gregoriocolor}{q} & \textcolor{gregoriocolor}{r} & \textcolor{gregoriocolor}{s} & \textcolor{gregoriocolor}{t} & \textcolor{gregoriocolor}{u} \\
 20 & 21 & 22 & 23 & 24 & 25 & 26 & 27 & 28 & 29 & 1 & 2 & 3 & 4 & 5 & 6 & 7 & 8 & 9 \\
\end{tabularx}
\vspace{0.5\baselineskip}
\noindent\begin{tabularx}{\linewidth}{*{12}{>{\centering\arraybackslash}X}}
 \textcolor{gregoriocolor}{A} & \textcolor{gregoriocolor}{B} & \textcolor{gregoriocolor}{C} & \textcolor{gregoriocolor}{D} & \textcolor{gregoriocolor}{E} & F & \textcolor{gregoriocolor}{F} & \textcolor{gregoriocolor}{G} & \textcolor{gregoriocolor}{H} & \textcolor{gregoriocolor}{M} & \textcolor{gregoriocolor}{N} & \textcolor{gregoriocolor}{P} \\
 10 & 11 & 12 & 13 & 14 & 15 & 14 & 15 & 16 & 17 & 18 & 19 \\
\end{tabularx}

\begin{paracol}{2}
\selectlanguage{latin}
\lettrine[lines=2]{V}{igília} Assumptiónis 
 beátæ Maríæ Vírginis.
\switchcolumn
\selectlanguage{english}
\lettrine[lines=2]{T}{he} Vigil of the Assumption of the 
 Blessed Virgin Mary.
\switchcolumn*
\selectlanguage{latin}
Romæ natális beáti 
 Eusébii, Presbyteri et Confessóris, qui, ab Ariáno Imperatóre Constántio, ob 
 cathólicæ fídei defensiónem, in quodam domus suæ cubículo inclúsus, ibi, cum 
 menses septem in oratióne constánter perseverásset, dormitiónem accépit. 
 Ipsíus autem corpus collegérunt Gregórius et Orósius Presbyteri, et in 
 cœmetério Callísti, via Appia, sepeliérunt.
\switchcolumn
\selectlanguage{english}
At Rome, the birthday of the blessed 
 priest Eusebius, who for the defence of the Catholic faith was shut up in a 
 room of his own house by the Arian emperor Constantius, where constantly 
 persevering in prayer for seven months, he rested in peace. His body 
 was removed by the priests Gregory and Orosius, and buried in the cemetery 
 of Callistus, on the Appian Way.
\switchcolumn*
\selectlanguage{latin}
Apaméæ, in Syria, 
 sancti Marcélli, Epíscopi et Mártyris; qui, cum Jovis delúbrum diruísset, a 
 furéntibus Gentílibus occísus est.
\switchcolumn
\selectlanguage{english}
At Apamea in Syria, St. Marcellus, 
 bishop and martyr, who was killed by the enraged heathen for having pulled 
 down a temple of Jupiter.
\switchcolumn*
\selectlanguage{latin}
Tudérti, in Umbria, 
 sancti Callísti, Epíscopi et Mártyris.
\switchcolumn
\selectlanguage{english}
At Todi in Umbria, St. Callistus, 
 bishop and martyr.
\switchcolumn*
\selectlanguage{latin}
In Illyrico sancti 
 Ursícii Mártyris, qui, sub Maximiáno Imperatóre et Aristíde Præside, post 
 multa et divérsa torménta, pro Christi nómine, gládio cæsus est.
\switchcolumn
\selectlanguage{english}
In Illyria, St. Ursicius, martyr, 
 who was beheaded for Christ after suffering various torments under Emperor 
 Maximian and the governor Aristides.
\switchcolumn*
\selectlanguage{latin}
In Africa sancti 
 Demétrii Mártyris.
\switchcolumn
\selectlanguage{english}
In Africa, St. Demetrius, martyr.
\switchcolumn*
\selectlanguage{latin}
In Ægína ínsula sanctæ 
 Athanásiæ Víduæ, monástica observántia et miraculórum dono illústris.
\switchcolumn
\selectlanguage{english}
In the island of Aegina, St. 
 Athanasia, widow, celebrated for monastical observance and the gift of 
 miracles.
\switchcolumn*
\selectlanguage{latin}
\end{paracol}


% ---- martyrology/mart08/mart0815.htm
\needspace{10\baselineskip}
\begin{paracol}{2}
\selectlanguage{latin}
\begin{center}{\color{gregoriocolor} Décimo octávo Kaléndas Septémbris. 
 Luna\dots\ }\end{center}
\switchcolumn
\selectlanguage{english}
\begin{center}{\color{gregoriocolor} The 
 Fifteenth Day of 
 August. The\dots\ Day of the Moon.}\end{center}
\end{paracol}

\noindent\begin{tabularx}{\linewidth}{*{19}{>{\centering\arraybackslash}X}}
 \textcolor{gregoriocolor}{a} & \textcolor{gregoriocolor}{b} & \textcolor{gregoriocolor}{c} & \textcolor{gregoriocolor}{d} & \textcolor{gregoriocolor}{e} & \textcolor{gregoriocolor}{f} & \textcolor{gregoriocolor}{g} & \textcolor{gregoriocolor}{h} & \textcolor{gregoriocolor}{i} & \textcolor{gregoriocolor}{k} & \textcolor{gregoriocolor}{l} & \textcolor{gregoriocolor}{m} & \textcolor{gregoriocolor}{n} & \textcolor{gregoriocolor}{p} & \textcolor{gregoriocolor}{q} & \textcolor{gregoriocolor}{r} & \textcolor{gregoriocolor}{s} & \textcolor{gregoriocolor}{t} & \textcolor{gregoriocolor}{u} \\
 21 & 22 & 23 & 24 & 25 & 26 & 27 & 28 & 29 & 1 & 2 & 3 & 4 & 5 & 6 & 7 & 8 & 9 & 10 \\
\end{tabularx}
\vspace{0.5\baselineskip}
\noindent\begin{tabularx}{\linewidth}{*{12}{>{\centering\arraybackslash}X}}
 \textcolor{gregoriocolor}{A} & \textcolor{gregoriocolor}{B} & \textcolor{gregoriocolor}{C} & \textcolor{gregoriocolor}{D} & \textcolor{gregoriocolor}{E} & F & \textcolor{gregoriocolor}{F} & \textcolor{gregoriocolor}{G} & \textcolor{gregoriocolor}{H} & \textcolor{gregoriocolor}{M} & \textcolor{gregoriocolor}{N} & \textcolor{gregoriocolor}{P} \\
 11 & 12 & 13 & 14 & 15 & 16 & 15 & 16 & 17 & 18 & 19 & 20 \\
\end{tabularx}

\begin{paracol}{2}
\selectlanguage{latin}
\lettrine[lines=2]{A}{ssúmptio} sanctíssimæ 
 Dei Genitrícis Vírginis Maríæ.
\switchcolumn
\selectlanguage{english}
\lettrine[lines=2]{T}{he} Assumption of the most holy 
 Virgin Mary, Mother of God.
\switchcolumn*
\selectlanguage{latin}
Cracóviæ, in Polónia, 
 natális sancti Hyacínthi, ex Ordine Prædicatórum, Confessóris, quem Clemens 
 Octávus, Póntifex Máximus, in Sanctórum númerum rétulit. Ipsíus autem 
 festum sextodécimo Kaléndas Septémbris celebrátur.
\switchcolumn
\selectlanguage{english}
At Cracow in Poland, St. Hyacinth, 
 confessor of the Order of Preachers, whom Pope Clement VIII placed in the 
 number of the saints. His feast is observed on the 17th of August.
\switchcolumn*
\selectlanguage{latin}
Apud Albam Regálem, in 
 Pannónia, item natális sancti Stéphani, Regis Hungarórum et Confessóris; 
 qui, divínis virtútibus exornátus, primus Húngaros ad Christi fidem 
 convértit, et a Deípara Vírgine, ipso die Assumptiónis suæ, in cælum 
 recéptus fuit. Ejus vero festívitas quarto Nonas Septémbris, quo die 
 munitíssima Budæ arx, sancti Regis ope, recólitur, ex dispositióne 
 Innocéntii Papæ Undécimi.
\switchcolumn
\selectlanguage{english}
At Alba Regalis in Hungary, St. 
 Stephen, King of Hungary, who was graced with divine virtues, was the first 
 to convert the Hungarians to the faith of Christ, and was received into 
 heaven by the Virgin Mother of God on the very day of her Assumption. 
 By decree of Pope Innocent XI, his feast is kept on the 2nd of September, on 
 which day the strong city of Buda, by the aid of the holy king, was 
 recovered by the Christian army.
\switchcolumn*
\selectlanguage{latin}
Romæ, via Appia, sancti 
 Tharsícii Acolythi, quem Pagáni, cum inveníssent Córporis Christi Sacraménta 
 portántem, cœpérunt disquírere quid géreret. At ipse, indígnum 
 júdicans porcis pródere margarítas, támdiu ab illis mactátus luto ejus 
 córpore, sacrílegi discussóres nihil Sacramentórum Christi in occísi mánibus 
 aut in véstibus invenérunt. Christiáni autem collegérunt Mártyris 
 corpus, et in cœmetério Callísti honorifice sepeliérunt.
\switchcolumn
\selectlanguage{english}
At Rome, on the Appian Way, St. 
 Tarsicius, acolyte. The pagans accosted him as he was carrying the 
 Sacrament of Christ's Body, and began to inquire what it was. But he 
 judged it an unworthy thing to cast pearls before swine. They 
 therefore beat him with sticks and stones until he expired. The 
 sacrilegious searchers examined his body, but found no vestige of the 
 Sacrament of Christ, either in his hands or in his clothes. The 
 Christians took up the body of the martyr, and buried it reverently in the 
 cemetery of Callistus.
\switchcolumn*
\selectlanguage{latin}
Tagáste, in Africa, 
 sancti Alípii Epíscopi, qui beáti Augustíni olim discípulus, póstea in 
 conversióne sócius, in múnere pastoráli colléga, et in certamínibus advérsus 
 hæréticos commílito strénuus, ac demum in cælésti glória consors fuit.
\switchcolumn
\selectlanguage{english}
At Tagaste in Africa, St. Alipius, 
 bishop, who was the disciple of blessed Augustine, and the companion of his 
 conversion, his colleague in the pastoral charge, his valiant fellow-soldier 
 in disputing heretics, and finally his partner in the glory of heaven.
\switchcolumn*
\selectlanguage{latin}
Suessióne, in Gálliis, 
 sancti Arnúlfi, Epíscopi et Confessóris.
\switchcolumn
\selectlanguage{english}
At Soissons in France, St. Arnold, 
 bishop and confessor.
\switchcolumn*
\selectlanguage{latin}
Romæ sancti Stanislái 
 Kostkæ Polóni, novítii e Societáte Jesu et Confessóris; qui, consummátus in 
 brevi, per angélicam vitæ innocéntiam explévit témpora multa, et a Benedícto 
 Décimo tértio, Pontífice Máximo, in Sanctórum album relátus est.
\switchcolumn
\selectlanguage{english}
At Rome, St. Stanislas Kostka, a 
 native of Poland, confessor of the Society of Jesus, who being made perfect 
 in a short time, fulfilled a long time by the angelic innocence of his life. 
 He was inscribed on the list of the saints by the Sovereign Pontiff, 
 Benedict XIII.
\switchcolumn*
\selectlanguage{latin}
\end{paracol}


% ---- martyrology/mart08/mart0816.htm
\needspace{10\baselineskip}
\begin{paracol}{2}
\selectlanguage{latin}
\begin{center}{\color{gregoriocolor} Décimo séptimo Kaléndas Septémbris. 
 Luna\dots\ }\end{center}
\switchcolumn
\selectlanguage{english}
\begin{center}{\color{gregoriocolor} The 
 Sixteenth Day of 
 August. The\dots\ Day of the Moon.}\end{center}
\end{paracol}

\noindent\begin{tabularx}{\linewidth}{*{19}{>{\centering\arraybackslash}X}}
 \textcolor{gregoriocolor}{a} & \textcolor{gregoriocolor}{b} & \textcolor{gregoriocolor}{c} & \textcolor{gregoriocolor}{d} & \textcolor{gregoriocolor}{e} & \textcolor{gregoriocolor}{f} & \textcolor{gregoriocolor}{g} & \textcolor{gregoriocolor}{h} & \textcolor{gregoriocolor}{i} & \textcolor{gregoriocolor}{k} & \textcolor{gregoriocolor}{l} & \textcolor{gregoriocolor}{m} & \textcolor{gregoriocolor}{n} & \textcolor{gregoriocolor}{p} & \textcolor{gregoriocolor}{q} & \textcolor{gregoriocolor}{r} & \textcolor{gregoriocolor}{s} & \textcolor{gregoriocolor}{t} & \textcolor{gregoriocolor}{u} \\
 22 & 23 & 24 & 25 & 26 & 27 & 28 & 29 & 1 & 2 & 3 & 4 & 5 & 6 & 7 & 8 & 9 & 10 & 11 \\
\end{tabularx}
\vspace{0.5\baselineskip}
\noindent\begin{tabularx}{\linewidth}{*{12}{>{\centering\arraybackslash}X}}
 \textcolor{gregoriocolor}{A} & \textcolor{gregoriocolor}{B} & \textcolor{gregoriocolor}{C} & \textcolor{gregoriocolor}{D} & \textcolor{gregoriocolor}{E} & F & \textcolor{gregoriocolor}{F} & \textcolor{gregoriocolor}{G} & \textcolor{gregoriocolor}{H} & \textcolor{gregoriocolor}{M} & \textcolor{gregoriocolor}{N} & \textcolor{gregoriocolor}{P} \\
 12 & 13 & 14 & 15 & 16 & 17 & 16 & 17 & 18 & 19 & 20 & 21 \\
\end{tabularx}

\begin{paracol}{2}
\selectlanguage{latin}
\lettrine[lines=2]{S}{ancti} Jóachim, patris 
 immaculátæ Vírginis Genetrícis Dei Maríæ, Confessóris, cujus dies natális 
 refértur tertiodécimo Kaléndas Aprílis.
\switchcolumn
\selectlanguage{english}
\lettrine[lines=2]{S}{t.} Joachim, father of the most 
 Blessed Virgin Mary, Mother of God, Confessor. His birthday is noted 
 on the 20th of March.
\switchcolumn*
\selectlanguage{latin}
Romæ sancti Titi 
 Diáconi, qui, Urbe a Gothis occupáta, pecúnias paupéribus distríbuens, a 
 Tribúno bárbaro jussus est occídi.
\switchcolumn
\selectlanguage{english}
At Rome, St. Titus, deacon, who, 
 when the city was taken by the Goths, was put to death by a barbarous 
 tribune while distributing money to the poor.
\switchcolumn*
\selectlanguage{latin}
Nicææ, in Bithynia, 
 sancti Diomédis médici, qui in persecutióne Diocletiáni Imperatóris, pro 
 Christi fide cæsus gládio, martyrium complévit.
\switchcolumn
\selectlanguage{english}
At Nicaea in Bithynia, St. Diomede, 
 a physician who underwent martyrdom by being beheaded during the persecution 
 of Diocletian.
\switchcolumn*
\selectlanguage{latin}
In Palæstína sanctórum 
 trigínta trium Mártyrum.
\switchcolumn
\selectlanguage{english}
In Palestine thirty-three holy 
 martyrs.
\switchcolumn*
\selectlanguage{latin}
Ferentíni, in Hérnicis, 
 sancti Ambrósii Centuriónis, qui, in persecutióne Diocletiáni, váriis modis 
 cruciátus, et novíssime, cum per ignem illæsus transísset, demérsus in aquam, 
 edúctus est in refrigérium.
\switchcolumn
\selectlanguage{english}
At Ferentino in Campania, St. 
 Ambrose, centurion. In the persecution of Diocletian he was subjected 
 to different kinds of tortures, and finally passing through fire without 
 injury, was cast into the waters, and thus reached the place of eternal 
 rest.
\switchcolumn*
\selectlanguage{latin}
Medioláni deposítio 
 sancti Simpliciáni Epíscopi, sanctórum Ambrósii et Augustíni testimónio 
 célebris.
\switchcolumn
\selectlanguage{english}
At Milan, the death of St. 
 Simplician, bishop, renowned by the testimony of given of him by St. Ambrose 
 and St. Augustine.
\switchcolumn*
\selectlanguage{latin}
Antisiodóri sancti 
 Eleuthérii Epíscopi.
\switchcolumn
\selectlanguage{english}
At Auxerre, St. Eleutherius, bishop.
\switchcolumn*
\selectlanguage{latin}
Nicomedíæ sancti Arsácii 
 Confessóris, qui, sub Licínio persecutóre, milítia relícta, solitáriam vitam 
 ducens, tantis virtútibus cláruit, ut et dæmones expulísse et orándo 
 interemísse ingéntem dracónem legátur; dénique, futúram civitátis cladem 
 prænúntians, in oratióne spíritum Deo réddidit.
\switchcolumn
\selectlanguage{english}
At Nicomedia, St. Arsacius, 
 confessor. Under the persecution of Licinius he left the military 
 service, and leading a solitary life, became so famous for working miracles 
 that we read of his expelling the demons and killing a huge dragon by his 
 prayers. Finally he foretold the destruction of the city, and gave up 
 his soul to God in prayer.
\switchcolumn*
\selectlanguage{latin}
Apud Montem Pessulánum, 
 in Gállia Narbonénsi, deposítio beáti Rochi Confessóris, qui multas Itáliæ 
 urbes signo Crucis a morbo epidémiæ liberávit. Ipsíus corpus Venétias 
 póstea translátum, et in Ecclésia, ejus nómine consecráta, 
 honorificentíssime cónditum fuit.
\switchcolumn
\selectlanguage{english}
In France, near Montpellier, in the 
 province of Narbonne, the death of blessed Roch, confessor, who by the sing 
 of the cross, delivered many cities of Italy from an epidemic. His 
 body was afterwards transferred to Venice, and deposited with the greatest 
 honours in the church dedicated under his invocation.
\switchcolumn*
\selectlanguage{latin}
Romæ sanctæ Serénæ, 
 uxóris quondam Diocletiáni Augústi.
\switchcolumn
\selectlanguage{english}
At Rome, St. Serena, who had been 
 the wife of Emperor Diocletian.
\switchcolumn*
\selectlanguage{latin}
\end{paracol}


% ---- martyrology/mart08/mart0817.htm
\needspace{10\baselineskip}
\begin{paracol}{2}
\selectlanguage{latin}
\begin{center}{\color{gregoriocolor} Sextodécimo Kaléndas Septémbris. 
 Luna\dots\ }\end{center}
\switchcolumn
\selectlanguage{english}
\begin{center}{\color{gregoriocolor} The 
 Seventeenth Day of 
 August. The\dots\ Day of the Moon.}\end{center}
\end{paracol}

\noindent\begin{tabularx}{\linewidth}{*{19}{>{\centering\arraybackslash}X}}
 \textcolor{gregoriocolor}{a} & \textcolor{gregoriocolor}{b} & \textcolor{gregoriocolor}{c} & \textcolor{gregoriocolor}{d} & \textcolor{gregoriocolor}{e} & \textcolor{gregoriocolor}{f} & \textcolor{gregoriocolor}{g} & \textcolor{gregoriocolor}{h} & \textcolor{gregoriocolor}{i} & \textcolor{gregoriocolor}{k} & \textcolor{gregoriocolor}{l} & \textcolor{gregoriocolor}{m} & \textcolor{gregoriocolor}{n} & \textcolor{gregoriocolor}{p} & \textcolor{gregoriocolor}{q} & \textcolor{gregoriocolor}{r} & \textcolor{gregoriocolor}{s} & \textcolor{gregoriocolor}{t} & \textcolor{gregoriocolor}{u} \\
 23 & 24 & 25 & 26 & 27 & 28 & 29 & 1 & 2 & 3 & 4 & 5 & 6 & 7 & 8 & 9 & 10 & 11 & 12 \\
\end{tabularx}
\vspace{0.5\baselineskip}
\noindent\begin{tabularx}{\linewidth}{*{12}{>{\centering\arraybackslash}X}}
 \textcolor{gregoriocolor}{A} & \textcolor{gregoriocolor}{B} & \textcolor{gregoriocolor}{C} & \textcolor{gregoriocolor}{D} & \textcolor{gregoriocolor}{E} & F & \textcolor{gregoriocolor}{F} & \textcolor{gregoriocolor}{G} & \textcolor{gregoriocolor}{H} & \textcolor{gregoriocolor}{M} & \textcolor{gregoriocolor}{N} & \textcolor{gregoriocolor}{P} \\
 13 & 14 & 15 & 16 & 17 & 18 & 17 & 18 & 19 & 20 & 21 & 22 \\
\end{tabularx}

\begin{paracol}{2}
\selectlanguage{latin}
\lettrine[lines=1]{O}{ctáva} sancti Lauréntii 
 Mártyris.
\switchcolumn
\selectlanguage{english}
\lettrine[lines=1]{T}{he} Octave of St. Lawrence, martyr.
\switchcolumn*
\selectlanguage{latin}
Sancti Hyacínthi, ex 
 Ordine Prædicatórum, Confessóris, qui décimo octávo Kaléndas Septémbris 
 obdormívit in Dómino.
\switchcolumn
\selectlanguage{english}
St. Hyacinth, confessor of the Order 
 of Preachers, who fell asleep in the Lord on the 15th of August.
\switchcolumn*
\selectlanguage{latin}
Carthágine sanctórum 
 Mártyrum Liberáti abbátis, Bonifátii Diáconi, Servi et Rústici Subdiaconórum, 
 Rogáti et Séptimi Monachórum, et Máximi púeri; qui, in persecutióne 
 Wandálica, sub Hunneríco Rege, pro confessióne cathólicæ fídei et pro únici 
 Baptísmatis defensióne, váriis et inaudítis supplíciis exagitáti, demum 
 super ligna, quibus concremándi erant, clavis confíxi, et, cum ignis sæpius 
 accénsus fuísset ac divínitus semper exstínctus, Regis jussu remórum 
 véctibus percússi, et, comminútis cérebris, enecáti, speciósum cursum 
 certáminis sui, coronánte Dómino, perfecérunt.
\switchcolumn
\selectlanguage{english}
At Carthage in Africa, the holy 
 martyrs Liberatus, abbot, Boniface, a deacon, Servus and Rusticus, 
 subdeacons, Rogatus and Septimus, monks, and Maximus, a young child. 
 In the persecution of the Vandals, under King Hunneric, they were subjected 
 to various and unheard-of torments for the confession of the Catholic faith 
 and the defence of one baptism. Finally, being nailed to the wood with 
 which they were to be burned, as the fire was always miraculously 
 extinguished whenever kindled, they were struck with iron bars by order of 
 the tyrant until their brains were dashed out. Thus they ended the 
 glorious series of their combats, and were crowned by our Lord.
\switchcolumn*
\selectlanguage{latin}
In Achája sancti 
 Myrónis, Presbyteri et Mártyris, qui, sub Décio Imperatóre et Antípatre 
 Præside, Cyzici, post multa torménta, cápite truncátus est.
\switchcolumn
\selectlanguage{english}
In Achaia, St. Myron, priest and 
 martyr, who was beheaded at Cyzicum after undergoing many torments, in the 
 time of Emperor Decius and the governor Antipater.
\switchcolumn*
\selectlanguage{latin}
Cæsaréæ, in Cappadócia, natális sancti Mamántis Mártyris, qui, sanctórum Theódoti et Rufínæ Mártyrum 
 fílius, longum a puerítia ad senectútem usque martyrium duxit, et tandem, 
 imperánte Aureliáno, sub Alexándro Præside, illud felíciter consummávit; 
 quem sancti Patres Basilíus et Gregórius Nazianzénus summis láudibus 
 celebrárunt.
\switchcolumn
\selectlanguage{english}
At Caesarea in Cappadocia, the 
 birthday of St. Mamas, martyr, the son of Saints Theodotus and Rufina, 
 martyrs, who, from childhood to old age, endured a long martyrdom, and at 
 length ended it happily in the reign of Aurelian, under the governor 
 Alexander. He has been highly praised by the holy Fathers Basil and 
 Gregory Nazianzen.
\switchcolumn*
\selectlanguage{latin}
Nicomedíæ sanctórum 
 Mártyrum Stratónis, Philíppi et Eutychiáni; qui, damnáti ad béstias et nil 
 læsi, per ignem martyrium consummárunt.
\switchcolumn
\selectlanguage{english}
At Nicomedia, the holy martyrs 
 Straton, Philip, and Eutychian, who were condemned to the beasts, but being 
 uninjured by them, ended their martyrdom by fire.
\switchcolumn*
\selectlanguage{latin}
Ptolemáide, in 
 Palæstína, pássio sanctórum Mártyrum Pauli, ejúsque soróris Juliánæ Vírginis; 
 qui ambo, sub Aureliáno Imperatóre, cum in Christi confessióne permanérent 
 immóbiles, jussi sunt váriis et diríssimis torméntis afflígi ac tandem 
 cápite obtruncári.
\switchcolumn
\selectlanguage{english}
At Ptolemais in Palestine, the holy 
 martyrs Paul and his sister Juliana, virgin, who suffered under Aurelian. 
 They were both punished with various cruel torments and were finally 
 beheaded for their constancy in confessing the name of Christ.
\switchcolumn*
\selectlanguage{latin}
Romæ sancti Eusébii 
 Papæ.
\switchcolumn
\selectlanguage{english}
At Rome, Pope St. Eusebius.
\switchcolumn*
\selectlanguage{latin}
Interámnæ sancti 
 Anastásii, Epíscopi et Confessóris.
\switchcolumn
\selectlanguage{english}
At Teramo, St. Anastasius, bishop 
 and confessor.
\switchcolumn*
\selectlanguage{latin}
In Monte Falco, in 
 Umbria, sanctæ Claræ, Moniális ex Ordine Eremitárum sancti Augustíni, 
 Vírginis; in cujus viscéribus renováta Domínicæ passiónis mystéria fidéles, 
 máxima cum devotióne, venerántur. Eam Leo Décimus tértius, Summus 
 Póntifex, sanctárum Vírginum albo adscrípsit.
\switchcolumn
\selectlanguage{english}
At Montefalco in Umbria, St. Clare, 
 a nun of the Order of Hermits of St. Augustine, virgin. In her flesh 
 were renewed the mysteries of the Lord's passion, which the faithful honour 
 with great devotion. Pope Leo XIII solemnly inscribed her in the list 
 of the holy virgins.
\switchcolumn*
\selectlanguage{latin}
\end{paracol}


% ---- martyrology/mart08/mart0818.htm
\needspace{10\baselineskip}
\begin{paracol}{2}
\selectlanguage{latin}
\begin{center}{\color{gregoriocolor} Quintodécimo Kaléndas Septémbris. 
 Luna\dots\ }\end{center}
\switchcolumn
\selectlanguage{english}
\begin{center}{\color{gregoriocolor} The 
 Eighteenth Day of 
 August. The\dots\ Day of the Moon.}\end{center}
\end{paracol}

\noindent\begin{tabularx}{\linewidth}{*{19}{>{\centering\arraybackslash}X}}
 \textcolor{gregoriocolor}{a} & \textcolor{gregoriocolor}{b} & \textcolor{gregoriocolor}{c} & \textcolor{gregoriocolor}{d} & \textcolor{gregoriocolor}{e} & \textcolor{gregoriocolor}{f} & \textcolor{gregoriocolor}{g} & \textcolor{gregoriocolor}{h} & \textcolor{gregoriocolor}{i} & \textcolor{gregoriocolor}{k} & \textcolor{gregoriocolor}{l} & \textcolor{gregoriocolor}{m} & \textcolor{gregoriocolor}{n} & \textcolor{gregoriocolor}{p} & \textcolor{gregoriocolor}{q} & \textcolor{gregoriocolor}{r} & \textcolor{gregoriocolor}{s} & \textcolor{gregoriocolor}{t} & \textcolor{gregoriocolor}{u} \\
 24 & 25 & 26 & 27 & 28 & 29 & 1 & 2 & 3 & 4 & 5 & 6 & 7 & 8 & 9 & 10 & 11 & 12 & 13 \\
\end{tabularx}
\vspace{0.5\baselineskip}
\noindent\begin{tabularx}{\linewidth}{*{12}{>{\centering\arraybackslash}X}}
 \textcolor{gregoriocolor}{A} & \textcolor{gregoriocolor}{B} & \textcolor{gregoriocolor}{C} & \textcolor{gregoriocolor}{D} & \textcolor{gregoriocolor}{E} & F & \textcolor{gregoriocolor}{F} & \textcolor{gregoriocolor}{G} & \textcolor{gregoriocolor}{H} & \textcolor{gregoriocolor}{M} & \textcolor{gregoriocolor}{N} & \textcolor{gregoriocolor}{P} \\
 14 & 15 & 16 & 17 & 18 & 19 & 18 & 19 & 20 & 21 & 22 & 23 \\
\end{tabularx}

\begin{paracol}{2}
\selectlanguage{latin}
\lettrine[lines=2]{P}{rænéste} natális sancti 
 Agapíti Mártyris, qui, cum esset annórum quíndecim et amóre Christi fervéret, 
 jussu Aureliáni Imperatóris tentus est, ac primo nervis crudis diutíssime 
 cæsus, deínde, sub Antíocho Præfécto, gravióra supplícia passus; exínde, cum 
 ex Imperatóris præcépto leónibus objicerétur et mínime læsus esset, gládio 
 ministrórum coronándus percútitur.
\switchcolumn
\selectlanguage{english}
\lettrine[lines=2]{A}{t} Palestrina, the birthday of the 
 holy martyr Agapitus. Although only fifteen years of age, because he 
 was fervent in the love of Christ, he was arrested by order of Emperor 
 Aurelian, and scourged for a long time. Afterwards, under the prefect 
 Antiochus, he endured more severe torments, and being delivered to the lions 
 by the emperor's order without receiving any injury, he was finally struck 
 with the sword, and thus merited his crown.
\switchcolumn*
\selectlanguage{latin}
Romæ beatórum Joánnis 
 et Crispi Presbyterórum, qui, in persecutióne Diocletiáni, multa Sanctórum 
 córpora officiosíssime sepeliérunt, quorum méritis et ipsi póstmodum sociáti, 
 gáudia vitæ ætérnæ sibi comparárunt.
\switchcolumn
\selectlanguage{english}
At Rome, during the persecution of 
 Diocletian, the blessed John and Crispus, priests, who charitably buried the 
 bodies of many saints; afterwards becoming partakers of their merits, they 
 deserved the joys of eternal life.
\switchcolumn*
\selectlanguage{latin}
Item Romæ sanctórum 
 Mártyrum Hermæ, Serapiónis et Polyæni, qui, per angústa, saxósa et áspera 
 loca raptáti, ánimas Deo reddidérunt.
\switchcolumn
\selectlanguage{english}
In the same city, the holy martyrs 
 Hermas, Serapion, and Polyaenus. Being dragged through narrow, stony, 
 and rough places, they yielded up their souls to God.
\switchcolumn*
\selectlanguage{latin}
In Illyrico sanctórum 
 Mártyrum Flori et Lauri, artis lapicidínæ, qui, sub Licióne Præside, 
 martyrio consúmptis eórum magístris Próculo et Máximo, ambo, post multa 
 torménta, in profúndum púteum sunt demérsi.
\switchcolumn
\selectlanguage{english}
In Illyria, the holy martyrs Florus 
 and Laurus, stonecutters, who, after the martyrdom of Proculus and Maximus, 
 their employers, were subjected to many torments under the governor Licion, 
 and plunged into a deep well.
\switchcolumn*
\selectlanguage{latin}
Myræ, in Lycia, 
 sanctórum Mártyrum Leónis et Juliánæ.
\switchcolumn
\selectlanguage{english}
At Myra in Lycia, the holy martyrs 
 Leo and Juliana.
\switchcolumn*
\selectlanguage{latin}
Metis, in Gállia, 
 sancti Firmíni, Epíscopi et Confessóris.
\switchcolumn
\selectlanguage{english}
At Metz in France, St. Firmin, 
 bishop and confessor.
\switchcolumn*
\selectlanguage{latin}
Romæ, via Lavicána, 
 sanctæ Hélenæ, matris Constantíni Magni, piíssimi Imperatóris, qui primus 
 egrégium Ecclésiæ tuéndæ atque amplificándæ exémplum céteris Princípibus 
 præbuit.
\switchcolumn
\selectlanguage{english}
At Rome, on the Via Lavicana, St. 
 Helena, mother of the religious emperor Constantine the Great, who was the 
 first to set the example to other princes of protecting and extending the 
 Church.
\switchcolumn*
\selectlanguage{latin}
\end{paracol}


% ---- martyrology/mart08/mart0819.htm
\needspace{10\baselineskip}
\begin{paracol}{2}
\selectlanguage{latin}
\begin{center}{\color{gregoriocolor} Quartodécimo Kaléndas Septémbris. 
 Luna\dots\ }\end{center}
\switchcolumn
\selectlanguage{english}
\begin{center}{\color{gregoriocolor} The 
 Nineteenth Day of 
 August. The\dots\ Day of the Moon.}\end{center}
\end{paracol}

\noindent\begin{tabularx}{\linewidth}{*{19}{>{\centering\arraybackslash}X}}
 \textcolor{gregoriocolor}{a} & \textcolor{gregoriocolor}{b} & \textcolor{gregoriocolor}{c} & \textcolor{gregoriocolor}{d} & \textcolor{gregoriocolor}{e} & \textcolor{gregoriocolor}{f} & \textcolor{gregoriocolor}{g} & \textcolor{gregoriocolor}{h} & \textcolor{gregoriocolor}{i} & \textcolor{gregoriocolor}{k} & \textcolor{gregoriocolor}{l} & \textcolor{gregoriocolor}{m} & \textcolor{gregoriocolor}{n} & \textcolor{gregoriocolor}{p} & \textcolor{gregoriocolor}{q} & \textcolor{gregoriocolor}{r} & \textcolor{gregoriocolor}{s} & \textcolor{gregoriocolor}{t} & \textcolor{gregoriocolor}{u} \\
 25 & 26 & 27 & 28 & 29 & 1 & 2 & 3 & 4 & 5 & 6 & 7 & 8 & 9 & 10 & 11 & 12 & 13 & 14 \\
\end{tabularx}
\vspace{0.5\baselineskip}
\noindent\begin{tabularx}{\linewidth}{*{12}{>{\centering\arraybackslash}X}}
 \textcolor{gregoriocolor}{A} & \textcolor{gregoriocolor}{B} & \textcolor{gregoriocolor}{C} & \textcolor{gregoriocolor}{D} & \textcolor{gregoriocolor}{E} & F & \textcolor{gregoriocolor}{F} & \textcolor{gregoriocolor}{G} & \textcolor{gregoriocolor}{H} & \textcolor{gregoriocolor}{M} & \textcolor{gregoriocolor}{N} & \textcolor{gregoriocolor}{P} \\
 15 & 16 & 17 & 18 & 19 & 20 & 19 & 20 & 21 & 22 & 23 & 24 \\
\end{tabularx}

\begin{paracol}{2}
\selectlanguage{latin}
\lettrine[lines=2]{C}{adómi,} in Gállia, 
 sancti Joánnis Eudes Confessóris, Missionárii Apostólici, Fundatóris 
 Congregatiónis Presbyterórum Jesu et Maríæ necnon Ordinis Moniálium Dóminæ 
 Nostræ a Caritáte, et promotóris litúrgici cultus erga Sacratíssima Christi 
 ejúsque Genitrícis Corda; quem Pius Papa Undécimus fastis Sanctórum 
 adscrípsit.
\switchcolumn
\selectlanguage{english}
\lettrine[lines=2]{A}{t} Caen in France, St. John Eudes, 
 apostolic missionary, founder of the Congregation of Priests of Jesus and 
 Mary and of the Order of Nuns of our Lady of Charity, and the promoter of 
 the liturgical cult towards the most sacred Hearts of Christ and his Mother. 
 He was canonized by Pope Pius XI.
\switchcolumn*
\selectlanguage{latin}
Romæ sancti Júlii, 
 Senatóris et Mártyris; qui, Vitéllio Júdici tráditus et ab eo in cárcerem 
 trusus, támdiu, jubénte Cómmodo Imperatóre, fústibus cæsus est, donec 
 emítteret spíritum. Ipsíus autem corpus in cœmetério Calepódii, via 
 Aurélia, sepúltum fuit.
\switchcolumn
\selectlanguage{english}
At Rome, St. Julius, senator and 
 martyr, who was delivered up to the judge Vitellius, and thrown into prison 
 by him. By order of Emperor Commodus, he was beaten with rods until he 
 expired. His body was buried in the cemetery of Caleposius on the 
 Aurelian Way.
\switchcolumn*
\selectlanguage{latin}
Anágniæ sancti Magni, 
 Epíscopi et Mártyris, qui in persecutióne Décii necátus est.
\switchcolumn
\selectlanguage{english}
At Anagni, St. Magnus, bishop and 
 martyr, who was put to death in the persecution of Decius.
\switchcolumn*
\selectlanguage{latin}
In Cilícia natális 
 sancti Andréæ Tribúni, et Sociórum mílitum; qui, victória de Persis 
 divínitus obténta, ad Christi fidem sunt convérsi, et, hoc nómine accusáti, 
 sub Maximiáno Imperatóre, in angústiis Tauri montis, a Seléuci Præsidis 
 exércitu trucidáti sunt.
\switchcolumn
\selectlanguage{english}
In Cilicia, the birthday of St. 
 Andrew, tribune, and his military companions, who were converted to 
 Christianity through a miraculous victory they had gained over the Persians. 
 Being accused on this account, they were massacred in the Mount Taurus pass, 
 by the army of the governor Seleucus, under Emperor Maximian.
\switchcolumn*
\selectlanguage{latin}
In Palæstína sancti 
 Timóthei Mártyris, qui, in persecutióne Diocletiáni, sub Urbáno Præside, 
 post multa superáta supplícia, lento igne combústus est. Passi sunt 
 étiam ibídem Thecla et Agápius, e quibus Thecla, feris expósita, eárum 
 laniáta déntibus transívit ad Sponsum; Agápius vero, plúrima torménta 
 perpéssus, ad majóra certámina fuit dilátus.
\switchcolumn
\selectlanguage{english}
In Palestine, St. Timothy, a martyr 
 in the persecution of Diocletian, under the governor Urbanus. After 
 overcoming many torments, he was consumed with a slow fire. In the 
 same country there suffered also Thecla and Agapius. The former, being 
 exposed to the beasts, was torn to pieces by their teeth, and went to her 
 Spouse; but Agapius, after enduring many torments, was reserved for greater 
 trials.
\switchcolumn*
\selectlanguage{latin}
Romæ sancti Xysti 
 Tértii, Papæ et Confessóris.
\switchcolumn
\selectlanguage{english}
At Rome, St. Sixtus III, pope and 
 confessor.
\switchcolumn*
\selectlanguage{latin}
Apud castrum Bríncolam, 
 in Província, deposítio sancti Ludovíci, ex Ordine Minórum, Epíscopi 
 Tolosiáni, vitæ sanctitáte et miráculis clari; cujus corpus, inde Massíliam 
 translátum, in Ecclésia Fratrum Minórum honorífice cónditum fuit, ac póstea 
 Valéntiam, in Hispánia, devéctum est, atque in cathedráli Ecclésia 
 collocátum.
\switchcolumn
\selectlanguage{english}
In Provence, at the village of 
 Brignoles, the death of St. Louis, bishop of Toulouse, of the Order of 
 Friars Minor, renowned for holiness of life and miracles. His body was 
 taken to Marseilles, and buried with due honours in the Church of the Friars 
 Minor, but afterwards it was taken to Valencia in Spain, and enshrined in 
 the cathedral.
\switchcolumn*
\selectlanguage{latin}
In pago Sigistérico, in 
 Gállia, beáti Donáti, Presbyteri et Confessóris; qui, ab ipsis usque 
 infántiæ rudiméntis mira Dei grátia præditus, anachoréticam vitam multis 
 annis exégit, et miraculórum glória clarus migrávit ad Christum.
\switchcolumn
\selectlanguage{english}
In the neighbourhood of Sisteron in 
 France, blessed Donatus, priest and confessor. Being from his very 
 infancy endowed with the grace of God in an extraordinary manner, he lived 
 the life of an anchoret for many years, and after having been renowned for 
 glorious miracles, went to Christ.
\switchcolumn*
\selectlanguage{latin}
In território 
 Bituricénsi sancti Mariáni Confessóris, cujus virtútes et mirácula beátus 
 Gregórius, Turonénsis Epíscopus, magnis láudibus celebrávit.
\switchcolumn
\selectlanguage{english}
In the territory of Bourges, St. 
 Marianus, confessor, whose virtues and miracles were described with great 
 praise by St. Gregory, bishop of Tours.
\switchcolumn*
\selectlanguage{latin}
Mántuæ sancti Rufíni 
 Confessóris.
\switchcolumn
\selectlanguage{english}
At Mantua, St. Rufinus, confessor.
\switchcolumn*
\selectlanguage{latin}
Norimbérgæ sancti 
 Sebáldi Eremítæ, virtútibus et miráculis præclári, qui Sanctórum catálogo a 
 Martíno Papa Quinto adjéctus est.
\switchcolumn
\selectlanguage{english}
At Nuremburg, St. Sebald, hermit, 
 noted for his virtues and miracles. Pope Martin V added his name to 
 the list of the saints.
\switchcolumn*
\selectlanguage{latin}
\end{paracol}


% ---- martyrology/mart08/mart0820.htm
\needspace{10\baselineskip}
\begin{paracol}{2}
\selectlanguage{latin}
\begin{center}{\color{gregoriocolor} Tertiodécimo Kaléndas Septémbris. 
 Luna\dots\ }\end{center}
\switchcolumn
\selectlanguage{english}
\begin{center}{\color{gregoriocolor} The 
 Twentieth Day of 
 August. The\dots\ Day of the Moon.}\end{center}
\end{paracol}

\noindent\begin{tabularx}{\linewidth}{*{19}{>{\centering\arraybackslash}X}}
 \textcolor{gregoriocolor}{a} & \textcolor{gregoriocolor}{b} & \textcolor{gregoriocolor}{c} & \textcolor{gregoriocolor}{d} & \textcolor{gregoriocolor}{e} & \textcolor{gregoriocolor}{f} & \textcolor{gregoriocolor}{g} & \textcolor{gregoriocolor}{h} & \textcolor{gregoriocolor}{i} & \textcolor{gregoriocolor}{k} & \textcolor{gregoriocolor}{l} & \textcolor{gregoriocolor}{m} & \textcolor{gregoriocolor}{n} & \textcolor{gregoriocolor}{p} & \textcolor{gregoriocolor}{q} & \textcolor{gregoriocolor}{r} & \textcolor{gregoriocolor}{s} & \textcolor{gregoriocolor}{t} & \textcolor{gregoriocolor}{u} \\
 26 & 27 & 28 & 29 & 1 & 2 & 3 & 4 & 5 & 6 & 7 & 8 & 9 & 10 & 11 & 12 & 13 & 14 & 15 \\
\end{tabularx}
\vspace{0.5\baselineskip}
\noindent\begin{tabularx}{\linewidth}{*{12}{>{\centering\arraybackslash}X}}
 \textcolor{gregoriocolor}{A} & \textcolor{gregoriocolor}{B} & \textcolor{gregoriocolor}{C} & \textcolor{gregoriocolor}{D} & \textcolor{gregoriocolor}{E} & F & \textcolor{gregoriocolor}{F} & \textcolor{gregoriocolor}{G} & \textcolor{gregoriocolor}{H} & \textcolor{gregoriocolor}{M} & \textcolor{gregoriocolor}{N} & \textcolor{gregoriocolor}{P} \\
 16 & 17 & 18 & 19 & 20 & 21 & 20 & 21 & 22 & 23 & 24 & 25 \\
\end{tabularx}

\begin{paracol}{2}
\selectlanguage{latin}
\lettrine[lines=2]{I}{n} território 
 Lingoniénsi deposítio sancti Bernárdi, primi Clarævallénsis Abbátis, vita, 
 doctrína et miráculis gloriósi, quem Pius Octávus, Póntifex Máximus, 
 universális Ecclésiæ Doctórem declarávit et confirmávit.
\switchcolumn
\selectlanguage{english}
\lettrine[lines=2]{I}{n} the territory of Langres, the 
 death of St. Bernard, first abbot of Clairvaux, illustrious for virtues, 
 learning, and miracles. He was declared and confirmed doctor of the 
 Universal Church by the Sovereign Pontiff, Pius VIII.
\switchcolumn*
\selectlanguage{latin}
Romæ deposítio sancti 
 Pii Décimi, Papæ et Confessóris, fídei integritátis et ecclesiásticæ 
 libertátis propugnatóris invícti, religionísque zelo insígnis, cujus festum 
 tértio Nonas Septémbris recólitur.
\switchcolumn
\selectlanguage{english}
At Rome, the death of St. Pius X, 
 pope and confessor, who championed the integrity of the faith and the 
 liberty of the Church, and was renowned for his religious zeal. His 
 feastday is celebrated on the 3rd of September.
\switchcolumn*
\selectlanguage{latin}
Apud montem Senárium, 
 in Etrúria, natális sancti Manétti Confessóris, e septem Fundatóribus 
 Ordinis Servórum beátæ Maríæ Vírginis; qui, eídem hymnos dicens, exspirávit. 
 Ipsíus autem ac Sociórum festum prídie Idus Februárii celebrátur.
\switchcolumn
\selectlanguage{english}
On Mount Senario in Tuscany, the 
 birthday of St. Manetto, confessor, one of the seven founders of the Order 
 of the Servites of the Blessed Virgin Mary, who died as he was repeating a 
 hymn to her. His feast, with that of his companions, is kept on the 
 12th of February.
\switchcolumn*
\selectlanguage{latin}
In Judæa sancti 
 Samuélis Prophétæ, cujus sacra ossa (ut beátus Hierónymus scribit) Arcádius 
 Augústus Constantinópolim tránstulit, et prope Séptimum collocávit.
\switchcolumn
\selectlanguage{english}
In Judea, the holy prophet Samuel, 
 whose holy relics (as is related by St. Jerome) were taken to Constantinople 
 by Emperor Arcadius, and deposited near Septimum.
\switchcolumn*
\selectlanguage{latin}
In Cypro sancti Lúcii 
 Senatóris, qui, perspécta constántia Theodóri, Cyrenénsis Epíscopi, in 
 martyrio pósiti, ad Christi fidem est convérsus, et ad eam étiam Digniánum 
 Præsidem pertráxit; cum eóque Cyprum proféctus, ibi, cum álios Christiános 
 pro confessióne Dómini coronári vidéret, ultro se ipsum óbtulit, et cápitis 
 obtruncatióne eándem martyrii corónam proméruit.
\switchcolumn
\selectlanguage{english}
In Cyprus, St. Lucius, senator, who 
 was converted to the faith on seeing the constancy of Theodore, bishop of 
 Cyrene, during his martyrdom. He also converted the governor Dignian, 
 with whom he set out for Cyprus, where, seeing other Christians crowned for 
 the confession of the Lord, he offered himself voluntarily, and merited the 
 same crown of martyrdom by beheading.
\switchcolumn*
\selectlanguage{latin}
In Thrácia sanctórum 
 trigínta septem Mártyrum, qui, sub Præside Apelliáno, pro Christi fide, 
 mánibus pedibúsque præcísis, in camínum ardéntem injécti sunt.
\switchcolumn
\selectlanguage{english}
In Thrace, in the time of the 
 governor Apellian, thirty-seven holy martyrs, who had their hands and feet 
 cut off for the faith of Christ, and were cast into a burning furnace.
\switchcolumn*
\selectlanguage{latin}
Ibídem sanctórum 
 Mártyrum Sevéri, et Memnónis Centuriónis; qui, eódem mortis génere 
 consummáti, victóres abiérunt in cælum.
\switchcolumn
\selectlanguage{english}
Also, the holy martyrs Severus, and 
 the centurion Memnon, who, suffering the same kind of death, went 
 victoriously to heaven.
\switchcolumn*
\selectlanguage{latin}
Córdubæ, in Hispánia, 
 sanctórum Mártyrum Leovigíldi et Christóphori Monachórum, qui, in Arabum 
 persecutióne, pro Christiánæ fídei defensióne in cárcerem conjécti, ac mox, 
 cervícibus abscíssis, igni tráditi, martyrii palmam adépti sunt.
\switchcolumn
\selectlanguage{english}
At Cordova, during the persecution 
 of the Arabs, the holy martyrs Leovigild and Christopher, monks, who were 
 thrust into prison for the defence of the Christian faith, and soon after, 
 being beheaded and cast into the fire, thus obtained the palm of martyrdom.
\switchcolumn*
\selectlanguage{latin}
In Hério ínsula sancti 
 Philibérti Abbátis.
\switchcolumn
\selectlanguage{english}
In the island of Hermoutier, St. 
 Philibert, abbot.
\switchcolumn*
\selectlanguage{latin}
Romæ beáti Porphyrii, 
 qui fuit homo Dei, et sanctum Mártyrem Agapítum erudívit in fide et doctrína 
 Christi.
\switchcolumn
\selectlanguage{english}
At Rome, blessed Porphyry, a man of 
 God, who instructed the holy martyr Agapitus in the faith and doctrine of 
 Christ.
\switchcolumn*
\selectlanguage{latin}
In castro Cainóne, in 
 Gállia, sancti Máximi Confessóris, qui éxstitit discípulus beáti Mártini 
 Epíscopi.
\switchcolumn
\selectlanguage{english}
At Chinon, St. Maximus, confessor, 
 disciple of the blessed bishop Martin.
\switchcolumn*
\selectlanguage{latin}
\end{paracol}


% ---- martyrology/mart08/mart0821.htm
\needspace{10\baselineskip}
\begin{paracol}{2}
\selectlanguage{latin}
\begin{center}{\color{gregoriocolor} Duodécimo Kaléndas Septémbris. 
 Luna\dots\ }\end{center}
\switchcolumn
\selectlanguage{english}
\begin{center}{\color{gregoriocolor} The 
 Twenty-First Day of 
 August. The\dots\ Day of the Moon.}\end{center}
\end{paracol}

\noindent\begin{tabularx}{\linewidth}{*{19}{>{\centering\arraybackslash}X}}
 \textcolor{gregoriocolor}{a} & \textcolor{gregoriocolor}{b} & \textcolor{gregoriocolor}{c} & \textcolor{gregoriocolor}{d} & \textcolor{gregoriocolor}{e} & \textcolor{gregoriocolor}{f} & \textcolor{gregoriocolor}{g} & \textcolor{gregoriocolor}{h} & \textcolor{gregoriocolor}{i} & \textcolor{gregoriocolor}{k} & \textcolor{gregoriocolor}{l} & \textcolor{gregoriocolor}{m} & \textcolor{gregoriocolor}{n} & \textcolor{gregoriocolor}{p} & \textcolor{gregoriocolor}{q} & \textcolor{gregoriocolor}{r} & \textcolor{gregoriocolor}{s} & \textcolor{gregoriocolor}{t} & \textcolor{gregoriocolor}{u} \\
 27 & 28 & 29 & 1 & 2 & 3 & 4 & 5 & 6 & 7 & 8 & 9 & 10 & 11 & 12 & 13 & 14 & 15 & 16 \\
\end{tabularx}
\vspace{0.5\baselineskip}
\noindent\begin{tabularx}{\linewidth}{*{12}{>{\centering\arraybackslash}X}}
 \textcolor{gregoriocolor}{A} & \textcolor{gregoriocolor}{B} & \textcolor{gregoriocolor}{C} & \textcolor{gregoriocolor}{D} & \textcolor{gregoriocolor}{E} & F & \textcolor{gregoriocolor}{F} & \textcolor{gregoriocolor}{G} & \textcolor{gregoriocolor}{H} & \textcolor{gregoriocolor}{M} & \textcolor{gregoriocolor}{N} & \textcolor{gregoriocolor}{P} \\
 17 & 18 & 19 & 20 & 21 & 22 & 21 & 22 & 23 & 24 & 25 & 26 \\
\end{tabularx}

\begin{paracol}{2}
\selectlanguage{latin}
\lettrine[lines=2]{S}{anctæ} Joánnæ-Francíscæ 
 Frémiot de Chantal, Víduæ, quæ Ordinis Sanctimoniálium Visitatiónis sanctæ 
 Maríæ fuit Institútrix, cujus dies natális recólitur Idibus Decémbris.
\switchcolumn
\selectlanguage{english}
\lettrine[lines=2]{T}{he} festival of St. Jane Frances 
 Fremiot de Chantal, foundress of the Order of Nuns of the Visitation of St. 
 Mary, whose birthday is commemorated on the 13th of December.
\switchcolumn*
\selectlanguage{latin}
Romæ, in agro Veráno, 
 sanctæ Cyríacæ, Víduæ et Mártyris; quæ, in persecutióne Valeriáni, cum se 
 súaque ómnia in Sanctórum ministéria impendísset, demum, martyrium pro 
 Christo súbiens, vitam quoque ipsam libénter impéndit.
\switchcolumn
\selectlanguage{english}
At Rome, in the Agro Verano, St. 
 Cyriaca, widow and martyr. In the persecution of Valerian, after 
 devoting herself and all her goods in the service of the saints, she gave up 
 her life by suffering martyrdom for Christ.
\switchcolumn*
\selectlanguage{latin}
In território 
 Gavalitáno sancti Priváti, Epíscopi et Mártyris; qui passus est in 
 persecutióne Valeriáni et Galliéni.
\switchcolumn
\selectlanguage{english}
In Gevaudan, St. Privatus, bishop 
 and martyr, who suffered in the persecution of Valerian and Gallienus.
\switchcolumn*
\selectlanguage{latin}
Salónæ, in Dalmátia, sancti Anastásii Corniculárii, qui, cum vidéret beátum Agapítum constánter 
 torménta perferéntem, convérsus est ad fidem, et, pro confessióne nóminis 
 Christi, jubénte Aureliáno Imperatóre, interémptus, Martyr migrávit ad 
 Dóminum.
\switchcolumn
\selectlanguage{english}
At Salona in Dalmatia, St. 
 Anastasius, a law officer, who was converted to the faith by seeing the 
 fortitude with which blessed Agapitus bore his torments, and being put to 
 death by order of Emperor Aurelian for confessing the name of Christ, went 
 to our Lord, a martyr.
\switchcolumn*
\selectlanguage{latin}
In Sardínia natális 
 sanctórum Mártyrum Luxórii, Cisélli et Cameríni; qui, in persecutióne 
 Diocletiáni, sub Délphio Præside, gládio cæsi sunt.
\switchcolumn
\selectlanguage{english}
In Sardinia, the birthday of the 
 holy martyrs Luxorius, Cisellus, and Camerinus, who were put to the sword in 
 the persecution of Diocletian, under the governor Delphius.
\switchcolumn*
\selectlanguage{latin}
Eódem die sanctórum 
 Mártyrum Bonósi et Maximiáni.
\switchcolumn
\selectlanguage{english}
On the same day, the holy martyrs 
 Bononus and Maximian.
\switchcolumn*
\selectlanguage{latin}
Fundis, in Látio, 
 sancti Patérni Mártyris, qui, ab Alexandría Romam venit ad Apostolórum 
 memórias; et inde in agrum Fundánum secéssit, atque ibi, cum Mártyrum 
 córpora sepelíret, a Tribúno comprehénsus est, et in vínculis exspirávit.
\switchcolumn
\selectlanguage{english}
At Fundi in Campania, St. Paternus, 
 a martyr, who came from Alexandria to Rome to visit the tomb of the 
 apostles. Thence he retired to the neighbourhood of Fundi, where, 
 being seized by the tribune while he was burying the bodies of the martyrs, 
 he died in captivity.
\switchcolumn*
\selectlanguage{latin}
Edéssæ, in Syria, 
 sanctórum Mártyrum Bassæ, ac trium ejus filiórum, id est Theogónii, Agápii 
 et Fidélis; quos, in persecutióne Maximiáni, pia mater exhórtans, martyrio 
 coronátos præmísit ad palmam, et, truncáto cápite, gaudens secúta est cum 
 victória.
\switchcolumn
\selectlanguage{english}
At Edessa in Syria, during the 
 persecution of Maximian, the holy martyrs Bassa, and her sons Theogonius, 
 Agapius, and Fidelis, whom their pious mother exhorted to martyrdom and sent 
 before her bearing their crowns. Being herself beheaded, she joyfully 
 followed them and shared their victory.
\switchcolumn*
\selectlanguage{latin}
Verónæ sancti Euprépii, 
 Epíscopi et Confessóris.
\switchcolumn
\selectlanguage{english}
At Verona, St. Euprepius, bishop and 
 confessor.
\switchcolumn*
\selectlanguage{latin}
Item sancti Quadráti 
 Epíscopi.
\switchcolumn
\selectlanguage{english}
Also, St. Quadratus, bishop.
\switchcolumn*
\selectlanguage{latin}
Arvérnis, in Gállia, 
 sancti Sidónii Epíscopi, doctrína et sanctitáte conspícui.
\switchcolumn
\selectlanguage{english}
In Auvergne in France, St. Sidonius, 
 bishop, noted for learning and holiness.
\switchcolumn*
\selectlanguage{latin}
Senis, in Túscia, beáti 
 Bernárdi Ptolomæi Abbátis, Congregatiónis Olivetánæ Fundatóris.
\switchcolumn
\selectlanguage{english}
At Siena in Tuscany, blessed Bernard 
 Ptolemy, abbot and founder of the Congregation of Olivetans.
\switchcolumn*
\selectlanguage{latin}
\end{paracol}


% ---- martyrology/mart08/mart0822.htm
\needspace{10\baselineskip}
\begin{paracol}{2}
\selectlanguage{latin}
\begin{center}{\color{gregoriocolor} Undécimo Kaléndas Septémbris. 
 Luna\dots\ }\end{center}
\switchcolumn
\selectlanguage{english}
\begin{center}{\color{gregoriocolor} The 
 Twenty-Second Day of 
 August. The\dots\ Day of the Moon.}\end{center}
\end{paracol}

\noindent\begin{tabularx}{\linewidth}{*{19}{>{\centering\arraybackslash}X}}
 \textcolor{gregoriocolor}{a} & \textcolor{gregoriocolor}{b} & \textcolor{gregoriocolor}{c} & \textcolor{gregoriocolor}{d} & \textcolor{gregoriocolor}{e} & \textcolor{gregoriocolor}{f} & \textcolor{gregoriocolor}{g} & \textcolor{gregoriocolor}{h} & \textcolor{gregoriocolor}{i} & \textcolor{gregoriocolor}{k} & \textcolor{gregoriocolor}{l} & \textcolor{gregoriocolor}{m} & \textcolor{gregoriocolor}{n} & \textcolor{gregoriocolor}{p} & \textcolor{gregoriocolor}{q} & \textcolor{gregoriocolor}{r} & \textcolor{gregoriocolor}{s} & \textcolor{gregoriocolor}{t} & \textcolor{gregoriocolor}{u} \\
 28 & 29 & 1 & 2 & 3 & 4 & 5 & 6 & 7 & 8 & 9 & 10 & 11 & 12 & 13 & 14 & 15 & 16 & 17 \\
\end{tabularx}
\vspace{0.5\baselineskip}
\noindent\begin{tabularx}{\linewidth}{*{12}{>{\centering\arraybackslash}X}}
 \textcolor{gregoriocolor}{A} & \textcolor{gregoriocolor}{B} & \textcolor{gregoriocolor}{C} & \textcolor{gregoriocolor}{D} & \textcolor{gregoriocolor}{E} & F & \textcolor{gregoriocolor}{F} & \textcolor{gregoriocolor}{G} & \textcolor{gregoriocolor}{H} & \textcolor{gregoriocolor}{M} & \textcolor{gregoriocolor}{N} & \textcolor{gregoriocolor}{P} \\
 18 & 19 & 20 & 21 & 22 & 23 & 22 & 23 & 24 & 25 & 26 & 27 \\
\end{tabularx}

\begin{paracol}{2}
\selectlanguage{latin}
\lettrine[lines=2]{O}{ctava} Assumptiónis 
 beátæ Maríæ Vírginis.
\switchcolumn
\selectlanguage{english}
\lettrine[lines=2]{T}{he} Octave of the Assumption of the 
 Blessed Virgin Mary.
\switchcolumn*
\selectlanguage{latin}
Festum Immaculáti 
 Cordis ejúsdem beátæ Vírginis Maríæ.
\switchcolumn
\selectlanguage{english}
Feast of the Immaculate Heart of the 
 same Blessed Virgin Mary.
\switchcolumn*
\selectlanguage{latin}
Romæ, via Ostiénsi, 
 natális sancti Timóthei Mártyris, qui, a Præfécto Urbis Tarquínio tentus, et 
 longa cárceris custódia macerátus, et, cum sacrificáre idólis noluísset, 
 tértio cæsus et gravíssimis supplíciis attrectátus, ad últimum decollátus 
 est.
\switchcolumn
\selectlanguage{english}
At Rome, on the Ostian Way, the 
 birthday of the holy martyr Timothy. After he had been arrested by 
 Tarquin, prefect of the city, and kept for a long time in prison, because he 
 refused to sacrifice to idols, he was scourged three times, subjected to the 
 most severe torments, and finally beheaded.
\switchcolumn*
\selectlanguage{latin}
In Portu Románo sancti 
 Hippóloyti Epíscopi, eruditióne claríssimi, qui, sub Alexándro Imperatóre, ob 
 præcláram fídei confessiónem, mánibus pedibúsque ligátis in altam fóveam, 
 aquis plenam, præcipitátus, martyrii palmam accépit; cujus corpus apud 
 eúndem locum a Christiánis sepúltum fuit.
\switchcolumn
\selectlanguage{english}
At Porto, St. Hippolytus, bishop, 
 most renowned for learning. Having gloriously confessed the faith, in 
 the time of Emperor Alexander, he was bound hand and foot, thrown into a 
 deep ditch filled with water, and thus received the palm of martyrdom. 
 His body was buried by the Christians at that place.
\switchcolumn*
\selectlanguage{latin}
Augustodúni sancti 
 Symphoriáni Mártyris, qui, témpore Aureliáni Imperatóris, cum sacrificáre 
 nollet idólis, primo verbéribus afflíctus est, deínde cárceri mancipátus; 
 atque ad últimum, cæso cápite, martyrium consummávit.
\switchcolumn
\selectlanguage{english}
At Autun, St. Symphorian, a martyr, 
 in the time of Emperor Aurelian. Refusing to offer sacrifice to the 
 idols, he was first scourged, then confined to prison, and finally ended his 
 martyrdom by being beheaded.
\switchcolumn*
\selectlanguage{latin}
Tudérti, in Umbria, 
 natális sancti Philíppi Benítii, Confessóris, Florentíni, qui Ordinis 
 Servórum beátæ Maríæ Vírginis éxstitit propagátor et exímiæ humilitátis vir; 
 atque a Cleménte Décimo, Pontífice Máximo, Sanctórum número adscríptus est. 
 Ipsíus autem festívitas sequénti die celebrátur.
\switchcolumn
\selectlanguage{english}
At Todi in Umbria, the birthday of 
 St. Philip Beniti, confessor, of Florence. He was a zealous promoter 
 of the Order of the Servants of the Blessed Virgin Mary, and was a man of 
 great humility. He was canonized by Pope Clement X; his feast, 
 however, is observed on the day following
\switchcolumn*
\selectlanguage{latin}
Romæ sancti Antoníni 
 Mártyris, qui cum se Christiánum líbera voce faterétur, senténtia capitáli a 
 Júdice Vitéllio damnátus est, et a Rufíno Presbytero via Aurélia sepúltus.
\switchcolumn
\selectlanguage{english}
At Rome, St. Antoninus, martyr, who, 
 openly declaring himself a Christian, was condemned to capital punishment by 
 the judge Vitellius, and buried on the Aurelian Way.
\switchcolumn*
\selectlanguage{latin}
Tarsi, in Cilícia, commemorátio sanctórum Athanásii, Epíscopi et Mártyris, Anthúsæ, nóbilis 
 féminæ, quam ipse baptizáverat, ac simul Charísii et Neóphyti Mártyrum, 
 ejúsdem Anthúsæ servórum, qui sub Valeriáno Imperatóre passi sunt.
\switchcolumn
\selectlanguage{english}
At Tarsus in Cilicia, the 
 commemoration of Saints Athanasius, bishop and martyr, Anthusa, a noble 
 woman he had baptized, and two of her servants, Charisius and Neophytus, 
 martyrs who suffered under the Emperor Valerian.
\switchcolumn*
\selectlanguage{latin}
In Portu Románo sanctórum Mártyrum Martiális, Saturníni, Epictéti, Maprílis et Felícis, cum 
 Sóciis eórum.
\switchcolumn
\selectlanguage{english}
At Porto, the holy martyrs Martial, 
 Saturninus, Epictetus, Maprilis, and Felix, with their companions.
\switchcolumn*
\selectlanguage{latin}
Nicomedíæ pássio 
 sanctórum Agathoníci, Zótici et Sociórum Mártyrum, sub Maximiáno Imperatóre 
 et Eutólmio Præside.
\switchcolumn
\selectlanguage{english}
At Nicomedia, the passion of Saints 
 Agathonicus, Zoticus, and their fellow-martyrs, under Emperor Maximian and 
 the governor Eutholomius.
\switchcolumn*
\selectlanguage{latin}
Rhemis, in Gállia, 
 sanctórum Mártyrum Mauri et Sociórum.
\switchcolumn
\selectlanguage{english}
At Rheims in France, the holy 
 martyrs Maur and his companions.
\switchcolumn*
\selectlanguage{latin}
In Hispánia sanctórum 
 Mártyrum Fabriciáni et Filibérti.
\switchcolumn
\selectlanguage{english}
In Spain, the holy martyrs Fabrician 
 and Philibert.
\switchcolumn*
\selectlanguage{latin}
Papíæ sancti Gunifórti 
 Mártyris.
\switchcolumn
\selectlanguage{english}
At Pavia, St. Gunifort, martyr.
\switchcolumn*
\selectlanguage{latin}
\end{paracol}


% ---- martyrology/mart08/mart0823.htm
\needspace{10\baselineskip}
\begin{paracol}{2}
\selectlanguage{latin}
\begin{center}{\color{gregoriocolor} Décimo Kaléndas Septémbris. 
 Luna\dots\ }\end{center}
\switchcolumn
\selectlanguage{english}
\begin{center}{\color{gregoriocolor} The 
 Twenty-Third Day of 
 August. The\dots\ Day of the Moon.}\end{center}
\end{paracol}

\noindent\begin{tabularx}{\linewidth}{*{19}{>{\centering\arraybackslash}X}}
 \textcolor{gregoriocolor}{a} & \textcolor{gregoriocolor}{b} & \textcolor{gregoriocolor}{c} & \textcolor{gregoriocolor}{d} & \textcolor{gregoriocolor}{e} & \textcolor{gregoriocolor}{f} & \textcolor{gregoriocolor}{g} & \textcolor{gregoriocolor}{h} & \textcolor{gregoriocolor}{i} & \textcolor{gregoriocolor}{k} & \textcolor{gregoriocolor}{l} & \textcolor{gregoriocolor}{m} & \textcolor{gregoriocolor}{n} & \textcolor{gregoriocolor}{p} & \textcolor{gregoriocolor}{q} & \textcolor{gregoriocolor}{r} & \textcolor{gregoriocolor}{s} & \textcolor{gregoriocolor}{t} & \textcolor{gregoriocolor}{u} \\
 29 & 1 & 2 & 3 & 4 & 5 & 6 & 7 & 8 & 9 & 10 & 11 & 12 & 13 & 14 & 15 & 16 & 17 & 18 \\
\end{tabularx}
\vspace{0.5\baselineskip}
\noindent\begin{tabularx}{\linewidth}{*{12}{>{\centering\arraybackslash}X}}
 \textcolor{gregoriocolor}{A} & \textcolor{gregoriocolor}{B} & \textcolor{gregoriocolor}{C} & \textcolor{gregoriocolor}{D} & \textcolor{gregoriocolor}{E} & F & \textcolor{gregoriocolor}{F} & \textcolor{gregoriocolor}{G} & \textcolor{gregoriocolor}{H} & \textcolor{gregoriocolor}{M} & \textcolor{gregoriocolor}{N} & \textcolor{gregoriocolor}{P} \\
 19 & 20 & 21 & 22 & 23 & 24 & 23 & 24 & 25 & 26 & 27 & 28 \\
\end{tabularx}

\begin{paracol}{2}
\selectlanguage{latin}
\lettrine[lines=1]{V}{igília} sancti Bartholomǽi Apóstoli.
\switchcolumn
\selectlanguage{english}
\lettrine[lines=1]{T}{he} Vigil of St. Bartholomew, Apostle.
\switchcolumn*
\selectlanguage{latin}
Sancti Philíppi Benítii, 
 Confessóris, qui Ordinis Servórum beátæ Maríæ Vírginis éxstitit propagátor, 
 ac prídie hujus diéi migrávit ad Dóminum.
\switchcolumn
\selectlanguage{english}
St. Philip Beniti, confessor, 
 promoter of the Order of the Servants of the Blessed Virgin Mary, who 
 departed to the Lord on the previous day.
\switchcolumn*
\selectlanguage{latin}
Apud Ostia Tiberína 
 sanctórum Mártyrum Quiríaci Epíscopi, Máximi Presbyteri, Archelái Diáconi, 
 et Sociórum, qui sub Ulpiáno Præfécto, témpore Alexándri, passi sunt.
\switchcolumn
\selectlanguage{english}
At Ostia, the holy martyrs Quiriacus, 
 bishop, Maximus, priest, Archelaus, deacon, and their companions, who 
 suffered under prefect Ulpian, in the time of Alexander.
\switchcolumn*
\selectlanguage{latin}
Antiochíæ natális 
 sanctórum Mártyrum Restitúti, Donáti, Valeriáni et Fructuósæ, cum áliis 
 duódecim; qui præclaríssimo confessiónis honóre coronáti sunt.
\switchcolumn
\selectlanguage{english}
At Antioch, the birthday of the holy 
 martyrs Restitutus, Donatus, Valerian, and Fructuosa, with twelve others, 
 who were crowned after having distinguished themselves by a glorious 
 confession.
\switchcolumn*
\selectlanguage{latin}
Ægǽæ, in Cilícia, sanctórum Mártyrum fratrum Cláudii, Astérii et Neónis, qui, Christiánæ 
 religiónis a novérca accusáti, sub Diocletiáno Imperatóre et Lysia Præside, 
 post acérba torménta cruci sunt affíxi, in qua victóres cum Christo 
 triumphárunt. Passæ sunt post eos Donvína et Theonílla.
\switchcolumn
\selectlanguage{english}
At Aegaea in Cilicia, the holy 
 martyrs Claudius, Asterius, and Neon, brothers, who were accused of being 
 Christians by their stepmother, under Emperor Diocletian and the governor 
 Lysias. After enduring bitter torments, they were fastened to a cross, 
 and thus conquered and triumphed with Christ. After them suffered 
 Dovina and Theonilla.
\switchcolumn*
\selectlanguage{latin}
Rhemis, in Gállia, 
 natális sanctórum Timóthei et Apollináris, qui, ibídem consummáto martyrio, 
 cæléstia regna meruérunt.
\switchcolumn
\selectlanguage{english}
At Rheims in France, the birthday of 
 the Saints Timothy and Apollinaris, who merited to enter the heavenly 
 kingdom by completing their martyrdom in that city.
\switchcolumn*
\selectlanguage{latin}
Lugdúni, in Gállia, 
 sanctórum Mártyrum Minérvi, et Eleazári cum fíliis octo.
\switchcolumn
\selectlanguage{english}
At Lyons, the holy martyrs Minercus 
 and Eleazar, with his eight sons.
\switchcolumn*
\selectlanguage{latin}
Item sancti Luppi 
 Mártyris, qui, ex sérvili conditióne, Christi libertáte donátus, martyrii 
 quoque coróna dignátus est.
\switchcolumn
\selectlanguage{english}
Also St. Luppus, martyr, who, though 
 a slave, enjoyed the liberty of Christ, and was likewise deemed worthy of 
 the crown of martyrdom.
\switchcolumn*
\selectlanguage{latin}
Hierosólymis sancti 
 Zachæi Epíscopi, qui, quartus a beáto Jacóbo Apóstolo, Hierosolymitánum 
 Ecclésiam rexit.
\switchcolumn
\selectlanguage{english}
At Jerusalem, St. Zachaeus, bishop, 
 who governed the Church in that city the fourth after the blessed apostle 
 James.
\switchcolumn*
\selectlanguage{latin}
Alexandríæ sancti 
 Theónæ, Epíscopi et Confessóris.
\switchcolumn
\selectlanguage{english}
At Alexandria, St. Theonas, bishop 
 and confessor.
\switchcolumn*
\selectlanguage{latin}
Uticæ, in Africa, beáti 
 Victóris Epíscopi.
\switchcolumn
\selectlanguage{english}
At Utica in Africa, blessed Victor, 
 bishop.
\switchcolumn*
\selectlanguage{latin}
Augustodúni sancti 
 Flaviáni Epíscopi.
\switchcolumn
\selectlanguage{english}
At Autun, St. Flavian, bishop.
\switchcolumn*
\selectlanguage{latin}
\end{paracol}


% ---- martyrology/mart08/mart0824.htm
\needspace{10\baselineskip}
\begin{paracol}{2}
\selectlanguage{latin}
\begin{center}{\color{gregoriocolor} Nono Kaléndas Septémbris. 
 Luna\dots\ }\end{center}
\switchcolumn
\selectlanguage{english}
\begin{center}{\color{gregoriocolor} The 
 Twenty-Fourth Day of 
 August. The\dots\ Day of the Moon.}\end{center}
\end{paracol}

\noindent\begin{tabularx}{\linewidth}{*{19}{>{\centering\arraybackslash}X}}
 \textcolor{gregoriocolor}{a} & \textcolor{gregoriocolor}{b} & \textcolor{gregoriocolor}{c} & \textcolor{gregoriocolor}{d} & \textcolor{gregoriocolor}{e} & \textcolor{gregoriocolor}{f} & \textcolor{gregoriocolor}{g} & \textcolor{gregoriocolor}{h} & \textcolor{gregoriocolor}{i} & \textcolor{gregoriocolor}{k} & \textcolor{gregoriocolor}{l} & \textcolor{gregoriocolor}{m} & \textcolor{gregoriocolor}{n} & \textcolor{gregoriocolor}{p} & \textcolor{gregoriocolor}{q} & \textcolor{gregoriocolor}{r} & \textcolor{gregoriocolor}{s} & \textcolor{gregoriocolor}{t} & \textcolor{gregoriocolor}{u} \\
 1 & 2 & 3 & 4 & 5 & 6 & 7 & 8 & 9 & 10 & 11 & 12 & 13 & 14 & 15 & 16 & 17 & 18 & 19 \\
\end{tabularx}
\vspace{0.5\baselineskip}
\noindent\begin{tabularx}{\linewidth}{*{12}{>{\centering\arraybackslash}X}}
 \textcolor{gregoriocolor}{A} & \textcolor{gregoriocolor}{B} & \textcolor{gregoriocolor}{C} & \textcolor{gregoriocolor}{D} & \textcolor{gregoriocolor}{E} & F & \textcolor{gregoriocolor}{F} & \textcolor{gregoriocolor}{G} & \textcolor{gregoriocolor}{H} & \textcolor{gregoriocolor}{M} & \textcolor{gregoriocolor}{N} & \textcolor{gregoriocolor}{P} \\
 20 & 21 & 22 & 23 & 24 & 25 & 24 & 25 & 26 & 27 & 28 & 29 \\
\end{tabularx}

\begin{paracol}{2}
\selectlanguage{latin}
\lettrine[lines=2]{S}{ancti} Bartholomǽi 
 Apóstoli, qui Christi Evangélium in India prædicávit; inde in majórem 
 Arméniam proféctus, ibi, cum plúrimos ad fidem convertísset, vivus a 
 bárbaris decoriátus est, atque, Astyagis Regis jussu, cápitis decollatióne 
 martyrium complévit. Ipsíus sacrum corpus, primo ad Líparam ínsulam, 
 deínde Benevéntum, postrémo Romam ad Tiberínam translátum ínsulam, ibi pia 
 fidélium veneratióne honorátur.
\switchcolumn
\selectlanguage{english}
\lettrine[lines=2]{T}{he} apostle St. Bartholomew, who 
 preached the Gospel of Christ in India. Passing thence into the 
 Greater Armenia where, after converting many to the faith, he was flayed 
 alive by the barbarians, and having his head cut off by order of King 
 Astyages, he fulfilled his martyrdom. His holy body was first carried 
 to the island of Lipara, then to Benevento, and finally to Rome in the 
 Island of the Tiber, where it is venerated by the pious faithful.
\switchcolumn*
\selectlanguage{latin}
Limæ, in Perúvia, 
 natális sanctæ Rosæ a sancta María, Vírginis, e tértio Ordine sancti 
 Domínici. Ejus vero festívitas tértio Kaléndas Septémbris celebrátur.
\switchcolumn
\selectlanguage{english}
At Lima in Peru, the birthday of St. 
 Rose of St. Mary, virgin of the Third Order of St. Dominic. Her feast 
 is observed on the 30th of August.
\switchcolumn*
\selectlanguage{latin}
Népete sancti Ptolomǽi 
 Epíscopi, qui fuit discípulus beáti Petri Apóstoli; atque, ab eo missus in 
 Túsciam ad prædicándum Evangélium, in eádem civitáte gloriósus Christi 
 Martyr occúbuit.
\switchcolumn
\selectlanguage{english}
At Nepi, St. Ptolemy, bishop, 
 disciple of the blessed apostle Peter. Being sent by him to preach the 
 Gospel in Tuscany, he died a glorious martyr of Christ in the city of Nepi.
\switchcolumn*
\selectlanguage{latin}
Eódem die sancti 
 Eutychii, qui fuit discípulus beáti Joánnis Evangelístæ; atque, ob Evangélii 
 prædicatiónem in multis regiónibus cárceres, vérbera et ignes perpéssus, in 
 pace tandem quiévit.
\switchcolumn
\selectlanguage{english}
Also, St. Eutychius, disciple of the 
 blessed evangelist John. He preached the Gospel in many countries, and 
 was subjected to imprisonment, to stripes and fire, but finally he rested in 
 peace.
\switchcolumn*
\selectlanguage{latin}
Népete sancti Románi, 
 ejúsdem civitátis Epíscopi, qui, cum esset sancti Ptolomǽi discípulus, fuit 
 étiam in passióne sócius.
\switchcolumn
\selectlanguage{english}
Also at Nepi, St. Romanus, bishop of 
 that city, who was the disciple of St. Ptolemy, and his companion in 
 martyrdom.
\switchcolumn*
\selectlanguage{latin}
Carthágine sanctórum 
 trecentórum Mártyrum, témpore Valeriáni et Galliéni. Hi Mártyres 
 magnánimi, inter ália supplícia, cum Præses fornácem calcáriam accéndi 
 jussísset, et, in præséntia ejus, prunas cum thure exhibéri, atque illis 
 dixísset: « Elígite e duóbus unum, aut thura super his carbónibus offérte 
 Jovi, aut in calcem demergímini », fide armáti, Christum Dei Fílium 
 confiténtes, ictu rapidíssimo se injecérunt in ignem, et inter calcis 
 vapóres in púlverem sunt redácti; ex quo candidátus ille beatórum exércitus 
 appellári Massa cándida méruit.
\switchcolumn
\selectlanguage{english}
At Carthage, three hundred holy 
 martyrs, in the time of Valerian and Gallienus. Among other torments 
 inflicted on them, a pit filled with burning lime was prepared by order of 
 the governor, who, live coals with incense being brought to him, said to the 
 confessors: ``Choose one of these two things: either offer incense to Jupiter 
 upon these coals, or be thrown into the lime.'' Armed with faith, and 
 confessing Christ to be the Son of God, they quickly threw themselves into 
 the pit, and amid the vapours of the lime were reduced to dust. From 
 this circumstance, this white-robed company of the blessed earned for itself 
 the name of the White Mass.
\switchcolumn*
\selectlanguage{latin}
In Isáuria sancti 
 Tatiónis Mártyris, qui, in persecutióne Diocletiáni, sub Urbáno Præside, 
 gládio cæsus, martyrii corónam accépit.
\switchcolumn
\selectlanguage{english}
In Isauria, St. Tation, martyr, who 
 received the crown of martyrdom by being beheaded in the persecution of 
 Diocletian, under the governor Urbanus.
\switchcolumn*
\selectlanguage{latin}
Item sancti Geórgii 
 Limniótæ Mónachi, qui, cum ímpium Leónem Imperatórem, quod sacras Imágines, 
 frángeret Sanctorúmque relíquias combúreret, reprehendísset, hanc ob causam, 
 ejus jussu mánibus abscíssis et cápite incénso, Martyr migrávit ad Dóminum.
\switchcolumn
\selectlanguage{english}
Also, St. George Limniota, monk. 
 Because he reprehended the wicked emperor Leo for breaking holy images, and 
 burning the relics of the saints, he had his hands cut off and his head 
 burned by order of the tyrant, and went to our Lord to receive the 
 recompence of a martyr.
\switchcolumn*
\selectlanguage{latin}
Apud Ostia Tiberína 
 sanctæ Aureæ, Vírginis et Mártyris; quæ saxo ad collum ligáto, in mare 
 demérsa est. Ipsíus autem corpus, ejéctum ad littus, beátus Nonnus 
 sepelívit.
\switchcolumn
\selectlanguage{english}
At Ostia, on the Tiber, St. Aurea, 
 virgin and martyr, who was plunged into the sea with a stone tied to her 
 neck. Her body being driven to the shore was buried by blessed Nonnus.
\switchcolumn*
\selectlanguage{latin}
Rotómagi sancti Audoéni, 
 Epíscopi et Confessóris.
\switchcolumn
\selectlanguage{english}
At Rouen, St. Owen, bishop and 
 confessor.
\switchcolumn*
\selectlanguage{latin}
Nivérnis, in Gállia, 
 sancti Patrícii Abbátis.
\switchcolumn
\selectlanguage{english}
At Nevers in France, St. Patrick, 
 abbot.
\switchcolumn*
\selectlanguage{latin}
Neápoli in Campánia, sanctæ Joánnæ Antidæ Thouret, Vírginis, Institúti Sorórum a Caritáte 
 Fundatrícis, quam Pius Papa Undécimus in album sanctárum Vírginum rétulit.
\switchcolumn
\selectlanguage{english}
At Naples in Campania, St. Joan 
 Antide Thouret, virgin, who founded the Daughters of Saint Vincent de Paul, 
 and whom Pope Pius XI added to the catalogue of holy virgins.
\switchcolumn*
\selectlanguage{latin}
Massíliæ, in Gállia, 
 sanctæ Æmíliæ de Vialár, Vírginis, Fundatrícis Institúti Sorórum a sancto 
 Joseph ab Apparitióne, fortitúdine, patiéntia et caritáte insígnis, quam 
 Pius Duodécimus, Póntifex Máximus, in Sanctárum númerum rétulit.
\switchcolumn
\selectlanguage{english}
At Marseilles in France, St. Emily 
 de Vialar, virgin, foundress of the Congregation of the Sisters of Saint 
 Joseph of the Apparition. A shining example of fortitude, patience and 
 charity, the Sovereign Pontiff Pius XII added her to the number of the 
 saints.
\switchcolumn*
\selectlanguage{latin}
Valéntiæ, in Hispánia, 
 natális sanctæ Maríæ Michaélæ, Vírginis, Fundatrícis Congregatiónis 
 Ancillárum a Sanctíssimo Sacraménto et Caritátis, patiéndi stúdio ac desidério ánimas Deo lucrándi inflammátæ, quam Pius Papa Undécimus sanctis 
 Virgínibus accénsuit.
\switchcolumn
\selectlanguage{english}
At Valencia in Spain, the birthday 
 of St. Mary Micaela, virgin, who founded the Institute of Religious 
 Adorer-Slaves of the Blessed Sacrament and of Charity. Burning with 
 the desire to suffer and draw souls to God, she was numbered among the holy 
 virgins by Pope Pius XI.
\switchcolumn*
\selectlanguage{latin}
\end{paracol}


% ---- martyrology/mart08/mart0825.htm
\needspace{10\baselineskip}
\begin{paracol}{2}
\selectlanguage{latin}
\begin{center}{\color{gregoriocolor} Octávo Kaléndas Septémbris. 
 Luna\dots\ }\end{center}
\switchcolumn
\selectlanguage{english}
\begin{center}{\color{gregoriocolor} The 
 Twenty-Fifth Day of 
 August. The\dots\ Day of the Moon.}\end{center}
\end{paracol}

\noindent\begin{tabularx}{\linewidth}{*{19}{>{\centering\arraybackslash}X}}
 \textcolor{gregoriocolor}{a} & \textcolor{gregoriocolor}{b} & \textcolor{gregoriocolor}{c} & \textcolor{gregoriocolor}{d} & \textcolor{gregoriocolor}{e} & \textcolor{gregoriocolor}{f} & \textcolor{gregoriocolor}{g} & \textcolor{gregoriocolor}{h} & \textcolor{gregoriocolor}{i} & \textcolor{gregoriocolor}{k} & \textcolor{gregoriocolor}{l} & \textcolor{gregoriocolor}{m} & \textcolor{gregoriocolor}{n} & \textcolor{gregoriocolor}{p} & \textcolor{gregoriocolor}{q} & \textcolor{gregoriocolor}{r} & \textcolor{gregoriocolor}{s} & \textcolor{gregoriocolor}{t} & \textcolor{gregoriocolor}{u} \\
 2 & 3 & 4 & 5 & 6 & 7 & 8 & 9 & 10 & 11 & 12 & 13 & 14 & 15 & 16 & 17 & 18 & 19 & 20 \\
\end{tabularx}
\vspace{0.5\baselineskip}
\noindent\begin{tabularx}{\linewidth}{*{12}{>{\centering\arraybackslash}X}}
 \textcolor{gregoriocolor}{A} & \textcolor{gregoriocolor}{B} & \textcolor{gregoriocolor}{C} & \textcolor{gregoriocolor}{D} & \textcolor{gregoriocolor}{E} & F & \textcolor{gregoriocolor}{F} & \textcolor{gregoriocolor}{G} & \textcolor{gregoriocolor}{H} & \textcolor{gregoriocolor}{M} & \textcolor{gregoriocolor}{N} & \textcolor{gregoriocolor}{P} \\
 21 & 22 & 23 & 24 & 25 & 26 & 25 & 26 & 27 & 28 & 29 & 1 \\
\end{tabularx}

\begin{paracol}{2}
\selectlanguage{latin}
\lettrine[lines=2]{A}{pud} Cartháginem sancti 
 Ludovíci Noni, Regis Francórum et Confessóris, vitæ sanctitáte ac 
 miraculórum glória præclári; cujus ossa póstmodum Lutétiam Parisiórum sunt 
 reláta.
\switchcolumn
\selectlanguage{english}
\lettrine[lines=2]{A}{t} Carthage, St. Louis IX, king of 
 France and confessor, illustrious for holiness of life and glorious 
 miracles. His bones were later translated to Paris.
\switchcolumn*
\selectlanguage{latin}
Romæ natális sancti 
 Joséphi Calasánctii, Presbyteri et Confessóris, vitæ innocéntia et miráculis 
 illústris; qui, ad erudiéndam pietáte ac lítteris juventútem, Ordinem 
 Clericórum Regulárium Páuperum Matris Dei Scholárum Piárum fundávit. 
 Eum Pius Duodécimus, Póntifex Máximus, ómnium Scholárum populárium 
 christianárum ubíque exsisténtium cæléstem apud Deum Patrónum constítuit. 
 Ipsíus tamen festívitas sexto Kaléndas Septémbris recólitur.
\switchcolumn
\selectlanguage{english}
At Rome, the birthday of St. Joseph 
 Calasanctius, priest and confessor, noteworthy for his holy life and 
 miracles. He founded the Order of Poor Clerics Regular of the Mother 
 of God of the Christian Schools. The Sovereign Pontiff, Pius XII, 
 named him as heavenly patron of all Christian schoolchildren. His 
 feast is on the 27th of August.
\switchcolumn*
\selectlanguage{latin}
Item Romæ sanctórum 
 Mártyrum Eusébii, Pontiáni, Vincéntii et Peregríni; qui, sub Cómmodo 
 Imperatóre, primum in equúleo leváti, nervis quoque disténti, ac deínde 
 fústibus cæsi sunt, flammis circa eórum látera appósitis; et, cum in laude 
 Christi fidelíssime permanérent, plumbátis usque ad emissiónem spíritus sunt 
 mactáti.
\switchcolumn
\selectlanguage{english}
Also at Rome, in the time of Emperor 
 Commodus, the holy martyrs Eusebius, Pontian, Vincent, and Peregrinus, who 
 were first racked, distended by ropes, then beaten with rods and burned 
 about their sides. As they continued faithfully to praise Christ, they 
 were scourged with leaded whips until they expired.
\switchcolumn*
\selectlanguage{latin}
Romæ prætérea natális 
 beáti Nemésii Diáconi, et fíliæ Lucillæ Vírginis; qui, cum de fide Christi 
 flecti nequáquam possent, decolláti sunt, jubénte Valeriáno Imperatóre. 
 Ipsórum córpora, a beáto Stéphano Papa sepúlta, deínde a beáto Xysto Secúndo 
 via Appia, prídie Kaléndas Novémbris, honéstius tumuláta, Gregórius Quintus 
 in Diacóniam sanctæ Maríæ Novæ tránstulit, una cum sanctis Symphrónio, 
 Olympio Tribúno, hujúsque uxóre Exsupéria et Theodúlo fílio; qui omnes, 
 Symphrónii ópera convérsi et ab eódem sancto Stéphano baptizáti, martyrio 
 coronáti fúerant. Eadem Sanctórum córpora, Gregório Décimo tértio 
 Summo Pontífice, ibídem invénta, sub altári ejúsdem Ecclésiæ honorificéntius 
 collocáta sunt sexto Idus Decémbris.
\switchcolumn
\selectlanguage{english}
In the same city of Rome, the 
 birthday of blessed Nemesius, deacon, and his daughter, the virgin Lucilla. 
 As they could not be prevailed upon to abandon the faith of Christ, they 
 were beheaded by order of Emperor Valerian. Their bodies were buried 
 by blessed Pope Stephen, and afterwards more decently entombed on the 31st 
 of October, by blessed Sixtus on the Appian Way. Gregory V translated 
 them into the sacristy of Santa Maria Nova, together with the Saints 
 Symphronius, Olympius, a tribune, Exuperia, his wife, and Theodulus, his 
 son, who, being all converted by the exertions of Symphonius, and baptized 
 by the same St. Stephen, had been crowned with martyrdom. These holy 
 bodies were found there during the pontificate of Gregory XIII, and placed 
 more honourably beneath the altar of the same church, on the 8th of 
 December.
\switchcolumn*
\selectlanguage{latin}
Item Romæ sancti 
 Genésii Mártyris, qui, primum sub Gentilitáte mimus, cum in theátro, 
 spectánte Diocletiáno Imperatóre, Mystériis Christianórum illúderet, repénte, 
 inspirátus a Deo, convérsus est ad fidem et baptizátus. Mox, 
 Imperatóris jussu, fústibus crudelíssime cæsus, deínde suspénsus in equúleo, 
 et ungulárum diutíssima laceratióne vexátus, lampádibus étiam adústus est; 
 ac tandem, cum in fide Christi persísteret, dicens: « Non est Rex præter 
 Christum, pro quo si míllies occídar, ipsum mihi de ore, ipsum mihi de corde 
 auférre non potéritis », martyrii palmam obtruncatióne cápitis proméruit.
\switchcolumn
\selectlanguage{english}
Also at Rome, St. Genesius, martyr, 
 who had embraced the profession of actor while he was a pagan. One day 
 he was deriding the Christian mysteries in the theatre in the presence of 
 Emperor Diocletian; but by the inspiration of God he was suddenly converted 
 to the faith and baptized. By command of the emperor he was forthwith 
 most cruelly beaten with rods, then racked, and a long time lacerated with 
 iron hooks, and burned with torches. As he remained firm in the faith 
 of Christ, even saying: ``There is no king besides Christ. Should you 
 kill me a thousand times, you shall not be able to take him from my lips or 
 my heart.'' He was then beheaded, and thus merited the palm of 
 martyrdom.
\switchcolumn*
\selectlanguage{latin}
Areláte, in Gállia, 
 beáti item Genésii, qui, cum ímpia edícta, quibus Christiáni puníri 
 jubebántur, exceptóris offício fungens, nollet excípere, et, projéctis in 
 públicum tábulis, se Christiánum esse testarétur, comprehénsus et decollátus 
 est, atque ita martyrii glóriam, próprio cruóre baptizátus, accépit.
\switchcolumn
\selectlanguage{english}
At Arles in France, another blessed 
 Genesius, who, filling the office of notary, and refusing to record the 
 impious edicts by which Christians were commanded to be punished, threw away 
 his books publicly, and declared himself a Christian. He was seized 
 and beheaded, and thus attained the glory of martyrdom through baptism in 
 his own blood.
\switchcolumn*
\selectlanguage{latin}
In Syria sancti Juliáni 
 Mártyris.
\switchcolumn
\selectlanguage{english}
In Syria, St. Julian, martyr.
\switchcolumn*
\selectlanguage{latin}
Tarracóne, in Hispánia, 
 sancti Magíni Mártyris.
\switchcolumn
\selectlanguage{english}
At Tarragona in Spain, St. Maginus, 
 martyr.
\switchcolumn*
\selectlanguage{latin}
Itálicæ, in Hispánia, 
 sancti Gerúntii Epíscopi, qui, Apostolórum témpore, Evangélium in ea 
 província prædicávit, et, post multos labóres, in cárcere quiévit.
\switchcolumn
\selectlanguage{english}
At Italica in Spain, St. Gerontius, 
 bishop, who preached the Gospel in that country in apostolic times, and 
 after many labours died in prison.
\switchcolumn*
\selectlanguage{latin}
Constantinópoli sancti 
 Mennæ Epíscopi.
\switchcolumn
\selectlanguage{english}
At Constantinople, St. Mennas, 
 bishop.
\switchcolumn*
\selectlanguage{latin}
Trajécti sancti 
 Gregórii Epíscopi.
\switchcolumn
\selectlanguage{english}
At Utrecht, St. Gregory, bishop.
\switchcolumn*
\selectlanguage{latin}
Apud Montem Falíscum, 
 in Etrúria, sancti Thomæ, Confessóris, qui Herfordiénsis Ecclésiæ, in 
 Anglia, Epíscopus éxstitit.
\switchcolumn
\selectlanguage{english}
At Monte Falisco in Etruria, St. 
 Thomas, bishop of the church of Hereford in England, and confessor.
\switchcolumn*
\selectlanguage{latin}
Neápoli, in Campánia, 
 sanctæ Patríciæ Vírginis.
\switchcolumn
\selectlanguage{english}
At Naples in Campania, St. Patricia, 
 virgin.
\switchcolumn*
\selectlanguage{latin}
\end{paracol}


% ---- martyrology/mart08/mart0826.htm
\needspace{10\baselineskip}
\begin{paracol}{2}
\selectlanguage{latin}
\begin{center}{\color{gregoriocolor} Séptimo Kaléndas Septémbris. 
 Luna\dots\ }\end{center}
\switchcolumn
\selectlanguage{english}
\begin{center}{\color{gregoriocolor} The 
 Twenty-Sixth Day of 
 August. The\dots\ Day of the Moon.}\end{center}
\end{paracol}

\noindent\begin{tabularx}{\linewidth}{*{19}{>{\centering\arraybackslash}X}}
 \textcolor{gregoriocolor}{a} & \textcolor{gregoriocolor}{b} & \textcolor{gregoriocolor}{c} & \textcolor{gregoriocolor}{d} & \textcolor{gregoriocolor}{e} & \textcolor{gregoriocolor}{f} & \textcolor{gregoriocolor}{g} & \textcolor{gregoriocolor}{h} & \textcolor{gregoriocolor}{i} & \textcolor{gregoriocolor}{k} & \textcolor{gregoriocolor}{l} & \textcolor{gregoriocolor}{m} & \textcolor{gregoriocolor}{n} & \textcolor{gregoriocolor}{p} & \textcolor{gregoriocolor}{q} & \textcolor{gregoriocolor}{r} & \textcolor{gregoriocolor}{s} & \textcolor{gregoriocolor}{t} & \textcolor{gregoriocolor}{u} \\
 3 & 4 & 5 & 6 & 7 & 8 & 9 & 10 & 11 & 12 & 13 & 14 & 15 & 16 & 17 & 18 & 19 & 20 & 21 \\
\end{tabularx}
\vspace{0.5\baselineskip}
\noindent\begin{tabularx}{\linewidth}{*{12}{>{\centering\arraybackslash}X}}
 \textcolor{gregoriocolor}{A} & \textcolor{gregoriocolor}{B} & \textcolor{gregoriocolor}{C} & \textcolor{gregoriocolor}{D} & \textcolor{gregoriocolor}{E} & F & \textcolor{gregoriocolor}{F} & \textcolor{gregoriocolor}{G} & \textcolor{gregoriocolor}{H} & \textcolor{gregoriocolor}{M} & \textcolor{gregoriocolor}{N} & \textcolor{gregoriocolor}{P} \\
 22 & 23 & 24 & 25 & 26 & 27 & 26 & 27 & 28 & 29 & 1 & 2 \\
\end{tabularx}

\begin{paracol}{2}
\selectlanguage{latin}
\lettrine[lines=2]{S}{ancti} Zephyríni, Papæ 
 et Mártyris; cujus dies natális tertiodécimo Kaléndas Januárii recensétur.
\switchcolumn
\selectlanguage{english}
\lettrine[lines=2]{A}{t} Rome, St. Zephyrinus, pope and 
 martyr, whose birthday falls on the 20th of December.
\switchcolumn*
\selectlanguage{latin}
Cardónæ, in Hispánia, 
 tránsitus sancti Raymúndi Nonnáti, Cardinális et Confessóris, ex Ordine 
 beátæ Maríæ de Mercéde redemptiónis captivórum, vitæ sanctitáte et miráculis 
 clari. Ipsíus tamen festum recólitur prídie Kaléndas Septémbris.
\switchcolumn
\selectlanguage{english}
At Cardona in Spain, the birthday of 
 St. Raymund Nonnatus, cardinal and confessor, of the Order of our Lady of 
 Ransom for the Redemption of Captives, renowned for holiness of life and for 
 miracles, whose feast is observed on the 31st of August.
\switchcolumn*
\selectlanguage{latin}
Romæ sanctórum Mártyrum 
 Irenǽi et Abúndii, qui, in persecutióne Valeriáni, eo quod corpus beátæ 
 Concórdiæ, in cloácam projéctum, leváverant, in eándem cloácam demérsi 
 fuérunt; quorum córpora, a Justíno Presbytero inde extrácta, in crypta, 
 juxta beátum Lauréntium, sepúlta sunt.
\switchcolumn
\selectlanguage{english}
At Rome, during the persecution of 
 Valerian, the holy martyrs Irenaeus and Abundius, who were thrown into a 
 sewer from which they had taken the body of blessed of Concordia. 
 Their bodies were drawn out by the priest Justin, and buried in a crypt near 
 St. Lawrence.
\switchcolumn*
\selectlanguage{latin}
Apud Albintimélium, 
 Ligúriæ civitátem, sancti Secúndi Mártyris, viri spectábilis et Ducis ex 
 legióne Thebæórum.
\switchcolumn
\selectlanguage{english}
At Ventimiglia, a city of Liguria, 
 St. Secundus, martyr, a distinguished man and officer in the Theban Legion.
\switchcolumn*
\selectlanguage{latin}
Bérgomi sancti 
 Alexándri Mártyris, qui, et ipse unus ex eádem legióne, cum nomen Dómini 
 Jesu Christi constantíssime faterétur, cápitis abscissióne martyrium 
 complévit.
\switchcolumn
\selectlanguage{english}
At Bergamo in Lombardy, St. 
 Alexander, martyr, who was one of the same legion, and endured martyrdom, 
 being beheaded for the constant confession of the name of our Lord Jesus 
 Christ.
\switchcolumn*
\selectlanguage{latin}
Apud Marsos sanctórum 
 Simplícii, et ejus filiórum Constántii et Victoriáni, qui, sub Antoníno 
 Imperatóre, várie primum excruciáti, tum demum, secúris ictu percússi, 
 martyrii corónam, adépti sunt.
\switchcolumn
\selectlanguage{english}
Among the Marcians, the saints 
 Simplicius, and his sons Constantius and Victorian, who were first tortured 
 in different manners, and lastly, struck with the axe, obtained the crown of 
 martyrdom, in the time of Emperor Antoninus.
\switchcolumn*
\selectlanguage{latin}
Nicomedíæ pássio sancti 
 Hadriáni, e Probo Cæsare progéniti, qui, Licínio persecutiónem in 
 Christiános commótam éxprobrans, ab eódem jussus est occídi. Ipsíus 
 corpus Domítius, Byzántii Epíscopus, ejus pátruus, in ipsíus civitátis 
 subúrbio cui nomen Argyrópolis sepelívit.
\switchcolumn
\selectlanguage{english}
At Nicomedia, the martyrdom of St. 
 Adrian, son of Emperor Probus. For reproaching Licinius because of the 
 persecution of Christians, he was put to death by his order. His body 
 was buried at Argyropolis by his uncle Domitius, bishop of Byzantium.
\switchcolumn*
\selectlanguage{latin}
In Hispánia sancti 
 Victoris Mártyris, qui pro Christi fide, a Mauris occísus, martyrii coróna 
 donátus est.
\switchcolumn
\selectlanguage{english}
In Spain, St. Victor, martyr, who 
 merited the crown of martyrs by being slain by the Moors for the faith of 
 Christ.
\switchcolumn*
\selectlanguage{latin}
Cápuæ sancti Rufíni, 
 Epíscopi et Confessóris.
\switchcolumn
\selectlanguage{english}
At Capua, St. Rufinus, bishop and 
 confessor.
\switchcolumn*
\selectlanguage{latin}
Pistórii, in Túscia, 
 sancti Felícis, Presbyteri et Confessóris.
\switchcolumn
\selectlanguage{english}
At Pistoia, St. Felix, priest and 
 confessor.
\switchcolumn*
\selectlanguage{latin}
Pódii, in diœcési 
 Pictaviénsi, sanctæ Joánnæ-Elisabeth Bichier des Ages, Vírginis, 
 Congregatiónis Filiárum a Cruce una cum sancto Andréa Hubérto Fournet 
 Fundatrícis, jugi mortificatióne et vitæ innocéntia claræ, quam Pius Papa 
 Duodécimus sanctárum Vírginum fastis accénsuit.
\switchcolumn
\selectlanguage{english}
In the diocese of Poitiers, St. 
 Joan-Elizabeth Bichier des Ages, virgin, who with St. André Hubert Fournet 
 co-founded the Congregation of the Daughters of the Cross, and who was 
 renowned for her spirit of mortification and life of innocence. Pope 
 Pius XII added her name to the list of holy virgins.
\switchcolumn*
\selectlanguage{latin}
\end{paracol}


% ---- martyrology/mart08/mart0827.htm
\needspace{10\baselineskip}
\begin{paracol}{2}
\selectlanguage{latin}
\begin{center}{\color{gregoriocolor} Sexto Kaléndas Septémbris. 
 Luna\dots\ }\end{center}
\switchcolumn
\selectlanguage{english}
\begin{center}{\color{gregoriocolor} The 
 Twenty-Seventh Day of 
 August. The\dots\ Day of the Moon.}\end{center}
\end{paracol}

\noindent\begin{tabularx}{\linewidth}{*{19}{>{\centering\arraybackslash}X}}
 \textcolor{gregoriocolor}{a} & \textcolor{gregoriocolor}{b} & \textcolor{gregoriocolor}{c} & \textcolor{gregoriocolor}{d} & \textcolor{gregoriocolor}{e} & \textcolor{gregoriocolor}{f} & \textcolor{gregoriocolor}{g} & \textcolor{gregoriocolor}{h} & \textcolor{gregoriocolor}{i} & \textcolor{gregoriocolor}{k} & \textcolor{gregoriocolor}{l} & \textcolor{gregoriocolor}{m} & \textcolor{gregoriocolor}{n} & \textcolor{gregoriocolor}{p} & \textcolor{gregoriocolor}{q} & \textcolor{gregoriocolor}{r} & \textcolor{gregoriocolor}{s} & \textcolor{gregoriocolor}{t} & \textcolor{gregoriocolor}{u} \\
 4 & 5 & 6 & 7 & 8 & 9 & 10 & 11 & 12 & 13 & 14 & 15 & 16 & 17 & 18 & 19 & 20 & 21 & 22 \\
\end{tabularx}
\vspace{0.5\baselineskip}
\noindent\begin{tabularx}{\linewidth}{*{12}{>{\centering\arraybackslash}X}}
 \textcolor{gregoriocolor}{A} & \textcolor{gregoriocolor}{B} & \textcolor{gregoriocolor}{C} & \textcolor{gregoriocolor}{D} & \textcolor{gregoriocolor}{E} & F & \textcolor{gregoriocolor}{F} & \textcolor{gregoriocolor}{G} & \textcolor{gregoriocolor}{H} & \textcolor{gregoriocolor}{M} & \textcolor{gregoriocolor}{N} & \textcolor{gregoriocolor}{P} \\
 23 & 24 & 25 & 26 & 27 & 28 & 27 & 28 & 29 & 1 & 2 & 3 \\
\end{tabularx}

\begin{paracol}{2}
\selectlanguage{latin}
\lettrine[lines=2]{S}{ancti} Joséphi 
 Calasánctii, Presbyteri et Confessóris, qui Ordinis Clericórum Regulárium 
 Páuperum Matris Dei Scholárum Piárum éxstitit Fundátor, atque octávo 
 Kaléndas Septémbris obdormívit in Dómino.
\switchcolumn
\selectlanguage{english}
\lettrine[lines=2]{S}{t.} Joseph Calasanctius, priest and 
 confessor, who founded the Order of Poor Clerics Regular of the Mother of 
 God of the Christian Schools. He fell asleep in the Lord on the 25th 
 of August.
\switchcolumn*
\selectlanguage{latin}
Poténtiæ, in Lucánia, 
 pássio sanctórum Aróntii, Honoráti, Fortunáti et Sabiniáni; qui, sanctórum 
 Bonifátii et Theclæ fílii, a Valeriáno Júdice, sub Maximiáno Imperatóre, 
 jussi sunt capitálem subíre senténtiam. Eórum tamen ac reliquórum ex 
 duódecim frátribus festum Kaléndis Septémbris celebrátur.
\switchcolumn
\selectlanguage{english}
At Potenza in Lucania, the passion 
 of Saints Arontius, Honoratus, Fortunatus, and Sabinian. They were the 
 sons of Saints Boniface and Thecla, and were condemned to death by the judge 
 Valerian in the reign of Emperor Maximian. Their feast, together with 
 that of the other twelve holy brethren, is celebrated on the first of 
 September.
\switchcolumn*
\selectlanguage{latin}
Bérgomi sancti Narni, 
 qui, a beáto Bárnaba baptizátus, primus ab ipso ejúsdem civitátis Epíscopus 
 ordinátus est.
\switchcolumn
\selectlanguage{english}
At Bergamo, St. Narnus, who was 
 baptized by blessed Barnabas and consecrated by him first bishop of that 
 city.
\switchcolumn*
\selectlanguage{latin}
Cápuæ natális sancti 
 Rufi, Epíscopi et Mártyris; qui, cum esset patríciæ dignitátis, a beáto 
 Apollináre, sancti Petri discípulo, cum univérsa família baptizátus est.
\switchcolumn
\selectlanguage{english}
At Capua, the birthday of St. Rufus, 
 bishop and martyr, a patrician, who was baptized with all his family by 
 blessed Apollinaris, disciple of St. Peter.
\switchcolumn*
\selectlanguage{latin}
Ibídem sanctórum 
 Mártyrum Rufi et Carpóphori, qui sub Diocletiáno et Maximiáno passi sunt.
\switchcolumn
\selectlanguage{english}
In the same place, the holy martyrs 
 Rufus and Carpophorus, who suffered under Diocletian and Maximian.
\switchcolumn*
\selectlanguage{latin}
Tomis, in Ponto, 
 sanctórum Mártyrum Marcellíni Tribúni, et uxóris Mannéæ, ac filiórum Joánnis, 
 Serapiónis et Petri.
\switchcolumn
\selectlanguage{english}
At Tomis in Pontus, the holy martyrs 
 Marcellinus, a tribune, and Mannea, his wife, and his sons John, Serapion, 
 and Peter.
\switchcolumn*
\selectlanguage{latin}
Apud Leonínos, in 
 Sicília, sanctæ Eutháliæ Vírginis, quæ, cum esset Christiána, ad cæléstem 
 Sponsum, a fratre suo Sermiliáno cæsa gládio, migrávit.
\switchcolumn
\selectlanguage{english}
At Lentini in Sicily, St. Euthalia, 
 virgin. Because she was a Christian she was put to the sword by her 
 brother Sermilian, and went to her Spouse.
\switchcolumn*
\selectlanguage{latin}
Eódem die pássio sanctæ 
 Anthúsæ junióris, quæ ob Christi fidem, in púteum mersa, martyrium sumpsit.
\switchcolumn
\selectlanguage{english}
The same day, the martyrdom of St. 
 Anthusa the Younger, who was made a martyr by being cast into a well for the 
 faith of Christ.
\switchcolumn*
\selectlanguage{latin}
Areláte, in Gállia, 
 sancti Cæsárii Epíscopi, miræ sanctitátis et pietátis viri.
\switchcolumn
\selectlanguage{english}
At Arles in France, the holy bishop 
 Caesarius, a man of great sanctity and piety.
\switchcolumn*
\selectlanguage{latin}
Augustodúni sancti 
 Syágrii, Epíscopi et Confessóris.
\switchcolumn
\selectlanguage{english}
At Autun, St. Syagrius, bishop and 
 confessor.
\switchcolumn*
\selectlanguage{latin}
Papíæ sancti Joánnis 
 Epíscopi.
\switchcolumn
\selectlanguage{english}
At Pavia, St. John, bishop.
\switchcolumn*
\selectlanguage{latin}
Ilérdæ, in Hispánia 
 Tarraconénsi, sancti Licérii Epíscopi.
\switchcolumn
\selectlanguage{english}
At Lerida in Spain, St. Licerius, 
 bishop.
\switchcolumn*
\selectlanguage{latin}
In Thebáide sancti 
 Pœmenis Anachorétæ.
\switchcolumn
\selectlanguage{english}
In Thebais, St. Poemen, abbot.
\switchcolumn*
\selectlanguage{latin}
Apud Septempedános, in 
 Picéno, sanctæ Margarítæ Víduæ.
\switchcolumn
\selectlanguage{english}
At San Severino, in Piceno, St. 
 Margaret, widow.
\switchcolumn*
\selectlanguage{latin}
\end{paracol}


% ---- martyrology/mart08/mart0828.htm
\needspace{10\baselineskip}
\begin{paracol}{2}
\selectlanguage{latin}
\begin{center}{\color{gregoriocolor} Quinto Kaléndas Septémbris. 
 Luna\dots\ }\end{center}
\switchcolumn
\selectlanguage{english}
\begin{center}{\color{gregoriocolor} The 
 Twenty-Eighth Day of 
 August. The\dots\ Day of the Moon.}\end{center}
\end{paracol}

\noindent\begin{tabularx}{\linewidth}{*{19}{>{\centering\arraybackslash}X}}
 \textcolor{gregoriocolor}{a} & \textcolor{gregoriocolor}{b} & \textcolor{gregoriocolor}{c} & \textcolor{gregoriocolor}{d} & \textcolor{gregoriocolor}{e} & \textcolor{gregoriocolor}{f} & \textcolor{gregoriocolor}{g} & \textcolor{gregoriocolor}{h} & \textcolor{gregoriocolor}{i} & \textcolor{gregoriocolor}{k} & \textcolor{gregoriocolor}{l} & \textcolor{gregoriocolor}{m} & \textcolor{gregoriocolor}{n} & \textcolor{gregoriocolor}{p} & \textcolor{gregoriocolor}{q} & \textcolor{gregoriocolor}{r} & \textcolor{gregoriocolor}{s} & \textcolor{gregoriocolor}{t} & \textcolor{gregoriocolor}{u} \\
 5 & 6 & 7 & 8 & 9 & 10 & 11 & 12 & 13 & 14 & 15 & 16 & 17 & 18 & 19 & 20 & 21 & 22 & 23 \\
\end{tabularx}
\vspace{0.5\baselineskip}
\noindent\begin{tabularx}{\linewidth}{*{12}{>{\centering\arraybackslash}X}}
 \textcolor{gregoriocolor}{A} & \textcolor{gregoriocolor}{B} & \textcolor{gregoriocolor}{C} & \textcolor{gregoriocolor}{D} & \textcolor{gregoriocolor}{E} & F & \textcolor{gregoriocolor}{F} & \textcolor{gregoriocolor}{G} & \textcolor{gregoriocolor}{H} & \textcolor{gregoriocolor}{M} & \textcolor{gregoriocolor}{N} & \textcolor{gregoriocolor}{P} \\
 24 & 25 & 26 & 27 & 28 & 29 & 28 & 29 & 1 & 2 & 3 & 4 \\
\end{tabularx}

\begin{paracol}{2}
\selectlanguage{latin}
\lettrine[lines=2]{H}{ippóne} Régio, in 
 Africa, natális sancti Augustíni Epíscopi, Confessóris et Ecclésiæ Doctóris 
 exímii, qui, beáti Ambrósii Epíscopi ópera ad cathólicam fidem convérsus et 
 baptizátus, eam advérsus Manichæos aliósque hæréticos acérrimus propugnátor 
 deféndit, multísque áliis pro Ecclésia Dei perfúnctus labóribus, ad præmia 
 migrávit in cælum. Ejus relíquiæ, primo de sua civitáte propter 
 bárbaros in Sardíniam advéctæ, et póstea a Rege Longobardórum Luitprándo 
 Papíam translátæ, ibi honorífice cónditæ sunt.
\switchcolumn
\selectlanguage{english}
\lettrine[lines=2]{A}{t} Hippo in Africa, the birthday of 
 St. Augustine, bishop and famous doctor of the Church. Converted and 
 baptized by the blessed bishop Ambrose, he defended the Catholic faith with 
 the greatest zeal against the Manicheans and other heretics, and after 
 having sustained many other labours for the Church of God, he went to his 
 reward in heaven. His relics, owing to the invasion of barbarians, 
 were first brought from his own city into Sardinia, and afterwards taken by Luitprand, king of the Lombards, to Pavia, where they were deposited with 
 due honours.
\switchcolumn*
\selectlanguage{latin}
Romæ item natális 
 sancti Hermétis, viri illústris, qui (ut in Actis beáti Alexándri Papæ 
 légitur), prius carceráli custódiæ mancipátus, deínde, cum áliis plúrimis, 
 gládio cædénte, martyrium complévit, sub Aureliáno Júdice.
\switchcolumn
\selectlanguage{english}
At Rome, the birthday of St. Hermes, 
 an illustrious man, who, as we read in the Acts of blessed Pope Alexander, 
 was first confined in prison, and afterwards fulfilled his martyrdom by the 
 sword, at the time of the judge Aurelian.
\switchcolumn*
\selectlanguage{latin}
Venúsiæ, in Apúlia, pássio sanctórum Septimíni, Januárii et Felícis, qui, sanctórum Bonifátii et 
 Theclæ fílii, a Valeriáno Júdice, sub Maximiáno Imperatóre, jussi sunt 
 decollári. Ipsórum tamen ac reliquórum ex duódecim frátribus 
 festívitas ágitur Kaléndis Septémbris.
\switchcolumn
\selectlanguage{english}
At Venosa in Apulia, the passion of 
 Saints Septiminus, Januarius, and Felix. During the reign of Emperor 
 Maximian, the judge Valerian ordered these sons of Saints Boniface and 
 Thecla to be beheaded. Their feast, however, is observed with that of 
 the other Twelve Holy Brethren on the first of September.
\switchcolumn*
\selectlanguage{latin}
Briváte, apud Arvérnos, 
 item pássio sancti Juliáni Mártyris, qui, cum esset beáti Ferreóli Tribúni 
 comes et in hábitu militári occúlte Christo servíret, in persecutióne 
 Diocletiáni, a milítibus tentus est, et, desécto gútture, morte horríbili 
 necátus.
\switchcolumn
\selectlanguage{english}
At Prinde in Auvergne, St. Julian, 
 martyr, during the persecution of Diocletian. He was the companion of 
 the blessed tribune Ferreol, and under a military garb he secretly served 
 Christ until arrested by the soldiers, and killed in a barbarous manner by 
 having his throat cut.
\switchcolumn*
\selectlanguage{latin}
Constántiæ, in Germánia, sancti Pelágii Mártyris, qui sub Numeriáno Imperatóre et Evilásio Júdice, 
 cápite amputátus, martyrii corónam accépit.
\switchcolumn
\selectlanguage{english}
At Constance, in Germany, St. 
 Pelagius, martyr, who was beheaded and received the crown of martyrdom under Emperor Numerian 
 and the judge Evilasius.
\switchcolumn*
\selectlanguage{latin}
Salérni sanctórum 
 Mártyrum Fortunáti, Caji et Anthis; qui, sub Diocletiáno Imperatóre et 
 Leóntio Procónsule, decolláti sunt.
\switchcolumn
\selectlanguage{english}
At Salerno, the holy martyrs 
 Fortunatus, Caius, and Anthes, beheaded under Emperor Diocletian and the 
 proconsul Leontius.
\switchcolumn*
\selectlanguage{latin}
Constantinópoli sancti 
 Alexándri Epíscopi, gloriósi senis; ob cujus oratiónem Arius, divíno judício 
 damnátus, crépuit médius, et effúsa sunt víscera ejus.
\switchcolumn
\selectlanguage{english}
At Constantinople, the holy bishop 
 Alexander, an aged and celebrated man, through whose efficacious prayers 
 Arius, by the judgment of God, burst asunder and his bowels were poured 
 out.
\switchcolumn*
\selectlanguage{latin}
Apud Sántonas, in 
 Gállia, sancti Viviáni, Epíscopi et Confessóris.
\switchcolumn
\selectlanguage{english}
At Saintes, St. Vivian, bishop and 
 confessor.
\switchcolumn*
\selectlanguage{latin}
Item sancti Móysis 
 Æthíopis, qui, ex insígni latróne insígnis Anachoréta, multos latrónes 
 convértit et secum duxit ad monastérium.
\switchcolumn
\selectlanguage{english}
Also, St. Moses the Ethiopian, who gave up a life of 
 robbery and became a renowned anchoret. He converted many robbers, and 
 led them to a monastery.
\switchcolumn*
\selectlanguage{latin}
\end{paracol}


% ---- martyrology/mart08/mart0829.htm
\needspace{10\baselineskip}
\begin{paracol}{2}
\selectlanguage{latin}
\begin{center}{\color{gregoriocolor} Quarto Kaléndas Septémbris. 
 Luna\dots\ }\end{center}
\switchcolumn
\selectlanguage{english}
\begin{center}{\color{gregoriocolor} The 
 Twenty-Ninth Day of 
 August. The\dots\ Day of the Moon.}\end{center}
\end{paracol}

\noindent\begin{tabularx}{\linewidth}{*{19}{>{\centering\arraybackslash}X}}
 \textcolor{gregoriocolor}{a} & \textcolor{gregoriocolor}{b} & \textcolor{gregoriocolor}{c} & \textcolor{gregoriocolor}{d} & \textcolor{gregoriocolor}{e} & \textcolor{gregoriocolor}{f} & \textcolor{gregoriocolor}{g} & \textcolor{gregoriocolor}{h} & \textcolor{gregoriocolor}{i} & \textcolor{gregoriocolor}{k} & \textcolor{gregoriocolor}{l} & \textcolor{gregoriocolor}{m} & \textcolor{gregoriocolor}{n} & \textcolor{gregoriocolor}{p} & \textcolor{gregoriocolor}{q} & \textcolor{gregoriocolor}{r} & \textcolor{gregoriocolor}{s} & \textcolor{gregoriocolor}{t} & \textcolor{gregoriocolor}{u} \\
 6 & 7 & 8 & 9 & 10 & 11 & 12 & 13 & 14 & 15 & 16 & 17 & 18 & 19 & 20 & 21 & 22 & 23 & 24 \\
\end{tabularx}
\vspace{0.5\baselineskip}
\noindent\begin{tabularx}{\linewidth}{*{12}{>{\centering\arraybackslash}X}}
 \textcolor{gregoriocolor}{A} & \textcolor{gregoriocolor}{B} & \textcolor{gregoriocolor}{C} & \textcolor{gregoriocolor}{D} & \textcolor{gregoriocolor}{E} & F & \textcolor{gregoriocolor}{F} & \textcolor{gregoriocolor}{G} & \textcolor{gregoriocolor}{H} & \textcolor{gregoriocolor}{M} & \textcolor{gregoriocolor}{N} & \textcolor{gregoriocolor}{P} \\
 25 & 26 & 27 & 28 & 29 & 30 & 29 & 1 & 2 & 3 & 4 & 5 \\
\end{tabularx}

\begin{paracol}{2}
\selectlanguage{latin}
\lettrine[lines=2]{D}{ecollátio} sancti 
 Joánnis Baptístæ, quem Heródes circa festum Paschæ decollári præcépit. 
 Ipsíus tamen memória solémniter hac die cólitur, qua venerándum ejus caput 
 secúndo invéntum fuit; quod, póstea Romam translátum, in Ecclésia sancti 
 Silvéstri, ad Campum Mártium, summa pópuli devotióne asservátur.
\switchcolumn
\selectlanguage{english}
\lettrine[lines=2]{T}{he} beheading of St. John Baptist, 
 who was put to death by Herod about the feast of Easter. However, his 
 solemn commemoration takes place today, when his venerable head was found 
 for the second time. It was afterwards solemnly carried to Rome, where 
 it is kept in the church of St. Sylvester, near the Campus Martius, and 
 honoured by the people with the greatest devotion.
\switchcolumn*
\selectlanguage{latin}
Romæ, in monte Aventíno, 
 natális sanctæ Sabínæ Mártyris, quæ sub Hadriáno Imperatóre, gládio percússa, 
 martyrii palmam adépta est.
\switchcolumn
\selectlanguage{english}
At Rome, on Mount Aventine, the 
 birthday of St. Sabina, martyr. Under Emperor Hadrian, she was struck 
 with the sword, and thus obtained the palm of martyrdom.
\switchcolumn*
\selectlanguage{latin}
Veliniáni, in 
 confínibus Apúliæ, pássio sanctórum Vitális, Satóris, et Repósiti; qui, 
 sanctórum Bonifátii et Theclæ fílii, a Valeriáno Júdice, sub Maximiáno 
 Imperatóre, capitálem senténtiam pertulérunt. Eórum tamen ac ceterórum 
 ex duódecim frátribus memória Kaléndis Septémbris recólitur.
\switchcolumn
\selectlanguage{english}
At Valiniano in Apulia, the passion 
 of Saints Vitalis, Sator, and Repositus. They were the sons of Saints 
 Boniface and Thecla, and were condemned to death by the judge Valerian in 
 the reign of Emperor Maximian. Their feast along with that of the 
 other Twelve Holy Brethren is observed on the first of September.
\switchcolumn*
\selectlanguage{latin}
Romæ sanctæ Cándidæ, 
 Vírginis et Mártyris; cujus corpus beátus Paschális Primus Papa in Ecclésiam 
 sanctæ Praxédis tránstulit.
\switchcolumn
\selectlanguage{english}
At Rome, St. Candida, virgin and 
 martyr, whose body was transferred to the Church of St. Praxedes by Pope 
 Paschal I.
\switchcolumn*
\selectlanguage{latin}
Constantinópoli 
 sanctórum Mártyrum Hypátii, Asiáni Epíscopi, et Andréæ Presbyteri; qui ambo, 
 ob cultum sanctárum Imáginum, sub Leóne Isáurico, barba pice íllita atque 
 incénsa, et cute cápitis extrácta, juguláti sunt.
\switchcolumn
\selectlanguage{english}
At Constantinople, the holy martyrs 
 Hypatius, an Asiatic bishop, and Andrew, a priest, who for the veneration of 
 holy images, under Leo the Isaurian had their beards besmirched with pitch 
 and set on fire, the skin of the heads torn off, and were beheaded.
\switchcolumn*
\selectlanguage{latin}
Antiochíæ natális 
 sanctórum Mártyrum Nicǽæ et Pauli.
\switchcolumn
\selectlanguage{english}
At Antioch, the birthday of the holy 
 martyrs Nicaeas and Paul.
\switchcolumn*
\selectlanguage{latin}
Metis, in Gállia, 
 sancti Adélphi, Epíscopi et Confessóris.
\switchcolumn
\selectlanguage{english}
At Metz in France, St. Adelphus, 
 bishop and confessor.
\switchcolumn*
\selectlanguage{latin}
Lutétiæ Parisiórum 
 deposítio sancti Mederíci Presbyteri.
\switchcolumn
\selectlanguage{english}
At Paris, the death of St. Merry, 
 priest.
\switchcolumn*
\selectlanguage{latin}
Perúsiæ sancti Euthymii 
 Románi, qui, cum uxóre et Crescéntio fílio persecutiónem Diocletiáni fúgiens, 
 ad eam urbem secéssit, et ibi póstmodum quiévit in Dómino.
\switchcolumn
\selectlanguage{english}
At Perugia, St. Euthymius, a Roman, 
 who fled from the persecution of Diocletian with this wife and his son 
 Crescentius, and there rested in the Lord.
\switchcolumn*
\selectlanguage{latin}
In Anglia sancti Sebbi 
 Regis.
\switchcolumn
\selectlanguage{english}
In England, St. Sebbe, king.
\switchcolumn*
\selectlanguage{latin}
Apud Sírmium natális 
 sanctæ Basíllæ Vírginis.
\switchcolumn
\selectlanguage{english}
At Smyrna, the birthday of St. 
 Basilla, virgin.
\switchcolumn*
\selectlanguage{latin}
In pago Tricassíno 
 sanctæ Sabínæ Vírginis, virtútibus et miráculis gloriósæ.
\switchcolumn
\selectlanguage{english}
In the vicinity of Troyes, St. 
 Sabina, a virgin, celebrated for virtues and miracles.
\switchcolumn*
\selectlanguage{latin}
\end{paracol}


% ---- martyrology/mart08/mart0830.htm
\needspace{10\baselineskip}
\begin{paracol}{2}
\selectlanguage{latin}
\begin{center}{\color{gregoriocolor} Tértio Kaléndas Septémbris. 
 Luna\dots\ }\end{center}
\switchcolumn
\selectlanguage{english}
\begin{center}{\color{gregoriocolor} The 
 Thirtieth Day of 
 August. The\dots\ Day of the Moon.}\end{center}
\end{paracol}

\noindent\begin{tabularx}{\linewidth}{*{19}{>{\centering\arraybackslash}X}}
 \textcolor{gregoriocolor}{a} & \textcolor{gregoriocolor}{b} & \textcolor{gregoriocolor}{c} & \textcolor{gregoriocolor}{d} & \textcolor{gregoriocolor}{e} & \textcolor{gregoriocolor}{f} & \textcolor{gregoriocolor}{g} & \textcolor{gregoriocolor}{h} & \textcolor{gregoriocolor}{i} & \textcolor{gregoriocolor}{k} & \textcolor{gregoriocolor}{l} & \textcolor{gregoriocolor}{m} & \textcolor{gregoriocolor}{n} & \textcolor{gregoriocolor}{p} & \textcolor{gregoriocolor}{q} & \textcolor{gregoriocolor}{r} & \textcolor{gregoriocolor}{s} & \textcolor{gregoriocolor}{t} & \textcolor{gregoriocolor}{u} \\
 7 & 8 & 9 & 10 & 11 & 12 & 13 & 14 & 15 & 16 & 17 & 18 & 19 & 20 & 21 & 22 & 23 & 24 & 25 \\
\end{tabularx}
\vspace{0.5\baselineskip}
\noindent\begin{tabularx}{\linewidth}{*{12}{>{\centering\arraybackslash}X}}
 \textcolor{gregoriocolor}{A} & \textcolor{gregoriocolor}{B} & \textcolor{gregoriocolor}{C} & \textcolor{gregoriocolor}{D} & \textcolor{gregoriocolor}{E} & F & \textcolor{gregoriocolor}{F} & \textcolor{gregoriocolor}{G} & \textcolor{gregoriocolor}{H} & \textcolor{gregoriocolor}{M} & \textcolor{gregoriocolor}{N} & \textcolor{gregoriocolor}{P} \\
 26 & 27 & 28 & 29 & 30 & 1 & 1 & 2 & 3 & 4 & 5 & 6 \\
\end{tabularx}

\begin{paracol}{2}
\selectlanguage{latin}
\lettrine[lines=2]{S}{anctæ} Rosæ a Sancta María, e tértio Ordine sancti Domínici, Vírginis; cujus dies natális nono 
 Kaléndas Septémbris recensétur.
\switchcolumn
\selectlanguage{english}
\lettrine[lines=2]{T}{he} feast of St. Rose of St. Mary, 
 virgin of the Third Order of St. Dominic, whose birthday is recalled on the 
 24th of August.
\switchcolumn*
\selectlanguage{latin}
Romæ, via Ostiénsi, 
 pássio beáti Felícis Presbyteri, sub Diocletiáno et Maximiáno Imperatóribus. 
 Hic, post equúlei vexatiónem, data senténtia, cum ducerétur ad decollándum, 
 óbvius ei fuit quidam Christiánus, qui, dum se Christiánum esse sponte 
 profiterétur, mox cum eódem páriter decollátus est; cujus nomen ignorántes, 
 Christiáni Adáuctum eum appellavérunt, eo quod sancto Felíci auctus sit ad 
 corónam.
\switchcolumn
\selectlanguage{english}
At Rome, on the Ostian Way, the 
 martyrdom of the blessed priest Felix, under Emperors Diocletian and 
 Maximian. After being racked he was sentenced to death, and as they 
 led him to execution, he met a man who spontaneously declared himself a 
 Christian, and was forthwith beheaded with him. The Christians, not 
 knowing his name, called him Adauctus, because he was added to St. Felix and 
 shared his crown.
\switchcolumn*
\selectlanguage{latin}
Item Romæ sanctæ 
 Gaudéntiæ, Vírginis et Mártyris, cum áliis tribus.
\switchcolumn
\selectlanguage{english}
Also at Rome, St. Gaudentia, virgin 
 and martyr, with three others.
\switchcolumn*
\selectlanguage{latin}
Colóniæ Suffetulánæ, in 
 Africa, beatórum sexagínta Mártyrum, qui furóre Gentílium cæsi sunt.
\switchcolumn
\selectlanguage{english}
At Colonia Suffetulana in Africa, 
 sixty blessed martyrs, who were murdered by the furious heathen.
\switchcolumn*
\selectlanguage{latin}
Bonóniæ sancti Bonónii 
 Abbátis.
\switchcolumn
\selectlanguage{english}
At Bologna, St. Bononius, abbot.
\switchcolumn*
\selectlanguage{latin}
Romæ sancti Pammáchii 
 Presbyteri, qui fuit doctrína et sanctitáte conspícuus.
\switchcolumn
\selectlanguage{english}
At Rome, St Pammachius, priest, who 
 was noteworthy for learning and sanctity.
\switchcolumn*
\selectlanguage{latin}
Adruméti, in Africa, 
 sanctórum Bonifátii et Theclæ, qui beatórum duódecim filiórum Mártyrum 
 paréntes fuérunt.
\switchcolumn
\selectlanguage{english}
At Adrumetum, also in Africa, the 
 Saints Boniface and Thecla, who were the parents of twelve blessed sons, all 
 martyrs.
\switchcolumn*
\selectlanguage{latin}
Thessalonícæ sancti 
 Fantíni Confessóris, qui, multa a Saracénis perpéssus, atque e monastério, 
 in quo abstinéntia víxerat admirábili, expúlsus, demum, cum plúrimos ad viam 
 salútis perduxísset, in senectúte bona quiévit.
\switchcolumn
\selectlanguage{english}
At Thessalonica, St. Fantinus, 
 confessor, who suffered much from the Saracens, and was driven from his 
 monastery, in which he had lived in great abstinence. After having 
 brought many to the way of salvation, he rested at last at an advanced age.
\switchcolumn*
\selectlanguage{latin}
In território Meldénsi 
 sancti Fiácrii Confessóris.
\switchcolumn
\selectlanguage{english}
In the diocese of Meaux, St. Fiacre, 
 confessor.
\switchcolumn*
\selectlanguage{latin}
Trebis, in Látio, 
 sancti Petri Confessóris, qui, multis clarus virtútibus et miráculis, ibídem 
 migrávit ad Dóminum, et honorífice cólitur.
\switchcolumn
\selectlanguage{english}
At Trevi in Lazio, St. Peter, 
 confessor, who was distinguished for many virtues and miracles. He is 
 honoured in that place from which he departed for heaven.
\switchcolumn*
\selectlanguage{latin}
\end{paracol}


% ---- martyrology/mart08/mart0831.htm
\needspace{10\baselineskip}
\begin{paracol}{2}
\selectlanguage{latin}
\begin{center}{\color{gregoriocolor} Prídie Kaléndas Septémbris. 
 Luna\dots\ }\end{center}
\switchcolumn
\selectlanguage{english}
\begin{center}{\color{gregoriocolor} The 
 Thirty-First Day of 
 August. The\dots\ Day of the Moon.}\end{center}
\end{paracol}

\noindent\begin{tabularx}{\linewidth}{*{19}{>{\centering\arraybackslash}X}}
 \textcolor{gregoriocolor}{a} & \textcolor{gregoriocolor}{b} & \textcolor{gregoriocolor}{c} & \textcolor{gregoriocolor}{d} & \textcolor{gregoriocolor}{e} & \textcolor{gregoriocolor}{f} & \textcolor{gregoriocolor}{g} & \textcolor{gregoriocolor}{h} & \textcolor{gregoriocolor}{i} & \textcolor{gregoriocolor}{k} & \textcolor{gregoriocolor}{l} & \textcolor{gregoriocolor}{m} & \textcolor{gregoriocolor}{n} & \textcolor{gregoriocolor}{p} & \textcolor{gregoriocolor}{q} & \textcolor{gregoriocolor}{r} & \textcolor{gregoriocolor}{s} & \textcolor{gregoriocolor}{t} & \textcolor{gregoriocolor}{u} \\
 8 & 9 & 10 & 11 & 12 & 13 & 14 & 15 & 16 & 17 & 18 & 19 & 20 & 21 & 22 & 23 & 24 & 25 & 26 \\
\end{tabularx}
\vspace{0.5\baselineskip}
\noindent\begin{tabularx}{\linewidth}{*{12}{>{\centering\arraybackslash}X}}
 \textcolor{gregoriocolor}{A} & \textcolor{gregoriocolor}{B} & \textcolor{gregoriocolor}{C} & \textcolor{gregoriocolor}{D} & \textcolor{gregoriocolor}{E} & F & \textcolor{gregoriocolor}{F} & \textcolor{gregoriocolor}{G} & \textcolor{gregoriocolor}{H} & \textcolor{gregoriocolor}{M} & \textcolor{gregoriocolor}{N} & \textcolor{gregoriocolor}{P} \\
 27 & 28 & 29 & 30 & 1 & 2 & 2 & 3 & 4 & 5 & 6 & 7 \\
\end{tabularx}

\begin{paracol}{2}
\selectlanguage{latin}
\lettrine[lines=2]{S}{ancti} Raymúndi Nonnáti, 
 ex Ordine beátæ Maríæ de Mercéde redemptiónis captivórum, Cardinális et 
 Confessóris; cujus dies natális séptimo Kaléndas Septémbris recólitur.
\switchcolumn
\selectlanguage{english}
\lettrine[lines=2]{S}{t.} Raymund Nonnatus, cardinal and 
 confessor, of the Order of our Lady of Ransom for the Redemption of 
 Captives. His birthday is commemorated on the 26th of August.
\switchcolumn*
\selectlanguage{latin}
Apud montem Senárium, 
 in Etrúria, natális sancti Bonajúnctæ Confessóris, e septem Fundatóribus 
 Ordinis Servórum beátæ Maríæ Vírginis; qui, de Passióne Domínica ad fratres 
 verba fáciens, in manus Dómini trádidit spíritum. Ipsíus autem ac 
 Sociórum festum prídie Idus Februárii celebrátur.
\switchcolumn
\selectlanguage{english}
In Tuscany, on Mount Senario, the 
 birthday of St. Bonajuncta, confessor, one of the seven founders of the 
 Order of Servites of the Blessed Virgin Mary, who gave up his soul into the 
 hands of the Lord while he was preaching to his brethren on the Passion of 
 our Saviour. his feast is kept with that of his companions on the 12th 
 of February.
\switchcolumn*
\selectlanguage{latin}
Tréviris item natális 
 sancti Paulíni Epíscopi, qui, témpore Ariánæ infestatiónis, ab Ariáno 
 Imperatóre Constántio ob cathólicam fidem relegátus exsílio, et extra 
 Christiánum nomen usque ad mortem mutándo exsília fatigátus, tandem, apud 
 Phrygiam defúnctus, beátæ passiónis corónam percépit a Dómino.
\switchcolumn
\selectlanguage{english}
At Treves, the birthday of St. 
 Paulinus, a bishop, who was exiled for the Catholic faith by the Arian 
 emperor Constantius, in the time of the Arian persecution. By having 
 to change the place of his exile, which was beyond the limits of 
 Christendom, he became wearied unto death, and finally, dying in Phrygia, 
 received a crown from the Lord for his blessed martyrdom.
\switchcolumn*
\selectlanguage{latin}
Tránsaquis, ad lacum 
 Fúcinum, in Marsis, natális quoque sanctórum Mártyrum Cæsídii Presbyteri, et 
 Sociórum; qui martyrio coronáti sunt in persecutióne Maximíni.
\switchcolumn
\selectlanguage{english}
At Transaco, in the Marches near 
 Lake Fucino, the birthday of the holy martyrs Caesidius, priest, and his 
 companions, who were crowned with martyrdom in the persecution of Maximinus.
\switchcolumn*
\selectlanguage{latin}
Item sanctórum Mártyrum 
 Robustiáni et Marci.
\switchcolumn
\selectlanguage{english}
Also, the holy martyrs Robustian and 
 Mark.
\switchcolumn*
\selectlanguage{latin}
Cæsaréæ, in Cappadócia, sanctórum Theódoti, Rufínæ et Ammiæ; quorum duo primi paréntes fuérunt 
 sancti Mamántis Mártyris, quem Rufína in cárcere péperit, et Ammia educávit.
\switchcolumn
\selectlanguage{english}
At Caesarea in Cappadocia, the 
 Saints Theodotus, Rufina, and Ammia. The first two were the parents of 
 the martyr St. Mamas, who was born in prison, and whom Ammia brought up.
\switchcolumn*
\selectlanguage{latin}
Antisiodóri sancti 
 Optáti, Epíscopi et Confessóris.
\switchcolumn
\selectlanguage{english}
At Auxerre, St. Optatus, bishop and 
 confessor.
\switchcolumn*
\selectlanguage{latin}
In Anglia sancti Aidáni, 
 Epíscopi Lindisfarnénsis; cujus ánimam, cum sanctus Cuthbértus, cujus 
 memória tertiodécimo Kaléndas Aprilis cólitur, tunc óvium pastor, in cælum 
 ferri vidísset, relíctis ovibus, factus est Mónachus.
\switchcolumn
\selectlanguage{english}
In England, St. Aidan, bishop of 
 Lindisfarne. When St. Cuthbert, then a shepherd, saw his soul going up 
 to heaven, he left his sheep and became a monk. Mention is made of St. 
 Cuthbert on the 20th of March.
\switchcolumn*
\selectlanguage{latin}
Apud Nuscum sancti 
 Amáti Epíscopi.
\switchcolumn
\selectlanguage{english}
At Nosco, St. Amatus, bishop.
\switchcolumn*
\selectlanguage{latin}
Athénis sancti 
 Aristídis, fide et sapiéntia claríssimi, qui Hadriáno Príncipi egrégium de 
 religióne Christiána volúmen óbtulit, nostri dógmatis cóntinens ratiónem; et 
 quod Christus Jesus solus esset Deus, præsénte ipso Imperatóre, 
 luculentíssime perorávit.
\switchcolumn
\selectlanguage{english}
At Athens, St. Aristides, most 
 celebrated for his faith and wisdom, who presented to Emperor Hadrian a 
 treatise on the Christian religion, containing the exposition of our 
 doctrine. In the presence of the emperor, he also delivered a 
 discourse in which he clearly demonstrated that Jesus Christ is the only God.
\switchcolumn*
\selectlanguage{latin}
\end{paracol}

\setrunningtitles{September}{September}

% ---- martyrology/mart09/mart0901.htm
\needspace{10\baselineskip}
\begin{paracol}{2}
\selectlanguage{latin}
\begin{center}{\color{gregoriocolor} Kaléndis Septémbris. 
 Luna\dots\ }\end{center}
\switchcolumn
\selectlanguage{english}
\begin{center}{\color{gregoriocolor} The   First Day of 
 September. The\dots\ Day of the Moon.}\end{center}
\end{paracol}

\noindent\begin{tabularx}{\linewidth}{*{19}{>{\centering\arraybackslash}X}}
 \textcolor{gregoriocolor}{a} & \textcolor{gregoriocolor}{b} & \textcolor{gregoriocolor}{c} & \textcolor{gregoriocolor}{d} & \textcolor{gregoriocolor}{e} & \textcolor{gregoriocolor}{f} & \textcolor{gregoriocolor}{g} & \textcolor{gregoriocolor}{h} & \textcolor{gregoriocolor}{i} & \textcolor{gregoriocolor}{k} & \textcolor{gregoriocolor}{l} & \textcolor{gregoriocolor}{m} & \textcolor{gregoriocolor}{n} & \textcolor{gregoriocolor}{p} & \textcolor{gregoriocolor}{q} & \textcolor{gregoriocolor}{r} & \textcolor{gregoriocolor}{s} & \textcolor{gregoriocolor}{t} & \textcolor{gregoriocolor}{u} \\
 9 & 10 & 11 & 12 & 13 & 14 & 15 & 16 & 17 & 18 & 19 & 20 & 21 & 22 & 23 & 24 & 25 & 26 & 27 \\
\end{tabularx}
\vspace{0.5\baselineskip}
\noindent\begin{tabularx}{\linewidth}{*{12}{>{\centering\arraybackslash}X}}
 \textcolor{gregoriocolor}{A} & \textcolor{gregoriocolor}{B} & \textcolor{gregoriocolor}{C} & \textcolor{gregoriocolor}{D} & \textcolor{gregoriocolor}{E} & F & \textcolor{gregoriocolor}{F} & \textcolor{gregoriocolor}{G} & \textcolor{gregoriocolor}{H} & \textcolor{gregoriocolor}{M} & \textcolor{gregoriocolor}{N} & \textcolor{gregoriocolor}{P} \\
 28 & 29 & 30 & 1 & 2 & 3 & 3 & 4 & 5 & 6 & 7 & 8 \\
\end{tabularx}

\begin{paracol}{2}
\selectlanguage{latin}
\lettrine[lines=2]{I}{n} província Narbonénsi 
 sancti Ægídii, Abbátis et Confessóris, cujus nómine est appellátum óppidum, 
 quod póstea crevit in loco, ubi ipse monastérium eréxerat et mortális vitæ 
 cursum absólverat.
\switchcolumn
\selectlanguage{english}
\lettrine[lines=2]{I}{n} the province of Narbonne, St. 
 Giles, abbot and confessor. A town which later arose in the place 
 where he had built his monastery and where he died was named after him.
\switchcolumn*
\selectlanguage{latin}
Sentiáni, in fínibus 
 Apúliæ, pássio sanctórum Donáti et altérius Felícis; qui, sanctórum 
 Bonifátii et Theclæ fílii, a Valeriáno Júdice, sub Maximiáno Imperatóre, 
 jussi sunt, post vária torménta, cápite præcídi hodiérna die, in qua et 
 festívitas aliórum ex duódecim frátribus, quorum natális respectívis diébus 
 ágitur, institúta est celebrári. Ipsórum vero duódecim fratrum córpora 
 Benevéntum póstea transláta sunt, ibíque honorífice asserváta.
\switchcolumn
\selectlanguage{english}
At Sentiano, in the district of 
 Apulia, the passion of Saints Donatus and a second Felix who were the sons 
 of Saints Boniface and Thecla. After they had endured various torments 
 under the judge Valerian in the reign of Emperor Maximian, they were 
 condemned to be beheaded on this day. Today also is kept the festival 
 of the others of the Twelve Holy Brethren, whose birthdays are noted in 
 their proper place. The bodies of these Twelve Holy Brethren were 
 later translated to Benevento where they are honourably enshrined.
\switchcolumn*
\selectlanguage{latin}
In Palæstína sanctórum 
 Jósue et Gedeónis.
\switchcolumn
\selectlanguage{english}
In Palestine, the Saints Joshua and 
 Gideon.
\switchcolumn*
\selectlanguage{latin}
Hierosólymis beátæ Annæ 
 Prophetíssæ, cujus sanctitátem sermo Evangélicus prodit.
\switchcolumn
\selectlanguage{english}
At Jerusalem, blessed Anna, 
 prophetess, whose sanctity is revealed in the Gospel.
\switchcolumn*
\selectlanguage{latin}
Cápuæ, via Aquária, 
 sancti Prisci Mártyris, qui fuit unus de antíquis Christi discípulis.
\switchcolumn
\selectlanguage{english}
At Capua, on the Via Aquaria, St. 
 Priscus, martyr, who was formerly one of the disciples of Christ.
\switchcolumn*
\selectlanguage{latin}
Tudérti, in Umbria, 
 sancti Terentiáni, Epíscopi et Mártyris; qui, sub Hadriáno Imperatóre, 
 Lætiáni Procónsulis jussu, equúleo et scorpiónibus cruciátus est, ac demum, 
 abscíssa lingua, cápitis damnátus martyrium complévit.
\switchcolumn
\selectlanguage{english}
At Todi in Umbria, St. Terentian, 
 bishop and martyr. Under Emperor Hadrian, by order of the proconsul 
 Laetian, he was racked, scourged with whips set with metal, and finally 
 having had his tongue cut out, he ended his martyrdom by undergoing capital 
 punishment.
\switchcolumn*
\selectlanguage{latin}
Heracléæ, in Thrácia, 
 sancti Ammónis Diáconi, et sanctárum quadragínta Vírginum, quas ille 
 erudívit in fide, et, sub Licínio tyránno, ad martyrii glóriam secum 
 perdúxit.
\switchcolumn
\selectlanguage{english}
At Heraclea, under the tyrant 
 Licinius, St. Ammon, deacon, and forty holy virgins whom he instructed in 
 the faith and led with him to the glory of martyrdom.
\switchcolumn*
\selectlanguage{latin}
In Hispániæ sanctórum 
 Mártyrum Vincéntii et Læti.
\switchcolumn
\selectlanguage{english}
In Spain, the holy martyrs Vincent 
 and Laetus.
\switchcolumn*
\selectlanguage{latin}
Populónii, in Túscia, 
 sancti Réguli Mártyris, qui ex Africa illuc venit, ibíque, sub Tótila, 
 martyrium consummávit.
\switchcolumn
\selectlanguage{english}
At Piombino in Tuscany, St. Regulus, 
 martyr, who went thither from Africa, and consummated his martyrdom under 
 Totila.
\switchcolumn*
\selectlanguage{latin}
Cápuæ sancti Prisci 
 Epíscopi, qui unus fuit ex illis Sacerdótibus, qui, in persecutióne 
 Wandalórum, ob fidem cathólicam várie afflícti et vetústæ navi impósiti, ex 
 Africa ad Campániæ líttora pervenérunt, et Christiánam religiónem, in iis 
 locis dispérsi diversísque Ecclésiis præfécti, mirífice propagárunt. 
 Ipsíus autem fuérunt sócii Castrénsis, cujus dies natális tértio Idus 
 Februárii recólitur, Támmarus, Rósius, Heráclius, Secundínus, Adjútor, 
 Marcus, Augústus, Elpídius, Cánion et Vindónius.
\switchcolumn
\selectlanguage{english}
At Capua, St. Priscus, bishop. 
 He was one of those priests who were subjected to various trials for the 
 Catholic faith during the persecution of the Vandals. Being put in an 
 old ship on the coast of Africa, they reached the shores of Campania, and 
 separating, they were placed at the head of various churches, and thus 
 greatly extended the Christian religion. The companions of Priscus 
 were Castrensis, whose birthday is mentioned on the 11th of February, 
 Tammarius, Rosius, Heraclius, Secundinus, Adjutor, Mark, Augustus, Elpidius, 
 Canion, and Vindonius.
\switchcolumn*
\selectlanguage{latin}
Apud Sénonas beáti Lupi, 
 Epíscopi et Confessóris; de quo refértur quod quadam die, præsénte Clero, 
 dum sacris altáribus adstáret, lapsa est cælitus gemma in ejus cálicem 
 sanctum.
\switchcolumn
\selectlanguage{english}
At Sens, St. Lupus, bishop and 
 confessor, of whom it is related that on a certain day, while he stood at 
 the holy altar in the presence of the clergy, a gem fell from heaven into 
 the consecrated chalice which he was using.
\switchcolumn*
\selectlanguage{latin}
Rhemis, in Gállia, 
 sancti Xysti, qui fuit primus ejúsdem civitátis Epíscopus.
\switchcolumn
\selectlanguage{english}
At Rheims in France, St. Sixtus, 
 disciple of the blessed apostle Peter, who consecrated him the first bishop 
 of that city. He received the crown of martyrdom under Nero.
\switchcolumn*
\selectlanguage{latin}
Apud Cenómanos, in 
 Gállia, sancti Victórii Epíscopi.
\switchcolumn
\selectlanguage{english}
At Le Mans in France, St. Victorinus, 
 bishop.
\switchcolumn*
\selectlanguage{latin}
Apud Aquínum sancti 
 Constántii Epíscopi, prophetíæ dono multísque virtútibus clari.
\switchcolumn
\selectlanguage{english}
At Aquino, St. Constantius, a bishop 
 renowned for the gift of prophecy and many virtues.
\switchcolumn*
\selectlanguage{latin}
Ad Aquas Duras, in 
 Constantiénsi Germániæ território, sanctæ Verénæ Vírginis.
\switchcolumn
\selectlanguage{english}
In Baden, in the province of 
 Constance, St. Verena, virgin.
\switchcolumn*
\selectlanguage{latin}
\end{paracol}


% ---- martyrology/mart09/mart0902.htm
\needspace{10\baselineskip}
\begin{paracol}{2}
\selectlanguage{latin}
\begin{center}{\color{gregoriocolor} Quarto Nonas Septémbris. 
 Luna\dots\ }\end{center}
\switchcolumn
\selectlanguage{english}
\begin{center}{\color{gregoriocolor} The Second Day of September. The\dots\ Day 
 of the Moon.}\end{center}
\end{paracol}

\noindent\begin{tabularx}{\linewidth}{*{19}{>{\centering\arraybackslash}X}}
 \textcolor{gregoriocolor}{a} & \textcolor{gregoriocolor}{b} & \textcolor{gregoriocolor}{c} & \textcolor{gregoriocolor}{d} & \textcolor{gregoriocolor}{e} & \textcolor{gregoriocolor}{f} & \textcolor{gregoriocolor}{g} & \textcolor{gregoriocolor}{h} & \textcolor{gregoriocolor}{i} & \textcolor{gregoriocolor}{k} & \textcolor{gregoriocolor}{l} & \textcolor{gregoriocolor}{m} & \textcolor{gregoriocolor}{n} & \textcolor{gregoriocolor}{p} & \textcolor{gregoriocolor}{q} & \textcolor{gregoriocolor}{r} & \textcolor{gregoriocolor}{s} & \textcolor{gregoriocolor}{t} & \textcolor{gregoriocolor}{u} \\
 10 & 11 & 12 & 13 & 14 & 15 & 16 & 17 & 18 & 19 & 20 & 21 & 22 & 23 & 24 & 25 & 26 & 27 & 28 \\
\end{tabularx}
\vspace{0.5\baselineskip}
\noindent\begin{tabularx}{\linewidth}{*{12}{>{\centering\arraybackslash}X}}
 \textcolor{gregoriocolor}{A} & \textcolor{gregoriocolor}{B} & \textcolor{gregoriocolor}{C} & \textcolor{gregoriocolor}{D} & \textcolor{gregoriocolor}{E} & F & \textcolor{gregoriocolor}{F} & \textcolor{gregoriocolor}{G} & \textcolor{gregoriocolor}{H} & \textcolor{gregoriocolor}{M} & \textcolor{gregoriocolor}{N} & \textcolor{gregoriocolor}{P} \\
 29 & 30 & 1 & 2 & 3 & 4 & 4 & 5 & 6 & 7 & 8 & 9 \\
\end{tabularx}

\begin{paracol}{2}
\selectlanguage{latin}
\lettrine[lines=2]{S}{ancti} Stéphani, Regis 
 Hungarórum et Confessóris; qui décimo octávo Kaléndas Septémbris obdormívit 
 in Dómino.
\switchcolumn
\selectlanguage{english}
\lettrine[lines=2]{S}{t.} Stephen, king of Hungary and 
 confessor, who fell asleep in the Lord on the 15th of August.
\switchcolumn*
\selectlanguage{latin}
Romæ sanctæ Máximæ 
 Mártyris, quæ, simul cum sancto Ansáno Christum conféssa, in persecutióne 
 Diocletiáni, dum fústibus cæditur, réddidit spíritum.
\switchcolumn
\selectlanguage{english}
At Rome, the holy martyr Maxima, who 
 confessed Christ with St. Ansanus in the persecution of Diocletian, and 
 yielded up her soul while being beaten with rods.
\switchcolumn*
\selectlanguage{latin}
Pámiæ, in Gállia, 
 sancti Antoníni Mártyris, cujus relíquiæ apud Ecclésiam Palentínam, in 
 Hispánia, magna veneratióne asservántur.
\switchcolumn
\selectlanguage{english}
At Pamiers in France, St. Antoninus, 
 martyr, whose relics are kept with great veneration in the church of 
 Palencia, in Spain.
\switchcolumn*
\selectlanguage{latin}
Item sanctórum Mártyrum 
 Diomédis, Juliáni, Philíppi, Euthychiáni, Hesychii, Leónidæ, Philadélphi, 
 Menalíppi et Pantágapæ; quorum álii igne, álii aqua, ense álii et cruce 
 martyrium complevérunt.
\switchcolumn
\selectlanguage{english}
Also, the holy martyrs, Diomedes, 
 Julian, Philip, Eutychian, Hesychius, Leonides, Philadelphus, Menalippus, 
 and Pantagapas. Their martyrdoms were completed, some by fire, some 
 water, others by the sword or by the cross.
\switchcolumn*
\selectlanguage{latin}
Nicomedíæ sanctórum 
 Mártyrum Zenónis, atque Concórdii et Theodóri, filiórum ejus.
\switchcolumn
\selectlanguage{english}
At Nicomedia, the holy martyrs Zeno, 
 and his sons Concordius and Theodore.
\switchcolumn*
\selectlanguage{latin}
Lugdúni, in Gállia, 
 sancti Elpídii, Epíscopi et Confessóris.
\switchcolumn
\selectlanguage{english}
At Lyons in France, St. Elpidius, 
 bishop and confessor.
\switchcolumn*
\selectlanguage{latin}
In Picéno item sancti 
 Elpídii Abbátis, cujus nómine óppidum est appellátum, quod ejus sacrum 
 corpus se possidére congáudet.
\switchcolumn
\selectlanguage{english}
In Piceno, another St. Elpidius, an 
 abbot. A town bearing his name glories in the possession of his holy 
 body.
\switchcolumn*
\selectlanguage{latin}
In monte Sorácte sancti 
 Nonnósi Abbátis, qui ingéntis molis saxum oratióne sua tránstulit, aliísque 
 miráculis coruscávit.
\switchcolumn
\selectlanguage{english}
On Mount Soracte, Abbot St. Nonnosus, 
 who by his prayers moved a rock of huge proportions, and was renowned for 
 other miracles.
\switchcolumn*
\selectlanguage{latin}
Eódem die Commemorátio 
 sanctórum Mártyrum germanórum Evódii, Hermógenis et Callístæ; de quibus, in 
 Syracusána Sicíliæ urbe martyrium passis, ágitur étiam séptimo Kaléndas Maji.
\switchcolumn
\selectlanguage{english}
On the same day, the commemoration 
 of the holy martyrs Evodius and Hermogenes, brothers, and Callista, their 
 sister. Mention is made of them that they died on the 25th of April in 
 the city of Syracuse in Italy.
\switchcolumn*
\selectlanguage{latin}
Lugdúni, in Gállia, 
 Translátio sanctórum Justi, Epíscopi et Confessóris, ac Viatóris, qui ejus 
 fúerat miníster; quorum natális dies respectíve prídie Idus Octóbris et 
 duodécimo Kaléndas Novémbris recensétur.
\switchcolumn
\selectlanguage{english}
At Lyons in France, the translation 
 of St. Justus, bishop and confessor, and Viator, his servant, whose 
 birthdays occur on the 14th of October and the 21st of October.
\switchcolumn*
\selectlanguage{latin}
\end{paracol}


% ---- martyrology/mart09/mart0903.htm
\needspace{10\baselineskip}
\begin{paracol}{2}
\selectlanguage{latin}
\begin{center}{\color{gregoriocolor} Tértio Nonas Septémbris. 
 Luna\dots\ }\end{center}
\switchcolumn
\selectlanguage{english}
\begin{center}{\color{gregoriocolor} The   Third Day of 
 September. The\dots\ Day of the Moon.}\end{center}
\end{paracol}

\noindent\begin{tabularx}{\linewidth}{*{19}{>{\centering\arraybackslash}X}}
 \textcolor{gregoriocolor}{a} & \textcolor{gregoriocolor}{b} & \textcolor{gregoriocolor}{c} & \textcolor{gregoriocolor}{d} & \textcolor{gregoriocolor}{e} & \textcolor{gregoriocolor}{f} & \textcolor{gregoriocolor}{g} & \textcolor{gregoriocolor}{h} & \textcolor{gregoriocolor}{i} & \textcolor{gregoriocolor}{k} & \textcolor{gregoriocolor}{l} & \textcolor{gregoriocolor}{m} & \textcolor{gregoriocolor}{n} & \textcolor{gregoriocolor}{p} & \textcolor{gregoriocolor}{q} & \textcolor{gregoriocolor}{r} & \textcolor{gregoriocolor}{s} & \textcolor{gregoriocolor}{t} & \textcolor{gregoriocolor}{u} \\
 11 & 12 & 13 & 14 & 15 & 16 & 17 & 18 & 19 & 20 & 21 & 22 & 23 & 24 & 25 & 26 & 27 & 28 & 29 \\
\end{tabularx}
\vspace{0.5\baselineskip}
\noindent\begin{tabularx}{\linewidth}{*{12}{>{\centering\arraybackslash}X}}
 \textcolor{gregoriocolor}{A} & \textcolor{gregoriocolor}{B} & \textcolor{gregoriocolor}{C} & \textcolor{gregoriocolor}{D} & \textcolor{gregoriocolor}{E} & F & \textcolor{gregoriocolor}{F} & \textcolor{gregoriocolor}{G} & \textcolor{gregoriocolor}{H} & \textcolor{gregoriocolor}{M} & \textcolor{gregoriocolor}{N} & \textcolor{gregoriocolor}{P} \\
 30 & 1 & 2 & 3 & 4 & 5 & 5 & 6 & 7 & 8 & 9 & 10 \\
\end{tabularx}

\begin{paracol}{2}
\selectlanguage{latin}
\lettrine[lines=2]{S}{ancti} Pii Papæ Décimi, 
 cujus natális dies tertiodécimo Kaléndas Septémbris recensétur.
\switchcolumn
\selectlanguage{english}
\lettrine[lines=2]{P}{ope} St. Pius X, whose birthday is 
 mentioned on the 20th of August.
\switchcolumn*
\selectlanguage{latin}
Corínthi natális sanctæ 
 Phœbes, cujus méminit beátus Apóstolus Paulus ad Romános scribens.
\switchcolumn
\selectlanguage{english}
At Corinth the birthday of St. 
 Phoebe, mentioned by the blessed apostle Paul in his Epistle to the Romans.
\switchcolumn*
\selectlanguage{latin}
Cápuæ sanctórum 
 Mártyrum Aristǽi Epíscopi, et Antoníni púeri.
\switchcolumn
\selectlanguage{english}
At Capua, the holy martyrs Aristaeus, 
 bishop, and Antoninus, a young boy.
\switchcolumn*
\selectlanguage{latin}
Eódem die natális 
 sanctórum Mártyrum Aigúlfi, Abbátis Lirinénsis, et Sociórum ipsíus 
 Monachórum, qui, linguis præcísis oculísque effóssis, gládio obtruncáti sunt.
\switchcolumn
\selectlanguage{english}
Also, the birthday of the holy 
 martyrs Aigulphus, abbot of Lerins, and the monks, his companions, who, 
 after their tongues were cut off and their eyes plucked out, were killed 
 with the sword.
\switchcolumn*
\selectlanguage{latin}
Item sanctórum Mártyrum 
 Zenónis et Charitónis; quorum alter in lebétem liquáti plumbi conjéctus est, 
 alter in ignis fornácem immíssus.
\switchcolumn
\selectlanguage{english}
Also, the holy martyrs Zeno and 
 Chariton. The one was cast into a cauldron of melted lead, the other 
 into a burning furnace.
\switchcolumn*
\selectlanguage{latin}
Córdubæ, in Hispánia, 
 sancti Sándali Mártyris.
\switchcolumn
\selectlanguage{english}
At Cordova in Spain, St. Sandal the 
 martyr.
\switchcolumn*
\selectlanguage{latin}
Aquiléjæ sanctárum 
 Vírginum et Mártyrum Euphémiæ, Dorótheæ, Theclæ et Erásmæ, quæ, sub Neróne 
 Imperatóre et Sebásto Præside, post multa supplícia, gládio cæsæ sunt, et a 
 sancto Hermágora sepúltæ.
\switchcolumn
\selectlanguage{english}
At Aquileia, the holy virgins and 
 martyrs Euphemia, Dorothy, Thecla, and Erasma. Under Nero, after 
 enduring many torments, they were slain with the sword and buried by St. 
 Hermagoras.
\switchcolumn*
\selectlanguage{latin}
Nicomedíæ pássio sanctæ 
 Basilíssæ, Vírginis et Mártyris; quæ, annos novem nata, cum in persecutióne 
 Diocletiáni Imperatóris, sub Alexándro Præside, vérbera, ignes ac béstias 
 divína virtúte superásset, ipsum Præsidem ad Christi fidem convértit, ac 
 tandem extra urbem in oratióne spíritum Deo réddidit.
\switchcolumn
\selectlanguage{english}
At Nicomedia, the passion of St. 
 Basilissa, virgin and martyr, in the persecution of Diocletian, under the 
 governor Alexander. At the age of nine years, after having, through 
 the power of God, overcome scourging, fire, and the beasts—by which she 
 converted the governor to the faith of Christ—she at length gave up her soul 
 to God while at prayer outside the city.
\switchcolumn*
\selectlanguage{latin}
Tulli, in Gállia, 
 sancti Mansuéti, Epíscopi et Confessóris.
\switchcolumn
\selectlanguage{english}
At Toul in France, St. Mansuetus, 
 bishop and confessor.
\switchcolumn*
\selectlanguage{latin}
Medioláni deposítio 
 sancti Auxáni Epíscopi.
\switchcolumn
\selectlanguage{english}
At Milan, the death of St. Auxanus, 
 bishop.
\switchcolumn*
\selectlanguage{latin}
Eódem die sancti 
 Simeónis Stylítæ junióris.
\switchcolumn
\selectlanguage{english}
The same day, St. Simon Stylites the 
 Younger.
\switchcolumn*
\selectlanguage{latin}
Romæ Translátio sanctæ 
 Serápiæ, Vírginis et Mártyris; quæ passa est quarto Kaléndas Augústi.
\switchcolumn
\selectlanguage{english}
At Rome, the translation of St. 
 Serapia, virgin and martyr, who suffered on the 29th of July.
\switchcolumn*
\selectlanguage{latin}
Item Romæ Ordinátio 
 incomparábilis viri sancti Gregórii Magni in Summum Pontíficem; qui, onus 
 illud subíre coáctus, e sublimióri throno clarióribus sanctitátis rádiis in 
 Orbe refúlsit.
\switchcolumn
\selectlanguage{english}
Also at Rome, the raising to the 
 Sovereign Pontificate of St. Gregory the Great. This incomparable man, 
 being forced to take that burden upon himself, sent forth from the exalted 
 throne brighter rays of sanctity upon the world.
\switchcolumn*
\selectlanguage{latin}
\end{paracol}


% ---- martyrology/mart09/mart0904.htm
\needspace{10\baselineskip}
\begin{paracol}{2}
\selectlanguage{latin}
\begin{center}{\color{gregoriocolor} Prídie Nonas Septémbris. 
 Luna\dots\ }\end{center}
\switchcolumn
\selectlanguage{english}
\begin{center}{\color{gregoriocolor} The   Fourth Day of 
 September. The\dots\ Day of the Moon.}\end{center}
\end{paracol}

\noindent\begin{tabularx}{\linewidth}{*{19}{>{\centering\arraybackslash}X}}
 \textcolor{gregoriocolor}{a} & \textcolor{gregoriocolor}{b} & \textcolor{gregoriocolor}{c} & \textcolor{gregoriocolor}{d} & \textcolor{gregoriocolor}{e} & \textcolor{gregoriocolor}{f} & \textcolor{gregoriocolor}{g} & \textcolor{gregoriocolor}{h} & \textcolor{gregoriocolor}{i} & \textcolor{gregoriocolor}{k} & \textcolor{gregoriocolor}{l} & \textcolor{gregoriocolor}{m} & \textcolor{gregoriocolor}{n} & \textcolor{gregoriocolor}{p} & \textcolor{gregoriocolor}{q} & \textcolor{gregoriocolor}{r} & \textcolor{gregoriocolor}{s} & \textcolor{gregoriocolor}{t} & \textcolor{gregoriocolor}{u} \\
 12 & 13 & 14 & 15 & 16 & 17 & 18 & 19 & 20 & 21 & 22 & 23 & 24 & 25 & 26 & 27 & 28 & 29 & 30 \\
\end{tabularx}
\vspace{0.5\baselineskip}
\noindent\begin{tabularx}{\linewidth}{*{12}{>{\centering\arraybackslash}X}}
 \textcolor{gregoriocolor}{A} & \textcolor{gregoriocolor}{B} & \textcolor{gregoriocolor}{C} & \textcolor{gregoriocolor}{D} & \textcolor{gregoriocolor}{E} & F & \textcolor{gregoriocolor}{F} & \textcolor{gregoriocolor}{G} & \textcolor{gregoriocolor}{H} & \textcolor{gregoriocolor}{M} & \textcolor{gregoriocolor}{N} & \textcolor{gregoriocolor}{P} \\
 1 & 2 & 3 & 4 & 5 & 6 & 6 & 7 & 8 & 9 & 10 & 11 \\
\end{tabularx}

\begin{paracol}{2}
\selectlanguage{latin}
\lettrine[lines=2]{I}{n} monte Nebo, terræ 
 Moab, sancti Móysis, legislatóris et Prophétæ.
\switchcolumn
\selectlanguage{english}
\lettrine[lines=2]{O}{n} Mount Nebo, in the land of Moab, 
 the holy lawgiver and prophet Moses.
\switchcolumn*
\selectlanguage{latin}
Neápoli, in Campánia, natális sanctæ Cándidæ, quæ sancto Petro Apóstolo, ad eam urbem veniénti, 
 prima occúrrit, atque, ab eo baptizáta, póstea sancto fine quiévit.
\switchcolumn
\selectlanguage{english}
At Naples in Campania, the birthday 
 of St. Candida, who was the first to meet St. Peter when he came to that 
 city, and being baptized by him afterwards ended her holy life in peace.
\switchcolumn*
\selectlanguage{latin}
Tréviris sancti 
 Marcélli, Epíscopi et Mártyris.
\switchcolumn
\selectlanguage{english}
At Treves, St. Marcellus, bishop and 
 martyr.
\switchcolumn*
\selectlanguage{latin}
Ancyræ, in Galátia, natális sanctórum trium puerórum Mártyrum, id est Rufíni, Silváni et 
 Vitálici.
\switchcolumn
\selectlanguage{english}
At Ancyra in Galatia, the birthday 
 of three saintly boys, Rufinus, Silvanus, and Vitalicus, martyrs.
\switchcolumn*
\selectlanguage{latin}
Eódem die sanctórum 
 Mártyrum Magni, Casti et Máximi.
\switchcolumn
\selectlanguage{english}
On the same day, the holy martyrs 
 Magnus, Castus and Maximus.
\switchcolumn*
\selectlanguage{latin}
Cabillóne, in Gálliis, 
 sancti Marcélli Mártyris, qui, sub Antoníno Imperatóre, cum a Præside Prisco 
 ad profánum convívium fuísset invitátus, et, hujúsmodi épulas éxsecrans, 
 omnes qui áderant, cur idólis deservírent, líbera increpatióne corríperet, 
 ab eódem Præside, inaudíto crudelitátis génere, cíngulo tenus defóssus est 
 in terra; sicque, cum in Dei láudibus tríduo perseverásset, incontaminátum 
 spíritum réddidit.
\switchcolumn
\selectlanguage{english}
At Chalons in France, under Emperor 
 Antoninus, St. Marcellus, martyr. Being invited to a profane banquet 
 by the governor Priscus, he scorned to partake of the meats that were 
 served, and reproved with great freedom all persons present for worshipping 
 idols. For this, with unheard-of cruelty, the same governor had him 
 buried alive up to the waist. After persevering for three days in 
 praising God, he yielded up his undefiled spirit.
\switchcolumn*
\selectlanguage{latin}
Eódem die sanctórum 
 Thamélis, ántea idolórum sacerdótis, et Sociórum Mártyrum, sub Hadriáno 
 Imperatóre.
\switchcolumn
\selectlanguage{english}
On the same day, St. Thamel, 
 previously a pagan priest, and his companions, martyrs under Emperor 
 Hadrian.
\switchcolumn*
\selectlanguage{latin}
Item sanctórum Mártyrum 
 Theódori, Océani, Ammiáni et Juliáni; qui sub Maximiáno Imperatóre, 
 disséctis pédibus in ignem conjécti, martyrium consummárunt.
\switchcolumn
\selectlanguage{english}
Also, the holy martyrs Theodore, 
 Oceanus, Ammian, and Julian, who had their feet cut off, and completed their 
 martyrdom by being thrown into the fire, in the time of Emperor Maximian.
\switchcolumn*
\selectlanguage{latin}
Romæ sancti Bonifátii 
 Primi, Papæ et Confessóris.
\switchcolumn
\selectlanguage{english}
At Rome, St. Boniface I, pope and 
 confessor.
\switchcolumn*
\selectlanguage{latin}
Arímini sancti Maríni 
 Diáconi.
\switchcolumn
\selectlanguage{english}
At Rimini, St. Marinus, deacon.
\switchcolumn*
\selectlanguage{latin}
Panórmi natális sanctæ 
 Rosáliæ, Vírginis Panormitánæ, ex régio Cároli Magni sánguine ortæ; quæ, pro 
 Christi amóre, patérnum principátum aulámque profúgit, et, in móntibus ac 
 spelúncis solitária, cæléstem vitam duxit.
\switchcolumn
\selectlanguage{english}
At Palermo, the birthday of St. 
 Rosalia, virgin, a native of that city, born of the royal blood of 
 Charlemagne. For the love of Christ, she forsook the princely court of 
 her father, and led a saintly life alone in mountains and caverns.
\switchcolumn*
\selectlanguage{latin}
Vitérbii Translátio 
 beátæ Rosæ Vírginis, ex tértio Ordine sancti Francísci, témpore Alexándri 
 Papæ Quarti.
\switchcolumn
\selectlanguage{english}
At Viterbo, the translation of St. 
 Rose the Virgin, of the Third Order of St. Francis, during the pontificate 
 of Pope Alexander IV.
\switchcolumn*
\selectlanguage{latin}
\end{paracol}


% ---- martyrology/mart09/mart0905.htm
\needspace{10\baselineskip}
\begin{paracol}{2}
\selectlanguage{latin}
\begin{center}{\color{gregoriocolor} Nonis Septémbris. 
 Luna\dots\ }\end{center}
\switchcolumn
\selectlanguage{english}
\begin{center}{\color{gregoriocolor} The   Fifth Day of 
 September. The\dots\ Day of the Moon.}\end{center}
\end{paracol}

\noindent\begin{tabularx}{\linewidth}{*{19}{>{\centering\arraybackslash}X}}
 \textcolor{gregoriocolor}{a} & \textcolor{gregoriocolor}{b} & \textcolor{gregoriocolor}{c} & \textcolor{gregoriocolor}{d} & \textcolor{gregoriocolor}{e} & \textcolor{gregoriocolor}{f} & \textcolor{gregoriocolor}{g} & \textcolor{gregoriocolor}{h} & \textcolor{gregoriocolor}{i} & \textcolor{gregoriocolor}{k} & \textcolor{gregoriocolor}{l} & \textcolor{gregoriocolor}{m} & \textcolor{gregoriocolor}{n} & \textcolor{gregoriocolor}{p} & \textcolor{gregoriocolor}{q} & \textcolor{gregoriocolor}{r} & \textcolor{gregoriocolor}{s} & \textcolor{gregoriocolor}{t} & \textcolor{gregoriocolor}{u} \\
 13 & 14 & 15 & 16 & 17 & 18 & 19 & 20 & 21 & 22 & 23 & 24 & 25 & 26 & 27 & 28 & 29 & 30 & 1 \\
\end{tabularx}
\vspace{0.5\baselineskip}
\noindent\begin{tabularx}{\linewidth}{*{12}{>{\centering\arraybackslash}X}}
 \textcolor{gregoriocolor}{A} & \textcolor{gregoriocolor}{B} & \textcolor{gregoriocolor}{C} & \textcolor{gregoriocolor}{D} & \textcolor{gregoriocolor}{E} & F & \textcolor{gregoriocolor}{F} & \textcolor{gregoriocolor}{G} & \textcolor{gregoriocolor}{H} & \textcolor{gregoriocolor}{M} & \textcolor{gregoriocolor}{N} & \textcolor{gregoriocolor}{P} \\
 2 & 3 & 4 & 5 & 6 & 7 & 7 & 8 & 9 & 10 & 11 & 12 \\
\end{tabularx}

\begin{paracol}{2}
\selectlanguage{latin}
\lettrine[lines=2]{S}{ancti} Lauréntii 
 Justiniáni, primi Patriárchæ Venetiárum et Confessóris, qui pontificálem 
 Cáthedram hac die invítus ascéndit, et sexto Idus Januárii obdormívit in 
 Dómino.
\switchcolumn
\selectlanguage{english}
\lettrine[lines=2]{S}{aint} Lawrence Justinian, first 
 patriarch of Venice and confessor, who on this day unwillingly ascended the 
 episcopal throne. His birthday is the 8th of January.
\switchcolumn*
\selectlanguage{latin}
Romæ, in suburbáno, 
 beáti Victoríni, Epíscopi et Mártyris; qui, sanctitáte et miráculis clarus, 
 sacerdótium Amiternínæ urbis, totíus pópuli electióne, adéptus est. 
 Póstmodum, sub Nerva Trajáno, apud Cutílias, ubi puténtes et sulphúreæ 
 emánant aquæ, cum áliis Dei servis, relegátus, ab Aureliáno Júdice jussus 
 est suspéndi cápite deórsum; idque cum per tríduum pro nómine Christi passus 
 fuísset, tandem, glorióse coronátus, victor migrávit ad Dóminum. Ejus 
 corpus rapuérunt Christiáni, et honorífica sepultúra Amitérni, in Vestínis, 
 condidérunt.
\switchcolumn
\selectlanguage{english}
In the suburbs of Rome, blessed 
 Victorinus, bishop and martyr, in the time of Nerva Trajan. Being 
 renowned for sanctity and miracles, he was elected bishop of Amiterno by the 
 whole populace, but afterwards he was banished, with other servants of God, 
 to Contigliano, where fetid sulphurous waters spring forth, and was 
 suspended with his head downward by order of the judge Aurelian. 
 Having for the name of Christ endured this torment for three days, he was 
 gloriously crowned and went victoriously to our Lord. His body was 
 taken away by the Christians and buried with due honours at Amiterno.
\switchcolumn*
\selectlanguage{latin}
Constantinópoli 
 sanctórum Mártyrum Urbáni, Theodóri, Menedémi, et Sociórum septuagínta 
 septem ex órdine ecclesiástico; qui a Valénte Imperatóre, pro fide cathólica, 
 in navígio impósiti, jussi sunt in mari combúri.
\switchcolumn
\selectlanguage{english}
At Constantinople, the holy martyrs 
 Urbanus, Theodore, Menedemus, and their companions of ecclesiastical rank, 
 seventy-seven in number, who were put in a ship by the command of Emperor 
 Valens, and burned on the sea for the Catholic faith.
\switchcolumn*
\selectlanguage{latin}
In Portu Románo pássio 
 sancti Herculáni mílitis, qui, sub Gallo Imperatóre, ob Christi fidem cæsus 
 flagris et cápite obtruncátus est.
\switchcolumn
\selectlanguage{english}
At Porto, the birthday of St. 
 Herculanus, martyr, who was scourged and beheaded in the reign of Emperor 
 Gallus because of the Christian faith.
\switchcolumn*
\selectlanguage{latin}
Cápuæ sanctórum 
 Mártyrum Quínctii, Arcóntii et Donáti.
\switchcolumn
\selectlanguage{english}
At Capua, the holy martyrs Quinctus, 
 Arcontius, and Donatus.
\switchcolumn*
\selectlanguage{latin}
Eódem die sancti Rómuli, 
 qui, cum esset Trajáni aulæ præféctus ac sævítiam Imperatóris in Christiános 
 detestarétur, cæsus est virgis, et cápite truncátus.
\switchcolumn
\selectlanguage{english}
On the same day, St. Romulus, 
 prefect of Trajan's court. For reproving the cruelty of the emperor 
 towards Christians, he was scourged with rods and beheaded.
\switchcolumn*
\selectlanguage{latin}
Melitínæ, in Arménia, pássio sanctórum mílitum Eudóxii, Zenónis, Macárii, et Sociórum mille centum 
 et quátuor; qui, cum abjecíssent milítiæ cíngulum, in persecutióne 
 Diocletiáni, pro Christi confessióne necáti sunt.
\switchcolumn
\selectlanguage{english}
At Melitine in Armenia, during the 
 persecution of Diocletian, the martyrdom of the holy soldiers Eudoxius, 
 Zeno, Macarius, and their companions to the number of eleven hundred and 
 four, who threw away their military belts and were put to death for the 
 confession of Christ.
\switchcolumn*
\selectlanguage{latin}
In pago Tarvanénsi, 
 monastério Sithinénsi, in Gállia, sancti Bertíni Abbátis.
\switchcolumn
\selectlanguage{english}
In the neighbourhood of Terouanne, 
 in the monastery of Sithiu, in France, St. Bertinus, abbot.
\switchcolumn*
\selectlanguage{latin}
Toléti, in Hispánia, 
 sanctæ Obdúliæ Vírginis.
\switchcolumn
\selectlanguage{english}
At Toledo in Spain, St. Obdulia, 
 virgin.
\switchcolumn*
\selectlanguage{latin}
\end{paracol}


% ---- martyrology/mart09/mart0906.htm
\needspace{10\baselineskip}
\begin{paracol}{2}
\selectlanguage{latin}
\begin{center}{\color{gregoriocolor} Octávo Idus Septémbris. 
 Luna\dots\ }\end{center}
\switchcolumn
\selectlanguage{english}
\begin{center}{\color{gregoriocolor} The   Sixth Day of 
 September. The\dots\ Day of the Moon.}\end{center}
\end{paracol}

\noindent\begin{tabularx}{\linewidth}{*{19}{>{\centering\arraybackslash}X}}
 \textcolor{gregoriocolor}{a} & \textcolor{gregoriocolor}{b} & \textcolor{gregoriocolor}{c} & \textcolor{gregoriocolor}{d} & \textcolor{gregoriocolor}{e} & \textcolor{gregoriocolor}{f} & \textcolor{gregoriocolor}{g} & \textcolor{gregoriocolor}{h} & \textcolor{gregoriocolor}{i} & \textcolor{gregoriocolor}{k} & \textcolor{gregoriocolor}{l} & \textcolor{gregoriocolor}{m} & \textcolor{gregoriocolor}{n} & \textcolor{gregoriocolor}{p} & \textcolor{gregoriocolor}{q} & \textcolor{gregoriocolor}{r} & \textcolor{gregoriocolor}{s} & \textcolor{gregoriocolor}{t} & \textcolor{gregoriocolor}{u} \\
 14 & 15 & 16 & 17 & 18 & 19 & 20 & 21 & 22 & 23 & 24 & 25 & 26 & 27 & 28 & 29 & 30 & 1 & 2 \\
\end{tabularx}
\vspace{0.5\baselineskip}
\noindent\begin{tabularx}{\linewidth}{*{12}{>{\centering\arraybackslash}X}}
 \textcolor{gregoriocolor}{A} & \textcolor{gregoriocolor}{B} & \textcolor{gregoriocolor}{C} & \textcolor{gregoriocolor}{D} & \textcolor{gregoriocolor}{E} & F & \textcolor{gregoriocolor}{F} & \textcolor{gregoriocolor}{G} & \textcolor{gregoriocolor}{H} & \textcolor{gregoriocolor}{M} & \textcolor{gregoriocolor}{N} & \textcolor{gregoriocolor}{P} \\
 3 & 4 & 5 & 6 & 7 & 8 & 8 & 9 & 10 & 11 & 12 & 13 \\
\end{tabularx}

\begin{paracol}{2}
\selectlanguage{latin}
\lettrine[lines=2]{I}{n} Palæstína sancti 
 Zacharíæ Prophétæ, qui, de Chaldǽa senex in pátriam revérsus, ibíque 
 defúnctus, juxta Aggǽum Prophétam cónditus jacet.
\switchcolumn
\selectlanguage{english}
\lettrine[lines=2]{I}{n} Palestine, the prophet Zachary, 
 who returned in his old age from Chaldea to his own country, and lies buried 
 near the prophet Aggeus.
\switchcolumn*
\selectlanguage{latin}
In Hellespónto sancti 
 Onesíphori, Apostolórum discípuli, cujus méminit sanctus Paulus ad Timótheum 
 scribens. Ipse autem Onesíphorus ibídem, una cum sancto Porphyrio, 
 jussu Hadriáni Procónsulis ácriter verberátus et a ferócibus raptátus equis, 
 spíritum Deo réddidit.
\switchcolumn
\selectlanguage{english}
In the Hellespont, St. Onesiphorus, 
 disciple of the apostles, of whom St. Paul speaks in his Letter to Timothy. 
 He was severely scourged with St. Porphyry, by order of the proconsul 
 Adrian, and being dragged by wild horses, gave up his soul unto God.
\switchcolumn*
\selectlanguage{latin}
In Africa sanctórum 
 Episcopórum Donatiáni, Præsidii, Mansuéti, Germáni et Fúsculi; qui, in 
 persecutióne Wandálica, jussu Ariáni Regis Hunneríci, pro assertióne 
 cathólicæ veritátis, fústibus diríssime cæsi et exsílio elimináti sunt. 
 Inter eos étiam erat Epíscopus, nómine Lætus, strénuus atque doctíssimus vir, 
 qui, post diutúrnos cárceris squalóres, incéndio concremátus est.
\switchcolumn
\selectlanguage{english}
In Africa, in the persecution of the 
 Vandals, the holy bishops Donatian, Praesidius, Mansuetus, Germanus, and 
 Fusculus, who were most cruelly scourged and sent into exile by order of the 
 Arian king Hunneric, because they proclaimed the Catholic truth. Among 
 them was one named Laetus, also a bishop, a courageous and very learned man, 
 who was burned alive after a long imprisonment in a loathsome dungeon.
\switchcolumn*
\selectlanguage{latin}
Alexandríæ pássio 
 sanctórum Mártyrum Fausti Presbyteri, Macárii et Sociórum decem; qui, sub 
 Décio Imperatóre et Valério Præside, pro Christi nómine, abscíssis 
 cervícibus, martyrium complevérunt.
\switchcolumn
\selectlanguage{english}
At Alexandria, in the time of 
 Emperor Decius and the governor Valerius, the holy martyrs Faustus, a 
 priest, Macarius, and ten companions, who received the martyr's crown by 
 being beheaded for the name of Christ.
\switchcolumn*
\selectlanguage{latin}
In Cappadócia sanctórum 
 Mártyrum Cóttidi Diáconi, Eugénii et Sociórum.
\switchcolumn
\selectlanguage{english}
In Cappadocia, the holy martyrs 
 Cottidus, deacon, Eugene, and their companions.
\switchcolumn*
\selectlanguage{latin}
Verónæ sancti Petrónii, 
 Epíscopi et Confessóris.
\switchcolumn
\selectlanguage{english}
At Verona, St. Petronius, bishop and 
 confessor.
\switchcolumn*
\selectlanguage{latin}
Romæ sancti Eleuthérii 
 Abbátis, qui Dei servus fuit, atque (ut sanctus Gregórius Papa scribit) 
 oratióne et lácrimis mórtuum suscitávit.
\switchcolumn
\selectlanguage{english}
At Rome, the holy abbot Eleutherius, 
 a servant of God, who, according to the testimony of Pope St. Gregory, 
 raised a dead man to life by his prayers and tears.
\switchcolumn*
\selectlanguage{latin}
\end{paracol}


% ---- martyrology/mart09/mart0907.htm
\needspace{10\baselineskip}
\begin{paracol}{2}
\selectlanguage{latin}
\begin{center}{\color{gregoriocolor} Séptimo Idus Septémbris. 
 Luna\dots\ }\end{center}
\switchcolumn
\selectlanguage{english}
\begin{center}{\color{gregoriocolor} The   Seventh Day of 
 September. The\dots\ Day of the Moon.}\end{center}
\end{paracol}

\noindent\begin{tabularx}{\linewidth}{*{19}{>{\centering\arraybackslash}X}}
 \textcolor{gregoriocolor}{a} & \textcolor{gregoriocolor}{b} & \textcolor{gregoriocolor}{c} & \textcolor{gregoriocolor}{d} & \textcolor{gregoriocolor}{e} & \textcolor{gregoriocolor}{f} & \textcolor{gregoriocolor}{g} & \textcolor{gregoriocolor}{h} & \textcolor{gregoriocolor}{i} & \textcolor{gregoriocolor}{k} & \textcolor{gregoriocolor}{l} & \textcolor{gregoriocolor}{m} & \textcolor{gregoriocolor}{n} & \textcolor{gregoriocolor}{p} & \textcolor{gregoriocolor}{q} & \textcolor{gregoriocolor}{r} & \textcolor{gregoriocolor}{s} & \textcolor{gregoriocolor}{t} & \textcolor{gregoriocolor}{u} \\
 15 & 16 & 17 & 18 & 19 & 20 & 21 & 22 & 23 & 24 & 25 & 26 & 27 & 28 & 29 & 30 & 1 & 2 & 3 \\
\end{tabularx}
\vspace{0.5\baselineskip}
\noindent\begin{tabularx}{\linewidth}{*{12}{>{\centering\arraybackslash}X}}
 \textcolor{gregoriocolor}{A} & \textcolor{gregoriocolor}{B} & \textcolor{gregoriocolor}{C} & \textcolor{gregoriocolor}{D} & \textcolor{gregoriocolor}{E} & F & \textcolor{gregoriocolor}{F} & \textcolor{gregoriocolor}{G} & \textcolor{gregoriocolor}{H} & \textcolor{gregoriocolor}{M} & \textcolor{gregoriocolor}{N} & \textcolor{gregoriocolor}{P} \\
 4 & 5 & 6 & 7 & 8 & 9 & 9 & 10 & 11 & 12 & 13 & 14 \\
\end{tabularx}

\begin{paracol}{2}
\selectlanguage{latin}
\lettrine[lines=2]{T}{recis,} in Gállia, 
 sancti Nemórii Diáconi, et Sociórum Mártyrum, quos Attila, Rex Hunnórum, 
 interfécit.
\switchcolumn
\selectlanguage{english}
\lettrine[lines=2]{A}{t} Troyes, St. Nemorius, deacon, and 
 his companions, all martyrs, who were slain by Attila, king of the Huns.
\switchcolumn*
\selectlanguage{latin}
Nicomedíæ natális beáti 
 Joánnis Mártyris, qui, cum vidéret crudélia advérsus Christiános edícta in 
 foro pendére, hinc fídei ardóre accénsus, injécta manu, illa detráxit atque 
 discérpsit. Cumque hoc relátum esset Diocletiáno et Maximiáno Augústis, 
 in eádem urbe constitútis, ómnia suppliciórum génera in eum experíri 
 jussérunt; quæ vir nobilíssimus tanta vultus ac spíritus alacritáte pértulit, 
 ut ne tristis quidem pro his vidéri potúerit.
\switchcolumn
\selectlanguage{english}
At Nicomedia, the birthday of the 
 blessed martyr John, who upon seeing the cruel edicts against Christians, 
 posted in the public square, and being inflamed with an ardent faith, 
 reached out his hand, took them away and tore them up. This was 
 related to Emperors Diocletian and Maximian, then residing in the city, who 
 gave orders that he should be subjected to many kinds of torments. The 
 noble champion bore them with such cheerfulness of spirit as not to shew on 
 his countenance the least trace of pain or grief.
\switchcolumn*
\selectlanguage{latin}
Cæsaréæ, in Cappadócia, sancti Eupsychii Mártyris, qui, sub Hadriáno Imperatóre, accusátus quod 
 Christiánus esset, in cárcerem conjéctus est; et, paulo post inde emíssus, 
 patrimónium statim véndidit, et prétium partim paupéribus, partim 
 accusatóribus, tamquam benefactóribus, distríbuit. Sed, íterum 
 comprehénsus, atque, cum sacrificáre nollet idólis, sævíssime dilaniátus et 
 gládio confóssus, sub Saprítio Júdice martyrium consummávit.
\switchcolumn
\selectlanguage{english}
At Caesarea in Cappadocia, in the 
 time of Emperor Adrian, St. Eupsychius, martyr, who was accused of 
 professing Christianity and who was cast into prison. Having been 
 released shortly after, he immediately sold his inheritance, and distributed 
 the price of it partly to his accusers, whom he regarded as his benefactors. 
 But being again arrested, under the judge Sapritius, he was tortured, 
 pierced through with a sword, and thus completed his martyrdom.
\switchcolumn*
\selectlanguage{latin}
Pompejópoli, in Cilícia, sancti Sozóntis Mártyris, qui, sub Maximiáno Imperatóre, in ignem injéctus, 
 réddidit spíritum.
\switchcolumn
\selectlanguage{english}
At Pompeiopolis in Cilicia, in the 
 time of Emperor Maximian, St. Sozon, a martyr who was thrown into the fire 
 and yielded up his spirit.
\switchcolumn*
\selectlanguage{latin}
Aquiléjæ sancti 
 Anastásii Mártyris.
\switchcolumn
\selectlanguage{english}
At Aquileia, St. Anastasius, martyr.
\switchcolumn*
\selectlanguage{latin}
Apud Aléxiam véterem, 
 in território Augustodunénsi, sanctæ Regínæ, Vírginis et Mártyris; quæ, sub 
 Procónsule Olybrio, cárceris et equúlei ac lampadárum afflícta supplíciis, 
 demum, cápite damnáta, migrávit ad Sponsum.
\switchcolumn
\selectlanguage{english}
In the diocese of Autun, under the 
 proconsul Olybrius, St. Regina, virgin and martyr. After having 
 suffered imprisonment, the rack, and burning with torches, she was finally 
 condemned to capital punishment, and so went to her spouse.
\switchcolumn*
\selectlanguage{latin}
Aureliánis, in Gállia, 
 deposítio sancti Evótii Epíscopi, qui, primo Románæ Ecclésiæ Subdiáconus, 
 dehinc divíno múnere per colúmbam designátus est Póntifex præfátæ urbis.
\switchcolumn
\selectlanguage{english}
At Orleans in France, the departure 
 from this life of the holy bishop Evortius, who was first a subdeacon of the 
 Roman Church, and afterwards, through a divine favour, was designated by a 
 dove as bishop of that city.
\switchcolumn*
\selectlanguage{latin}
In Gállia sancti 
 Augustális, Epíscopi et Confessóris.
\switchcolumn
\selectlanguage{english}
In France, St. Augustalis, bishop 
 and confessor.
\switchcolumn*
\selectlanguage{latin}
Cápuæ sancti Pámphili 
 Epíscopi.
\switchcolumn
\selectlanguage{english}
At Capua, St. Pamphilus, bishop.
\switchcolumn*
\selectlanguage{latin}
In território Parisiénsi sancti Clodoáldi, Presbyteri et Confessóris.
\switchcolumn
\selectlanguage{english}
In the territory of Paris, St. 
 Cloud, priest and confessor.
\switchcolumn*
\selectlanguage{latin}
\end{paracol}


% ---- martyrology/mart09/mart0908.htm
\needspace{10\baselineskip}
\begin{paracol}{2}
\selectlanguage{latin}
\begin{center}{\color{gregoriocolor} Sexto Idus Septémbris. 
 Luna\dots\ }\end{center}
\switchcolumn
\selectlanguage{english}
\begin{center}{\color{gregoriocolor} The   Eighth Day of 
 September. The\dots\ Day of the Moon.}\end{center}
\end{paracol}

\noindent\begin{tabularx}{\linewidth}{*{19}{>{\centering\arraybackslash}X}}
 \textcolor{gregoriocolor}{a} & \textcolor{gregoriocolor}{b} & \textcolor{gregoriocolor}{c} & \textcolor{gregoriocolor}{d} & \textcolor{gregoriocolor}{e} & \textcolor{gregoriocolor}{f} & \textcolor{gregoriocolor}{g} & \textcolor{gregoriocolor}{h} & \textcolor{gregoriocolor}{i} & \textcolor{gregoriocolor}{k} & \textcolor{gregoriocolor}{l} & \textcolor{gregoriocolor}{m} & \textcolor{gregoriocolor}{n} & \textcolor{gregoriocolor}{p} & \textcolor{gregoriocolor}{q} & \textcolor{gregoriocolor}{r} & \textcolor{gregoriocolor}{s} & \textcolor{gregoriocolor}{t} & \textcolor{gregoriocolor}{u} \\
 16 & 17 & 18 & 19 & 20 & 21 & 22 & 23 & 24 & 25 & 26 & 27 & 28 & 29 & 30 & 1 & 2 & 3 & 4 \\
\end{tabularx}
\vspace{0.5\baselineskip}
\noindent\begin{tabularx}{\linewidth}{*{12}{>{\centering\arraybackslash}X}}
 \textcolor{gregoriocolor}{A} & \textcolor{gregoriocolor}{B} & \textcolor{gregoriocolor}{C} & \textcolor{gregoriocolor}{D} & \textcolor{gregoriocolor}{E} & F & \textcolor{gregoriocolor}{F} & \textcolor{gregoriocolor}{G} & \textcolor{gregoriocolor}{H} & \textcolor{gregoriocolor}{M} & \textcolor{gregoriocolor}{N} & \textcolor{gregoriocolor}{P} \\
 5 & 6 & 7 & 8 & 9 & 10 & 10 & 11 & 12 & 13 & 14 & 15 \\
\end{tabularx}

\begin{paracol}{2}
\selectlanguage{latin}
\lettrine[lines=2]{N}{atívitas} beatíssimæ 
 semper Vírginis Genitrícis Dei Maríæ.
\switchcolumn
\selectlanguage{english}
\lettrine[lines=2]{T}{he} Nativity of the most Blessed and 
 ever Virgin Mary, Mother of God.
\switchcolumn*
\selectlanguage{latin}
Sancti Hadriáni 
 Mártyris; cujus dies natális quarto Nonas Mártii recensétur, sed festívitas 
 hac die, qua sacrum ejus corpus Romam translátum fuit, potíssime celebrátur.
\switchcolumn
\selectlanguage{english}
St. Hadrian, martyr, whose birthday 
 is on the 4th of March. His feast, however, is observed today, the day 
 on which his holy body was translated to Rome.
\switchcolumn*
\selectlanguage{latin}
Valéntiæ, in Hispánia 
 Tarraconénsi, natális sancti Thomæ a Villa Nova, ex Eremitárum sancti 
 Augustíni Ordine, Epíscopi et Confessóris, insígnis propter flagrántem in 
 páuperes caritátem; qui ab Alexándro Papa Séptimo in Sanctórum númerum 
 adscríptus est. Ipsíus autem festum décimo Kaléndas Octóbris 
 celebrátur.
\switchcolumn
\selectlanguage{english}
At Valencia in Spain, the birthday 
 of St. Thomas of Villanova, bishop and confessor, of the order of the 
 Hermits of St. Augustine, distinguished by his ardent love for the poor. 
 He was inscribed among the saints by Pope Alexander VII, and his festival is 
 observed on the 22nd of this month.
\switchcolumn*
\selectlanguage{latin}
Alexandríæ sanctórum 
 Mártyrum Ammónis, Theóphili, Neotérii et aliórum vigínti duórum.
\switchcolumn
\selectlanguage{english}
At Alexandria, the holy martyrs 
 Ammon, Theophilus, Neoterius, and twenty-two others.
\switchcolumn*
\selectlanguage{latin}
Antiochíæ sanctórum 
 Timóthei et Fausti Mártyrum.
\switchcolumn
\selectlanguage{english}
At Antioch, the Saints Timothy and 
 Faustus, martyrs.
\switchcolumn*
\selectlanguage{latin}
Gazæ, in Palæstína, 
 sanctórum Mártyrum fratrum Eusébii, Néstabi et Zenónis; qui, témpore Juliáni 
 Apóstatæ, irruénte in eos turba Gentílium, discérpti atque necáti sunt.
\switchcolumn
\selectlanguage{english}
At Gaza in Palestine, in the time of 
 Julian the Apostate, the holy martyrs Eusebius, Nestabus, and Zeno, 
 brothers, who were torn to pieces by a multitude of pagans that rushed upon 
 them.
\switchcolumn*
\selectlanguage{latin}
Ibídem sancti Néstoris 
 Mártyris, qui sub eódem Juliáno, ab iísdem Gentílibus furéntibus sævíssime 
 cruciátus, emísit spíritum.
\switchcolumn
\selectlanguage{english}
In the same place, and under the 
 same Julian, St. Nestor, martyr, who breathed his last after being most 
 cruelly tortured by the same furious heathen.
\switchcolumn*
\selectlanguage{latin}
Romæ sancti Sérgii 
 Primi, Papæ et Confessóris.
\switchcolumn
\selectlanguage{english}
At Rome, St. Sergius I, pope and 
 confessor.
\switchcolumn*
\selectlanguage{latin}
Frisíngæ sancti 
 Corbiniáni, qui fuit primus ejúsdem civitátis Epíscopus. Hic, a sancto 
 Gregório Secúndo Pontífice ordinátus et ad prædicándum Evangélium missus, 
 úberes fructus in Gállia et Germánia rétulit, ac demum, virtútibus et 
 miráculis clarus, in pace quiévit.
\switchcolumn
\selectlanguage{english}
At Freisingen, St. Corbinian, first 
 bishop of that city. Being consecrated by Pope Gregory II and sent to 
 preach the Gospel, he reaped abundant fruits in France and Germany, and 
 finally rested in peace, renowned for virtues and miracles.
\switchcolumn*
\selectlanguage{latin}
Carthágine nova, in 
 Ameríca meridionáli, sancti Petri Claver, Sacerdótis e Societáte Jesu et 
 Confessóris; qui mira sui abnegatióne et exímia caritáte Nigrítis in 
 servitútem abdúctis, annos ámplius quadragínta, óperam impéndens, tercénta 
 fere eórum míllia Christo sua ipse manu regenerávit; et a Leóne Décimo 
 tértio, Pontífice Máximo, in Sanctórum númerum relátus est, ad dein étiam 
 cæléstis Patrónus peculiáris sacrárum ad Nigrítas Missiónum constitútus et 
 declarátus.
\switchcolumn
\selectlanguage{english}
In New Carthage in South America, 
 St. Peter Claver, priest of the Society of Jesus and confessor. He 
 devoted more than forty years with wonderful mortification and exceeding 
 charity to the service of the Negroes who had been enslaved, and with his 
 own hand baptized in Christ almost three hundred thousand of them. 
 Pope Leo XIII added him to the list of the saints, and then declared him to 
 be the special heavenly patron of all missions for the Negroes.
\switchcolumn*
\selectlanguage{latin}
\end{paracol}


% ---- martyrology/mart09/mart0909.htm
\needspace{10\baselineskip}
\begin{paracol}{2}
\selectlanguage{latin}
\begin{center}{\color{gregoriocolor} Quinto Idus Septémbris. 
 Luna\dots\ }\end{center}
\switchcolumn
\selectlanguage{english}
\begin{center}{\color{gregoriocolor} The   Ninth Day of 
 September. The\dots\ Day of the Moon.}\end{center}
\end{paracol}

\noindent\begin{tabularx}{\linewidth}{*{19}{>{\centering\arraybackslash}X}}
 \textcolor{gregoriocolor}{a} & \textcolor{gregoriocolor}{b} & \textcolor{gregoriocolor}{c} & \textcolor{gregoriocolor}{d} & \textcolor{gregoriocolor}{e} & \textcolor{gregoriocolor}{f} & \textcolor{gregoriocolor}{g} & \textcolor{gregoriocolor}{h} & \textcolor{gregoriocolor}{i} & \textcolor{gregoriocolor}{k} & \textcolor{gregoriocolor}{l} & \textcolor{gregoriocolor}{m} & \textcolor{gregoriocolor}{n} & \textcolor{gregoriocolor}{p} & \textcolor{gregoriocolor}{q} & \textcolor{gregoriocolor}{r} & \textcolor{gregoriocolor}{s} & \textcolor{gregoriocolor}{t} & \textcolor{gregoriocolor}{u} \\
 17 & 18 & 19 & 20 & 21 & 22 & 23 & 24 & 25 & 26 & 27 & 28 & 29 & 30 & 1 & 2 & 3 & 4 & 5 \\
\end{tabularx}
\vspace{0.5\baselineskip}
\noindent\begin{tabularx}{\linewidth}{*{12}{>{\centering\arraybackslash}X}}
 \textcolor{gregoriocolor}{A} & \textcolor{gregoriocolor}{B} & \textcolor{gregoriocolor}{C} & \textcolor{gregoriocolor}{D} & \textcolor{gregoriocolor}{E} & F & \textcolor{gregoriocolor}{F} & \textcolor{gregoriocolor}{G} & \textcolor{gregoriocolor}{H} & \textcolor{gregoriocolor}{M} & \textcolor{gregoriocolor}{N} & \textcolor{gregoriocolor}{P} \\
 6 & 7 & 8 & 9 & 10 & 11 & 11 & 12 & 13 & 14 & 15 & 16 \\
\end{tabularx}

\begin{paracol}{2}
\selectlanguage{latin}
\lettrine[lines=2]{N}{icomedíæ} pássio 
 sanctórum Mártyrum Doróthei et Gorgónii, qui, cum essent apud Diocletiánum 
 Augústum honóres amplíssimos consecúti, et persecutiónem, quam ille 
 Christiánis inferébat, detestaréntur, præsénte eo, jussi sunt primo appéndi, 
 et flagris toto córpore laniári; deínde, viscéribus pelle nudátis, acéto et 
 sale perfúndi, sicque assári in cratícula; atque ad últimum, láqueo necári. 
 Interjécto autem témpore, beáti Gorgónii corpus Romam delátus fuit, ac via 
 Latína pósitum, et inde ad Basílicam sancti Petri translátum.
\switchcolumn
\selectlanguage{english}
\lettrine[lines=2]{A}{t} Nicomedia, the holy martyrs 
 Dorothy and Gorgonius. The greatest honours had been conferred on them 
 by Emperor Diocletian, but as they detested the cruelty which he exercised 
 against the Christians, they were by his order hung up in his presence and 
 lacerated with whips. Then, having the skin torn off from their bodies 
 and vinegar and salt poured over them, they were burned on a gridiron, and 
 finally strangled. After some time the body of blessed Gorgonius was 
 brought to Rome and deposited on the Latin Way. From there it was 
 transferred to the basilica of St. Peter.
\switchcolumn*
\selectlanguage{latin}
In Sabínis, trigésimo 
 ab Urbe milliário, sanctórum Mártyrum Hyacínthi, Alexándri et Tibúrtii.
\switchcolumn
\selectlanguage{english}
Among the Sabines, thirty miles from 
 Rome, the holy martyrs Hyacinth, Alexander, and Tiburtius.
\switchcolumn*
\selectlanguage{latin}
Sebáste, in Arménia, sancti Severiáni, qui, cum Licínii Imperatóris esset miles, et Quadragínta 
 Mártyres in cárcere deténtos frequens visitáret, hinc, jussu Lysiæ Præsidis, 
 saxo ad pedes ligáto suspénsus est, ac, verbéribus cæsus et flagris laniátus, 
 in torméntis réddidit spíritum.
\switchcolumn
\selectlanguage{english}
At Sebaste in Armenia, St. Severian, 
 a soldier of Emperor Licinius. For frequently visiting the Forty 
 Martyrs in prison, he was suspended in the air with a stone tied to his feet 
 by order of the governor Lysias, and being scourged and torn with whips, 
 yielded up his soul in the midst of his torments.
\switchcolumn*
\selectlanguage{latin}
Eódem die pássio sancti 
 Stratónis, qui pro Christo, ad duas árbores ligátus atque discérptus, 
 martyrium consummávit.
\switchcolumn
\selectlanguage{english}
On the same day, St. Strato, who 
 ended his martyrdom for Christ by being tied to two trees and torn asunder.
\switchcolumn*
\selectlanguage{latin}
Item sanctórum Mártyrum 
 Rufíni et Rufiniáni fratrum.
\switchcolumn
\selectlanguage{english}
Also, the holy martyrs Rufinus and 
 Rufinian, brothers.
\switchcolumn*
\selectlanguage{latin}
In territorio 
 Tarvanénsi, in Gállia, sancti Audomári Epíscopi.
\switchcolumn
\selectlanguage{english}
In the territory of Terouanne, St. 
 Omer, bishop.
\switchcolumn*
\selectlanguage{latin}
In monastério Cluanénsi, 
 in Hibérnia, sancti Queráni, Presbyteri et Abbátis.
\switchcolumn
\selectlanguage{english}
In the monastery of Clonmacnoise in 
 Ireland, St. Kiaran, priest and abbot.
\switchcolumn*
\selectlanguage{latin}
\end{paracol}


% ---- martyrology/mart09/mart0910.htm
\needspace{10\baselineskip}
\begin{paracol}{2}
\selectlanguage{latin}
\begin{center}{\color{gregoriocolor} Quarto Idus Septémbris. 
 Luna\dots\ }\end{center}
\switchcolumn
\selectlanguage{english}
\begin{center}{\color{gregoriocolor} The   Tenth Day of 
 September. The\dots\ Day of the Moon.}\end{center}
\end{paracol}

\noindent\begin{tabularx}{\linewidth}{*{19}{>{\centering\arraybackslash}X}}
 \textcolor{gregoriocolor}{a} & \textcolor{gregoriocolor}{b} & \textcolor{gregoriocolor}{c} & \textcolor{gregoriocolor}{d} & \textcolor{gregoriocolor}{e} & \textcolor{gregoriocolor}{f} & \textcolor{gregoriocolor}{g} & \textcolor{gregoriocolor}{h} & \textcolor{gregoriocolor}{i} & \textcolor{gregoriocolor}{k} & \textcolor{gregoriocolor}{l} & \textcolor{gregoriocolor}{m} & \textcolor{gregoriocolor}{n} & \textcolor{gregoriocolor}{p} & \textcolor{gregoriocolor}{q} & \textcolor{gregoriocolor}{r} & \textcolor{gregoriocolor}{s} & \textcolor{gregoriocolor}{t} & \textcolor{gregoriocolor}{u} \\
 18 & 19 & 20 & 21 & 22 & 23 & 24 & 25 & 26 & 27 & 28 & 29 & 30 & 1 & 2 & 3 & 4 & 5 & 6 \\
\end{tabularx}
\vspace{0.5\baselineskip}
\noindent\begin{tabularx}{\linewidth}{*{12}{>{\centering\arraybackslash}X}}
 \textcolor{gregoriocolor}{A} & \textcolor{gregoriocolor}{B} & \textcolor{gregoriocolor}{C} & \textcolor{gregoriocolor}{D} & \textcolor{gregoriocolor}{E} & F & \textcolor{gregoriocolor}{F} & \textcolor{gregoriocolor}{G} & \textcolor{gregoriocolor}{H} & \textcolor{gregoriocolor}{M} & \textcolor{gregoriocolor}{N} & \textcolor{gregoriocolor}{P} \\
 7 & 8 & 9 & 10 & 11 & 12 & 12 & 13 & 14 & 15 & 16 & 17 \\
\end{tabularx}

\begin{paracol}{2}
\selectlanguage{latin}
\lettrine[lines=2]{T}{olentíni,} in Picéno, deposítio sancti Nicolái Confessóris, ex Ordine Eremitárum sancti Augustíni.
\switchcolumn
\selectlanguage{english}
\lettrine[lines=2]{A}{t} Tolentino in Piceno, the 
 departure from this life of St. Nicholas, confessor, of the order of the 
 Hermits of St. Augustine.
\switchcolumn*
\selectlanguage{latin}
In Africa natális 
 sanctórum Episcopórum Nemesiáni, Felícis, Lúcii, altérius item Felícis, 
 Littéi, Polyáni, Victóris, Jadéris, Datívi et aliórum; qui, sub Valeriáno et 
 Galliéno, exsurgénte persecutiónis rábie, ad primam confessiónis Christi 
 constántiam gráviter fústibus cæsi sunt, deínde, compédibus vincti et ad 
 fodiénda metálla deputáti, gloriósæ confessiónis agónem consummárunt.
\switchcolumn
\selectlanguage{english}
In Africa, the birthday of the holy 
 bishops Nemesian, Felix, Lucius, another Felix, Litteus, Polyanus, Victor, 
 Jader, Dativus, and others. Because a violent persecution was breaking 
 out under Valerian and Gallienus, they were at their first courageous 
 confession of Christ beaten with rods, placed in irons, and sent to dig in 
 the metal mines where they completed their combat with a glorious 
 confession.
\switchcolumn*
\selectlanguage{latin}
Leódii, in Bélgio, 
 sancti Theodárdi, Epíscopi et Mártyris; qui ánimam suam pósuit pro óvibus 
 suis, et miraculórum signis post mortem illúxit.
\switchcolumn
\selectlanguage{english}
At Liege in Belgium, St. Theodard, 
 bishop and martyr, who laid down his life for his flock, and after his death 
 was renowned for the gift of miracles.
\switchcolumn*
\selectlanguage{latin}
Chalcédone sanctórum 
 Mártyrum Sósthenis et Victóris, qui, in persecutióne Diocletiáni, sub Asiæ 
 Procónsule Prisco, post víncula et béstias superátas, jussi sunt incéndi; at 
 illi, salutántes se ínvicem in ósculo sancto, in oratióne pósiti emisérunt 
 spíritum.
\switchcolumn
\selectlanguage{english}
At Chalcedon, in the persecution of 
 Diocletian, the holy martyrs Sosthenes and Victor. Under Priscus, 
 proconsul of Asia, they were loaded with fetters and exposed to the beasts, 
 after which they were condemned to be burned. But while they were 
 saluting each other with a holy kiss and praying, they expired.
\switchcolumn*
\selectlanguage{latin}
Item sanctórum Mártyrum 
 Apéllii, Lucæ et Cleméntis
\switchcolumn
\selectlanguage{english}
Also the holy martyrs Apellius, 
 Luke, and Clement.
\switchcolumn*
\selectlanguage{latin}
In Bithynia sanctárum Vírginum sorórum Menodóræ, Metrodóræ et Nymphodóræ; quæ sub Maximiáno Imperatóre et Frontóne Præside, ob intrépidam in Christi fide constántiam martyrio coronátæ, pervenérunt ad glóriam.
\switchcolumn
\selectlanguage{english}
In Bithynia, the holy virgins 
 Menodora, Metrodora, and Nymphodora, sisters. Under Emperor Maximian 
 and the governor Fronto, they were crowned with martyrdom, and went to 
 eternal glory.
\switchcolumn*
\selectlanguage{latin}
Compostéllæ sancti 
 Petri Epíscopi, qui multis virtútibus et miráculis cláruit.
\switchcolumn
\selectlanguage{english}
At Compostella, St. Peter, bishop, who was celebrated for his many virtues and miracles.
\switchcolumn*
\selectlanguage{latin}
In civitáte Albigénsi, in Gállia, sancti Sálvii, Epíscopi et Confessóris.
\switchcolumn
\selectlanguage{english}
In the city of Albi in France, St. Salvius, bishop and confessor.
\switchcolumn*
\selectlanguage{latin}
Nováriæ sancti Agápii 
 Epíscopi.
\switchcolumn
\selectlanguage{english}
At Novara, St. Agapius, bishop.
\switchcolumn*
\selectlanguage{latin}
Constantinópoli sanctæ 
 Pulchériæ Augústæ, Vírginis, religióne et pietáte insígnis.
\switchcolumn
\selectlanguage{english}
At Constantinople, St. Pulcheria, 
 empress and virgin, distinguished by her piety and zeal for religion.
\switchcolumn*
\selectlanguage{latin}
Neápoli, in Campánia, sanctæ Cándidæ junióris, miráculis claræ.
\switchcolumn
\selectlanguage{english}
At Naples in Campania, St. Candida 
 the Younger, famed for miracles.
\switchcolumn*
\selectlanguage{latin}
\end{paracol}


% ---- martyrology/mart09/mart0911.htm
\needspace{10\baselineskip}
\begin{paracol}{2}
\selectlanguage{latin}
\begin{center}{\color{gregoriocolor} Tértio Idus Septémbris. 
 Luna\dots\ }\end{center}
\switchcolumn
\selectlanguage{english}
\begin{center}{\color{gregoriocolor} The   Eleventh Day of 
 September. The\dots\ Day of the Moon.}\end{center}
\end{paracol}

\noindent\begin{tabularx}{\linewidth}{*{19}{>{\centering\arraybackslash}X}}
 \textcolor{gregoriocolor}{a} & \textcolor{gregoriocolor}{b} & \textcolor{gregoriocolor}{c} & \textcolor{gregoriocolor}{d} & \textcolor{gregoriocolor}{e} & \textcolor{gregoriocolor}{f} & \textcolor{gregoriocolor}{g} & \textcolor{gregoriocolor}{h} & \textcolor{gregoriocolor}{i} & \textcolor{gregoriocolor}{k} & \textcolor{gregoriocolor}{l} & \textcolor{gregoriocolor}{m} & \textcolor{gregoriocolor}{n} & \textcolor{gregoriocolor}{p} & \textcolor{gregoriocolor}{q} & \textcolor{gregoriocolor}{r} & \textcolor{gregoriocolor}{s} & \textcolor{gregoriocolor}{t} & \textcolor{gregoriocolor}{u} \\
 19 & 20 & 21 & 22 & 23 & 24 & 25 & 26 & 27 & 28 & 29 & 30 & 1 & 2 & 3 & 4 & 5 & 6 & 7 \\
\end{tabularx}
\vspace{0.5\baselineskip}
\noindent\begin{tabularx}{\linewidth}{*{12}{>{\centering\arraybackslash}X}}
 \textcolor{gregoriocolor}{A} & \textcolor{gregoriocolor}{B} & \textcolor{gregoriocolor}{C} & \textcolor{gregoriocolor}{D} & \textcolor{gregoriocolor}{E} & F & \textcolor{gregoriocolor}{F} & \textcolor{gregoriocolor}{G} & \textcolor{gregoriocolor}{H} & \textcolor{gregoriocolor}{M} & \textcolor{gregoriocolor}{N} & \textcolor{gregoriocolor}{P} \\
 8 & 9 & 10 & 11 & 12 & 13 & 13 & 14 & 15 & 16 & 17 & 18 \\
\end{tabularx}

\begin{paracol}{2}
\selectlanguage{latin}
\lettrine[lines=2]{R}{omæ,} via Salária 
 véteri, in cœmetério Basíllæ, natális sanctórum Mártyrum Proti et Hyacínthi 
 fratrum, eunuchórum beátæ Eugéniæ. Hi, sub Galliéno Imperatóre, 
 deprehénsi quod essent Christiáni, sacrificáre cogúntur; sed non 
 consentiéntes, primo duríssime verberáti sunt, ac tandem páriter decolláti.
\switchcolumn
\selectlanguage{english}
\lettrine[lines=2]{A}{t} Rome, on the old Salarian Way in 
 the cemetery of Basilla, the birthday of the holy martyrs Protus and 
 Hyacinth, brothers, and eunuchs in the service of blessed Eugenia. 
 They were arrested in the time of Emperor Gallienus on the charge of being 
 Christians, and urged to offer sacrifice to the gods. Because they 
 refused, they were most severely scourged and finally beheaded.
\switchcolumn*
\selectlanguage{latin}
Legióne, in Hispánia, 
 sancti Vincéntii, Abbátis et Mártyris.
\switchcolumn
\selectlanguage{english}
At Leon in Spain, St. Vincent, abbot 
 and martyr.
\switchcolumn*
\selectlanguage{latin}
Laodicéæ, in Syria, 
 pássio sanctórum Diodóri, Diomédis et Dídymi.
\switchcolumn
\selectlanguage{english}
At Laodicea in Syria, the martyrdom 
 of Saints Diodorus, Diomedes, and Didymus.
\switchcolumn*
\selectlanguage{latin}
In Ægypto sancti 
 Paphnútii Epíscopi, qui unus fuit ex iis Confessóribus, qui, sub Galério 
 Maximiáno Imperatóre, dextro óculo effósso et sinístro póplite excíso, ad 
 metálla damnáti fuérunt; deínde, sub Constantíno Magno, advérsus Ariános, 
 pro fide cathólica, strénue decertávit; et demum, multis corónis auctus, in 
 pace quiévit.
\switchcolumn
\selectlanguage{english}
In Egypt, the holy bishop Paphnutius, 
 one of those confessors who, under Emperor Galerius Maximinus, having the 
 right eye plucked out and the joint of the left knee cut, were condemned to 
 work in the metal mines. Afterwards, under Constantine the Great, he 
 courageously strove for the Catholic faith against the Arians, and at 
 length, adorned with many crowns, rested in peace.
\switchcolumn*
\selectlanguage{latin}
Lugdúni, in Gállia, 
 deposítio sancti Patiéntis Epíscopi.
\switchcolumn
\selectlanguage{english}
At Lyons in France, the death of St. 
 Patiens, bishop.
\switchcolumn*
\selectlanguage{latin}
Vercéllis sancti 
 Æmiliáni Epíscopi.
\switchcolumn
\selectlanguage{english}
At Vercelli, St. Aemilian, bishop.
\switchcolumn*
\selectlanguage{latin}
Alexandríæ sanctæ 
 Theodóræ, quæ, cum incáute deliquísset, inde, facti pænitens, mirábili 
 abstinéntia et patiéntia in hábitu sancto perseverávit incógnita usque ad 
 mortem.
\switchcolumn
\selectlanguage{english}
At Alexandria, St. Theodora, who 
 having committed a fault through imprudence and repenting of it, remained 
 unknown in a religious habit, and persevered until her death in practices of 
 extraordinary abstinence and patience.
\switchcolumn*
\selectlanguage{latin}
\end{paracol}


% ---- martyrology/mart09/mart0912.htm
\needspace{10\baselineskip}
\begin{paracol}{2}
\selectlanguage{latin}
\begin{center}{\color{gregoriocolor} Prídie Idus Septémbris. 
 Luna\dots\ }\end{center}
\switchcolumn
\selectlanguage{english}
\begin{center}{\color{gregoriocolor} The   Twelfth Day of 
 September. The\dots\ Day of the Moon.}\end{center}
\end{paracol}

\noindent\begin{tabularx}{\linewidth}{*{19}{>{\centering\arraybackslash}X}}
 \textcolor{gregoriocolor}{a} & \textcolor{gregoriocolor}{b} & \textcolor{gregoriocolor}{c} & \textcolor{gregoriocolor}{d} & \textcolor{gregoriocolor}{e} & \textcolor{gregoriocolor}{f} & \textcolor{gregoriocolor}{g} & \textcolor{gregoriocolor}{h} & \textcolor{gregoriocolor}{i} & \textcolor{gregoriocolor}{k} & \textcolor{gregoriocolor}{l} & \textcolor{gregoriocolor}{m} & \textcolor{gregoriocolor}{n} & \textcolor{gregoriocolor}{p} & \textcolor{gregoriocolor}{q} & \textcolor{gregoriocolor}{r} & \textcolor{gregoriocolor}{s} & \textcolor{gregoriocolor}{t} & \textcolor{gregoriocolor}{u} \\
 20 & 21 & 22 & 23 & 24 & 25 & 26 & 27 & 28 & 29 & 30 & 1 & 2 & 3 & 4 & 5 & 6 & 7 & 8 \\
\end{tabularx}
\vspace{0.5\baselineskip}
\noindent\begin{tabularx}{\linewidth}{*{12}{>{\centering\arraybackslash}X}}
 \textcolor{gregoriocolor}{A} & \textcolor{gregoriocolor}{B} & \textcolor{gregoriocolor}{C} & \textcolor{gregoriocolor}{D} & \textcolor{gregoriocolor}{E} & F & \textcolor{gregoriocolor}{F} & \textcolor{gregoriocolor}{G} & \textcolor{gregoriocolor}{H} & \textcolor{gregoriocolor}{M} & \textcolor{gregoriocolor}{N} & \textcolor{gregoriocolor}{P} \\
 9 & 10 & 11 & 12 & 13 & 14 & 14 & 15 & 16 & 17 & 18 & 19 \\
\end{tabularx}

\begin{paracol}{2}
\selectlanguage{latin}
\lettrine[lines=2]{F}{estum} sanctíssimi 
 Nóminis beátæ Maríæ, quod Innocéntius Undécimus, Póntifex Máximus, ob 
 insígnem victóriam de Turcis, ipsíus Vírginis præsídio, Vindobónæ in Austria 
 reportátam, celebrári jussit.
\switchcolumn
\selectlanguage{english}
\lettrine[lines=2]{T}{he} feast of the most holy Name of 
 the Blessed Virgin Mary, celebrated by order of the Sovereign Pontiff, 
 Innocent XI, on account of the signal victory gained over the Turks at 
 Vienna in Austria through her protection.
\switchcolumn*
\selectlanguage{latin}
In Bithynia sancti 
 Autónomi, Epíscopi et Mártyris; qui ex Itália, Diocletiáni Imperatóris 
 persecutiónem declínans, illuc proféctus, ibi, cum plúrimos convertísset ad 
 fidem, a furéntibus Gentílibus, dum sacra Mystéria perágeret, ad altáre 
 mactátus est, et hóstia Christi efféctus.
\switchcolumn
\selectlanguage{english}
In Bithynia, St. Autonomus, bishop 
 and martyr, who went to that country from Italy to avoid the persecution of 
 Diocletian. After he had converted many to the faith, he was killed at 
 the altar by the furious heathen while celebrating the sacred mysteries, and 
 thus he became a victim for Christ.
\switchcolumn*
\selectlanguage{latin}
Icónii, in Lycaónia, 
 sancti Curónoti Epíscopi, qui sub Perénnio Præside, cápite truncátus, 
 martyrii palmam accépit.
\switchcolumn
\selectlanguage{english}
At Iconium in Lycaonia, the holy 
 bishop Curonotus, who received the crown of martyrdom by being beheaded 
 under the governor Perennius.
\switchcolumn*
\selectlanguage{latin}
Alexandríæ natális 
 sanctórum Mártyrum Hierónidis, Leóntii, Serapiónis, Selésii, Valeriáni et 
 Stratónis, qui, sub Maximíno Imperatóre, ob confessiónem nóminis Christi, in 
 mare sunt demérsi.
\switchcolumn
\selectlanguage{english}
At Alexandria, in the time of 
 Emperor Maximinus, the birthday of the holy martyrs Hieronides, Leontius, 
 Serapion, Selesius, Valerian, and Strato, who were drowned in the sea for 
 the confession of the name of Christ.
\switchcolumn*
\selectlanguage{latin}
Meri, in Phrygia, 
 pássio sanctórum Mártyrum Macedónii, Theodúli et Tatiáni, qui, sub Juliáno Apóstata, ab Almáchio Præside, post ália torménta, super crates férreas 
 ignítas pósiti, exsultántes martyrium complevérunt.
\switchcolumn
\selectlanguage{english}
At Merum in Phrygia, the holy 
 martyrs Macedonius, Theodulus, and Tatian, under Julian the Apostate. 
 After other torments, they joyfully completed their martyrdom by being laid 
 on burning gridirons by order of the governor Almachius.
\switchcolumn*
\selectlanguage{latin}
Apud Papíam sancti 
 Juvéntii Epíscopi, de quo ágitur sexto Idus Februárii. Ipse, a beáto 
 Hermágora, discípulo sancti Marci Evangelístæ, ad eam urbem, una cum sancto 
 Syro, cujus memória recólitur quinto Idus Decémbris, diréctus est; et ambo, 
 prædicántes illic Christi Evangélium et magnis virtútibus ac miráculis 
 coruscántes, étiam vicínas urbes divínis opéribus illustrárunt, sicque in 
 pontificáli honóre, glorióso fine, quievérunt in pace.
\switchcolumn
\selectlanguage{english}
At Pavia, St. Juventius, bishop, 
 mentioned on the 8th of February. The blessed Hermagoras, disciple of 
 the evangelist St. Mark, sent him to that city along with St. Cyrus, who is 
 mentioned on the 9th of December. They both preached the Gospel of 
 Christ there, and being renowned for great virtues and miracles, enlightened 
 the neighbouring cities by divine works. They closed their glorious 
 careers in peace, invested with the episcopal office.
\switchcolumn*
\selectlanguage{latin}
Lugdúni, in Gállia, 
 deposítio sancti Sacerdótis Epíscopi.
\switchcolumn
\selectlanguage{english}
At Lyons in France, the death of St. 
 Sacerdos, bishop.
\switchcolumn*
\selectlanguage{latin}
Verónæ sancti Silvíni 
 Epíscopi.
\switchcolumn
\selectlanguage{english}
At Verona, St. Silvinus, bishop.
\switchcolumn*
\selectlanguage{latin}
Anderláci, prope 
 Bruxéllas, in Brabántia, sancti Guidónis Confessóris.
\switchcolumn
\selectlanguage{english}
At Anderlecht, near Brussels in 
 Belgium, St. Guy, confessor.
\switchcolumn*
\selectlanguage{latin}
\end{paracol}


% ---- martyrology/mart09/mart0913.htm
\needspace{10\baselineskip}
\begin{paracol}{2}
\selectlanguage{latin}
\begin{center}{\color{gregoriocolor} Idibus Septémbris. 
 Luna\dots\ }\end{center}
\switchcolumn
\selectlanguage{english}
\begin{center}{\color{gregoriocolor} The   Thirteenth Day of 
 September. The\dots\ Day of the Moon.}\end{center}
\end{paracol}

\noindent\begin{tabularx}{\linewidth}{*{19}{>{\centering\arraybackslash}X}}
 \textcolor{gregoriocolor}{a} & \textcolor{gregoriocolor}{b} & \textcolor{gregoriocolor}{c} & \textcolor{gregoriocolor}{d} & \textcolor{gregoriocolor}{e} & \textcolor{gregoriocolor}{f} & \textcolor{gregoriocolor}{g} & \textcolor{gregoriocolor}{h} & \textcolor{gregoriocolor}{i} & \textcolor{gregoriocolor}{k} & \textcolor{gregoriocolor}{l} & \textcolor{gregoriocolor}{m} & \textcolor{gregoriocolor}{n} & \textcolor{gregoriocolor}{p} & \textcolor{gregoriocolor}{q} & \textcolor{gregoriocolor}{r} & \textcolor{gregoriocolor}{s} & \textcolor{gregoriocolor}{t} & \textcolor{gregoriocolor}{u} \\
 21 & 22 & 23 & 24 & 25 & 26 & 27 & 28 & 29 & 30 & 1 & 2 & 3 & 4 & 5 & 6 & 7 & 8 & 9 \\
\end{tabularx}
\vspace{0.5\baselineskip}
\noindent\begin{tabularx}{\linewidth}{*{12}{>{\centering\arraybackslash}X}}
 \textcolor{gregoriocolor}{A} & \textcolor{gregoriocolor}{B} & \textcolor{gregoriocolor}{C} & \textcolor{gregoriocolor}{D} & \textcolor{gregoriocolor}{E} & F & \textcolor{gregoriocolor}{F} & \textcolor{gregoriocolor}{G} & \textcolor{gregoriocolor}{H} & \textcolor{gregoriocolor}{M} & \textcolor{gregoriocolor}{N} & \textcolor{gregoriocolor}{P} \\
 10 & 11 & 12 & 13 & 14 & 15 & 15 & 16 & 17 & 18 & 19 & 20 \\
\end{tabularx}

\begin{paracol}{2}
\selectlanguage{latin}
\lettrine[lines=2]{A}{lexandríæ} natális 
 beáti Philíppi, patris sanctæ Eugéniæ Vírginis. Hic, dignitátem 
 Præfectúræ Ægypti déserens, Baptísmatis grátiam assecútus est; quem, in 
 oratióne constitútum, jussit Teréntius Præféctus, ejus succéssor, gládio 
 jugulári.
\switchcolumn
\selectlanguage{english}
\lettrine[lines=2]{A}{t} Alexandria, the birthday of 
 blessed Philip, father of the virgin St. Eugenia. Resigning the 
 dignity of prefect of Egypt, he received the grace of baptism. His 
 successor, the prefect Terentius, had him pierced through the throat with a 
 sword while he was praying.
\switchcolumn*
\selectlanguage{latin}
Item sanctórum Mártyrum 
 Macróbii et Juliáni, qui sub Licínio passi sunt.
\switchcolumn
\selectlanguage{english}
Also, the holy martyrs Macrobius and 
 Julian, who suffered under Licinius.
\switchcolumn*
\selectlanguage{latin}
Eódem die sancti 
 Ligórii Mártyris, qui a Gentílibus, in erémo degens, ob Christi fidem 
 necátus est.
\switchcolumn
\selectlanguage{english}
On the same day, St. Ligorius, 
 martyr. While living in the desert, he was murdered by heathens for 
 the faith of Christ.
\switchcolumn*
\selectlanguage{latin}
Alexandríæ sancti 
 Eulógii Epíscopi, doctrína et sanctitáte célebris.
\switchcolumn
\selectlanguage{english}
At Alexandria, St. Eulogius, a 
 bishop celebrated for learning and sanctity.
\switchcolumn*
\selectlanguage{latin}
Andégavi, in Gállia, 
 sancti Maurílii Epíscopi, qui innúmeris miráculis cláruit.
\switchcolumn
\selectlanguage{english}
At Angers in France, St. Maurilius, 
 a bishop renowned for numberless miracles.
\switchcolumn*
\selectlanguage{latin}
Apud Sénonas sancti 
 Amáti, Epíscopi et Confessóris.
\switchcolumn
\selectlanguage{english}
At Sens, St. Amatus, bishop and 
 confessor.
\switchcolumn*
\selectlanguage{latin}
In monastério Romárico, 
 in Gállia, sancti Amáti, Presbyteri et Abbátis, abstinéntia et miraculórum 
 dono illústris.
\switchcolumn
\selectlanguage{english}
In the monastery of Remiremont in 
 France, St. Amatus, priest and abbot, illustrious for the virtue of 
 abstinence and the gift of miracles.
\switchcolumn*
\selectlanguage{latin}
Ipso die sancti Venérii 
 Confessóris, admirándæ sanctitátis viri, qui in ínsula Palmária vitam 
 eremíticam duxit.
\switchcolumn
\selectlanguage{english}
The same day, St. Venerius, 
 confessor, a man of admirable sanctity who led the life of a hermit on the 
 island of Palmaria.
\switchcolumn*
\selectlanguage{latin}
\end{paracol}


% ---- martyrology/mart09/mart0914.htm
\needspace{10\baselineskip}
\begin{paracol}{2}
\selectlanguage{latin}
\begin{center}{\color{gregoriocolor} Décimo octávo Kaléndas Octóbris. 
 Luna\dots\ }\end{center}
\switchcolumn
\selectlanguage{english}
\begin{center}{\color{gregoriocolor} The   Fourteenth Day of 
 September. The\dots\ Day of the Moon.}\end{center}
\end{paracol}

\noindent\begin{tabularx}{\linewidth}{*{19}{>{\centering\arraybackslash}X}}
 \textcolor{gregoriocolor}{a} & \textcolor{gregoriocolor}{b} & \textcolor{gregoriocolor}{c} & \textcolor{gregoriocolor}{d} & \textcolor{gregoriocolor}{e} & \textcolor{gregoriocolor}{f} & \textcolor{gregoriocolor}{g} & \textcolor{gregoriocolor}{h} & \textcolor{gregoriocolor}{i} & \textcolor{gregoriocolor}{k} & \textcolor{gregoriocolor}{l} & \textcolor{gregoriocolor}{m} & \textcolor{gregoriocolor}{n} & \textcolor{gregoriocolor}{p} & \textcolor{gregoriocolor}{q} & \textcolor{gregoriocolor}{r} & \textcolor{gregoriocolor}{s} & \textcolor{gregoriocolor}{t} & \textcolor{gregoriocolor}{u} \\
 22 & 23 & 24 & 25 & 26 & 27 & 28 & 29 & 30 & 1 & 2 & 3 & 4 & 5 & 6 & 7 & 8 & 9 & 10 \\
\end{tabularx}
\vspace{0.5\baselineskip}
\noindent\begin{tabularx}{\linewidth}{*{12}{>{\centering\arraybackslash}X}}
 \textcolor{gregoriocolor}{A} & \textcolor{gregoriocolor}{B} & \textcolor{gregoriocolor}{C} & \textcolor{gregoriocolor}{D} & \textcolor{gregoriocolor}{E} & F & \textcolor{gregoriocolor}{F} & \textcolor{gregoriocolor}{G} & \textcolor{gregoriocolor}{H} & \textcolor{gregoriocolor}{M} & \textcolor{gregoriocolor}{N} & \textcolor{gregoriocolor}{P} \\
 11 & 12 & 13 & 14 & 15 & 16 & 16 & 17 & 18 & 19 & 20 & 21 \\
\end{tabularx}

\begin{paracol}{2}
\selectlanguage{latin}
\lettrine[lines=2]{E}{xaltátio} sanctæ Crucis, quando Heracl\'ius Imperátor, Chósroa Rege devícto, eam de Pérside Hierosólymam reportávit.
\switchcolumn
\selectlanguage{english}
\lettrine[lines=2]{T}{he} Exaltation of the Holy Cross, 
 when Emperor Heraclius, after defeating King Chosroes, brought it back to 
 Jerusalem from Persia.
\switchcolumn*
\selectlanguage{latin}
Romæ, via Appia, beáti 
 Cornélii, Papæ et Mártyris; qui, in persecutióne Décii, post exsílii 
 relegatiónem, jussus est plumbátis cædi, et sic, cum áliis vigínti et uno 
 promíscui sexus, decollári. Sed et Cæreális, miles cum Sallústia uxóre, 
 quos idem Cornélius in fide instrúxerat, eódem die sunt cápite plexi.
\switchcolumn
\selectlanguage{english}
At Rome, on the Appian Way, during 
 the persecution of Decius, blessed Cornelius, pope and martyr. After 
 being banished, he was scourged with leaded whips and then beheaded with 
 twenty-one others of both sexes. On the same day were condemned to 
 capital punishment Caerealis, a soldier, and his wife Sallustia, who had 
 been instructed in the faith by the same Cornelius.
\switchcolumn*
\selectlanguage{latin}
In Africa pássio sancti 
 Cypriáni, Epíscopi Carthaginénsis, sanctitáte et doctrína claríssimi; qui, 
 sub Valeriáno et Galliéno Princípibus, post durum exsílium, cápitis 
 detruncatióne martyrium consummávit, sexto milliário a Carthágine, juxta 
 mare. Eorúndem vero sanctórum Cornélii et Cypriáni memória sextodécimo 
 Kaléndas Octóbris festíve celebrátur.
\switchcolumn
\selectlanguage{english}
In Africa, in the time of Emperors 
 Valerian and Gallienus, St. Cyprian, bishop of Carthage, most renowned for 
 holiness and learning. It was near the seashore, six miles from the 
 city, that he completed his martyrdom by beheading, after enduring a most 
 painful exile. The feast of the Saints Cornelius and Cyprian is kept 
 on the 16th of this month.
\switchcolumn*
\selectlanguage{latin}
Apud Cománam, in Ponto, 
 natális sancti Joánnis, Epíscopi Constantinopolitáni, Confessóris et 
 Ecclésiæ Doctóris, propter áureum eloquéntiæ flumen cognoménto Chrysóstomi; 
 qui, ab inimicórum factióne in exsílium ejéctus, et, cum e sancti Innocéntii 
 Primi, Summi Pontíficis, decréto inde revocarétur, in itínere, a 
 custodiéntibus milítibus multa mala perpéssus, ánimam Deo réddidit. 
 Ejus autem festívitas sexto Kaléndas Februárii celebrátur, quo die sacrum 
 ipsíus corpus a Theodósio junióre Constantinópolim fuit translátum. 
 Hunc vero præclaríssimum divíni verbi præcónem Pius Papa Décimus cæléstem 
 Oratórum sacrórum Patrónum declarávit atque constítuit.
\switchcolumn
\selectlanguage{english}
At Comana in Pontus, the birthday of 
 St. John, bishop of Constantinople, confessor and doctor of the Church, 
 surnamed Chrysostom because of his golden eloquence. He was cast into exile by a faction of his enemies, but was recalled by a decree of Pope 
 Innocent I. However, he suffered many evils on the journey at the 
 hands of the soldiers who guarded him, and he rendered up his soul unto God. 
 His feast is kept on the 27th of January, on which day his holy body was 
 translated to Constantinople by Theodosius the Younger. Pope Pius X 
 declared and appointed this glorious preacher of the divine Word as heavenly 
 patron of those preaching of holy things.
\switchcolumn*
\selectlanguage{latin}
Tréviris sancti Matérni Epíscopi, qui fuit discípulus beáti Petri Apóstoli; ac Tungrénses, Coloniénses et Trevirénses, aliósque finítimos pópulos ad Christi fidem perdúxit.
\switchcolumn
\selectlanguage{english}
At Treves, the holy bishop Maternus, 
 a disciple of the blessed apostle Peter, who brought to the faith of Christ 
 the inhabitants of Tongres, Cologne, Treves, and of the neighbouring 
 country.
\switchcolumn*
\selectlanguage{latin}
Romæ sancti Crescéntii 
 púeri, qui sancti Euthymii fílius éxstitit; atque, in persecutióne 
 Diocletiáni, sub Turpílio Júdice, via Salária, gládio percússus, occúbuit.
\switchcolumn
\selectlanguage{english}
On the Salarian Way at Rome, during 
 the persecution of Diocletian, St. Crescentius, the young son of St. 
 Euthymius, whose life was ended by the sword, under the judge Turpilius.
\switchcolumn*
\selectlanguage{latin}
In Africa pássio 
 sanctórum Mártyrum Crescentiáni, Victóris, Rósulæ et Generális.
\switchcolumn
\selectlanguage{english}
In Africa, the passion of the holy 
 martyrs Crescentian, Victor, Rosula, and Generalis.
\switchcolumn*
\selectlanguage{latin}
\end{paracol}


% ---- martyrology/mart09/mart0915.htm
\needspace{10\baselineskip}
\begin{paracol}{2}
\selectlanguage{latin}
\begin{center}{\color{gregoriocolor} Décimo séptimo Kaléndas Octóbris. 
 Luna\dots\ }\end{center}
\switchcolumn
\selectlanguage{english}
\begin{center}{\color{gregoriocolor} The   Fifteenth Day of 
 September. The\dots\ Day of the Moon.}\end{center}
\end{paracol}

\noindent\begin{tabularx}{\linewidth}{*{19}{>{\centering\arraybackslash}X}}
 \textcolor{gregoriocolor}{a} & \textcolor{gregoriocolor}{b} & \textcolor{gregoriocolor}{c} & \textcolor{gregoriocolor}{d} & \textcolor{gregoriocolor}{e} & \textcolor{gregoriocolor}{f} & \textcolor{gregoriocolor}{g} & \textcolor{gregoriocolor}{h} & \textcolor{gregoriocolor}{i} & \textcolor{gregoriocolor}{k} & \textcolor{gregoriocolor}{l} & \textcolor{gregoriocolor}{m} & \textcolor{gregoriocolor}{n} & \textcolor{gregoriocolor}{p} & \textcolor{gregoriocolor}{q} & \textcolor{gregoriocolor}{r} & \textcolor{gregoriocolor}{s} & \textcolor{gregoriocolor}{t} & \textcolor{gregoriocolor}{u} \\
 23 & 24 & 25 & 26 & 27 & 28 & 29 & 30 & 1 & 2 & 3 & 4 & 5 & 6 & 7 & 8 & 9 & 10 & 11 \\
\end{tabularx}
\vspace{0.5\baselineskip}
\noindent\begin{tabularx}{\linewidth}{*{12}{>{\centering\arraybackslash}X}}
 \textcolor{gregoriocolor}{A} & \textcolor{gregoriocolor}{B} & \textcolor{gregoriocolor}{C} & \textcolor{gregoriocolor}{D} & \textcolor{gregoriocolor}{E} & F & \textcolor{gregoriocolor}{F} & \textcolor{gregoriocolor}{G} & \textcolor{gregoriocolor}{H} & \textcolor{gregoriocolor}{M} & \textcolor{gregoriocolor}{N} & \textcolor{gregoriocolor}{P} \\
 12 & 13 & 14 & 15 & 16 & 17 & 17 & 18 & 19 & 20 & 21 & 22 \\
\end{tabularx}

\begin{paracol}{2}
\selectlanguage{latin}
\lettrine[lines=2]{O}{ctáva} Nativitátis 
 beátæ Maríæ Vírginis.
\switchcolumn
\selectlanguage{english}
\lettrine[lines=2]{T}{he} Octave of the Nativity of the 
 Blessed Virgin Mary.
\switchcolumn*
\selectlanguage{latin}
Festum Septem Dolórum 
 ejúsdem beatíssimæ Vírginis Maríæ.
\switchcolumn
\selectlanguage{english}
The feast of the Seven Sorrows of 
 the same most Blessed Virgin Mary.
\switchcolumn*
\selectlanguage{latin}
Romæ, via Nomentána, 
 natális beáti Nicomédis, Presbyteri et Mártyris, qui, cum díceret 
 compelléntibus se sacrificáre: « Ego non sacrífico nisi Deo omnipoténti, qui 
 regnat in cælis », plumbátis diutíssime cæsus est, atque in eo torménto 
 migrávit ad Dóminum.
\switchcolumn
\selectlanguage{english}
At Rome, on the Via Nomentana, the 
 birthday of blessed Nicomedes, priest and martyr. Because he said to 
 those who would compel him to sacrifice: ``I offer sacrifice only to the 
 omnipotent God who reigneth in heaven,'' he was for a long time scourged with 
 leaded whips, and thus passed to the Lord.
\switchcolumn*
\selectlanguage{latin}
Córdubæ, in Hispánia, 
 sanctórum Mártyrum Emilæ Diáconi, et Jeremíæ, qui, in persecutióne Arábica, 
 post longam cárceris maceratiónem, demum, cervícibus pro Christo abscíssis, 
 martyrium complevérunt.
\switchcolumn
\selectlanguage{english}
At Cordova in Spain, the holy 
 martyrs Emilas, deacon, and Jeremias, who ended their martyrdom in the 
 persecution of the Arabs by being beheaded after a long stay in prison.
\switchcolumn*
\selectlanguage{latin}
In território 
 Cabillonénsi sancti Valeriáni Mártyris, quem Priscus Præses, suspénsum et 
 gravi ungulárum laceratióne cruciátum, tandem, cum in Christi confessióne 
 vidéret immóbilem ac læto ánimo in ejus láudibus permanéntem, gládio 
 animadvérti præcépit.
\switchcolumn
\selectlanguage{english}
In the diocese of Chalons, St. 
 Valerian, martyr, who was suspended on high by the governor Priscus, and 
 tortured with iron hooks. Remaining immovable in the confession of 
 Christ, and continuing joyfully to praise him, he was struck with the sword 
 by order of the same magistrate.
\switchcolumn*
\selectlanguage{latin}
Hadrianópoli, in 
 Thrácia, sanctórum Mártyrum Máximi, Theodóri et Asclepiódoti; qui sub 
 Maximiáno Imperatóre coronáti sunt.
\switchcolumn
\selectlanguage{english}
At Adrianople in Thrace, the holy martyrs 
 Maximus, Theodore, and Asclepiodotus, who were crowned under Emperor 
 Maximian.
\switchcolumn*
\selectlanguage{latin}
Item sancti Porphyrii 
 mimi, qui, coram Juliáno Apóstata, per jocum suscípiens baptísmum, Dei 
 virtúte derepénte mutátur, et Christiánum se esse profitétur; ac mox, ipsíus 
 Imperatóris mandáto secúri percússus, martyrio coronátur.
\switchcolumn
\selectlanguage{english}
Also, St. Porphyry, a comedian, who 
 was baptized in jest in the presence of Julian the Apostate, but was 
 suddenly converted by the power of God and declared himself a Christian. 
 By order of the emperor he was thereupon struck with an axe, and thus 
 crowned with martyrdom.
\switchcolumn*
\selectlanguage{latin}
Eódem die sancti Nicétæ 
 Gothi, qui ab Athanaríco Rege, ob cathólicam fidem, jussus est igne combúri.
\switchcolumn
\selectlanguage{english}
On the same day, St. Nicetas, a 
 Goth, who was burned alive for the Catholic faith by order of King Athanaric.
\switchcolumn*
\selectlanguage{latin}
Marcianópoli, in 
 Thrácia, sanctæ Melitínæ Mártyris, quæ, sub Antoníno Imperatóre et Antíocho 
 Præside, cum ad Gentílium fana semel et íterum ducta esset, atque idóla 
 semper corrúerent, ídeo suspénsa et laniáta est, ac demum cápite plexa.
\switchcolumn
\selectlanguage{english}
At Marcianapolis in Thrace, St. 
 Melitina, a martyr, in the time of Emperor Antoninus and the governor 
 Antiochus. She was twice led to the temples of the heathens, and since 
 the idols fell to the ground each time, she was hanged and torn, and finally 
 beheaded.
\switchcolumn*
\selectlanguage{latin}
Tulli, in Gállia, 
 sancti Apri Epíscopi.
\switchcolumn
\selectlanguage{english}
At Toul in France, St. Aper, bishop.
\switchcolumn*
\selectlanguage{latin}
Item sancti Leobíni, 
 Epíscopi Carnuténsis.
\switchcolumn
\selectlanguage{english}
Also, St. Leobinus, bishop of 
 Chartres.
\switchcolumn*
\selectlanguage{latin}
Lugdúni, in Gállia, 
 sancti Albíni Epíscopi.
\switchcolumn
\selectlanguage{english}
At Lyons in France, St. Albinus, 
 bishop.
\switchcolumn*
\selectlanguage{latin}
Eódem die deposítio 
 sancti Aichárdi Abbátis.
\switchcolumn
\selectlanguage{english}
On the same day, the death of St. 
 Aichard, abbot.
\switchcolumn*
\selectlanguage{latin}
In Gállia sanctæ 
 Eutrópiæ Víduæ.
\switchcolumn
\selectlanguage{english}
In France, St. Eutropia, widow.
\switchcolumn*
\selectlanguage{latin}
Génuæ sanctæ Catharínæ 
 Víduæ, contémptu mundi et caritáte in Deum insígnis.
\switchcolumn
\selectlanguage{english}
In Genoa, St. Catherine, a widow, 
 renowned for her contempt of the world and her love of God.
\switchcolumn*
\selectlanguage{latin}
\end{paracol}


% ---- martyrology/mart09/mart0916.htm
\needspace{10\baselineskip}
\begin{paracol}{2}
\selectlanguage{latin}
\begin{center}{\color{gregoriocolor} Sextodécimo Kaléndas Octóbris. 
 Luna\dots\ }\end{center}
\switchcolumn
\selectlanguage{english}
\begin{center}{\color{gregoriocolor} The   Sixteenth Day of 
 September. The\dots\ Day of the Moon.}\end{center}
\end{paracol}

\noindent\begin{tabularx}{\linewidth}{*{19}{>{\centering\arraybackslash}X}}
 \textcolor{gregoriocolor}{a} & \textcolor{gregoriocolor}{b} & \textcolor{gregoriocolor}{c} & \textcolor{gregoriocolor}{d} & \textcolor{gregoriocolor}{e} & \textcolor{gregoriocolor}{f} & \textcolor{gregoriocolor}{g} & \textcolor{gregoriocolor}{h} & \textcolor{gregoriocolor}{i} & \textcolor{gregoriocolor}{k} & \textcolor{gregoriocolor}{l} & \textcolor{gregoriocolor}{m} & \textcolor{gregoriocolor}{n} & \textcolor{gregoriocolor}{p} & \textcolor{gregoriocolor}{q} & \textcolor{gregoriocolor}{r} & \textcolor{gregoriocolor}{s} & \textcolor{gregoriocolor}{t} & \textcolor{gregoriocolor}{u} \\
 24 & 25 & 26 & 27 & 28 & 29 & 30 & 1 & 2 & 3 & 4 & 5 & 6 & 7 & 8 & 9 & 10 & 11 & 12 \\
\end{tabularx}
\vspace{0.5\baselineskip}
\noindent\begin{tabularx}{\linewidth}{*{12}{>{\centering\arraybackslash}X}}
 \textcolor{gregoriocolor}{A} & \textcolor{gregoriocolor}{B} & \textcolor{gregoriocolor}{C} & \textcolor{gregoriocolor}{D} & \textcolor{gregoriocolor}{E} & F & \textcolor{gregoriocolor}{F} & \textcolor{gregoriocolor}{G} & \textcolor{gregoriocolor}{H} & \textcolor{gregoriocolor}{M} & \textcolor{gregoriocolor}{N} & \textcolor{gregoriocolor}{P} \\
 13 & 14 & 15 & 16 & 17 & 18 & 18 & 19 & 20 & 21 & 22 & 23 \\
\end{tabularx}

\begin{paracol}{2}
\selectlanguage{latin}
\lettrine[lines=2]{S}{anctórum} Mártyrum 
 Cornélii Papæ, et Cypriáni, Carthaginénsis Epíscopi, quorum memória décimo 
 octávo Kaléndas Octóbris recólitur.
\switchcolumn
\selectlanguage{english}
\lettrine[lines=2]{T}{he} holy martyrs Cornelius, pope, 
 and Cyprian, bishop of Carthage, who were mentioned on the14th of September.
\switchcolumn*
\selectlanguage{latin}
Chalcédone natális 
 sanctæ Euphémiæ, Vírginis et Mártyris; quæ, sub Diocletiáno Imperatóre et 
 Prisco Procónsule, torménta, cárceres, vérbera, arguménta rotárum, ignes, 
 póndera lápidum, béstias, plagas virgárum, serras acútas, sartágines ignítas 
 pro Christo superávit. Sed, rursus in theátrum ad béstias ducta, ibi, 
 cum orásset ad Dóminum ut jam spíritum suum suscíperet, una ex iis morsum 
 sancto córpori infigénte, céteris pedes ejus lambéntibus, immaculátum 
 spíritum Deo réddidit.
\switchcolumn
\selectlanguage{english}
At Chalcedon, the birthday of St. 
 Euphemia, virgin and martyr, under Emperor Diocletian and the proconsul 
 Priscus. For her faith in our Lord she was subjected to tortures, 
 imprisonment, blows, the torment of the wheel, fire, the crushing weight of 
 stones, the teeth of the beasts, scourging with rods, the cutting of sharp 
 saws, and burning pans, all of which she survived. But when she was 
 again exposed to the beasts in the amphitheatre, praying to our Lord to 
 receive her spirit, one of the animals inflicted a bite on her holy body 
 although the rest of them licked her feet, and she yielded her unspotted 
 soul unto God.
\switchcolumn*
\selectlanguage{latin}
Romæ sanctórum Mártyrum 
 Lúciæ, nóbilis matrónæ, et Geminiáni; quos ambos Diocletiánus Imperátor, 
 pœnis gravíssimis afflíctos diúque tortos, tandem, post laudábilem martyrii 
 victóriam, gládio animadvérti præcépit.
\switchcolumn
\selectlanguage{english}
At Rome, the holy martyrs Lucy, a 
 noble matron, and Geminian, who were subjected to grievous afflictions and 
 were for a long time tortured by the command of Emperor Diocletian. 
 Finally, being put to the sword, they obtained the glorious victory of 
 martyrdom.
\switchcolumn*
\selectlanguage{latin}
Natális sancti Martíni 
 Primi, Papæ et Mártyris, qui, cum Sérgium ac Paulum et Pyrrhum hæréticos, 
 Romæ coácta Synodo, condemnásset, ídeo, jussu Constántis, Imperatóris 
 hærétici, per fraudem captus et Constantinópolim perdúctus, in Chersonésum 
 relegátus est, ibíque, ob cathólicam fidem ærúmnis conféctus, vitam finívit, 
 multísque miráculis cláruit. Ejus corpus, póstea Romam translátum, in 
 Ecclésia sanctórum Silvéstri et Martíni cónditum fuit. Ipsíus tamen 
 festívitas prídie Idus Novémbris celebrátur.
\switchcolumn
\selectlanguage{english}
The birthday of St. Martin I, pope 
 and martyr. He had called together a council at Rome and condemned the 
 heretics Sergius, Paul and Pyrrhus. By order of the heretical Emperor 
 Constantius he was taken prisoner through a deceit, brought to 
 Constantinople, and exiled to the Chersonese. There he ended his life, 
 worn out with his labours for the Catholic faith and favoured with many 
 virtues. His body was afterwards brought to Rome and buried in the 
 church of Saints Sylvester and Martin. His feast, however, is observed 
 on the 12th of November.
\switchcolumn*
\selectlanguage{latin}
Romæ item natális 
 sanctæ Cæcíliæ, Vírginis et Mártyris, quæ sponsum suum Valeriánum et fratrem 
 ejus Tibúrtium ad credéndum in Christum perdúxit, et ad martyrium incitávit. 
 Hanc Almáchius, Urbis Præféctus, post eórum martyrium tenéri, atque illústri 
 passióne post ignem superátum, fecit gládio consummári, témpore Marci 
 Aurélii Sevéri Alexándri Imperatóris. Ejus vero festum recólitur décimo Kaléndas Decémbris.
\switchcolumn
\selectlanguage{english}
Also at Rome, the birthday of St. 
 Cecilia, virgin and martyr. She brought her husband and brother 
 Tiburtius to the faith of Christ and afterwards encouraged them on to 
 martyrdom. Almachius, prefect of the city, after their martyrdom, had 
 her arrested and slain by the sword, after she had endured many trials and 
 had passed through fire unhurt. This was in the reign of Emperor 
 Marcus Aurelius Severus Alexander. Her feast is celebrated on the 22nd 
 of November.
\switchcolumn*
\selectlanguage{latin}
Heracléæ, in Thrácia, 
 sanctæ Sebastiánæ Mártyris, quæ a beáto Paulo Apóstolo ad Christi fidem est 
 perdúcta; atque, sub Domitiáno Imperatóre et Sérgio Præside, váriis modis 
 tentáta, gládio tandem cæsa est.
\switchcolumn
\selectlanguage{english}
At Heraclea in Thrace, under Emperor 
 Domitian and the governor Sergius, St. Sebastiana, martyr. Being 
 brought to the faith of Christ by the blessed apostle Paul, she was 
 tormented in various ways and finally beheaded.
\switchcolumn*
\selectlanguage{latin}
Romæ, via Flamínia, 
 sanctórum Mártyrum Abúndii Presbyteri, et Abundántii Diáconi, quos 
 Diocletiánus Imperátor, una cum illústri viro Marciáno et ejus fílio Joánne, 
 quem illi a mórtuis suscitáverant, gládio feríri jussit, décimo ab Urbe 
 lápide.
\switchcolumn
\selectlanguage{english}
At Rome, at a place on the Flaminian 
 Way ten miles from the city, the holy martyrs Abundius, a priest, and 
 Abundantius, a deacon, whom Emperor Diocletian ordered to be struck with the 
 sword, together with Marcian, an illustrious man, and his son John, whom 
 they raised from the dead.
\switchcolumn*
\selectlanguage{latin}
Córdubæ, in Hispánia, 
 sanctórum Mártyrum Rogélli et Servidéi, qui, mánibus pedibúsque abscíssis, 
 ad últimum decolláti sunt.
\switchcolumn
\selectlanguage{english}
At Cordova in Spain, the holy 
 martyrs Rogellus and Servusdeus, who were beheaded after their hands and 
 feet had been cut off.
\switchcolumn*
\selectlanguage{latin}
Apud Cándidam Casam, in 
 Scótia, sancti Niniáni, Epíscopi et Confessóris.
\switchcolumn
\selectlanguage{english}
At Whithorn in Scotland, St. Ninian, 
 bishop and confessor.
\switchcolumn*
\selectlanguage{latin}
In Anglia sanctæ Edíthæ 
 Vírginis, Regis Anglórum Edgári fíliæ, quæ, in monastério a téneris annis 
 Deo dicáta, sæculum hoc ignorávit pótius quam relíquit.
\switchcolumn
\selectlanguage{english}
In England, St. Edith, virgin, 
 daughter of the English King Edgar. She was consecrated to God in a 
 monastery from her earliest years, and it may be said rather that she never 
 knew the world than that she forsook it.
\switchcolumn*
\selectlanguage{latin}
In monte Cassíno beáti 
 Victóris Papæ Tértii, qui, sancti Gregórii Séptimi succéssor, Apostólicam 
 Sedem novo splendóre illustrávit, insígnem de Saracénis triúmphum divína ope 
 consecútus. Cultum, ab immemorábili témpore eídem exhíbitum, Leo 
 Décimus tértius, Póntifex Máximus, ratum hábuit et confirmávit.
\switchcolumn
\selectlanguage{english}
At Monte Cassino, the blessed Pope 
 Victor III, successor of Pope St. Gregory VII, who shed a fresh lustre on 
 the Apostolic See, and by God's help gained a famous victory over the 
 Saracens. Pope Leo XIII approved and confirmed the veneration given 
 him from time immemorial.
\switchcolumn*
\selectlanguage{latin}
\end{paracol}


% ---- martyrology/mart09/mart0917.htm
\needspace{10\baselineskip}
\begin{paracol}{2}
\selectlanguage{latin}
\begin{center}{\color{gregoriocolor} Quintodécimo Kaléndas Octóbris. 
 Luna\dots\ }\end{center}
\switchcolumn
\selectlanguage{english}
\begin{center}{\color{gregoriocolor} The   Seventeenth Day of 
 September. The\dots\ Day of the Moon.}\end{center}
\end{paracol}

\noindent\begin{tabularx}{\linewidth}{*{19}{>{\centering\arraybackslash}X}}
 \textcolor{gregoriocolor}{a} & \textcolor{gregoriocolor}{b} & \textcolor{gregoriocolor}{c} & \textcolor{gregoriocolor}{d} & \textcolor{gregoriocolor}{e} & \textcolor{gregoriocolor}{f} & \textcolor{gregoriocolor}{g} & \textcolor{gregoriocolor}{h} & \textcolor{gregoriocolor}{i} & \textcolor{gregoriocolor}{k} & \textcolor{gregoriocolor}{l} & \textcolor{gregoriocolor}{m} & \textcolor{gregoriocolor}{n} & \textcolor{gregoriocolor}{p} & \textcolor{gregoriocolor}{q} & \textcolor{gregoriocolor}{r} & \textcolor{gregoriocolor}{s} & \textcolor{gregoriocolor}{t} & \textcolor{gregoriocolor}{u} \\
 25 & 26 & 27 & 28 & 29 & 30 & 1 & 2 & 3 & 4 & 5 & 6 & 7 & 8 & 9 & 10 & 11 & 12 & 13 \\
\end{tabularx}
\vspace{0.5\baselineskip}
\noindent\begin{tabularx}{\linewidth}{*{12}{>{\centering\arraybackslash}X}}
 \textcolor{gregoriocolor}{A} & \textcolor{gregoriocolor}{B} & \textcolor{gregoriocolor}{C} & \textcolor{gregoriocolor}{D} & \textcolor{gregoriocolor}{E} & F & \textcolor{gregoriocolor}{F} & \textcolor{gregoriocolor}{G} & \textcolor{gregoriocolor}{H} & \textcolor{gregoriocolor}{M} & \textcolor{gregoriocolor}{N} & \textcolor{gregoriocolor}{P} \\
 14 & 15 & 16 & 17 & 18 & 19 & 19 & 20 & 21 & 22 & 23 & 24 \\
\end{tabularx}

\begin{paracol}{2}
\selectlanguage{latin}
\lettrine[lines=2]{I}{n} monte Alvérniæ, in 
 Etrúria, commemorátio Impressiónis sacrórum Stígmatum, quibus sanctus 
 Francíscus, Ordinis Minórum Institútor, in suis mánibus, pédibus et látere, 
 mirábili Dei grátia, impréssus fuit.
\switchcolumn
\selectlanguage{english}
\lettrine[lines=2]{T}{he} commemoration of the Impression 
 of the Sacred Stigmata which St. Francis, founder of the Order of Friars 
 Minor, received through a wonderful favour of God in his hands, feet, and 
 side, at Mount Alverina in Etruria.
\switchcolumn*
\selectlanguage{latin}
Romæ natális sancti 
 Robérti Bellarmíno, Confessóris, e Societáte Jesu, atque Cardinális et 
 Capuáni olim Epíscopi, sanctitáte, doctrína, et plúrimis ad cathólicæ fídei 
 et Apostólicæ Sedis defensiónem suscéptis labóribus claríssimi; quem Pius 
 Undécimus, Póntifex Máximus, Sanctórum honóribus auxit et universális 
 Ecclésiæ Doctórem declarávit, ejúsque festum tértio Idus Maji recoléndum 
 indíxit.
\switchcolumn
\selectlanguage{english}
At Rome, the birthday of St. Robert 
 Bellarmine of the Society of Jesus, confessor and cardinal, and also 
 formerly bishop of Capua. He is noted for his holiness, learning, and 
 the many great tasks he performed in defence of the Catholic faith and the 
 Apostolic See. Pope Pius XI bestowed on him the honours of the saints, 
 declared him to be a doctor of the universal Church, and appointed the 13th 
 of May as his feast day.
\switchcolumn*
\selectlanguage{latin}
Romæ, via Tiburtína, 
 natális sancti Justíni, Presbyteri et Mártyris; qui, in persecutióne 
 Valeriáni et Galliéni, ob confessiónis glóriam fuit insígnis. Hic 
 beáti Pontíficis Xysti Secúndi, Lauréntii, Hippólyti aliorúmque plurimórum 
 Sanctórum córpora sepelívit, ac demum, sub Cláudio, martyrium consummávit.
\switchcolumn
\selectlanguage{english}
At Rome, on the road to Tivoli, the 
 birthday of St. Justin, priest and martyr, who distinguished himself by a 
 glorious confession of the faith during the persecution of Valerian and 
 Gallienus. He buried the bodies of the blessed Pontiff Sixtus II, of 
 Lawrence, Hippolytus, and many other saints, and finally completed his 
 martyrdom under Claudius.
\switchcolumn*
\selectlanguage{latin}
Item Romæ sanctórum 
 Mártyrum Narcíssi et Crescentiónis.
\switchcolumn
\selectlanguage{english}
Also at Rome, the holy martyrs 
 Narcissus and Crescentio.
\switchcolumn*
\selectlanguage{latin}
Apud Leódium, in Bélgio, 
 beáti Lambérti, Epíscopi Trajecténsis, qui, cum régiam domum zelo religiónis 
 increpásset, a nocéntibus ínnocens occísus est, sicque aulam regni cæléstis 
 perpétuo victúrus intrávit.
\switchcolumn
\selectlanguage{english}
At Liege in Belgium, blessed 
 Lambert, bishop of Maestricht. Through his zeal for religion he 
 rebuked the royal family, and was undeservedly put to death by the guilty, 
 and thus he entered the court of the heavenly kingdom, to enjoy it forever.
\switchcolumn*
\selectlanguage{latin}
Cæsaraugústæ, in 
 Hispánia, sancti Petri de Arbues, primi in Aragóniæ regno Quæsitóris fídei; 
 quem, a relápsis Judæis ob eándem, quam pro múnere suo strénue tuebátur, 
 cathólicam fidem, immániter trucidátum, sanctórum Mártyrum catálogo Pius 
 Papa Nonus adjúnxit.
\switchcolumn
\selectlanguage{english}
At Saragossa in Spain, St. Peter of 
 Arbues, first inquisitor of the faith in the kingdom of Aragon, who received 
 the palm of martyrdom by being barbarously massacred by apostate Jews for 
 courageously defending the Catholic faith, according to the duties of his 
 office. He was added to the list of martyr saints by Pius IX.
\switchcolumn*
\selectlanguage{latin}
In Británnia sanctórum 
 Mártyrum Sócratis et Stéphani.
\switchcolumn
\selectlanguage{english}
In England, the holy martyrs 
 Socrates and Stephen.
\switchcolumn*
\selectlanguage{latin}
Noviodúni, in Gálliis, 
 sanctórum Mártyrum Valeriáni, Macríni et Gordiáni.
\switchcolumn
\selectlanguage{english}
At Noyon in France, the holy martyrs 
 Valerian, Macrinus, and Gordian.
\switchcolumn*
\selectlanguage{latin}
Augustodúni sancti 
 Flocélli púeri, qui, sub Antoníno Imperatóre et Valeriáno Præside, multa 
 passus, demum, a feris discérptus, martyrii corónam adéptus est.
\switchcolumn
\selectlanguage{english}
At Autun, under Emperor Antoninus 
 and the governor Valerian, St. Flocellus, a boy, who, after many sufferings, 
 was torn to pieces by wild beasts, and thus won the crown of martyrs.
\switchcolumn*
\selectlanguage{latin}
Córdubæ, in Hispánia, 
 sanctæ Colúmbæ, Vírginis et Mártyris.
\switchcolumn
\selectlanguage{english}
At Cordova in Spain, St. Columba, 
 virgin and martyr.
\switchcolumn*
\selectlanguage{latin}
In Phrygia sanctæ 
 Ariádne Mártyris, sub Hadriáno Imperatóre.
\switchcolumn
\selectlanguage{english}
In Phrygia, St. Ariadne, martyr, 
 under Emperor Hadrian.
\switchcolumn*
\selectlanguage{latin}
Eódem die sanctæ 
 Agathoclíæ, quæ, cum esset ancílla cujúsdam mulíeris infidélis, a dómina sua 
 longo témpore verbéribus aliísque ærúmnis est vexáta ut Christum negáret; 
 demum, obláta Júdici ac sævius laniáta, et nihilóminus in confessióne fídei 
 persístens, ídeo, post excisiónem linguæ, in ignem projécta est.
\switchcolumn
\selectlanguage{english}
On the same day, St. Agathoclia, 
 servant of an infidel woman, who was for a long time subjected by her to 
 blows and other afflictions that she might deny Christ. She was 
 finally presented to the judge and cruelly lacerated, but since she 
 persisted in confessing the faith, they cut off her tongue and threw her 
 into the flames.
\switchcolumn*
\selectlanguage{latin}
Medioláni deposítio 
 sancti Sátyri Confessóris, cujus insígnia mérita sanctus Ambrósius, ejus 
 frater, commémorat.
\switchcolumn
\selectlanguage{english}
At Milan, the death of St. Satyrus, 
 confessor, whose distinguished merits are mentioned by his brother, St. 
 Ambrose.
\switchcolumn*
\selectlanguage{latin}
Apud Bíngiam, in 
 diœcési Moguntinénsi, sanctæ Hildegárdis Vírginis.
\switchcolumn
\selectlanguage{english}
At Bingen, in the diocese of Mainz, 
 St. Hildegard, virgin.
\switchcolumn*
\selectlanguage{latin}
Romæ sanctæ Theodóræ 
 matrónæ, quæ, in persecutióne Diocletiáni, sanctis Martyribus sédulo 
 ministrábat.
\switchcolumn
\selectlanguage{english}
At Rome, St. Theodora, a matron who 
 zealously ministered to the martyrs in the persecution of Diocletian.
\switchcolumn*
\selectlanguage{latin}
\end{paracol}


% ---- martyrology/mart09/mart0918.htm
\needspace{10\baselineskip}
\begin{paracol}{2}
\selectlanguage{latin}
\begin{center}{\color{gregoriocolor} Quartodécimo Kaléndas Octóbris. 
 Luna\dots\ }\end{center}
\switchcolumn
\selectlanguage{english}
\begin{center}{\color{gregoriocolor} The   Eighteenth Day of 
 September. The\dots\ Day of the Moon.}\end{center}
\end{paracol}

\noindent\begin{tabularx}{\linewidth}{*{19}{>{\centering\arraybackslash}X}}
 \textcolor{gregoriocolor}{a} & \textcolor{gregoriocolor}{b} & \textcolor{gregoriocolor}{c} & \textcolor{gregoriocolor}{d} & \textcolor{gregoriocolor}{e} & \textcolor{gregoriocolor}{f} & \textcolor{gregoriocolor}{g} & \textcolor{gregoriocolor}{h} & \textcolor{gregoriocolor}{i} & \textcolor{gregoriocolor}{k} & \textcolor{gregoriocolor}{l} & \textcolor{gregoriocolor}{m} & \textcolor{gregoriocolor}{n} & \textcolor{gregoriocolor}{p} & \textcolor{gregoriocolor}{q} & \textcolor{gregoriocolor}{r} & \textcolor{gregoriocolor}{s} & \textcolor{gregoriocolor}{t} & \textcolor{gregoriocolor}{u} \\
 26 & 27 & 28 & 29 & 30 & 1 & 2 & 3 & 4 & 5 & 6 & 7 & 8 & 9 & 10 & 11 & 12 & 13 & 14 \\
\end{tabularx}
\vspace{0.5\baselineskip}
\noindent\begin{tabularx}{\linewidth}{*{12}{>{\centering\arraybackslash}X}}
 \textcolor{gregoriocolor}{A} & \textcolor{gregoriocolor}{B} & \textcolor{gregoriocolor}{C} & \textcolor{gregoriocolor}{D} & \textcolor{gregoriocolor}{E} & F & \textcolor{gregoriocolor}{F} & \textcolor{gregoriocolor}{G} & \textcolor{gregoriocolor}{H} & \textcolor{gregoriocolor}{M} & \textcolor{gregoriocolor}{N} & \textcolor{gregoriocolor}{P} \\
 15 & 16 & 17 & 18 & 19 & 20 & 20 & 21 & 22 & 23 & 24 & 25 \\
\end{tabularx}

\begin{paracol}{2}
\selectlanguage{latin}
\lettrine[lines=2]{A}{uximi,} in Picéno, sancti Joséphi a Cupertíno, Sacerdótis ex Ordine Minórum Conventuálium et 
 Confessóris; quem Clemens Papa Décimus tértius in Sanctórum númerum rétulit.
\switchcolumn
\selectlanguage{english}
\lettrine[lines=2]{A}{t} Osimo in Piceno, St. Joseph of 
 Cupertino, priest and confessor of the Order of Friars Minor Conventual, who was placed 
 among the saints by Clement XIII.
\switchcolumn*
\selectlanguage{latin}
In Chálcide Græciæ 
 natális sancti Methódii, qui prius Olympii, in Lycia, et póstea Tyri, in 
 Phœnícia, éxstitit Epíscopus, sermónis nitóre ac doctrína claríssimus; atque, 
 ad extrémum novíssimæ persecutiónis (ut scribit sanctus Hierónymus), 
 martyrio coronátus est.
\switchcolumn
\selectlanguage{english}
In Chalcis of Greece, the birthday 
 of St. Methodius, bishop of Olympius in Lycia and afterwards of Tyre in 
 Phoenicia, most renowned for eloquence and learning. St. Jerome says 
 that he won the martyr's crown at the end of the last persecution.
\switchcolumn*
\selectlanguage{latin}
In território Viennénsi 
 sancti Ferréoli Mártyris, qui, cum esset tribuníciæ potestátis, jussu 
 impiíssimi Præsidis Crispíni tentus, et primo crudelíssime verberátus, 
 deínde, gravi catenárum póndere onústus, in tetérrimum cárcerem trusus est; 
 unde, solútis Dei nutu vínculis et jánuis cárceris patefáctis, éxiens, ab 
 insequéntibus íterum est captus, ac martyrii palmam obtruncatióne cápitis 
 percépit.
\switchcolumn
\selectlanguage{english}
In the diocese of Vienne, the holy 
 martyr Ferreol, a tribune, who was arrested by order of the impious governor 
 Crispinus, most cruelly scourged, loaded with heavy chains, and cast into a 
 dark dungeon. A miracle broke his bonds and opened the doors of the 
 prison, from which he made his escape, but he was taken again by his 
 pursuers and received the palm of martyrdom by being beheaded.
\switchcolumn*
\selectlanguage{latin}
Item sanctárum Mártyrum 
 Sophíæ et Irénes.
\switchcolumn
\selectlanguage{english}
Also, the Saints Sophia and Irene, 
 martyrs.
\switchcolumn*
\selectlanguage{latin}
Medioláni sancti 
 Eustórgii Primi, ejúsdem civitátis Epíscopi, beáti Ambrósii testimónio 
 célebris.
\switchcolumn
\selectlanguage{english}
At Milan, St. Eustorgius, first 
 bishop of that city, highly praised by blessed Ambrose.
\switchcolumn*
\selectlanguage{latin}
Gortynæ, in Creta, 
 sancti Euménii, Epíscopi et Confessóris.
\switchcolumn
\selectlanguage{english}
At Gortyna in Crete, St. Eumenius, 
 bishop and confessor.
\switchcolumn*
\selectlanguage{latin}
\end{paracol}


% ---- martyrology/mart09/mart0919.htm
\needspace{10\baselineskip}
\begin{paracol}{2}
\selectlanguage{latin}
\begin{center}{\color{gregoriocolor} Tertiodécimo Kaléndas Octóbris. 
 Luna\dots\ }\end{center}
\switchcolumn
\selectlanguage{english}
\begin{center}{\color{gregoriocolor} The   Nineteenth Day of 
 September. The\dots\ Day of the Moon.}\end{center}
\end{paracol}

\noindent\begin{tabularx}{\linewidth}{*{19}{>{\centering\arraybackslash}X}}
 \textcolor{gregoriocolor}{a} & \textcolor{gregoriocolor}{b} & \textcolor{gregoriocolor}{c} & \textcolor{gregoriocolor}{d} & \textcolor{gregoriocolor}{e} & \textcolor{gregoriocolor}{f} & \textcolor{gregoriocolor}{g} & \textcolor{gregoriocolor}{h} & \textcolor{gregoriocolor}{i} & \textcolor{gregoriocolor}{k} & \textcolor{gregoriocolor}{l} & \textcolor{gregoriocolor}{m} & \textcolor{gregoriocolor}{n} & \textcolor{gregoriocolor}{p} & \textcolor{gregoriocolor}{q} & \textcolor{gregoriocolor}{r} & \textcolor{gregoriocolor}{s} & \textcolor{gregoriocolor}{t} & \textcolor{gregoriocolor}{u} \\
 27 & 28 & 29 & 30 & 1 & 2 & 3 & 4 & 5 & 6 & 7 & 8 & 9 & 10 & 11 & 12 & 13 & 14 & 15 \\
\end{tabularx}
\vspace{0.5\baselineskip}
\noindent\begin{tabularx}{\linewidth}{*{12}{>{\centering\arraybackslash}X}}
 \textcolor{gregoriocolor}{A} & \textcolor{gregoriocolor}{B} & \textcolor{gregoriocolor}{C} & \textcolor{gregoriocolor}{D} & \textcolor{gregoriocolor}{E} & F & \textcolor{gregoriocolor}{F} & \textcolor{gregoriocolor}{G} & \textcolor{gregoriocolor}{H} & \textcolor{gregoriocolor}{M} & \textcolor{gregoriocolor}{N} & \textcolor{gregoriocolor}{P} \\
 16 & 17 & 18 & 19 & 20 & 21 & 21 & 22 & 23 & 24 & 25 & 26 \\
\end{tabularx}

\begin{paracol}{2}
\selectlanguage{latin}
\lettrine[lines=2]{P}{utéolis,} in Campánia, sanctórum Mártyrum Januárii, Beneventánæ civitátis Epíscopi, ejúsque Diáconi 
 Festi, et Desidérii Lectóris, Sósii, Diáconi Ecclésiæ Misenátis; Próculi, 
 Diáconi Puteoláni; Eutychii et Acútii. Hi omnes, post víncula et cárceres, cápite cæsi sunt, sub Diocletiáno Príncipe. Corpus sancti 
 Januárii delátum fuit Neápolim, atque honorífice in Ecclésia tumulátum; ubi 
 étiam beatíssimi Mártyris sanguis in ampúlla vítrea adhuc servátur, qui, in 
 conspéctu cápitis illíus pósitus, velut recens liquéscere et ebullíre 
 conspícitur.
\switchcolumn
\selectlanguage{english}
\lettrine[lines=2]{A}{t} Pozzuoli in Campania, the holy 
 martyrs Januarius, bishop of Benevento; Festus, his deacon, and Desiderius, 
 a lector, together with Sosius, a deacon of the Church of Miseno; Proculus, 
 deacon of Pozzuoli; Eutychius, and Acutius, who were bound and imprisoned 
 and then beheaded during the reign of Diocletian. The body of St. Januarius 
 was brought to Naples and buried in the church with due honours, where even 
 now the blood of the blessed martyr is kept in a vial, and when placed close 
 to his head is seen to become liquid and bubble up as if it were just taken 
 from his veins.
\switchcolumn*
\selectlanguage{latin}
In Palæstína sanctórum 
 Mártyrum et Ægypti Episcopórum Pélei, Nili et Elíæ; qui, témpore 
 persecutiónis Diocletiáni, cum plúrimis Cléricis, pro Christo sunt igne 
 consúmpti.
\switchcolumn
\selectlanguage{english}
In Palestine, the holy martyrs 
 Peleus, Nilus, and Elias, bishops in Egypt, with many others of the clergy, 
 who were consumed by fire for the sake of Christ during the persecution of 
 Diocletian.
\switchcolumn*
\selectlanguage{latin}
Nucériæ natális 
 sanctórum Mártyrum Felícis et Constántiæ, qui passi sunt sub Neróne.
\switchcolumn
\selectlanguage{english}
At Nocera, the birthday of the holy 
 martyrs Felix and Constantia, who suffered under Nero.
\switchcolumn*
\selectlanguage{latin}
Eódem die sanctórum 
 Mártyrum Tróphimi, Sabbátii et Dorymedóntis, sub Probo Imperatóre. Ex 
 eis Sabbátius Antiochíæ, jussu Attici Præsidis, támdiu flagris cæsus est, 
 donec emítteret spíritum; Tróphimus vero, Synnadam, in Phrygia, ad Perénnium 
 Præsidem missus, ibi, post multos cruciátus, cum Dorymedónte Senatóre, 
 cápitis decollatióne martyrium consummávit.
\switchcolumn
\selectlanguage{english}
Also, the holy martyrs Trophimus, 
 Sabbatius, and Dorymedon, senator, under Emperor Probus. By command of 
 the governor Atticus at Antioch, Sabbatius was scourged until he expired. 
 Trophimus was sent to the governor Perennius at Synnada, where he and the 
 senator Dorymedon completed their martyrdom by being beheaded after enduring 
 many torments.
\switchcolumn*
\selectlanguage{latin}
Eleutherópoli, in 
 Palæstína, sanctæ Susánnæ, Vírginis et Mártyris; quæ, orta ex idolórum 
 sacerdóte Arthémio et Judæa mulíere Martha, ad Christiánam fidem, paréntibus 
 mórtuis, est convérsa, atque ob eándem fidem ab Alexándro Præfécto várie 
 torta et in cárcerem trusa, ibi orans migrávit ad Sponsum.
\switchcolumn
\selectlanguage{english}
At Eleutheropolis in Palestine, St. 
 Susanna, virgin and martyr. She was the daughter of Arthemius, a pagan 
 priest, and of Martha, a Jewish woman, and after the death of her parents 
 she was converted to the Christian faith. For this she was tortured in 
 various ways, and cast in prison by the prefect Alexander, and there gave up 
 her spirit while at prayer.
\switchcolumn*
\selectlanguage{latin}
Córdubæ, in Hispánia, 
 sanctæ Pompósæ, Vírginis et Mártyris; quæ in persecutióne Arábica, ob impávidam Christi confessiónem decolláta gládio, palmam consecúta est.
\switchcolumn
\selectlanguage{english}
At Cordova in Spain, St. Pomposa, 
 virgin and martyr. Because of her fearless witness to Christ she was 
 beheaded in the Arab persecution, and thus obtained the palm of martyrdom.
\switchcolumn*
\selectlanguage{latin}
Cantuáriæ sancti 
 Theodóri Epíscopi, qui, a beáto Vitaliáno Papa in Angliam missus, doctrína 
 et sanctitáte refúlsit.
\switchcolumn
\selectlanguage{english}
At Canterbury, the holy bishop 
 Theodore, who was sent to England by Pope Vitalian, and who was renowned for 
 learning and holiness.
\switchcolumn*
\selectlanguage{latin}
Turónis, in Gállia, 
 sancti Eustóchii Epíscopi, magnárum virtútum viri.
\switchcolumn
\selectlanguage{english}
At Tours in France, St. Eustochius, 
 bishop, a man of great virtue.
\switchcolumn*
\selectlanguage{latin}
In território 
 Lingoniénsi sancti Sequáni, Presbyteri et Confessóris.
\switchcolumn
\selectlanguage{english}
In the diocese of Langres, St. 
 Sequanus, priest and confessor.
\switchcolumn*
\selectlanguage{latin}
Barcinóne, in Hispánia, 
 beátæ Maríæ de Cervellióne, ex Ordine beátæ Maríæ de Mercéde redemptiónis 
 captivórum, Vírginis; quæ, ob præséntem quam invocántibus confert opem, 
 María de Subsídio vulgo nuncupátur.
\switchcolumn
\selectlanguage{english}
At Barcelona in Spain, blessed Mary 
 de Cervellione, virgin, of the Order of Our Lady of Ransom. She is 
 commonly called Mary of Help on account of the prompt assistance she renders 
 to those who invoke her.
\switchcolumn*
\selectlanguage{latin}
In vico Druelle, in 
 diœcési Ruthenénsi, in Gállia, sanctæ Maríæ Guliélmæ-Æmíliæ de Rodat, 
 Vírginis, Congregatiónis Sorórum a sancta Família Fundatrícis, puéllis 
 erudiéndis et egénis sublevándis addictíssimæ, quæ a Pio Duodécimo, 
 Pontífice Máximo, inter sanctas Vírgines reláta est.
\switchcolumn
\selectlanguage{english}
In the village of Druelle, in the 
 diocese of Rodez in France, St. Marie Guillemette Emilie de Rodat, virgin, 
 and foundress of the Congregation of Sisters of the Holy Family, which was 
 established to teach poor and orphaned girls. Pius XII added her name 
 to the number of holy virgins.
\switchcolumn*
\selectlanguage{latin}
\end{paracol}


% ---- martyrology/mart09/mart0920.htm
\needspace{10\baselineskip}
\begin{paracol}{2}
\selectlanguage{latin}
\begin{center}{\color{gregoriocolor} Duodécimo Kaléndas Octóbris. 
 Luna\dots\ }\end{center}
\switchcolumn
\selectlanguage{english}
\begin{center}{\color{gregoriocolor} The   Twentieth Day of 
 September. The\dots\ Day of the Moon.}\end{center}
\end{paracol}

\noindent\begin{tabularx}{\linewidth}{*{19}{>{\centering\arraybackslash}X}}
 \textcolor{gregoriocolor}{a} & \textcolor{gregoriocolor}{b} & \textcolor{gregoriocolor}{c} & \textcolor{gregoriocolor}{d} & \textcolor{gregoriocolor}{e} & \textcolor{gregoriocolor}{f} & \textcolor{gregoriocolor}{g} & \textcolor{gregoriocolor}{h} & \textcolor{gregoriocolor}{i} & \textcolor{gregoriocolor}{k} & \textcolor{gregoriocolor}{l} & \textcolor{gregoriocolor}{m} & \textcolor{gregoriocolor}{n} & \textcolor{gregoriocolor}{p} & \textcolor{gregoriocolor}{q} & \textcolor{gregoriocolor}{r} & \textcolor{gregoriocolor}{s} & \textcolor{gregoriocolor}{t} & \textcolor{gregoriocolor}{u} \\
 28 & 29 & 30 & 1 & 2 & 3 & 4 & 5 & 6 & 7 & 8 & 9 & 10 & 11 & 12 & 13 & 14 & 15 & 16 \\
\end{tabularx}
\vspace{0.5\baselineskip}
\noindent\begin{tabularx}{\linewidth}{*{12}{>{\centering\arraybackslash}X}}
 \textcolor{gregoriocolor}{A} & \textcolor{gregoriocolor}{B} & \textcolor{gregoriocolor}{C} & \textcolor{gregoriocolor}{D} & \textcolor{gregoriocolor}{E} & F & \textcolor{gregoriocolor}{F} & \textcolor{gregoriocolor}{G} & \textcolor{gregoriocolor}{H} & \textcolor{gregoriocolor}{M} & \textcolor{gregoriocolor}{N} & \textcolor{gregoriocolor}{P} \\
 17 & 18 & 19 & 20 & 21 & 22 & 22 & 23 & 24 & 25 & 26 & 27 \\
\end{tabularx}

\begin{paracol}{2}
\selectlanguage{latin}
\lettrine[lines=1]{V}{igília} sancti Matthæi, Apóstoli et Evangelístæ.
\switchcolumn
\selectlanguage{english}
\lettrine[lines=1]{T}{he} vigil of St. Matthew, apostle and evangelist.
\switchcolumn*
\selectlanguage{latin}
Romæ pássio sanctórum 
 Mártyrum Eustáchii et Theopístis uxóris, cum duóbus fíliis Agapíto et 
 Theopísto, qui, sub Hadriáno Imperatóre, damnáti ad béstias, sed Dei ope ab 
 iis nullátenus læsi, tandem, in bovem æneum candéntem inclúsi, martyrium 
 consummárunt.
\switchcolumn
\selectlanguage{english}
At Rome, the holy martyrs Eustace, 
 and Theopistes, his wife, with their two sons, Agapitus and Theopistus. 
 Under Emperor Hadrian they were condemned to be cast to the beasts, but by 
 the power of God they were uninjured by them, so they were shut up in a 
 heated brazen ox, and thus completed their martyrdom.
\switchcolumn*
\selectlanguage{latin}
Cyzici, in Propóntide, 
 natális sanctórum Mártyrum Faustæ Vírginis, et Evilásii, sub Maximiáno 
 Imperatóre; e quibus Fausta, ab eódem Evilásio, idolórum sacerdóte, 
 decalváta et ad turpitúdinem rasa, suspénsa et torta est. Deínde, cum 
 eam vellet médiam secáre, et carnífices lǽdere non valérent, stupens 
 crédidit in Christum Evilásius; et, dum ipse quoque, Imperatóris jussu, 
 fórtiter torquerétur, Fausta, cápite terebráta, clavis toto córpore confíxa 
 et sartágine ignítæ impósita, tandem, cum eódem Evilásio, illam voce de 
 cælis vocánte, transívit ad Dóminum.
\switchcolumn
\selectlanguage{english}
At Cyzicum, on the sea of Marmora, 
 the birthday of the holy martyrs Evilasius and the virgin Fausta, in the 
 time of Emperor Maximian. Fausta's head was shaved to shame her, and 
 she was hung up and tortured by Evilasius, then a pagan priest. But 
 when he wished to have her body cut in two, the executioners could not 
 inflict any injury upon her. Amazed at this prodigy, Evilasius 
 believed in Christ and was cruelly tortured by order of the emperor; at the 
 same time Fausta had her head bored through and her whole body pierced with 
 nails. She was then laid on a heated gridiron, and being called by a 
 celestial voice, went in company with Evilasius to enjoy the blessedness of 
 heaven.
\switchcolumn*
\selectlanguage{latin}
In Phrygia sanctórum 
 Mártyrum Dionysii et Priváti.
\switchcolumn
\selectlanguage{english}
In Phrygia, the holy martyrs Denis 
 and Privatus.
\switchcolumn*
\selectlanguage{latin}
Item sancti Prisci 
 Mártyris, qui, punctim pugiónibus transverberátus, cápite plexus est.
\switchcolumn
\selectlanguage{english}
Also St. Priscus, martyr, whose body 
 was pierced throughout with daggers, after which he was beheaded.
\switchcolumn*
\selectlanguage{latin}
Perge, in Pamphylia, 
 sanctórum Theodóri, et Philíppæ matris, ac Sociórum Mártyrum, sub Antoníno 
 Imperatóre.
\switchcolumn
\selectlanguage{english}
At Pergen in Pamphylia, the Saints 
 Theodore, his mother Philippa, and their fellow martyrs, in the time of 
 Emperor Antoninus.
\switchcolumn*
\selectlanguage{latin}
Carthágine sanctæ 
 Cándidæ, Vírginis et Mártyris; quæ, sub Maximiáno Imperatóre, plagis toto 
 córpore dilaceráta, martyrio coronátur.
\switchcolumn
\selectlanguage{english}
At Carthage, under Emperor Maximian, 
 St. Candida, virgin and martyr. After her body was lacerated by whips 
 she was crowned with martyrdom.
\switchcolumn*
\selectlanguage{latin}
Medioláni sancti 
 Clicérii, Epíscopi et Confessóris.
\switchcolumn
\selectlanguage{english}
At Milan, St. Clicerius, bishop and 
 confessor.
\switchcolumn*
\selectlanguage{latin}
Romæ Translátio 
 córporis sancti Agapíti Primi, Papæ et Confessóris, ex urbe Constantinópoli, 
 in qua Póntifex décimo Kaléndas Maji obdormíerat in Dómino.
\switchcolumn
\selectlanguage{english}
At Rome, the translation of the body 
 of St. Agapitus I, pope and confessor, from the city of Constantinople, in 
 which he died on the 22nd of April.
\switchcolumn*
\selectlanguage{latin}
\end{paracol}


% ---- martyrology/mart09/mart0921.htm
\needspace{10\baselineskip}
\begin{paracol}{2}
\selectlanguage{latin}
\begin{center}{\color{gregoriocolor} Undécimo Kaléndas Octóbris. 
 Luna\dots\ }\end{center}
\switchcolumn
\selectlanguage{english}
\begin{center}{\color{gregoriocolor} The   Twenty-First Day of 
 September. The\dots\ Day of the Moon.}\end{center}
\end{paracol}

\noindent\begin{tabularx}{\linewidth}{*{19}{>{\centering\arraybackslash}X}}
 \textcolor{gregoriocolor}{a} & \textcolor{gregoriocolor}{b} & \textcolor{gregoriocolor}{c} & \textcolor{gregoriocolor}{d} & \textcolor{gregoriocolor}{e} & \textcolor{gregoriocolor}{f} & \textcolor{gregoriocolor}{g} & \textcolor{gregoriocolor}{h} & \textcolor{gregoriocolor}{i} & \textcolor{gregoriocolor}{k} & \textcolor{gregoriocolor}{l} & \textcolor{gregoriocolor}{m} & \textcolor{gregoriocolor}{n} & \textcolor{gregoriocolor}{p} & \textcolor{gregoriocolor}{q} & \textcolor{gregoriocolor}{r} & \textcolor{gregoriocolor}{s} & \textcolor{gregoriocolor}{t} & \textcolor{gregoriocolor}{u} \\
 29 & 30 & 1 & 2 & 3 & 4 & 5 & 6 & 7 & 8 & 9 & 10 & 11 & 12 & 13 & 14 & 15 & 16 & 17 \\
\end{tabularx}
\vspace{0.5\baselineskip}
\noindent\begin{tabularx}{\linewidth}{*{12}{>{\centering\arraybackslash}X}}
 \textcolor{gregoriocolor}{A} & \textcolor{gregoriocolor}{B} & \textcolor{gregoriocolor}{C} & \textcolor{gregoriocolor}{D} & \textcolor{gregoriocolor}{E} & F & \textcolor{gregoriocolor}{F} & \textcolor{gregoriocolor}{G} & \textcolor{gregoriocolor}{H} & \textcolor{gregoriocolor}{M} & \textcolor{gregoriocolor}{N} & \textcolor{gregoriocolor}{P} \\
 18 & 19 & 20 & 21 & 22 & 23 & 23 & 24 & 25 & 26 & 27 & 28 \\
\end{tabularx}

\begin{paracol}{2}
\selectlanguage{latin}
\lettrine[lines=2]{I}{n} Æthiópia natális 
 sancti Matthæi, Apóstoli et Evangelístæ; qui, in ea regióne prædicans, 
 martyrium passus est. Hujus Evangélium, Hebræo sermóne conscríptum, 
 ipso Matthæo revelánte, invéntum est, una cum córpore beáti Bárnabæ Apóstoli, 
 témpore Zenónis Imperatóris.
\switchcolumn
\selectlanguage{english}
\lettrine[lines=2]{T}{he} birthday of St. Matthew, apostle 
 and evangelist, who suffered martyrdom in Ethiopia while engaged in 
 preaching. The Gospel written by him in Hebrew was found by his own 
 revelation during the time of Emperor Zeno, together with the relics of the 
 blessed apostle Barnabas.
\switchcolumn*
\selectlanguage{latin}
In terra Saar sancti 
 Jonæ Prophétæ, qui sepúltus est in Geth.
\switchcolumn
\selectlanguage{english}
In the land of the Saar, the holy 
 prophet Jonas, who was buried in Geth.
\switchcolumn*
\selectlanguage{latin}
In Æthiópia sanctæ 
 Iphigéniæ Vírginis, quæ, baptizáta a beáto Matthæo Apóstolo et Deo dicáta, 
 sancto fine quiévit.
\switchcolumn
\selectlanguage{english}
In Ethiopia, St. Iphigenia, virgin, 
 who was baptized and consecrated to God by the blessed apostle Matthew, and 
 who ended her holy life in peace.
\switchcolumn*
\selectlanguage{latin}
Romæ sancti Pámphili 
 Mártyris.
\switchcolumn
\selectlanguage{english}
At Rome, St Pamphilius, martyr.
\switchcolumn*
\selectlanguage{latin}
Eódem die, via Cláudia, vigésimo ab Urbe milliário, pássio sancti Alexándri Epíscopi, qui, sub 
 Antoníno Imperatóre, pro Christi fide, víncula fustes, equúleum, lámpades 
 ardéntes, ungulárum laniatiónem, béstias ac fornácis superávit flammas, ac 
 tandem, gládio cæsus, vitam adéptus est gloriósam. Ejus corpus póstea 
 beátus Dámasus Papa in Urbem tránstulit sexto Kaléndas Decémbris.
\switchcolumn
\selectlanguage{english}
On the Claudian Way, twenty miles 
 from Rome, in the time of Emperor Antoninus, the martyrdom of St. Alexander, 
 bishop. For the faith of Christ he was loaded with fetters, scourged, 
 tortured, burned with torches, torn with iron hooks, exposed to the beasts, 
 and cast into the flames, but having overcome all these torments, he was 
 finally beheaded, and thus attained the glory of eternal life. His 
 body was afterwards carried into the city by blessed Pope Damasus on the 
 26th of November.
\switchcolumn*
\selectlanguage{latin}
In Cypro sancti Isácii, 
 Epíscopi et Mártyris.
\switchcolumn
\selectlanguage{english}
In Cyprus, St. Isacius, bishop and 
 martyr.
\switchcolumn*
\selectlanguage{latin}
In Phœnícia sancti 
 Eusébii Mártyris, qui, cum ultro Præféctum adísset et se Christiánum esse 
 denuntiásset, ab eo, multis torméntis afflíctus, cápite cæsus est.
\switchcolumn
\selectlanguage{english}
In Phoenicia, St. Eusebius, martyr, 
 who of his own accord went to the prefect and declared himself a Christian. 
 He was subjected by him to many torments, and finally beheaded.
\switchcolumn*
\selectlanguage{latin}
In Cypro sancti Melétii 
 Epíscopi et Confessóris.
\switchcolumn
\selectlanguage{english}
In Cyprus, St. Meletius, bishop and 
 confessor.
\switchcolumn*
\selectlanguage{latin}
\end{paracol}


% ---- martyrology/mart09/mart0922.htm
\needspace{10\baselineskip}
\begin{paracol}{2}
\selectlanguage{latin}
\begin{center}{\color{gregoriocolor} Décimo Kaléndas Octóbris. 
 Luna\dots\ }\end{center}
\switchcolumn
\selectlanguage{english}
\begin{center}{\color{gregoriocolor} The   Twenty-Second Day of 
 September. The\dots\ Day of the Moon.}\end{center}
\end{paracol}

\noindent\begin{tabularx}{\linewidth}{*{19}{>{\centering\arraybackslash}X}}
 \textcolor{gregoriocolor}{a} & \textcolor{gregoriocolor}{b} & \textcolor{gregoriocolor}{c} & \textcolor{gregoriocolor}{d} & \textcolor{gregoriocolor}{e} & \textcolor{gregoriocolor}{f} & \textcolor{gregoriocolor}{g} & \textcolor{gregoriocolor}{h} & \textcolor{gregoriocolor}{i} & \textcolor{gregoriocolor}{k} & \textcolor{gregoriocolor}{l} & \textcolor{gregoriocolor}{m} & \textcolor{gregoriocolor}{n} & \textcolor{gregoriocolor}{p} & \textcolor{gregoriocolor}{q} & \textcolor{gregoriocolor}{r} & \textcolor{gregoriocolor}{s} & \textcolor{gregoriocolor}{t} & \textcolor{gregoriocolor}{u} \\
 30 & 1 & 2 & 3 & 4 & 5 & 6 & 7 & 8 & 9 & 10 & 11 & 12 & 13 & 14 & 15 & 16 & 17 & 18 \\
\end{tabularx}
\vspace{0.5\baselineskip}
\noindent\begin{tabularx}{\linewidth}{*{12}{>{\centering\arraybackslash}X}}
 \textcolor{gregoriocolor}{A} & \textcolor{gregoriocolor}{B} & \textcolor{gregoriocolor}{C} & \textcolor{gregoriocolor}{D} & \textcolor{gregoriocolor}{E} & F & \textcolor{gregoriocolor}{F} & \textcolor{gregoriocolor}{G} & \textcolor{gregoriocolor}{H} & \textcolor{gregoriocolor}{M} & \textcolor{gregoriocolor}{N} & \textcolor{gregoriocolor}{P} \\
 19 & 20 & 21 & 22 & 23 & 24 & 24 & 25 & 26 & 27 & 28 & 29 \\
\end{tabularx}

\begin{paracol}{2}
\selectlanguage{latin}
\lettrine[lines=2]{S}{ancti} Thomæ a Villa 
 Nova, ex Eremitárum sancti Augustíni Ordine, Epíscopi Valentíni et 
 Confessóris; cujus dies natális recólitur sexto Idus Septémbris.
\switchcolumn
\selectlanguage{english}
\lettrine[lines=2]{S}{t.} Thomas of Villanova, of the 
 Order of Hermits of St. Augustine, archbishop of Valencia and confessor, 
 whose birthday is the 8th of September.
\switchcolumn*
\selectlanguage{latin}
Sedúni, in Gállia, in 
 loco Agáuno, natális sanctórum Mártyrum Thebæórum Maurítii, Exsupérii, 
 Cándidi, Victóris, Innocéntii et Vitális, cum Sóciis ejúsdem legiónis; qui, 
 sub Maximiáno, pro Christo necáti, gloriósa passióne mundum illustrárunt.
\switchcolumn
\selectlanguage{english}
At St. Maurice, near Sion in 
 Switzerland, the birthday of the holy Theban martyrs Maurice, Exuperius, 
 Candidus, Victor, Innocent, and Vitalis, with their companions of the same 
 legion, whose martyrdom for the faith during the time of Maximian filled the 
 world with the glory of their sufferings.
\switchcolumn*
\selectlanguage{latin}
Romæ pássio sanctárum 
 Vírginum et Mártyrum Dignæ et Eméritæ, sub Valeriáno et Gallieno; quarum 
 relíquiæ in Ecclésia sancti Marcélli asservántur.
\switchcolumn
\selectlanguage{english}
At Rome, the martyrdom of the holy 
 virgins and martyrs Digna and Emerita, under Valerian and Gallienus. 
 Their relics are kept in the church of St. Marcellus.
\switchcolumn*
\selectlanguage{latin}
Ratisbónæ, in Bavária, sancti Emmerámi, Epíscopi et Mártyris; qui, ut álios liberáret, mortem 
 sævíssimam, Christi causa, patiénter súbiit.
\switchcolumn
\selectlanguage{english}
At Ratisbon in Bavaria, St. 
 Emmeramus, bishop and martyr, who patiently endured a most cruel death for 
 the sake of our Lord, in order to set others free.
\switchcolumn*
\selectlanguage{latin}
Apud pagum Castrénsium 
 sancti Jonæ, Presbyteri et Mártyris; qui, cum sancto Dionysio proféctus in 
 Gálliam, ibídem, Juliáni Præfécti jussu verbéribus cæsus, gládio martyrium 
 consummávit.
\switchcolumn
\selectlanguage{english}
At Arpajon, near Paris, St. Jonas, 
 priest and martyr, who went to France along with St. Denis. After he 
 was scourged by the order of the prefect Julian, his martyrdom was ended by 
 the sword.
\switchcolumn*
\selectlanguage{latin}
Antinoópoli, in Ægypto, 
 sanctæ Iráidis, Vírginis Alexandrínæ, et Sociórum Mártyrum. Ipsa 
 Virgo, cum esset ad hauriéndam e próximo fonte aquam egréssa, et navim 
 vidísset Confessóribus Christi onústam, prótinus, relícta hydria, se illis 
 adjúnxit, ac, simul cum iis in urbem ducta, prima ómnium, post multa 
 supplícia, cápite cæsa est; deínde Presbyteri, Diáconi, et Vírgines, aliíque 
 omnes eódem mortis génere consúmpti sunt.
\switchcolumn
\selectlanguage{english}
At Antinopolis in Egypt, the holy 
 martyrs Irais, an Alexandrian virgin, and her companions. Having gone 
 out to draw water at a near-by fountain, and seeing a boat loaded with 
 Christian confessors, she immediately left her vessel and joined them. 
 She was conducted to the city with them, and after many torments she was the 
 first to have her head struck off. After her, priests, deacons, 
 virgins, and all others underwent the same kind of death.
\switchcolumn*
\selectlanguage{latin}
Romæ sancti Felícis 
 Papæ Quarti, qui pro fide cathólica plúrimum laborávit.
\switchcolumn
\selectlanguage{english}
At Rome, Pope St. Felix IV, who 
 laboured exceedingly for the Catholic faith.
\switchcolumn*
\selectlanguage{latin}
Apud civitátem 
 Meldénsem beáti Sanctíni Epíscopi, qui fuit discípulus sancti Dionysii 
 Areopaíitæ; et ejúsdem civitátis Epíscopus ab eo consecrátus, primus illic 
 Evangélium prædicávit.
\switchcolumn
\selectlanguage{english}
At Meaux, blessed Sanctinus, bishop, 
 a disciple of St. Denis the Areopagite, by whom he was consecrated bishop of 
 that city, and was the first to preach the Gospel there.
\switchcolumn*
\selectlanguage{latin}
In território 
 Constantiénsi, in Gállia, sancti Lautónis Epíscopi.
\switchcolumn
\selectlanguage{english}
In the territory of Coutances, St. 
 Lauto, bishop.
\switchcolumn*
\selectlanguage{latin}
In monte Glonna, ad 
 Lígerim flumen, in Gállia, sancti Floréntii Presbyteri.
\switchcolumn
\selectlanguage{english}
At Mount Glonna in France, the holy 
 priest Florentius.
\switchcolumn*
\selectlanguage{latin}
In óppido cui Leprósii 
 nomen, in território Bituricénsi, sancti Silváni Confessóris.
\switchcolumn
\selectlanguage{english}
In the territory of Bourges, St. 
 Sylvanus, confessor.
\switchcolumn*
\selectlanguage{latin}
Laudúni, in Gállia, 
 sanctæ Salabérgæ Abbatíssæ.
\switchcolumn
\selectlanguage{english}
At Laon in France, St. Salaberga, 
 abbess.
\switchcolumn*
\selectlanguage{latin}
\end{paracol}


% ---- martyrology/mart09/mart0923.htm
\needspace{10\baselineskip}
\begin{paracol}{2}
\selectlanguage{latin}
\begin{center}{\color{gregoriocolor} Nono Kaléndas Octóbris. 
 Luna\dots\ }\end{center}
\switchcolumn
\selectlanguage{english}
\begin{center}{\color{gregoriocolor} The   Twenty-Third Day of 
 September. The\dots\ Day of the Moon.}\end{center}
\end{paracol}

\noindent\begin{tabularx}{\linewidth}{*{19}{>{\centering\arraybackslash}X}}
 \textcolor{gregoriocolor}{a} & \textcolor{gregoriocolor}{b} & \textcolor{gregoriocolor}{c} & \textcolor{gregoriocolor}{d} & \textcolor{gregoriocolor}{e} & \textcolor{gregoriocolor}{f} & \textcolor{gregoriocolor}{g} & \textcolor{gregoriocolor}{h} & \textcolor{gregoriocolor}{i} & \textcolor{gregoriocolor}{k} & \textcolor{gregoriocolor}{l} & \textcolor{gregoriocolor}{m} & \textcolor{gregoriocolor}{n} & \textcolor{gregoriocolor}{p} & \textcolor{gregoriocolor}{q} & \textcolor{gregoriocolor}{r} & \textcolor{gregoriocolor}{s} & \textcolor{gregoriocolor}{t} & \textcolor{gregoriocolor}{u} \\
 1 & 2 & 3 & 4 & 5 & 6 & 7 & 8 & 9 & 10 & 11 & 12 & 13 & 14 & 15 & 16 & 17 & 18 & 19 \\
\end{tabularx}
\vspace{0.5\baselineskip}
\noindent\begin{tabularx}{\linewidth}{*{12}{>{\centering\arraybackslash}X}}
 \textcolor{gregoriocolor}{A} & \textcolor{gregoriocolor}{B} & \textcolor{gregoriocolor}{C} & \textcolor{gregoriocolor}{D} & \textcolor{gregoriocolor}{E} & F & \textcolor{gregoriocolor}{F} & \textcolor{gregoriocolor}{G} & \textcolor{gregoriocolor}{H} & \textcolor{gregoriocolor}{M} & \textcolor{gregoriocolor}{N} & \textcolor{gregoriocolor}{P} \\
 20 & 21 & 22 & 23 & 24 & 25 & 25 & 26 & 27 & 28 & 29 & 30 \\
\end{tabularx}

\begin{paracol}{2}
\selectlanguage{latin}
\lettrine[lines=2]{R}{omæ} sancti Lini, Papæ 
 et Mártyris, qui, primus post beátum Petrum Apóstolum, Románum Ecclésiam 
 gubernávit, et, martyrio coronátus, sepúltus est in Vaticáno, prope eúndem 
 Apóstolum.
\switchcolumn
\selectlanguage{english}
\lettrine[lines=2]{A}{t} Rome, St. Linus, pope and martyr, 
 who governed the Roman Church next after the blessed apostle Peter. He 
 was crowned with martyrdom and was buried on the Vatican Hill beside the 
 same apostle.
\switchcolumn*
\selectlanguage{latin}
Icónii, in Lycaónia, 
 sanctæ Theclæ, Vírginis et Mártyris; quæ, a sancto Paulo Apóstolo ad fidem 
 perdúcta, ignes ac béstias, sub Neróne Imperatóre, in Christi confessióne 
 devícit; et, post plúrima ad multórum doctrínam superáta certámina, venit 
 Seleucíam, ibíque requiévit in pace. Ipsam vero sancti Patres summis 
 láudibus celebrárunt.
\switchcolumn
\selectlanguage{english}
At Iconium in Lycaonia, St. Thecla, 
 virgin and martyr, who was brought to the faith by the apostle St. Paul. 
 Under Emperor Nero, she was victorious over the flames and the beasts to 
 which she was exposed for the faith of Christ. After many combats 
 endured for the instruction of others, she went to Seleucia, where she ended 
 her days in peace. Her memory has been eulogized by the holy Fathers.
\switchcolumn*
\selectlanguage{latin}
In Hispánia sanctárum 
 mulíerum Xantíppæ et Polyxenæ, quæ fuérunt Apostolórum discípulæ.
\switchcolumn
\selectlanguage{english}
In Spain, the holy women Xantippa 
 and Polyxena, who were disciples of the apostles.
\switchcolumn*
\selectlanguage{latin}
In Africa sanctórum 
 Mártyrum Andréæ, Joánnis, Petri et Antónii.
\switchcolumn
\selectlanguage{english}
In Africa, the holy martyrs Andrew, 
 John, Peter and Anthony.
\switchcolumn*
\selectlanguage{latin}
Ancónæ sancti 
 Constántii, Ecclésiæ Mansionárii, miraculórum grátia conspícui.
\switchcolumn
\selectlanguage{english}
At Ancona, St. Constantius, 
 sacristan of the church, renowned for the gift of miracles.
\switchcolumn*
\selectlanguage{latin}
In Campánia 
 commemorátio beáti Sósii, Diáconi, Misenátis, quem sanctus Epíscopus 
 Januárius, cum de cápite illíus, Evangélium in Ecclésia legéntis, flammam 
 vidéret exsúrgere, Mártyrem futúrum præannuntiávit; et, non post multos 
 dies, ipse Sósius, cum esset annórum trigínta, cápitis detruncatióne 
 martyrium cum eódem Epíscopo suscépit.
\switchcolumn
\selectlanguage{english}
In Campania, the commemoration of 
 the blessed Sosius, deacon of the church of Miseno. The holy bishop 
 Januarius, upon seeing a flame arise from his head as he was reading the 
 Gospel in the church, foretold that he would be a martyr. Not many 
 days after, when he was thirty years of age, he and the holy bishop suffered 
 martyrdom by beheading.
\switchcolumn*
\selectlanguage{latin}
Sescíaci, in 
 Constantiénsi Gálliæ território, item Commemorátio sancti Patérni, Epíscopi 
 Abrincénsis et Confessóris; cujus dies natális sextodécimo Kaléndas Maji 
 recólitur.
\switchcolumn
\selectlanguage{english}
At Scicy in the district of 
 Coutances in France, the commemoration of St. Paternus, bishop of Avranches 
 and confessor, whose birthday is recalled on the 16th of April.
\switchcolumn*
\selectlanguage{latin}
\end{paracol}


% ---- martyrology/mart09/mart0924.htm
\needspace{10\baselineskip}
\begin{paracol}{2}
\selectlanguage{latin}
\begin{center}{\color{gregoriocolor} Octávo Kaléndas Octóbris. 
 Luna\dots\ }\end{center}
\switchcolumn
\selectlanguage{english}
\begin{center}{\color{gregoriocolor} The   Twenty-Fourth Day of 
 September. The\dots\ Day of the Moon.}\end{center}
\end{paracol}

\noindent\begin{tabularx}{\linewidth}{*{19}{>{\centering\arraybackslash}X}}
 \textcolor{gregoriocolor}{a} & \textcolor{gregoriocolor}{b} & \textcolor{gregoriocolor}{c} & \textcolor{gregoriocolor}{d} & \textcolor{gregoriocolor}{e} & \textcolor{gregoriocolor}{f} & \textcolor{gregoriocolor}{g} & \textcolor{gregoriocolor}{h} & \textcolor{gregoriocolor}{i} & \textcolor{gregoriocolor}{k} & \textcolor{gregoriocolor}{l} & \textcolor{gregoriocolor}{m} & \textcolor{gregoriocolor}{n} & \textcolor{gregoriocolor}{p} & \textcolor{gregoriocolor}{q} & \textcolor{gregoriocolor}{r} & \textcolor{gregoriocolor}{s} & \textcolor{gregoriocolor}{t} & \textcolor{gregoriocolor}{u} \\
 2 & 3 & 4 & 5 & 6 & 7 & 8 & 9 & 10 & 11 & 12 & 13 & 14 & 15 & 16 & 17 & 18 & 19 & 20 \\
\end{tabularx}
\vspace{0.5\baselineskip}
\noindent\begin{tabularx}{\linewidth}{*{12}{>{\centering\arraybackslash}X}}
 \textcolor{gregoriocolor}{A} & \textcolor{gregoriocolor}{B} & \textcolor{gregoriocolor}{C} & \textcolor{gregoriocolor}{D} & \textcolor{gregoriocolor}{E} & F & \textcolor{gregoriocolor}{F} & \textcolor{gregoriocolor}{G} & \textcolor{gregoriocolor}{H} & \textcolor{gregoriocolor}{M} & \textcolor{gregoriocolor}{N} & \textcolor{gregoriocolor}{P} \\
 21 & 22 & 23 & 24 & 25 & 26 & 26 & 27 & 28 & 29 & 30 & 1 \\
\end{tabularx}

\begin{paracol}{2}
\selectlanguage{latin}
\lettrine[lines=2]{F}{estum} beátæ Maríæ 
 Vírginis de Mercéde nuncupátæ, Ordinis redemptiónis captivórum sub ejus 
 nómine Institutrícis, de cujus Apparitióne ágitur quarto Idus Augústi.
\switchcolumn
\selectlanguage{english}
\lettrine[lines=2]{T}{he} feast of our Lady of Ransom, 
 Foundress of the Order for the Redemption of Captives. The apparition 
 of the same Blessed Virgin occurred on the 10th of August.
\switchcolumn*
\selectlanguage{latin}
Bríxiæ deposítio sancti 
 Anathalónis Epíscopi, qui, beáti Bárnabæ Apóstoli discípulus, in ejus locum 
 Epíscopus Ecclésiæ Mediolanénsis succéssit.
\switchcolumn
\selectlanguage{english}
At Brescia, the death of St. 
 Anathalo, bishop. He was a disciple of the blessed apostle Barnabas, 
 and succeeded him as bishop of the Milanese church.
\switchcolumn*
\selectlanguage{latin}
In Pannónia sancti 
 Gerárdi, Epíscopi Morisénæ sedis et Mártyris, Hungarórum Apóstoli nuncupáti, 
 patrícii Véneti; qui, cum e Chanadiénsi óppido Albam Regálem se conférret, 
 prope flumen Danúbium ab infidélibus impetitus, lapídibus óbrutus ac tandem 
 láncea transfíxus occúbuit, sicque primus pátriam nóbili martyrio 
 illustrávit.
\switchcolumn
\selectlanguage{english}
In Hungary, St. Gerard, bishop of 
 Chzonad and martyr, patrician of Venice, called the apostle of the 
 Hungarians. During a journey from the town of Chzonad to Alba Regalis 
 he was attacked by the pagans near the river Danube, stoned by them, and 
 then pierced with a lance. He was thus the first to adorn his native 
 land with a noble martyrdom.
\switchcolumn*
\selectlanguage{latin}
Augustodúni natális 
 sanctórum Mártyrum Andóchii Presbyteri, Thyrsi Diáconi, et Felícis. 
 Hi, a beáto Polycárpo, Smyrnénsi Epíscopo, ab Oriénte dirécti ad docéndam 
 Gálliam, ibídem flagéllis duríssime cæsi, ac tota die invérsis mánibus 
 suspénsi, et in ignem missi sunt, sed non combústi; tandem eórum colla 
 véctibus feriúntur, et ita Mártyres gloriosíssime coronántur.
\switchcolumn
\selectlanguage{english}
At Autun, the birthday of the holy 
 martyrs Andochius, a priest, Thyrsus, a deacon, and Felix, who were sent 
 from the East by blessed Polycarp, bishop of Smyrna, to preach in France. 
 There they were severely scourged, hanged by the hands for a whole day, and 
 cast into the fire. Remaining uninjured, they had their necks broken 
 with heavy bars, and thus won a most glorious crown.
\switchcolumn*
\selectlanguage{latin}
In Ægypto pássio 
 sanctórum Paphnútii et Sociórum Mártyrum. Ipse, vitam in solitúdine 
 agens, cum audíret multos Christiános in vínculis retinéri, sponte, divíno 
 Spíritu cóncitus, Præfécto se offert, et Christiánam religiónem líbere 
 profitétur; a quo primum caténis férreis constríngitur, et in equúleo 
 diutíssime torquétur, deínde cum áliis plúrimis ad Diocletiánum míttitur, 
 cujus jussu, ipse palmæ affígitur, céteri autem ferro necántur.
\switchcolumn
\selectlanguage{english}
In Egypt, the holy martyrs 
 Paphnutius and his companions. While leading a solitary life, St. 
 Paphnutius heard that many Christians were kept in bonds. Moved by the 
 spirit of God, he voluntarily offered himself to the prefect, and freely 
 confessed the Christian faith. He was bound by him with iron chains, 
 and for a long time tortured on the rack. Then, being sent with many 
 others to Diocletian, by his order he was fastened to a palm tree, and the 
 rest were struck with the sword.
\switchcolumn*
\selectlanguage{latin}
Chalcédone sanctórum 
 quadragínta novem Mártyrum, qui, post martyrium sanctæ Euphémiæ, sub 
 Diocletiáno Imperatóre, damnáti ad béstias, et, cum ab iis divínitus líberi 
 evasíssent, demum, gládio percússi, migravérunt in cælum.
\switchcolumn
\selectlanguage{english}
At Chalcedon, under Emperor 
 Diocletian, after the martyrdom of St. Euphemia, forty-nine holy martyrs who 
 were condemned to be devoured by the beasts, but being miraculously 
 delivered, were finally struck with the sword and went to heaven.
\switchcolumn*
\selectlanguage{latin}
Arvérnis, in Gállia, 
 deposítio sancti Rústici, Epíscopi et Confessóris.
\switchcolumn
\selectlanguage{english}
In Auvergne, the death of St. 
 Rusticus, bishop and confessor.
\switchcolumn*
\selectlanguage{latin}
Flavíaci, in território Bellovacénsi, sancti Geremari, Presbyteri et Abbátis.
\switchcolumn
\selectlanguage{english}
In the diocese of Beauvais, St. 
 Geremarus, priest and abbot.
\switchcolumn*
\selectlanguage{latin}
Septémpedæ, in Picéno, 
 deposítio sancti Pacífici, Sacerdótis ex Ordine Minórum et Confessóris, 
 exímiæ patiéntiæ viri et solitúdinis amóre præclári, quem Gregórius Papa 
 Décimus sextus in Sanctórum cánonem rétulit.
\switchcolumn
\selectlanguage{english}
At San Severino in Piceno, the death 
 of St. Pacificus, priest and confessor of the Order of Friars Minor of St. 
 Francis of the Reformed Observance. Illustrious for his great patience 
 and his love of solitude, he was enrolled in the canon of the saints by Pope 
 Gregory XVI.
\switchcolumn*
\selectlanguage{latin}
\end{paracol}


% ---- martyrology/mart09/mart0925.htm
\needspace{10\baselineskip}
\begin{paracol}{2}
\selectlanguage{latin}
\begin{center}{\color{gregoriocolor} Séptimo Kaléndas Octóbris. 
 Luna\dots\ }\end{center}
\switchcolumn
\selectlanguage{english}
\begin{center}{\color{gregoriocolor} The   Twenty-Fifth Day of 
 September. The\dots\ Day of the Moon.}\end{center}
\end{paracol}

\noindent\begin{tabularx}{\linewidth}{*{19}{>{\centering\arraybackslash}X}}
 \textcolor{gregoriocolor}{a} & \textcolor{gregoriocolor}{b} & \textcolor{gregoriocolor}{c} & \textcolor{gregoriocolor}{d} & \textcolor{gregoriocolor}{e} & \textcolor{gregoriocolor}{f} & \textcolor{gregoriocolor}{g} & \textcolor{gregoriocolor}{h} & \textcolor{gregoriocolor}{i} & \textcolor{gregoriocolor}{k} & \textcolor{gregoriocolor}{l} & \textcolor{gregoriocolor}{m} & \textcolor{gregoriocolor}{n} & \textcolor{gregoriocolor}{p} & \textcolor{gregoriocolor}{q} & \textcolor{gregoriocolor}{r} & \textcolor{gregoriocolor}{s} & \textcolor{gregoriocolor}{t} & \textcolor{gregoriocolor}{u} \\
 3 & 4 & 5 & 6 & 7 & 8 & 9 & 10 & 11 & 12 & 13 & 14 & 15 & 16 & 17 & 18 & 19 & 20 & 21 \\
\end{tabularx}
\vspace{0.5\baselineskip}
\noindent\begin{tabularx}{\linewidth}{*{12}{>{\centering\arraybackslash}X}}
 \textcolor{gregoriocolor}{A} & \textcolor{gregoriocolor}{B} & \textcolor{gregoriocolor}{C} & \textcolor{gregoriocolor}{D} & \textcolor{gregoriocolor}{E} & F & \textcolor{gregoriocolor}{F} & \textcolor{gregoriocolor}{G} & \textcolor{gregoriocolor}{H} & \textcolor{gregoriocolor}{M} & \textcolor{gregoriocolor}{N} & \textcolor{gregoriocolor}{P} \\
 22 & 23 & 24 & 25 & 26 & 27 & 27 & 28 & 29 & 30 & 1 & 2 \\
\end{tabularx}

\begin{paracol}{2}
\selectlanguage{latin}
\lettrine[lines=2]{A}{pud} castéllum Emmaus 
 natális beáti Cléophæ, qui fuit Christi discípulus, quem et in eádem domo in 
 qua mensam Dómino paráverat, pro confessióne illíus a Judæis occísum tradunt, 
 et gloriósa memória sepúltum.
\switchcolumn
\selectlanguage{english}
\lettrine[lines=2]{A}{t} Emmaus, the birthday of blessed 
 Cleophas, disciple of Christ. It is related that he was killed by the 
 Jews for the confession of our Lord, and honourably buried in the same house 
 in which he had entertained him.
\switchcolumn*
\selectlanguage{latin}
Ambiáni, in Gállia, 
 beáti Firmíni Epíscopi, qui, in persecutióne Diocletiáni, sub Rictiováro 
 Præside, post vária torménta, cápitis decollatióne martyrium sumpsit.
\switchcolumn
\selectlanguage{english}
At Amiens in France, in the 
 persecution of Diocletian, blessed Firminus, bishop. Under the 
 governor Rictiovarus, after many torments he suffered martyrdom by being 
 beheaded.
\switchcolumn*
\selectlanguage{latin}
Eódem die, via Cláudia, sancti Herculáni, mílitis et Mártyris; qui, sub Antoníno Imperatóre, 
 miráculis in passióne beáti Alexándri Epíscopi ad Christum convérsus, atque 
 ob fídei confessiónem, post multa torménta, gládio cæsus est.
\switchcolumn
\selectlanguage{english}
At Rome, on the Claudian Way, under 
 Emperor Antoninus, St. Herculanus, soldier and martyr, who was converted to 
 Christ by the miracle wrought during the martyrdom of the blessed bishop 
 Alexander. After enduring many torments he was put to the sword.
\switchcolumn*
\selectlanguage{latin}
Damásci sanctórum 
 Mártyrum Pauli, et Tattæ cónjugis, ac Sabiniáni, Máximi, Rufi et Eugénii 
 filiórum; qui, Christiánæ religiónis accusáti, verbéribus aliísque 
 supplíciis torti sunt, et in cruciátibus ánimas Deo reddidérunt.
\switchcolumn
\selectlanguage{english}
At Damascus, the holy martyrs Paul, 
 his wife Tatta, and their sons Sabinian, Maximus, Rufus, and Eugene. 
 Accused of professing the Christian religion, they were scourged and 
 tortured in other ways until they gave up their souls unto God.
\switchcolumn*
\selectlanguage{latin}
In Asia pássio 
 sanctórum Bardomiáni, Eucárpi et aliórum vigínti sex Mártyrum.
\switchcolumn
\selectlanguage{english}
In Asia, the holy martyrs Bardomian, 
 Eucarpus, and twenty-six others.
\switchcolumn*
\selectlanguage{latin}
Lugdúni, in Gállia, 
 deposítio sancti Lupi, qui ex Anachoréta factus est Epíscopus.
\switchcolumn
\selectlanguage{english}
At Lyons in France, the death of St. 
 Lupus, at one time an anchoret, but later a bishop.
\switchcolumn*
\selectlanguage{latin}
Antisiodóri sancti 
 Anachárii, Epíscopi et Confessóris.
\switchcolumn
\selectlanguage{english}
At Auxerre, St. Anacharius, bishop 
 and confessor.
\switchcolumn*
\selectlanguage{latin}
Blesis, in Gállia, 
 sancti Solémnii, Epíscopus Carnuténsis, miráculis clari.
\switchcolumn
\selectlanguage{english}
At Blois in France, St. Solemnius, 
 bishop of Chartres, renowned for miracles.
\switchcolumn*
\selectlanguage{latin}
Eódem die sancti 
 Princípii, qui fuit Epíscopus Suessionénsis et frater beáti Remígii Epíscopi.
\switchcolumn
\selectlanguage{english}
On the same day, St. Principius, 
 bishop of Soissons, brother of the blessed bishop Remigius.
\switchcolumn*
\selectlanguage{latin}
Anágniæ sanctárum 
 Vírginum Auréliæ et Neomísiæ.
\switchcolumn
\selectlanguage{english}
At Anagni, the holy virgins Aurelia 
 and Neomysia.
\switchcolumn*
\selectlanguage{latin}
\end{paracol}


% ---- martyrology/mart09/mart0926.htm
\needspace{10\baselineskip}
\begin{paracol}{2}
\selectlanguage{latin}
\begin{center}{\color{gregoriocolor} Sexto Kaléndas Octóbris. 
 Luna\dots\ }\end{center}
\switchcolumn
\selectlanguage{english}
\begin{center}{\color{gregoriocolor} The   Twenty-Sixth Day of 
 September. The\dots\ Day of the Moon.}\end{center}
\end{paracol}

\noindent\begin{tabularx}{\linewidth}{*{19}{>{\centering\arraybackslash}X}}
 \textcolor{gregoriocolor}{a} & \textcolor{gregoriocolor}{b} & \textcolor{gregoriocolor}{c} & \textcolor{gregoriocolor}{d} & \textcolor{gregoriocolor}{e} & \textcolor{gregoriocolor}{f} & \textcolor{gregoriocolor}{g} & \textcolor{gregoriocolor}{h} & \textcolor{gregoriocolor}{i} & \textcolor{gregoriocolor}{k} & \textcolor{gregoriocolor}{l} & \textcolor{gregoriocolor}{m} & \textcolor{gregoriocolor}{n} & \textcolor{gregoriocolor}{p} & \textcolor{gregoriocolor}{q} & \textcolor{gregoriocolor}{r} & \textcolor{gregoriocolor}{s} & \textcolor{gregoriocolor}{t} & \textcolor{gregoriocolor}{u} \\
 4 & 5 & 6 & 7 & 8 & 9 & 10 & 11 & 12 & 13 & 14 & 15 & 16 & 17 & 18 & 19 & 20 & 21 & 22 \\
\end{tabularx}
\vspace{0.5\baselineskip}
\noindent\begin{tabularx}{\linewidth}{*{12}{>{\centering\arraybackslash}X}}
 \textcolor{gregoriocolor}{A} & \textcolor{gregoriocolor}{B} & \textcolor{gregoriocolor}{C} & \textcolor{gregoriocolor}{D} & \textcolor{gregoriocolor}{E} & F & \textcolor{gregoriocolor}{F} & \textcolor{gregoriocolor}{G} & \textcolor{gregoriocolor}{H} & \textcolor{gregoriocolor}{M} & \textcolor{gregoriocolor}{N} & \textcolor{gregoriocolor}{P} \\
 23 & 24 & 25 & 26 & 27 & 28 & 28 & 29 & 30 & 1 & 2 & 3 \\
\end{tabularx}

\begin{paracol}{2}
\selectlanguage{latin}
\lettrine[lines=2]{N}{icomedíæ} natális 
 sanctórum Mártyrum Cypriáni et Justínæ Vírginis. Hæc, sub Diocletiáno 
 Imperatóre et Eutólmio Præside, cum multa pro Christo pertulísset, ipsum 
 quoque Cypriánum, qui erat magnus et suis mágicis ártibus eam dementáre 
 conabátur, ad Christiánam fidem convértit; cum quo póstea martyrium sumpsit. 
 Eórum córpora, feris objécta, rapuérunt noctu quidam nautæ Christiáni, et 
 Romam detulérunt; quæ, póstmodum in Basílicam Constantiniánam transláta, 
 prope Baptistérium cóndita sunt.
\switchcolumn
\selectlanguage{english}
\lettrine[lines=2]{A}{t} Nicomedia, the birthday of the 
 holy martyrs Cyprian and the virgin Justina. Under Emperor Diocletian 
 and the governor Eutholmius, Justina suffered greatly for the faith of 
 Christ, and thus converted Cyprian, who, while a magician, had endeavoured 
 to bring her under the influence of his magical practices. She 
 afterwards suffered martyrdom with him. Their bodies were exposed to 
 the beasts, but were taken away in the night by some Christian sailors, and 
 carried to Rome. They were subsequently taken into the Constantinian 
 basilica, and buried near the baptistry.
\switchcolumn*
\selectlanguage{latin}
Romæ sancti Callístrati 
 Mártyris, et aliórum quadragínta novem mílitum; qui mílites, in persecutióne 
 Diocletiáni Imperatóris, cum Callístratus, insútus cúleo et in mare demérsus, 
 divína ope evasísset incólumis, ad Christiánam religiónem convérsi sunt, et 
 cum eo páriter martyrium subiérunt.
\switchcolumn
\selectlanguage{english}
At Rome, in the persecution of 
 Diocletian, the holy martyr Callistratus and forty-nine other soldiers who 
 endured martyrdom together. The companions of Callistratus were 
 converted to Christ upon seeing him miraculously delivered from drowning in 
 the sea, although he had been sewn up in a bag and thrown in.
\switchcolumn*
\selectlanguage{latin}
Bonóniæ sancti Eusébii, 
 Epíscopi et Confessóris.
\switchcolumn
\selectlanguage{english}
At Bologna, St. Eusebius, bishop and 
 confessor.
\switchcolumn*
\selectlanguage{latin}
Bríxiæ sancti Vigílii 
 Epíscopi.
\switchcolumn
\selectlanguage{english}
At Brescia, St. Vigilius, bishop.
\switchcolumn*
\selectlanguage{latin}
In agro Tusculáno beáti 
 Nili Abbátis, qui fundátor monastérii Cryptæ Ferrátæ ac vir magnæ sanctitátis 
 éxstitit.
\switchcolumn
\selectlanguage{english}
In the Tuscan plain, the blessed 
 Abbot Nilus, founder of the monastery of Grottaferrata, a man of eminent 
 sanctity.
\switchcolumn*
\selectlanguage{latin}
Tiférni, in Umbria, 
 sancti Amántii Presbyteri, virtúte miraculórum illústris.
\switchcolumn
\selectlanguage{english}
At Tiferno in Umbria, St. Amantius, 
 a priest distinguished for the gift of miracles.
\switchcolumn*
\selectlanguage{latin}
Albáni sancti Senatóris.
\switchcolumn
\selectlanguage{english}
At Albano, St. Senatore.
\switchcolumn*
\selectlanguage{latin}
\end{paracol}


% ---- martyrology/mart09/mart0927.htm
\needspace{10\baselineskip}
\begin{paracol}{2}
\selectlanguage{latin}
\begin{center}{\color{gregoriocolor} Quinto Kaléndas Octóbris. 
 Luna\dots\ }\end{center}
\switchcolumn
\selectlanguage{english}
\begin{center}{\color{gregoriocolor} The   Twenty-Seventh Day of 
 September. The\dots\ Day of the Moon.}\end{center}
\end{paracol}

\noindent\begin{tabularx}{\linewidth}{*{19}{>{\centering\arraybackslash}X}}
 \textcolor{gregoriocolor}{a} & \textcolor{gregoriocolor}{b} & \textcolor{gregoriocolor}{c} & \textcolor{gregoriocolor}{d} & \textcolor{gregoriocolor}{e} & \textcolor{gregoriocolor}{f} & \textcolor{gregoriocolor}{g} & \textcolor{gregoriocolor}{h} & \textcolor{gregoriocolor}{i} & \textcolor{gregoriocolor}{k} & \textcolor{gregoriocolor}{l} & \textcolor{gregoriocolor}{m} & \textcolor{gregoriocolor}{n} & \textcolor{gregoriocolor}{p} & \textcolor{gregoriocolor}{q} & \textcolor{gregoriocolor}{r} & \textcolor{gregoriocolor}{s} & \textcolor{gregoriocolor}{t} & \textcolor{gregoriocolor}{u} \\
 5 & 6 & 7 & 8 & 9 & 10 & 11 & 12 & 13 & 14 & 15 & 16 & 17 & 18 & 19 & 20 & 21 & 22 & 23 \\
\end{tabularx}
\vspace{0.5\baselineskip}
\noindent\begin{tabularx}{\linewidth}{*{12}{>{\centering\arraybackslash}X}}
 \textcolor{gregoriocolor}{A} & \textcolor{gregoriocolor}{B} & \textcolor{gregoriocolor}{C} & \textcolor{gregoriocolor}{D} & \textcolor{gregoriocolor}{E} & F & \textcolor{gregoriocolor}{F} & \textcolor{gregoriocolor}{G} & \textcolor{gregoriocolor}{H} & \textcolor{gregoriocolor}{M} & \textcolor{gregoriocolor}{N} & \textcolor{gregoriocolor}{P} \\
 24 & 25 & 26 & 27 & 28 & 29 & 29 & 30 & 1 & 2 & 3 & 4 \\
\end{tabularx}

\begin{paracol}{2}
\selectlanguage{latin}
Ægéæ natális sanctórum 
 Mártyrum Cosmæ et Damiáni fratrum, qui, in persecutióne Diocletiáni, post 
 multa torménta, víncula et cárceres, post mare et ignes, cruces, 
 lapidatiónem et sagíttas divínitus superátas, cápite plectúntur; cum quibus 
 étiam referúntur passi tres eórum fratres germáni, id est Anthimus, Leóntius 
 et Euprépius.
\switchcolumn
\selectlanguage{english}
\lettrine[lines=2]{A}{t} Aegea, during the persecution of 
 Diocletian, the birthday of the holy martyrs Cosmas and Damian, brothers. 
 After miraculously overcoming many torments from bonds, imprisonment, fire, 
 crucifixion, stoning, arrows, and from being cast into the sea, they were 
 beheaded. With them are said to have suffered three brothers: Anthimus, 
 Leontius, and Euprepius.
\switchcolumn*
\selectlanguage{latin}
Lutétiæ Parisiórum item 
 natális sancti Vincéntii a Paulo, Presbyteri et Confessóris, Congregatiónis 
 Presbyterórum Missiónis et Puellárum Caritátis Fundatóris, viri apostólici 
 et páuperum patris; quem Leo Décimus tértius, Póntifex Máximus, ómnium 
 Societátum caritátis, in toto cathólico Orbe exsisténtium et ab eódem Sancto 
 quomodólibet promanántium, cæléstium Patrónum apud Deum constítuit. 
 Ipsíus tamen festívitas quartodécimo Kaléndas Augústi celebrátur.
\switchcolumn
\selectlanguage{english}
At Paris, the birthday of St. 
 Vincent de Paul, priest and confessor, founder of the Congregation of the 
 Mission and of the Sisters of Charity, an apostolic man and father to the 
 poor. Pope Leo XIII appointed this saint as the heavenly patron before 
 God of all charitable societies in the world which in any way whatever draw 
 their origin from him. His feast is celebrated on the 19th of July.
\switchcolumn*
\selectlanguage{latin}
Bybli, in Phœnícia, 
 sancti Marci Epíscopi, qui et Joánnes a beáto Luca nominátur, atque fílius éxstitit illíus beátæ Maríæ, cujus memória tértio Kaléndas Júlii recensétur.
\switchcolumn
\selectlanguage{english}
At Byblos in Phoenicia, Bishop St. 
 Mark, whom St. Luke calls John, and who was the son of that blessed Mary who 
 is commemorated on the 29th of July.
\switchcolumn*
\selectlanguage{latin}
Medioláni sancti Caji 
 Epíscopi, qui fuit discípulus beáti Bárnabæ Apóstoli; et, multa in Nerónis 
 persecutióne passus, quiévit in pace.
\switchcolumn
\selectlanguage{english}
At Milan, the holy bishop Caius, a 
 disciple of the blessed apostle Barnabas, who passed calmly to rest after 
 suffering severely in the persecution of Nero.
\switchcolumn*
\selectlanguage{latin}
Romæ sanctæ Epicháridis, 
 mulíeris Senatóriæ, quæ, in eádem Diocletiáni persecutióne, plumbátis cæsa 
 est atque gládio percússa.
\switchcolumn
\selectlanguage{english}
At Rome, St. Epicharis, wife of a 
 senator, who was scourged with leaded whips and then struck with the sword 
 in the persecution of Diocletian.
\switchcolumn*
\selectlanguage{latin}
Tudérti, in Umbria, 
 sanctórum Mártyrum Fidéntii et Teréntii, sub eódem Diocletiáno.
\switchcolumn
\selectlanguage{english}
At Todi in Umbria, under the same 
 Diocletian, the holy martyrs Fidentius and Terence.
\switchcolumn*
\selectlanguage{latin}
Córdubæ, in Hispánia, 
 sanctórum Mártyrum Adúlfi et Joánnis fratrum, qui, in persecutióne Arábica, 
 pro Christo coronáti sunt; eorúmque animáta exémplo beáta Virgo Aurea, 
 ipsórum soror, ad fidem redúcta, et ipsa póstmodum martyrium fórtiter súbiit 
 quartodécimo Kaléndas Augústi.
\switchcolumn
\selectlanguage{english}
At Cordova in Spain, the holy 
 martyrs Adolph and John, brothers, who won the martyrs' crown in the 
 Arabian persecution. Their sister, the blessed virgin Aurea, was 
 inspired by their example to return to the faith and later bravely suffered 
 martyrdom on the 19th of July.
\switchcolumn*
\selectlanguage{latin}
Sedúni, in Gállia, 
 sancti Florentíni Mártyris, qui, una cum beáto Hilário, post abscissiónem 
 linguæ, jussus est gládio feríri.
\switchcolumn
\selectlanguage{english}
At Sion in Switzerland, St. 
 Florentius, martyr. After his tongue had been cut out, he was put to 
 the sword with blessed Hilary.
\switchcolumn*
\selectlanguage{latin}
Ravénnæ sancti Aderíti, 
 Epíscopi et Confessóris.
\switchcolumn
\selectlanguage{english}
At Ravenna, St. Aderitus, bishop and 
 confessor.
\switchcolumn*
\selectlanguage{latin}
Lutétiæ Parisiórum 
 sancti Elzeárii Cómitis.
\switchcolumn
\selectlanguage{english}
At Paris, St. Eleazar, a count.
\switchcolumn*
\selectlanguage{latin}
In Hannónia sanctæ 
 Hiltrúdis Vírginis.
\switchcolumn
\selectlanguage{english}
In Hainault, St. Hiltrude, virgin.
\switchcolumn*
\selectlanguage{latin}
\end{paracol}


% ---- martyrology/mart09/mart0928.htm
\needspace{10\baselineskip}
\begin{paracol}{2}
\selectlanguage{latin}
\begin{center}{\color{gregoriocolor} Quarto Kaléndas Octóbris. 
 Luna\dots\ }\end{center}
\switchcolumn
\selectlanguage{english}
\begin{center}{\color{gregoriocolor} The   Twenty-Eighth Day of 
 September. The\dots\ Day of the Moon.}\end{center}
\end{paracol}

\noindent\begin{tabularx}{\linewidth}{*{19}{>{\centering\arraybackslash}X}}
 \textcolor{gregoriocolor}{a} & \textcolor{gregoriocolor}{b} & \textcolor{gregoriocolor}{c} & \textcolor{gregoriocolor}{d} & \textcolor{gregoriocolor}{e} & \textcolor{gregoriocolor}{f} & \textcolor{gregoriocolor}{g} & \textcolor{gregoriocolor}{h} & \textcolor{gregoriocolor}{i} & \textcolor{gregoriocolor}{k} & \textcolor{gregoriocolor}{l} & \textcolor{gregoriocolor}{m} & \textcolor{gregoriocolor}{n} & \textcolor{gregoriocolor}{p} & \textcolor{gregoriocolor}{q} & \textcolor{gregoriocolor}{r} & \textcolor{gregoriocolor}{s} & \textcolor{gregoriocolor}{t} & \textcolor{gregoriocolor}{u} \\
 6 & 7 & 8 & 9 & 10 & 11 & 12 & 13 & 14 & 15 & 16 & 17 & 18 & 19 & 20 & 21 & 22 & 23 & 24 \\
\end{tabularx}
\vspace{0.5\baselineskip}
\noindent\begin{tabularx}{\linewidth}{*{12}{>{\centering\arraybackslash}X}}
 \textcolor{gregoriocolor}{A} & \textcolor{gregoriocolor}{B} & \textcolor{gregoriocolor}{C} & \textcolor{gregoriocolor}{D} & \textcolor{gregoriocolor}{E} & F & \textcolor{gregoriocolor}{F} & \textcolor{gregoriocolor}{G} & \textcolor{gregoriocolor}{H} & \textcolor{gregoriocolor}{M} & \textcolor{gregoriocolor}{N} & \textcolor{gregoriocolor}{P} \\
 25 & 26 & 27 & 28 & 29 & 1 & 30 & 1 & 2 & 3 & 4 & 5 \\
\end{tabularx}

\begin{paracol}{2}
\selectlanguage{latin}
\lettrine[lines=2]{A}{pud} Bolesláviam 
 véterem, in Bohémia, sancti Wenceslái, Ducis Bohemórum et Mártyris, 
 sanctitáte et miráculis gloriósi, qui, dolo fratris sui necátus, victor 
 pervénit ad palmam.
\switchcolumn
\selectlanguage{english}
\lettrine[lines=2]{I}{n} Bohemia, St. Wenceslas, duke of 
 Bohemia and martyr, renowned for holiness and miracles. Being murdered 
 by the deceit of his brother, he went triumphantly to heaven.
\switchcolumn*
\selectlanguage{latin}
Romæ sancti Priváti 
 Mártyris, qui, ulcéribus plenus, a beáto Callísto Papa est sanátus; inde, 
 sub Alexándro Imperatóre, ob Christi fidem plumbátis cæsus est usque ad 
 mortem.
\switchcolumn
\selectlanguage{english}
At Rome, St. Privatus, martyr, who 
 was cured of ulcers by blessed Pope Callistus. In the time of Emperor 
 Alexander he was scourged to death with leaded whips for the faith of 
 Christ.
\switchcolumn*
\selectlanguage{latin}
Item Romæ sancti 
 Stáctei Mártyris.
\switchcolumn
\selectlanguage{english}
In the same place, St. Stacteus, 
 martyr.
\switchcolumn*
\selectlanguage{latin}
In Africa sanctórum 
 Mártyrum Martiális, Lauréntii et aliórum vigínti.
\switchcolumn
\selectlanguage{english}
In Africa, the Saints Martial, 
 Lawrence, and twenty other martyrs.
\switchcolumn*
\selectlanguage{latin}
Antiochíæ Pisídiæ 
 sancti Marci Mártyris, pastóris óvium; itémque commemorátio sanctórum Alphíi, 
 Alexándri et Zósimi fratrum, Nicónis, Neónis, Heliodóri et trigínta mílitum, 
 qui, cum ad mirácula beáti Marci credidíssent in Christum, divérsis in locis 
 ac diébus variísque modis martyrio coronáti sunt.
\switchcolumn
\selectlanguage{english}
At Antioch in Pisidia, the holy 
 martyrs Mark, a shepherd, Alphius, Alexander, and Zosimus, his brothers; 
 also Nicon, Neon, Heliodorus, and thirty soldiers, who were converted to 
 Christ upon seeing the miracles of blessed Mark, and were crowned with 
 martyrdom in different places and in diverse manners.
\switchcolumn*
\selectlanguage{latin}
Eódem die pássio sancti 
 Máximi, sub Décio Imperatóre.
\switchcolumn
\selectlanguage{english}
On the same day, under Emperor 
 Decius, the martyrdom of St. Maximus.
\switchcolumn*
\selectlanguage{latin}
Tolósæ sancti Exsupérii, 
 Epíscopi et Confessóris; qui beátus vir quantum sibi parcus exstíterit 
 quantúmque áliis largus, sanctus Hierónymus relátu prosecútus est memorábili.
\switchcolumn
\selectlanguage{english}
At Toulouse, St. Exuperius, bishop 
 and confessor. St. Jerome gives a memorable testimony of this blessed 
 man, relating how severe he was towards himself and how liberal towards 
 others.
\switchcolumn*
\selectlanguage{latin}
Génuæ sancti Salomónis, 
 Epíscopi et Confessóris.
\switchcolumn
\selectlanguage{english}
At Genoa, St. Solomon, bishop and 
 confessor.
\switchcolumn*
\selectlanguage{latin}
Bríxiæ sancti Silvíni 
 Epíscopi.
\switchcolumn
\selectlanguage{english}
At Brescia, St. Silvinus, bishop.
\switchcolumn*
\selectlanguage{latin}
In Béthlehem Judæ 
 sanctæ Eustóchii Vírginis, quæ cum beáta Paula, matre sua, ex urbe Roma in 
 Palæstínam profécta est; ibíque, ad Præsépe Dómini cum áliis Virgínibus 
 enutríta, præcláris méritis fulgens migrávit ad Dóminum.
\switchcolumn
\selectlanguage{english}
At Bethlehem of Juda, the holy 
 virgin Eustochium, daughter of blessed Paula, who was brought up at 
 the manger of our Lord with other virgins, and being celebrated for her 
 merits, went to our Lord.
\switchcolumn*
\selectlanguage{latin}
Schorneshémii, prope 
 Mogúntiam, sanctæ Líobæ Vírginis, miráculis claræ.
\switchcolumn
\selectlanguage{english}
At Fulda near Mayence, St. Lioba, 
 virgin, renowned for miracles.
\switchcolumn*
\selectlanguage{latin}
\end{paracol}


% ---- martyrology/mart09/mart0929.htm
\needspace{10\baselineskip}
\begin{paracol}{2}
\selectlanguage{latin}
\begin{center}{\color{gregoriocolor} Tértio Kaléndas Octóbris. 
 Luna\dots\ }\end{center}
\switchcolumn
\selectlanguage{english}
\begin{center}{\color{gregoriocolor} The   Twenty-Ninth Day of 
 September. The\dots\ Day of the Moon.}\end{center}
\end{paracol}

\noindent\begin{tabularx}{\linewidth}{*{19}{>{\centering\arraybackslash}X}}
 \textcolor{gregoriocolor}{a} & \textcolor{gregoriocolor}{b} & \textcolor{gregoriocolor}{c} & \textcolor{gregoriocolor}{d} & \textcolor{gregoriocolor}{e} & \textcolor{gregoriocolor}{f} & \textcolor{gregoriocolor}{g} & \textcolor{gregoriocolor}{h} & \textcolor{gregoriocolor}{i} & \textcolor{gregoriocolor}{k} & \textcolor{gregoriocolor}{l} & \textcolor{gregoriocolor}{m} & \textcolor{gregoriocolor}{n} & \textcolor{gregoriocolor}{p} & \textcolor{gregoriocolor}{q} & \textcolor{gregoriocolor}{r} & \textcolor{gregoriocolor}{s} & \textcolor{gregoriocolor}{t} & \textcolor{gregoriocolor}{u} \\
 7 & 8 & 9 & 10 & 11 & 12 & 13 & 14 & 15 & 16 & 17 & 18 & 19 & 20 & 21 & 22 & 23 & 24 & 25 \\
\end{tabularx}
\vspace{0.5\baselineskip}
\noindent\begin{tabularx}{\linewidth}{*{12}{>{\centering\arraybackslash}X}}
 \textcolor{gregoriocolor}{A} & \textcolor{gregoriocolor}{B} & \textcolor{gregoriocolor}{C} & \textcolor{gregoriocolor}{D} & \textcolor{gregoriocolor}{E} & F & \textcolor{gregoriocolor}{F} & \textcolor{gregoriocolor}{G} & \textcolor{gregoriocolor}{H} & \textcolor{gregoriocolor}{M} & \textcolor{gregoriocolor}{N} & \textcolor{gregoriocolor}{P} \\
 26 & 27 & 28 & 29 & 1 & 2 & 1 & 2 & 3 & 4 & 5 & 6 \\
\end{tabularx}

\begin{paracol}{2}
\selectlanguage{latin}
\lettrine[lines=2]{I}{n} monte Gargáno 
 venerábilis memória beáti Michaélis Archángeli, quando ipsíus nómine ibi 
 consecráta fuit Ecclésia, vili quidem facta schémate, sed cælésti 
 præstans virtúte.
\switchcolumn
\selectlanguage{english}
\lettrine[lines=2]{O}{n} Mount Gargano, the commemoration 
 of the blessed archangel Michael. This festival is kept in memory of 
 the day when, under his invocation, there was consecrated a church, 
 unpretending in its exterior, but endowed with celestial virtue.
\switchcolumn*
\selectlanguage{latin}
Antisiodóri sancti 
 Fratérni, Epíscopi et Mártyris.
\switchcolumn
\selectlanguage{english}
At Auxerre, St. Fraternus, bishop 
 and martyr.
\switchcolumn*
\selectlanguage{latin}
In Thrácia natális 
 sanctórum Mártyrum Eutychii, Plauti et Heracléæ.
\switchcolumn
\selectlanguage{english}
In Thrace, the birthday of the holy 
 martyrs Eutychius, Plautus, and Heracleas.
\switchcolumn*
\selectlanguage{latin}
In Pérside sanctórum 
 Mártyrum Dadæ, e Sáporis Regis consanguíneis, Cásdoæ uxóris, et Gabdélæ 
 fílii; qui, honóribus exúti ac váriis torméntis dilaniáti, tandem, post 
 longos cárceres, gládio sunt animadvérsi.
\switchcolumn
\selectlanguage{english}
In Persia, the holy martyrs Dadas, a 
 blood relative of King Sapor, Casdoa, his wife, and Gabdelas, his son. 
 After being deprived of their dignities, and subjected to various torments, 
 they were imprisoned for a long time and finally put to the sword.
\switchcolumn*
\selectlanguage{latin}
In Arménia sanctárum 
 Vírginum Rípsimis et Sociárum Mártyrum, sub Tiridáte Rege.
\switchcolumn
\selectlanguage{english}
In Armenia, under King Tiridates, 
 the holy virgin Ripsimis and her martyr companions.
\switchcolumn*
\selectlanguage{latin}
In Pérside sanctæ 
 Gudéliæ Mártyris, quæ, cum plúrimos convertísset ad Christum, ac Solem et 
 Ignem adoráre noluísset, ideo, sub Sápore Rege, post multa torménta, cute 
 cápitis detrácta, ligno affíxa, méruit obtinére triúmphum.
\switchcolumn
\selectlanguage{english}
In Persia, under King Sapor, the 
 holy martyr Gudelia. After converting many to the faith, and having 
 refused to adore the sun and the fire, she was subjected to numerous 
 torments. Having the skin torn off her head, and being fastened to a 
 post, she merited an eternal triumph.
\switchcolumn*
\selectlanguage{latin}
In Ponte Curvo, apud 
 Aquínum, sancti Grimoáldi, Presbyteri et Confessóris.
\switchcolumn
\selectlanguage{english}
At Pontecorvo near Aquino, St. 
 Grimoaldus, priest and confessor.
\switchcolumn*
\selectlanguage{latin}
In Palæstína sancti 
 Quiríaci Anachorétæ.
\switchcolumn
\selectlanguage{english}
In Palestine, St. Quiriacus, an 
 anchoret.
\switchcolumn*
\selectlanguage{latin}
\end{paracol}


% ---- martyrology/mart09/mart0930.htm
\needspace{10\baselineskip}
\begin{paracol}{2}
\selectlanguage{latin}
\begin{center}{\color{gregoriocolor} Prídie Kaléndas Octóbris. 
 Luna\dots\ }\end{center}
\switchcolumn
\selectlanguage{english}
\begin{center}{\color{gregoriocolor} The   Thirtieth Day of 
 September. The\dots\ Day of the Moon.}\end{center}
\end{paracol}

\noindent\begin{tabularx}{\linewidth}{*{19}{>{\centering\arraybackslash}X}}
 \textcolor{gregoriocolor}{a} & \textcolor{gregoriocolor}{b} & \textcolor{gregoriocolor}{c} & \textcolor{gregoriocolor}{d} & \textcolor{gregoriocolor}{e} & \textcolor{gregoriocolor}{f} & \textcolor{gregoriocolor}{g} & \textcolor{gregoriocolor}{h} & \textcolor{gregoriocolor}{i} & \textcolor{gregoriocolor}{k} & \textcolor{gregoriocolor}{l} & \textcolor{gregoriocolor}{m} & \textcolor{gregoriocolor}{n} & \textcolor{gregoriocolor}{p} & \textcolor{gregoriocolor}{q} & \textcolor{gregoriocolor}{r} & \textcolor{gregoriocolor}{s} & \textcolor{gregoriocolor}{t} & \textcolor{gregoriocolor}{u} \\
 8 & 9 & 10 & 11 & 12 & 13 & 14 & 15 & 16 & 17 & 18 & 19 & 20 & 21 & 22 & 23 & 24 & 25 & 26 \\
\end{tabularx}
\vspace{0.5\baselineskip}
\noindent\begin{tabularx}{\linewidth}{*{12}{>{\centering\arraybackslash}X}}
 \textcolor{gregoriocolor}{A} & \textcolor{gregoriocolor}{B} & \textcolor{gregoriocolor}{C} & \textcolor{gregoriocolor}{D} & \textcolor{gregoriocolor}{E} & F & \textcolor{gregoriocolor}{F} & \textcolor{gregoriocolor}{G} & \textcolor{gregoriocolor}{H} & \textcolor{gregoriocolor}{M} & \textcolor{gregoriocolor}{N} & \textcolor{gregoriocolor}{P} \\
 27 & 28 & 29 & 1 & 2 & 3 & 2 & 3 & 4 & 5 & 6 & 7 \\
\end{tabularx}

\begin{paracol}{2}
\selectlanguage{latin}
\lettrine[lines=2]{I}{n} Béthlehem Judæ 
 deposítio sancti Hierónymi Presbyteri, Confessóris et Ecclésiæ Doctóris, 
 qui, ómnium stúdia litterárum adéptus ac probatórum Monachórum imitátor 
 factus, multa hæresum monstra gládio suæ doctrínæ confódit; demum, cum ad 
 decrépitam usque vixísset ætátem, in pace quiévit, sepultúsque est ad 
 Præsépe Dómini. Ejus corpus, póstea Romam delátum, in Basílica sanctæ 
 Maríæ Majóris cónditum fuit.
\switchcolumn
\selectlanguage{english}
\lettrine[lines=2]{I}{n} Bethlehem of Juda, the death of 
 St. Jerome, priest and doctor of the Church. Excelling in all kinds of 
 learning, he imitated the life of the most approved monks, and disposed of 
 many monstrous heresies with the sword of his doctrine. Having at 
 length reached a very advanced age, he rested in peace and was buried near 
 the manger of our Lord. His body was afterwards transferred to Rome, 
 and placed in the basilica of St. Mary Major.
\switchcolumn*
\selectlanguage{latin}
Romæ natális sancti 
 Francísci Bórgiæ, Sacerdótis et Confessóris; qui Præpósitus Generális fuit 
 Societátis Jesu, ac vitæ asperitáte, oratiónis dono, abdicátis sæculi et 
 recusátis Ecclésiæ dignitátibus, vir memorábilis éxstitit. Ipsíus 
 autem festum sexto Idus Octóbris celebrátur.
\switchcolumn
\selectlanguage{english}
At Rome, the birthday of St. Francis 
 Borgia, priest and confessor. He was the General of the Society of 
 Jesus, and is memorable for his mortification, gift of prayer, the forsaking 
 of the world, and the refusal of ecclesiastical dignities. His feast 
 is observed on the 10th of October.
\switchcolumn*
\selectlanguage{latin}
Lexóvii, in Gállia, 
 item natális sanctæ Terésiæ a Jesu Infánte, ex Ordine Carmelitárum 
 Excalceatórum; quam, vitæ innocéntia et simplicitáte claríssimam, Pius 
 Undécimus, Póntifex Máximus, sanctárum Vírginum albo adscrípsit, peculiárem 
 ómnium Missiónum Patrónam declarávit, ejúsque festum quinto Nonas Octóbris 
 recoléndum esse decrévit.
\switchcolumn
\selectlanguage{english}
At Lisieux in France, the birthday 
 of St. Theresa of the Child Jesus, of the Order of Discalced Carmelites. 
 Seeing her to be most wonderful for her innocence of life and simplicity, 
 Pope Pius XI placed her name among the holy virgins and appointed her as 
 special patron before God of all missions, decreeing that her feast should 
 be observed on the 3rd of October.
\switchcolumn*
\selectlanguage{latin}
Romæ sancti Leopárdi 
 Mártyris, qui ex Juliáni Apóstatæ domésticis fuit; cui caput amputátum est, 
 et corpus ejus Aquisgránum póstea translátum.
\switchcolumn
\selectlanguage{english}
At Rome, the holy martyr Leopardus, 
 of the household of Julian the Apostate. He was beheaded at Rome, and 
 his body afterwards taken to Aix-la-Chapelle.
\switchcolumn*
\selectlanguage{latin}
Solodóri, in Gállia, 
 pássio sanctórum Mártyrum Victóris et Ursi, ex gloriósa legióne Thebæórum, 
 qui, sub Maximiáno Imperatóre, primum diris supplíciis cruciáti, sed cælésti 
 super eos lúmine coruscánte, ruéntibus in terram minístris, erépti; deínde 
 in ignem missi, sed in nullo pénitus læsi; novíssime gládio consummáti sunt.
\switchcolumn
\selectlanguage{english}
At Soleure in Switzerland, in the 
 time of Emperor Maximian, the passion of the holy martyrs Victor and Ursus, 
 of the glorious Theban legion. They were subjected to horrible 
 tortures, but a heavenly light shone over them causing the executioners to 
 fall to the ground, and they were delivered. Being then cast into the 
 fire without sustaining any injury, they finally perished by the sword.
\switchcolumn*
\selectlanguage{latin}
Placéntiæ sancti 
 Antoníni Mártyris, ex eádem legióne.
\switchcolumn
\selectlanguage{english}
At Piacenza, the holy martyr 
 Antoninus, a soldier of the same legion.
\switchcolumn*
\selectlanguage{latin}
Eódem die sancti 
 Gregórii, Epíscopi magnæ Arméniæ, qui, sub Diocletiáno, multa passus est; ac 
 tandem, Constantíni Magni Imperatóris témpore, in pace quiévit.
\switchcolumn
\selectlanguage{english}
On the same day, St. Gregory, bishop 
 of Greater Armenia, who, after many sufferings under Diocletian, rested in 
 peace.
\switchcolumn*
\selectlanguage{latin}
Cantuáriæ, in Anglia, 
 sancti Honórii, Epíscopi et Confessóris.
\switchcolumn
\selectlanguage{english}
At Canterbury in England, St. 
 Honorius, bishop and confessor.
\switchcolumn*
\selectlanguage{latin}
Romæ sanctæ Sophíæ 
 Víduæ, matris sanctárum Vírginum et Mártyrum Fídei, Spei et Caritátis.
\switchcolumn
\selectlanguage{english}
At Rome, St. Sophia, widow, mother 
 of the holy virgin martyrs Faith, Hope, and Charity.
\switchcolumn*
\selectlanguage{latin}
\end{paracol}

\setrunningtitles{October}{October}

% ---- martyrology/mart10/mart1001.htm
\needspace{10\baselineskip}
\begin{paracol}{2}
\selectlanguage{latin}
\begin{center}{\color{gregoriocolor} Kaléndis Octóbris. 
 Luna\dots\ }\end{center}
\switchcolumn
\selectlanguage{english}
\begin{center}{\color{gregoriocolor} The   First Day of 
 October. The\dots\ Day of the Moon.}\end{center}
\end{paracol}

\noindent\begin{tabularx}{\linewidth}{*{19}{>{\centering\arraybackslash}X}}
 \textcolor{gregoriocolor}{a} & \textcolor{gregoriocolor}{b} & \textcolor{gregoriocolor}{c} & \textcolor{gregoriocolor}{d} & \textcolor{gregoriocolor}{e} & \textcolor{gregoriocolor}{f} & \textcolor{gregoriocolor}{g} & \textcolor{gregoriocolor}{h} & \textcolor{gregoriocolor}{i} & \textcolor{gregoriocolor}{k} & \textcolor{gregoriocolor}{l} & \textcolor{gregoriocolor}{m} & \textcolor{gregoriocolor}{n} & \textcolor{gregoriocolor}{p} & \textcolor{gregoriocolor}{q} & \textcolor{gregoriocolor}{r} & \textcolor{gregoriocolor}{s} & \textcolor{gregoriocolor}{t} & \textcolor{gregoriocolor}{u} \\
 9 & 10 & 11 & 12 & 13 & 14 & 15 & 16 & 17 & 18 & 19 & 20 & 21 & 22 & 23 & 24 & 25 & 26 & 27 \\
\end{tabularx}
\vspace{0.5\baselineskip}
\noindent\begin{tabularx}{\linewidth}{*{12}{>{\centering\arraybackslash}X}}
 \textcolor{gregoriocolor}{A} & \textcolor{gregoriocolor}{B} & \textcolor{gregoriocolor}{C} & \textcolor{gregoriocolor}{D} & \textcolor{gregoriocolor}{E} & F & \textcolor{gregoriocolor}{F} & \textcolor{gregoriocolor}{G} & \textcolor{gregoriocolor}{H} & \textcolor{gregoriocolor}{M} & \textcolor{gregoriocolor}{N} & \textcolor{gregoriocolor}{P} \\
 28 & 29 & 1 & 2 & 3 & 4 & 3 & 4 & 5 & 6 & 7 & 8 \\
\end{tabularx}

\begin{paracol}{2}
\selectlanguage{latin}
\lettrine[lines=2]{S}{ancti} Remígii, 
 Epíscopi Rheménsis et Confessóris, qui Idibus Januárii obdormívit in Dómino, 
 sed hac die, ob Translatiónem córporis ejus, potíssimum cólitur.
\switchcolumn
\selectlanguage{english}
\lettrine[lines=2]{S}{t.} Remigius, bishop of Rheims and 
 confessor, who fell asleep in the Lord on the 13th of January, but is 
 commemorated on this day because of the translation of his body.
\switchcolumn*
\selectlanguage{latin}
Romæ beáti Arétæ 
 Mártyris, et aliórum quingentórum quátuor.
\switchcolumn
\selectlanguage{english}
At Rome, blessed Aretas and five 
 hundred and four other martyrs.
\switchcolumn*
\selectlanguage{latin}
Tornáci, in Gálliis, 
 sancti Piatónis, Presbyteri et Mártyris; qui, prædicatiónis causa, cum beáto 
 Quinctíno ejúsque Sóciis, ab urbe Roma in Gálliam perréxit, ac póstea, in 
 persecutióne Maximiáni, consummáto martyrio, migrávit ad Dóminum.
\switchcolumn
\selectlanguage{english}
At Tournai in France, St. Piaton, 
 priest and martyr, who went from Rome to France to preach, together with 
 blessed Quinctinus and his companions. Afterwards, his martyrdom was 
 completed in the persecution of Maximian and he passed from earth to heaven.
\switchcolumn*
\selectlanguage{latin}
Tomis, in Ponto, 
 sanctórum Mártyrum Prisci, Crescéntis et Evágrii.
\switchcolumn
\selectlanguage{english}
At Tomis in Pontus, the holy martyrs 
 Priscus, Crescens, and Evagrius.
\switchcolumn*
\selectlanguage{latin}
Ulyssipóne, in 
 Lusitánia, sanctórum Mártyrum Veríssimi, Máximæ et Júliæ, sorórum ejus; qui 
 in Diocletiáni Imperatóris persecutióne passi sunt.
\switchcolumn
\selectlanguage{english}
At Lisbon in Portugal, the holy 
 martyrs Verissimus, and his sisters Maxima and Julia, who suffered in the 
 persecution of Diocletian.
\switchcolumn*
\selectlanguage{latin}
Thessalonícæ sancti 
 Domníni Mártyris, sub Maximiáno Imperatóre.
\switchcolumn
\selectlanguage{english}
At Thessalonica, St. Domninus, 
 martyr, under Emperor Maximian.
\switchcolumn*
\selectlanguage{latin}
Urbe Véteri sancti 
 Sevéri, Presbyteri et Confessóris.
\switchcolumn
\selectlanguage{english}
At Orvieto, St. Severus, priest and 
 confessor.
\switchcolumn*
\selectlanguage{latin}
In Portu Gandæ sancti 
 Bavónis Confessóris.
\switchcolumn
\selectlanguage{english}
At the port of Ghent, St. Bavo, 
 confessor.
\switchcolumn*
\selectlanguage{latin}
\end{paracol}


% ---- martyrology/mart10/mart1002.htm
\needspace{10\baselineskip}
\begin{paracol}{2}
\selectlanguage{latin}
\begin{center}{\color{gregoriocolor} Sexto Nonas Octóbris. 
 Luna\dots\ }\end{center}
\switchcolumn
\selectlanguage{english}
\begin{center}{\color{gregoriocolor} The   Second Day of 
 October. The\dots\ Day of the Moon.}\end{center}
\end{paracol}

\noindent\begin{tabularx}{\linewidth}{*{19}{>{\centering\arraybackslash}X}}
 \textcolor{gregoriocolor}{a} & \textcolor{gregoriocolor}{b} & \textcolor{gregoriocolor}{c} & \textcolor{gregoriocolor}{d} & \textcolor{gregoriocolor}{e} & \textcolor{gregoriocolor}{f} & \textcolor{gregoriocolor}{g} & \textcolor{gregoriocolor}{h} & \textcolor{gregoriocolor}{i} & \textcolor{gregoriocolor}{k} & \textcolor{gregoriocolor}{l} & \textcolor{gregoriocolor}{m} & \textcolor{gregoriocolor}{n} & \textcolor{gregoriocolor}{p} & \textcolor{gregoriocolor}{q} & \textcolor{gregoriocolor}{r} & \textcolor{gregoriocolor}{s} & \textcolor{gregoriocolor}{t} & \textcolor{gregoriocolor}{u} \\
 10 & 11 & 12 & 13 & 14 & 15 & 16 & 17 & 18 & 19 & 20 & 21 & 22 & 23 & 24 & 25 & 26 & 27 & 28 \\
\end{tabularx}
\vspace{0.5\baselineskip}
\noindent\begin{tabularx}{\linewidth}{*{12}{>{\centering\arraybackslash}X}}
 \textcolor{gregoriocolor}{A} & \textcolor{gregoriocolor}{B} & \textcolor{gregoriocolor}{C} & \textcolor{gregoriocolor}{D} & \textcolor{gregoriocolor}{E} & F & \textcolor{gregoriocolor}{F} & \textcolor{gregoriocolor}{G} & \textcolor{gregoriocolor}{H} & \textcolor{gregoriocolor}{M} & \textcolor{gregoriocolor}{N} & \textcolor{gregoriocolor}{P} \\
 29 & 1 & 2 & 3 & 4 & 5 & 4 & 5 & 6 & 7 & 8 & 9 \\
\end{tabularx}

\begin{paracol}{2}
\selectlanguage{latin}
\lettrine[lines=1]{F}{estum} sanctórum Angelórum Custódum.
\switchcolumn
\selectlanguage{english}
\lettrine[lines=1]{T}{he} Feast of the holy Guardian Angels.
\switchcolumn*
\selectlanguage{latin}
Romæ pássio sancti 
 Modésti Sardi, Levítæ et Mártyris; qui, sub Diocletiáno Imperatóre, equúleo 
 tortus atque igne adústus est. Ipsíus vero corpus, Benevéntum póstea 
 translátum, in Ecclésia suo insigníta nómine collocátum fuit.
\switchcolumn
\selectlanguage{english}
At Rome, the martyrdom of St. 
 Modestus, a Sardinian, deacon and martyr, who was racked and burned with 
 fire by Emperor Diocletian. His holy body was afterwards translated to 
 Benevento and buried there in a church named after him.
\switchcolumn*
\selectlanguage{latin}
In território 
 Atrebaténsi item pássio beáti Leodegárii, Augustodunénsis Epíscopi; quem, 
 váriis injúriis et divérsis supplíciis pro veritáte afflíctum, Ebroínus, 
 Major domus régiæ Theodoríci, intérfici jussit.
\switchcolumn
\selectlanguage{english}
In the vicinity of Arras, the 
 martyrdom of blessed Leodegar, bishop of Autun. After being 
 subjected to various insults and torments for the truth, he was put to death 
 by Ebroin, chief minister of Theodoric.
\switchcolumn*
\selectlanguage{latin}
Nicomedíæ sancti 
 Eleuthérii, mílitis et Mártyris, cum áliis innúmeris; qui, cum Diocletiáni 
 régia incéndio conflagrásset et falso hujus críminis essent accusáti, omnes, 
 jubénte eódem sævíssimo Imperatóre, acervátim necáti sunt. Horum porro 
 álii gládiis obtruncabántur, álii cremabántur ígnibus, álii in mare 
 præcipitabántur; sed inter eos primus Eleuthérius, cum per síngula torménta, 
 diu cruciátus, valídior redderétur, martyrium victóriæ suæ, ígnibus 
 velut aurum examinátus, complévit.
\switchcolumn
\selectlanguage{english}
At Nicomedia, St. Eleutherius, 
 soldier and martyr, with innumerable others. They were falsely accused 
 of having set fire to the palace of Diocletian and, by order of this cruel 
 emperor, were barbarously massacred in groups. Some were put to the 
 sword, some consumed by fire, while others were cast into the sea. But 
 the principal one, Eleutherius, after long tortures, being found stronger 
 after each torment, completed his victorious martyrdom by fire, as 
 well-tried gold.
\switchcolumn*
\selectlanguage{latin}
Antiochíæ sanctórum 
 Mártyrum Primi, Cyrílli et Secundárii.
\switchcolumn
\selectlanguage{english}
At Antioch, the holy martyrs Primus, 
 Cyril, and Secundarius.
\switchcolumn*
\selectlanguage{latin}
Eódem die sancti Geríni 
 Mártyris, qui frater éxstitit beáti Leodegárii, Augustodunénsis Epíscopi, 
 et, jubénte ipso Ebroíno, lapídibus óbrutus est.
\switchcolumn
\selectlanguage{english}
On the same day, St. Gerinus, 
 martyr, brother of blessed Leodegar, bishop of Autun. He was stoned 
 to death by the same Ebroin.
\switchcolumn*
\selectlanguage{latin}
Constantinópoli sancti 
 Theóphili Mónachi, qui pro defensióne sanctárum Imáginum a Leóne Isáurico 
 sævíssime cæsus et in exsílium pulsus, migrávit ad Dóminum.
\switchcolumn
\selectlanguage{english}
At Constantinople, St. Theophilus, a 
 monk. He was cruelly scourged by Leo the Isaurian for his defense of 
 holy images, was driven into exile, and there went gloriously to heaven.
\switchcolumn*
\selectlanguage{latin}
\end{paracol}


% ---- martyrology/mart10/mart1003.htm
\needspace{10\baselineskip}
\begin{paracol}{2}
\selectlanguage{latin}
\begin{center}{\color{gregoriocolor} Quinto Nonas Octóbris. 
 Luna\dots\ }\end{center}
\switchcolumn
\selectlanguage{english}
\begin{center}{\color{gregoriocolor} The   Third Day of 
 October. The\dots\ Day of the Moon.}\end{center}
\end{paracol}

\noindent\begin{tabularx}{\linewidth}{*{19}{>{\centering\arraybackslash}X}}
 \textcolor{gregoriocolor}{a} & \textcolor{gregoriocolor}{b} & \textcolor{gregoriocolor}{c} & \textcolor{gregoriocolor}{d} & \textcolor{gregoriocolor}{e} & \textcolor{gregoriocolor}{f} & \textcolor{gregoriocolor}{g} & \textcolor{gregoriocolor}{h} & \textcolor{gregoriocolor}{i} & \textcolor{gregoriocolor}{k} & \textcolor{gregoriocolor}{l} & \textcolor{gregoriocolor}{m} & \textcolor{gregoriocolor}{n} & \textcolor{gregoriocolor}{p} & \textcolor{gregoriocolor}{q} & \textcolor{gregoriocolor}{r} & \textcolor{gregoriocolor}{s} & \textcolor{gregoriocolor}{t} & \textcolor{gregoriocolor}{u} \\
 11 & 12 & 13 & 14 & 15 & 16 & 17 & 18 & 19 & 20 & 21 & 22 & 23 & 24 & 25 & 26 & 27 & 28 & 29 \\
\end{tabularx}
\vspace{0.5\baselineskip}
\noindent\begin{tabularx}{\linewidth}{*{12}{>{\centering\arraybackslash}X}}
 \textcolor{gregoriocolor}{A} & \textcolor{gregoriocolor}{B} & \textcolor{gregoriocolor}{C} & \textcolor{gregoriocolor}{D} & \textcolor{gregoriocolor}{E} & F & \textcolor{gregoriocolor}{F} & \textcolor{gregoriocolor}{G} & \textcolor{gregoriocolor}{H} & \textcolor{gregoriocolor}{M} & \textcolor{gregoriocolor}{N} & \textcolor{gregoriocolor}{P} \\
 1 & 2 & 3 & 4 & 5 & 6 & 5 & 6 & 7 & 8 & 9 & 10 \\
\end{tabularx}

\begin{paracol}{2}
\selectlanguage{latin}
\lettrine[lines=2]{S}{anctæ} Terésiæ a Jesu 
 Infánte, ex Ordine Carmelitárum Excalceatórum, Vírginis, peculiáris ómnium 
 Missiónum Patrónæ; cujus dies natális prídie Kaléndas Octóbris recensétur.
\switchcolumn
\selectlanguage{english}
\lettrine[lines=2]{S}{t.} Theresa of the Child Jesus, 
 virgin of the Order of Discalced Carmelites, special patroness of all 
 missions. Her birthday is commemorated on the 30th day of September.
\switchcolumn*
\selectlanguage{latin}
Romæ, ad Ursum pileátum, 
 sancti Cándidi Mártyris.
\switchcolumn
\selectlanguage{english}
At Rome, near the place called Ursus 
 Pileatus, St. Candidus, martyr.
\switchcolumn*
\selectlanguage{latin}
Apud antíquos Sáxones 
 sanctórum Mártyrum duórum Ewaldórum, qui, cum essent Presbyteri et Christum 
 ibi prædicáre cœpíssent, comprehénsi sunt a Pagánis et occísi; ad quorum 
 córpora noctu lux multa, diu appárens, et ubi essent et cujus essent mériti, 
 declarávit.
\switchcolumn
\selectlanguage{english}
Among the ancient Saxons, two holy 
 martyrs of the name of Ewald, priests who had been preaching in that 
 country. They were seized by the pagans and put to death. During 
 the night, a great light shone over the bodies for a long time, pointing out 
 where they were and also how distinguished were their merits.
\switchcolumn*
\selectlanguage{latin}
Eódem die sanctórum 
 Mártyrum Dionysii, Fausti, Caji, Petri, Pauli et aliórum quátuor, qui primum, 
 sub Décio, multa passi sunt, ac demum, sub Valeriáno, ab Æmiliáno Præside 
 diu torméntis vexáti, martyrii palmam meruérunt.
\switchcolumn
\selectlanguage{english}
Also, the holy martyrs Denis, 
 Faustus, Caius, Peter, Paul, and four others who had suffered greatly under 
 Decius. In the time of Valerian, they were long subjected to torments 
 by the governor Aemilian, and merited the palm of martyrdom.
\switchcolumn*
\selectlanguage{latin}
In Africa sancti 
 Maximiáni, Epíscopi Bagajénsis, qui, a Donatístis íterum atque íterum 
 sævíssima perpéssus, ex alta dénique turri præcipitátus est, et pro mórtuo 
 derelíctus; sed, póstmodum a transeúntibus colléctus et pia curatióne 
 sanátus, cathólicam fidem propugnáre non déstitit, donec, glória 
 confessiónis clarus, quiévit in Dómino.
\switchcolumn
\selectlanguage{english}
In Africa, St. Maximian, bishop of 
 Bagaia. Again and again he suffered great cruelties from the Donatists, 
 was finally cast headlong from a high tower, and left for dead. He was 
 found by passers-by, and having been healed by their zealous care, he did 
 not cease to defend the Catholic faith until he rested in the Lord, renowned 
 for the glory of his witness to the faith.
\switchcolumn*
\selectlanguage{latin}
Legióne, in Hispánia, 
 sancti Froiláni, ejúsdem civitátis Epíscopi, monásticæ vitæ propagándæ 
 stúdio, beneficéntia in páuperes, ceterísque virtútibus et miráculis clari.
\switchcolumn
\selectlanguage{english}
At Leon in Spain, St. Froylan, 
 bishop of that city, noted for his zeal in spreading the monastic life, his 
 generosity to the poor and other virtues, and for his miracles.
\switchcolumn*
\selectlanguage{latin}
In diœcési Namurcénsi, 
 apud Belgas, sancti Gerárdi Abbátis.
\switchcolumn
\selectlanguage{english}
In Belgium, in the diocese of Namur, 
 St. Gerard, abbot.
\switchcolumn*
\selectlanguage{latin}
In Palæstína sancti 
 Hesychii Confessóris, qui fuit sancti Hilariónis discípulus et in 
 peregrinatióne sócius.
\switchcolumn
\selectlanguage{english}
In Palestine, St. Hesychius, 
 confessor, disciple of St. Hilarion, and the companion of his travels.
\switchcolumn*
\selectlanguage{latin}
Savónæ, in Ligúria, 
 sanctæ Maríæ Joséphæ Rosséllo, Institúti Filiárum Nostræ Dóminæ a 
 Misericórdia Fundatrícis, quam, caritátis opéribus præcláram, Pius Papa 
 Duodécimus sanctis Virgínibus adnumerávit.
\switchcolumn
\selectlanguage{english}
At Savona in Liguria, St. Maria 
 Giuseppe Rossello, foundress of the Daughters of our Lady of Mercy. 
 Renowned for her charitable works, Pope Pius XII placed her among the number 
 of holy virgins.
\switchcolumn*
\selectlanguage{latin}
\end{paracol}


% ---- martyrology/mart10/mart1004.htm
\needspace{10\baselineskip}
\begin{paracol}{2}
\selectlanguage{latin}
\begin{center}{\color{gregoriocolor} Quarto Nonas Octóbris. 
 Luna\dots\ }\end{center}
\switchcolumn
\selectlanguage{english}
\begin{center}{\color{gregoriocolor} The   Fourth Day of 
 October. The\dots\ Day of the Moon.}\end{center}
\end{paracol}

\noindent\begin{tabularx}{\linewidth}{*{19}{>{\centering\arraybackslash}X}}
 \textcolor{gregoriocolor}{a} & \textcolor{gregoriocolor}{b} & \textcolor{gregoriocolor}{c} & \textcolor{gregoriocolor}{d} & \textcolor{gregoriocolor}{e} & \textcolor{gregoriocolor}{f} & \textcolor{gregoriocolor}{g} & \textcolor{gregoriocolor}{h} & \textcolor{gregoriocolor}{i} & \textcolor{gregoriocolor}{k} & \textcolor{gregoriocolor}{l} & \textcolor{gregoriocolor}{m} & \textcolor{gregoriocolor}{n} & \textcolor{gregoriocolor}{p} & \textcolor{gregoriocolor}{q} & \textcolor{gregoriocolor}{r} & \textcolor{gregoriocolor}{s} & \textcolor{gregoriocolor}{t} & \textcolor{gregoriocolor}{u} \\
 12 & 13 & 14 & 15 & 16 & 17 & 18 & 19 & 20 & 21 & 22 & 23 & 24 & 25 & 26 & 27 & 28 & 29 & 1 \\
\end{tabularx}
\vspace{0.5\baselineskip}
\noindent\begin{tabularx}{\linewidth}{*{12}{>{\centering\arraybackslash}X}}
 \textcolor{gregoriocolor}{A} & \textcolor{gregoriocolor}{B} & \textcolor{gregoriocolor}{C} & \textcolor{gregoriocolor}{D} & \textcolor{gregoriocolor}{E} & F & \textcolor{gregoriocolor}{F} & \textcolor{gregoriocolor}{G} & \textcolor{gregoriocolor}{H} & \textcolor{gregoriocolor}{M} & \textcolor{gregoriocolor}{N} & \textcolor{gregoriocolor}{P} \\
 2 & 3 & 4 & 5 & 6 & 7 & 6 & 7 & 8 & 9 & 10 & 11 \\
\end{tabularx}

\begin{paracol}{2}
\selectlanguage{latin}
\lettrine[lines=2]{A}{ssísii,} in Umbria, 
 natális sancti Francísci, Levítæ et Confessóris; qui trium Ordinum, 
 scílicet Fratrum Minórum, Páuperum Dominárum, ac Fratrum et Sorórum de 
 Pæniténtia Fundátor éxstitit. Ipsíus autem vitam, sanctitáte ac 
 miráculis plenam, sanctus Bonaventúra conscrípsit.
\switchcolumn
\selectlanguage{english}
\lettrine[lines=2]{A}{t} Assisi in Umbria, the birthday of 
 St. Francis, cleric and confessor, founder of three orders: the Friars 
 Minor, the Poor Clares, and the Brothers and Sisters of Penance. His 
 life, filled with holy deeds and miracles, were written by St. Bonaventure.
\switchcolumn*
\selectlanguage{latin}
Corínthi item natális 
 sanctórum Crispi et Caji, quorum méminit sanctus Paulus Apóstolus ad 
 Corínthios scribens.
\switchcolumn
\selectlanguage{english}
At Corinth, the birthday of the 
 Saints Crispus and Caius, who are mentioned by the apostle St. Paul in his 
 Epistle to the Corinthians.
\switchcolumn*
\selectlanguage{latin}
Athénis sancti 
 Hieróthei, qui fuit discípulus ipsíus beáti Pauli Apóstoli.
\switchcolumn
\selectlanguage{english}
At Athens, St. Hierotheos, disciple 
 of the blessed apostle Paul.
\switchcolumn*
\selectlanguage{latin}
Damásci sancti Petri, 
 Epíscopi et Mártyris; qui, accusátus apud Agarenórum Príncipem quod fidem 
 Christi docéret, ídeo, lingua et mánibus pedibúsque amputátis, cruci affíxus 
 martyrium consummávit.
\switchcolumn
\selectlanguage{english}
At Damascus, St. Peter, bishop and 
 martyr, who was accused before the king of the Agarenians of teaching the 
 faith of Christ. His tongue, hands, and feet were cut off, and being 
 fastened to a cross, his martyrdom was fulfilled.
\switchcolumn*
\selectlanguage{latin}
Alexandríæ sanctórum 
 Presbyterórum, et Diaconórum Caji, Fausti, Eusébii, Chærémonis, Lúcii et 
 Sociórum; ex quibus álii, in persecutióne Valeriáni, Mártyres facti sunt, 
 álii, Martyribus serviéntes, mercédem Mártyrum recepérunt.
\switchcolumn
\selectlanguage{english}
At Alexandria, the holy priests and 
 deacons Caius, Faustus, Eusebius, Chaeremon, Lucius, and their companions. 
 Some of them were martyred in the persecution of Valerian; others, for 
 serving the martyrs, received the reward of martyrs.
\switchcolumn*
\selectlanguage{latin}
In Ægypto sanctórum 
 Mártyrum Marci et Marciáni fratrum, et aliórum ferme innumerabílium 
 utriúsque sexus atque omnis ætátis; quorum álii post vérbera, álii post 
 divérsa géneris horríbiles cruciátus flammis tráditi, álii in mare 
 præcipitáti, nonnúlli cápite cæsi, plúrimi inédia consúmpti, álii partíbulis 
 affíxi, quidam étiam, cápite deórsum verso et pédibus in sublíme sublátis, 
 appénsi, beatíssimam martyrii corónam meruérunt.
\switchcolumn
\selectlanguage{english}
In Egypt, the holy martyrs Mark and 
 Marcian, brothers, and an almost countless number of both sexes and of all 
 ages, who merited the blessed crown of martyrdom, some after being scourged, 
 others when they had suffered horrible torment, and others after being 
 delivered to the flames. Some were cast into the sea, some others were 
 beheaded; many were starved to death; many were fastened to gibbets; and 
 others again were suspended by the feet with their heads downward.
\switchcolumn*
\selectlanguage{latin}
Bonóniæ sancti Petrónii, 
 Epíscopi et Confessóris; qui doctrína, miráculis et sanctitáte cláruit.
\switchcolumn
\selectlanguage{english}
At Bologna, St. Petronius, bishop 
 and confessor, celebrated for learning, miracles, and sanctity.
\switchcolumn*
\selectlanguage{latin}
Lutétiæ Parisiórum 
 sanctæ Aureæ Vírginis.
\switchcolumn
\selectlanguage{english}
At Paris, St. Aurea, virgin.
\switchcolumn*
\selectlanguage{latin}
\end{paracol}


% ---- martyrology/mart10/mart1005.htm
\needspace{10\baselineskip}
\begin{paracol}{2}
\selectlanguage{latin}
\begin{center}{\color{gregoriocolor} Tértio Nonas Octóbris. 
 Luna\dots\ }\end{center}
\switchcolumn
\selectlanguage{english}
\begin{center}{\color{gregoriocolor} The   Fifth Day of 
 October. The\dots\ Day of the Moon.}\end{center}
\end{paracol}

\noindent\begin{tabularx}{\linewidth}{*{19}{>{\centering\arraybackslash}X}}
 \textcolor{gregoriocolor}{a} & \textcolor{gregoriocolor}{b} & \textcolor{gregoriocolor}{c} & \textcolor{gregoriocolor}{d} & \textcolor{gregoriocolor}{e} & \textcolor{gregoriocolor}{f} & \textcolor{gregoriocolor}{g} & \textcolor{gregoriocolor}{h} & \textcolor{gregoriocolor}{i} & \textcolor{gregoriocolor}{k} & \textcolor{gregoriocolor}{l} & \textcolor{gregoriocolor}{m} & \textcolor{gregoriocolor}{n} & \textcolor{gregoriocolor}{p} & \textcolor{gregoriocolor}{q} & \textcolor{gregoriocolor}{r} & \textcolor{gregoriocolor}{s} & \textcolor{gregoriocolor}{t} & \textcolor{gregoriocolor}{u} \\
 13 & 14 & 15 & 16 & 17 & 18 & 19 & 20 & 21 & 22 & 23 & 24 & 25 & 26 & 27 & 28 & 29 & 1 & 2 \\
\end{tabularx}
\vspace{0.5\baselineskip}
\noindent\begin{tabularx}{\linewidth}{*{12}{>{\centering\arraybackslash}X}}
 \textcolor{gregoriocolor}{A} & \textcolor{gregoriocolor}{B} & \textcolor{gregoriocolor}{C} & \textcolor{gregoriocolor}{D} & \textcolor{gregoriocolor}{E} & F & \textcolor{gregoriocolor}{F} & \textcolor{gregoriocolor}{G} & \textcolor{gregoriocolor}{H} & \textcolor{gregoriocolor}{M} & \textcolor{gregoriocolor}{N} & \textcolor{gregoriocolor}{P} \\
 3 & 4 & 5 & 6 & 7 & 8 & 7 & 8 & 9 & 10 & 11 & 12 \\
\end{tabularx}

\begin{paracol}{2}
\selectlanguage{latin}
\lettrine[lines=2]{M}{essánæ,} in Sicília, 
 natális sanctórum Mártyrum Plácidi Mónachi, e beáti Benedícti Abbátis 
 discípulis, et ejus fratrum Eutychii et Victoríni, ac soróris eórum Fláviæ 
 Vírginis, itémque Donáti, Firmáti Diáconi, Fausti et aliórum trigínta 
 Monachórum, qui omnes a Manúcha piráta, pro Christi fide, necáti sunt.
\switchcolumn
\selectlanguage{english}
\lettrine[lines=2]{A}{t} Messina in Sicily, the birthday 
 of the holy martyrs Placidus, a monk who was a disciple of the blessed Abbot 
 Benedict, and of his brothers Eutychius and Victorinus, and the virgin 
 Flavia, their sister; also of Donatus, Firmatus, a deacon, Faustus, and 
 thirty other monks, who were murdered for the faith of Christ by the pirate 
 Manuchas.
\switchcolumn*
\selectlanguage{latin}
Apud Smyrnam item 
 natális beáti Thraséæ, Epíscopi Euméniæ, martyrio consummáti.
\switchcolumn
\selectlanguage{english}
At Smyrna, the birthday of blessed 
 Thraseas, bishop of Eumenia, who ended his career through martyrdom.
\switchcolumn*
\selectlanguage{latin}
Antisiodóri deposítio 
 sanctórum germanórum Firmáti Diáconi, et Flaviánæ Vírginis.
\switchcolumn
\selectlanguage{english}
At Auxerre, the death of the saintly 
 deacon Firmatus and the virgin Flaviana, his sister.
\switchcolumn*
\selectlanguage{latin}
Tréviris sanctórum 
 Mártyrum Palmátii et Sociórum; qui, in persecutióne Diocletiáni, sub 
 Rictiováro Præside, martyrium subiérunt.
\switchcolumn
\selectlanguage{english}
At Treves, the holy martyrs 
 Palmatius and his companions, who suffered martyrdom in the persecution of 
 Diocletian, under the governor Rictiovarus.
\switchcolumn*
\selectlanguage{latin}
Eódem die pássio sanctæ 
 Charitínæ Vírginis, quæ, sub Diocletiáno Imperatóre et Domítio Consulári, 
 ígnibus est cruciáta et in mare projécta, et, cum inde incólumis evasísset, 
 tandem, mánibus et pédibus abscíssis dentibúsque convúlsis, in oratióne 
 spíritum emísit.
\switchcolumn
\selectlanguage{english}
Also, under Emperor Diocletian and 
 the proconsul Domitius, St. Charitina, virgin. She was exposed to the 
 fire and thrown into the sea, but escaping uninjured, her hands and feet 
 were cut off and her teeth torn out, and finally she yielded up her spirit 
 in prayer.
\switchcolumn*
\selectlanguage{latin}
Ravénnæ sancti 
 Marcellíni, Epíscopi et Confessóris.
\switchcolumn
\selectlanguage{english}
At Ravenna, St. Marcellinus, bishop 
 and confessor.
\switchcolumn*
\selectlanguage{latin}
Valéntiæ, in Gállia, 
 sancti Apollináris Epíscopi, cujus vita virtútibus fuit illústris, et mors 
 signis ac prodígiis decoráta.
\switchcolumn
\selectlanguage{english}
At Valence in France, St. 
 Apollinaris, a bishop, renowned in life for virtues and in death for 
 miracles and prodigies.
\switchcolumn*
\selectlanguage{latin}
Eódem die sancti 
 Attiláni, Epíscopi Zamorénsis, quem beátus Urbánus Papa Secúndus in 
 Sanctórum númerum rétulit.
\switchcolumn
\selectlanguage{english}
Also, St. Attilanus, bishop of 
 Zamora, who was ranked among the saints by Pope Urban II.
\switchcolumn*
\selectlanguage{latin}
Romæ sanctæ Gallæ Víduæ, 
 fíliæ Symmachi Cónsulis, quæ, viro suo defúncto, apud Ecclésiam beáti Petri 
 multis annis oratióni, eleemósynis, jejúniis aliísque sanctis opéribus 
 inténta permánsit; cujus felicíssimum tránsitum sanctus Gregórius Papa 
 descrípsit.
\switchcolumn
\selectlanguage{english}
At Rome, St. Galla, widow, daughter 
 of the consul Symmachus. After the death of her husband, she remained 
 for many years near the church of St. Peter, devoted to prayer, almsgiving, 
 fasting, and other pious works. Her most happy death has been 
 described by Pope St. Gregory.
\switchcolumn*
\selectlanguage{latin}
\end{paracol}


% ---- martyrology/mart10/mart1006.htm
\needspace{10\baselineskip}
\begin{paracol}{2}
\selectlanguage{latin}
\begin{center}{\color{gregoriocolor} Prídie Nonas Octóbris. 
 Luna\dots\ }\end{center}
\switchcolumn
\selectlanguage{english}
\begin{center}{\color{gregoriocolor} The   Sixth Day of 
 October. The\dots\ Day of the Moon.}\end{center}
\end{paracol}

\noindent\begin{tabularx}{\linewidth}{*{19}{>{\centering\arraybackslash}X}}
 \textcolor{gregoriocolor}{a} & \textcolor{gregoriocolor}{b} & \textcolor{gregoriocolor}{c} & \textcolor{gregoriocolor}{d} & \textcolor{gregoriocolor}{e} & \textcolor{gregoriocolor}{f} & \textcolor{gregoriocolor}{g} & \textcolor{gregoriocolor}{h} & \textcolor{gregoriocolor}{i} & \textcolor{gregoriocolor}{k} & \textcolor{gregoriocolor}{l} & \textcolor{gregoriocolor}{m} & \textcolor{gregoriocolor}{n} & \textcolor{gregoriocolor}{p} & \textcolor{gregoriocolor}{q} & \textcolor{gregoriocolor}{r} & \textcolor{gregoriocolor}{s} & \textcolor{gregoriocolor}{t} & \textcolor{gregoriocolor}{u} \\
 14 & 15 & 16 & 17 & 18 & 19 & 20 & 21 & 22 & 23 & 24 & 25 & 26 & 27 & 28 & 29 & 1 & 2 & 3 \\
\end{tabularx}
\vspace{0.5\baselineskip}
\noindent\begin{tabularx}{\linewidth}{*{12}{>{\centering\arraybackslash}X}}
 \textcolor{gregoriocolor}{A} & \textcolor{gregoriocolor}{B} & \textcolor{gregoriocolor}{C} & \textcolor{gregoriocolor}{D} & \textcolor{gregoriocolor}{E} & F & \textcolor{gregoriocolor}{F} & \textcolor{gregoriocolor}{G} & \textcolor{gregoriocolor}{H} & \textcolor{gregoriocolor}{M} & \textcolor{gregoriocolor}{N} & \textcolor{gregoriocolor}{P} \\
 4 & 5 & 6 & 7 & 8 & 9 & 8 & 9 & 10 & 11 & 12 & 13 \\
\end{tabularx}

\begin{paracol}{2}
\selectlanguage{latin}
\lettrine[lines=2]{I}{n} monastério Turris, 
 diœcésis Squillacénsis, in Calábria, sancti Brunónis Confessóris, qui 
 Ordinis Carthusianórum fuit Institútor.
\switchcolumn
\selectlanguage{english}
\lettrine[lines=2]{I}{n} the Monastery De Torre, in the 
 diocese of Squillace in Calabria, St. Bruno, confessor, founder of the Order 
 of the Carthusians.
\switchcolumn*
\selectlanguage{latin}
Laodicéæ, in Phrygia, 
 beáti Ságaris, Epíscopi et Mártyris; qui éxstitit unus de antíquis Pauli 
 Apóstoli discípulis.
\switchcolumn
\selectlanguage{english}
At Laodicea, the blessed bishop and 
 martyr Sagar, one of the first disciples of the apostle Paul.
\switchcolumn*
\selectlanguage{latin}
Antisiodóri sancti 
 Románi Epíscopi et Mártyris.
\switchcolumn
\selectlanguage{english}
At Auxerre, St. Romanus, bishop and 
 martyr.
\switchcolumn*
\selectlanguage{latin}
Cápuæ natális sanctórum 
 Mártyrum Marcélli, Cásti, Æmílii et Saturníni.
\switchcolumn
\selectlanguage{english}
At Capua, the birthday of the holy 
 martyrs Marcellus, Castus, Aemílius, and Saturninus.
\switchcolumn*
\selectlanguage{latin}
Tréviris commemorátio 
 innumerabílium fere Mártyrum, qui, in persecutióne Diocletiáni, sub 
 Rictiováro Præside, ob Christi fidem, vário mortis génere necáti sunt.
\switchcolumn
\selectlanguage{english}
At Treves, the commemoration of 
 innumerable martyrs, who were put death for the faith in various manners, 
 under the governor Rictiovarus, in the persecution of Diocletian.
\switchcolumn*
\selectlanguage{latin}
Agénni, in Gállia, 
 natális sanctæ Fídei, Vírginis et Mártyris; cujus exémplo beátus Caprásius, 
 ad martyrium animátus, agónem suum tertiodécimo Kaléndas Novémbris felíciter 
 consummávit.
\switchcolumn
\selectlanguage{english}
At Agen in France, the birthday of 
 St. Faith, virgin and martyr, by whose example blessed Caprasius was aroused 
 to martyrdom, and by martyrdom happily fulfilled his own trial.
\switchcolumn*
\selectlanguage{latin}
Item sanctæ Erótidis 
 Mártyris, quæ, Christi amóre succénsa, ignis superávit incéndium.
\switchcolumn
\selectlanguage{english}
Also, St. Erotis martyr, who, aflame 
 with love for Christ, triumphed over the flames of fire.
\switchcolumn*
\selectlanguage{latin}
Opitérgii, in Venetórum 
 fínibus, sancti Magni Epíscopi, cujus corpus Venétiis requiéscit.
\switchcolumn
\selectlanguage{english}
At Oderzo, in the neighbourhood of 
 Venice, St. Magnus, bishop, whose body rests at Venice.
\switchcolumn*
\selectlanguage{latin}
Neápoli, in Campánia, 
 deposítio sanctæ Maríæ-Francíscæ a Quinque Vulnéribus Dómini nostri Jesu 
 Christi, Vírginis, ex tértio Ordine sancti Francísci, quæ, virtútibus et 
 miráculis clara, a Pio Papa Nono sanctis Virgínibus adscrípta fuit.
\switchcolumn
\selectlanguage{english}
At Naples in Campania, the death of 
 St. Mary Frances of the Five Wounds of Our Lord Jesus Christ, a nun of the 
 Third Order of St. Francis. Because of her reputation for virtues and 
 the working of miracles, she was placed among the holy virgins by Pope Pius 
 IX.
\switchcolumn*
\selectlanguage{latin}
\end{paracol}


% ---- martyrology/mart10/mart1007.htm
\needspace{10\baselineskip}
\begin{paracol}{2}
\selectlanguage{latin}
\begin{center}{\color{gregoriocolor} Nonis Octóbris. 
 Luna\dots\ }\end{center}
\switchcolumn
\selectlanguage{english}
\begin{center}{\color{gregoriocolor} The   Seventh Day of 
 October. The\dots\ Day of the Moon.}\end{center}
\end{paracol}

\noindent\begin{tabularx}{\linewidth}{*{19}{>{\centering\arraybackslash}X}}
 \textcolor{gregoriocolor}{a} & \textcolor{gregoriocolor}{b} & \textcolor{gregoriocolor}{c} & \textcolor{gregoriocolor}{d} & \textcolor{gregoriocolor}{e} & \textcolor{gregoriocolor}{f} & \textcolor{gregoriocolor}{g} & \textcolor{gregoriocolor}{h} & \textcolor{gregoriocolor}{i} & \textcolor{gregoriocolor}{k} & \textcolor{gregoriocolor}{l} & \textcolor{gregoriocolor}{m} & \textcolor{gregoriocolor}{n} & \textcolor{gregoriocolor}{p} & \textcolor{gregoriocolor}{q} & \textcolor{gregoriocolor}{r} & \textcolor{gregoriocolor}{s} & \textcolor{gregoriocolor}{t} & \textcolor{gregoriocolor}{u} \\
 15 & 16 & 17 & 18 & 19 & 20 & 21 & 22 & 23 & 24 & 25 & 26 & 27 & 28 & 29 & 1 & 2 & 3 & 4 \\
\end{tabularx}
\vspace{0.5\baselineskip}
\noindent\begin{tabularx}{\linewidth}{*{12}{>{\centering\arraybackslash}X}}
 \textcolor{gregoriocolor}{A} & \textcolor{gregoriocolor}{B} & \textcolor{gregoriocolor}{C} & \textcolor{gregoriocolor}{D} & \textcolor{gregoriocolor}{E} & F & \textcolor{gregoriocolor}{F} & \textcolor{gregoriocolor}{G} & \textcolor{gregoriocolor}{H} & \textcolor{gregoriocolor}{M} & \textcolor{gregoriocolor}{N} & \textcolor{gregoriocolor}{P} \\
 5 & 6 & 7 & 8 & 9 & 10 & 9 & 10 & 11 & 12 & 13 & 14 \\
\end{tabularx}

\begin{paracol}{2}
\selectlanguage{latin}
\lettrine[lines=2]{F}{estum} sacratíssimi 
 Rosárii beátæ Maríæ Vírginis; itémque sanctæ Maríæ de Victória commemorátio, 
 quam sanctus Pius Quintus, Póntifex Máximus, ob insígnem victóriam a 
 Christiánis bello naváli, ejúsdem sanctíssimæ Dei Genitrícis auxílio, hac 
 ipsa die de Turcis reportátam, quotánnis fíeri instítuit.
\switchcolumn
\selectlanguage{english}
\lettrine[lines=2]{T}{he} Feast of the Most Holy Rosary of 
 the blessed Virgin Mary, and the commemoration of St. Mary of Victory, which 
 Pope Pius V instituted to be kept yearly in memory of the great victory 
 granted on this day in a naval battle to the Christians over the Turks, by 
 the help of the Mother of God.
\switchcolumn*
\selectlanguage{latin}
Romæ, via Ardeatína, 
 sancti Marci, Papæ et Confessóris.
\switchcolumn
\selectlanguage{english}
At Rome, on the Ardeatine Way, the 
 death of St. Mark, pope and confessor.
\switchcolumn*
\selectlanguage{latin}
In Província quæ 
 nuncupátur Augústa Euphratésia, sanctórum Mártyrum Sérgii et Bacchi, 
 nobílium Romanórum, sub Maximiáno Imperatóre. Ex his Bacchus támdiu 
 nervis crudis cæsus est, quoadúsque, toto córpore discíssus, in Christi 
 confessióne emítteret spíritum; Sérgius vero clavátis cothúrnis pedes 
 indútus, et, cum in fide fixus manéret, data senténtia, jussus est decollári. 
 Beáti autem Sérgii nómine locus ubi quiéscit, Sergiópolis appellátus est, 
 et, ob præclára mirácula, frequénti Christianórum concúrsu honorátur.
\switchcolumn
\selectlanguage{english}
In the province of the Euphrates, 
 the holy martyrs Sergius and Bacchus, noble Romans, in the time of Emperor 
 Maximian. Bacchus was scourged with rough sinews until his body was 
 completely mangled, and breathed his last in the confession of Christ. 
 Sergius had his feet forced into shoes full of sharp-pointed nails, but, 
 remaining unshaken in the faith, he was sentenced to be beheaded. The 
 place where he rests is called after him Sergiopolis, and, on account of the 
 frequent miracles wrought there, is honoured by large gatherings of 
 Christians.
\switchcolumn*
\selectlanguage{latin}
Romæ sanctórum Mártyrum 
 Marcélli et Apuléji, qui prius quidem Simóni mago adhæsérunt; sed, vidéntes 
 mirabília quæ per Apóstolum Petrum Dóminus operabátur, ambo, relícto Simóne, 
 se doctrínæ Apostólicæ tradidérunt, ac, post passiónem Apostolórum, sub 
 Aureliáno Consulári, corónam martyrii reportárunt, sepultíque sunt non longe 
 ab Urbe.
\switchcolumn
\selectlanguage{english}
At Rome, the holy martyrs Marcellus 
 and Apulcius, who at first were followers of Simon Magus, but seeing the 
 wonders which the Lord performed through the apostle Peter, they abandoned 
 Simon and embraced the apostolic doctrine. After the death of the 
 apostles, under the proconsul Aurelian, they won the crown of martyrdom and 
 were buried near the city.
\switchcolumn*
\selectlanguage{latin}
Item apud Augústam 
 Euphratésiam sanctæ Júliæ Vírginis, quæ, sub Marciáno Præside, martyrium 
 consummávit.
\switchcolumn
\selectlanguage{english}
Also in the province of the 
 Euphrates, St. Julia, virgin, who suffered martyrdom under the governor 
 Marcian.
\switchcolumn*
\selectlanguage{latin}
Patávii sanctæ Justínæ, 
 Vírginis et Mártyris; quæ, a beáto Prosdócimo, sancti Petri discípulo, 
 baptizáta, et, cum in fide Christi constánter persísteret, Máximi Præsidis 
 jussu, transverberáta gládio, migrávit ad Dóminum.
\switchcolumn
\selectlanguage{english}
At Padua, St. Justina, virgin and 
 martyr, who was baptized by blessed Prosdocimus, a disciple of St. Peter. 
 Because she remained firm in the faith of Christ, she was put to the sword 
 by order of the governor Maximus, and thus went to God.
\switchcolumn*
\selectlanguage{latin}
Apud Bitúricas, in 
 Aquitánia, sancti Augústi, Presbyteri et Confessóris.
\switchcolumn
\selectlanguage{english}
At Bourges, St. Augustus, priest and 
 confessor.
\switchcolumn*
\selectlanguage{latin}
In pago Rheménsi sancti 
 Heláni Presbyteri.
\switchcolumn
\selectlanguage{english}
In the diocese of Rheims, St. 
 Helanus, priest.
\switchcolumn*
\selectlanguage{latin}
In Suécia Translátio 
 córporis sanctæ Birgíttæ Víduæ.
\switchcolumn
\selectlanguage{english}
In Sweden, the translation of the 
 body of St. Bridget, widow.
\switchcolumn*
\selectlanguage{latin}
\end{paracol}


% ---- martyrology/mart10/mart1008.htm
\needspace{10\baselineskip}
\begin{paracol}{2}
\selectlanguage{latin}
\begin{center}{\color{gregoriocolor} Octávo Idus Octóbris. 
 Luna\dots\ }\end{center}
\switchcolumn
\selectlanguage{english}
\begin{center}{\color{gregoriocolor} The   Eighth Day of 
 October. The\dots\ Day of the Moon.}\end{center}
\end{paracol}

\noindent\begin{tabularx}{\linewidth}{*{19}{>{\centering\arraybackslash}X}}
 \textcolor{gregoriocolor}{a} & \textcolor{gregoriocolor}{b} & \textcolor{gregoriocolor}{c} & \textcolor{gregoriocolor}{d} & \textcolor{gregoriocolor}{e} & \textcolor{gregoriocolor}{f} & \textcolor{gregoriocolor}{g} & \textcolor{gregoriocolor}{h} & \textcolor{gregoriocolor}{i} & \textcolor{gregoriocolor}{k} & \textcolor{gregoriocolor}{l} & \textcolor{gregoriocolor}{m} & \textcolor{gregoriocolor}{n} & \textcolor{gregoriocolor}{p} & \textcolor{gregoriocolor}{q} & \textcolor{gregoriocolor}{r} & \textcolor{gregoriocolor}{s} & \textcolor{gregoriocolor}{t} & \textcolor{gregoriocolor}{u} \\
 16 & 17 & 18 & 19 & 20 & 21 & 22 & 23 & 24 & 25 & 26 & 27 & 28 & 29 & 1 & 2 & 3 & 4 & 5 \\
\end{tabularx}
\vspace{0.5\baselineskip}
\noindent\begin{tabularx}{\linewidth}{*{12}{>{\centering\arraybackslash}X}}
 \textcolor{gregoriocolor}{A} & \textcolor{gregoriocolor}{B} & \textcolor{gregoriocolor}{C} & \textcolor{gregoriocolor}{D} & \textcolor{gregoriocolor}{E} & F & \textcolor{gregoriocolor}{F} & \textcolor{gregoriocolor}{G} & \textcolor{gregoriocolor}{H} & \textcolor{gregoriocolor}{M} & \textcolor{gregoriocolor}{N} & \textcolor{gregoriocolor}{P} \\
 6 & 7 & 8 & 9 & 10 & 11 & 10 & 11 & 12 & 13 & 14 & 15 \\
\end{tabularx}

\begin{paracol}{2}
\selectlanguage{latin}
\lettrine[lines=2]{S}{anctæ} Birgíttæ Víduæ, 
 cujus dies natális décimo Kaléndas Augústi, ac Translátio Nonis Octóbris 
 recensétur.
\switchcolumn
\selectlanguage{english}
\lettrine[lines=2]{S}{t.} Bridget, widow, whose birthday 
 is observed on the 23rd of July, and the translation of her holy body on the 
 7th of October.
\switchcolumn*
\selectlanguage{latin}
Eódem die natális beáti 
 Simeónis senis, qui in Evangélio Dóminum Jesum, præsentátum in Templo, suis 
 in ulnis accepísse ac de illo prophetásse légitur.
\switchcolumn
\selectlanguage{english}
Also, the birthday of blessed 
 Simeon, an aged man, who as we read in the Gospel, took our Lord Jesus in 
 his arms and prophesied concerning him when he was presented in the Temple.
\switchcolumn*
\selectlanguage{latin}
Laodicéæ, in Phrygia, 
 sancti Artémonis Presbyteri, qui per ignem, sub Diocletiáno, martyrii 
 corónam accépit.
\switchcolumn
\selectlanguage{english}
At Laodicea in Phrygia, during the 
 reign of Diocletian, St. Artemon, a priest, who gained the crown of 
 martyrdom by fire.
\switchcolumn*
\selectlanguage{latin}
Thessalonícæ sancti 
 Demétrii Procónsulis, qui, cum plúrimos ad Christi fidem perdúceret, ideo, 
 Maximiáni Imperatóris jussu lánceis confóssus, martyrium consummávit.
\switchcolumn
\selectlanguage{english}
At Thessalonica, St. Demetrius, a 
 proconsul. For having brought many to the faith of Christ he was 
 pierced with spears by order of Emperor Maximian, and thus completed his 
 martyrdom.
\switchcolumn*
\selectlanguage{latin}
Ibídem sancti Néstoris 
 Mártyris.
\switchcolumn
\selectlanguage{english}
In the same place, St. Nestor, 
 martyr.
\switchcolumn*
\selectlanguage{latin}
Híspali, in Hispánia, 
 sancti Petri Mártyris.
\switchcolumn
\selectlanguage{english}
At Seville in Spain, St. Peter, 
 martyr.
\switchcolumn*
\selectlanguage{latin}
Cæsaréæ, in Palæstína, 
 pássio sanctæ Reparátæ, Vírginis et Mártyris; quæ, cum nollet idólis 
 sacrificáre, sub Décio Imperatóre, váriis tormentórum genéribus cruciátur, 
 ac demum gládio percútitur. Ipsíus autem ánima, in colúmbæ spécie, de 
 córpore égredi cælúmque conscéndere visa est.
\switchcolumn
\selectlanguage{english}
At Caesarea in Palestine, in the 
 reign of Decius, St. Reparata, virgin and martyr. For refusing to 
 sacrifice to idols, she was subjected to various kinds of torments and was 
 finally struck with the sword. Her soul was seen to leave her body in 
 the form of a dove and ascend to heaven.
\switchcolumn*
\selectlanguage{latin}
In território 
 Laudunénsi natális sanctæ Benedíctæ, Vírginis et Mártyris.
\switchcolumn
\selectlanguage{english}
In the country of Laon, St. 
 Benedicta, virgin and martyr.
\switchcolumn*
\selectlanguage{latin}
Ancónæ sanctárum 
 Palatiátis et Lauréntiæ, quæ, in persecutióne Diocletiáni, sub Dióne Præside, 
 in exsílium deportátæ, labóribus et ærúmnis conféctæ sunt.
\switchcolumn
\selectlanguage{english}
At Ancona, Saints Palatius and 
 Laurentia, who were sent into exile during the persecution of Diocletian, 
 under the governor Dion, and were overcome by the weight of toil and misery.
\switchcolumn*
\selectlanguage{latin}
Rotómagi sancti Evódii, 
 Epíscopi et Confessóris.
\switchcolumn
\selectlanguage{english}
At Rouen, St. Evodius, bishop and 
 confessor.
\switchcolumn*
\selectlanguage{latin}
Hierosólymis sanctæ 
 Pelágiæ, cognoménto Pæniténtis.
\switchcolumn
\selectlanguage{english}
At Jerusalem, St. Palagia, surnamed 
 the Penitent.
\switchcolumn*
\selectlanguage{latin}
\end{paracol}


% ---- martyrology/mart10/mart1009.htm
\needspace{10\baselineskip}
\begin{paracol}{2}
\selectlanguage{latin}
\begin{center}{\color{gregoriocolor} Séptimo Idus Octóbris. 
 Luna\dots\ }\end{center}
\switchcolumn
\selectlanguage{english}
\begin{center}{\color{gregoriocolor} The   Ninth Day of 
 October. The\dots\ Day of the Moon.}\end{center}
\end{paracol}

\noindent\begin{tabularx}{\linewidth}{*{19}{>{\centering\arraybackslash}X}}
 \textcolor{gregoriocolor}{a} & \textcolor{gregoriocolor}{b} & \textcolor{gregoriocolor}{c} & \textcolor{gregoriocolor}{d} & \textcolor{gregoriocolor}{e} & \textcolor{gregoriocolor}{f} & \textcolor{gregoriocolor}{g} & \textcolor{gregoriocolor}{h} & \textcolor{gregoriocolor}{i} & \textcolor{gregoriocolor}{k} & \textcolor{gregoriocolor}{l} & \textcolor{gregoriocolor}{m} & \textcolor{gregoriocolor}{n} & \textcolor{gregoriocolor}{p} & \textcolor{gregoriocolor}{q} & \textcolor{gregoriocolor}{r} & \textcolor{gregoriocolor}{s} & \textcolor{gregoriocolor}{t} & \textcolor{gregoriocolor}{u} \\
 17 & 18 & 19 & 20 & 21 & 22 & 23 & 24 & 25 & 26 & 27 & 28 & 29 & 1 & 2 & 3 & 4 & 5 & 6 \\
\end{tabularx}
\vspace{0.5\baselineskip}
\noindent\begin{tabularx}{\linewidth}{*{12}{>{\centering\arraybackslash}X}}
 \textcolor{gregoriocolor}{A} & \textcolor{gregoriocolor}{B} & \textcolor{gregoriocolor}{C} & \textcolor{gregoriocolor}{D} & \textcolor{gregoriocolor}{E} & F & \textcolor{gregoriocolor}{F} & \textcolor{gregoriocolor}{G} & \textcolor{gregoriocolor}{H} & \textcolor{gregoriocolor}{M} & \textcolor{gregoriocolor}{N} & \textcolor{gregoriocolor}{P} \\
 7 & 8 & 9 & 10 & 11 & 12 & 11 & 12 & 13 & 14 & 15 & 16 \\
\end{tabularx}

\begin{paracol}{2}
\selectlanguage{latin}
\lettrine[lines=2]{R}{omæ} sancti Joánnis 
 Leonárdi, Confessóris, Fundatóris Congregatiónis Clericórum Regulárium a 
 Matre Dei, labóribus et miráculis clari, cujus ópera Missiónes a Propagánda 
 Fide institútæ sunt.
\switchcolumn
\selectlanguage{english}
\lettrine[lines=2]{A}{t} Rome, St. John Leonard, 
 confessor, founder of the Congregation of Clerks Regular of the Mother of 
 God, renowned for his labours and miracles, and by whose zeal were begun 
 missions for the propagation of the faith.
\switchcolumn*
\selectlanguage{latin}
Lutétiæ Parisiórum 
 natális sanctórum Mártyrum Dionysii Areopagítæ Epíscopi, Rústici Presbyteri, 
 et Eleuthérii Diáconi. Ex his Dionysius, ab Apóstolo Paulo baptizátus, 
 primus Atheniénsium Epíscopus ordinátus est; deínde Romam venit, atque inde 
 a beáto Cleménte, Románo Pontífice, in Gállias prædicándi grátia diréctus 
 est, et ad præfátam urbem devénit; ibíque, cum per áliquot annos commíssum 
 sibi opus fidéliter prosecútus esset, tandem, a Præfécto Fescénnio, post 
 gravíssima tormentórum génera, una cum Sóciis, gládio animadvérsus, 
 martyrium complévit.
\switchcolumn
\selectlanguage{english}
At Paris, the birthday of the holy 
 martyrs Denis the Areopagite, a bishop, Rusticus, a priest, and Eleutherius, 
 a deacon. Denis was baptized by the apostle St. Paul, and consecrated 
 first bishop of Athens. Then going to Rome, he was sent to France by 
 the blessed Roman Pontiff Clement to preach the Gospel. He proceeded 
 to Paris, and after having for some years faithfully filled the office 
 entrusted to him, he was subjected to the severest kinds of torments by the 
 prefect Fescennius, and at length was beheaded with his companions, thus 
 completing his martyrdom.
\switchcolumn*
\selectlanguage{latin}
Eódem die memória 
 sancti Abrahæ, Patriárchæ et ómnium credéntium Patris.
\switchcolumn
\selectlanguage{english}
On the same day, the commemoration 
 of the holy patriarch Abraham, father of all believers.
\switchcolumn*
\selectlanguage{latin}
Apud Cassínum sancti 
 Deúsdedit Abbátis, qui, a Sicárdo tyránno in cárcerem trusus, illic, 
 fame et ærúmnis conféctus, réddidit spíritum.
\switchcolumn
\selectlanguage{english}
At Monte Cassino, St. Deusdedit, 
 abbot, who was cast into prison by the tyrant Sicardus, and being there 
 consumed with hunger and misery, yielded up his soul.
\switchcolumn*
\selectlanguage{latin}
Apud Júliam, in 
 território Parménsi, via Cláudia, sancti Domníni Mártyris; qui, sub 
 Maximiáno Imperatóre, cum vellet persecutiónis rábiem declináre, a 
 persequéntibus est comprehénsus, et, gládio transverberátus, glorióse 
 occúbuit.
\switchcolumn
\selectlanguage{english}
At Julia, in the region of Parma, on 
 the Via Claudia, St. Domninus, martyr. Under the Emperor Maximian, in 
 the rage of persecution, he was taken by the persecutors and died gloriously 
 by being pierced with a sword.
\switchcolumn*
\selectlanguage{latin}
In Hannónia sancti 
 Gisléni, Epíscopi et Confessóris; qui, relícto Episcopátu, Mónachi vitam in 
 monastério a se constrúcto exércuit, et multis virtútibus cláruit.
\switchcolumn
\selectlanguage{english}
In Hainault, St. Gislenus, bishop 
 and confessor, who resigning his bishopric, led the monastic life in a 
 monastery built by himself, and was distinguished by many virtues.
\switchcolumn*
\selectlanguage{latin}
Valéntiæ, in Hispánia 
 Tarraconénsi, sancti Ludovíci Bertrándi, ex Ordine Prædicatórum, Confessóris; 
 qui, apostólico spíritu clarus, Evangélium quod Americánis prædicáverat, 
 vitæ innocéntia multísque éditis miráculis confirmávit.
\switchcolumn
\selectlanguage{english}
At Valencia in Spain, St. Louis 
 Bertrand, of the Order of Preachers. Being filled with the apostolic 
 spirit, he confirmed by the innocency of his life and the working of many 
 miracles the Gospel which he had preached in America.
\switchcolumn*
\selectlanguage{latin}
Hierosólymis sanctórum 
 Androníci et Athanásiæ cónjugis.
\switchcolumn
\selectlanguage{english}
At Jerusalem, Saints Andronicus and 
 his wife Athanasia.
\switchcolumn*
\selectlanguage{latin}
Antiochíæ sanctæ Públiæ 
 Abbatíssæ, quæ, transeúnte Juliáno Apóstata, Davídicum illud cum suis 
 Virgínibus canens: « Simulácra Géntium argéntum et aurum », et « Símiles 
 illis fiant qui fáciunt ea », Imperatóris jussu, álapis cæsa est, et 
 gráviter objurgáta.
\switchcolumn
\selectlanguage{english}
At Antioch, St. Publia, abbess. 
 While Julian the Apostate was passing by, she and her religious sang these 
 words of David: ``The idols of the Gentiles are silver and gold,'' and ``Let 
 them that make them be like unto them.'' By the command of the emperor, 
 she was struck on the face and severely rebuked.
\switchcolumn*
\selectlanguage{latin}
\end{paracol}


% ---- martyrology/mart10/mart1010.htm
\needspace{10\baselineskip}
\begin{paracol}{2}
\selectlanguage{latin}
\begin{center}{\color{gregoriocolor} Sexto Idus Octóbris. 
 Luna\dots\ }\end{center}
\switchcolumn
\selectlanguage{english}
\begin{center}{\color{gregoriocolor} The   Tenth Day of 
 October. The\dots\ Day of the Moon.}\end{center}
\end{paracol}

\noindent\begin{tabularx}{\linewidth}{*{19}{>{\centering\arraybackslash}X}}
 \textcolor{gregoriocolor}{a} & \textcolor{gregoriocolor}{b} & \textcolor{gregoriocolor}{c} & \textcolor{gregoriocolor}{d} & \textcolor{gregoriocolor}{e} & \textcolor{gregoriocolor}{f} & \textcolor{gregoriocolor}{g} & \textcolor{gregoriocolor}{h} & \textcolor{gregoriocolor}{i} & \textcolor{gregoriocolor}{k} & \textcolor{gregoriocolor}{l} & \textcolor{gregoriocolor}{m} & \textcolor{gregoriocolor}{n} & \textcolor{gregoriocolor}{p} & \textcolor{gregoriocolor}{q} & \textcolor{gregoriocolor}{r} & \textcolor{gregoriocolor}{s} & \textcolor{gregoriocolor}{t} & \textcolor{gregoriocolor}{u} \\
 18 & 19 & 20 & 21 & 22 & 23 & 24 & 25 & 26 & 27 & 28 & 29 & 1 & 2 & 3 & 4 & 5 & 6 & 7 \\
\end{tabularx}
\vspace{0.5\baselineskip}
\noindent\begin{tabularx}{\linewidth}{*{12}{>{\centering\arraybackslash}X}}
 \textcolor{gregoriocolor}{A} & \textcolor{gregoriocolor}{B} & \textcolor{gregoriocolor}{C} & \textcolor{gregoriocolor}{D} & \textcolor{gregoriocolor}{E} & F & \textcolor{gregoriocolor}{F} & \textcolor{gregoriocolor}{G} & \textcolor{gregoriocolor}{H} & \textcolor{gregoriocolor}{M} & \textcolor{gregoriocolor}{N} & \textcolor{gregoriocolor}{P} \\
 8 & 9 & 10 & 11 & 12 & 13 & 12 & 13 & 14 & 15 & 16 & 17 \\
\end{tabularx}

\begin{paracol}{2}
\selectlanguage{latin}
\lettrine[lines=2]{S}{ancti} Francísci Bórgiæ, 
 Sacerdótis e Societáte Jesu et Confessóris, cujus dies natális prídie 
 Kaléndas Octóbris recensétur.
\switchcolumn
\selectlanguage{english}
\lettrine[lines=2]{S}{t.} Francis Borgia, confessor, 
 priest of the Society of Jesus, whose birthday is mentioned on the 30th of 
 September.
\switchcolumn*
\selectlanguage{latin}
Apud Septam, in 
 Mauritánia Tingitána, pássio sanctórum septem Mártyrum, ex Ordine Minórum, 
 scílicet Daniélis, Samuélis, Angeli, Leónis, Nicolái, Hugolíni, et Domni; 
 qui, cum essent omnes præter Domnum Sacerdótes, ibídem, ob Evangélii 
 prædicatiónem et Mahuméticæ confutatiónem sectæ, a Saracénis contumélias, 
 víncula et flagélla perpéssi, demum, capítibus abscíssis, martyrii palmam 
 adépti sunt.
\switchcolumn
\selectlanguage{english}
At Ceuta in Morocco, the passion of 
 seven holy martyrs of the Order of Friars Minor: Daniel, Samuel, Angelus, 
 Leo, Nicholas, Ugolino, and Domnus, all of whom were priests except Domnus. 
 Because they had preached the Gospel and put to silence the doctrines of 
 Mohammed, they suffered insults, fetters, and scourgings from the Saracens 
 in that place. They were at last beheaded and thus obtained the palm 
 of martyrdom.
\switchcolumn*
\selectlanguage{latin}
Colóniæ Agrippínæ 
 sancti Gereónis Mártyris, cum áliis trecéntis decem et octo; qui pro vera 
 pietáte, in persecutióne Maximiáni, gládiis patiéntes colla subdidérunt.
\switchcolumn
\selectlanguage{english}
At Cologne, in the persecution of 
 Maximian, St. Gereon and three hundred and eighteen other martyrs who 
 patiently bowed to the sword for the true religion.
\switchcolumn*
\selectlanguage{latin}
In território ejúsdem 
 urbis sanctórum Victóris et Sociórum Mártyrum.
\switchcolumn
\selectlanguage{english}
In the neighbourhood of the same 
 city, the holy martyrs Victor and his companions.
\switchcolumn*
\selectlanguage{latin}
Bonnæ, in Germánia, sanctórum Mártyrum Cássii et Floréntii, cum áliis plúrimis.
\switchcolumn
\selectlanguage{english}
At Bonn in Germany, the holy martyrs 
 Cassius and Florentius, with many others.
\switchcolumn*
\selectlanguage{latin}
Nicomedíæ sanctórum 
 Mártyrum Eulámpii, et soróris Eulámpiæ Vírginis. Hæc porro, cum 
 audísset pro Christo fratrem torquéri, in médiam turbam exsíluit, et, 
 fratrem amplexáta, huic sóciam se adjúnxit; atque ambo, conjécti in ollam 
 fervéntis ólei, sed nulla ex parte læsi, tandem, una cum áliis ducéntis, qui 
 eo miráculo permóti credidérunt in Christum, cápitis obtruncatióne martyrium 
 complevérunt.
\switchcolumn
\selectlanguage{english}
At Nicomedia, the holy martyrs 
 Eulampius, and his sister, the virgin Eulampia. Upon hearing that her 
 brother was tortured for Christ, she rushed through the crowd, embraced him, 
 and became his companion. Both were cast into a cauldron of boiling 
 oil, but being uninjured, their martyrdom was completed by beheading along 
 with two hundred others, who, impressed by the miracle, had believed in 
 Christ.
\switchcolumn*
\selectlanguage{latin}
In Creta ínsula beáti 
 Pinyti, inter Epíscopos nobilíssimi. Hic, Gnósiæ urbis Epíscopus, sub 
 Marco Antoníno Vero et Lúcio Aurélio Cómmodo flóruit, et in scriptis suis, 
 velut in quodam spéculo, vivéntem sui relíquit imáginem.
\switchcolumn
\selectlanguage{english}
On the island of Crete, blessed 
 Pinytus, most noble of bishops. He was bishop of Gnosia, and 
 flourished under Marcus Antoninus Verus and Lucius Aurelius Commodus. 
 He left in his writings, as in a mirror, a vivid picture of himself.
\switchcolumn*
\selectlanguage{latin}
Eboráci, in Anglia, 
 sancti Paulíni Epíscopi, qui fuit beáti Gregórii Papæ discípulus; et, una 
 cum áliis, ad prædicándum Evangélium illuc ab eo missus, Edwínum Regem 
 ejúsque pópulum ad Christi fidem convértit.
\switchcolumn
\selectlanguage{english}
At York in England, the holy bishop 
 Paulinus, disciple of the blessed pope Gregory. He was sent there by 
 that pope along with others to preach the Gospel, and he converted King 
 Edwin and his people to the faith of Christ.
\switchcolumn*
\selectlanguage{latin}
Populónii, in Túscia, 
 sancti Cerbónii, Epíscopi et Confessóris, qui (ut sanctus Gregórius Papa 
 refert) in vita et morte miráculis cláruit.
\switchcolumn
\selectlanguage{english}
At Piombino in Tuscany, St. 
 Cerbonius, bishop and confessor. St. Gregory relates that he was 
 renowned for miracles, both during life and after death.
\switchcolumn*
\selectlanguage{latin}
Verónæ sancti Cerbónii 
 Epíscopi.
\switchcolumn
\selectlanguage{english}
At Verona, another St. Cerbonius, 
 bishop.
\switchcolumn*
\selectlanguage{latin}
Cápuæ sancti Paulíni 
 Epíscopi.
\switchcolumn
\selectlanguage{english}
At Capua, St. Paulinus, bishop.
\switchcolumn*
\selectlanguage{latin}
\end{paracol}


% ---- martyrology/mart10/mart1011.htm
\needspace{10\baselineskip}
\begin{paracol}{2}
\selectlanguage{latin}
\begin{center}{\color{gregoriocolor} Quinto Idus Octóbris. 
 Luna\dots\ }\end{center}
\switchcolumn
\selectlanguage{english}
\begin{center}{\color{gregoriocolor} The   Eleventh Day of 
 October. The\dots\ Day of the Moon.}\end{center}
\end{paracol}

\noindent\begin{tabularx}{\linewidth}{*{19}{>{\centering\arraybackslash}X}}
 \textcolor{gregoriocolor}{a} & \textcolor{gregoriocolor}{b} & \textcolor{gregoriocolor}{c} & \textcolor{gregoriocolor}{d} & \textcolor{gregoriocolor}{e} & \textcolor{gregoriocolor}{f} & \textcolor{gregoriocolor}{g} & \textcolor{gregoriocolor}{h} & \textcolor{gregoriocolor}{i} & \textcolor{gregoriocolor}{k} & \textcolor{gregoriocolor}{l} & \textcolor{gregoriocolor}{m} & \textcolor{gregoriocolor}{n} & \textcolor{gregoriocolor}{p} & \textcolor{gregoriocolor}{q} & \textcolor{gregoriocolor}{r} & \textcolor{gregoriocolor}{s} & \textcolor{gregoriocolor}{t} & \textcolor{gregoriocolor}{u} \\
 19 & 20 & 21 & 22 & 23 & 24 & 25 & 26 & 27 & 28 & 29 & 1 & 2 & 3 & 4 & 5 & 6 & 7 & 8 \\
\end{tabularx}
\vspace{0.5\baselineskip}
\noindent\begin{tabularx}{\linewidth}{*{12}{>{\centering\arraybackslash}X}}
 \textcolor{gregoriocolor}{A} & \textcolor{gregoriocolor}{B} & \textcolor{gregoriocolor}{C} & \textcolor{gregoriocolor}{D} & \textcolor{gregoriocolor}{E} & F & \textcolor{gregoriocolor}{F} & \textcolor{gregoriocolor}{G} & \textcolor{gregoriocolor}{H} & \textcolor{gregoriocolor}{M} & \textcolor{gregoriocolor}{N} & \textcolor{gregoriocolor}{P} \\
 9 & 10 & 11 & 12 & 13 & 14 & 13 & 14 & 15 & 16 & 17 & 18 \\
\end{tabularx}

\begin{paracol}{2}
\selectlanguage{latin}
\lettrine[lines=1]{F}{estum} Maternitátis beátæ Maríæ Vírginis.
\switchcolumn
\selectlanguage{english}
\lettrine[lines=1]{T}{he} Motherhood of the Blessed Virgin Mary.
\switchcolumn*
\selectlanguage{latin}
Tarsi, in Cilícia, sanctárum mulíerum Zenáidis et Philoníllæ sorórum, quæ beáti Pauli Apóstoli 
 consanguíneæ et in fide fuérunt discípulæ.
\switchcolumn
\selectlanguage{english}
At Tarsus in Cilicia, the holy women 
 Zenaides and Philonilla, sisters, who were relatives of the blessed apostle 
 Paul and his disciples in the faith.
\switchcolumn*
\selectlanguage{latin}
In pago Vilcassíno, in 
 Gállia, pássio sanctórum Mártyrum Nicásii, qui erat Rotomagénsis Epíscopus, 
 Quiríni Presbyteri, Scubículi Diáconi, et Piéntiæ Vírginis, sub Præside 
 Fescénnio.
\switchcolumn
\selectlanguage{english}
In the neighbourhood of Vexin in 
 France, in the time of the governor Fescennius, the passion of the holy 
 martyrs Nicasius, bishop of Rouen, the priest Quirinus, the deacon 
 Scubiculus, and Pientia, a virgin.
\switchcolumn*
\selectlanguage{latin}
Vesontióne, in Gálliis, 
 sancti Germáni, Epíscopi et Mártyris.
\switchcolumn
\selectlanguage{english}
At Besançon in France, St. Germanus, 
 bishop and martyr.
\switchcolumn*
\selectlanguage{latin}
Item pássio sanctórum 
 Anastásii Presbyteri, Plácidi, Genésii et Sociórum.
\switchcolumn
\selectlanguage{english}
Also, the martyrdom of the Saints 
 Anastasius, a priest, Placidus, Genesius, and their companions.
\switchcolumn*
\selectlanguage{latin}
Tarsi, in Cilícia, natális sanctórum Mártyrum Tháraci, Probi et Andrónici; qui, in persecutióne 
 Diocletiáni, longo témpore cárceris squalóre afflícti, et tértio divérsis 
 torméntis et supplíciis examináti, tandem, in confessióne Christi, abscíssis 
 cervícibus, triúmphum glóriæ sunt adépti.
\switchcolumn
\selectlanguage{english}
At Tarsus in Cilicia, the birthday 
 of the holy martyrs Tharacus, Probus, and Andronicus, who endured a long and 
 painful imprisonment during the persecution of Diocletian. They were 
 three times subjected to diverse punishments and tortures, and finally 
 obtained a glorious triumph for the confession of Christ by having their 
 heads struck off.
\switchcolumn*
\selectlanguage{latin}
In Thebáide sancti 
 Sarmátæ, qui fuit discípulus beáti Antónii Abbátis, et a Saracénis pro 
 Christo necátus est.
\switchcolumn
\selectlanguage{english}
In Thebais, St. Sarmata, disciple of 
 the blessed abbot Anthony, who was put to death for Christ by the Saracens.
\switchcolumn*
\selectlanguage{latin}
Ucétiæ, in Gállia 
 Narbonénsi, sancti Firmíni, Epíscopi et Confessóris.
\switchcolumn
\selectlanguage{english}
At Uzea in France, St. Firmin, 
 bishop and confessor.
\switchcolumn*
\selectlanguage{latin}
Calótii, in diœcési 
 Asténsi, olim Papiénsi, sancti Alexándri Sauli, e Clericórum Regulárium 
 sancti Pauli Congregatióne, Epíscopi et Confessóris; quem, génere, 
 virtútibus, doctrína et miráculis clarum, Pius Décimus, Póntifex Máximus, 
 Sanctórum fastis adscrípsit.
\switchcolumn
\selectlanguage{english}
At Calozzo, in the diocese of Asti, 
 formerly that of Pavia, St. Alexander Sauli, bishop and confessor of the 
 Clerics Regular of St. Paul. He was of noble birth and renowned for 
 virtues, learning, and miracles. Pope Pius X placed him in the canon 
 of the saints.
\switchcolumn*
\selectlanguage{latin}
In monastério 
 Achadh-boénsi, in Hibérnia, sancti Cánici, Presbyteri et Abbátis.
\switchcolumn
\selectlanguage{english}
In the monastery of Aghaboe in 
 Ireland, St. Kenny, priest and abbot.
\switchcolumn*
\selectlanguage{latin}
Lyræ, in Bélgio, 
 deposítio sancti Gummári Confessóris,
\switchcolumn
\selectlanguage{english}
At Lier in Belgium, the death of St. 
 Gummarus, confessor.
\switchcolumn*
\selectlanguage{latin}
Apud Rhédones, in 
 Gállia, sancti Æmiliáni Confessóris.
\switchcolumn
\selectlanguage{english}
At Rennes in France, St. Emilian, 
 confessor.
\switchcolumn*
\selectlanguage{latin}
Verónæ sanctæ Placídiæ 
 Vírginis.
\switchcolumn
\selectlanguage{english}
At Verona, St. Placidia, virgin.
\switchcolumn*
\selectlanguage{latin}
\end{paracol}


% ---- martyrology/mart10/mart1012.htm
\needspace{10\baselineskip}
\begin{paracol}{2}
\selectlanguage{latin}
\begin{center}{\color{gregoriocolor} Quarto Idus Octóbris. 
 Luna\dots\ }\end{center}
\switchcolumn
\selectlanguage{english}
\begin{center}{\color{gregoriocolor} The   Twelfth Day of 
 October. The\dots\ Day of the Moon.}\end{center}
\end{paracol}

\noindent\begin{tabularx}{\linewidth}{*{19}{>{\centering\arraybackslash}X}}
 \textcolor{gregoriocolor}{a} & \textcolor{gregoriocolor}{b} & \textcolor{gregoriocolor}{c} & \textcolor{gregoriocolor}{d} & \textcolor{gregoriocolor}{e} & \textcolor{gregoriocolor}{f} & \textcolor{gregoriocolor}{g} & \textcolor{gregoriocolor}{h} & \textcolor{gregoriocolor}{i} & \textcolor{gregoriocolor}{k} & \textcolor{gregoriocolor}{l} & \textcolor{gregoriocolor}{m} & \textcolor{gregoriocolor}{n} & \textcolor{gregoriocolor}{p} & \textcolor{gregoriocolor}{q} & \textcolor{gregoriocolor}{r} & \textcolor{gregoriocolor}{s} & \textcolor{gregoriocolor}{t} & \textcolor{gregoriocolor}{u} \\
 20 & 21 & 22 & 23 & 24 & 25 & 26 & 27 & 28 & 29 & 1 & 2 & 3 & 4 & 5 & 6 & 7 & 8 & 9 \\
\end{tabularx}
\vspace{0.5\baselineskip}
\noindent\begin{tabularx}{\linewidth}{*{12}{>{\centering\arraybackslash}X}}
 \textcolor{gregoriocolor}{A} & \textcolor{gregoriocolor}{B} & \textcolor{gregoriocolor}{C} & \textcolor{gregoriocolor}{D} & \textcolor{gregoriocolor}{E} & F & \textcolor{gregoriocolor}{F} & \textcolor{gregoriocolor}{G} & \textcolor{gregoriocolor}{H} & \textcolor{gregoriocolor}{M} & \textcolor{gregoriocolor}{N} & \textcolor{gregoriocolor}{P} \\
 10 & 11 & 12 & 13 & 14 & 15 & 14 & 15 & 16 & 17 & 18 & 19 \\
\end{tabularx}

\begin{paracol}{2}
\selectlanguage{latin}
\lettrine[lines=2]{R}{omæ} sanctórum Mártyrum 
 Evágrii, Prisciáni, et Sociórum.
\switchcolumn
\selectlanguage{english}
\lettrine[lines=2]{A}{t} Rome, the holy martyrs Evagrius, 
 Priscian, and their companions.
\switchcolumn*
\selectlanguage{latin}
In Africa sanctórum 
 Confessórum et Mártyrum quátuor míllium nongentórum sexagínta sex, in 
 persecutióne Wandálica, sub Hunneríco, Rege Ariáno. Hi, cum essent 
 partim Epíscopi Ecclesiárum Dei, partim Presbyteri et Diáconi, associátis 
 sibi turbis fidélium populórum, pro defensióne cathólicæ veritátis in 
 horríbilis erémi exsílium trusi sunt; ex quibus plúrimi, dum crudéliter a 
 Mauris duceréntur, hastílium cuspídibus impúlsi ad curréndum et lapídibus 
 tunsi, álii, ligátis pédibus, velut cadávera per dura et áspera loca tracti 
 et síngulis membris discérpti, ad extrémum, várie excruciáti, martyrium 
 celebrárunt. Erant inter eos præcípui Sacerdótes Dómini, Felix et 
 Cypriánus Epíscopi.
\switchcolumn
\selectlanguage{english}
In Africa, four thousand nine 
 hundred and sixty-six holy confessors and martyrs in the persecution of the 
 Vandals under the Arian king Hunneric. Some of them were bishops of 
 the churches of God, some priests and deacons, and there was a multitude of 
 the faithful who were driven into a frightful wilderness for the defence of 
 the Catholic truth. Many of them were cruelly molested by the Moorish 
 leaders, and with sharp-pointed spears and stones were forced to hasten 
 their march; others, with their feet tied, were dragged like corpses through 
 rough places and were mangled in all their limbs. At the end they were 
 tortured in different manners and won the honours of martyrdom. The 
 principal ones among them were the bishops Felix and Cyprian.
\switchcolumn*
\selectlanguage{latin}
Ravénnæ, via Laurentína, natális sancti Edístii Mártyris.
\switchcolumn
\selectlanguage{english}
At Ravenna, on the Via Laurentina, 
 the birthday of St. Edistus, martyr.
\switchcolumn*
\selectlanguage{latin}
In Lycia sanctæ Domnínæ 
 Mártyris, sub Diocletiáno Imperatóre.
\switchcolumn
\selectlanguage{english}
In Lycia, under Emperor Diocletian, 
 St. Domnina, martyr.
\switchcolumn*
\selectlanguage{latin}
Celénæ, in Pannónia, 
 sancti Maximiliáni, Epíscopi Laureacénsis.
\switchcolumn
\selectlanguage{english}
At Cilli in Austria, St. Maximilian, 
 bishop of Lorsch.
\switchcolumn*
\selectlanguage{latin}
Eboráci, in Anglia, 
 sancti Walfrídi, Epíscopi et Confessóris.
\switchcolumn
\selectlanguage{english}
At York in England, St. Wilfrid, 
 bishop and confessor.
\switchcolumn*
\selectlanguage{latin}
Medioláni sancti Monæ 
 Epíscopi, qui, cum de Epíscopo eligéndo agerétur, cælésti lúmine circumfúsus, 
 eo signo mirabíliter in Pontíficem illíus Ecclésiæ est assúmptus.
\switchcolumn
\selectlanguage{english}
At Milan, St. Monas, bishop. 
 He was chosen as head of that church because a miraculous light from heaven 
 surrounded him when they were deliberating on the choice of a bishop.
\switchcolumn*
\selectlanguage{latin}
Verónæ sancti Salvíni 
 Epíscopi.
\switchcolumn
\selectlanguage{english}
At Verona, St. Salvinus, bishop.
\switchcolumn*
\selectlanguage{latin}
In Syria sancti 
 Eustáchii, Presbyteri et Confessóris.
\switchcolumn
\selectlanguage{english}
In Syria, St. Eustace, priest and 
 confessor.
\switchcolumn*
\selectlanguage{latin}
Asculi, in Picéno, sancti Seraphíni Confessóris, ex Ordine Minórum Capuccinórum, vitæ 
 sanctimónia et humilitáte conspícui; quem Clemens Décimus tértius, Póntifex 
 Máximus, Sanctórum fastis adscrípsit.
\switchcolumn
\selectlanguage{english}
At Ascoli in Piceno, St. Seraphinus, 
 confessor, of the Order of Friars Minor Capuchin, distinguished by his 
 humility and holiness of life. He was enrolled among the saints by the 
 Sovereign Pontiff Clement XIII.
\switchcolumn*
\selectlanguage{latin}
\end{paracol}


% ---- martyrology/mart10/mart1013.htm
\needspace{10\baselineskip}
\begin{paracol}{2}
\selectlanguage{latin}
\begin{center}{\color{gregoriocolor} Tértio Idus Octóbris. 
 Luna\dots\ }\end{center}
\switchcolumn
\selectlanguage{english}
\begin{center}{\color{gregoriocolor} The   Thirteenth Day of 
 October. The\dots\ Day of the Moon.}\end{center}
\end{paracol}

\noindent\begin{tabularx}{\linewidth}{*{19}{>{\centering\arraybackslash}X}}
 \textcolor{gregoriocolor}{a} & \textcolor{gregoriocolor}{b} & \textcolor{gregoriocolor}{c} & \textcolor{gregoriocolor}{d} & \textcolor{gregoriocolor}{e} & \textcolor{gregoriocolor}{f} & \textcolor{gregoriocolor}{g} & \textcolor{gregoriocolor}{h} & \textcolor{gregoriocolor}{i} & \textcolor{gregoriocolor}{k} & \textcolor{gregoriocolor}{l} & \textcolor{gregoriocolor}{m} & \textcolor{gregoriocolor}{n} & \textcolor{gregoriocolor}{p} & \textcolor{gregoriocolor}{q} & \textcolor{gregoriocolor}{r} & \textcolor{gregoriocolor}{s} & \textcolor{gregoriocolor}{t} & \textcolor{gregoriocolor}{u} \\
 21 & 22 & 23 & 24 & 25 & 26 & 27 & 28 & 29 & 1 & 2 & 3 & 4 & 5 & 6 & 7 & 8 & 9 & 10 \\
\end{tabularx}
\vspace{0.5\baselineskip}
\noindent\begin{tabularx}{\linewidth}{*{12}{>{\centering\arraybackslash}X}}
 \textcolor{gregoriocolor}{A} & \textcolor{gregoriocolor}{B} & \textcolor{gregoriocolor}{C} & \textcolor{gregoriocolor}{D} & \textcolor{gregoriocolor}{E} & F & \textcolor{gregoriocolor}{F} & \textcolor{gregoriocolor}{G} & \textcolor{gregoriocolor}{H} & \textcolor{gregoriocolor}{M} & \textcolor{gregoriocolor}{N} & \textcolor{gregoriocolor}{P} \\
 11 & 12 & 13 & 14 & 15 & 16 & 15 & 16 & 17 & 18 & 19 & 20 \\
\end{tabularx}

\begin{paracol}{2}
\selectlanguage{latin}
\lettrine[lines=2]{S}{ancti} Eduárdi, Regis 
 Anglórum et Confessóris, qui Nonis Januárii obdormívit in Dómino, sed hac 
 die, ob Translatiónem córporis ejus, potíssimum cólitur.
\switchcolumn
\selectlanguage{english}
\lettrine[lines=2]{S}{t.} Edward, king of England and 
 confessor, who died on the 5th day of January. He is specially 
 honoured on this day because of the translation of his body.
\switchcolumn*
\selectlanguage{latin}
Apud Tróadem, Asiæ 
 minóris urbem, natális sancti Carpi, qui fuit discípulus beáti Pauli 
 Apóstoli.
\switchcolumn
\selectlanguage{english}
At Troas in Asia Minor, the birthday 
 of St. Carpus, a disciple of the blessed apostle Paul.
\switchcolumn*
\selectlanguage{latin}
Córdubæ, in Hispánia, 
 item natális sanctórum Mártyrum Fausti, Januárii et Martiális; qui, primo 
 equúlei pœna cruciáti, deínde, supercíliis rasis, déntibus evúlsis, áuribus 
 quoque et náribus præcísis, ignis passióne martyrium consummárunt.
\switchcolumn
\selectlanguage{english}
At Cordova in Spain, the birthday of 
 the holy martyrs Faustus, Januarius, and Martial. They were first 
 tortured on the rack, their eyebrows were then shaven, their teeth torn out, 
 their ears and noses cut off, and the martyrdom was completed by fire.
\switchcolumn*
\selectlanguage{latin}
Thessalonícæ sancti 
 Floréntii Mártyris, qui, post vária torménta, igne combústus est.
\switchcolumn
\selectlanguage{english}
At Thessalonica, St. Florentius, a 
 martyr, who, after enduring various torments, was burned alive.
\switchcolumn*
\selectlanguage{latin}
Apud Stokeráviam, in 
 Austria, sancti Colmánni Mártyris.
\switchcolumn
\selectlanguage{english}
At Stockerau in Austria, St. Colman, 
 martyr.
\switchcolumn*
\selectlanguage{latin}
Antiochíæ sancti 
 Theóphili Epíscopi, qui, sextus post beátum Petrum Apóstolum, ejúsdem 
 Ecclésiæ Pontificátum ténuit.
\switchcolumn
\selectlanguage{english}
At Antioch, St. Theophilus, the 
 bishop who held the pontificate in that church, the sixth after the blessed 
 apostle Peter.
\switchcolumn*
\selectlanguage{latin}
Turónis, in Gállia, 
 sancti Venántii, Abbátis et Confessóris.
\switchcolumn
\selectlanguage{english}
At Tours in France, St. Venantius, 
 abbot and confessor.
\switchcolumn*
\selectlanguage{latin}
Apud Sublácum, in Látio, 
 sanctæ Chelidóniæ Vírginis.
\switchcolumn
\selectlanguage{english}
At Subiaco in Italy, St. Chelidonia, 
 virgin.
\switchcolumn*
\selectlanguage{latin}
\end{paracol}


% ---- martyrology/mart10/mart1014.htm
\needspace{10\baselineskip}
\begin{paracol}{2}
\selectlanguage{latin}
\begin{center}{\color{gregoriocolor} Prídie Idus Octóbris. 
 Luna\dots\ }\end{center}
\switchcolumn
\selectlanguage{english}
\begin{center}{\color{gregoriocolor} The   Fourteenth Day of 
 October. The\dots\ Day of the Moon.}\end{center}
\end{paracol}

\noindent\begin{tabularx}{\linewidth}{*{19}{>{\centering\arraybackslash}X}}
 \textcolor{gregoriocolor}{a} & \textcolor{gregoriocolor}{b} & \textcolor{gregoriocolor}{c} & \textcolor{gregoriocolor}{d} & \textcolor{gregoriocolor}{e} & \textcolor{gregoriocolor}{f} & \textcolor{gregoriocolor}{g} & \textcolor{gregoriocolor}{h} & \textcolor{gregoriocolor}{i} & \textcolor{gregoriocolor}{k} & \textcolor{gregoriocolor}{l} & \textcolor{gregoriocolor}{m} & \textcolor{gregoriocolor}{n} & \textcolor{gregoriocolor}{p} & \textcolor{gregoriocolor}{q} & \textcolor{gregoriocolor}{r} & \textcolor{gregoriocolor}{s} & \textcolor{gregoriocolor}{t} & \textcolor{gregoriocolor}{u} \\
 22 & 23 & 24 & 25 & 26 & 27 & 28 & 29 & 1 & 2 & 3 & 4 & 5 & 6 & 7 & 8 & 9 & 10 & 11 \\
\end{tabularx}
\vspace{0.5\baselineskip}
\noindent\begin{tabularx}{\linewidth}{*{12}{>{\centering\arraybackslash}X}}
 \textcolor{gregoriocolor}{A} & \textcolor{gregoriocolor}{B} & \textcolor{gregoriocolor}{C} & \textcolor{gregoriocolor}{D} & \textcolor{gregoriocolor}{E} & F & \textcolor{gregoriocolor}{F} & \textcolor{gregoriocolor}{G} & \textcolor{gregoriocolor}{H} & \textcolor{gregoriocolor}{M} & \textcolor{gregoriocolor}{N} & \textcolor{gregoriocolor}{P} \\
 12 & 13 & 14 & 15 & 16 & 17 & 16 & 17 & 18 & 19 & 20 & 21 \\
\end{tabularx}

\begin{paracol}{2}
\selectlanguage{latin}
\lettrine[lines=2]{R}{omæ,} via Aurélia, natális beáti Callísti Primi, Papæ et Mártyris; qui, Alexándri Imperatóris 
 jussu, diútius fame in cárcere cruciátus, et quotídie fústibus cæsus, 
 tandem, præcipitátus e fenéstra domus in qua custodiebátur, atque in púteum 
 demérsus, victóriæ triúmphum proméruit.
\switchcolumn
\selectlanguage{english}
\lettrine[lines=2]{A}{t} Rome, on the Aurelian Way, the 
 birthday of blessed Callistus I, pope and martyr. By order of Emperor 
 Alexander, he was kept in prison for a long time without food, and was daily 
 scourged with rods. He was finally hurled from a window of the house 
 in which he had been shut up, and was cast into a well, and thus merited the 
 triumph of victory.
\switchcolumn*
\selectlanguage{latin}
Arímini sancti 
 Gaudéntii, Epíscopi et Mártyris.
\switchcolumn
\selectlanguage{english}
At Rimini, St. Gaudentius, bishop 
 and martyr.
\switchcolumn*
\selectlanguage{latin}
Cæsaréæ, in Palæstína, 
 sanctórum Carpónii, Evarísti et Prisciáni, fratrum beátæ Fortunátæ, qui, 
 gládio juguláti, páriter martyrii corónam percepérunt.
\switchcolumn
\selectlanguage{english}
At Caesarea in Palestine, the Saints 
 Carponius, Evaristus, and Priscian, brothers of blessed Fortunata, who 
 obtained the crown of martyrdom together, their throats being cut with the 
 sword.
\switchcolumn*
\selectlanguage{latin}
Item sanctórum 
 Saturníni et Lupi.
\switchcolumn
\selectlanguage{english}
Also, the Saints Saturninus and 
 Lupus.
\switchcolumn*
\selectlanguage{latin}
Cæsaréæ, in Palæstína, 
 sanctæ Fortunátæ, Vírginis et Mártyris, ac prædicatórum Mártyrum Carpónii, 
 Evarísti et Prisciáni soróris; quæ, in persecutióne Diocletiáni, post 
 equúleum, ignes, béstias et ália torménta superáta, spíritum Deo réddidit. 
 Ipsíus corpus póstea Neápolim, in Campánia, delátum fuit.
\switchcolumn
\selectlanguage{english}
At Caesarea in Palestine, St. 
 Fortunata, virgin and martyr, the sister of the martyrs Carponius, Evaristus, 
 and Priscian. After having been subjected to the rack, to fire, to the 
 teeth of beasts, and other torments during the persecution of Diocletian, she 
 gave up her soul to God. Her body was afterwards conveyed to Naples in 
 Campania.
\switchcolumn*
\selectlanguage{latin}
Tudérti, in Umbria, 
 sancti Fortunáti Epíscopi, qui (ut beátus Gregórius Papa refert) in immúndis 
 spirítibus effugándis imménsæ grátia virtútis emícuit.
\switchcolumn
\selectlanguage{english}
At Todi in Umbria, St. Fortunatus, 
 bishop, who, as is mentioned by blessed Gregory, was endowed with an 
 extraordinary gift for casting out unclean spirits.
\switchcolumn*
\selectlanguage{latin}
Herbípoli, in Germánia, sancti Burchárdi, qui fuit primus illíus civitátis Epíscopus.
\switchcolumn
\selectlanguage{english}
At Wurzburg in Germany, St. Burchard, 
 first bishop of that city.
\switchcolumn*
\selectlanguage{latin}
Brugis Flandrórum 
 sancti Donatiáni, Rheménsis Epíscopi.
\switchcolumn
\selectlanguage{english}
At Bruges in Belgium, St. Donatian, 
 bishop of Rheims.
\switchcolumn*
\selectlanguage{latin}
Tréviris sancti Rústici 
 Epíscopi.
\switchcolumn
\selectlanguage{english}
At Treves, St. Rusticus, bishop.
\switchcolumn*
\selectlanguage{latin}
Lugdúni, in Gállia, 
 sancti Justi, Epíscopi et Confessóris, miræ sanctitátis et prophétici 
 spíritus viri; qui, Episcopátu demísso, in erémum Ægypti, una cum Lectóre 
 suo Viatóre, secéssit, ibíque, cum áliquot annos próximam Angelis egísset 
 vitam, et dignus suórum labórum finis advenísset, corónam justítiæ 
 perceptúrus migrávit ad Dóminum. Ipsíus sanctum corpus, una cum 
 óssibus beáti Viatóris, qui ejúsdem Epíscopi fúerat miníster, Lugdúnum 
 póstea quarto Nonas Septémbris delátum fuit.
\switchcolumn
\selectlanguage{english}
At Lyons in France, St. Justus, 
 bishop and confessor, a man of extraordinary sanctity and endowed with the 
 spirit of prophecy. He resigned his bishopric and retired into a 
 desert in Egypt with his lector Viator. When he had for some years led 
 an almost angelic life, and the end of his meritorious labours had come, he 
 went to our Lord to receive the crown of justice. His holy body and 
 the relics of his lector, blessed Viator, were afterwards taken to Lyons on 
 the 2nd of September.
\switchcolumn*
\selectlanguage{latin}
Eódem die deposítio 
 beáti Domínici Loricáti.
\switchcolumn
\selectlanguage{english}
On the same day, the death of 
 blessed Dominic Loricatus.
\switchcolumn*
\selectlanguage{latin}
Arpíni, in Látio, 
 sancti Bernárdi Confessóris.
\switchcolumn
\selectlanguage{english}
At Arpino in Italy, St. Bernard, 
 confessor.
\switchcolumn*
\selectlanguage{latin}
\end{paracol}


% ---- martyrology/mart10/mart1015.htm
\needspace{10\baselineskip}
\begin{paracol}{2}
\selectlanguage{latin}
\begin{center}{\color{gregoriocolor} Idibus Octóbris. 
 Luna\dots\ }\end{center}
\switchcolumn
\selectlanguage{english}
\begin{center}{\color{gregoriocolor} The   Fifteenth Day of 
 October. The\dots\ Day of the Moon.}\end{center}
\end{paracol}

\noindent\begin{tabularx}{\linewidth}{*{19}{>{\centering\arraybackslash}X}}
 \textcolor{gregoriocolor}{a} & \textcolor{gregoriocolor}{b} & \textcolor{gregoriocolor}{c} & \textcolor{gregoriocolor}{d} & \textcolor{gregoriocolor}{e} & \textcolor{gregoriocolor}{f} & \textcolor{gregoriocolor}{g} & \textcolor{gregoriocolor}{h} & \textcolor{gregoriocolor}{i} & \textcolor{gregoriocolor}{k} & \textcolor{gregoriocolor}{l} & \textcolor{gregoriocolor}{m} & \textcolor{gregoriocolor}{n} & \textcolor{gregoriocolor}{p} & \textcolor{gregoriocolor}{q} & \textcolor{gregoriocolor}{r} & \textcolor{gregoriocolor}{s} & \textcolor{gregoriocolor}{t} & \textcolor{gregoriocolor}{u} \\
 23 & 24 & 25 & 26 & 27 & 28 & 29 & 1 & 2 & 3 & 4 & 5 & 6 & 7 & 8 & 9 & 10 & 11 & 12 \\
\end{tabularx}
\vspace{0.5\baselineskip}
\noindent\begin{tabularx}{\linewidth}{*{12}{>{\centering\arraybackslash}X}}
 \textcolor{gregoriocolor}{A} & \textcolor{gregoriocolor}{B} & \textcolor{gregoriocolor}{C} & \textcolor{gregoriocolor}{D} & \textcolor{gregoriocolor}{E} & F & \textcolor{gregoriocolor}{F} & \textcolor{gregoriocolor}{G} & \textcolor{gregoriocolor}{H} & \textcolor{gregoriocolor}{M} & \textcolor{gregoriocolor}{N} & \textcolor{gregoriocolor}{P} \\
 13 & 14 & 15 & 16 & 17 & 18 & 17 & 18 & 19 & 20 & 21 & 22 \\
\end{tabularx}

\begin{paracol}{2}
\selectlanguage{latin}
\lettrine[lines=2]{A}{lbæ,} in Hispánia, 
 sanctæ Terésiæ Vírginis, quæ Fratrum ac Sorórum Ordinis Carmelitárum 
 arctióris observántiæ mater éxstitit et magístra.
\switchcolumn
\selectlanguage{english}
\lettrine[lines=2]{A}{t} Avila in Spain, the virgin St. 
 Teresa, mother and mistress of the Brothers and Sisters of the Carmelite 
 Order of the Strict Observance.
\switchcolumn*
\selectlanguage{latin}
Cracóviæ, in Polónia, 
 natális sanctæ Hedwígis Víduæ, Polonórum Ducíssæ, quæ, páuperum obséquio 
 dédita, étiam miráculis cláruit; et a Cleménte Quarto, Pontífice Máximo, 
 Sanctórum número adscrípta est. Ipsíus autem festívitas sequénti die 
 celebrátur.
\switchcolumn
\selectlanguage{english}
At Cracow in Poland, St. Hedwig, 
 duchess of Poland, who devoted herself to the service of the poor, and was 
 renowned for miracles. She was inscribed among the saints by Pope 
 Clement IV. Her feast is celebrated on the following day.
\switchcolumn*
\selectlanguage{latin}
Romæ, via Aurélia, sancti Fortunáti Mártyris.
\switchcolumn
\selectlanguage{english}
At Rome, on the Aurelian Way, St. 
 Fortunatus, martyr.
\switchcolumn*
\selectlanguage{latin}
In Borússia sancti 
 Brunónis, Epíscopi Ruthenórum et Mártyris; qui, Evangélium in ea regióne 
 prædicans, ab ímpiis tentus est, ac, mánibus pedibúsque præcísis, cápite 
 truncátus.
\switchcolumn
\selectlanguage{english}
In Prussia, St. Bruno, bishop of the 
 Ruthenians and martyr. While preaching the Gospel in that region he 
 was arrested by impious men, his hands and feet were cut off, and he was 
 then beheaded.
\switchcolumn*
\selectlanguage{latin}
Apud Colóniam 
 Agrippínam natális sanctórum trecentórum Mártyrum, qui, in Maximiáni 
 persecutióne, cursum sui agónis complevérunt.
\switchcolumn
\selectlanguage{english}
At Cologne, the birthday of three 
 hundred holy martyrs, who met their trials in the persecution of Maximian.
\switchcolumn*
\selectlanguage{latin}
Carthágine sancti 
 Agiléi Mártyris, in cujus die natáli sanctus Augustínus tractátum de ipso ad 
 pópulum hábuit.
\switchcolumn
\selectlanguage{english}
At Carthage, St. Agileus, martyr, on 
 whose birthday St. Augustine delivered a discourse to the people concerning 
 him.
\switchcolumn*
\selectlanguage{latin}
Lugdúni, in Gállia, 
 sancti Antíochi Epíscopi, qui, strénue administráto Pontifícii cúlmine ad 
 quod assúmptus fúerat, regnum cæléste adéptus est.
\switchcolumn
\selectlanguage{english}
At Lyons in France, St. Antiochus, 
 bishop, who entered the heavenly kingdom after having courageously 
 fulfilled the duties of the high station to which he had been called.
\switchcolumn*
\selectlanguage{latin}
Tréviris sancti Sevéri, 
 Epíscopi et Confessóris.
\switchcolumn
\selectlanguage{english}
At Treves, St. Severus, bishop and 
 confessor.
\switchcolumn*
\selectlanguage{latin}
Argentoráti sanctæ 
 Auréliæ Vírginis.
\switchcolumn
\selectlanguage{english}
At Strasbourg, St. Aurelia, virgin.
\switchcolumn*
\selectlanguage{latin}
In Germánia sanctæ 
 Theclæ, Abbatíssæ et Vírginis, quæ, monastériis Kitzíngæ et Ochsenfúrti 
 præpósita, multis cumuláta méritis in cælum migrávit.
\switchcolumn
\selectlanguage{english}
In Germany, St. Thecla, abbess and 
 virgin. She governed the convents of Kitzingen and Ochsenfurt, and 
 departed to heaven filled with merits.
\switchcolumn*
\selectlanguage{latin}
\end{paracol}


% ---- martyrology/mart10/mart1016.htm
\needspace{10\baselineskip}
\begin{paracol}{2}
\selectlanguage{latin}
\begin{center}{\color{gregoriocolor} Décimo séptimo Kaléndas Novémbris. 
 Luna\dots\ }\end{center}
\switchcolumn
\selectlanguage{english}
\begin{center}{\color{gregoriocolor} The   Sixteenth Day of 
 October. The\dots\ Day of the Moon.}\end{center}
\end{paracol}

\noindent\begin{tabularx}{\linewidth}{*{19}{>{\centering\arraybackslash}X}}
 \textcolor{gregoriocolor}{a} & \textcolor{gregoriocolor}{b} & \textcolor{gregoriocolor}{c} & \textcolor{gregoriocolor}{d} & \textcolor{gregoriocolor}{e} & \textcolor{gregoriocolor}{f} & \textcolor{gregoriocolor}{g} & \textcolor{gregoriocolor}{h} & \textcolor{gregoriocolor}{i} & \textcolor{gregoriocolor}{k} & \textcolor{gregoriocolor}{l} & \textcolor{gregoriocolor}{m} & \textcolor{gregoriocolor}{n} & \textcolor{gregoriocolor}{p} & \textcolor{gregoriocolor}{q} & \textcolor{gregoriocolor}{r} & \textcolor{gregoriocolor}{s} & \textcolor{gregoriocolor}{t} & \textcolor{gregoriocolor}{u} \\
 24 & 25 & 26 & 27 & 28 & 29 & 1 & 2 & 3 & 4 & 5 & 6 & 7 & 8 & 9 & 10 & 11 & 12 & 13 \\
\end{tabularx}
\vspace{0.5\baselineskip}
\noindent\begin{tabularx}{\linewidth}{*{12}{>{\centering\arraybackslash}X}}
 \textcolor{gregoriocolor}{A} & \textcolor{gregoriocolor}{B} & \textcolor{gregoriocolor}{C} & \textcolor{gregoriocolor}{D} & \textcolor{gregoriocolor}{E} & F & \textcolor{gregoriocolor}{F} & \textcolor{gregoriocolor}{G} & \textcolor{gregoriocolor}{H} & \textcolor{gregoriocolor}{M} & \textcolor{gregoriocolor}{N} & \textcolor{gregoriocolor}{P} \\
 14 & 15 & 16 & 17 & 18 & 19 & 18 & 19 & 20 & 21 & 22 & 23 \\
\end{tabularx}

\begin{paracol}{2}
\selectlanguage{latin}
\lettrine[lines=2]{S}{anctæ} Hedwígis Víduæ, 
 Polonórum Ducíssæ, quæ prídie hujus diéi obdormívit in Dómino.
\switchcolumn
\selectlanguage{english}
\lettrine[lines=2]{S}{t.} Hedwig, widow, duchess of 
 Poland, who went to her rest in the Lord on the day previous.
\switchcolumn*
\selectlanguage{latin}
In monastério Dervénsi, 
 in Gállia, sancti Berchárii, Abbátis et Mártyris.
\switchcolumn
\selectlanguage{english}
In the monastery of Moutier-en-Der, 
 in France, St. Bercharius, abbot and martyr.
\switchcolumn*
\selectlanguage{latin}
In Africa sanctórum 
 Mártyrum ducentórum septuagínta, páriter coronatórum.
\switchcolumn
\selectlanguage{english}
In Africa, two hundred and seventy 
 holy martyrs, crowned together.
\switchcolumn*
\selectlanguage{latin}
Ibídem sanctórum 
 Martiniáni et Saturiáni, cum duóbus eórum frátribus; qui, témpore Wandálicæ 
 persecutiónis, sub Rege Ariáno Genseríco, cum servi essent cujúsdam Wándali, 
 et a sancta Máxima Vírgine, ipsórum consérva, ad Christi fidem convérsi 
 fuíssent, omnes ab hærético illórum dómino, pro constántia fídei cathólicæ, 
 primum nodósis fústibus cæsi sunt et usque ad ossa laniáti. Sed, cum 
 tália multo témpore pateréntur, et sequénti die nihilóminus redderéntur 
 semper incólumes, exsílio tandem relegántur; ubi, cum multos barbarórum ad 
 Christi convertíssent fidem, et a Románo Pontífice Presbyterum aliósque 
 minístros, qui eos baptizárent, obtinuíssent, novíssime, vinctis pédibus 
 post terga curréntium quadrigárum, inter spinósa loca silvárum jussi sunt 
 páriter interíre. Máxima vero, post multos superátos agónes divínitus 
 liberáta, in monastério, multárum Vírginum Mater, sancto fine quiévit.
\switchcolumn
\selectlanguage{english}
Likewise, the Saints Martinian and 
 Saturnian, with their two brothers. While the persecution of the 
 Vandals was raging in the reign of the Arian king Genseric, they were slaves 
 to a man of that race. They were converted to the faith of Christ by 
 Maxima, a slave like themselves, and they manifested their attachment to the 
 truth with such courage that they were beaten with rough clubs and lacerated 
 in all parts of their bodies to the very bones. Although this 
 barbarous treatment was continued for a considerable period, their wounds 
 were each time healed overnight. They were at length sent into exile 
 where they converted many barbarians to the faith, and obtained from the 
 Roman Pontiff a priest and other ministers to baptize them. Finally 
 there were condemned to die by having their feet tied behind running 
 chariots and being dragged through thorns. Maxima, after enduring many 
 tribulations, was miraculously delivered and became the superior of a large 
 monastery of virgins, where she ended her days in peace.
\switchcolumn*
\selectlanguage{latin}
Item sanctórum 
 Saturníni, Nérei et aliórum trecentórum sexagínta quinque Mártyrum.
\switchcolumn
\selectlanguage{english}
Also, the Saints Saturninus, Nereus, 
 and three hundred and sixty-five other martyrs.
\switchcolumn*
\selectlanguage{latin}
Colóniæ Agrippínæ 
 sancti Elíphii Mártyris, sub Juliáno Apóstata.
\switchcolumn
\selectlanguage{english}
At Cologne, under Julian the 
 Apostate, the martyr St. Eliphius.
\switchcolumn*
\selectlanguage{latin}
In território Bituricénsi sancti Ambrósii, Epíscopi Caturcénsis.
\switchcolumn
\selectlanguage{english}
Near Bourges, St. Ambrose, bishop of 
 Cahors.
\switchcolumn*
\selectlanguage{latin}
Mogúntiæ sancti Lulli, 
 Epíscopi et Confessóris.
\switchcolumn
\selectlanguage{english}
At Mainz, St. Lullus, bishop and 
 confessor.
\switchcolumn*
\selectlanguage{latin}
Tréviris sancti 
 Florentíni Epíscopi.
\switchcolumn
\selectlanguage{english}
At Treves, St. Florentinus, bishop.
\switchcolumn*
\selectlanguage{latin}
Apud Arbónam, in 
 Germánia, sancti Galli Abbátis, qui fuit discípulus beáti Columbáni.
\switchcolumn
\selectlanguage{english}
At Arbon in Germany, St. Gall, 
 abbot, a disciple of blessed Columban.
\switchcolumn*
\selectlanguage{latin}
Muri, in Lucánia, 
 sancti Gerárdi Majélla, Confessóris, Láici proféssi Congregatiónis a 
 sanctíssimo Redemptóre nuncupátæ, quem, miráculis clarum, Pius Décimus, 
 Póntifex Máximus, Sanctórum albo accénsuit.
\switchcolumn
\selectlanguage{english}
At Muro in Italy, St. Gerard Majella, 
 confessor and professed lay brother of the Congregation of the Most Holy 
 Redeemer. Renowned for miracles, he was added to the list of the 
 saints by Pope Pius X.
\switchcolumn*
\selectlanguage{latin}
\end{paracol}


% ---- martyrology/mart10/mart1017.htm
\needspace{10\baselineskip}
\begin{paracol}{2}
\selectlanguage{latin}
\begin{center}{\color{gregoriocolor} Sextodécimo Kaléndas Novémbris. 
 Luna\dots\ }\end{center}
\switchcolumn
\selectlanguage{english}
\begin{center}{\color{gregoriocolor} The   Seventeenth Day of 
 October. The\dots\ Day of the Moon.}\end{center}
\end{paracol}

\noindent\begin{tabularx}{\linewidth}{*{19}{>{\centering\arraybackslash}X}}
 \textcolor{gregoriocolor}{a} & \textcolor{gregoriocolor}{b} & \textcolor{gregoriocolor}{c} & \textcolor{gregoriocolor}{d} & \textcolor{gregoriocolor}{e} & \textcolor{gregoriocolor}{f} & \textcolor{gregoriocolor}{g} & \textcolor{gregoriocolor}{h} & \textcolor{gregoriocolor}{i} & \textcolor{gregoriocolor}{k} & \textcolor{gregoriocolor}{l} & \textcolor{gregoriocolor}{m} & \textcolor{gregoriocolor}{n} & \textcolor{gregoriocolor}{p} & \textcolor{gregoriocolor}{q} & \textcolor{gregoriocolor}{r} & \textcolor{gregoriocolor}{s} & \textcolor{gregoriocolor}{t} & \textcolor{gregoriocolor}{u} \\
 25 & 26 & 27 & 28 & 29 & 1 & 2 & 3 & 4 & 5 & 6 & 7 & 8 & 9 & 10 & 11 & 12 & 13 & 14 \\
\end{tabularx}
\vspace{0.5\baselineskip}
\noindent\begin{tabularx}{\linewidth}{*{12}{>{\centering\arraybackslash}X}}
 \textcolor{gregoriocolor}{A} & \textcolor{gregoriocolor}{B} & \textcolor{gregoriocolor}{C} & \textcolor{gregoriocolor}{D} & \textcolor{gregoriocolor}{E} & F & \textcolor{gregoriocolor}{F} & \textcolor{gregoriocolor}{G} & \textcolor{gregoriocolor}{H} & \textcolor{gregoriocolor}{M} & \textcolor{gregoriocolor}{N} & \textcolor{gregoriocolor}{P} \\
 15 & 16 & 17 & 18 & 19 & 20 & 19 & 20 & 21 & 22 & 23 & 24 \\
\end{tabularx}

\begin{paracol}{2}
\selectlanguage{latin}
\lettrine[lines=2]{P}{arédii,} in diœcési 
 Augustodunénsi, sanctæ Margarítæ Maríæ Alacóque, quæ, Ordinem Visitatiónis 
 beátæ Maríæ Vírginis proféssa, exímiis in devotióne erga sacratíssimum Cor 
 Jesu propagánda et público ejúsdem cultu provehéndo méritis excélluit; 
 atque in sanctárum Vírginum album a Benedícto Papa Décimo quinto reláta fuit.
\switchcolumn
\selectlanguage{english}
\lettrine[lines=2]{A}{t} Paray, in the diocese of Autun, 
 St. Margaret Mary Alacoque. She made her profession in the Order of 
 the Visitation of Blessed Mary the Virgin, and she excelled with great merit 
 in spreading devotion to the Most Sacred Heart of Jesus and in furthering 
 its public veneration. Pope Benedict XV added her name to the list of 
 holy virgins.
\switchcolumn*
\selectlanguage{latin}
Antiochíæ natális 
 sancti Herónis, qui fuit discípulus beáti Ignátii; atque, post eum factus 
 Epíscopus, viam magístri pius imitátor est secútus, et pro commendáto sibi 
 grege amátor Christi occúbuit.
\switchcolumn
\selectlanguage{english}
At Antioch, the birthday of St. 
 Heron, a disciple of blessed Ignatius. Being made bishop after him, he 
 religiously followed his master's footsteps, and, as a true lover of Christ, 
 died for the flock entrusted to his keeping.
\switchcolumn*
\selectlanguage{latin}
Eódem die pássio 
 sanctórum Victóris, Alexándri et Mariáni.
\switchcolumn
\selectlanguage{english}
The same day, the martyrdom of the 
 Saints Victor, Alexander, and Marian.
\switchcolumn*
\selectlanguage{latin}
In Pérside sanctæ 
 Maméltæ Mártyris, quæ, a cultu idolórum ad fidem Angélico mónitu convérsa, a 
 Gentílibus lapidáta est et in profúndum lacum demérsa.
\switchcolumn
\selectlanguage{english}
In Persia, St. Mamelta, martyr. 
 He was converted from idolatry to the faith by the warning of an angel, and 
 was later stoned by heathens and cast into a deep lake.
\switchcolumn*
\selectlanguage{latin}
Aráusicæ, in Gálliis, 
 sancti Floréntii Epíscopi, qui, multis clarus virtútibus, quiévit in pace.
\switchcolumn
\selectlanguage{english}
At Orange in France, St. Florentinus, 
 bishop, who died leaving a reputation for many virtues.
\switchcolumn*
\selectlanguage{latin}
\end{paracol}


% ---- martyrology/mart10/mart1018.htm
\needspace{10\baselineskip}
\begin{paracol}{2}
\selectlanguage{latin}
\begin{center}{\color{gregoriocolor} Quintodécimo Kaléndas Novémbris. 
 Luna\dots\ }\end{center}
\switchcolumn
\selectlanguage{english}
\begin{center}{\color{gregoriocolor} The   Eighteenth Day of 
 October. The\dots\ Day of the Moon.}\end{center}
\end{paracol}

\noindent\begin{tabularx}{\linewidth}{*{19}{>{\centering\arraybackslash}X}}
 \textcolor{gregoriocolor}{a} & \textcolor{gregoriocolor}{b} & \textcolor{gregoriocolor}{c} & \textcolor{gregoriocolor}{d} & \textcolor{gregoriocolor}{e} & \textcolor{gregoriocolor}{f} & \textcolor{gregoriocolor}{g} & \textcolor{gregoriocolor}{h} & \textcolor{gregoriocolor}{i} & \textcolor{gregoriocolor}{k} & \textcolor{gregoriocolor}{l} & \textcolor{gregoriocolor}{m} & \textcolor{gregoriocolor}{n} & \textcolor{gregoriocolor}{p} & \textcolor{gregoriocolor}{q} & \textcolor{gregoriocolor}{r} & \textcolor{gregoriocolor}{s} & \textcolor{gregoriocolor}{t} & \textcolor{gregoriocolor}{u} \\
 26 & 27 & 28 & 29 & 1 & 2 & 3 & 4 & 5 & 6 & 7 & 8 & 9 & 10 & 11 & 12 & 13 & 14 & 15 \\
\end{tabularx}
\vspace{0.5\baselineskip}
\noindent\begin{tabularx}{\linewidth}{*{12}{>{\centering\arraybackslash}X}}
 \textcolor{gregoriocolor}{A} & \textcolor{gregoriocolor}{B} & \textcolor{gregoriocolor}{C} & \textcolor{gregoriocolor}{D} & \textcolor{gregoriocolor}{E} & F & \textcolor{gregoriocolor}{F} & \textcolor{gregoriocolor}{G} & \textcolor{gregoriocolor}{H} & \textcolor{gregoriocolor}{M} & \textcolor{gregoriocolor}{N} & \textcolor{gregoriocolor}{P} \\
 16 & 17 & 18 & 19 & 20 & 21 & 20 & 21 & 22 & 23 & 24 & 25 \\
\end{tabularx}

\begin{paracol}{2}
\selectlanguage{latin}
\lettrine[lines=2]{I}{n} Bithynia natális 
 beáti Lucæ Evangelístæ, qui, multa passus pro Christi nómine, obiit Spíritu 
 Sancto plenus. Ipsíus autem ossa póstea Constantinópolim transláta 
 sunt, et inde Patávium deláta.
\switchcolumn
\selectlanguage{english}
\lettrine[lines=2]{I}{n} Bithynia, the birthday of St. 
 Luke the Evangelist. He died, filled with the Holy Ghost, after having 
 suffered much for the Name of Christ. His relics were translated to 
 Constantinople, and thence taken to Pavia.
\switchcolumn*
\selectlanguage{latin}
Romæ item natális 
 sancti Pauli a Cruce, Presbyteri et Confessóris; qui Congregatiónis a Cruce 
 et Passióne Dómini nostri Jesu Christi nuncupátæ Institútor fuit. 
 Ipsum vero, mira innocéntia ac pæniténtia conspícuum et singulári in 
 Christum crucifíxum caritáte incénsum, Pius Papa Nonus fastis Sanctórum 
 adjúnxit, et ejúsdem festivitátem quarto Kaléndas Maji recoléndam indíxit.
\switchcolumn
\selectlanguage{english}
At Rome, the birthday of St. Paul of 
 the Cross, priest, confessor, and founder of the Congregation of the Cross 
 and Passion of our Lord Jesus Christ. Known for his remarkable 
 innocency of life and his penitential spirit, and aflame with love for 
 Christ crucified, he was canonized by Pope Pius IX, and the 28th of April 
 was assigned as his feast day.
\switchcolumn*
\selectlanguage{latin}
Arénis, in Hispánia, 
 natális páriter sancti Petri de Alcántara, Sacerdótis ex Ordine Minórum et 
 Confessóris; quem, propter admirábilem pæniténtiam múltaque mirácula, 
 Clemens Nonus, Póntifex Máximus, Sanctórum número adscrípsit. Ejus 
 autem festum sequénti die celebrátur.
\switchcolumn
\selectlanguage{english}
At Arenas in Spain, the birthday of 
 St. Peter of Alcantara, confessor and priest of the Order of Friars Minor. 
 He was canonized by Pope Clement IX because of his admirable penance and 
 many miracles, and his feast is observed on the day following.
\switchcolumn*
\selectlanguage{latin}
Antiochíæ sancti 
 Asclepíadis Epíscopi, qui fuit unus ex præcláro illórum Mártyrum número, qui 
 glorióse sub Macríno passi sunt.
\switchcolumn
\selectlanguage{english}
At Antioch, the bishop St. 
 Asclepiades, who was one of the celebrated band of martyrs who suffered so 
 gloriously under Macrinus.
\switchcolumn*
\selectlanguage{latin}
Neocæsaréæ, in Ponto, 
 sancti Athenodóri Epíscopi, qui fuit frater sancti Gregórii Thaumatúrgi; et, 
 doctrína clarus, in persecutióne Aureliáni, martyrium consummávit.
\switchcolumn
\selectlanguage{english}
At Neocaesarea in Pontus, the holy 
 and learned Bishop Athenodorus, brother of St. Gregory Thaumaturgus, who 
 underwent martyrdom in the persecution of Aurelian.
\switchcolumn*
\selectlanguage{latin}
Sinomovíci, in 
 território Bellovacénsi, sancti Justi Mártyris, qui adhuc puer, in 
 persecutióne Diocletiáni Imperatóris, sub Rictiováro Præside, cápite 
 amputátus est.
\switchcolumn
\selectlanguage{english}
At Louvres, in the diocese of 
 Beauvais, St. Justus, martyr, who, being but a boy, was put to death in the 
 persecution of Diocletian, under the governor Rictiovarus.
\switchcolumn*
\selectlanguage{latin}
Romæ sanctæ Tryphóniæ, 
 quæ Décii Cæsaris quondam uxor ac sanctæ Vírginis et Mártyris Cyrillæ mater 
 éxstitit; cujus corpus in crypta, juxta sanctum Hippólytum, sepúltum est.
\switchcolumn
\selectlanguage{english}
At Rome, St. Tryphonia, at one time 
 the wife of Caesar Decius, the mother of St. Cyrilla, virgin and martyr. 
 She was buried in a crypt, near that of St. Hippolytus.
\switchcolumn*
\selectlanguage{latin}
Apud Auriesville, in 
 statu Neo-Eboracénsi, sanctórum Mártyrum e Societáte Jesu, Isaáci Jogues, 
 Sacerdotis, et Joánnis de La Lande, Coadjutóris temporális, qui hac et 
 sequénti die ab Iroquénsibus dire necáti sunt, eódem loco ubi, paucis ante 
 annis, Renátus Goupil, et ipse Coadjútor temporális, martyrii palmam 
 consecútus fúerat.
\switchcolumn
\selectlanguage{english}
At Auriesville, in the state of New 
 York, the birthday of the holy martyrs Isaac Jogues, priest of the Society 
 of Jesus, and John de la Lande, a temporary helper to the same Society, who 
 came from France to teach the faith. On this and the following day 
 they were cruelly tortured and killed by the Iroquois in the same place 
 where, a few years before, one of the companions, René Goupil, also a 
 temporary assistant, had received the palm of martyrdom.
\switchcolumn*
\selectlanguage{latin}
In fínibus Edessénæ 
 regiónis, in Mesopotámia, commemorátio sancti Juliáni Eremítæ, cognoménto Sabæ, de quo 
 ágitur étiam sextodécimo Kaléndas Februárii.
\switchcolumn
\selectlanguage{english}
In Mesopotamia, in the neighbourhood 
 of Edessa, the commemoration of St. Julian the Hermit, surnamed Sabas, who 
 is mentioned also on the 17th of January.
\switchcolumn*
\selectlanguage{latin}
\end{paracol}


% ---- martyrology/mart10/mart1019.htm
\needspace{10\baselineskip}
\begin{paracol}{2}
\selectlanguage{latin}
\begin{center}{\color{gregoriocolor} Quartodécimo Kaléndas Novémbris. 
 Luna\dots\ }\end{center}
\switchcolumn
\selectlanguage{english}
\begin{center}{\color{gregoriocolor} The   Nineteenth Day of 
 October. The\dots\ Day of the Moon.}\end{center}
\end{paracol}

\noindent\begin{tabularx}{\linewidth}{*{19}{>{\centering\arraybackslash}X}}
 \textcolor{gregoriocolor}{a} & \textcolor{gregoriocolor}{b} & \textcolor{gregoriocolor}{c} & \textcolor{gregoriocolor}{d} & \textcolor{gregoriocolor}{e} & \textcolor{gregoriocolor}{f} & \textcolor{gregoriocolor}{g} & \textcolor{gregoriocolor}{h} & \textcolor{gregoriocolor}{i} & \textcolor{gregoriocolor}{k} & \textcolor{gregoriocolor}{l} & \textcolor{gregoriocolor}{m} & \textcolor{gregoriocolor}{n} & \textcolor{gregoriocolor}{p} & \textcolor{gregoriocolor}{q} & \textcolor{gregoriocolor}{r} & \textcolor{gregoriocolor}{s} & \textcolor{gregoriocolor}{t} & \textcolor{gregoriocolor}{u} \\
 27 & 28 & 29 & 1 & 2 & 3 & 4 & 5 & 6 & 7 & 8 & 9 & 10 & 11 & 12 & 13 & 14 & 15 & 16 \\
\end{tabularx}
\vspace{0.5\baselineskip}
\noindent\begin{tabularx}{\linewidth}{*{12}{>{\centering\arraybackslash}X}}
 \textcolor{gregoriocolor}{A} & \textcolor{gregoriocolor}{B} & \textcolor{gregoriocolor}{C} & \textcolor{gregoriocolor}{D} & \textcolor{gregoriocolor}{E} & F & \textcolor{gregoriocolor}{F} & \textcolor{gregoriocolor}{G} & \textcolor{gregoriocolor}{H} & \textcolor{gregoriocolor}{M} & \textcolor{gregoriocolor}{N} & \textcolor{gregoriocolor}{P} \\
 17 & 18 & 19 & 20 & 21 & 22 & 21 & 22 & 23 & 24 & 25 & 26 \\
\end{tabularx}

\begin{paracol}{2}
\selectlanguage{latin}
\lettrine[lines=2]{S}{ancti} Petri de 
 Alcántara, Sacerdótis ex Ordine Minórum et Confessóris, qui migrávit in 
 cælum prídie hujus diéi.
\switchcolumn
\selectlanguage{english}
\lettrine[lines=2]{S}{t.} Peter of Alcantara, priest of 
 the Order of Friars Minor and confessor, whose birthday was mentioned in the 
 day previous to this.
\switchcolumn*
\selectlanguage{latin}
Romæ natális sanctórum 
 Mártyrum Ptolomǽi et Lúcii, sub Marco Antoníno. Horum prior (ut 
 scribit Justínus Martyr), cum impúdicam mulíerem ad Christi convertísset 
 fidem, et castitátem cólere docuísset, ídeo, ab impúro viro apud Præféctum 
 Urbícium accusátus, multo témpore squalóre cárceris macerátus est, et ad 
 últimum, cum de Christi magistério pública confessióne testarétur, jussus 
 est duci ad mortem; Lúcius quoque, cum Urbícii senténtiam improbáret et se 
 Christiánum líbere faterétur, símilem senténtiam excépit; quibus et álius 
 tértius adjúnctus est, qui étiam eódem supplício damnátus fuit.
\switchcolumn
\selectlanguage{english}
At Rome, the birthday of the holy 
 martyrs Ptolemy and Lucius, in the time of Marcus Antoninus. The 
 former, as we learn from the martyr Justin, converted a certain immodest 
 woman to the faith of Christ and induced her to practice chastity. He 
 was accused by an evil man before the prefect Urbicius and made to undergo a 
 long imprisonment in a foul dungeon. At length, because he declared by 
 a public confession that Christ was his master, he was led to execution. 
 Lucius protested against the sentence of Urbicius, and freely proclaimed 
 himself to be a Christian, whereby he received the same sentence. To 
 them was added still a third martyr, who was condemned to suffer a like 
 punishment.
\switchcolumn*
\selectlanguage{latin}
Antiochíæ sanctórum 
 Mártyrum Beroníci, Pelágiæ Vírginis, et aliórum quadragínta novem.
\switchcolumn
\selectlanguage{english}
At Antioch, the holy martyrs 
 Beronicus, the virgin Pelagia, and forty-nine others.
\switchcolumn*
\selectlanguage{latin}
In Ægypto sancti Vari 
 mílitis, qui, sub Maximíno Imperatóre, dum sanctos septem Mónachos in cárcere deténtos visitáret atque refíceret, vóluit, uno ex ipsis defúncto, 
 in ejus locum subrogári; atque ita cum illis, sævíssima passus, martyrii 
 palmam adéptus est.
\switchcolumn
\selectlanguage{english}
In Egypt, St. Varus, a soldier, who, 
 under Emperor Maximian, visited and comforted seven holy monks who were kept 
 in prison. When one of them died he wished to be accepted in his 
 place, and after suffering most cruel torments with them he obtained the 
 palm of martyrdom.
\switchcolumn*
\selectlanguage{latin}
Ebróicis, in Gállia, 
 sancti Aquilíni, Epíscopi et Confessóris.
\switchcolumn
\selectlanguage{english}
At Evreux in France, St. Aquilinus, 
 bishop and confessor.
\switchcolumn*
\selectlanguage{latin}
In território 
 Aurelianénsi deposítio sancti Veráni Epíscopi.
\switchcolumn
\selectlanguage{english}
In the diocese of Orleans, the death 
 of St. Veranus, bishop.
\switchcolumn*
\selectlanguage{latin}
Apud Salérnum sancti 
 Eustérii Epíscopi.
\switchcolumn
\selectlanguage{english}
At Salerno, St. Eusterius, bishop.
\switchcolumn*
\selectlanguage{latin}
In monastério Silvæ 
 Necténsis, in Hibérnia, sancti Ethbíni Abbátis.
\switchcolumn
\selectlanguage{english}
In Ireland, in the monastery of the 
 Forest of Kildare, St. Ethbin, abbot.
\switchcolumn*
\selectlanguage{latin}
Oxónii, in Anglia, 
 sanctæ Fredeswíndæ Vírginis.
\switchcolumn
\selectlanguage{english}
At Oxford in England, St. Frideswide, 
 virgin.
\switchcolumn*
\selectlanguage{latin}
\end{paracol}


% ---- martyrology/mart10/mart1020.htm
\needspace{10\baselineskip}
\begin{paracol}{2}
\selectlanguage{latin}
\begin{center}{\color{gregoriocolor} Tertiodécimo Kaléndas Novémbris. 
 Luna\dots\ }\end{center}
\switchcolumn
\selectlanguage{english}
\begin{center}{\color{gregoriocolor} The   Twentieth Day of 
 October. The\dots\ Day of the Moon.}\end{center}
\end{paracol}

\noindent\begin{tabularx}{\linewidth}{*{19}{>{\centering\arraybackslash}X}}
 \textcolor{gregoriocolor}{a} & \textcolor{gregoriocolor}{b} & \textcolor{gregoriocolor}{c} & \textcolor{gregoriocolor}{d} & \textcolor{gregoriocolor}{e} & \textcolor{gregoriocolor}{f} & \textcolor{gregoriocolor}{g} & \textcolor{gregoriocolor}{h} & \textcolor{gregoriocolor}{i} & \textcolor{gregoriocolor}{k} & \textcolor{gregoriocolor}{l} & \textcolor{gregoriocolor}{m} & \textcolor{gregoriocolor}{n} & \textcolor{gregoriocolor}{p} & \textcolor{gregoriocolor}{q} & \textcolor{gregoriocolor}{r} & \textcolor{gregoriocolor}{s} & \textcolor{gregoriocolor}{t} & \textcolor{gregoriocolor}{u} \\
 28 & 29 & 1 & 2 & 3 & 4 & 5 & 6 & 7 & 8 & 9 & 10 & 11 & 12 & 13 & 14 & 15 & 16 & 17 \\
\end{tabularx}
\vspace{0.5\baselineskip}
\noindent\begin{tabularx}{\linewidth}{*{12}{>{\centering\arraybackslash}X}}
 \textcolor{gregoriocolor}{A} & \textcolor{gregoriocolor}{B} & \textcolor{gregoriocolor}{C} & \textcolor{gregoriocolor}{D} & \textcolor{gregoriocolor}{E} & F & \textcolor{gregoriocolor}{F} & \textcolor{gregoriocolor}{G} & \textcolor{gregoriocolor}{H} & \textcolor{gregoriocolor}{M} & \textcolor{gregoriocolor}{N} & \textcolor{gregoriocolor}{P} \\
 18 & 19 & 20 & 21 & 22 & 23 & 22 & 23 & 24 & 25 & 26 & 27 \\
\end{tabularx}

\begin{paracol}{2}
\selectlanguage{latin}
\lettrine[lines=2]{S}{ancti} Joánnis Cántii, 
 Presbyteri et Confessóris, qui nono Kaléndas Januárii obdormívit in Dómino.
\switchcolumn
\selectlanguage{english}
\lettrine[lines=2]{S}{t.} John Cantius, priest and 
 confessor, who fell asleep in the Lord on the 24th of December.
\switchcolumn*
\selectlanguage{latin}
In Aviénsi civitáte, 
 prope Aquilam, in Vestínis, natális beáti Máximi, Levítæ et Mártyris; qui, 
 patiéndi desidério, inquiréntibus se persecutóribus palam osténdit, et, post 
 responsiónis constántiam, equúleo suspénsus ac tortus, deínde fústibus cæsus, 
 ad últimum, e sublími loco præcipitátus, occúbuit.
\switchcolumn
\selectlanguage{english}
At Abia, near Aquila in Abruzzo, the 
 birthday of blessed Maximus, deacon and martyr. Because of his desire 
 to suffer he shewed himself to the persecutors of his own accord. 
 After answering with great constancy, he was racked and tortured, then 
 beaten with rods, and he finally died by being cast headlong from a high 
 place.
\switchcolumn*
\selectlanguage{latin}
Agénni, in Gállia, 
 sancti Caprásii Mártyris, qui, cum rábiem persecutiónis declínans latéret in 
 spelúnca, tandem, áudiens quáliter beáta Fides Virgo pro Christo agonizáret, 
 índeque animátus ad tolerántiam passiónum, orávit ad Dóminum, ut, si eum 
 glória martyrii dignum judicáret, ex lápide spelúncæ limpidíssima aqua 
 manáret; quod cum Dóminum præstitísset, secúrus ad áream certáminis 
 properávit, et palmam martyrii, sub Maximiáno Imperatóre, fórtiter dimicándo 
 proméruit.
\switchcolumn
\selectlanguage{english}
At Agen in France, St. Caprasius, 
 martyr. He was hiding in a cavern to avoid the violence of the 
 persecution when the report of the blessed virgin Faith's courage in 
 suffering for Christ roused him to endure the torments. He prayed to 
 God that, if he were deemed worthy of the glory of martyrdom, clear water 
 might flow from the rock of his cave. God granted his prayer, and he 
 went with confidence to the scene of the trial, where, after a valiant 
 struggle, he merited the palm of martyrdom under Maximian.
\switchcolumn*
\selectlanguage{latin}
Antiochíæ sancti 
 Artémii, Ducis Augustális, qui, sub Constantíno Magno præcláris milítiæ 
 honóribus functus, a Juliáno Apóstata, quem sævítiæ in Christiános argúerat, 
 fústibus cædi, aliísque torméntis afflígi, ac demum cápite truncári jubétur.
\switchcolumn
\selectlanguage{english}
At Antioch, St. Artemius, an 
 imperial officer who had filled high positions in the army under Constantine 
 the Great. Julian the Apostate, however, whom he rebuked for his 
 cruelty towards Christians, ordered him to be beaten with rods, subjected to 
 other torments, and finally beheaded.
\switchcolumn*
\selectlanguage{latin}
Constantinópoli sancti 
 Andréæ Creténsis Mónachi, qui ob cultum sacrárum Imáginum, sub Constantíno 
 Coprónymo, sæpius verberátus, tandem, amputáto áltero pede, réddidit 
 spíritum.
\switchcolumn
\selectlanguage{english}
At Constantinople, St. Andrew of 
 Crete, a monk who had often been scourged by Constantine Copronymus for his 
 veneration of holy images. After one of his feet had been cut off he 
 rendered up his soul.
\switchcolumn*
\selectlanguage{latin}
Colóniæ Agrippínæ 
 pássio sanctárum Vírginum Marthæ et Saulæ, cum áliis plúribus.
\switchcolumn
\selectlanguage{english}
At Cologne, the martyrdom of the 
 holy virgins Martha and Saula, with many others.
\switchcolumn*
\selectlanguage{latin}
Apud Nabántiam, in 
 Lusitánia, sanctæ Irénes, Vírginis et Mártyris; cujus corpus honorífice 
 sepúltum fuit in óppido Scálabi, quod ipsíus Sanctæ nómine insígnitum inde 
 permánsit.
\switchcolumn
\selectlanguage{english}
In Portugal, St. Irene, virgin and 
 martyr. Her body was honourably buried in the town of Scalabris. 
 Since that time the town has been named Santarem, which is derived from her 
 name.
\switchcolumn*
\selectlanguage{latin}
Alsóntiæ, in território 
 Rheménsi, sancti Sindúlphi, Presbyteri et Confessóris.
\switchcolumn
\selectlanguage{english}
At Aussonce, in the diocese of 
 Rheims, St. Sindulphus, priest and confessor.
\switchcolumn*
\selectlanguage{latin}
Apud Mindam, in 
 Germánia, Translátio sancti Feliciáni, Epíscopi Fulginátis et Mártyris; 
 cujus ibi depósita fuit sacrárum pars reliquiárum, quæ in Germániam advéctæ 
 sunt ex Umbriæ civitáte Fulgíneo, ubi ipse nono Kaléndas Februárii passus 
 quondam fúerat.
\switchcolumn
\selectlanguage{english}
At Minden in Germany, the 
 translation of St. Felician, bishop of Foligno and martyr. From his 
 holy relics a portion was placed in an urn and brought to Germany from the 
 city of Foligno in Umbria, where he had died on the 24th of January.
\switchcolumn*
\selectlanguage{latin}
Lutétiæ Parisiórum item 
 Translátio sanctórum Mártyrum Geórgii Diáconi, et Aurélii, ex urbe Hispániæ 
 Córduba, in qua olim, una cum áliis tribus Sóciis, ambo martyrium 
 compléverant sexto Kaléndas Augústi.
\switchcolumn
\selectlanguage{english}
At Paris, the translation of the 
 holy martyrs George, a deacon, and Aurelius from Cordova, a city of Spain, 
 where they had died with three companions on the 27th of July.
\switchcolumn*
\selectlanguage{latin}
\end{paracol}


% ---- martyrology/mart10/mart1021.htm
\needspace{10\baselineskip}
\begin{paracol}{2}
\selectlanguage{latin}
\begin{center}{\color{gregoriocolor} Duodécimo Kaléndas Novémbris. 
 Luna\dots\ }\end{center}
\switchcolumn
\selectlanguage{english}
\begin{center}{\color{gregoriocolor} The 
 Twenty-First Day of 
 October. The\dots\ Day of the Moon.}\end{center}
\end{paracol}

\noindent\begin{tabularx}{\linewidth}{*{19}{>{\centering\arraybackslash}X}}
 \textcolor{gregoriocolor}{a} & \textcolor{gregoriocolor}{b} & \textcolor{gregoriocolor}{c} & \textcolor{gregoriocolor}{d} & \textcolor{gregoriocolor}{e} & \textcolor{gregoriocolor}{f} & \textcolor{gregoriocolor}{g} & \textcolor{gregoriocolor}{h} & \textcolor{gregoriocolor}{i} & \textcolor{gregoriocolor}{k} & \textcolor{gregoriocolor}{l} & \textcolor{gregoriocolor}{m} & \textcolor{gregoriocolor}{n} & \textcolor{gregoriocolor}{p} & \textcolor{gregoriocolor}{q} & \textcolor{gregoriocolor}{r} & \textcolor{gregoriocolor}{s} & \textcolor{gregoriocolor}{t} & \textcolor{gregoriocolor}{u} \\
 29 & 1 & 2 & 3 & 4 & 5 & 6 & 7 & 8 & 9 & 10 & 11 & 12 & 13 & 14 & 15 & 16 & 17 & 18 \\
\end{tabularx}
\vspace{0.5\baselineskip}
\noindent\begin{tabularx}{\linewidth}{*{12}{>{\centering\arraybackslash}X}}
 \textcolor{gregoriocolor}{A} & \textcolor{gregoriocolor}{B} & \textcolor{gregoriocolor}{C} & \textcolor{gregoriocolor}{D} & \textcolor{gregoriocolor}{E} & F & \textcolor{gregoriocolor}{F} & \textcolor{gregoriocolor}{G} & \textcolor{gregoriocolor}{H} & \textcolor{gregoriocolor}{M} & \textcolor{gregoriocolor}{N} & \textcolor{gregoriocolor}{P} \\
 19 & 20 & 21 & 22 & 23 & 24 & 23 & 24 & 25 & 26 & 27 & 28 \\
\end{tabularx}

\begin{paracol}{2}
\selectlanguage{latin}
\lettrine[lines=2]{I}{n} Cypro natális sancti 
 Hilariónis Abbátis, cujus vitam, virtútibus atque miráculis plenam, sanctus 
 Hierónymus scripsit.
\switchcolumn
\selectlanguage{english}
\lettrine[lines=2]{I}{n} Cyprus, the birthday of the holy 
 abbot Hilarion. His life, full of virtues and miracles, was written by 
 St. Jerome.
\switchcolumn*
\selectlanguage{latin}
Apud Colóniam 
 Aggripínam item natális sanctárum Ursulæ et Sociárum ejus; quæ, pro 
 Christiána religióne et virginitátis constántia ab Hunnis interféctæ, 
 martyrio vitam consummárunt, earúmque plúrima córpora fuérunt Colóniæ 
 cóndita.
\switchcolumn
\selectlanguage{english}
At Cologne, the birthday of St. 
 Ursula and her companions, who gained the martyr's crown by being slain by 
 the Huns for the Christian religion and their constancy in keeping their 
 virginity. Many of their bodies are buried in Cologne.
\switchcolumn*
\selectlanguage{latin}
Apud Ostia Tiberína 
 sancti Astérii, Presbyteri et Mártyris; qui (ut in passióne beáti Callísti 
 Papæ légitur) sub Alexándro Imperatóre passus est.
\switchcolumn
\selectlanguage{english}
At Ostia, St. Asterius, priest and 
 martyr, who suffered under Emperor Alexander, as we read in the Acts of 
 blessed Pope Callistus.
\switchcolumn*
\selectlanguage{latin}
Nicomedíæ natális 
 sanctórum Dásii, Zótici, Caji et aliórum duódecim mílitum; qui, post divérsa 
 torménta, in mare demérsi sunt.
\switchcolumn
\selectlanguage{english}
At Nicomedia, the birthday of Saints 
 Dasius, Zoticus, Caius, and twelve other soldiers, who, after suffering 
 various torments, were drowned in the sea.
\switchcolumn*
\selectlanguage{latin}
Lugdúni, in Gállia, 
 sancti Viatóris, qui éxstitit miníster beáti Justi, Lugdunénsis Epíscopi.
\switchcolumn
\selectlanguage{english}
At Lyons in France, St. Viator, 
 deacon of blessed Justus, bishop of that city.
\switchcolumn*
\selectlanguage{latin}
Maróniæ, prope 
 Antiochíam, in Syria, sancti Malchi Mónachi.
\switchcolumn
\selectlanguage{english}
At Maronia, near Antioch in Syria, 
 St. Malchus, a monk.
\switchcolumn*
\selectlanguage{latin}
In castro Laudunénsi 
 sanctæ Cilíniæ, matris beáti Remígii, Epíscopi Rheménsis.
\switchcolumn
\selectlanguage{english}
At Laon, St. Cilinia, mother of 
 blessed Remigius, bishop of Rheims.
\switchcolumn*
\selectlanguage{latin}
\end{paracol}


% ---- martyrology/mart10/mart1022.htm
\needspace{10\baselineskip}
\begin{paracol}{2}
\selectlanguage{latin}
\begin{center}{\color{gregoriocolor} Undécimo Kaléndas Novémbris. 
 Luna\dots\ }\end{center}
\switchcolumn
\selectlanguage{english}
\begin{center}{\color{gregoriocolor} The 
 Twenty-Second Day of 
 October. The\dots\ Day of the Moon.}\end{center}
\end{paracol}

\noindent\begin{tabularx}{\linewidth}{*{19}{>{\centering\arraybackslash}X}}
 \textcolor{gregoriocolor}{a} & \textcolor{gregoriocolor}{b} & \textcolor{gregoriocolor}{c} & \textcolor{gregoriocolor}{d} & \textcolor{gregoriocolor}{e} & \textcolor{gregoriocolor}{f} & \textcolor{gregoriocolor}{g} & \textcolor{gregoriocolor}{h} & \textcolor{gregoriocolor}{i} & \textcolor{gregoriocolor}{k} & \textcolor{gregoriocolor}{l} & \textcolor{gregoriocolor}{m} & \textcolor{gregoriocolor}{n} & \textcolor{gregoriocolor}{p} & \textcolor{gregoriocolor}{q} & \textcolor{gregoriocolor}{r} & \textcolor{gregoriocolor}{s} & \textcolor{gregoriocolor}{t} & \textcolor{gregoriocolor}{u} \\
 1 & 2 & 3 & 4 & 5 & 6 & 7 & 8 & 9 & 10 & 11 & 12 & 13 & 14 & 15 & 16 & 17 & 18 & 19 \\
\end{tabularx}
\vspace{0.5\baselineskip}
\noindent\begin{tabularx}{\linewidth}{*{12}{>{\centering\arraybackslash}X}}
 \textcolor{gregoriocolor}{A} & \textcolor{gregoriocolor}{B} & \textcolor{gregoriocolor}{C} & \textcolor{gregoriocolor}{D} & \textcolor{gregoriocolor}{E} & F & \textcolor{gregoriocolor}{F} & \textcolor{gregoriocolor}{G} & \textcolor{gregoriocolor}{H} & \textcolor{gregoriocolor}{M} & \textcolor{gregoriocolor}{N} & \textcolor{gregoriocolor}{P} \\
 20 & 21 & 22 & 23 & 24 & 25 & 24 & 25 & 26 & 27 & 28 & 29 \\
\end{tabularx}

\begin{paracol}{2}
\selectlanguage{latin}
\lettrine[lines=2]{H}{ierosólymis} sanctæ 
 Maríæ Salóme, matris sanctórum Jacóbi et Joánnis Apostolórum, quæ in 
 Evangelio légitur circa Dómini sepultúram sollícita.
\switchcolumn
\selectlanguage{english}
\lettrine[lines=2]{A}{t} Jerusalem, St. Mary Salome, the 
 mother of the apostles James and John, who is referred to in the Gospel as 
 having cared for the burial of our Lord.
\switchcolumn*
\selectlanguage{latin}
Item Hierosólymis beáti 
 Marci Epíscopi, claríssimi et doctíssimi viri, qui primus ex Géntibus 
 Ecclésiam Hierosolymórum suscépit regéndam, ac, non multo post, sub Antoníno 
 Imperatóre, martyrii méruit palmam.
\switchcolumn
\selectlanguage{english}
At Jerusalem, blessed Bishop Mark, a 
 noble and learned man, who was the first Gentile to govern the Church of 
 Jerusalem. His brief episcopate was rewarded by the palm of martyrdom 
 under Emperor Antoninus.
\switchcolumn*
\selectlanguage{latin}
Hadrianópoli, in 
 Thrácia, natális sanctórum Mártyrum Philíppi Epíscopi, Sevéri Presbyteri, 
 Eusébii et Hermétis; qui, sub Juliáno Apóstata, post cárceres et flagélla, 
 incéndio cremáti sunt.
\switchcolumn
\selectlanguage{english}
At Adrianople in Thrace, the 
 birthday of the holy martyrs Philip, a bishop, Severus, a priest, Eusebius, 
 and Hermes. After being imprisoned and scourged, they were burned 
 alive in the time of Julian the Apostate.
\switchcolumn*
\selectlanguage{latin}
Item sanctórum Mártyrum 
 Alexándri Epíscopi, Heraclíi mílitis, et Sociórum.
\switchcolumn
\selectlanguage{english}
Also, the holy martyrs Alexander, a 
 bishop, Heraclius, a soldier, and their companions.
\switchcolumn*
\selectlanguage{latin}
Apud Firmum, in Picéno, natális sancti Philíppi, Epíscopi et Mártyris.
\switchcolumn
\selectlanguage{english}
At Fermo in Piceno, the birthday of 
 St. Philip, bishop and martyr.
\switchcolumn*
\selectlanguage{latin}
Apud Colóniam 
 Agrippínam sanctæ Córdulæ, quæ, cum esset una ex sodálibus sanctæ Ursulæ, 
 atque aliárum supplíciis et cædibus pertérrita se occultásset, postrídie, 
 ejus rei pænitens, se ultro patefécit Hunnis, et, novíssima ómnium, martyrii 
 corónam accépit.
\switchcolumn
\selectlanguage{english}
At Cologne, St. Cordula, who was one 
 of the companions of St. Ursula. Being terrified by the punishments 
 and slaughter of the others, she hid herself, but repenting her deed, on the 
 next day she declared herself to the Huns of her own accord, and thus was 
 the last of them all to receive the crown of martyrdom.
\switchcolumn*
\selectlanguage{latin}
Oscæ, in Hispánia, 
 sanctárum Vírginum Nunilónis et Alódiæ sorórum, quæ, a Saracénis ob fídei 
 confessiónem capitáli senténtia punítæ, martyrium consummárunt.
\switchcolumn
\selectlanguage{english}
At Huesca in Spain, the holy virgins 
 Nunilo and Alodia, sisters, who endured martyrdom by being condemned to 
 capital punishment by the Saracens for the confession of the faith.
\switchcolumn*
\selectlanguage{latin}
Hierápoli, in Phrygia, 
 sancti Abércii Epíscopi, qui sub Marco Antoníno Imperatóre cláruit.
\switchcolumn
\selectlanguage{english}
At Hieropolis in Phrygia, St. 
 Abercius, bishop, who flourished under Emperor Marcus Antoninus.
\switchcolumn*
\selectlanguage{latin}
Rotómagi sancti Melánii 
 Epíscopi, qui, a sancto Stéphano Papa ordinátus, illuc ad prædicándum 
 Evangélium missus est.
\switchcolumn
\selectlanguage{english}
At Rouen, St. Melanius, bishop, who 
 was ordained by Pope St. Stephen and sent there to preach the Gospel.
\switchcolumn*
\selectlanguage{latin}
In Túscia sancti Donáti 
 Scoti, Epíscopi Fæsuláni.
\switchcolumn
\selectlanguage{english}
In Tuscany, St. Donatus of Scotland, 
 bishop of Fiesole.
\switchcolumn*
\selectlanguage{latin}
Verónæ sancti Verecúndi, 
 Epíscopi et Confessóris.
\switchcolumn
\selectlanguage{english}
At Verona, St. Verecundius, bishop 
 and confessor.
\switchcolumn*
\selectlanguage{latin}
\end{paracol}


% ---- martyrology/mart10/mart1023.htm
\needspace{10\baselineskip}
\begin{paracol}{2}
\selectlanguage{latin}
\begin{center}{\color{gregoriocolor} Décimo Kaléndas Novémbris. 
 Luna\dots\ }\end{center}
\switchcolumn
\selectlanguage{english}
\begin{center}{\color{gregoriocolor} The 
 Twenty-Third Day of 
 October. The\dots\ Day of the Moon.}\end{center}
\end{paracol}

\noindent\begin{tabularx}{\linewidth}{*{19}{>{\centering\arraybackslash}X}}
 \textcolor{gregoriocolor}{a} & \textcolor{gregoriocolor}{b} & \textcolor{gregoriocolor}{c} & \textcolor{gregoriocolor}{d} & \textcolor{gregoriocolor}{e} & \textcolor{gregoriocolor}{f} & \textcolor{gregoriocolor}{g} & \textcolor{gregoriocolor}{h} & \textcolor{gregoriocolor}{i} & \textcolor{gregoriocolor}{k} & \textcolor{gregoriocolor}{l} & \textcolor{gregoriocolor}{m} & \textcolor{gregoriocolor}{n} & \textcolor{gregoriocolor}{p} & \textcolor{gregoriocolor}{q} & \textcolor{gregoriocolor}{r} & \textcolor{gregoriocolor}{s} & \textcolor{gregoriocolor}{t} & \textcolor{gregoriocolor}{u} \\
 2 & 3 & 4 & 5 & 6 & 7 & 8 & 9 & 10 & 11 & 12 & 13 & 14 & 15 & 16 & 17 & 18 & 19 & 20 \\
\end{tabularx}
\vspace{0.5\baselineskip}
\noindent\begin{tabularx}{\linewidth}{*{12}{>{\centering\arraybackslash}X}}
 \textcolor{gregoriocolor}{A} & \textcolor{gregoriocolor}{B} & \textcolor{gregoriocolor}{C} & \textcolor{gregoriocolor}{D} & \textcolor{gregoriocolor}{E} & F & \textcolor{gregoriocolor}{F} & \textcolor{gregoriocolor}{G} & \textcolor{gregoriocolor}{H} & \textcolor{gregoriocolor}{M} & \textcolor{gregoriocolor}{N} & \textcolor{gregoriocolor}{P} \\
 21 & 22 & 23 & 24 & 25 & 26 & 25 & 26 & 27 & 28 & 29 & 1 \\
\end{tabularx}

\begin{paracol}{2}
\selectlanguage{latin}
\lettrine[lines=2]{A}{pud} Villáckum, in 
 Pannónia, natális sancti Joánnis de Capistráno, Sacerdótis ex Ordine Minórum 
 et Confessóris, vitæ sanctitáte ac fídei cathólicæ propagándæ zelo illústris; 
 qui Taurunénsem arcem, validíssimo Turcárum exércitu profligáto, suis 
 précibus et miráculis ab obsidióne liberávit. Ejus tamen festívitas 
 quinto Kaléndas Aprílis recólitur.
\switchcolumn
\selectlanguage{english}
\lettrine[lines=2]{A}{t} Vilak in Hungary, the birthday of 
 St. John Capistran, priest and confessor of the Order of Friars Minor, 
 illustrious for the sanctity of his life and his zeal for the propagation of 
 the Catholic faith. By his prayers and miracles, he routed a powerful 
 army of Turks, and forced them to quit the siege of Tornau. His 
 feastday, however, is celebrated on the 28th of March.
\switchcolumn*
\selectlanguage{latin}
Antiochíæ item natális 
 sancti Theodóri Presbyteri, qui, in persecutióne ímpii Juliáni comprehénsus, 
 et, post equúlei pœnam et multos ac duríssimos cruciátus, lampádibus étiam 
 circa látera appósitis adústus, tandem, cum in confessióne Christi 
 persísteret, gládii occisióne martyrium consummávit.
\switchcolumn
\selectlanguage{english}
At Antioch, the birthday of the holy 
 priest Theodore, who was arrested in the persecution of the impious Julian. 
 After the torment of the rack and many severe tortures, including the 
 burning of his sides with torches, he persisted in the confession of Christ, 
 and so his martyrdom was completed by death with the sword.
\switchcolumn*
\selectlanguage{latin}
Ad fundum Ursoniánum 
 prope Gades, in Hispánia, sanctórum Mártyrum Servándi et Germáni, qui, in 
 persecutióne Diocletiáni, sub Viatóre Vicário, post vérbera, squalórum cárceris, famis ac sitis injúriam, et longíssimi itíneris labórem, quem 
 pertulérunt ferro onústi, novíssime martyrii sui cursum, cæsis cervícibus, 
 implevérunt; ex quibus Germánus Eméritæ, Servándus autem Híspali cónditus 
 est.
\switchcolumn
\selectlanguage{english}
At Osuma, near Cadiz in Spain, in 
 the persecution of Diocletian, under the subgovernor Viator, the holy 
 martyrs Servandus and Germanus. They were subjected to scourging, 
 imprisonment in a foul dungeon, want of food and drink, and the fatigue of a 
 long journey while loaded with fetters, and at length reached the end of 
 their martyrdom by having their heads stricken off. Germanus was 
 buried at Merida, and Servandus at Seville.
\switchcolumn*
\selectlanguage{latin}
Constantinópoli sancti 
 Ignátii Epíscopi, qui, cum Bardam Cæsarem ob repudiátam uxórem arguísset, ab 
 eo multis injúriis afféctus est, et in exsílium pulsus; sed, a sancto 
 Nicoláo, Románo Pontífice, restitútus, tandem in pace quiévit.
\switchcolumn
\selectlanguage{english}
At Constantinople, St. Ignatius, 
 bishop, who rebuked Bardas Caesar for putting away his wife, for which he 
 was subjected to many insults and driven into banishment. He was, 
 however, restored to his See by the Roman Pontiff Nicholas, and there died 
 in peace.
\switchcolumn*
\selectlanguage{latin}
Burdígalæ sancti 
 Severíni, Epíscopi Coloniénsis et Confessóris.
\switchcolumn
\selectlanguage{english}
At Bordeaux, St. Severin, bishop of 
 Cologne and confessor.
\switchcolumn*
\selectlanguage{latin}
Rotómagi sancti Románi 
 Epíscopi.
\switchcolumn
\selectlanguage{english}
At Rouen, Bishop St. Romanus.
\switchcolumn*
\selectlanguage{latin}
Apud Salérnum sancti 
 Veri Epíscopi.
\switchcolumn
\selectlanguage{english}
At Salerno, Bishop St. Verus.
\switchcolumn*
\selectlanguage{latin}
In território Ambianénsi sancti Domítii Presbyteri.
\switchcolumn
\selectlanguage{english}
In the district of Amiens, St. 
 Domitius, a priest.
\switchcolumn*
\selectlanguage{latin}
In pago Pictaviénsi 
 sancti Benedícti Confessóris.
\switchcolumn
\selectlanguage{english}
In the country of Poitiers, St. 
 Benedict, confessor.
\switchcolumn*
\selectlanguage{latin}
Mántuæ beáti Joánnis 
 Boni, ex Eremitárum sancti Augustíni Ordine, Confessóris; cujus præcláram 
 vitam sanctus Antonínus conscrípsit.
\switchcolumn
\selectlanguage{english}
At Mantua, blessed John the Good, of 
 the Order of Hermits of St. Augustine, whose celebrated life was written by 
 St. Antoninus.
\switchcolumn*
\selectlanguage{latin}
\end{paracol}


% ---- martyrology/mart10/mart1024.htm
\needspace{10\baselineskip}
\begin{paracol}{2}
\selectlanguage{latin}
\begin{center}{\color{gregoriocolor} Nono Kaléndas Novémbris. 
 Luna\dots\ }\end{center}
\switchcolumn
\selectlanguage{english}
\begin{center}{\color{gregoriocolor} The 
 Twenty-Fourth Day of 
 October. The\dots\ Day of the Moon.}\end{center}
\end{paracol}

\noindent\begin{tabularx}{\linewidth}{*{19}{>{\centering\arraybackslash}X}}
 \textcolor{gregoriocolor}{a} & \textcolor{gregoriocolor}{b} & \textcolor{gregoriocolor}{c} & \textcolor{gregoriocolor}{d} & \textcolor{gregoriocolor}{e} & \textcolor{gregoriocolor}{f} & \textcolor{gregoriocolor}{g} & \textcolor{gregoriocolor}{h} & \textcolor{gregoriocolor}{i} & \textcolor{gregoriocolor}{k} & \textcolor{gregoriocolor}{l} & \textcolor{gregoriocolor}{m} & \textcolor{gregoriocolor}{n} & \textcolor{gregoriocolor}{p} & \textcolor{gregoriocolor}{q} & \textcolor{gregoriocolor}{r} & \textcolor{gregoriocolor}{s} & \textcolor{gregoriocolor}{t} & \textcolor{gregoriocolor}{u} \\
 3 & 4 & 5 & 6 & 7 & 8 & 9 & 10 & 11 & 12 & 13 & 14 & 15 & 16 & 17 & 18 & 19 & 20 & 21 \\
\end{tabularx}
\vspace{0.5\baselineskip}
\noindent\begin{tabularx}{\linewidth}{*{12}{>{\centering\arraybackslash}X}}
 \textcolor{gregoriocolor}{A} & \textcolor{gregoriocolor}{B} & \textcolor{gregoriocolor}{C} & \textcolor{gregoriocolor}{D} & \textcolor{gregoriocolor}{E} & F & \textcolor{gregoriocolor}{F} & \textcolor{gregoriocolor}{G} & \textcolor{gregoriocolor}{H} & \textcolor{gregoriocolor}{M} & \textcolor{gregoriocolor}{N} & \textcolor{gregoriocolor}{P} \\
 22 & 23 & 24 & 25 & 26 & 27 & 26 & 27 & 28 & 29 & 1 & 2 \\
\end{tabularx}

\begin{paracol}{2}
\selectlanguage{latin}
\lettrine[lines=2]{F}{estum} sancti Raphaélis 
 Archángeli, cujus dígnitas ac benefícia in sacro Tobíæ libro celebrántur.
\switchcolumn
\selectlanguage{english}
\lettrine[lines=2]{T}{he} Feast of St. Raphael the 
 Archangel, whose dignity and benefits to mankind are set forth in the holy 
 book of Tobias.
\switchcolumn*
\selectlanguage{latin}
Venúsiæ, in Apúlia, natális sanctórum Mártyrum Felícis, Epíscopi Africáni; Audácti et Januárii, 
 Presbyterórum; Fortunáti et Séptimi, Lectórum. Hi omnes, témpore 
 Diocletiáni, a Magdelliáno Procuratóre multis diu vínculis et carcéribus in 
 Africa et Sicília maceráti sunt, et, cum Felix sacros Libros juxta ipsíus 
 Imperatóris edíctum trádere nullátenus voluísset, gládii tandem occisióne 
 consummáti sunt.
\switchcolumn
\selectlanguage{english}
At Venosa in Apulia, the birthday of 
 the holy martyrs Felix, an African bishop, Audactus and Januarius, priests, 
 and the lectors Fortunatus and Septimus. In the time of Diocletian, 
 under the governor Magdellian, they were loaded with fetters and imprisoned 
 for a long time in Africa and Sicily. Because Felix refused to deliver 
 the sacred books, they were at last slain with the sword.
\switchcolumn*
\selectlanguage{latin}
Tungris, in Bélgio, 
 sancti Evergísli, Epíscopi Coloniénsis et Mártyris; qui, ob pastorális 
 offícii curam illuc proféctus, ibídem, dum nocte solus ad monastérium 
 sanctíssimæ Genitrícis Dei Maríæ oratúrus pérgeret, a latrónibus sagítta 
 percússus occúbuit.
\switchcolumn
\selectlanguage{english}
At Tongres in Belgium, St. 
 Evergislus, bishop of Cologne and martyr. Because of his duties in the 
 pastoral office he journeyed there, and on the way stopped to pray alone at 
 the monastery of the Blessed Virgin Mary where he was killed by robbers who 
 struck him with an arrow.
\switchcolumn*
\selectlanguage{latin}
In civitáte Nagran, 
 apud Homerítas, in Arábia, pássio sanctórum Arétæ, et Sociórum ejus 
 trecentórum et quadragínta, témpore Justíni Imperatóris, sub Dúnaan, Judæo 
 tyránno. Post eos Christiána múlier incéndio trádita est; cujus fílius 
 annórum quinque, cum Christum balbutiéndo confiterétur, nec blandítiis nec 
 minis retinéri posset, in ignem ubi mater ardébat, se præcípitem dedit.
\switchcolumn
\selectlanguage{english}
In the city of Nagran in Arabia 
 Felix, the passion of St. Aretas and his companions, to the number of three 
 hundred and forty, in the time of Emperor Justin, under the Jewish tyrant 
 Dunaan. After them, a Christian woman was burned alive, whose 
 five-year-old son confessed Christ in a lisping voice and could not be 
 prevented by caresses or threats from rushing into the fire in which his 
 mother was burning
\switchcolumn*
\selectlanguage{latin}
Constantinópoli sancti 
 Procli Epíscopi.
\switchcolumn
\selectlanguage{english}
At Constantinople, St. Proclus, 
 bishop.
\switchcolumn*
\selectlanguage{latin}
In ínsula Sargiénsi sancti Maglórii Epíscopi, qui ibídem, dimísso Episcopáli offício, quod erga 
 sparsos in Armórica Británnos per triénnium exercúerat, monastérium 
 constrúxit, in quo sancte réliquum vitæ tempus exégit; cujus corpus Lutétiam 
 Parisiórum póstea translátum fuit.
\switchcolumn
\selectlanguage{english}
On the island of Jersey, St. 
 Maglorius, bishop, who laid down the Episcopal office after exercising it 
 for three years towards a few scattered people in Brittany. He built a 
 monastery on that island, and there spent the remainder of his life in holy 
 conversation. His body was later translated to Paris.
\switchcolumn*
\selectlanguage{latin}
In monastério Montis 
 Frígidi, diœcésis Carcassonénsis, in Gállia, sancti Antónii Maríæ Claret, 
 olim Archiepíscopi Cubáni, Fundatóris Missionariórum Filiórum Immaculáti 
 Cordis beátæ Maríæ Vírginis, animárum zelo et mansuetúdine præclári; quem 
 Pius Duodécimus, Póntifex Máximus, Sanctórum fastis adscrípsit.
\switchcolumn
\selectlanguage{english}
In the monastery of Fontfroide in 
 the diocese of Carcassonne in France, St. Anthony Mary Claret, formerly 
 Archbishop of Cuba, and founder of the Missionary Sons of the Immaculate 
 Heart of the Blessed Virgin Mary. He was renowned for his meekness and 
 zeal for souls, and was canonized by the Supreme Pontiff, Pius XII.
\switchcolumn*
\selectlanguage{latin}
In monastério Duríni, 
 in Gállia, sancti Martíni, Diáconi et Abbátis, cujus corpus inde ad Vertávum 
 monastérium delátum est.
\switchcolumn
\selectlanguage{english}
In the monastery of Durin in France, 
 St. Martin, abbot and deacon. His body was translated to the monastery 
 of Vertou.
\switchcolumn*
\selectlanguage{latin}
In Campánia sancti 
 Marci Solitárii, cujus præclára ópera sanctus Gregórius Papa descrípsit.
\switchcolumn
\selectlanguage{english}
In Campania, St. Mark, a solitary, 
 whose noble accomplishments have been recorded by St. Gregory.
\switchcolumn*
\selectlanguage{latin}
\end{paracol}


% ---- martyrology/mart10/mart1025.htm
\needspace{10\baselineskip}
\begin{paracol}{2}
\selectlanguage{latin}
\begin{center}{\color{gregoriocolor} Octávo Kaléndas Novémbris. 
 Luna\dots\ }\end{center}
\switchcolumn
\selectlanguage{english}
\begin{center}{\color{gregoriocolor} The 
 Twenty-Fifth Day of 
 October. The\dots\ Day of the Moon.}\end{center}
\end{paracol}

\noindent\begin{tabularx}{\linewidth}{*{19}{>{\centering\arraybackslash}X}}
 \textcolor{gregoriocolor}{a} & \textcolor{gregoriocolor}{b} & \textcolor{gregoriocolor}{c} & \textcolor{gregoriocolor}{d} & \textcolor{gregoriocolor}{e} & \textcolor{gregoriocolor}{f} & \textcolor{gregoriocolor}{g} & \textcolor{gregoriocolor}{h} & \textcolor{gregoriocolor}{i} & \textcolor{gregoriocolor}{k} & \textcolor{gregoriocolor}{l} & \textcolor{gregoriocolor}{m} & \textcolor{gregoriocolor}{n} & \textcolor{gregoriocolor}{p} & \textcolor{gregoriocolor}{q} & \textcolor{gregoriocolor}{r} & \textcolor{gregoriocolor}{s} & \textcolor{gregoriocolor}{t} & \textcolor{gregoriocolor}{u} \\
 4 & 5 & 6 & 7 & 8 & 9 & 10 & 11 & 12 & 13 & 14 & 15 & 16 & 17 & 18 & 19 & 20 & 21 & 22 \\
\end{tabularx}
\vspace{0.5\baselineskip}
\noindent\begin{tabularx}{\linewidth}{*{12}{>{\centering\arraybackslash}X}}
 \textcolor{gregoriocolor}{A} & \textcolor{gregoriocolor}{B} & \textcolor{gregoriocolor}{C} & \textcolor{gregoriocolor}{D} & \textcolor{gregoriocolor}{E} & F & \textcolor{gregoriocolor}{F} & \textcolor{gregoriocolor}{G} & \textcolor{gregoriocolor}{H} & \textcolor{gregoriocolor}{M} & \textcolor{gregoriocolor}{N} & \textcolor{gregoriocolor}{P} \\
 23 & 24 & 25 & 26 & 27 & 28 & 27 & 28 & 29 & 1 & 2 & 3 \\
\end{tabularx}

\begin{paracol}{2}
\selectlanguage{latin}
\lettrine[lines=2]{R}{omæ} sanctórum Mártyrum 
 Chrysánthi et Daríæ uxóris, qui, post multas, quas sub Celeríno Præfécto pro 
 Christo sustinuérunt, passiónes, a Numeriáno Imperatóre jussi sunt via 
 Salária in Arenário depóni, atque vivéntes illic terra et lapídibus óbrui.
\switchcolumn
\selectlanguage{english}
\lettrine[lines=2]{A}{t} Rome, the holy martyrs 
 Chrysanthus and his wife Daria. After many sufferings endured for 
 Christ under the prefect Celerinus, they were ordered by Emperor Numerian to 
 be thrown into a sandpit on the Salarian Way, where, being still alive, were 
 covered with earth and stones.
\switchcolumn*
\selectlanguage{latin}
Ibídem natális sancti 
 Marcellíni, Papæ et Mártyris; qui, sub Maximiáno, pro fide Christi, una cum 
 Cláudio, Cyríno et Antoníno, cápite truncátus est. Quo témpore ita 
 magna fuit persecútio, ut decem et septem míllia Christianórum, intra unum 
 mensem, martyrio coronaréntur. Ipsíus tamen sancti Marcellíni festum, 
 una cum festo sancti Cleti, Papæ et Mártyris, sexto Kaléndas Maji celebrátur.
\switchcolumn
\selectlanguage{english}
Also, the birthday of St. 
 Marcellinus, pope and martyr, who was beheaded for the faith of Christ in 
 the reign of Maximian along with Claudius Cyrinus and Antoninus. So 
 great was the persecution then that seventeen thousand Christians received 
 the crown of martyrdom in the space of one month. The feast of St. 
 Marcellinus is celebrated with that of St. Cletus, pope and martyr, on the 
 26th of April.
\switchcolumn*
\selectlanguage{latin}
Petragóricis, in 
 Gállia, sancti Frontónis, qui, a beáto Petro Apóstolo Epíscopus ordinátus, 
 cum Geórgio Presbytero magnam illíus gentis multitúdinem convértit ad 
 Christum, et miráculis clarus, in pace quiévit.
\switchcolumn
\selectlanguage{english}
At Perigueux in France, St. Fronto, 
 who was made bishop by the blessed apostle Peter. Along with a priest 
 named George, he converted to Christ a large number of people of that place, 
 and, renowned for miracles, rested in peace.
\switchcolumn*
\selectlanguage{latin}
Romæ natális sanctórum 
 quadragínta sex mílitum, qui, simul baptizáti a sancto Dionysio Papa, mox 
 Cláudii Imperatóris jussu decolláti sunt, ac via Salária sepúlti; ubi et 
 álii Mártyres centum vigínti et unus pósiti sunt, inter quos fuérunt quátuor 
 mílites Christi, scílicet Theodósius, Lúcius, Marcus et Petrus.
\switchcolumn
\selectlanguage{english}
Also at Rome, the birthday of 
 forty-six holy soldiers, who were baptized at the same time by Pope Denis, 
 and soon after beheaded by order of Emperor Claudius. They were buried 
 on the Salarian Way with one hundred and twenty-one other martyrs. 
 Among them are named four soldiers of Christ: Theodosius, Lucius, Mark, and 
 Peter.
\switchcolumn*
\selectlanguage{latin}
Túrribus, in Sardínia, sanctórum Mártyrum Proti Presbyteri, et Januárii Diáconi, qui, a sancto Cajo 
 Papa ad eam ínsulam missi, ibídem, témpore Diocletiáni, sub Bárbaro Præside, 
 consummáti sunt.
\switchcolumn
\selectlanguage{english}
At Sassari in Sardinia, the holy 
 martyrs Protus, a priest, and Januarius, a deacon, who were sent to that 
 island Pope St. Caius, and were martyred in the time of Diocletian under the 
 governor Barbarus.
\switchcolumn*
\selectlanguage{latin}
Constantinópoli pássio 
 sanctórum Martyrii Subdiáconi, et Marciáni Cantóris, qui ab hæréticis, sub 
 Constántio Imperatóre, necáti sunt.
\switchcolumn
\selectlanguage{english}
At Constantinople, the martyrdom of 
 the Saints Martyrius, subdeacon, and Marcian, a cantor, who were slain by 
 the heretics during the reign of Emperor Constantius.
\switchcolumn*
\selectlanguage{latin}
Suessióne, in Gálliis, 
 sanctórum Mártyrum Crispíni et Crispiniáni, nobílium Romanórum, qui, in 
 persecutióne Diocletiáni, sub Rictiováro Præside, post immánia torménta 
 gládio trucidáti, corónam martyrii sunt consecúti; quorum córpora póstea 
 Romam deláta fuérunt, atque in Ecclésia sancti Lauréntii in Pane et Perna 
 honorífice tumuláta.
\switchcolumn
\selectlanguage{english}
At Soissons in France, in the 
 persecution of Diocletian, the holy martyrs Crispin and Crispinian, noble 
 Romans. Under Governor Rictiovarus, after horrible torments, they were 
 put to the sword, and thus obtained the crown of martyrdom. Their 
 bodies were afterwards conveyed to Rome and entombed with due honours in the 
 church of St. Lawrence in Panisperna.
\switchcolumn*
\selectlanguage{latin}
Floréntiæ pássio beáti 
 Miniátis mílitis, qui, sub Décio Príncipe, pro fide Christi egrégie certans, 
 nóbili martyrio coronátur.
\switchcolumn
\selectlanguage{english}
At Florence, St. Minias, a soldier, 
 who fought valorously for the faith of Christ and was gloriously crowned 
 with martyrdom during the reign of Decius.
\switchcolumn*
\selectlanguage{latin}
Bríxiæ natális sancti 
 Gaudéntii Epíscopi, eruditióne et sanctitáte conspícui.
\switchcolumn
\selectlanguage{english}
At Brescia, the birthday of St. 
 Gaudentius, bishop, distinguished for his learning and holiness.
\switchcolumn*
\selectlanguage{latin}
Gavális, in Gállia, 
 sancti Hilárii Epíscopi.
\switchcolumn
\selectlanguage{english}
At Javoux in France, St. Hilary, 
 bishop.
\switchcolumn*
\selectlanguage{latin}
\end{paracol}


% ---- martyrology/mart10/mart1026.htm
\needspace{10\baselineskip}
\begin{paracol}{2}
\selectlanguage{latin}
\begin{center}{\color{gregoriocolor} Séptimo Kaléndas Novémbris. 
 Luna\dots\ }\end{center}
\switchcolumn
\selectlanguage{english}
\begin{center}{\color{gregoriocolor} The 
 Twenty-Sixth Day of 
 October. The\dots\ Day of the Moon.}\end{center}
\end{paracol}

\noindent\begin{tabularx}{\linewidth}{*{19}{>{\centering\arraybackslash}X}}
 \textcolor{gregoriocolor}{a} & \textcolor{gregoriocolor}{b} & \textcolor{gregoriocolor}{c} & \textcolor{gregoriocolor}{d} & \textcolor{gregoriocolor}{e} & \textcolor{gregoriocolor}{f} & \textcolor{gregoriocolor}{g} & \textcolor{gregoriocolor}{h} & \textcolor{gregoriocolor}{i} & \textcolor{gregoriocolor}{k} & \textcolor{gregoriocolor}{l} & \textcolor{gregoriocolor}{m} & \textcolor{gregoriocolor}{n} & \textcolor{gregoriocolor}{p} & \textcolor{gregoriocolor}{q} & \textcolor{gregoriocolor}{r} & \textcolor{gregoriocolor}{s} & \textcolor{gregoriocolor}{t} & \textcolor{gregoriocolor}{u} \\
 5 & 6 & 7 & 8 & 9 & 10 & 11 & 12 & 13 & 14 & 15 & 16 & 17 & 18 & 19 & 20 & 21 & 22 & 23 \\
\end{tabularx}
\vspace{0.5\baselineskip}
\noindent\begin{tabularx}{\linewidth}{*{12}{>{\centering\arraybackslash}X}}
 \textcolor{gregoriocolor}{A} & \textcolor{gregoriocolor}{B} & \textcolor{gregoriocolor}{C} & \textcolor{gregoriocolor}{D} & \textcolor{gregoriocolor}{E} & F & \textcolor{gregoriocolor}{F} & \textcolor{gregoriocolor}{G} & \textcolor{gregoriocolor}{H} & \textcolor{gregoriocolor}{M} & \textcolor{gregoriocolor}{N} & \textcolor{gregoriocolor}{P} \\
 24 & 25 & 26 & 27 & 28 & 29 & 28 & 29 & 1 & 2 & 3 & 4 \\
\end{tabularx}

\begin{paracol}{2}
\selectlanguage{latin}
\lettrine[lines=2]{R}{omæ} sancti Evarísti, 
 Papæ et Mártyris, qui Dei Ecclésiam, sub Hadriáno Imperatóre, suo sánguine 
 purpurávit.
\switchcolumn
\selectlanguage{english}
\lettrine[lines=2]{A}{t} Rome, St. Evaristus, pope and 
 martyr, who enriched the Church of God with his blood under Emperor Hadrian.
\switchcolumn*
\selectlanguage{latin}
In Africa sanctórum 
 Mártyrum Rogatiáni Presbyteri, et Felicíssimi, qui, in persecutióne 
 Valeriáni et Galliéni, illústri martyrio coronáti sunt; de quibus étiam 
 scribit sanctus Cypriánus in epístola ad Confessóres.
\switchcolumn
\selectlanguage{english}
In Africa, the holy martyrs 
 Felicissimus and the priest Rogatian, who received the bright crown of 
 martyrs in the persecution of Valerian and Gallienus. They are 
 mentioned by St. Cyprian in his Epistle to the Confessors.
\switchcolumn*
\selectlanguage{latin}
Nicomedíæ sanctórum 
 Mártyrum Luciáni, Flórii et Sociórum.
\switchcolumn
\selectlanguage{english}
At Nicomedia, the holy martyrs 
 Lucian, Florius, and their companions.
\switchcolumn*
\selectlanguage{latin}
Narbóne, in Gállia, 
 sancti Rústici, Epíscopi et Confessóris; qui cláruit tempóribus Valentiniáni 
 et Leónis Imperatórum.
\switchcolumn
\selectlanguage{english}
At Narbonne, St. Rusticus, bishop 
 and confessor, who flourished in the reigns of Emperors Leo and Valentian.
\switchcolumn*
\selectlanguage{latin}
Apud Salérnum sancti 
 Gaudiósi Epíscopi.
\switchcolumn
\selectlanguage{english}
At Salerno, St. Gaudiosus, bishop.
\switchcolumn*
\selectlanguage{latin}
Papíæ sancti Fulci 
 Epíscopi.
\switchcolumn
\selectlanguage{english}
At Pavia, Bishop St. Fulk.
\switchcolumn*
\selectlanguage{latin}
Item sancti 
 Quadragésimi Subdiáconi, qui et mórtuum resuscitávit.
\switchcolumn
\selectlanguage{english}
Also St. Quadragesimus, subdeacon, 
 who raised a dead man to life.
\switchcolumn*
\selectlanguage{latin}
\end{paracol}


% ---- martyrology/mart10/mart1027.htm
\needspace{10\baselineskip}
\begin{paracol}{2}
\selectlanguage{latin}
\begin{center}{\color{gregoriocolor} Sexto Kaléndas Novémbris. 
 Luna\dots\ }\end{center}
\switchcolumn
\selectlanguage{english}
\begin{center}{\color{gregoriocolor} The 
 Twenty-Seventh Day of 
 October. The\dots\ Day of the Moon.}\end{center}
\end{paracol}

\noindent\begin{tabularx}{\linewidth}{*{19}{>{\centering\arraybackslash}X}}
 \textcolor{gregoriocolor}{a} & \textcolor{gregoriocolor}{b} & \textcolor{gregoriocolor}{c} & \textcolor{gregoriocolor}{d} & \textcolor{gregoriocolor}{e} & \textcolor{gregoriocolor}{f} & \textcolor{gregoriocolor}{g} & \textcolor{gregoriocolor}{h} & \textcolor{gregoriocolor}{i} & \textcolor{gregoriocolor}{k} & \textcolor{gregoriocolor}{l} & \textcolor{gregoriocolor}{m} & \textcolor{gregoriocolor}{n} & \textcolor{gregoriocolor}{p} & \textcolor{gregoriocolor}{q} & \textcolor{gregoriocolor}{r} & \textcolor{gregoriocolor}{s} & \textcolor{gregoriocolor}{t} & \textcolor{gregoriocolor}{u} \\
 6 & 7 & 8 & 9 & 10 & 11 & 12 & 13 & 14 & 15 & 16 & 17 & 18 & 19 & 20 & 21 & 22 & 23 & 24 \\
\end{tabularx}
\vspace{0.5\baselineskip}
\noindent\begin{tabularx}{\linewidth}{*{12}{>{\centering\arraybackslash}X}}
 \textcolor{gregoriocolor}{A} & \textcolor{gregoriocolor}{B} & \textcolor{gregoriocolor}{C} & \textcolor{gregoriocolor}{D} & \textcolor{gregoriocolor}{E} & F & \textcolor{gregoriocolor}{F} & \textcolor{gregoriocolor}{G} & \textcolor{gregoriocolor}{H} & \textcolor{gregoriocolor}{M} & \textcolor{gregoriocolor}{N} & \textcolor{gregoriocolor}{P} \\
 25 & 26 & 27 & 28 & 29 & 30 & 29 & 1 & 2 & 3 & 4 & 5 \\
\end{tabularx}

\begin{paracol}{2}
\selectlanguage{latin}
\lettrine[lines=1]{V}{igília} sanctórum Apostolórum Simónis et Judæ.
\switchcolumn
\selectlanguage{english}
\lettrine[lines=1]{T}{he} vigil of the holy apostles Simon and Jude.
\switchcolumn*
\selectlanguage{latin}
Abulæ, in Hispánia, 
 pássio sanctórum Vincéntii, Sabínæ et Christétæ. Hi primum in equúleo 
 ádeo sunt exténti, ut omnes membrórum compáges laxaréntur; deínde cápita 
 eórum, lapídibus superpósita, usque ad cérebri excussiónem válidis véctibus 
 sunt contúsa, atque ita ipsi martyrium complevérunt, sub Præside Daciáno.
\switchcolumn
\selectlanguage{english}
At Avila in Spain, under the 
 governor Dacian, the Saints Vincent, Sabina, and Christeta. They were 
 first stretched on the rack in such a manner that all their limbs were 
 dislocated; then stones being laid on their heads, and their brains beaten 
 out with heavy bars, their martyrdom was fulfilled.
\switchcolumn*
\selectlanguage{latin}
Apud castrum Tyle, in 
 Gállia, sancti Floréntii Mártyris.
\switchcolumn
\selectlanguage{english}
At Tilchatel in France, St. 
 Florentius, martyr.
\switchcolumn*
\selectlanguage{latin}
In Cappadócia sanctárum 
 Mártyrum Capitolínæ, ejúsque ancíllæ Erothéidis, quæ sub Diocletiáno sunt 
 passæ.
\switchcolumn
\selectlanguage{english}
In Cappadocia, the holy martyrs 
 Capitolina, and Erotheides, her handmaid, who suffered under Diocletian.
\switchcolumn*
\selectlanguage{latin}
Apud Indos sancti 
 Fruménti Epíscopi, qui, primum ibi captívus, deínde, Epíscopus a sancto 
 Athanásio ordinátus, Evangélium in ea província propagávit.
\switchcolumn
\selectlanguage{english}
In India, St. Frumentius, bishop. 
 While he was a captive there he was consecrated bishop by St. Athanasius, 
 and propagated the Gospel in that country.
\switchcolumn*
\selectlanguage{latin}
Neápoli, in Campánia, sancti Gaudiósi, Epíscopi Africáni, qui ob Wandalórum persecutiónem venit in 
 Campániam, et, in monastério apud eam urbem, sancto fine quiévit.
\switchcolumn
\selectlanguage{english}
At Naples, St. Gaudiosus, an African 
 bishop who came to Campania because of the Vandal persecution, and died a 
 holy death in a monastery in that city.
\switchcolumn*
\selectlanguage{latin}
In Æthiópia sancti 
 Elésbaan Regis, qui, Christi hóstibus expugnátis, ac, témpore Justíni 
 Imperatóris, misso régio diadémate Hierosólymam, monásticam vitam, ut 
 vóverat agens, migrávit ad Dóminum.
\switchcolumn
\selectlanguage{english}
In Ethiopia, in the time of Emperor 
 Justin, St. Elesbaan, king. After having defeated the enemies of 
 Christ and sent his royal diadem to Jerusalem, he led a monastic life, as he 
 had vowed, and went to his reward.
\switchcolumn*
\selectlanguage{latin}
\end{paracol}


% ---- martyrology/mart10/mart1028.htm
\needspace{10\baselineskip}
\begin{paracol}{2}
\selectlanguage{latin}
\begin{center}{\color{gregoriocolor} Quinto Kaléndas Novémbris. 
 Luna\dots\ }\end{center}
\switchcolumn
\selectlanguage{english}
\begin{center}{\color{gregoriocolor} The 
 Twenty-Eighth Day of 
 October. The\dots\ Day of the Moon.}\end{center}
\end{paracol}

\noindent\begin{tabularx}{\linewidth}{*{19}{>{\centering\arraybackslash}X}}
 \textcolor{gregoriocolor}{a} & \textcolor{gregoriocolor}{b} & \textcolor{gregoriocolor}{c} & \textcolor{gregoriocolor}{d} & \textcolor{gregoriocolor}{e} & \textcolor{gregoriocolor}{f} & \textcolor{gregoriocolor}{g} & \textcolor{gregoriocolor}{h} & \textcolor{gregoriocolor}{i} & \textcolor{gregoriocolor}{k} & \textcolor{gregoriocolor}{l} & \textcolor{gregoriocolor}{m} & \textcolor{gregoriocolor}{n} & \textcolor{gregoriocolor}{p} & \textcolor{gregoriocolor}{q} & \textcolor{gregoriocolor}{r} & \textcolor{gregoriocolor}{s} & \textcolor{gregoriocolor}{t} & \textcolor{gregoriocolor}{u} \\
 7 & 8 & 9 & 10 & 11 & 12 & 13 & 14 & 15 & 16 & 17 & 18 & 19 & 20 & 21 & 22 & 23 & 24 & 25 \\
\end{tabularx}
\vspace{0.5\baselineskip}
\noindent\begin{tabularx}{\linewidth}{*{12}{>{\centering\arraybackslash}X}}
 \textcolor{gregoriocolor}{A} & \textcolor{gregoriocolor}{B} & \textcolor{gregoriocolor}{C} & \textcolor{gregoriocolor}{D} & \textcolor{gregoriocolor}{E} & F & \textcolor{gregoriocolor}{F} & \textcolor{gregoriocolor}{G} & \textcolor{gregoriocolor}{H} & \textcolor{gregoriocolor}{M} & \textcolor{gregoriocolor}{N} & \textcolor{gregoriocolor}{P} \\
 26 & 27 & 28 & 29 & 30 & 1 & 1 & 2 & 3 & 4 & 5 & 6 \\
\end{tabularx}

\begin{paracol}{2}
\selectlanguage{latin}
\lettrine[lines=2]{I}{n} Pérside natális 
 beatórum Apostolórum Simónis Chananæi, et Thaddǽi, qui et Judas dícitur. 
 Ex ipsis autem Simon in Ægypto, Thaddǽus in Mesopotámia Evangélium 
 prædicávit; deínde, in Pérsidem simul ingréssi, ibi, cum innúmeram gentis illíus multitúdinem Christo subdidíssent, martyrium consummárunt.
\switchcolumn
\selectlanguage{english}
\lettrine[lines=2]{I}{n} Persia, the birthday of the 
 blessed apostles Simon the Canaanite and Thaddeus, who is also called Jude. 
 Simon preached the Gospel in Egypt, Thaddeus in Mesopotamia. 
 Afterwards, entering Persia together, they converted to Christ a numberless 
 multitude of the inhabitants, then underwent martyrdom.
\switchcolumn*
\selectlanguage{latin}
Romæ sanctórum Mártyrum 
 Anastásiæ senióris Vírginis, et Cyrílli. Ipsa Virgo, in persecutióne 
 Valeriáni, sub Probo Præfécto, vínculis constrícta, cólaphis cæsa, igne et 
 verbéribus est cruciáta, et, cum in confessióne Christi permanéret immóbilis, 
 tandem, abscíssis mamíllis, evúlsis únguibus, déntibus comminútis, mánibus 
 pedibúsque præcísis, truncáta cápite, tot passiónum ornáta monílibus 
 migrávit ad Sponsum; Cyríllus autem, ei peténti aquam propínans, martyrium 
 pro mercéde accépit.
\switchcolumn
\selectlanguage{english}
At Rome, the holy martyrs Cyril and 
 Anastasia the Elder, virgin. In the persecution of Valerian, under the 
 prefect Probus, Anastasia was bound with chains, buffeted, subjected to fire 
 and scourging, and, as she remained immovable in the confession of Christ, 
 her breasts were cut away, her nails plucked out, her teeth broken, and her 
 hands, feet, and head severed from her body. Adorned with her 
 sufferings as with so many jewels, she went to her Spouse. At her 
 request, Cyril gave her some water to drink, and for his reward became a 
 martyr.
\switchcolumn*
\selectlanguage{latin}
Item Romæ sanctæ 
 Cyríllæ Vírginis, quæ fília éxstitit sanctæ Tryphóniæ, et, sub Cláudio 
 Príncipe, pro Christo juguláta est.
\switchcolumn
\selectlanguage{english}
In the same city, during the reign 
 of Claudius, St. Cyrilla, virgin, daughter of St. Tryphonia, who was pierced 
 through the throat for the faith of Christ.
\switchcolumn*
\selectlanguage{latin}
Apud Comum sancti 
 Fidélis Mártyris, sub Maximiáno Imperatóre.
\switchcolumn
\selectlanguage{english}
At Como, under Emperor Maximian, St. 
 Fidelis, martyr.
\switchcolumn*
\selectlanguage{latin}
Mogúntiæ sancti 
 Ferrútii Mártyris.
\switchcolumn
\selectlanguage{english}
At Mainz, St. Ferrutius, martyr.
\switchcolumn*
\selectlanguage{latin}
Meldis, in Gállia, 
 sancti Farónis, Epíscopi et Confessóris.
\switchcolumn
\selectlanguage{english}
At Meaux, in France, St. Faro, 
 bishop and confessor.
\switchcolumn*
\selectlanguage{latin}
Vercéllis sancti 
 Honoráti Epíscopi.
\switchcolumn
\selectlanguage{english}
At Vercelli, St. Honoratus, bishop.
\switchcolumn*
\selectlanguage{latin}
\end{paracol}


% ---- martyrology/mart10/mart1029.htm
\needspace{10\baselineskip}
\begin{paracol}{2}
\selectlanguage{latin}
\begin{center}{\color{gregoriocolor} Quarto Kaléndas Novémbris. 
 Luna\dots\ }\end{center}
\switchcolumn
\selectlanguage{english}
\begin{center}{\color{gregoriocolor} The 
 Twenty-Ninth Day of 
 October. The\dots\ Day of the Moon.}\end{center}
\end{paracol}

\noindent\begin{tabularx}{\linewidth}{*{19}{>{\centering\arraybackslash}X}}
 \textcolor{gregoriocolor}{a} & \textcolor{gregoriocolor}{b} & \textcolor{gregoriocolor}{c} & \textcolor{gregoriocolor}{d} & \textcolor{gregoriocolor}{e} & \textcolor{gregoriocolor}{f} & \textcolor{gregoriocolor}{g} & \textcolor{gregoriocolor}{h} & \textcolor{gregoriocolor}{i} & \textcolor{gregoriocolor}{k} & \textcolor{gregoriocolor}{l} & \textcolor{gregoriocolor}{m} & \textcolor{gregoriocolor}{n} & \textcolor{gregoriocolor}{p} & \textcolor{gregoriocolor}{q} & \textcolor{gregoriocolor}{r} & \textcolor{gregoriocolor}{s} & \textcolor{gregoriocolor}{t} & \textcolor{gregoriocolor}{u} \\
 8 & 9 & 10 & 11 & 12 & 13 & 14 & 15 & 16 & 17 & 18 & 19 & 20 & 21 & 22 & 23 & 24 & 25 & 26 \\
\end{tabularx}
\vspace{0.5\baselineskip}
\noindent\begin{tabularx}{\linewidth}{*{12}{>{\centering\arraybackslash}X}}
 \textcolor{gregoriocolor}{A} & \textcolor{gregoriocolor}{B} & \textcolor{gregoriocolor}{C} & \textcolor{gregoriocolor}{D} & \textcolor{gregoriocolor}{E} & F & \textcolor{gregoriocolor}{F} & \textcolor{gregoriocolor}{G} & \textcolor{gregoriocolor}{H} & \textcolor{gregoriocolor}{M} & \textcolor{gregoriocolor}{N} & \textcolor{gregoriocolor}{P} \\
 27 & 28 & 29 & 30 & 1 & 2 & 2 & 3 & 4 & 5 & 6 & 7 \\
\end{tabularx}

\begin{paracol}{2}
\selectlanguage{latin}
\lettrine[lines=2]{S}{anctórum} Episcopórum 
 Maximiliáni Mártyris, et Valentíni Confessóris.
\switchcolumn
\selectlanguage{english}
\lettrine[lines=2]{T}{he} holy bishop Maximian, martyr, 
 and Valentine, confessor.
\switchcolumn*
\selectlanguage{latin}
Sidóne, in Phœnícia, 
 sancti Zenóbii Presbyteri, qui, sub novíssimæ persecutiónis acerbitáte, ad 
 martyrium álios exhórtans, martyrio et ipse dignátus est.
\switchcolumn
\selectlanguage{english}
At Sidon in Phoenicia, St. Zenobius, 
 a priest. When the last persecution was raging, by exhorting others to 
 martyrdom, he himself was deemed worthy of it.
\switchcolumn*
\selectlanguage{latin}
In Lucánia sanctórum 
 Mártyrum Hyacínthi, Quincti, Feliciáni, et Lúcii.
\switchcolumn
\selectlanguage{english}
In Lucania, the holy martyrs 
 Hyacinth, Quinctus, Felician, and Lucius.
\switchcolumn*
\selectlanguage{latin}
Bérgomi sanctæ Eusébiæ, 
 Vírginis et Mártyris.
\switchcolumn
\selectlanguage{english}
At Bergamo, St. Eusebia, virgin and 
 martyr.
\switchcolumn*
\selectlanguage{latin}
Hierosólymis natális 
 beáti Narcíssi Epíscopi, sanctitáte, patiéntia ac fide laudábilis, qui, 
 centum et séxdecim annórum senex, felíciter migrávit ad Dóminum.
\switchcolumn
\selectlanguage{english}
At Jerusalem, the birthday of 
 blessed Narcissus, a bishop distinguished for holiness, patience, and faith, 
 who went to the kingdom of God at the age of one hundred and sixteen years.
\switchcolumn*
\selectlanguage{latin}
Augustodúni sancti 
 Joánnis, Epíscopi et Confessóris.
\switchcolumn
\selectlanguage{english}
At Autun, St. John, bishop and 
 confessor.
\switchcolumn*
\selectlanguage{latin}
Cassíope, in ínsula 
 Corcyra, sancti Donáti Epíscopi, de quo scribit beátus Gregórius Papa.
\switchcolumn
\selectlanguage{english}
At Cassiope, in the island of Corfu, 
 Bishop St. Donatus, mentioned by blessed Pope Gregory.
\switchcolumn*
\selectlanguage{latin}
Viénnæ, in Gállia, 
 deposítio beáti Theodóri Abbátis.
\switchcolumn
\selectlanguage{english}
At Vienne in France, the death of 
 blessed Theodore, abbot.
\switchcolumn*
\selectlanguage{latin}
\end{paracol}


% ---- martyrology/mart10/mart1030.htm
\needspace{10\baselineskip}
\begin{paracol}{2}
\selectlanguage{latin}
\begin{center}{\color{gregoriocolor} Tértio Kaléndas Novémbris. 
 Luna\dots\ }\end{center}
\switchcolumn
\selectlanguage{english}
\begin{center}{\color{gregoriocolor} The 
 Thirtieth Day of 
 October. The\dots\ Day of the Moon.}\end{center}
\end{paracol}

\noindent\begin{tabularx}{\linewidth}{*{19}{>{\centering\arraybackslash}X}}
 \textcolor{gregoriocolor}{a} & \textcolor{gregoriocolor}{b} & \textcolor{gregoriocolor}{c} & \textcolor{gregoriocolor}{d} & \textcolor{gregoriocolor}{e} & \textcolor{gregoriocolor}{f} & \textcolor{gregoriocolor}{g} & \textcolor{gregoriocolor}{h} & \textcolor{gregoriocolor}{i} & \textcolor{gregoriocolor}{k} & \textcolor{gregoriocolor}{l} & \textcolor{gregoriocolor}{m} & \textcolor{gregoriocolor}{n} & \textcolor{gregoriocolor}{p} & \textcolor{gregoriocolor}{q} & \textcolor{gregoriocolor}{r} & \textcolor{gregoriocolor}{s} & \textcolor{gregoriocolor}{t} & \textcolor{gregoriocolor}{u} \\
 9 & 10 & 11 & 12 & 13 & 14 & 15 & 16 & 17 & 18 & 19 & 20 & 21 & 22 & 23 & 24 & 25 & 26 & 27 \\
\end{tabularx}
\vspace{0.5\baselineskip}
\noindent\begin{tabularx}{\linewidth}{*{12}{>{\centering\arraybackslash}X}}
 \textcolor{gregoriocolor}{A} & \textcolor{gregoriocolor}{B} & \textcolor{gregoriocolor}{C} & \textcolor{gregoriocolor}{D} & \textcolor{gregoriocolor}{E} & F & \textcolor{gregoriocolor}{F} & \textcolor{gregoriocolor}{G} & \textcolor{gregoriocolor}{H} & \textcolor{gregoriocolor}{M} & \textcolor{gregoriocolor}{N} & \textcolor{gregoriocolor}{P} \\
 28 & 29 & 30 & 1 & 2 & 3 & 3 & 4 & 5 & 6 & 7 & 8 \\
\end{tabularx}

\begin{paracol}{2}
\selectlanguage{latin}
\lettrine[lines=2]{I}{n} Sardínia natális 
 sancti Pontiáni, Papæ et Mártyris, qui, ab Alexándro Imperatóre, una cum 
 Hippólyto Presbytero, in eam ínsulam deportátus, ibídem, mactátus fústibus, 
 martyrium consummávit. Ejus corpus a beáto Fabiáno Papa Romam delátum 
 est, atque in cœmetério Callísti sepúltum. Ipsíus tamen festum 
 recólitur tertiodécimo Kaléndas Decémbris.
\switchcolumn
\selectlanguage{english}
\lettrine[lines=2]{I}{n} Sardinia, the birthday of St. 
 Pontianus, pope and martyr. In the company of the priest Hippolytus, he 
 was exiled by Emperor Alexander, and achieved martyrdom by being scourged. 
 His body was brought to Rome by blessed Pope Fabian and buried in the 
 cemetery of Callistus. His feast, however, is celebrated on the 19th 
 of November.
\switchcolumn*
\selectlanguage{latin}
Ægéæ, in Cilícia, pássio sanctórum Zenóbii Epíscopi, et Zenóbiæ soróris, sub Diocletiáno 
 Imperatóre et Lysia Præside.
\switchcolumn
\selectlanguage{english}
At Aegea in Cilicia, in the reign of 
 Diocletian, under the governor Lysias, the martyrdom of Saints Zenobius, 
 bishop, and his sister Zenobia.
\switchcolumn*
\selectlanguage{latin}
Altíni, in Venetórum 
 fínibus, sancti Theonésti, Epíscopi et Mártyris, qui ab Ariánis occísus est.
\switchcolumn
\selectlanguage{english}
At Altino, in the neighbourhood of 
 Venice, St. Theonestus, bishop and martyr, who was slain by the Arians.
\switchcolumn*
\selectlanguage{latin}
In Africa natális 
 sanctórum Mártyrum ducentórum vigínti.
\switchcolumn
\selectlanguage{english}
In Africa, the birthday of two 
 hundred and twenty holy martyrs.
\switchcolumn*
\selectlanguage{latin}
Tingi, in Mauritánia, pássio sancti Marcélli Centuriónis, qui, sanctórum Cláudii ac Lupérci et 
 Victórii Mártyrum pater, cápitis abscissióne martyrium complévit sub 
 Agricoláo, agénte vices Præfécti prætório.
\switchcolumn
\selectlanguage{english}
At Tangier in Morocco, St. 
 Marcellus, a centurion, the father of Saints Claudius, Lupercus, and 
 Victorius. He achieved martyrdom by beheading under Agricola, deputy 
 praetor for Praefectus.
\switchcolumn*
\selectlanguage{latin}
Alexandríæ sanctórum 
 trédecim Mártyrum, qui, cum sanctis Juliáno, Euno et Macário, passi sunt sub 
 Décio Imperatóre.
\switchcolumn
\selectlanguage{english}
At Alexandria, in the reign of 
 Decius, thirteen holy martyrs who suffered with Saints Julian, Eunus, and 
 Macarius.
\switchcolumn*
\selectlanguage{latin}
Cárali, in Sardínia, sancti Saturníni Mártyris, qui, in persecutióne Diocletiáni, sub Bárbaro 
 Præside, cápite truncátus est.
\switchcolumn
\selectlanguage{english}
At Cagliari in Sardinia, St. 
 Saturninus, martyr, who was beheaded under the governor Barbarus, during the 
 persecution of Diocletian.
\switchcolumn*
\selectlanguage{latin}
Apaméæ, in Phrygia, 
 sancti Máximi Mártyris, sub eódem Diocletiáno
\switchcolumn
\selectlanguage{english}
At Apamea in Phrygia, St. Maximus, 
 martyr, under the same Diocletian.
\switchcolumn*
\selectlanguage{latin}
Legióne, in Hispánia, 
 sanctórum Mártyrum Cláudii, Lupérci et Victórii, filiórum sancti Marcélli 
 Centuriónis; qui, in persecutióne Diocletiáni et Maximiáni, sub Diogeniáno 
 Præside, jussi sunt decollári.
\switchcolumn
\selectlanguage{english}
At Leon in Spain, the holy martyrs 
 Claudius, Lupercus, and Victorius, the sons of St. Marcellus the centurion. 
 They were condemned to be beheaded by Diogenian, the governor, in the 
 persecution of Diocletian and Maximian.
\switchcolumn*
\selectlanguage{latin}
Lutétiæ Parisiórum 
 sancti Lucáni Mártyris.
\switchcolumn
\selectlanguage{english}
At Paris, St. Lucanus, martyr.
\switchcolumn*
\selectlanguage{latin}
Alexandríæ sanctæ 
 Eutrópiæ Mártyris, quæ, Mártyres vísitans, apprehénsa est, et, cum illis 
 sævíssime cruciáta, réddidit spíritum.
\switchcolumn
\selectlanguage{english}
At Alexandria, the martyr St. 
 Eutropia, who was arrested while visiting the martyrs, and rendered up her 
 soul after being cruelly tortured with them.
\switchcolumn*
\selectlanguage{latin}
Antiochíæ sancti 
 Serapiónis Epíscopi, eruditióne claríssimi.
\switchcolumn
\selectlanguage{english}
At Antioch, St. Serapion, a bishop 
 very celebrated for his learning.
\switchcolumn*
\selectlanguage{latin}
Cápuæ sancti Germáni, 
 Epíscopi et Confessóris, magnæ sanctitátis viri; cujus ánimam, in hora 
 óbitus ejus, ab Angelis in cælum deférri sanctus Benedíctus aspéxit.
\switchcolumn
\selectlanguage{english}
At Capua, St. Germanus, bishop and 
 confessor, a man of great sanctity, whose soul, at the very hour of death, 
 was seen by St. Benedict taken to heaven by angels.
\switchcolumn*
\selectlanguage{latin}
Poténtiæ, in Lucánia, 
 sancti Gerárdi Epíscopi.
\switchcolumn
\selectlanguage{english}
At Potenza in Lucania, St. Gerard, 
 bishop.
\switchcolumn*
\selectlanguage{latin}
\end{paracol}


% ---- martyrology/mart10/mart1031.htm
\needspace{10\baselineskip}
\begin{paracol}{2}
\selectlanguage{latin}
\begin{center}{\color{gregoriocolor} Prídie Kaléndas Novémbris. 
 Luna\dots\ }\end{center}
\switchcolumn
\selectlanguage{english}
\begin{center}{\color{gregoriocolor} The 
 Thirty-First Day of 
 October. The\dots\ Day of the Moon.}\end{center}
\end{paracol}

\noindent\begin{tabularx}{\linewidth}{*{19}{>{\centering\arraybackslash}X}}
 \textcolor{gregoriocolor}{a} & \textcolor{gregoriocolor}{b} & \textcolor{gregoriocolor}{c} & \textcolor{gregoriocolor}{d} & \textcolor{gregoriocolor}{e} & \textcolor{gregoriocolor}{f} & \textcolor{gregoriocolor}{g} & \textcolor{gregoriocolor}{h} & \textcolor{gregoriocolor}{i} & \textcolor{gregoriocolor}{k} & \textcolor{gregoriocolor}{l} & \textcolor{gregoriocolor}{m} & \textcolor{gregoriocolor}{n} & \textcolor{gregoriocolor}{p} & \textcolor{gregoriocolor}{q} & \textcolor{gregoriocolor}{r} & \textcolor{gregoriocolor}{s} & \textcolor{gregoriocolor}{t} & \textcolor{gregoriocolor}{u} \\
 10 & 11 & 12 & 13 & 14 & 15 & 16 & 17 & 18 & 19 & 20 & 21 & 22 & 23 & 24 & 25 & 26 & 27 & 28 \\
\end{tabularx}
\vspace{0.5\baselineskip}
\noindent\begin{tabularx}{\linewidth}{*{12}{>{\centering\arraybackslash}X}}
 \textcolor{gregoriocolor}{A} & \textcolor{gregoriocolor}{B} & \textcolor{gregoriocolor}{C} & \textcolor{gregoriocolor}{D} & \textcolor{gregoriocolor}{E} & F & \textcolor{gregoriocolor}{F} & \textcolor{gregoriocolor}{G} & \textcolor{gregoriocolor}{H} & \textcolor{gregoriocolor}{M} & \textcolor{gregoriocolor}{N} & \textcolor{gregoriocolor}{P} \\
 29 & 30 & 1 & 2 & 3 & 4 & 4 & 5 & 6 & 7 & 8 & 9 \\
\end{tabularx}

\begin{paracol}{2}
\selectlanguage{latin}
\lettrine[lines=1]{V}{igília} ómnium 
 Sanctórum.
\switchcolumn
\selectlanguage{english}
\lettrine[lines=1]{T}{he} Vigil of All Saints.
\switchcolumn*
\selectlanguage{latin}
Romæ sanctórum Ampliáti, 
 Urbáni et Narcíssi, quorum méminit sanctus Paulus ad Romános scribens, qui, 
 ob Evangélium Christi, a Judæis et Gentílibus cæsi sunt.
\switchcolumn
\selectlanguage{english}
At Rome, the Saints Ampliatus, 
 Urbanus, and Narcissus, who are mentioned by St. Paul in his Epistle to the 
 Romans. They were put to death by the Jews and Gentiles for the Gospel 
 of Christ.
\switchcolumn*
\selectlanguage{latin}
Constantinópoli sancti 
 Stachis Epíscopi, qui a beáto Andréa Apóstolo primus ejúsdem civitátis 
 Epíscopus ordinátus est.
\switchcolumn
\selectlanguage{english}
At Constantinople, St. Stachis, 
 bishop who was consecrated first bishop of that city by the blessed apostle 
 Andrew.
\switchcolumn*
\selectlanguage{latin}
Apud Augústam 
 Veromanduórum, in Gállia, sancti Quinctíni, civis Románi et ex órdine 
 Senatório viri, qui sub Maximiáno Imperatóre martyrium passus est; cujus 
 corpus post annos quinquagínta quinque, revelánte Angelo, invéntum est 
 incorrúptum.
\switchcolumn
\selectlanguage{english}
At Saint Quentin in France, the 
 martyr St. Quentin, a Roman citizen and senator, who suffered under Emperor 
 Maximian. By the revelation of an angel, his body was found incorrupt 
 after a lapse of fifty-five years.
\switchcolumn*
\selectlanguage{latin}
Medioláni sancti 
 Antoníni, Epíscopi et Confessóris.
\switchcolumn
\selectlanguage{english}
At Milan, St. Antoninus, bishop and 
 confessor.
\switchcolumn*
\selectlanguage{latin}
Ratisbónæ, in Bavária, sancti Wolfgángi Epíscopi.
\switchcolumn
\selectlanguage{english}
At Ratisbon in Bavaria, St. 
 Wolfgang, bishop.
\switchcolumn*
\selectlanguage{latin}
Palmæ, in Majórca ínsula, sancti Alfónsi Rodríguez, Coadjutóris temporális formáti e Societáte 
 Jesu et Confessóris, humilitáte ac jugi mortificatiónis stúdio insígnis; 
 quem Leo Décimus tértius, Póntifex Máximus, Sanctórum fastis adscrípsit.
\switchcolumn
\selectlanguage{english}
At Palma, in the island of Majorca, 
 St. Alphonsus Rodriguez, a lay brother of the Society of Jesus, whom Leo 
 XIII canonized because of his remarkable humility and constant love of 
 mortification.
\switchcolumn*
\selectlanguage{latin}
Romæ Translátio beáti 
 Nemésii Diáconi, et fíliæ Lucíllæ Vírginis, qui octávo Kaléndas Septémbris 
 decolláti sunt.
\switchcolumn
\selectlanguage{english}
At Rome, the translation of blessed 
 Nemesius, deacon, and his daughter, the virgin Lucilla, who were beheaded on 
 the 25th of August.
\switchcolumn*
\selectlanguage{latin}
\end{paracol}

\setrunningtitles{November}{November}

% ---- martyrology/mart11/mart1101.htm
\needspace{10\baselineskip}
\begin{paracol}{2}
\selectlanguage{latin}
\begin{center}{\color{gregoriocolor} Kaléndis Novémbris. 
 Luna\dots\ }\end{center}
\switchcolumn
\selectlanguage{english}
\begin{center}{\color{gregoriocolor} The First Day of 
 November. The\dots\ Day of the Moon.}\end{center}
\end{paracol}

\noindent\begin{tabularx}{\linewidth}{*{19}{>{\centering\arraybackslash}X}}
 \textcolor{gregoriocolor}{a} & \textcolor{gregoriocolor}{b} & \textcolor{gregoriocolor}{c} & \textcolor{gregoriocolor}{d} & \textcolor{gregoriocolor}{e} & \textcolor{gregoriocolor}{f} & \textcolor{gregoriocolor}{g} & \textcolor{gregoriocolor}{h} & \textcolor{gregoriocolor}{i} & \textcolor{gregoriocolor}{k} & \textcolor{gregoriocolor}{l} & \textcolor{gregoriocolor}{m} & \textcolor{gregoriocolor}{n} & \textcolor{gregoriocolor}{p} & \textcolor{gregoriocolor}{q} & \textcolor{gregoriocolor}{r} & \textcolor{gregoriocolor}{s} & \textcolor{gregoriocolor}{t} & \textcolor{gregoriocolor}{u} \\
 11 & 12 & 13 & 14 & 15 & 16 & 17 & 18 & 19 & 20 & 21 & 22 & 23 & 24 & 25 & 26 & 27 & 28 & 29 \\
\end{tabularx}
\vspace{0.5\baselineskip}
\noindent\begin{tabularx}{\linewidth}{*{12}{>{\centering\arraybackslash}X}}
 \textcolor{gregoriocolor}{A} & \textcolor{gregoriocolor}{B} & \textcolor{gregoriocolor}{C} & \textcolor{gregoriocolor}{D} & \textcolor{gregoriocolor}{E} & F & \textcolor{gregoriocolor}{F} & \textcolor{gregoriocolor}{G} & \textcolor{gregoriocolor}{H} & \textcolor{gregoriocolor}{M} & \textcolor{gregoriocolor}{N} & \textcolor{gregoriocolor}{P} \\
 30 & 1 & 2 & 3 & 4 & 5 & 5 & 6 & 7 & 8 & 9 & 10 \\
\end{tabularx}

\begin{paracol}{2}
\selectlanguage{latin}
\lettrine[lines=2]{F}{estívitas} ómnium 
 Sanctórum, quam in honórem beátæ Dei Genitrícis Vírginis Maríæ et sanctórum 
 Mártyrum Bonifátius Papa Quartus, cum templum Pántheon tértio Idus Maji 
 dedicásset, célebrem et generálem instítuit agi quotánnis in urbe Roma. 
 Sed Gregórius item Quartus póstmodum decrévit, eándem festivitátem, quæ 
 váriis modis jam in divérsis Ecclésiis celebrabátur, in honórem ómnium 
 Sanctórum solémniter hac die ab univérsa Ecclésia perpétuo observári.
\switchcolumn
\selectlanguage{english}
\lettrine[lines=2]{T}{he} Festival of All Saints, which Pope Boniface IV, after the dedication 
 of the Pantheon, ordained to be kept generally and solemnly every year on 
 the 13th of May, in the city of Rome, in honour of the blessed Virgin Mary, 
 Mother of God, and of the holy martyrs. It was afterwards decreed by 
 Gregory IV that this feast, which was then celebrated in many dioceses, but 
 at different times, should be on this day kept by the whole Church in honour 
 of all the saints.
\switchcolumn*
\selectlanguage{latin}
In Pérside sanctórum 
 Mártyrum Joánnis Epíscopi, et Jacóbi Presbyteri, sub Sápore Rege.
\switchcolumn
\selectlanguage{english}
In Persia, the holy martyrs John, a 
 bishop, and James, a priest, under King Sapor.
\switchcolumn*
\selectlanguage{latin}
Tarracínæ, in Campánia, natális sancti Cæsárii Diáconi, qui, diébus multis in custódia macerátus, 
 póstea, cum sancto Juliáno Presbytero, in saccum missus et in mare 
 præcipitátus est.
\switchcolumn
\selectlanguage{english}
At Terracina in Campania, the 
 birthday of St. Caesarius, deacon, who was detained many days in prison, 
 afterwards put into a sack with the priest St. Julian, and then thrown into 
 the sea.
\switchcolumn*
\selectlanguage{latin}
In castro Divióne 
 sancti Benígni Presbyteri, qui a beáto Polycárpo missus est in Gálliam ad 
 prædicándum Evangélium; et, postquam, sub Marco Aurélio Imperatóre, a 
 Teréntio Júdice gravíssimis torméntis multiplíciter est afflíctus, tandem 
 ejus collum vecte férreo tundi et corpus láncea perforári jubétur.
\switchcolumn
\selectlanguage{english}
At Dijon, St. Benignus, a priest, 
 who was sent to France by blessed Polycarp to preach the Gospel. After 
 he had been subjected to many grievous torments by the judge Terentius, 
 under Emperor Marcus Aurelius, he was finally condemned to have his neck 
 struck with an iron bar and his body pierced with a lance.
\switchcolumn*
\selectlanguage{latin}
Damásci pássio 
 sanctórum Cæsárii, Dácii et aliórum quinque.
\switchcolumn
\selectlanguage{english}
At Damascus, the martyrdom of the 
 Saints Caesarius, Dacius, and five others.
\switchcolumn*
\selectlanguage{latin}
Eódem die sanctæ Maríæ 
 ancíllæ, quæ, Christiánæ religiónis nómine accusáta, hinc, sub Hadriáno 
 Imperatóre, diris verbéribus afflícta, equúlei extensiónem et ungulárum 
 laceratiónem passa, martyrium complévit.
\switchcolumn
\selectlanguage{english}
On the same day, St. Mary, a servant girl. Being 
 accused of professing the Christian religion in the time of Emperor Hadrian, 
 she was subjected to cruel scourging, to torture on the rack, and the 
 lacerating of her body with iron hooks, and thus completed her martyrdom
\switchcolumn*
\selectlanguage{latin}
Tarsi, in Cilícia, sanctárum Cyréniæ et Juliánæ Mártyrum, sub Maximiáno Imperatóre.
\switchcolumn
\selectlanguage{english}
At Tarsus in Cilicia, under Emperor 
 Maximian, the Saints Cyrenia and Juliana.
\switchcolumn*
\selectlanguage{latin}
Arvérnis, in Gállia, 
 sancti Austremónii, qui fuit primus ejúsdem civitátis Epíscopus.
\switchcolumn
\selectlanguage{english}
At Auvergne in France, St. 
 Austremonius, first bishop of Clermont.
\switchcolumn*
\selectlanguage{latin}
Lutétiæ Parisiórum 
 deposítio sancti Marcélli Epíscopi.
\switchcolumn
\selectlanguage{english}
At Paris, the death of St. 
 Marcellus, bishop.
\switchcolumn*
\selectlanguage{latin}
Bajócis, in Gállia, 
 sancti Vigóris Epíscopi, témpore Childebérti, Francórum Regis.
\switchcolumn
\selectlanguage{english}
At Bayeux, in the reign of the 
 Frankish king Childebert, St. Vigor, bishop.
\switchcolumn*
\selectlanguage{latin}
Andégavi, in Gállia, 
 deposítio sancti Licínii Epíscopi, venerábilis sanctitátis viri.
\switchcolumn
\selectlanguage{english}
At Angers in France, the death of 
 the aged holy man, St. Licinius, bishop.
\switchcolumn*
\selectlanguage{latin}
Tíbure sancti Severíni 
 Mónachi.
\switchcolumn
\selectlanguage{english}
At Tivoli, St. Severinus, monk.
\switchcolumn*
\selectlanguage{latin}
Lyricánti, in 
 Wastinénsi Gálliæ território, sancti Maturíni Confessóris.
\switchcolumn
\selectlanguage{english}
In Gatinais in France, St. Mathurin, 
 confessor.
\switchcolumn*
\selectlanguage{latin}
\end{paracol}


% ---- martyrology/mart11/mart1102.htm
\needspace{10\baselineskip}
\begin{paracol}{2}
\selectlanguage{latin}
\begin{center}{\color{gregoriocolor} Quarto Nonas Novémbris. 
 Luna\dots\ }\end{center}
\switchcolumn
\selectlanguage{english}
\begin{center}{\color{gregoriocolor} The Second Day of 
 November. The\dots\ Day of the Moon.}\end{center}
\end{paracol}

\noindent\begin{tabularx}{\linewidth}{*{19}{>{\centering\arraybackslash}X}}
 \textcolor{gregoriocolor}{a} & \textcolor{gregoriocolor}{b} & \textcolor{gregoriocolor}{c} & \textcolor{gregoriocolor}{d} & \textcolor{gregoriocolor}{e} & \textcolor{gregoriocolor}{f} & \textcolor{gregoriocolor}{g} & \textcolor{gregoriocolor}{h} & \textcolor{gregoriocolor}{i} & \textcolor{gregoriocolor}{k} & \textcolor{gregoriocolor}{l} & \textcolor{gregoriocolor}{m} & \textcolor{gregoriocolor}{n} & \textcolor{gregoriocolor}{p} & \textcolor{gregoriocolor}{q} & \textcolor{gregoriocolor}{r} & \textcolor{gregoriocolor}{s} & \textcolor{gregoriocolor}{t} & \textcolor{gregoriocolor}{u} \\
 12 & 13 & 14 & 15 & 16 & 17 & 18 & 19 & 20 & 21 & 22 & 23 & 24 & 25 & 26 & 27 & 28 & 29 & 30 \\
\end{tabularx}
\vspace{0.5\baselineskip}
\noindent\begin{tabularx}{\linewidth}{*{12}{>{\centering\arraybackslash}X}}
 \textcolor{gregoriocolor}{A} & \textcolor{gregoriocolor}{B} & \textcolor{gregoriocolor}{C} & \textcolor{gregoriocolor}{D} & \textcolor{gregoriocolor}{E} & F & \textcolor{gregoriocolor}{F} & \textcolor{gregoriocolor}{G} & \textcolor{gregoriocolor}{H} & \textcolor{gregoriocolor}{M} & \textcolor{gregoriocolor}{N} & \textcolor{gregoriocolor}{P} \\
 1 & 2 & 3 & 4 & 5 & 6 & 6 & 7 & 8 & 9 & 10 & 11 \\
\end{tabularx}

\begin{paracol}{2}
\selectlanguage{latin}
\lettrine[lines=1]{C}{ommemorátio} ómnium 
 Fidélium Defunctórum.
\switchcolumn
\selectlanguage{english}
\lettrine[lines=1]{T}{he} Commemoration of all the 
 Faithful Departed.
\switchcolumn*
\selectlanguage{latin}
Si dies secunda Novembris incidat in Dominicam, prædicta verba leguntur 
 primo loco, die sequenti.
\switchcolumn
\selectlanguage{english}
If the second 
 day of November should occur on a Sunday, the above words are omitted and read 
 instead at the beginning on the following day.
\switchcolumn*
\selectlanguage{latin}
Pœtovióne, in Pannónia 
 superióre, natális sancti Victoríni, ejúsdem civitátis Epíscopi, qui, post 
 multa édita scripta (ut sanctus Hierónymus testátur), in persecutióne 
 Diocletiáni, martyrio coronátus est.
\switchcolumn
\selectlanguage{english}
At Pettau in Styria, the birthday of 
 St. Victorinus, bishop of that city, who, after publishing many writings, as 
 is attested to by St. Jerome, was crowned with martyrdom in the persecution 
 of Diocletian.
\switchcolumn*
\selectlanguage{latin}
Tergéste pássio beáti 
 Justi, qui in eádem persecutióne, sub Manátio Præside, martyrium consummávit.
\switchcolumn
\selectlanguage{english}
At Trieste, blessed Justus, who 
 fulfilled his martyrdom in the same persecution under the governor Manatius.
\switchcolumn*
\selectlanguage{latin}
Sebáste, in Arménia, sanctórum Cartérii, Styríaci, Tobíæ, Eudóxii, Agápii et Sociórum Mártyrum, 
 sub Licínio Imperatóre.
\switchcolumn
\selectlanguage{english}
At Sebaste in Armenia, the Saints 
 Carterius, Styriacus, Tobias, Eudoxius, Agapius, and their companions, 
 martyrs under Emperor Licinius.
\switchcolumn*
\selectlanguage{latin}
In Pérside sanctórum 
 Mártyrum Acíndyni, Pegásii, Aphthónii, Elpidíphori et Anempodísti, cum 
 plúrimis Sóciis.
\switchcolumn
\selectlanguage{english}
In Persia, the holy martyrs 
 Acindynus, Pegasius, Aphthonius, Elpiderphorus, and Anempodistus, with many 
 companions.
\switchcolumn*
\selectlanguage{latin}
In Africa natális 
 sanctórum Mártyrum Públii, Victóris, Hermétis, et Pápiæ.
\switchcolumn
\selectlanguage{english}
In Africa, the birthday of the holy 
 martyrs Publius, Victor, Hermes, and Papias.
\switchcolumn*
\selectlanguage{latin}
Tarsi, in Cilícia, 
 sanctæ Eustóchii, Vírginis et Mártyris; quæ, sub Juliáno Apóstata, post dira 
 torménta, in oratióne réddidit spíritum.
\switchcolumn
\selectlanguage{english}
At Tarsus in Cilicia, in the reign 
 of Julian the Apostate, St. Eustochium, virgin and martyr, who breathed her 
 last in prayer in the midst of severe torments.
\switchcolumn*
\selectlanguage{latin}
Laodicéæ, in Syria, 
 sancti Theódoti Epíscopi, qui non solum verbis, sed rebus quoque et 
 virtútibus fuit ornátus.
\switchcolumn
\selectlanguage{english}
At Laodicea in Syria, St. Theodotus, 
 a bishop powerful in words and adorned with good works and virtues.
\switchcolumn*
\selectlanguage{latin}
Viénnæ, in Gállia, 
 sancti Geórgii Epíscopi.
\switchcolumn
\selectlanguage{english}
At Vienne in France, the bishop St. 
 George.
\switchcolumn*
\selectlanguage{latin}
In monastério Agaunénsi, 
 in Gállia, sancti Ambrósii Abbátis.
\switchcolumn
\selectlanguage{english}
In the monastery of St. Moritz in 
 Switzerland, St. Ambrose, abbot.
\switchcolumn*
\selectlanguage{latin}
Cyri, in Syria, sancti 
 Marciáni Confessóris.
\switchcolumn
\selectlanguage{english}
At Cyrus in Syria, St. Marcian, 
 confessor.
\switchcolumn*
\selectlanguage{latin}
\end{paracol}

% TODO: check the martyrology to see how this is to be formatted...

% ---- martyrology/mart11/mart1103.htm
\needspace{10\baselineskip}
\begin{paracol}{2}
\selectlanguage{latin}
\begin{center}{\color{gregoriocolor} Postea dicitur: Tértio Nonas 
 Novémbris, Luna\dots\, et continuatur Lectio usque ad 
 finem, more solito.}\end{center}
\switchcolumn
\selectlanguage{english}
\begin{center}{\color{gregoriocolor} After which 
 is read: The Third Day of 
 November, the... Day of the Moon, and the rest in the 
 usual manner.}\end{center}
\end{paracol}



\begin{paracol}{2}
\selectlanguage{latin}
\lettrine[lines=2]{T}{értio} Nonas Novémbris. 
 Luna\dots\
\switchcolumn
\selectlanguage{english}
\lettrine[lines=2]{T}{he} Third Day of 
 November. The\dots\ Day of the Moon.
\switchcolumn*
\selectlanguage{latin}
Commemorátio ómnium 
 Fidélium Defunctórum.
\switchcolumn
\selectlanguage{english}
The Commemoration of all the 
 Faithful Departed.
\switchcolumn*
\selectlanguage{latin}
Medioláni natália 
 sancti Cároli Borromæi Cardinális, Epíscopi Mediolanénsis et Confessóris, 
 quem, sanctitáte conspícuum et miráculis clarum, Paulus Papa Quintus in 
 Sanctórum númerum rétulit. Ipsíus tamen festívitas sequénti die 
 celebrátur.
\switchcolumn
\selectlanguage{english}
At Milan, St. Charles Borromeo, 
 cardinal, bishop of that city, and confessor, who was ranked among the 
 saints by Paul V for the holiness of his life and for his renown for 
 miracles. His feast is observed on the following day.
\switchcolumn*
\selectlanguage{latin}
Eódem die natális 
 quoque sancti Quarti, Apostolórum discípuli.
\switchcolumn
\selectlanguage{english}
On the same day, the birthday of St. 
 Quartus, a disciple of the apostles.
\switchcolumn*
\selectlanguage{latin}
Vitérbii sanctórum 
 Mártyrum Valentíni Presbyteri, et Hilárii Diáconi, qui, in persecutióne 
 Maximiáni, ob Christi fidem, cum saxi póndere in Tíberim præcipitáti et inde 
 ab Angelo divínitus erépti, demum, abscíssis cervícibus, corónam martyrii 
 percepérunt.
\switchcolumn
\selectlanguage{english}
At Viterbo, during the persecution 
 of Maximian, the holy martyrs Valentine, a priest, and Hilary, a deacon. 
 For their attachment to the faith of Christ, they were tied to a stone and 
 cast into the Tiber, but being miraculously delivered by an angel, they were 
 beheaded, and thus crowned with the glory of martyrdom.
\switchcolumn*
\selectlanguage{latin}
Cæsaréæ, in Cappadócia, sanctórum Mártyrum Germáni, Theóphili, Cæsárii et Vitális; qui, in 
 persecutióne Deciána, óptime duxérunt martyrium.
\switchcolumn
\selectlanguage{english}
At Caesarea in Cappadocia, the holy 
 martyrs Germanus, Theophilus, Caesarius, and Vitalis, who nobly endured 
 martyrdom in the Decian persecution.
\switchcolumn*
\selectlanguage{latin}
Cæsaraugústæ, in 
 Hispánia, sanctórum innumerabílium Mártyrum, qui, sub Hispaniárum Præside 
 Daciáno, mirabíliter occubuérunt pro Christo.
\switchcolumn
\selectlanguage{english}
At Saragossa in Spain, the countless 
 holy martyrs who lay down their lives with admirable fervour for the faith 
 of Christ under Dacian, governor of Spain.
\switchcolumn*
\selectlanguage{latin}
In Anglia sanctæ 
 Wenefrídæ, Vírginis et Mártyris.
\switchcolumn
\selectlanguage{english}
In England, St. Winifred, virgin and 
 martyr.
\switchcolumn*
\selectlanguage{latin}
In monastério 
 Clarævallénsi, in Gállia, deposítio sancti Malachíæ, Connerthénsis in 
 Hibérnia Epíscopi, qui multis virtútibus suo témpore cláruit; cujus vitam 
 sanctum Bernárdus Abbas conscrípsit.
\switchcolumn
\selectlanguage{english}
In the monastery of Clairvaux in 
 France, the death of St. Malachy, bishop of Armagh in Ireland, who won 
 renown in his own days for his many virtues, and whose life was written by 
 St. Bernard the Abbot.
\switchcolumn*
\selectlanguage{latin}
Eódem die sancti 
 Hubérti, Tungrénsis Epíscopi.
\switchcolumn
\selectlanguage{english}
On the same day, St. Hubert, bishop 
 of Tongres.
\switchcolumn*
\selectlanguage{latin}
Viénnæ, in Gállia, 
 sancti Domni, Epíscopi et Confessóris.
\switchcolumn
\selectlanguage{english}
At Vienne in France, St. Domnus, 
 bishop and confessor.
\switchcolumn*
\selectlanguage{latin}
Item deposítio sancti 
 Pirmíni, Meldénsis Epíscopi.
\switchcolumn
\selectlanguage{english}
Also, the death of St. Pirmin, 
 bishop of Meaux.
\switchcolumn*
\selectlanguage{latin}
Urgéllæ, in Hispánia 
 Tarraconénsi, sancti Hermengáudii Epíscopi.
\switchcolumn
\selectlanguage{english}
At Urgel in Spain, Bishop St. 
 Hermengaud.
\switchcolumn*
\selectlanguage{latin}
Romæ sanctæ Sílviæ, 
 matris sancti Gregórii Papæ.
\switchcolumn
\selectlanguage{english}
At Rome, St. Sylvia, mother of Pope 
 St. Gregory.
\switchcolumn*
\selectlanguage{latin}
\end{paracol}


% ---- martyrology/mart11/mart1104.htm
\needspace{10\baselineskip}
\begin{paracol}{2}
\selectlanguage{latin}
\begin{center}{\color{gregoriocolor} Postea dicitur: Prídie Nonas 
 Novémbris, Luna\dots\, et continuatur Lectio usque ad 
 finem, more solito.}\end{center}
\switchcolumn
\selectlanguage{english}
\begin{center}{\color{gregoriocolor} After which 
 is read: The Fourth Day of 
 November, the... Day of the Moon, and the rest in the 
 usual manner.}\end{center}
\end{paracol}



\begin{paracol}{2}
\selectlanguage{latin}
\lettrine[lines=2]{P}{rídie} Nonas Novémbris. 
 Luna\dots\
\switchcolumn
\selectlanguage{english}
\lettrine[lines=2]{T}{he} Fourth Day of 
 November. The\dots\ Day of the Moon.
\switchcolumn*
\selectlanguage{latin}
Sancti Cároli Borromæi Cardinális, Epíscopi Mediolanénsis et Confessóris, 
 qui migrávit in cælum prídie hujus diéi.
\switchcolumn
\selectlanguage{english}
St. Charles Borromeo, 
 cardinal, bishop of Milan, and confessor, whose birthday is on the day 
 previous.
\switchcolumn*
\selectlanguage{latin}
Bonóniæ sanctórum 
 Mártyrum Vitális et Agrícolæ; quorum prior postérioris servus ántea fuit, 
 póstea consors et colléga martyrii. In ipsum porro Vitálem 
 persecutóres ómnia tormentórum génera ita exercuérunt, ut non esset in 
 córpore ejus sine vúlnere locus, quæ ipse constánter pérferens, in oratióne 
 spíritum Deo réddidit; Agrícolam vero, plúrimis clavis cruci affigéntes, 
 interemérunt. Eórum translatióni sanctus Ambrósius cum interésset, 
 Mártyris clavos, sánguinem triumphálem et crucis lignum se collegísse refert, 
 ac sub sacris altáribus condidísse.
\switchcolumn
\selectlanguage{english}
At Bologna, the holy martyrs Vitalis 
 and Agricola. The former was first the servant of the latter, and 
 afterwards his partner and companion in martyrdom. He was subjected by 
 the persecutors to all kinds of torments, so that there was no part of his 
 body without wounds. After having suffered with constancy, he yielded 
 up his soul unto God in prayer. Agricola was put to death by being 
 fastened to a cross with many nails. St. Ambrose relates that being 
 present at the translation, he took the martyr's nails, his glorious blood, 
 and the wood of his cross, and deposited them under consecrated altars.
\switchcolumn*
\selectlanguage{latin}
In cœnóbio Cervi 
 Frígidi, territórii Meldénsis, natális sancti Felícis Valésii, Presbyteri et 
 Confessóris; qui Fundátor fuit Ordinis sanctíssimæ Trinitátis redemptiónis 
 captivórum. Ipsíus autem festum, ex dispositióne Innocéntii Papæ 
 Undécimi, celebrátur duodécimo Kaléndas Decémbris.
\switchcolumn
\selectlanguage{english}
In the monastery of Cerfroid, in the 
 territory of Meaux, St. Felix of Valois, priest and confessor, and founder of the Order of the Most 
 Holy Trinity for the Redemption of Captives, whose feast is celebrated on 
 the 20th of November by order of Pope Innocent XI.
\switchcolumn*
\selectlanguage{latin}
Eódem die natális 
 sanctórum Philólogi et Pátrobi, sancti Pauli Apóstoli discipulórum.
\switchcolumn
\selectlanguage{english}
On the same day, the birthday of the 
 Saints Philologus and Patrobas, disciples of the apostle St. Paul.
\switchcolumn*
\selectlanguage{latin}
Augustodúni sancti 
 Próculi, Epíscopi et Mártyris.
\switchcolumn
\selectlanguage{english}
At Autun, St. Proculus, bishop and 
 martyr.
\switchcolumn*
\selectlanguage{latin}
Myræ, in Lycia, 
 sanctórum Mártyrum Nicándri Epíscopi, et Hermæ Presbyteri, sub Libánio 
 Præside.
\switchcolumn
\selectlanguage{english}
At Myra in Lycia, under the governor 
 Libanius, the holy martyrs Nicander, a bishop, and Hermes, a priest.
\switchcolumn*
\selectlanguage{latin}
In pago Vilcassíno, in 
 Gállia, sancti Clari, Presbyteri et Mártyris.
\switchcolumn
\selectlanguage{english}
In the district of Vexin in France, 
 St. Clarus, priest and martyr.
\switchcolumn*
\selectlanguage{latin}
Ephesi sancti Porphyrii 
 Mártyris, sub Aureliáno Imperatóre.
\switchcolumn
\selectlanguage{english}
At Ephesus, St. Porphyrias, a martyr 
 under Emperor Aurelian.
\switchcolumn*
\selectlanguage{latin}
Apud Ruthénos, in 
 Gállia, beáti Amántii Epíscopi, cujus vita éxstitit sanctitáte et miráculis 
 gloriósa.
\switchcolumn
\selectlanguage{english}
At Rodez in France, blessed Bishop 
 Amantius, whose life stood out glorious by his sanctity and miracles.
\switchcolumn*
\selectlanguage{latin}
Romæ natális sancti 
 Piérii, Presbyteri Alexandríni, qui, in divínis Scriptúris nobíliter 
 erudítus, vita puríssimus, et ad Christiánam philosophíam nudus pénitus et 
 expedítus, sub Caro et Diocletiáno Princípibus, regénte Alexandrínam 
 Ecclésiam Theóna, florentíssime dócuit pópulum, et divérsos tractátus édidit; 
 post persecutiónem vero, omne vitæ suæ tempus Romæ versátus, in pace quiévit.
\switchcolumn
\selectlanguage{english}
At Rome, the birthday of St. Pierius, 
 priest of Alexandria, who was well versed in the Holy Scriptures, most pure 
 in his life, and highly skilled in Christian philosophy. He taught the 
 people and became famous under Emperors Carus and Diocletian, when Theonas 
 governed the Church of Alexandria. After the persecution, he spent the 
 remainder of his life at Rome, where he died in peace.
\switchcolumn*
\selectlanguage{latin}
In Bithynia sancti 
 Joannícii Abbátis.
\switchcolumn
\selectlanguage{english}
In Bithynia, St. Joannicius, abbot.
\switchcolumn*
\selectlanguage{latin}
Apud Albam Regálem, in 
 Pannónia, deposítio beáti Emeríci Confessóris, qui fuit fílius sancti 
 Stéphani, Hungarórum Regis.
\switchcolumn
\selectlanguage{english}
In Hungary at Alba Regalis, the 
 death of blessed Emeric, confessor, the son of St. Stephen, king of Hungary.
\switchcolumn*
\selectlanguage{latin}
Tréviris sanctæ Modéstæ 
 Vírginis.
\switchcolumn
\selectlanguage{english}
At Treves, St. Modesta, virgin.
\switchcolumn*
\selectlanguage{latin}
\end{paracol}


% ---- martyrology/mart11/mart1105.htm
\needspace{10\baselineskip}
\begin{paracol}{2}
\selectlanguage{latin}
\begin{center}{\color{gregoriocolor} Nonis Novémbris. 
 Luna\dots\ }\end{center}
\switchcolumn
\selectlanguage{english}
\begin{center}{\color{gregoriocolor} The Fifth Day of 
 November. The\dots\ Day of the Moon.}\end{center}
\end{paracol}

\noindent\begin{tabularx}{\linewidth}{*{19}{>{\centering\arraybackslash}X}}
 \textcolor{gregoriocolor}{a} & \textcolor{gregoriocolor}{b} & \textcolor{gregoriocolor}{c} & \textcolor{gregoriocolor}{d} & \textcolor{gregoriocolor}{e} & \textcolor{gregoriocolor}{f} & \textcolor{gregoriocolor}{g} & \textcolor{gregoriocolor}{h} & \textcolor{gregoriocolor}{i} & \textcolor{gregoriocolor}{k} & \textcolor{gregoriocolor}{l} & \textcolor{gregoriocolor}{m} & \textcolor{gregoriocolor}{n} & \textcolor{gregoriocolor}{p} & \textcolor{gregoriocolor}{q} & \textcolor{gregoriocolor}{r} & \textcolor{gregoriocolor}{s} & \textcolor{gregoriocolor}{t} & \textcolor{gregoriocolor}{u} \\
 15 & 16 & 17 & 18 & 19 & 20 & 21 & 22 & 23 & 24 & 25 & 26 & 27 & 28 & 29 & 30 & 1 & 2 & 3 \\
\end{tabularx}
\vspace{0.5\baselineskip}
\noindent\begin{tabularx}{\linewidth}{*{12}{>{\centering\arraybackslash}X}}
 \textcolor{gregoriocolor}{A} & \textcolor{gregoriocolor}{B} & \textcolor{gregoriocolor}{C} & \textcolor{gregoriocolor}{D} & \textcolor{gregoriocolor}{E} & F & \textcolor{gregoriocolor}{F} & \textcolor{gregoriocolor}{G} & \textcolor{gregoriocolor}{H} & \textcolor{gregoriocolor}{M} & \textcolor{gregoriocolor}{N} & \textcolor{gregoriocolor}{P} \\
 4 & 5 & 6 & 7 & 8 & 9 & 9 & 10 & 11 & 12 & 13 & 14 \\
\end{tabularx}

\begin{paracol}{2}
\selectlanguage{latin}
\lettrine[lines=2]{S}{ancti} Zacharíæ, 
 Sacerdótis et Prophétæ, qui pater éxstitit beáti Joánnis Baptístæ, 
 Præcursóris Dómini.
\switchcolumn
\selectlanguage{english}
\lettrine[lines=2]{S}{t.} Zachary, priest and prophet, the 
 father of blessed John Baptist, Forerunner of our Lord.
\switchcolumn*
\selectlanguage{latin}
Item sanctæ Elísabeth, 
 ejúsdem sanctíssimi Præcursóris matris.
\switchcolumn
\selectlanguage{english}
Also, St. Elizabeth, mother of the 
 same most holy Forerunner.
\switchcolumn*
\selectlanguage{latin}
Tarracínæ, in Campánia, natális sanctórum Mártyrum Felícis Presbyteri, et Eusébii Mónachi. Ex 
 his Eusébius, cum sepelísset sanctos Mártyres Juliánum et Cæsárium, et 
 multos convérteret ad fidem Christi, quos sanctus Felix Presbyter baptizábat, 
 una cum ipso Felíce tentus est; et, ad Júdicis forum ducti nec superáti, 
 inde in cárcerem inclúsi, ambo, nocte eádem, cum sacrificáre noluíssent, 
 decolláti sunt.
\switchcolumn
\selectlanguage{english}
At Terracina in Campania, the 
 birthday of the holy martyrs Felix, a priest, and Eusebius, a monk. 
 The latter buried the holy martyrs Julian and Caesarius, and converted to 
 the faith of Christ many whom the priest St. Felix baptized. They were 
 arrested together, and both were led to the tribunal of the judge, who could 
 not succeed in intimidating them; they were shut up in prison, and as they 
 refused to offer sacrifice, were beheaded that same night.
\switchcolumn*
\selectlanguage{latin}
Eméssæ, in Phœnícia, 
 sanctórum Mártyrum Galatiónis et Epistémis cónjugis, qui in Décii 
 persecutióne, flagris cæsi, mánibus pedibúsque et lingua ínsuper mutiláti, 
 dénique, truncáto cápite, martyrium consummárunt.
\switchcolumn
\selectlanguage{english}
At Emesa in Phoenicia, during the 
 persecution of Decius, the holy martyrs Galation and his wife Epistemis, who 
 were scourged, had their hands, feet, and tongue mutilated, and finally 
 fulfilled their martyrdom by beheading.
\switchcolumn*
\selectlanguage{latin}
Item sanctórum Mártyrum 
 Domníni, Theótimi, Philóthei, Silváni et Sociórum, sub Maximíno Imperatóre.
\switchcolumn
\selectlanguage{english}
Also, the holy martyrs Dominus, 
 Theotimus, Philotheus, Silvanus, and their companions, under Emperor 
 Maximinus.
\switchcolumn*
\selectlanguage{latin}
Medioláni sancti Magni, 
 Epíscopi et Confessóris.
\switchcolumn
\selectlanguage{english}
At Milan, St. Magnus, bishop and 
 confessor.
\switchcolumn*
\selectlanguage{latin}
Bríxiæ sancti 
 Dominatóris Epíscopi.
\switchcolumn
\selectlanguage{english}
At Brescia, St. Dominator, bishop.
\switchcolumn*
\selectlanguage{latin}
Tréviris sancti Fibítii, 
 qui ex Abbáte factus est ejúsdem civitátis Epíscopus.
\switchcolumn
\selectlanguage{english}
At Treves, St. Fibitius, first an 
 abbot and then bishop of that city.
\switchcolumn*
\selectlanguage{latin}
Aureliánis, in Gállia, 
 sancti Læti, Presbyteri et Confessóris.
\switchcolumn
\selectlanguage{english}
At Orleans in France, St. Laetus, 
 priest and confessor.
\switchcolumn*
\selectlanguage{latin}
\end{paracol}


% ---- martyrology/mart11/mart1106.htm
\needspace{10\baselineskip}
\begin{paracol}{2}
\selectlanguage{latin}
\begin{center}{\color{gregoriocolor} Octávo Idus Novémbris. 
 Luna\dots\ }\end{center}
\switchcolumn
\selectlanguage{english}
\begin{center}{\color{gregoriocolor} The Sixth Day of 
 November. The\dots\ Day of the Moon.}\end{center}
\end{paracol}

\noindent\begin{tabularx}{\linewidth}{*{19}{>{\centering\arraybackslash}X}}
 \textcolor{gregoriocolor}{a} & \textcolor{gregoriocolor}{b} & \textcolor{gregoriocolor}{c} & \textcolor{gregoriocolor}{d} & \textcolor{gregoriocolor}{e} & \textcolor{gregoriocolor}{f} & \textcolor{gregoriocolor}{g} & \textcolor{gregoriocolor}{h} & \textcolor{gregoriocolor}{i} & \textcolor{gregoriocolor}{k} & \textcolor{gregoriocolor}{l} & \textcolor{gregoriocolor}{m} & \textcolor{gregoriocolor}{n} & \textcolor{gregoriocolor}{p} & \textcolor{gregoriocolor}{q} & \textcolor{gregoriocolor}{r} & \textcolor{gregoriocolor}{s} & \textcolor{gregoriocolor}{t} & \textcolor{gregoriocolor}{u} \\
 16 & 17 & 18 & 19 & 20 & 21 & 22 & 23 & 24 & 25 & 26 & 27 & 28 & 29 & 30 & 1 & 2 & 3 & 4 \\
\end{tabularx}
\vspace{0.5\baselineskip}
\noindent\begin{tabularx}{\linewidth}{*{12}{>{\centering\arraybackslash}X}}
 \textcolor{gregoriocolor}{A} & \textcolor{gregoriocolor}{B} & \textcolor{gregoriocolor}{C} & \textcolor{gregoriocolor}{D} & \textcolor{gregoriocolor}{E} & F & \textcolor{gregoriocolor}{F} & \textcolor{gregoriocolor}{G} & \textcolor{gregoriocolor}{H} & \textcolor{gregoriocolor}{M} & \textcolor{gregoriocolor}{N} & \textcolor{gregoriocolor}{P} \\
 5 & 6 & 7 & 8 & 9 & 10 & 10 & 11 & 12 & 13 & 14 & 15 \\
\end{tabularx}

\begin{paracol}{2}
\selectlanguage{latin}
\lettrine[lines=2]{B}{arcinóne,} in Hispánia, 
 sancti Sevéri, Epíscopi et Mártyris; qui ob fidem cathólicam, confósso per 
 clavum cápite, martyrii corónam accépit.
\switchcolumn
\selectlanguage{english}
\lettrine[lines=2]{A}{t} Barcelona in Spain, St. Severus, 
 bishop and martyr, who had his head pierced with a spike, and thus received 
 the crown of martyrdom for the sake of the Catholic faith.
\switchcolumn*
\selectlanguage{latin}
Thiníssæ, in Africa, 
 natális sancti Felícis Mártyris, qui, conféssus et ad torménta dilátus, álio 
 die (ut refert sanctus Augustínus, Psalmum in ejus festivitáte ad pópulum 
 expónens) invéntus est in cárcere exánimis.
\switchcolumn
\selectlanguage{english}
At Tunis in Africa, the birthday of 
 St. Felix, martyr, who, having confessed Christ, was sent to prison. 
 His sentence had been deferred, but the next day he was found dead, as is 
 related by St. Augustine when he was expounding on a psalm to the people on 
 the feast of the saint.
\switchcolumn*
\selectlanguage{latin}
Theópoli, quæ est 
 Antiochía, sanctórum decem Mártyrum, qui a Saracénis passi legúntur.
\switchcolumn
\selectlanguage{english}
At Theopolis, which is Antioch, ten 
 holy martyrs who are said to have suffered at the hands of the Saracens.
\switchcolumn*
\selectlanguage{latin}
In Phrygia sancti 
 Attici Mártyris.
\switchcolumn
\selectlanguage{english}
In Phrygia, St. Atticus, martyr.
\switchcolumn*
\selectlanguage{latin}
Apud Bergas, in 
 Flándria, deposítio sancti Winóci Abbátis, qui, virtútibus et miráculis 
 clarus, étiam frátribus sibi súbditis multo témpore ministrávit.
\switchcolumn
\selectlanguage{english}
At Berg in Flanders, the death of 
 St. Winoc, abbot, who was renowned for virtues and miracles, and for a long 
 time was servant to the brethren subject to him.
\switchcolumn*
\selectlanguage{latin}
Fundis, in Látio, 
 sancti Felícis Mónachi.
\switchcolumn
\selectlanguage{english}
At Fondi in Lazio, St. Felix, monk.
\switchcolumn*
\selectlanguage{latin}
Lemóvicis, in Aquitánia, sancti Leonárdi Confessóris, qui fuit beáti Remígii Epíscopi discípulus. 
 Hic, nóbili génere ortus, solitáriam vitam delégit, et sanctitáte ac 
 miráculis cláruit; ejúsque virtus præcípue in liberándis captívis enítuit.
\switchcolumn
\selectlanguage{english}
At Limoges in Aquitaine, St. 
 Leonard, confessor, disciple of the blessed bishop Remigius, who was born of 
 a noble family but chose to lead a solitary life. He was celebrated 
 for holiness and miracles, but his virtue shone particularly in the 
 deliverance of captives.
\switchcolumn*
\selectlanguage{latin}
\end{paracol}


% ---- martyrology/mart11/mart1107.htm
\needspace{10\baselineskip}
\begin{paracol}{2}
\selectlanguage{latin}
\begin{center}{\color{gregoriocolor} Séptimo Idus Novémbris. 
 Luna\dots\ }\end{center}
\switchcolumn
\selectlanguage{english}
\begin{center}{\color{gregoriocolor} The Seventh Day of 
 November. The\dots\ Day of the Moon.}\end{center}
\end{paracol}

\noindent\begin{tabularx}{\linewidth}{*{19}{>{\centering\arraybackslash}X}}
 \textcolor{gregoriocolor}{a} & \textcolor{gregoriocolor}{b} & \textcolor{gregoriocolor}{c} & \textcolor{gregoriocolor}{d} & \textcolor{gregoriocolor}{e} & \textcolor{gregoriocolor}{f} & \textcolor{gregoriocolor}{g} & \textcolor{gregoriocolor}{h} & \textcolor{gregoriocolor}{i} & \textcolor{gregoriocolor}{k} & \textcolor{gregoriocolor}{l} & \textcolor{gregoriocolor}{m} & \textcolor{gregoriocolor}{n} & \textcolor{gregoriocolor}{p} & \textcolor{gregoriocolor}{q} & \textcolor{gregoriocolor}{r} & \textcolor{gregoriocolor}{s} & \textcolor{gregoriocolor}{t} & \textcolor{gregoriocolor}{u} \\
 17 & 18 & 19 & 20 & 21 & 22 & 23 & 24 & 25 & 26 & 27 & 28 & 29 & 30 & 1 & 2 & 3 & 4 & 5 \\
\end{tabularx}
\vspace{0.5\baselineskip}
\noindent\begin{tabularx}{\linewidth}{*{12}{>{\centering\arraybackslash}X}}
 \textcolor{gregoriocolor}{A} & \textcolor{gregoriocolor}{B} & \textcolor{gregoriocolor}{C} & \textcolor{gregoriocolor}{D} & \textcolor{gregoriocolor}{E} & F & \textcolor{gregoriocolor}{F} & \textcolor{gregoriocolor}{G} & \textcolor{gregoriocolor}{H} & \textcolor{gregoriocolor}{M} & \textcolor{gregoriocolor}{N} & \textcolor{gregoriocolor}{P} \\
 6 & 7 & 8 & 9 & 10 & 11 & 11 & 12 & 13 & 14 & 15 & 16 \\
\end{tabularx}

\begin{paracol}{2}
\selectlanguage{latin}
\lettrine[lines=2]{P}{atávii} deposítio 
 sancti Prosdócimi, qui fuit primus ejúsdem civitátis Epíscopus. Hic, a 
 beáto Petro Apóstolo Epíscopus ordinátus, ad prædicándum Dei verbum ad 
 prædíctam civitátem missus est; ibíque, multis virtútibus et prodígiis 
 corúscans, beáto fine quiévit.
\switchcolumn
\selectlanguage{english}
\lettrine[lines=2]{A}{t} Padua, the death of St. 
 Prosdocimus, consecrated as first bishop of that city by the blessed apostle 
 Peter. He was sent to that city to preach the word of God and there he 
 died a holy death, celebrated for many virtues and miracles.
\switchcolumn*
\selectlanguage{latin}
Perúsiæ sancti 
 Herculáni, Epíscopi et Mártyris.
\switchcolumn
\selectlanguage{english}
At Perugia, St. Herculanus, bishop 
 and martyr.
\switchcolumn*
\selectlanguage{latin}
Apud Swelménsem 
 civitátem, in Germánia, pássio sancti Engelbérti, Epíscopi Coloniénsis, qui, 
 cum illuc ex óppido Sosátio ad templum dedicándum pérgeret, a sicáriis 
 intercéptus in via multísque vulnéribus cæsus, gloriósum pro defensióne 
 ecclesiásticæ libertátis et Románæ Ecclésiæ obediéntia martyrium súbiit.
\switchcolumn
\selectlanguage{english}
At Schwelm in Germany, the martyrdom 
 of St. Engelbert, bishop of Cologne. He was on his way from that city 
 to the town of Essen in order to consecrate a church, when he was set upon 
 by ruffians on the road and slain by their many blows. Thus he 
 suffered martyrdom in defence of Church liberty and for obedience to the 
 Roman Church.
\switchcolumn*
\selectlanguage{latin}
Eódem die sancti 
 Amaránthi Mártyris, qui apud Albigénsem urbem, in Gállia, exácto agónis 
 fidélis cursu, sepúltus, vivit in glória.
\switchcolumn
\selectlanguage{english}
The same day, St. Amaranthus, 
 martyr. After successfully fulfilling the course of his trials he was 
 buried in the city of Albi, but lives in eternal glory.
\switchcolumn*
\selectlanguage{latin}
Melitínæ, in Arménia, pássio sanctórum Hierónis, Nicándri, Hesychii et aliórum trigínta; qui in 
 persecutióne Diocletiáni, sub Lysia Præside, coronáti sunt.
\switchcolumn
\selectlanguage{english}
At Melitina in Armenia, the 
 martyrdom of the Saints Hiero, Nicander, Hesychius, and thirty others, who 
 were crowned in the persecution of Diocletian under the governor Lysias.
\switchcolumn*
\selectlanguage{latin}
Amphípoli, in 
 Macedónia, sanctórum Mártyrum Aucti, Tauriónis et Thessalonícæ.
\switchcolumn
\selectlanguage{english}
At Amphipolis in Macedonia, the holy 
 martyrs Auctus, Taurio, and Thessalonica.
\switchcolumn*
\selectlanguage{latin}
Ancyræ, in Galátia, pássio sanctórum Melasíppi, Antónii et Carínæ, sub Juliáno Apóstata.
\switchcolumn
\selectlanguage{english}
At Ancyra in Galatia, the martyrdom 
 of Saints Melasippus, Anthony and Carina, under Julian the Apostate.
\switchcolumn*
\selectlanguage{latin}
Alexandríæ beáti 
 Achíllæ Epíscopi, qui eruditióne, fide, conversatióne ac móribus fuit 
 insígnis.
\switchcolumn
\selectlanguage{english}
At Alexandria, the blessed Achilles, 
 bishop, renowned for his learning, faith, and purity of life.
\switchcolumn*
\selectlanguage{latin}
In Frísia deposítio 
 sancti Willibrórdi, Epíscopi Trajecténsis; qui, a beáto Sérgio Papa ordinátus Epíscopus, in Frísia et Dánia Evangélium prædicávit.
\switchcolumn
\selectlanguage{english}
In Friesland, the death of St. 
 Willibrord, bishop of Utrecht, who was consecrated bishop by blessed Pope 
 Sergius, and preached the Gospel in Friesland and Denmark.
\switchcolumn*
\selectlanguage{latin}
Metis, in Gállia, 
 sancti Rufi, Epíscopi et Confessóris.
\switchcolumn
\selectlanguage{english}
At Metz, St. Rufus, bishop and 
 confessor.
\switchcolumn*
\selectlanguage{latin}
Argentoráti sancti 
 Floréntii Epíscopi.
\switchcolumn
\selectlanguage{english}
At Strasbourg, St. Florentius, 
 bishop.
\switchcolumn*
\selectlanguage{latin}
\end{paracol}


% ---- martyrology/mart11/mart1108.htm
\needspace{10\baselineskip}
\begin{paracol}{2}
\selectlanguage{latin}
\begin{center}{\color{gregoriocolor} Sexto Idus Novémbris. 
 Luna\dots\ }\end{center}
\switchcolumn
\selectlanguage{english}
\begin{center}{\color{gregoriocolor} The Eighth Day of 
 November. The\dots\ Day of the Moon.}\end{center}
\end{paracol}

\noindent\begin{tabularx}{\linewidth}{*{19}{>{\centering\arraybackslash}X}}
 \textcolor{gregoriocolor}{a} & \textcolor{gregoriocolor}{b} & \textcolor{gregoriocolor}{c} & \textcolor{gregoriocolor}{d} & \textcolor{gregoriocolor}{e} & \textcolor{gregoriocolor}{f} & \textcolor{gregoriocolor}{g} & \textcolor{gregoriocolor}{h} & \textcolor{gregoriocolor}{i} & \textcolor{gregoriocolor}{k} & \textcolor{gregoriocolor}{l} & \textcolor{gregoriocolor}{m} & \textcolor{gregoriocolor}{n} & \textcolor{gregoriocolor}{p} & \textcolor{gregoriocolor}{q} & \textcolor{gregoriocolor}{r} & \textcolor{gregoriocolor}{s} & \textcolor{gregoriocolor}{t} & \textcolor{gregoriocolor}{u} \\
 18 & 19 & 20 & 21 & 22 & 23 & 24 & 25 & 26 & 27 & 28 & 29 & 30 & 1 & 2 & 3 & 4 & 5 & 6 \\
\end{tabularx}
\vspace{0.5\baselineskip}
\noindent\begin{tabularx}{\linewidth}{*{12}{>{\centering\arraybackslash}X}}
 \textcolor{gregoriocolor}{A} & \textcolor{gregoriocolor}{B} & \textcolor{gregoriocolor}{C} & \textcolor{gregoriocolor}{D} & \textcolor{gregoriocolor}{E} & F & \textcolor{gregoriocolor}{F} & \textcolor{gregoriocolor}{G} & \textcolor{gregoriocolor}{H} & \textcolor{gregoriocolor}{M} & \textcolor{gregoriocolor}{N} & \textcolor{gregoriocolor}{P} \\
 7 & 8 & 9 & 10 & 11 & 12 & 12 & 13 & 14 & 15 & 16 & 17 \\
\end{tabularx}

\begin{paracol}{2}
\selectlanguage{latin}
\lettrine[lines=1]{O}{ctáva} ómnium 
 Sanctórum.
\switchcolumn
\selectlanguage{english}
\lettrine[lines=1]{T}{he} Octave of All Saints.
\switchcolumn*
\selectlanguage{latin}
Romæ, via Lavicána, 
 tértio ab Urbe milliário, pássio sanctórum Mártyrum Cláudii, Nicóstrati, 
 Symphoriáni, Castórii et Simplícii, qui, primo in cárcerem missi, deínde 
 scorpiónibus gravíssime cæsi, tandem, cum ex fide Christi dimovéri non 
 possent, a Diocletiáno jussi sunt in flúvium præcípites dari.
\switchcolumn
\selectlanguage{english}
At Rome, on the Lavican Way, three 
 miles from the city, the martyrdom of the Saints Claudius, Nicostratus, 
 Symphorian, Castorius, and Simplicius. They were first sent to prison, 
 then scourged with whips set with metal, but since they could not be made to 
 forsake the faith of Christ, Diocletian ordered them to be thrown into the 
 river.
\switchcolumn*
\selectlanguage{latin}
Ibídem, via Lavicána, 
 natális sanctórum Quátuor Coronatórum fratrum, id est Sevéri, Severiáni, 
 Carpóphori et Victoríni; qui, sub eódem Imperatóre, íctibus plumbatárum 
 usque ad mortem cæsi sunt. Horum autem nómina, quæ póstea, interjéctis 
 annis, Dómino revelánte, osténsa sunt, cum mínime reperíri tunc potuíssent, 
 statútum fuit ut anniversária dies ipsórum, una cum illis quinque, sub 
 nómine sanctórum Quátuor Coronatórum recolerétur; qui mos, étiam postquam 
 reveláta sunt, in Ecclésia perseverávit.
\switchcolumn
\selectlanguage{english}
Also, on the Lavican Way, the 
 birthday of the saintly brothers, Severus, Severian, Carpophorus, and 
 Victorinus, called the Four Crowned, who were scourged to death with leaded 
 whips, during the reign of the same emperor. Because their names, 
 known some years afterwards by revelation, could not then be ascertained, it 
 was ordered that their anniversary should be commemorated with the preceding 
 five, under the name of the Four Saints Crowned. This custom was 
 retained by the Church, even after their names had been revealed.
\switchcolumn*
\selectlanguage{latin}
Item Romæ sancti 
 Deúsdedit Papæ Primi, qui tanti mériti fuit, ut leprósum ósculo a lepra 
 sanáverit.
\switchcolumn
\selectlanguage{english}
Also at Rome, St. Deusdedit, pope, 
 whose merit was so great that he cured a leper by kissing him.
\switchcolumn*
\selectlanguage{latin}
In vico Blexen, ad 
 Visúrgim flúvium, in Germánia, sancti Willehádi, qui primus éxstitit 
 Breménsis civitátis Epíscopus; atque, una cum sancto Bonifátio, cujus 
 discípulus fuit, in Frísia et Saxónia Evangélium propagávit.
\switchcolumn
\selectlanguage{english}
In the village of Plexem, on the 
 Weser River in Germany, St. Willehad, first bishop of Bremen, who, together 
 with St. Boniface, whose disciple he was, spread the Gospel in Friesland and 
 Saxony.
\switchcolumn*
\selectlanguage{latin}
Suessíone, in Gálliis, 
 sancti Godefrídi, Ambianénsis Epíscopi, magnæ sanctitátis viri.
\switchcolumn
\selectlanguage{english}
At Soissons in France, St. Godfrey, 
 bishop of Amiens, a man of great sanctity.
\switchcolumn*
\selectlanguage{latin}
Apud Virodúnum, in 
 Gállia, sancti Mauri, Epíscopi et Confessóris.
\switchcolumn
\selectlanguage{english}
At Verdun in France, St. Maur, 
 bishop and confessor.
\switchcolumn*
\selectlanguage{latin}
Turónis, in Gállia, 
 sancti Clári Presbyteri, cujus sanctus Paulínus epitáphium scripsit.
\switchcolumn
\selectlanguage{english}
At Tours in France, St. Clarus, a 
 priest whose epitaph was written by St. Paulinus.
\switchcolumn*
\selectlanguage{latin}
\end{paracol}


% ---- martyrology/mart11/mart1109.htm
\needspace{10\baselineskip}
\begin{paracol}{2}
\selectlanguage{latin}
\begin{center}{\color{gregoriocolor} Quinto Idus Novémbris. 
 Luna\dots\ }\end{center}
\switchcolumn
\selectlanguage{english}
\begin{center}{\color{gregoriocolor} The Ninth Day of 
 November. The\dots\ Day of the Moon.}\end{center}
\end{paracol}

\noindent\begin{tabularx}{\linewidth}{*{19}{>{\centering\arraybackslash}X}}
 \textcolor{gregoriocolor}{a} & \textcolor{gregoriocolor}{b} & \textcolor{gregoriocolor}{c} & \textcolor{gregoriocolor}{d} & \textcolor{gregoriocolor}{e} & \textcolor{gregoriocolor}{f} & \textcolor{gregoriocolor}{g} & \textcolor{gregoriocolor}{h} & \textcolor{gregoriocolor}{i} & \textcolor{gregoriocolor}{k} & \textcolor{gregoriocolor}{l} & \textcolor{gregoriocolor}{m} & \textcolor{gregoriocolor}{n} & \textcolor{gregoriocolor}{p} & \textcolor{gregoriocolor}{q} & \textcolor{gregoriocolor}{r} & \textcolor{gregoriocolor}{s} & \textcolor{gregoriocolor}{t} & \textcolor{gregoriocolor}{u} \\
 19 & 20 & 21 & 22 & 23 & 24 & 25 & 26 & 27 & 28 & 29 & 30 & 1 & 2 & 3 & 4 & 5 & 6 & 7 \\
\end{tabularx}
\vspace{0.5\baselineskip}
\noindent\begin{tabularx}{\linewidth}{*{12}{>{\centering\arraybackslash}X}}
 \textcolor{gregoriocolor}{A} & \textcolor{gregoriocolor}{B} & \textcolor{gregoriocolor}{C} & \textcolor{gregoriocolor}{D} & \textcolor{gregoriocolor}{E} & F & \textcolor{gregoriocolor}{F} & \textcolor{gregoriocolor}{G} & \textcolor{gregoriocolor}{H} & \textcolor{gregoriocolor}{M} & \textcolor{gregoriocolor}{N} & \textcolor{gregoriocolor}{P} \\
 8 & 9 & 10 & 11 & 12 & 13 & 13 & 14 & 15 & 16 & 17 & 18 \\
\end{tabularx}

\begin{paracol}{2}
\selectlanguage{latin}
\lettrine[lines=2]{R}{omæ,} in Lateráno, 
 Dedicátio Basílicæ sanctíssimi Salvatóris, quæ ómnium Urbis et Orbis 
 Ecclesiárum est mater et caput.
\switchcolumn
\selectlanguage{english}
\lettrine[lines=2]{A}{t} Rome in the Lateran, the 
 Dedication of the Basilica of the Saviour, which is the Mother and Head of 
 all churches in the city and the world.
\switchcolumn*
\selectlanguage{latin}
Amaséæ, in Ponto, 
 natális sancti Theodóri mílitis, qui, témpore Maximiáni Imperatóris, pro 
 Christiánæ fídei confessióne, fórtiter cæsus et in cárcerem missus; deínde, 
 Dómino sibi apparénte ac monénte ut constánter et viríliter ágeret, 
 relevátus est; novíssime, postquam in equúleo suspénsus et úngulis ita 
 excarnificátus est, ut ejus interióra apparérent nuda, ardéntibus ígnibus 
 comburéndus tráditur. Ipsíus vero laudes sanctus Gregórius Nyssénus 
 præcláro encómio celebrávit.
\switchcolumn
\selectlanguage{english}
At Amasea in Pontus, the birthday of 
 St. Theodore, a soldier, in the time of Emperor Maximian. For the 
 confession of Christ he was severely scourged and sent to prison, where he 
 was comforted by an apparition of our Lord, who exhorted him to act with 
 courage and constancy. He was finally stretched on the rack, lacerated 
 with iron hooks until his bowels were laid bare, then cast into the flames 
 to be burned alive. His glorious deeds have been celebrated in a 
 eulogy by Gregory of Nyssa.
\switchcolumn*
\selectlanguage{latin}
Tyánæ, in Cappadócia, pássio sancti Oréstis, sub Diocletiáno Imperatóre.
\switchcolumn
\selectlanguage{english}
At Tyana in Cappadocia, the 
 martyrdom of St. Orestes under Emperor Diocletian.
\switchcolumn*
\selectlanguage{latin}
Thessalonícæ sancti 
 Alexándri Mártyris, sub Maximiáno Príncipe.
\switchcolumn
\selectlanguage{english}
At Thessalonica, under Emperor 
 Maximian, St. Alexander, martyr.
\switchcolumn*
\selectlanguage{latin}
Apud Bitúricas, in 
 Aquitánia, sancti Ursíni Confessóris, qui, Romæ ordinátus a successóribus 
 Apostolórum, primus eídem Bituricénsi urbi destinátur Epíscopus.
\switchcolumn
\selectlanguage{english}
At Bourges in Aquitaine, St. Ursinus, 
 confessor, who was ordained at Rome by the successors of the apostles and 
 appointed first bishop of that city.
\switchcolumn*
\selectlanguage{latin}
Neápoli, in Campánia, sancti Agrippíni Epíscopi, miráculis clari.
\switchcolumn
\selectlanguage{english}
At Naples in Campania, St. 
 Agrippinus, bishop, renowned for miracles.
\switchcolumn*
\selectlanguage{latin}
Constantinópoli 
 sanctárum Vírginum Eustóliæ Románæ, et Sópatræ, fíliæ Maurítii Imperatóris.
\switchcolumn
\selectlanguage{english}
At Constantinople, the holy virgins 
 Eustolia, a Roman maiden, and Sopatra, the daughter of Emperor Maurice
\switchcolumn*
\selectlanguage{latin}
Beryti, in Syria, 
 commemorátio Imáginis Salvatóris, quæ, a Judæis crucifíxa, tam copiósum 
 emísit sánguinem, ut Orientáles et Occidentáles Ecclésiæ ex eo ubértim 
 accéperint.
\switchcolumn
\selectlanguage{english}
At Berytus in Syria, the 
 Commemoration of the Image of our Saviour, which, being fastened to a cross 
 by the Jews, poured out blood so plentifully that the Eastern and Western 
 Churches received abundantly of it.
\switchcolumn*
\selectlanguage{latin}
\end{paracol}


% ---- martyrology/mart11/mart1110.htm
\needspace{10\baselineskip}
\begin{paracol}{2}
\selectlanguage{latin}
\begin{center}{\color{gregoriocolor} Quarto Idus Novémbris. 
 Luna\dots\ }\end{center}
\switchcolumn
\selectlanguage{english}
\begin{center}{\color{gregoriocolor} The Tenth Day of 
 November. The\dots\ Day of the Moon.}\end{center}
\end{paracol}

\noindent\begin{tabularx}{\linewidth}{*{19}{>{\centering\arraybackslash}X}}
 \textcolor{gregoriocolor}{a} & \textcolor{gregoriocolor}{b} & \textcolor{gregoriocolor}{c} & \textcolor{gregoriocolor}{d} & \textcolor{gregoriocolor}{e} & \textcolor{gregoriocolor}{f} & \textcolor{gregoriocolor}{g} & \textcolor{gregoriocolor}{h} & \textcolor{gregoriocolor}{i} & \textcolor{gregoriocolor}{k} & \textcolor{gregoriocolor}{l} & \textcolor{gregoriocolor}{m} & \textcolor{gregoriocolor}{n} & \textcolor{gregoriocolor}{p} & \textcolor{gregoriocolor}{q} & \textcolor{gregoriocolor}{r} & \textcolor{gregoriocolor}{s} & \textcolor{gregoriocolor}{t} & \textcolor{gregoriocolor}{u} \\
 20 & 21 & 22 & 23 & 24 & 25 & 26 & 27 & 28 & 29 & 30 & 1 & 2 & 3 & 4 & 5 & 6 & 7 & 8 \\
\end{tabularx}
\vspace{0.5\baselineskip}
\noindent\begin{tabularx}{\linewidth}{*{12}{>{\centering\arraybackslash}X}}
 \textcolor{gregoriocolor}{A} & \textcolor{gregoriocolor}{B} & \textcolor{gregoriocolor}{C} & \textcolor{gregoriocolor}{D} & \textcolor{gregoriocolor}{E} & F & \textcolor{gregoriocolor}{F} & \textcolor{gregoriocolor}{G} & \textcolor{gregoriocolor}{H} & \textcolor{gregoriocolor}{M} & \textcolor{gregoriocolor}{N} & \textcolor{gregoriocolor}{P} \\
 9 & 10 & 11 & 12 & 13 & 14 & 14 & 15 & 16 & 17 & 18 & 19 \\
\end{tabularx}

\begin{paracol}{2}
\selectlanguage{latin}
\lettrine[lines=2]{N}{eápoli,} in Campánia, natális sancti Andréæ Avellini, Clérici Reguláris et Confessóris, sanctitáte 
 et salútis proximórum procurándæ stúdio præcélebris, quem, miráculis clarum, 
 Clemens Undécimus, Póntifex Máximus, Sanctórum catálogo adscrípsit.
\switchcolumn
\selectlanguage{english}
\lettrine[lines=2]{A}{t} Naples in Campania, the birthday 
 of St. Andrew Avellini, Cleric Regular and confessor, celebrated for his 
 sanctity, his zeal in procuring the salvation of souls, and renowned for his 
 miracles. He was inscribed on the catalogue of the Saints by Pope 
 Clement XI.
\switchcolumn*
\selectlanguage{latin}
Eódem die natális 
 quoque sanctórum Mártyrum Tryphónis, et Respícii, ac Nymphæ Vírginis.
\switchcolumn
\selectlanguage{english}
On the same day, the birthday of the 
 holy martyrs Trypho and Respicius, and the virgin Nympha.
\switchcolumn*
\selectlanguage{latin}
Romæ item natális 
 sancti Leónis Papæ Primi, Confessóris et Ecclésiæ Doctóris; qui, virtútum 
 excéllens méritis, dictus est Magnus. Ejus tempóribus celebráta fuit 
 sancta Synodus Chalcedonénsis, in qua ipse per legátos damnávit Eutychen; 
 cujus étiam Synodi decréta póstmodum auctoritáte sua confirmávit. 
 Tandem, cum sanxísset multa luculentérque scripsísset, Pastor bonus, de 
 sancta Dei Ecclésia et univérso grege Domínico óptime méritus, quiévit in 
 pace. Ipsíus tamen festívitas tértio Idus Aprílis celebrátur.
\switchcolumn
\selectlanguage{english}
At Rome, Pope St. Leo I, confessor 
 and doctor of the Church, surnamed the Great because of his extraordinary 
 merits. During his pontificate the holy Council of Chalcedon was held 
 which condemned Eutyches through his legates, and whose decrees were 
 afterwards given the seal of his authority. After meriting the 
 gratitude of the Church of God and the whole flock of Christ by the many 
 decrees which he issued, and by the many excellent works which he wrote, 
 this good and zealous shepherd rested in peace. His feast is 
 celebrated on the 11th of April.
\switchcolumn*
\selectlanguage{latin}
Icónii, in Lycaónia, 
 sanctárum mulíerum Tryphénnæ et Tryphósæ, quæ, beáti Pauli prædicatióne et 
 exémplo Theclæ, in Christiána disciplína plúrimum profecérunt.
\switchcolumn
\selectlanguage{english}
At Iconium in Lycaonia, the holy 
 women Tryphenna and Tryphosa, who profited by the preaching of blessed Paul 
 and the example of Thecla to make great progress in Christian perfection.
\switchcolumn*
\selectlanguage{latin}
Antiochíæ sanctórum 
 Demétrii Epíscopi, Aniáni Diáconi, Eustósii et aliórum vigínti Mártyrum.
\switchcolumn
\selectlanguage{english}
At Antioch, Saints Demetrius, 
 bishop, Anian, deacon, Eustosius, and twenty other martyrs.
\switchcolumn*
\selectlanguage{latin}
In território Agathénsi, 
 in Gállia, sanctórum Mártyrum Tibérii, Modésti et Floréntiæ; qui, témpore 
 Diocletiáni, váriis torméntis cruciáti, martyrium complevérunt.
\switchcolumn
\selectlanguage{english}
In the diocese of Agde in France, 
 the holy martyrs Tiberius, Modestus, and Florence, who were subjected to 
 diverse torments and fulfilled their martyrdom in the time of Diocletian.
\switchcolumn*
\selectlanguage{latin}
Ravénnæ sancti Probi 
 Epíscopi, miráculis clari.
\switchcolumn
\selectlanguage{english}
At Ravenna, St. Probus, a bishop 
 renowned for miracles.
\switchcolumn*
\selectlanguage{latin}
Aureliánis, in Gállia, 
 sancti Monitóris, Epíscopi et Confessóris.
\switchcolumn
\selectlanguage{english}
At Orleans in France, St. Monitor, 
 bishop and confessor.
\switchcolumn*
\selectlanguage{latin}
In Anglia sancti Justi 
 Epíscopi, qui, una cum Augustíno, Mellíto et áliis a beáto Gregório Papa in 
 eam ínsulam missus ad prædicándum Evangélium, ibídem, sanctitáte célebris, 
 obdormívit in Dómino.
\switchcolumn
\selectlanguage{english}
In England, St. Justus, bishop, who 
 was sent by Pope Gregory with Augustine, Mellitus, and others to preach the 
 Gospel in that country. There he went to repose in the Lord, 
 celebrated for his sanctity.
\switchcolumn*
\selectlanguage{latin}
In óppido Milledúno, in 
 Gállia, sancti Leónis Confessóris.
\switchcolumn
\selectlanguage{english}
In the town of Melun in France, St. 
 Leo, confessor.
\switchcolumn*
\selectlanguage{latin}
In Paro ínsula sanctæ 
 Theoctístis Vírginis.
\switchcolumn
\selectlanguage{english}
In the island of Paros, St. 
 Theoctistis, virgin.
\switchcolumn*
\selectlanguage{latin}
\end{paracol}


% ---- martyrology/mart11/mart1111.htm
\needspace{10\baselineskip}
\begin{paracol}{2}
\selectlanguage{latin}
\begin{center}{\color{gregoriocolor} Tértio Idus Novémbris. 
 Luna\dots\ }\end{center}
\switchcolumn
\selectlanguage{english}
\begin{center}{\color{gregoriocolor} The 
 Eleventh Day of 
 November. The\dots\ Day of the Moon.}\end{center}
\end{paracol}

\noindent\begin{tabularx}{\linewidth}{*{19}{>{\centering\arraybackslash}X}}
 \textcolor{gregoriocolor}{a} & \textcolor{gregoriocolor}{b} & \textcolor{gregoriocolor}{c} & \textcolor{gregoriocolor}{d} & \textcolor{gregoriocolor}{e} & \textcolor{gregoriocolor}{f} & \textcolor{gregoriocolor}{g} & \textcolor{gregoriocolor}{h} & \textcolor{gregoriocolor}{i} & \textcolor{gregoriocolor}{k} & \textcolor{gregoriocolor}{l} & \textcolor{gregoriocolor}{m} & \textcolor{gregoriocolor}{n} & \textcolor{gregoriocolor}{p} & \textcolor{gregoriocolor}{q} & \textcolor{gregoriocolor}{r} & \textcolor{gregoriocolor}{s} & \textcolor{gregoriocolor}{t} & \textcolor{gregoriocolor}{u} \\
 21 & 22 & 23 & 24 & 25 & 26 & 27 & 28 & 29 & 30 & 1 & 2 & 3 & 4 & 5 & 6 & 7 & 8 & 9 \\
\end{tabularx}
\vspace{0.5\baselineskip}
\noindent\begin{tabularx}{\linewidth}{*{12}{>{\centering\arraybackslash}X}}
 \textcolor{gregoriocolor}{A} & \textcolor{gregoriocolor}{B} & \textcolor{gregoriocolor}{C} & \textcolor{gregoriocolor}{D} & \textcolor{gregoriocolor}{E} & F & \textcolor{gregoriocolor}{F} & \textcolor{gregoriocolor}{G} & \textcolor{gregoriocolor}{H} & \textcolor{gregoriocolor}{M} & \textcolor{gregoriocolor}{N} & \textcolor{gregoriocolor}{P} \\
 10 & 11 & 12 & 13 & 14 & 15 & 15 & 16 & 17 & 18 & 19 & 20 \\
\end{tabularx}

\begin{paracol}{2}
\selectlanguage{latin}
\lettrine[lines=2]{T}{urónis,} in Gállia, 
 natális beáti Martíni Epíscopi et Confessóris; cujus vita tantis éxstitit 
 miráculis gloriósa, ut trium mortuórum suscitátor esse merúerit.
\switchcolumn
\selectlanguage{english}
\lettrine[lines=2]{A}{t} Tours in France, the birthday of 
 blessed Martin, bishop and confessor, whose life was so renowned for 
 miracles that he received the power to raise three persons from the dead.
\switchcolumn*
\selectlanguage{latin}
Cotyǽi, in Phrygia, 
 insígnis pássio sancti Mennæ, Ægyptii mílitis, qui in persecutióne 
 Diocletiáni, postquam, abjécto milítiæ cíngulo, méruit cælésti Regi secréta 
 conversatióne in erémo militáre, procéssit in públicum, et, se Christiánum 
 líbera voce declárans, primo diris cruciátibus examinátur; novíssime, fixis 
 in oratióne génibus, Dómino Jesu Christo grátias agens, gládio cæsus est, ac 
 multis post mortem miráculis cláruit.
\switchcolumn
\selectlanguage{english}
At Cotyaeum in Phrygia, during the 
 persecution of Diocletian, the celebrated martyrdom of St. Mennas, an 
 Egyptian soldier, who cast off the military belt and obtained the grace of 
 serving the King of heaven secretly in the desert. Afterwards, coming 
 out publicly and freely declaring himself a Christian, he was first 
 subjected to severe torments; and finally kneeling in prayer, giving thanks 
 to our Lord Jesus Christ, he was slain with the sword. After his death 
 he became renowned for many miracles.
\switchcolumn*
\selectlanguage{latin}
Ravénnæ sanctórum 
 Mártyrum Valentíni, Feliciáni et Victoríni; qui in Diocletiáni persecutióne 
 coronáti sunt.
\switchcolumn
\selectlanguage{english}
At Ravenna, the holy martyrs 
 Valentine, Felician, and Victorinus, who were crowned during the persecution 
 of Diocletian.
\switchcolumn*
\selectlanguage{latin}
In Mesopotámia sancti 
 Athenodóri Mártyris, qui, sub eódem Diocletiáno et Eléusio Præside, ígnibus 
 cruciátus et áliis supplíciis tortus, demum cápitis damnátus est, et, cum 
 cárnifex corruísset neque ullus álius gládio illum feríre ausus esset, orans 
 obdormívit in Dómino.
\switchcolumn
\selectlanguage{english}
In Mesopotamia, St. Athenodorus, 
 martyr, who was subjected to fire and other torments under the same 
 Diocletian and the governor Eleusius. He was at length sentenced to be 
 beheaded, but when the executioner fell to the ground and no other person 
 would dare to strike him with the sword, he passed to his repose in the Lord 
 while praying.
\switchcolumn*
\selectlanguage{latin}
Lugdúni, in Gállia, 
 sancti Veráni Epíscopi, cujus vita fuit fide et virtútum méritis illústris.
\switchcolumn
\selectlanguage{english}
At Lyons in France, St. Veranus, 
 bishop, whose life was illustrious for his faith and his other virtues.
\switchcolumn*
\selectlanguage{latin}
Constantinópoli sancti 
 Theodóri, Abbátis Studítæ, qui, pro fide cathólica advérsus Iconoclástas 
 strénue pugnans, factus est apud univérsam Ecclésiam cathólicam célebris.
\switchcolumn
\selectlanguage{english}
At Constantinople, St. Theodore, 
 abbot of Studium, who fought valiantly for the Catholic faith against the 
 Iconoclasts, and became famed throughout the universal Church.
\switchcolumn*
\selectlanguage{latin}
In monastério Cryptæ 
 Ferrátæ, in agro Tusculáno, sancti Bartholomæi Abbátis, qui fuit sócius 
 beáti Nili, ejúsque vitam conscrípsit.
\switchcolumn
\selectlanguage{english}
In the monastery of Grottaferrata, 
 in the Tuscan plain, the holy abbot Bartholomew, a companion of blessed 
 Nilus, whose life he wrote.
\switchcolumn*
\selectlanguage{latin}
In província Sámnii 
 beáti Mennæ solitárii, cujus virtútes et mirácula sanctus Gregórius Papa 
 commémorat.
\switchcolumn
\selectlanguage{english}
In the province of Abruzzi, blessed 
 Mennas, a solitary whose virtues and miracles are mentioned by Pope St. 
 Gregory.
\switchcolumn*
\selectlanguage{latin}
\end{paracol}


% ---- martyrology/mart11/mart1112.htm
\needspace{10\baselineskip}
\begin{paracol}{2}
\selectlanguage{latin}
\begin{center}{\color{gregoriocolor} Prídie Idus Novémbris. 
 Luna\dots\ }\end{center}
\switchcolumn
\selectlanguage{english}
\begin{center}{\color{gregoriocolor} The 
 Twelfth Day of 
 November. The\dots\ Day of the Moon.}\end{center}
\end{paracol}

\noindent\begin{tabularx}{\linewidth}{*{19}{>{\centering\arraybackslash}X}}
 \textcolor{gregoriocolor}{a} & \textcolor{gregoriocolor}{b} & \textcolor{gregoriocolor}{c} & \textcolor{gregoriocolor}{d} & \textcolor{gregoriocolor}{e} & \textcolor{gregoriocolor}{f} & \textcolor{gregoriocolor}{g} & \textcolor{gregoriocolor}{h} & \textcolor{gregoriocolor}{i} & \textcolor{gregoriocolor}{k} & \textcolor{gregoriocolor}{l} & \textcolor{gregoriocolor}{m} & \textcolor{gregoriocolor}{n} & \textcolor{gregoriocolor}{p} & \textcolor{gregoriocolor}{q} & \textcolor{gregoriocolor}{r} & \textcolor{gregoriocolor}{s} & \textcolor{gregoriocolor}{t} & \textcolor{gregoriocolor}{u} \\
 22 & 23 & 24 & 25 & 26 & 27 & 28 & 29 & 30 & 1 & 2 & 3 & 4 & 5 & 6 & 7 & 8 & 9 & 10 \\
\end{tabularx}
\vspace{0.5\baselineskip}
\noindent\begin{tabularx}{\linewidth}{*{12}{>{\centering\arraybackslash}X}}
 \textcolor{gregoriocolor}{A} & \textcolor{gregoriocolor}{B} & \textcolor{gregoriocolor}{C} & \textcolor{gregoriocolor}{D} & \textcolor{gregoriocolor}{E} & F & \textcolor{gregoriocolor}{F} & \textcolor{gregoriocolor}{G} & \textcolor{gregoriocolor}{H} & \textcolor{gregoriocolor}{M} & \textcolor{gregoriocolor}{N} & \textcolor{gregoriocolor}{P} \\
 11 & 12 & 13 & 14 & 15 & 16 & 16 & 17 & 18 & 19 & 20 & 21 \\
\end{tabularx}

\begin{paracol}{2}
\selectlanguage{latin}
\lettrine[lines=2]{S}{ancti} Martíni Primi, 
 Papæ et Mártyris, cujus dies natális sextodécimo Kaléndas Octóbris 
 recensétur.
\switchcolumn
\selectlanguage{english}
\lettrine[lines=2]{T}{he} Feast of St. Martin I, pope and 
 martyr, whose birthday is mentioned on the 16th day of September.
\switchcolumn*
\selectlanguage{latin}
Vitépsci, in Polónia, 
 pássio sancti Jósaphat, e sancti Basilíi Ordine, Epíscopi Polocénsis et 
 Mártyris; qui a schismáticis, in ódium cathólicæ unitátis et veritátis, 
 crudéliter interféctus est, et a Pio Papa Nono inter sanctos Mártyres 
 adscríptus. Ejus tamen festívitas recólitur décimo octávo Kaléndas 
 Decémbris.
\switchcolumn
\selectlanguage{english}
At Witebsk in Poland, the martyrdom 
 of St. Josaphat, of the Order of St. Basil, a Polish archbishop and martyr, 
 who was cruelly slain by schismatics through hatred of Catholic unity and 
 truth. He was canonized by Pope Pius IX, and his feast is observed on 
 the 16th of November.
\switchcolumn*
\selectlanguage{latin}
Complúti, in Hispánia, 
 natális sancti Dídaci Confessóris, ex Ordine Minórum, humilitáte célebris; 
 quem Xystus Quintus, Póntifex Máximus, Sanctórum catálogo adscrípsit. 
 Ipsíus autem festum sequénti die celebrátur.
\switchcolumn
\selectlanguage{english}
At Alcala in Spain, the birthday of 
 St. Didacus, confessor, a member of the Order of Friars Minor well known for 
 his humility. Pope Sixtus V included him in the catalogue of the 
 saints and his feast is celebrated on the day following.
\switchcolumn*
\selectlanguage{latin}
In Asia pássio 
 sanctórum Aurélii et Públii Episcopórum.
\switchcolumn
\selectlanguage{english}
In Asia, the martyrdom of the holy 
 bishops Aurelius and Publius.
\switchcolumn*
\selectlanguage{latin}
Eschæ, in Bélgio, 
 sancti Livíni, Epíscopi et Mártyris; qui, cum plúrimos ad Christi fidem 
 convertísset, a Pagánis necátus est. Ipsíus vero corpus ad Portum 
 Gandæ póstea translátum fuit.
\switchcolumn
\selectlanguage{english}
At Eschen in Belgium, St. Livinus, 
 bishop and martyr. After converting many to the faith he was slain by 
 heathens. His body, however, was afterwards translated to Ghent.
\switchcolumn*
\selectlanguage{latin}
Apud Casimíriam, in 
 Polónia, sanctórum Mártyrum Eremitárum Benedícti, Joánnis, Matthæi, Isaac et 
 Christiáni; qui a prædónibus, divíno inténti servítio, dire vexáti sunt et 
 gládiis occísi.
\switchcolumn
\selectlanguage{english}
At Gnesen in Poland, the holy 
 hermits and martyrs Benedict, John, Matthew, Isaac, and Christian. 
 They were savagely attacked by robbers and slain by the sword while there 
 were at prayer.
\switchcolumn*
\selectlanguage{latin}
Apud óppidum Sergíniam, 
 in território Senonénsi, sancti Patérni, Mónachi et Mártyris; qui, dum eídem 
 occurréntes in ipsíus óppidi silva latrónes ad emendándam vitam incitáret, 
 ab illis trucidátus est.
\switchcolumn
\selectlanguage{english}
In the neighbourhood of Sens, St. 
 Paternus, monk and martyr. He had met some robbers in a nearby forest, 
 and for attempting to persuade them to amend their lives, they slew him.
\switchcolumn*
\selectlanguage{latin}
Avenióne sancti Rufi, 
 qui éxstitit primus ejúsdem civitátis Epíscopus.
\switchcolumn
\selectlanguage{english}
At Avignon, St. Rufus, the first 
 bishop of that city.
\switchcolumn*
\selectlanguage{latin}
Colóniæ Agrippínæ 
 deposítio sancti Cunibérti Epíscopi.
\switchcolumn
\selectlanguage{english}
At Cologne, the death of St. 
 Cunibert, bishop.
\switchcolumn*
\selectlanguage{latin}
Turiasóne, in Hispánia 
 Tarraconénsi, beáti Æmiliáni Presbyteri, qui innúmeris miráculis cláruit; 
 cujus admirábilem vitam sanctus Bráulio, Cæsaraugustánus Epíscopus, 
 descrípsit.
\switchcolumn
\selectlanguage{english}
At Tarazona in Aragon, blessed 
 Emilian, a priest favoured with many miracles. His admirable life was 
 recorded by St. Braulio, bishop of Saragossa.
\switchcolumn*
\selectlanguage{latin}
Constantinópoli sancti 
 Nili Abbátis, qui, sub Theodósio junióre, ex Præfécto ejúsdem civitátis 
 factus Mónachus, doctrína et sanctitáte cláruit.
\switchcolumn
\selectlanguage{english}
At Constantinople, St. Nilus, abbot, 
 who resigned as prefect of the city to become a monk, and during the reign 
 of Theodosius the Younger became distinguished for his learning and 
 sanctity.
\switchcolumn*
\selectlanguage{latin}
\end{paracol}


% ---- martyrology/mart11/mart1113.htm
\needspace{10\baselineskip}
\begin{paracol}{2}
\selectlanguage{latin}
\begin{center}{\color{gregoriocolor} Idibus Novémbris. 
 Luna\dots\ }\end{center}
\switchcolumn
\selectlanguage{english}
\begin{center}{\color{gregoriocolor} The 
 Thirteenth Day of 
 November. The\dots\ Day of the Moon.}\end{center}
\end{paracol}

\noindent\begin{tabularx}{\linewidth}{*{19}{>{\centering\arraybackslash}X}}
 \textcolor{gregoriocolor}{a} & \textcolor{gregoriocolor}{b} & \textcolor{gregoriocolor}{c} & \textcolor{gregoriocolor}{d} & \textcolor{gregoriocolor}{e} & \textcolor{gregoriocolor}{f} & \textcolor{gregoriocolor}{g} & \textcolor{gregoriocolor}{h} & \textcolor{gregoriocolor}{i} & \textcolor{gregoriocolor}{k} & \textcolor{gregoriocolor}{l} & \textcolor{gregoriocolor}{m} & \textcolor{gregoriocolor}{n} & \textcolor{gregoriocolor}{p} & \textcolor{gregoriocolor}{q} & \textcolor{gregoriocolor}{r} & \textcolor{gregoriocolor}{s} & \textcolor{gregoriocolor}{t} & \textcolor{gregoriocolor}{u} \\
 23 & 24 & 25 & 26 & 27 & 28 & 29 & 30 & 1 & 2 & 3 & 4 & 5 & 6 & 7 & 8 & 9 & 10 & 11 \\
\end{tabularx}
\vspace{0.5\baselineskip}
\noindent\begin{tabularx}{\linewidth}{*{12}{>{\centering\arraybackslash}X}}
 \textcolor{gregoriocolor}{A} & \textcolor{gregoriocolor}{B} & \textcolor{gregoriocolor}{C} & \textcolor{gregoriocolor}{D} & \textcolor{gregoriocolor}{E} & F & \textcolor{gregoriocolor}{F} & \textcolor{gregoriocolor}{G} & \textcolor{gregoriocolor}{H} & \textcolor{gregoriocolor}{M} & \textcolor{gregoriocolor}{N} & \textcolor{gregoriocolor}{P} \\
 12 & 13 & 14 & 15 & 16 & 17 & 17 & 18 & 19 & 20 & 21 & 22 \\
\end{tabularx}

\begin{paracol}{2}
\selectlanguage{latin}
\lettrine[lines=2]{S}{ancti} Dídaci, ex 
 Ordine Minórum, Confessóris; cujus dies natális recólitur prídie hujus diéi.
\switchcolumn
\selectlanguage{english}
\lettrine[lines=2]{S}{t.} Didacus, confessor of the Order 
 of Friars Minor, whose birthday occurred on the preceding day.
\switchcolumn*
\selectlanguage{latin}
Ravénnæ natális 
 sanctórum Mártyrum Valentíni, Solutóris et Victóris; qui sub Diocletiáno 
 Imperatóre passi sunt.
\switchcolumn
\selectlanguage{english}
At Ravenna, the birthday of the holy 
 martyrs Valentine, Salutor, and Victor, who suffered under Emperor 
 Diocletian.
\switchcolumn*
\selectlanguage{latin}
Aquis, in província 
 Narbonénsi, beáti Mítrii, claríssimi Mártyris.
\switchcolumn
\selectlanguage{english}
At Aix, in the province of Narbonne, 
 the renowned martyr, blessed Mitrius.
\switchcolumn*
\selectlanguage{latin}
Cæsaréæ, in Palæstína, 
 pássio sanctórum Antoníni, Zébinæ, Germáni et Ennathæ Vírginis. Hæc, 
 sub Galério Maximiáno Imperatóre, verbéribus cæsa, igne cremáta est; illi 
 vero, cum intrépidi ac líbera voce Firmiliánum Præsidem, diis immolántem, 
 impietátis argúerent, cápite cæsi sunt.
\switchcolumn
\selectlanguage{english}
At Caesarea in Palestine, the 
 martyrdom of the Saints Antoninus, Zebina, Germanus, and the virgin Ennatha. 
 Under Galerius Maximian, Ennatha was scourged and burned alive, while the 
 others, for boldly reproaching the governor Firmilian for his idolatry in 
 sacrificing to the gods, were beheaded.
\switchcolumn*
\selectlanguage{latin}
In Africa sanctórum 
 Mártyrum Hispanórum Arcádii, Paschásii, Probi et Eutychiáni; qui, in 
 persecutióne Wandálica, cum in Ariánam perfídiam nullátenus declináre 
 pateréntur, hinc a Genseríco, Rege Ariáno, primum proscrípti, deínde acti in 
 exsílium atque atrocíssimis supplíciis cruciáti, postrémum divérso mortis 
 génere interémpti sunt. Tunc et Paulílli puéruli, germáni sanctórum 
 Paschásii et Eutychiáni, constántia enítuit; qui, cum de fide cathólica 
 nullátenus posset avélli, fústibus diu cæsus est, atque ad ínfimam 
 servitútem damnátus.
\switchcolumn
\selectlanguage{english}
In Africa, the holy martyrs Arcadius, 
 Paschasius, Probus, and Eutychian, Spaniards who absolutely refused to yield 
 to the Arian perfidy, during the persecution of the Vandals. 
 Accordingly, they were condemned by the Arian king Genseric, driven into exile, and finally, after being subjected to fearful tortures, were put to 
 death in divers manners. At that time there was also seen the 
 constancy of the small boy Paulillus, brother of the Saints Paschasius and 
 Eutychian. Because he could not be turned from the Catholic faith, he 
 was long beaten with rods and sentenced to the lowest servitude.
\switchcolumn*
\selectlanguage{latin}
Romæ sancti Nicolái Papæ Primi, vigóre apostólico præstántis.
\switchcolumn
\selectlanguage{english}
At Rome, Pope St. Nicholas, 
 distinguished for the apostolic spirit.
\switchcolumn*
\selectlanguage{latin}
Turónis, in Gállia, 
 sancti Brítii Epíscopi, qui fuit discípulus beáti Martíni Epíscopi.
\switchcolumn
\selectlanguage{english}
At Tours in France, St. Brice, 
 bishop, a disciple of the blessed Bishop Martin.
\switchcolumn*
\selectlanguage{latin}
Toléti, in Hispánia, 
 sancti Eugénii Epíscopi.
\switchcolumn
\selectlanguage{english}
At Toledo in Spain, St. Eugene, 
 bishop.
\switchcolumn*
\selectlanguage{latin}
Arvérnis, in Gállia, 
 sancti Quinctiáni Epíscopi.
\switchcolumn
\selectlanguage{english}
In Auvergne in France, St. Quinctian, 
 bishop.
\switchcolumn*
\selectlanguage{latin}
Cremónæ, in Insúbria, 
 sancti Homobóni Confessóris; quem, miráculis clarum, Innocéntius Papa 
 Tértius in Sanctórum númerum rétulit.
\switchcolumn
\selectlanguage{english}
At Cremona, in the duchy of Milan, 
 St. Homobonus, confessor, renowned for miracles, whom Innocent III placed 
 among the saints.
\switchcolumn*
\selectlanguage{latin}
\end{paracol}


% ---- martyrology/mart11/mart1114.htm
\needspace{10\baselineskip}
\begin{paracol}{2}
\selectlanguage{latin}
\begin{center}{\color{gregoriocolor} Décimo octávo Kaléndas Decémbris. 
 Luna\dots\ }\end{center}
\switchcolumn
\selectlanguage{english}
\begin{center}{\color{gregoriocolor} The 
 Fourteenth Day of 
 November. The\dots\ Day of the Moon.}\end{center}
\end{paracol}

\noindent\begin{tabularx}{\linewidth}{*{19}{>{\centering\arraybackslash}X}}
 \textcolor{gregoriocolor}{a} & \textcolor{gregoriocolor}{b} & \textcolor{gregoriocolor}{c} & \textcolor{gregoriocolor}{d} & \textcolor{gregoriocolor}{e} & \textcolor{gregoriocolor}{f} & \textcolor{gregoriocolor}{g} & \textcolor{gregoriocolor}{h} & \textcolor{gregoriocolor}{i} & \textcolor{gregoriocolor}{k} & \textcolor{gregoriocolor}{l} & \textcolor{gregoriocolor}{m} & \textcolor{gregoriocolor}{n} & \textcolor{gregoriocolor}{p} & \textcolor{gregoriocolor}{q} & \textcolor{gregoriocolor}{r} & \textcolor{gregoriocolor}{s} & \textcolor{gregoriocolor}{t} & \textcolor{gregoriocolor}{u} \\
 24 & 25 & 26 & 27 & 28 & 29 & 30 & 1 & 2 & 3 & 4 & 5 & 6 & 7 & 8 & 9 & 10 & 11 & 12 \\
\end{tabularx}
\vspace{0.5\baselineskip}
\noindent\begin{tabularx}{\linewidth}{*{12}{>{\centering\arraybackslash}X}}
 \textcolor{gregoriocolor}{A} & \textcolor{gregoriocolor}{B} & \textcolor{gregoriocolor}{C} & \textcolor{gregoriocolor}{D} & \textcolor{gregoriocolor}{E} & F & \textcolor{gregoriocolor}{F} & \textcolor{gregoriocolor}{G} & \textcolor{gregoriocolor}{H} & \textcolor{gregoriocolor}{M} & \textcolor{gregoriocolor}{N} & \textcolor{gregoriocolor}{P} \\
 13 & 14 & 15 & 16 & 17 & 18 & 18 & 19 & 20 & 21 & 22 & 23 \\
\end{tabularx}

\begin{paracol}{2}
\selectlanguage{latin}
\lettrine[lines=2]{S}{ancti} Jósaphat, e 
 sancti Basilíi Ordine, Epíscopi Polocénsis et Mártyris, cujus dies natális 
 recensétur prídie Idus Novémbris.
\switchcolumn
\selectlanguage{english}
\lettrine[lines=2]{S}{t.} Josaphat, of the Order of St. 
 Basil, archbishop and martyr of Poland, whose birthday was observed on the 
 12th of November.
\switchcolumn*
\selectlanguage{latin}
Gangris, in Paphlagónia, 
 sancti Hypátii Epíscopi, qui, a magna Nicæna Synodo rédiens, a Novatiánis 
 hæréticis in via lapídibus impetítus, Martyr occúbuit.
\switchcolumn
\selectlanguage{english}
At Gangra in Paphlagonia, St. 
 Hypatius, bishop, who on his way home from the great Council of Nicaea, was 
 attacked with stones by the Novatian heretics, and died a martyr.
\switchcolumn*
\selectlanguage{latin}
Heracléæ, in Thrácia, 
 natális sanctórum Mártyrum Clementíni, Theodóti et Philómeni.
\switchcolumn
\selectlanguage{english}
At Heraclea in Thrace, the birthday 
 of the holy martyrs Clementinus, Theodotus and Philomenus.
\switchcolumn*
\selectlanguage{latin}
Alexandríæ sancti 
 Serapiónis Mártyris, quem persecutóres, sub Décio Príncipe, ita 
 crudelíssimis affecérunt supplíciis, ut cunctas et junctúras membrórum prius 
 sólverent, eúmque póstea de superióribus domus suæ præcipitárent; atque ille 
 sic gloriósus Christi Martyr efficerétur.
\switchcolumn
\selectlanguage{english}
At Alexandria, St. Serapion, martyr, 
 whom the persecutors under Emperor Decius subjected to torments so cruel 
 that all his limbs were disjointed. He became a martyr of Christ by 
 being hurled from the upper part of the house.
\switchcolumn*
\selectlanguage{latin}
Trecis, in Gállia, 
 sancti Venerándi Mártyris, sub Aureliáno Imperatóre.
\switchcolumn
\selectlanguage{english}
At Troyes in France, under Emperor 
 Aurelian, St. Venerandus, martyr.
\switchcolumn*
\selectlanguage{latin}
In Gállia sanctæ 
 Venerándæ Vírginis, quæ, sub Antoníno Imperatóre et Asclepíade Præside, 
 martyrii corónam accépit.
\switchcolumn
\selectlanguage{english}
Also in France, the holy virgin 
 Veneranda, who received the crown of martyrdom under Emperor Antoninus and 
 the governor Asclepiades.
\switchcolumn*
\selectlanguage{latin}
Eméssæ, in Phœnícia, 
 pássio plurimárum sanctárum mulíerum, quæ, sub sævíssimo Arabum duce Mady, 
 ob Christi fidem, crudelíssime tortæ atque necátæ sunt.
\switchcolumn
\selectlanguage{english}
At Emesa in Phoenicia, the martyrdom 
 of many holy women, who were barbarously tortured and massacred for the 
 faith of Christ under Mady, a savage Arabian chief.
\switchcolumn*
\selectlanguage{latin}
Bonóniæ sancti Jucúndi, 
 Epíscopi et Confessóris.
\switchcolumn
\selectlanguage{english}
At Bologna, St. Jucundus, bishop and 
 confessor.
\switchcolumn*
\selectlanguage{latin}
Augæ, in Gállia, 
 tránsitus sancti Lauréntii, Epíscopi Dublinénsis.
\switchcolumn
\selectlanguage{english}
At Eu in France, St. Laurence, 
 bishop of Dublin.
\switchcolumn*
\selectlanguage{latin}
Algáriæ, in Africa, 
 beáti Serapiónis, qui, primus ex Ordine beátæ Maríæ de Mercéde redemptiónis 
 captivórum, pro captívis fidélibus rediméndis et Christiánæ fídei 
 prædicatióne actus in crucem et membrátim disséctus, martyrii palmam méruit 
 obtinére.
\switchcolumn
\selectlanguage{english}
At Algiers in Africa, blessed 
 Serapion, of the Order of Our Blessed Lady of Ransom. For the 
 redemption of the faithful in captivity and the preaching of the Christian 
 faith, he was the first of his Order to merit the palm of martyrdom, being 
 crucified and torn limb from limb.
\switchcolumn*
\selectlanguage{latin}
\end{paracol}


% ---- martyrology/mart11/mart1115.htm
\needspace{10\baselineskip}
\begin{paracol}{2}
\selectlanguage{latin}
\begin{center}{\color{gregoriocolor} Décimo séptimo Kaléndas Decémbris. 
 Luna\dots\ }\end{center}
\switchcolumn
\selectlanguage{english}
\begin{center}{\color{gregoriocolor} The 
 Fifteenth Day of 
 November. The\dots\ Day of the Moon.}\end{center}
\end{paracol}

\noindent\begin{tabularx}{\linewidth}{*{19}{>{\centering\arraybackslash}X}}
 \textcolor{gregoriocolor}{a} & \textcolor{gregoriocolor}{b} & \textcolor{gregoriocolor}{c} & \textcolor{gregoriocolor}{d} & \textcolor{gregoriocolor}{e} & \textcolor{gregoriocolor}{f} & \textcolor{gregoriocolor}{g} & \textcolor{gregoriocolor}{h} & \textcolor{gregoriocolor}{i} & \textcolor{gregoriocolor}{k} & \textcolor{gregoriocolor}{l} & \textcolor{gregoriocolor}{m} & \textcolor{gregoriocolor}{n} & \textcolor{gregoriocolor}{p} & \textcolor{gregoriocolor}{q} & \textcolor{gregoriocolor}{r} & \textcolor{gregoriocolor}{s} & \textcolor{gregoriocolor}{t} & \textcolor{gregoriocolor}{u} \\
 25 & 26 & 27 & 28 & 29 & 30 & 1 & 2 & 3 & 4 & 5 & 6 & 7 & 8 & 9 & 10 & 11 & 12 & 13 \\
\end{tabularx}
\vspace{0.5\baselineskip}
\noindent\begin{tabularx}{\linewidth}{*{12}{>{\centering\arraybackslash}X}}
 \textcolor{gregoriocolor}{A} & \textcolor{gregoriocolor}{B} & \textcolor{gregoriocolor}{C} & \textcolor{gregoriocolor}{D} & \textcolor{gregoriocolor}{E} & F & \textcolor{gregoriocolor}{F} & \textcolor{gregoriocolor}{G} & \textcolor{gregoriocolor}{H} & \textcolor{gregoriocolor}{M} & \textcolor{gregoriocolor}{N} & \textcolor{gregoriocolor}{P} \\
 14 & 15 & 16 & 17 & 18 & 19 & 19 & 20 & 21 & 22 & 23 & 24 \\
\end{tabularx}

\begin{paracol}{2}
\selectlanguage{latin}
\lettrine[lines=2]{C}{olóniæ} Agrippínæ 
 sancti Albérti Epíscopi et Confessóris, ex Ordine Prædicatórum, cognoménto 
 Magni, sanctitáte et doctrína célebris, quem Pius Papa Undécimus Doctórem 
 universális Ecclésiæ declarávit, et Pius Duodécimus cultórum scientiárum 
 naturálium cæléstem apud Deum Patrónum constítuit.
\switchcolumn
\selectlanguage{english}
\lettrine[lines=2]{A}{t} Cologne, St. Albert, surnamed the 
 Great, bishop and confessor of the Order of Preachers, renowned for his 
 holiness and learning. Pope Pius XI appointed him as Doctor of the 
 universal Church, and Pius XII appointed him as heavenly patron of those 
 studying the natural sciences.
\switchcolumn*
\selectlanguage{latin}
Eódem die natális 
 sancti Eugénii, Epíscopi Toletáni et Mártyris; qui fuit beáti Dionysii 
 Areopagítæ discípulus, et in território Parisiénsi, consummáto martyrii 
 cursu, beátæ passiónis corónam percépit a Dómino. Ipsíus autem corpus 
 Tolétum, in Hispánia, póstea fuit translátum.
\switchcolumn
\selectlanguage{english}
Also, the birthday of St. Eugene, 
 bishop of Toledo and martyr, disciple of blessed Denis the Areopagite. 
 His martyrdom was completed near Paris, and he received from our Lord a 
 crown for his blessed sufferings. His body was afterwards translated 
 to Toledo in Spain.
\switchcolumn*
\selectlanguage{latin}
Nolæ, in Campánia, beáti Felícis, Epíscopi et Mártyris; qui, a quintodécimo ætátis suæ anno, 
 miráculis cláruit, et, sub Marciáno Præside, cum áliis trigínta Sóciis, 
 agónem martyrii complévit.
\switchcolumn
\selectlanguage{english}
At Nola in Campania, blessed Felix, 
 bishop and martyr, who was renowned for miracles from his fifteenth year. 
 He completed the combats of his martyrdom with thirty others, under the 
 governor Marcian.
\switchcolumn*
\selectlanguage{latin}
Edéssæ, in Mesopotámia, pássio sancti Abibi Diáconi, qui sub Licínio Imperatóre et Lysánia Præside, 
 únguibus lacerátus, in ignem conjéctus est.
\switchcolumn
\selectlanguage{english}
At Edessa in Mesopotamia, the 
 martyrdom of St. Abibus, deacon, who was torn with iron hooks and cast into 
 the fire in the time of Emperor Licinius and the governor Lysanias.
\switchcolumn*
\selectlanguage{latin}
Ibídem sanctórum 
 Mártyrum Guríæ et Samónæ, sub Diocletiáno Imperatóre et Antoníno Præside.
\switchcolumn
\selectlanguage{english}
In the same place, the holy martyrs 
 Gurias and Samonas, under Emperor Diocletian and the governor Antoninus.
\switchcolumn*
\selectlanguage{latin}
In Africa sanctórum 
 Mártyrum Secúndi, Fidentiáni et Várici.
\switchcolumn
\selectlanguage{english}
In Africa, the holy martyrs Secundus, 
 Fidentian, and Varicus.
\switchcolumn*
\selectlanguage{latin}
Apud Arcum, in 
 território Santonénsi, natális sancti Machúti, Aleténsis in Gállia Epíscopi; 
 qui, in Anglia natus, a primævo ætátis suæ tirocínio miráculis emícuit.
\switchcolumn
\selectlanguage{english}
At Archingeay, in the neighbourhood 
 of Saintes, the birthday of St. Malo, bishop of Aleth, in France. He 
 was born in England and from his earliest years was famed for his miracles.
\switchcolumn*
\selectlanguage{latin}
Verónæ sancti Lupérii, 
 Epíscopi et Confessóris.
\switchcolumn
\selectlanguage{english}
At Verona, St. Luperius, bishop and 
 confessor.
\switchcolumn*
\selectlanguage{latin}
Kahlembérgæ, prope 
 Vindobónam, in Austria, sancti Leopóldi, ejúsdem provínciæ Austriæ 
 Marchiónis, quem Innocéntius Papa Octávus in Sanctórum númerum adscrípsit.
\switchcolumn
\selectlanguage{english}
At Klosterneuburg, near Vienna in 
 Austria, St. Leopold, margrave of that province of Austria. He was 
 placed on the canon of the saints by Pope Innocent VIII.
\switchcolumn*
\selectlanguage{latin}
\end{paracol}


% ---- martyrology/mart11/mart1116.htm
\needspace{10\baselineskip}
\begin{paracol}{2}
\selectlanguage{latin}
\begin{center}{\color{gregoriocolor} Sextodécimo Kaléndas Decémbris. 
 Luna\dots\ }\end{center}
\switchcolumn
\selectlanguage{english}
\begin{center}{\color{gregoriocolor} The 
 Sixteenth Day of 
 November. The\dots\ Day of the Moon.}\end{center}
\end{paracol}

\noindent\begin{tabularx}{\linewidth}{*{19}{>{\centering\arraybackslash}X}}
 \textcolor{gregoriocolor}{a} & \textcolor{gregoriocolor}{b} & \textcolor{gregoriocolor}{c} & \textcolor{gregoriocolor}{d} & \textcolor{gregoriocolor}{e} & \textcolor{gregoriocolor}{f} & \textcolor{gregoriocolor}{g} & \textcolor{gregoriocolor}{h} & \textcolor{gregoriocolor}{i} & \textcolor{gregoriocolor}{k} & \textcolor{gregoriocolor}{l} & \textcolor{gregoriocolor}{m} & \textcolor{gregoriocolor}{n} & \textcolor{gregoriocolor}{p} & \textcolor{gregoriocolor}{q} & \textcolor{gregoriocolor}{r} & \textcolor{gregoriocolor}{s} & \textcolor{gregoriocolor}{t} & \textcolor{gregoriocolor}{u} \\
 26 & 27 & 28 & 29 & 30 & 1 & 2 & 3 & 4 & 5 & 6 & 7 & 8 & 9 & 10 & 11 & 12 & 13 & 14 \\
\end{tabularx}
\vspace{0.5\baselineskip}
\noindent\begin{tabularx}{\linewidth}{*{12}{>{\centering\arraybackslash}X}}
 \textcolor{gregoriocolor}{A} & \textcolor{gregoriocolor}{B} & \textcolor{gregoriocolor}{C} & \textcolor{gregoriocolor}{D} & \textcolor{gregoriocolor}{E} & F & \textcolor{gregoriocolor}{F} & \textcolor{gregoriocolor}{G} & \textcolor{gregoriocolor}{H} & \textcolor{gregoriocolor}{M} & \textcolor{gregoriocolor}{N} & \textcolor{gregoriocolor}{P} \\
 15 & 16 & 17 & 18 & 19 & 20 & 20 & 21 & 22 & 23 & 24 & 25 \\
\end{tabularx}

\begin{paracol}{2}
\selectlanguage{latin}
\lettrine[lines=2]{S}{anctæ} Gertrúdis 
 Vírginis, cujus natális sequénti die recensétur.
\switchcolumn
\selectlanguage{english}
\lettrine[lines=2]{S}{t.} Gertrude, virgin, whose birthday 
 is on the 17th of November.
\switchcolumn*
\selectlanguage{latin}
Edimbúrgi, in Scótia, sanctæ Margarítæ Víduæ, Scotórum Regínæ, amóre in páuperes et voluntária 
 paupertáte célebris. Ipsíus tamen festívitas quarto Idus Júnii 
 celebrátur.
\switchcolumn
\selectlanguage{english}
At Edinburgh in Scotland, the 
 birthday of St. Margaret, queen of the Scots and widow, renowned for her 
 love of the poor and her voluntary poverty. Her feast is celebrated on 
 the 10th of June.
\switchcolumn*
\selectlanguage{latin}
In Africa sanctórum 
 Mártyrum Rufíni, Marci, Valérii et Sociórum.
\switchcolumn
\selectlanguage{english}
In Africa, the holy martyrs Rufinus, 
 Mark, Valerius, and their fellows.
\switchcolumn*
\selectlanguage{latin}
Eódem die sanctórum 
 Mártyrum Elpídii, Marcélli, Eustóchii et Sociórum; ex quibus Elpídius, cum 
 esset órdinis Senatórii et coram Juliáno Apóstata Christiánam fidem 
 constantíssime profiterétur, ídeo, primum equis indómitis, una cum Sóciis, 
 alligátus atque pertráctus, deínde, in ignem conjéctus, gloriósum martyrium 
 consummávit.
\switchcolumn
\selectlanguage{english}
The same day, the holy martyrs 
 Elpidius, Marcellus, Eustochius, and their companions. Elpidius, who 
 was a senator, perseveringly confessed the Christian faith before Julian the 
 Apostate, and, with his companions, was tied to wild horses and dragged by 
 them, thus fulfilling a glorious martyrdom.
\switchcolumn*
\selectlanguage{latin}
Lugdúni, in Gállia, 
 natális sancti Euchérii, Epíscopi et Confessóris, viri admirándæ fídei et 
 doctrínæ. Hic, ex nobilíssimo Senatórum órdine ad religiósam vitam 
 habitúmque convérsus, diu, intra septa spelúncæ sponte conclúsus, in 
 oratiónibus et jejúniis Christo servívit; deínde apud præfátam urbem in 
 pontificáli Cáthedra, revelánte Angelo, solémniter collocátus est.
\switchcolumn
\selectlanguage{english}
At Lyons in France, the birthday of 
 St. Eucherius, bishop and confessor, a man of extraordinary faith and 
 learning. He renounced the senatorial dignity to embrace the religious 
 life, and for a long time voluntarily shut himself up in a cave, where he 
 served Christ in prayer and fasting. Afterwards, through the 
 revelation of an angel, he was solemnly installed in the episcopal chair of 
 the city of Lyons.
\switchcolumn*
\selectlanguage{latin}
Patávii sancti Fidéntii 
 Epíscopi.
\switchcolumn
\selectlanguage{english}
At Padua, St. Fidentius, bishop.
\switchcolumn*
\selectlanguage{latin}
Cantuáriæ, in Anglia, 
 sancti Edmúndi, Epíscopi et Confessóris; qui, pro Ecclésiæ suæ júribus 
 tuéndis in exsílium actus, apud Provínum, Sénonum óppidum, sanctíssime óbiit; 
 et Sanctórum cánoni ab Innocéntio Papa Quarto adscríptus est.
\switchcolumn
\selectlanguage{english}
At Canterbury in England, St. 
 Edmund, archbishop and confessor, who was sent into exile for having 
 maintained the rights of his church. He died a most holy death at 
 Provins, a town near Sens, and was canonized by Innocent IV.
\switchcolumn*
\selectlanguage{latin}
Eódem die deposítio 
 sancti Othmári Abbátis.
\switchcolumn
\selectlanguage{english}
The same day, the death of St. 
 Othmar, abbot.
\switchcolumn*
\selectlanguage{latin}
\end{paracol}


% ---- martyrology/mart11/mart1117.htm
\needspace{10\baselineskip}
\begin{paracol}{2}
\selectlanguage{latin}
\begin{center}{\color{gregoriocolor} Quintodécimo Kaléndas Decémbris. 
 Luna\dots\ }\end{center}
\switchcolumn
\selectlanguage{english}
\begin{center}{\color{gregoriocolor} The 
 Seventeenth Day of 
 November. The\dots\ Day of the Moon.}\end{center}
\end{paracol}

\noindent\begin{tabularx}{\linewidth}{*{19}{>{\centering\arraybackslash}X}}
 \textcolor{gregoriocolor}{a} & \textcolor{gregoriocolor}{b} & \textcolor{gregoriocolor}{c} & \textcolor{gregoriocolor}{d} & \textcolor{gregoriocolor}{e} & \textcolor{gregoriocolor}{f} & \textcolor{gregoriocolor}{g} & \textcolor{gregoriocolor}{h} & \textcolor{gregoriocolor}{i} & \textcolor{gregoriocolor}{k} & \textcolor{gregoriocolor}{l} & \textcolor{gregoriocolor}{m} & \textcolor{gregoriocolor}{n} & \textcolor{gregoriocolor}{p} & \textcolor{gregoriocolor}{q} & \textcolor{gregoriocolor}{r} & \textcolor{gregoriocolor}{s} & \textcolor{gregoriocolor}{t} & \textcolor{gregoriocolor}{u} \\
 27 & 28 & 29 & 30 & 1 & 2 & 3 & 4 & 5 & 6 & 7 & 8 & 9 & 10 & 11 & 12 & 13 & 14 & 15 \\
\end{tabularx}
\vspace{0.5\baselineskip}
\noindent\begin{tabularx}{\linewidth}{*{12}{>{\centering\arraybackslash}X}}
 \textcolor{gregoriocolor}{A} & \textcolor{gregoriocolor}{B} & \textcolor{gregoriocolor}{C} & \textcolor{gregoriocolor}{D} & \textcolor{gregoriocolor}{E} & F & \textcolor{gregoriocolor}{F} & \textcolor{gregoriocolor}{G} & \textcolor{gregoriocolor}{H} & \textcolor{gregoriocolor}{M} & \textcolor{gregoriocolor}{N} & \textcolor{gregoriocolor}{P} \\
 16 & 17 & 18 & 19 & 20 & 21 & 21 & 22 & 23 & 24 & 25 & 26 \\
\end{tabularx}

\begin{paracol}{2}
\selectlanguage{latin}
\lettrine[lines=2]{N}{eocæsaréæ,} in Ponto, 
 natális sancti Gregórii, Epíscopi et Confessóris, doctrína et sanctitáte 
 illústris, qui propter signa atque mirácula, quæ cum multa Ecclesiárum 
 glória perpetrávit, Thaumatúrgus est appellátus.
\switchcolumn
\selectlanguage{english}
\lettrine[lines=2]{A}{t} Neocaesarea in Pontus, the 
 birthday of St. Gregory, bishop and confessor, illustrious for his learning 
 and sanctity. The signs and miracles which he wrought to the great 
 glory of the Church gained for him the surname Wonderworker.
\switchcolumn*
\selectlanguage{latin}
Helpíthi, in Saxónia, 
 item natális sanctæ Gertrúdis Vírginis, ex Ordine sancti Benedícti, quæ dono 
 revelatiónum clara éxstitit. Ipsíus tamen festívitas prídie hujus diéi 
 celebrátur.
\switchcolumn
\selectlanguage{english}
At Hedelfs in Saxony, the birthday 
 of St. Gertrude, virgin of the Order of St. Benedict, who was famous for her 
 revelations. Her feast is observed on the preceding day.
\switchcolumn*
\selectlanguage{latin}
In Palæstína sanctórum 
 Mártyrum Alphæi et Zachæi, qui primo anno persecutiónis Diocletiáni, post 
 multa torménta, capitálem senténtiam subiére.
\switchcolumn
\selectlanguage{english}
In Palestine, in the first year of 
 Diocletian's persecution, the holy martyrs Alpheus and Zachaeus, who 
 underwent beheading after many tortures.
\switchcolumn*
\selectlanguage{latin}
Córdubæ, in Hispánia, 
 sanctórum Mártyrum Acíscli et Victóriæ germanórum, qui, in eádem 
 persecutióne, Diónis Præsidis jussu sævíssime cruciáti sunt, et illústri 
 passióne corónas a Dómino meruérunt.
\switchcolumn
\selectlanguage{english}
At Cordova in Spain, during the same 
 persecution, the holy martyrs Acisclus and his sister Victoria, who were 
 most cruelly tortured by order of the governor Dion, and thus merited to be 
 crowned by our Lord for their glorious sufferings.
\switchcolumn*
\selectlanguage{latin}
Alexandríæ sancti 
 Dionysii Epíscopi, summæ eruditiónis viri, qui, multis confessiónibus clarus 
 et pro passiónum tormentorúmque diversitáte magníficus, plenus diérum 
 Conféssor quiévit, Valeriáni et Galliéni Imperatórum tempóribus.
\switchcolumn
\selectlanguage{english}
At Alexandria, St. Denis, bishop, a 
 man of very great learning. In the time of Emperors Valerian and 
 Gallienus, renowned for often having confessed the faith, and illustrious 
 for the various sufferings and torments he had endured, full of days he 
 rested in peace a confessor.
\switchcolumn*
\selectlanguage{latin}
Aureliánis, in Gállia, 
 sancti Aniáni Epíscopi, cujus mortem in conspéctu Dómini pretiósam mirácula 
 crebra testántur.
\switchcolumn
\selectlanguage{english}
At Orleans in France, St. Anian, 
 bishop, the value of whose death in the sight of the Lord is attested by 
 frequent miracles.
\switchcolumn*
\selectlanguage{latin}
In Británnia sancti 
 Hugónis Epíscopi, qui, ex Mónacho Carthusiáno ad Ecclésiam Lincolniénsem 
 regéndam vocátus, multis cláruit miráculis, et sancto fine quiévit.
\switchcolumn
\selectlanguage{english}
In England, St. Hugh, bishop, who 
 was called to rule the church of Lincoln. He ended his holy life in 
 peace, renowned for many miracles.
\switchcolumn*
\selectlanguage{latin}
Turónis, in Gállia, 
 sancti Gregórii Epíscopi.
\switchcolumn
\selectlanguage{english}
At Tours in France, St. Gregory, 
 bishop.
\switchcolumn*
\selectlanguage{latin}
Floréntiæ sancti 
 Eugénii Confessóris, qui fuit Diáconus beáti Zenóbii, ejúsdem civitátis 
 Epíscopi.
\switchcolumn
\selectlanguage{english}
At Florence, St. Eugene, confessor, 
 the deacon of blessed Zenobius, bishop of that city.
\switchcolumn*
\selectlanguage{latin}
\end{paracol}


% ---- martyrology/mart11/mart1118.htm
\needspace{10\baselineskip}
\begin{paracol}{2}
\selectlanguage{latin}
\begin{center}{\color{gregoriocolor} Quartodécimo Kaléndas Decémbris. 
 Luna\dots\ }\end{center}
\switchcolumn
\selectlanguage{english}
\begin{center}{\color{gregoriocolor} The 
 Eighteenth Day of 
 November. The\dots\ Day of the Moon.}\end{center}
\end{paracol}

\noindent\begin{tabularx}{\linewidth}{*{19}{>{\centering\arraybackslash}X}}
 \textcolor{gregoriocolor}{a} & \textcolor{gregoriocolor}{b} & \textcolor{gregoriocolor}{c} & \textcolor{gregoriocolor}{d} & \textcolor{gregoriocolor}{e} & \textcolor{gregoriocolor}{f} & \textcolor{gregoriocolor}{g} & \textcolor{gregoriocolor}{h} & \textcolor{gregoriocolor}{i} & \textcolor{gregoriocolor}{k} & \textcolor{gregoriocolor}{l} & \textcolor{gregoriocolor}{m} & \textcolor{gregoriocolor}{n} & \textcolor{gregoriocolor}{p} & \textcolor{gregoriocolor}{q} & \textcolor{gregoriocolor}{r} & \textcolor{gregoriocolor}{s} & \textcolor{gregoriocolor}{t} & \textcolor{gregoriocolor}{u} \\
 28 & 29 & 30 & 1 & 2 & 3 & 4 & 5 & 6 & 7 & 8 & 9 & 10 & 11 & 12 & 13 & 14 & 15 & 16 \\
\end{tabularx}
\vspace{0.5\baselineskip}
\noindent\begin{tabularx}{\linewidth}{*{12}{>{\centering\arraybackslash}X}}
 \textcolor{gregoriocolor}{A} & \textcolor{gregoriocolor}{B} & \textcolor{gregoriocolor}{C} & \textcolor{gregoriocolor}{D} & \textcolor{gregoriocolor}{E} & F & \textcolor{gregoriocolor}{F} & \textcolor{gregoriocolor}{G} & \textcolor{gregoriocolor}{H} & \textcolor{gregoriocolor}{M} & \textcolor{gregoriocolor}{N} & \textcolor{gregoriocolor}{P} \\
 17 & 18 & 19 & 20 & 21 & 22 & 22 & 23 & 24 & 25 & 26 & 27 \\
\end{tabularx}

\begin{paracol}{2}
\selectlanguage{latin}
\lettrine[lines=2]{R}{omæ} Dedicátio 
 Basilicárum sanctórum Petri et Pauli Apostolórum. Eárum primam, 
 restitútam in ampliórem formam, Summus Póntifex Urbánus Octávus consecrávit 
 hac ipsa recurrénte die; álteram vero, miserándo incéndio pénitus consúmptam, 
 ac magnificéntius reædificátam, Pius Nonus die décima Decémbris
 solémni ritu 
 consecrávit, ejúsque ánnuam commemoratiónem hodiérna die agéndam indíxit.
\switchcolumn
\selectlanguage{english}
\lettrine[lines=2]{A}{t} Rome, the dedication of the 
 basilicas of the holy apostles Peter and Paul. The former, having been 
 enlarged, was on this day solemnly consecrated by Urban VIII; while the 
 latter, more beautifully rebuilt after its total destruction by fire, was 
 solemnly dedicated on the 10th of December by Pius IX, though the feast in 
 commemoration of that event was transferred to this day.
\switchcolumn*
\selectlanguage{latin}
Antiochíæ natális 
 sancti Románi Mártyris, qui, témpore Galérii Imperatóris, cum Asclepíades 
 Præféctus in Ecclésiam irrúmperet eámque fúnditus conarétur evértere, 
 céteros Christiános hortátus est ut ei contradícerent, ideóque, post dira 
 torménta et abscissiónem linguæ (sine qua tamen Dei præcónia loquebátur), in cárcere strangulátus láqueo, célebri martyrio coronátur. Passus est 
 étiam ante ipsum puérulus, nómine Bárula, qui, cum fuísset ab eódem 
 interrogátus Præfécto utrum mélius esset unum Deum cólere an plures deos, 
 atque in unum Deum, quem Christiáni colunt, credéndum esse respondísset, 
 proptérea, verbéribus cæsus, jussus est decollári.
\switchcolumn
\selectlanguage{english}
At Antioch, the birthday of St. 
 Romanus, martyr, in the time of Emperor Galerius. When the prefect 
 Asclepiades attacked the Church and attempted to destroy it, Romanus 
 exhorted the Christians to resist him. After being subjected to severe 
 torments and the cutting out of his tongue (without which, however, he spake 
 the praises of God), he was strangled in prison and crowned with glorious 
 martyrdom. Before him suffered a young boy named Barulas, who being 
 asked by him whether it was better to worship one God or several gods, and 
 having answered that we must believe in the one God whom the Christians 
 adore, was scourged and beheaded.
\switchcolumn*
\selectlanguage{latin}
Item Antiochíæ sancti 
 Hesychii Mártyris, qui, cum esset miles, et præcéptum audísset ut quisquis 
 non sacrificáret idólis, cíngulum milítiæ depóneret, repénte cíngulum solvit; 
 ob quam causam, ingénti saxo in déxtera ejus ligáto, in flúvium præcipitári 
 jussus est.
\switchcolumn
\selectlanguage{english}
Also at Antioch, the holy martyr 
 Hesychius, a soldier. Hearing the order that anyone refusing to 
 sacrifice to idols should lay aside his military belt, he immediately took 
 off his. For this reason he was cast into the river with a large stone 
 tied to his right hand.
\switchcolumn*
\selectlanguage{latin}
Eódem die sanctórum 
 Orículi et Sociórum, qui, in persecutióne Wandálica, pro fide cathólica 
 passi sunt.
\switchcolumn
\selectlanguage{english}
On the same day, St. Oriculus and 
 his companions, who suffered for the Catholic faith in the Vandal 
 persecution.
\switchcolumn*
\selectlanguage{latin}
Mogúntiæ sancti Máximi 
 Epíscopi, qui, témpore Constántii multa passus ab Ariánis, Conféssor occúbuit.
\switchcolumn
\selectlanguage{english}
At Mainz, St. Maximus, bishop, who 
 suffered greatly at the hands of the Arians, and died a confessor in the 
 time of Constantius.
\switchcolumn*
\selectlanguage{latin}
Turónis, in Gállia, 
 tránsitus beáti Odónis, Abbátis Cluniacénsis.
\switchcolumn
\selectlanguage{english}
At Tours in France, the passing of 
 blessed Odo, abbot of Cluny.
\switchcolumn*
\selectlanguage{latin}
Antiochíæ sancti Thomæ 
 Mónachi, quem Antiochéni, ob sedátam ejus précibus pestem, solemnitáte ánnua 
 coluérunt.
\switchcolumn
\selectlanguage{english}
At Antioch, St. Thomas, a monk 
 honoured with an annual solemnity by the people of Antioch, for bringing the 
 end of a plague by his prayers.
\switchcolumn*
\selectlanguage{latin}
Lucæ, in Túscia, 
 Translátio sancti Frigdiáni, Epíscopi et Confessóris.
\switchcolumn
\selectlanguage{english}
At Lucca in Tuscany, the translation 
 of St. Frigidian, bishop and confessor.
\switchcolumn*
\selectlanguage{latin}
\end{paracol}


% ---- martyrology/mart11/mart1119.htm
\needspace{10\baselineskip}
\begin{paracol}{2}
\selectlanguage{latin}
\begin{center}{\color{gregoriocolor} Tertiodécimo Kaléndas Decémbris. 
 Luna\dots\ }\end{center}
\switchcolumn
\selectlanguage{english}
\begin{center}{\color{gregoriocolor} The 
 Nineteenth Day of 
 November. The\dots\ Day of the Moon.}\end{center}
\end{paracol}

\noindent\begin{tabularx}{\linewidth}{*{19}{>{\centering\arraybackslash}X}}
 \textcolor{gregoriocolor}{a} & \textcolor{gregoriocolor}{b} & \textcolor{gregoriocolor}{c} & \textcolor{gregoriocolor}{d} & \textcolor{gregoriocolor}{e} & \textcolor{gregoriocolor}{f} & \textcolor{gregoriocolor}{g} & \textcolor{gregoriocolor}{h} & \textcolor{gregoriocolor}{i} & \textcolor{gregoriocolor}{k} & \textcolor{gregoriocolor}{l} & \textcolor{gregoriocolor}{m} & \textcolor{gregoriocolor}{n} & \textcolor{gregoriocolor}{p} & \textcolor{gregoriocolor}{q} & \textcolor{gregoriocolor}{r} & \textcolor{gregoriocolor}{s} & \textcolor{gregoriocolor}{t} & \textcolor{gregoriocolor}{u} \\
 29 & 30 & 1 & 2 & 3 & 4 & 5 & 6 & 7 & 8 & 9 & 10 & 11 & 12 & 13 & 14 & 15 & 16 & 17 \\
\end{tabularx}
\vspace{0.5\baselineskip}
\noindent\begin{tabularx}{\linewidth}{*{12}{>{\centering\arraybackslash}X}}
 \textcolor{gregoriocolor}{A} & \textcolor{gregoriocolor}{B} & \textcolor{gregoriocolor}{C} & \textcolor{gregoriocolor}{D} & \textcolor{gregoriocolor}{E} & F & \textcolor{gregoriocolor}{F} & \textcolor{gregoriocolor}{G} & \textcolor{gregoriocolor}{H} & \textcolor{gregoriocolor}{M} & \textcolor{gregoriocolor}{N} & \textcolor{gregoriocolor}{P} \\
 18 & 19 & 20 & 21 & 22 & 23 & 23 & 24 & 25 & 26 & 27 & 28 \\
\end{tabularx}

\begin{paracol}{2}
\selectlanguage{latin}
\lettrine[lines=2]{I}{n} óppido Marpúrgi, in 
 Germánia, deposítio sanctæ Elísabeth Víduæ, Regis Hungarórum Andréæ fíliæ, 
 ex tértio Ordine sancti Francísci, quæ, pietátis opéribus assídue inténta, 
 miráculis clara migrávit ad Dóminum.
\switchcolumn
\selectlanguage{english}
\lettrine[lines=2]{A}{t} Marburg in Germany, the death of 
 St. Elizabeth, widow, daughter of King Andrew of Hungary, and member of the 
 Third Order of St. Francis. After a life spent in the performance of 
 works of piety, she went to heaven, having a reputation for miracles.
\switchcolumn*
\selectlanguage{latin}
Sancti Pontiáni, Papæ 
 et Mártyris; cujus dies natális tértio Kaléndas Novémbris recensétur.
\switchcolumn
\selectlanguage{english}
St. Pontianus, pope and martyr, whose 
 birthday occurs on the 30th of October.
\switchcolumn*
\selectlanguage{latin}
Samaríæ, in Palæstína, 
 sancti Abdíæ Prophétæ.
\switchcolumn
\selectlanguage{english}
At Samaria in Palestine, the holy 
 prophet Abdias.
\switchcolumn*
\selectlanguage{latin}
Romæ, via Appia, 
 natális sancti Máximi, Presbyteri et Mártyris, qui, in persecutióne 
 Valeriáni passus, pósitus est ad sanctum Xystum.
\switchcolumn
\selectlanguage{english}
At Rome, on the Appian Way, the 
 birthday of St. Maximus, priest and martyr, who suffered in the persecution 
 of Valerian and was buried near St. Sixtus.
\switchcolumn*
\selectlanguage{latin}
In civitáte Astiagénsi, 
 in Hispánia, beáti Crispíni Epíscopi, qui, cápite amputáto, martyrii glóriam 
 adéptus est.
\switchcolumn
\selectlanguage{english}
At Ecijo in Spain, blessed Bishop 
 Crispin, who obtained the glory of martyrdom by beheading.
\switchcolumn*
\selectlanguage{latin}
Eódem die sancti Fausti, 
 Diáconi Alexandríni, qui primum, in persecutióne Valeriáni, cum sancto 
 Dionysio, in exsílium missus est; deínde, ætáte longævus, in persecutióne 
 Diocletiáni, animadvérsus gládio martyrium consummávit.
\switchcolumn
\selectlanguage{english}
St. Faustus, deacon of Alexandria, 
 who had been banished with St. Denis in the persecution of Valerian; later, 
 in the persecution of Diocletian, being advanced in age, his martyrdom was 
 accomplished by the sword.
\switchcolumn*
\selectlanguage{latin}
Cæsaréæ, in Cappadócia, sancti Bárlaam Mártyris, qui, agréstis licet et rudis, Christi sapiéntia 
 munítus tyránnum vicit, et ignem ipsum per invíctam fídei constántiam 
 superávit; in cujus die natáli sanctus Basilíus Magnus célebrem hábuit 
 oratiónem.
\switchcolumn
\selectlanguage{english}
At Caesarea in Cappadocia, St. 
 Barlaam, martyr, who, though unpolished and ignorant, was armed with the 
 wisdom of Christ to overcome the tyrant, and by the constancy of his faith, 
 subdue fire itself. On his birthday, St. Basil the Great delivered a 
 celebrated sermon.
\switchcolumn*
\selectlanguage{latin}
Viénnæ, in Gállia, 
 sanctórum Mártyrum Severíni, Exsupérii et Feliciáni; quorum córpora, post multa annórum currícula, ipsis revelántibus, invénta, et a Pontífice, clero 
 et pópulo illíus urbis honorífice subláta, condígno honóre cóndita sunt.
\switchcolumn
\selectlanguage{english}
At Vienne in France, the holy 
 martyrs Severinus, Exuperius and Felician. Their bodies, after the 
 lapse of many years, were found through their own revelation, and being 
 taken up with due honours by the bishop, clergy, and people of that city, 
 were buried with becoming solemnity.
\switchcolumn*
\selectlanguage{latin}
In Isáuria pássio 
 sanctórum Azæ et Sociórum centum quinquagínta mílitum, sub Diocletiáno 
 Imperatóre et Aquilíno Tribúno.
\switchcolumn
\selectlanguage{english}
In Isauria the martyrdom of St. Azas 
 and his soldier companions, to the number of one hundred and fifty, under 
 Emperor Diocletian and the tribune Aquilinus.
\switchcolumn*
\selectlanguage{latin}
\end{paracol}


% ---- martyrology/mart11/mart1120.htm
\needspace{10\baselineskip}
\begin{paracol}{2}
\selectlanguage{latin}
\begin{center}{\color{gregoriocolor} Duodécimo Kaléndas Decémbris. 
 Luna\dots\ }\end{center}
\switchcolumn
\selectlanguage{english}
\begin{center}{\color{gregoriocolor} The 
 Twentieth Day of 
 November. The\dots\ Day of the Moon.}\end{center}
\end{paracol}

\noindent\begin{tabularx}{\linewidth}{*{19}{>{\centering\arraybackslash}X}}
 \textcolor{gregoriocolor}{a} & \textcolor{gregoriocolor}{b} & \textcolor{gregoriocolor}{c} & \textcolor{gregoriocolor}{d} & \textcolor{gregoriocolor}{e} & \textcolor{gregoriocolor}{f} & \textcolor{gregoriocolor}{g} & \textcolor{gregoriocolor}{h} & \textcolor{gregoriocolor}{i} & \textcolor{gregoriocolor}{k} & \textcolor{gregoriocolor}{l} & \textcolor{gregoriocolor}{m} & \textcolor{gregoriocolor}{n} & \textcolor{gregoriocolor}{p} & \textcolor{gregoriocolor}{q} & \textcolor{gregoriocolor}{r} & \textcolor{gregoriocolor}{s} & \textcolor{gregoriocolor}{t} & \textcolor{gregoriocolor}{u} \\
 30 & 1 & 2 & 3 & 4 & 5 & 6 & 7 & 8 & 9 & 10 & 11 & 12 & 13 & 14 & 15 & 16 & 17 & 18 \\
\end{tabularx}
\vspace{0.5\baselineskip}
\noindent\begin{tabularx}{\linewidth}{*{12}{>{\centering\arraybackslash}X}}
 \textcolor{gregoriocolor}{A} & \textcolor{gregoriocolor}{B} & \textcolor{gregoriocolor}{C} & \textcolor{gregoriocolor}{D} & \textcolor{gregoriocolor}{E} & F & \textcolor{gregoriocolor}{F} & \textcolor{gregoriocolor}{G} & \textcolor{gregoriocolor}{H} & \textcolor{gregoriocolor}{M} & \textcolor{gregoriocolor}{N} & \textcolor{gregoriocolor}{P} \\
 19 & 20 & 21 & 22 & 23 & 24 & 24 & 25 & 26 & 27 & 28 & 29 \\
\end{tabularx}

\begin{paracol}{2}
\selectlanguage{latin}
\lettrine[lines=2]{S}{ancti} Felícis Valésii, 
 Presbyteri et Confessóris, qui Ordinis sanctíssimæ Trinitátis redemptiónis 
 captivórum éxstitit Fundátor, ac prídie Nonas Novémbris obdormívit in 
 Dómino.
\switchcolumn
\selectlanguage{english}
\lettrine[lines=2]{S}{t.} Felix of Valois, priest and 
 confessor, who founded the Order of the Most Holy Trinity for the Redemption 
 of Captives, and who fell asleep in the Lord on the 4th of November.
\switchcolumn*
\selectlanguage{latin}
In Pérside pássio 
 sanctórum Nersæ Epíscopi, et Sociórum.
\switchcolumn
\selectlanguage{english}
In Persia, the martyrdom of St. 
 Nersas, bishop, and his companions.
\switchcolumn*
\selectlanguage{latin}
Messánæ, in Sicília, 
 sanctórum Mártyrum Ampeli et Caji.
\switchcolumn
\selectlanguage{english}
At Messina in Sicily, the holy 
 martyrs Ampelus and Caius.
\switchcolumn*
\selectlanguage{latin}
Tauríni sanctórum 
 Mártyrum Octávii, Solutóris et Adventóris, Thebánæ legiónis mílitum; qui, 
 sub Maximiáno Imperatóre, egrégie decertántes, martyrio coronáti sunt.
\switchcolumn
\selectlanguage{english}
At Turin, the holy martyrs Octavius, 
 Solutor, and Adventor, soldiers of the Theban Legion, who fought valiantly 
 for the faith under Emperor Maximian and who were crowned with martyrdom.
\switchcolumn*
\selectlanguage{latin}
Cæsaréæ, in Palæstína, 
 sancti Agápii Mártyris, qui, sub Galério Maximiáno Imperatóre, damnátus ad 
 béstias et ab iis nil læsus, tandem, lapídibus ac pedes appénsis, in mare 
 demérgitur.
\switchcolumn
\selectlanguage{english}
At Caesarea in Palestine, in the 
 time of Emperor Galerius Maximian, the holy martyr Agapius, who was 
 condemned to be devoured by the beasts; but being unhurt by them, he was 
 cast into the sea with stones tied to his feet.
\switchcolumn*
\selectlanguage{latin}
Doróstori, in Mysia 
 inferióre, sancti Dásii Mártyris, qui, cum in festo Satúrni nollet 
 impudicítiis ejus consentíre, sub Basso Præfécto cæsus est.
\switchcolumn
\selectlanguage{english}
At Silistria in Rumania, St. Dasius, 
 bishop, who, for refusing to consent to the unholy rites of the Saturnalia, 
 was put to death under the governor Bassus.
\switchcolumn*
\selectlanguage{latin}
Nicéæ, in Bithynia, 
 sanctórum Mártyrum Eustáchii, Thespésii et Anatólii, in persecutióne 
 Maximíni.
\switchcolumn
\selectlanguage{english}
At Nicaea in Bithynia, the holy 
 martyrs Eustace, Thespesius, and Anatolius, in the persecution of Maximinus.
\switchcolumn*
\selectlanguage{latin}
Heracléæ, in Thrácia, 
 sanctórum Mártyrum Bassi, Dionysii, Agapíti et aliórum quadragínta.
\switchcolumn
\selectlanguage{english}
At Heraclea in Thrace, the holy 
 martyrs Bassus, Denis, Agapitus, and forty others.
\switchcolumn*
\selectlanguage{latin}
In Anglia sancti 
 Eadmúndi, Regis et Mártyris.
\switchcolumn
\selectlanguage{english}
In England, St. Edmund, king and 
 martyr.
\switchcolumn*
\selectlanguage{latin}
Constantinópoli sancti 
 Gregórii Decapolítæ, qui ob cultum sanctárum Imáginum multa passus est.
\switchcolumn
\selectlanguage{english}
At Constantinople, St. Gregory of 
 Decapolis, who suffered many things for the veneration of sacred images.
\switchcolumn*
\selectlanguage{latin}
Medioláni sancti 
 Benígni Epíscopi, qui, in magna barbarórum perturbatióne, commíssam sibi 
 Ecclésiam summa constántia et religióne administrávit.
\switchcolumn
\selectlanguage{english}
At Milan, St. Benignus, bishop, who, 
 amid great troubles caused by the barbarians, governed the Church entrusted 
 to him with greatest constancy and piety.
\switchcolumn*
\selectlanguage{latin}
Cabillóne, in Gálliis, 
 sancti Silvéstri Epíscopi, qui quadragésimo secúndo sui sacerdótii anno, 
 plenus diérum atque virtútum, migrávit ad Dóminum.
\switchcolumn
\selectlanguage{english}
At Chalons in France, St. Sylvester, 
 bishop, who went to God in the forty-second year of his priesthood, full of 
 days and virtues.
\switchcolumn*
\selectlanguage{latin}
Verónæ sancti Simplícii, 
 Epíscopi et Confessóris.
\switchcolumn
\selectlanguage{english}
At Verona, St. Simplicius, bishop 
 and confessor.
\switchcolumn*
\selectlanguage{latin}
Hildeshémii, in Saxónia, 
 sancti Bernwárdi, Epíscopi et Confessóris, qui a Cælestíno Papa Tértio in 
 Sanctórum númerum adscríptus est.
\switchcolumn
\selectlanguage{english}
At Hildesheim in Saxony, St. 
 Bernward, bishop and confessor, who was numbered among the saints by Pope 
 Celestine III.
\switchcolumn*
\selectlanguage{latin}
\end{paracol}


% ---- martyrology/mart11/mart1121.htm
\needspace{10\baselineskip}
\begin{paracol}{2}
\selectlanguage{latin}
\begin{center}{\color{gregoriocolor} Undécimo Kaléndas Decémbris. 
 Luna\dots\ }\end{center}
\switchcolumn
\selectlanguage{english}
\begin{center}{\color{gregoriocolor} The 
 Twenty-First Day of 
 November. The\dots\ Day of the Moon.}\end{center}
\end{paracol}

\noindent\begin{tabularx}{\linewidth}{*{19}{>{\centering\arraybackslash}X}}
 \textcolor{gregoriocolor}{a} & \textcolor{gregoriocolor}{b} & \textcolor{gregoriocolor}{c} & \textcolor{gregoriocolor}{d} & \textcolor{gregoriocolor}{e} & \textcolor{gregoriocolor}{f} & \textcolor{gregoriocolor}{g} & \textcolor{gregoriocolor}{h} & \textcolor{gregoriocolor}{i} & \textcolor{gregoriocolor}{k} & \textcolor{gregoriocolor}{l} & \textcolor{gregoriocolor}{m} & \textcolor{gregoriocolor}{n} & \textcolor{gregoriocolor}{p} & \textcolor{gregoriocolor}{q} & \textcolor{gregoriocolor}{r} & \textcolor{gregoriocolor}{s} & \textcolor{gregoriocolor}{t} & \textcolor{gregoriocolor}{u} \\
 1 & 2 & 3 & 4 & 5 & 6 & 7 & 8 & 9 & 10 & 11 & 12 & 13 & 14 & 15 & 16 & 17 & 18 & 19 \\
\end{tabularx}
\vspace{0.5\baselineskip}
\noindent\begin{tabularx}{\linewidth}{*{12}{>{\centering\arraybackslash}X}}
 \textcolor{gregoriocolor}{A} & \textcolor{gregoriocolor}{B} & \textcolor{gregoriocolor}{C} & \textcolor{gregoriocolor}{D} & \textcolor{gregoriocolor}{E} & F & \textcolor{gregoriocolor}{F} & \textcolor{gregoriocolor}{G} & \textcolor{gregoriocolor}{H} & \textcolor{gregoriocolor}{M} & \textcolor{gregoriocolor}{N} & \textcolor{gregoriocolor}{P} \\
 20 & 21 & 22 & 23 & 24 & 25 & 25 & 26 & 27 & 28 & 29 & 30 \\
\end{tabularx}

\begin{paracol}{2}
\selectlanguage{latin}
\lettrine[lines=2]{H}{ierosólymis} Præsentátio beátæ Dei Genitrícis Vírginis Maríæ in Templo.
\switchcolumn
\selectlanguage{english}
\lettrine[lines=2]{I}{n} the temple at Jerusalem, the 
 Presentation of the Blessed Virgin Mary, Mother of God.
\switchcolumn*
\selectlanguage{latin}
Eódem die natális beáti 
 Rufi, de quo sanctus Paulus Apóstolus ad Romános scribit.
\switchcolumn
\selectlanguage{english}
Also, the birthday of blessed Rufus, 
 mentioned by the apostle St. Paul in his Epistle to the Romans.
\switchcolumn*
\selectlanguage{latin}
Romæ pássio sanctórum 
 Celsi et Cleméntis.
\switchcolumn
\selectlanguage{english}
At Rome, the martyrdom of the Saints 
 Celsus and Clement.
\switchcolumn*
\selectlanguage{latin}
Rhemis, in Gállia, 
 sancti Albérti, Epíscopi Leodiénsis et Mártyris; qui pro tuénda 
 ecclesiástica libertáte necátus est.
\switchcolumn
\selectlanguage{english}
At Rheims, St. Albert, bishop of 
 Liege and martyr, who was put to death for defending the liberty of the 
 Church.
\switchcolumn*
\selectlanguage{latin}
Apud Ostia Tiberína 
 natális sanctórum Mártyrum Demétrii et Honórii.
\switchcolumn
\selectlanguage{english}
At Ostia, the holy martyrs Demetrius 
 and Honorius.
\switchcolumn*
\selectlanguage{latin}
In Hispánia sanctórum 
 Mártyrum Honórii, Eutychii et Stéphani.
\switchcolumn
\selectlanguage{english}
In Spain, the holy martyrs Honorius, 
 Eutychius, and Stephen.
\switchcolumn*
\selectlanguage{latin}
In Pamphylia sancti 
 Heliodóri Mártyris, in persecutióne Aureliáni, sub Aétio Præside. Post 
 eum vero ipsi tortóres, convérsi ad fidem, in mare demérsi sunt.
\switchcolumn
\selectlanguage{english}
In Pamphylia, St. Heliodorus, 
 martyr, in the persecution of Aurelian under the governor Aetius. 
 After his death his executioners were converted to the faith and were cast 
 into the sea.
\switchcolumn*
\selectlanguage{latin}
Romæ sancti Gelásii 
 Papæ Primi, doctrína et sanctitáte conspícui.
\switchcolumn
\selectlanguage{english}
At Rome, Pope St. Gelasius, 
 distinguished for learning and sanctity.
\switchcolumn*
\selectlanguage{latin}
Verónæ sancti Mauri, 
 Epíscopi et Confessóris.
\switchcolumn
\selectlanguage{english}
At Verona, St. Maur, bishop and 
 confessor.
\switchcolumn*
\selectlanguage{latin}
In monastério Bobiénsi 
 deposítio sancti Columbáni Abbátis, qui, multórum cœnobiórum Fundátor, 
 plurimórum Monachórum éxstitit Pater, multísque virtútibus clarus, in 
 senectúte bona quiévit.
\switchcolumn
\selectlanguage{english}
In the monastery of Bobbio, the 
 death of St. Columban, abbot who founded many monasteries and governed a 
 large number of monks. He died at an advanced age, celebrated for many 
 virtues.
\switchcolumn*
\selectlanguage{latin}
\end{paracol}


% ---- martyrology/mart11/mart1122.htm
\needspace{10\baselineskip}
\begin{paracol}{2}
\selectlanguage{latin}
\begin{center}{\color{gregoriocolor} Décimo Kaléndas Decémbris. 
 Luna\dots\ }\end{center}
\switchcolumn
\selectlanguage{english}
\begin{center}{\color{gregoriocolor} The 
 Twenty-Second Day of 
 November. The\dots\ Day of the Moon.}\end{center}
\end{paracol}

\noindent\begin{tabularx}{\linewidth}{*{19}{>{\centering\arraybackslash}X}}
 \textcolor{gregoriocolor}{a} & \textcolor{gregoriocolor}{b} & \textcolor{gregoriocolor}{c} & \textcolor{gregoriocolor}{d} & \textcolor{gregoriocolor}{e} & \textcolor{gregoriocolor}{f} & \textcolor{gregoriocolor}{g} & \textcolor{gregoriocolor}{h} & \textcolor{gregoriocolor}{i} & \textcolor{gregoriocolor}{k} & \textcolor{gregoriocolor}{l} & \textcolor{gregoriocolor}{m} & \textcolor{gregoriocolor}{n} & \textcolor{gregoriocolor}{p} & \textcolor{gregoriocolor}{q} & \textcolor{gregoriocolor}{r} & \textcolor{gregoriocolor}{s} & \textcolor{gregoriocolor}{t} & \textcolor{gregoriocolor}{u} \\
 2 & 3 & 4 & 5 & 6 & 7 & 8 & 9 & 10 & 11 & 12 & 13 & 14 & 15 & 16 & 17 & 18 & 19 & 20 \\
\end{tabularx}
\vspace{0.5\baselineskip}
\noindent\begin{tabularx}{\linewidth}{*{12}{>{\centering\arraybackslash}X}}
 \textcolor{gregoriocolor}{A} & \textcolor{gregoriocolor}{B} & \textcolor{gregoriocolor}{C} & \textcolor{gregoriocolor}{D} & \textcolor{gregoriocolor}{E} & F & \textcolor{gregoriocolor}{F} & \textcolor{gregoriocolor}{G} & \textcolor{gregoriocolor}{H} & \textcolor{gregoriocolor}{M} & \textcolor{gregoriocolor}{N} & \textcolor{gregoriocolor}{P} \\
 21 & 22 & 23 & 24 & 25 & 26 & 26 & 27 & 28 & 29 & 30 & 1 \\
\end{tabularx}

\begin{paracol}{2}
\selectlanguage{latin}
\lettrine[lines=2]{S}{anctæ} Cæcíliæ, 
 Vírginis et Mártyris, quæ ad cæléstem Sponsum, próprio sánguine purpuráta, 
 transívit sextodécimo Kaléndas Octóbris.
\switchcolumn
\selectlanguage{english}
\lettrine[lines=2]{S}{t.} Cecilia, virgin and martyr, who 
 on the 16th of September, purpled with her own blood, departed to her 
 heavenly Spouse.
\switchcolumn*
\selectlanguage{latin}
Colóssis, in Phrygia, 
 sanctórum Philémonis et Apphíæ, sancti Pauli discipulórum; qui, sub Neróne 
 Imperatóre, cum Gentíles, in die festo Diánæ, invasíssent Ecclésiam, ambo 
 céteris fugiéntibus, tenti, jussu Artoclis Præsidis verbéribus cæsi sunt, 
 et, usque ad renes in fóveam inclúsi, lapídibus opprimúntur.
\switchcolumn
\selectlanguage{english}
At Colossae in Phrygia, during the 
 reign of Nero, Saints Philemon and Apphias, disciples of St. Paul. 
 When the heathen rushed into the church on the feast of Diana, they were 
 arrested and the rest of the Christians fled. By command of the 
 governor Artocles they were scourged, enclosed up to their waists in a pit, 
 then overwhelmed with stones.
\switchcolumn*
\selectlanguage{latin}
Romæ sancti Mauri 
 Mártyris, qui, cum ex Africa venísset ad sepúlcra Apostolórum, sub 
 Imperatóre Numeriáno et Urbis Præfécto Celeríno agonizávit.
\switchcolumn
\selectlanguage{english}
At Rome, St. Maurus, martyr. He 
 came from Africa to visit the tombs of the apostles, and suffered martyrdom 
 there under Celerinus, prefect of the city in the reign of Emperor Numerian.
\switchcolumn*
\selectlanguage{latin}
Antiochíæ Pisídiæ 
 pássio sanctórum Marci et Stéphani, sub Diocletiáno Imperatóre.
\switchcolumn
\selectlanguage{english}
At Antioch in Pisidia, the martyrdom 
 of the Saints Mark and Stephen, under Emperor Diocletian.
\switchcolumn*
\selectlanguage{latin}
Augustodúni sancti 
 Pragmátii, Epíscopi et Confessóris.
\switchcolumn
\selectlanguage{english}
At Autun, St. Pragmatius, bishop and 
 confessor.
\switchcolumn*
\selectlanguage{latin}
\end{paracol}


% ---- martyrology/mart11/mart1123.htm
\needspace{10\baselineskip}
\begin{paracol}{2}
\selectlanguage{latin}
\begin{center}{\color{gregoriocolor} Nono Kaléndas Decémbris. 
 Luna\dots\ }\end{center}
\switchcolumn
\selectlanguage{english}
\begin{center}{\color{gregoriocolor} The 
 Twenty-Third Day of 
 November. The\dots\ Day of the Moon.}\end{center}
\end{paracol}

\noindent\begin{tabularx}{\linewidth}{*{19}{>{\centering\arraybackslash}X}}
 \textcolor{gregoriocolor}{a} & \textcolor{gregoriocolor}{b} & \textcolor{gregoriocolor}{c} & \textcolor{gregoriocolor}{d} & \textcolor{gregoriocolor}{e} & \textcolor{gregoriocolor}{f} & \textcolor{gregoriocolor}{g} & \textcolor{gregoriocolor}{h} & \textcolor{gregoriocolor}{i} & \textcolor{gregoriocolor}{k} & \textcolor{gregoriocolor}{l} & \textcolor{gregoriocolor}{m} & \textcolor{gregoriocolor}{n} & \textcolor{gregoriocolor}{p} & \textcolor{gregoriocolor}{q} & \textcolor{gregoriocolor}{r} & \textcolor{gregoriocolor}{s} & \textcolor{gregoriocolor}{t} & \textcolor{gregoriocolor}{u} \\
 3 & 4 & 5 & 6 & 7 & 8 & 9 & 10 & 11 & 12 & 13 & 14 & 15 & 16 & 17 & 18 & 19 & 20 & 21 \\
\end{tabularx}
\vspace{0.5\baselineskip}
\noindent\begin{tabularx}{\linewidth}{*{12}{>{\centering\arraybackslash}X}}
 \textcolor{gregoriocolor}{A} & \textcolor{gregoriocolor}{B} & \textcolor{gregoriocolor}{C} & \textcolor{gregoriocolor}{D} & \textcolor{gregoriocolor}{E} & F & \textcolor{gregoriocolor}{F} & \textcolor{gregoriocolor}{G} & \textcolor{gregoriocolor}{H} & \textcolor{gregoriocolor}{M} & \textcolor{gregoriocolor}{N} & \textcolor{gregoriocolor}{P} \\
 22 & 23 & 24 & 25 & 26 & 27 & 27 & 28 & 29 & 30 & 1 & 2 \\
\end{tabularx}

\begin{paracol}{2}
\selectlanguage{latin}
\lettrine[lines=2]{N}{atális} sancti 
 Cleméntis Primi, Papæ et Mártyris, qui, tértius post beátum Petrum Apóstolum, 
 Pontificátum ténuit, et, in persecutióne Trajáni, apud Chersonésum relegátus, 
 ibi, alligáta ad ejus collum ánchora, præcipitátus in mare, martyrio 
 coronátur. Ipsíus autem corpus, Hadriáno Secúndo Summo Pontífice, a 
 sanctis Cyríllo et Methódio frátribus Romam translátum, in Ecclésia quæ ejus 
 nómine ántea fúerat exstrúcta, honorífice recónditum est.
\switchcolumn
\selectlanguage{english}
\lettrine[lines=2]{T}{he} birthday of Pope St. Clement, 
 who held the sovereign pontificate the third after the blessed apostle 
 Peter. In the persecution of Trajan, he was banished to Chersonesus, 
 where, being thrown into the sea with an anchor tied to his neck, he was 
 crowned with martyrdom. During the pontificate of Pope Adrian II, his 
 body was translated to Rome by the brothers Saints Cyril and Methodius, and 
 buried with honour in the church that had already been built and named 
 for him.
\switchcolumn*
\selectlanguage{latin}
Romæ sanctæ Felicitátis 
 Mártyris, septem filiórum Mártyrum matris; quæ, post eos, jubénte Marco 
 Antoníno Imperatóre, decolláta est pro Christo.
\switchcolumn
\selectlanguage{english}
At Rome, St. Felicitas, mother of 
 seven martyred sons. After them she was beheaded for Christ by order 
 of Emperor Marcus Antoninus.
\switchcolumn*
\selectlanguage{latin}
Cyzici, in Hellespónto, 
 sancti Sisínii Mártyris, qui in persecutióne Diocletiáni Imperatóris, post 
 multa torménta, gládio cæsus est.
\switchcolumn
\selectlanguage{english}
At Cyzicum, in the Hellespont, St. 
 Sisinius, martyr, who after many torments was put to the sword in the 
 persecution of Diocletian.
\switchcolumn*
\selectlanguage{latin}
Eméritæ, in Hispánia, 
 sanctæ Lucrétiæ, Vírginis et Mártyris; quæ in eádem persecutióne, sub 
 Daciáno Præside, martyrium consummávit.
\switchcolumn
\selectlanguage{english}
At Merida in Spain, St. Lucretia, 
 virgin and martyr, whose martyrdom was fulfilled in the same persecution, 
 under the governor Dacian.
\switchcolumn*
\selectlanguage{latin}
Icónii, in Lycaónia, 
 sancti Amphilóchii Epíscopi, qui, sanctórum Basilíi et Gregórii Nazianzéni 
 in erémo sócius et in Episcopátu colléga, tandem, post multa quæ suscépit 
 pro cathólica fide certámina, sanctitáte et doctrína clarus, quiévit in 
 pace.
\switchcolumn
\selectlanguage{english}
At Iconium in Lycaonia, the holy 
 bishop Amphilochius, who was the companion of St. Basil and St. Gregory 
 Nazianzen in the desert, and their colleague in the episcopate. After 
 enduring many trials for the Catholic faith, he rested in peace, renowned 
 for holiness and learning.
\switchcolumn*
\selectlanguage{latin}
Agrigénti deposítio 
 sancti Gregórii Epíscopi.
\switchcolumn
\selectlanguage{english}
At Girgenti, the death of St. 
 Gregory, bishop.
\switchcolumn*
\selectlanguage{latin}
In óppido Hasbániæ, in 
 Bélgio, sancti Trudónis, Presbyteri et Confessóris, cujus nómine póstmodum 
 insignítum fuit tum monastérium illic ab eódem Sancto in suis prædiis 
 eréctum, tum ipsum óppidum in eo loco paulátim exstrúctum.
\switchcolumn
\selectlanguage{english}
In the town of Hasbein in Belgium, 
 St. Trudo, priest and confessor. Both the monastery which he had 
 erected on his land, and the town which soon afterwards arose, were later 
 named for him.
\switchcolumn*
\selectlanguage{latin}
\end{paracol}


% ---- martyrology/mart11/mart1124.htm
\needspace{10\baselineskip}
\begin{paracol}{2}
\selectlanguage{latin}
\begin{center}{\color{gregoriocolor} Octávo Kaléndas Decémbris. 
 Luna\dots\ }\end{center}
\switchcolumn
\selectlanguage{english}
\begin{center}{\color{gregoriocolor} The 
 Twenty-Fourth Day of 
 November. The\dots\ Day of the Moon.}\end{center}
\end{paracol}

\noindent\begin{tabularx}{\linewidth}{*{19}{>{\centering\arraybackslash}X}}
 \textcolor{gregoriocolor}{a} & \textcolor{gregoriocolor}{b} & \textcolor{gregoriocolor}{c} & \textcolor{gregoriocolor}{d} & \textcolor{gregoriocolor}{e} & \textcolor{gregoriocolor}{f} & \textcolor{gregoriocolor}{g} & \textcolor{gregoriocolor}{h} & \textcolor{gregoriocolor}{i} & \textcolor{gregoriocolor}{k} & \textcolor{gregoriocolor}{l} & \textcolor{gregoriocolor}{m} & \textcolor{gregoriocolor}{n} & \textcolor{gregoriocolor}{p} & \textcolor{gregoriocolor}{q} & \textcolor{gregoriocolor}{r} & \textcolor{gregoriocolor}{s} & \textcolor{gregoriocolor}{t} & \textcolor{gregoriocolor}{u} \\
 4 & 5 & 6 & 7 & 8 & 9 & 10 & 11 & 12 & 13 & 14 & 15 & 16 & 17 & 18 & 19 & 20 & 21 & 22 \\
\end{tabularx}
\vspace{0.5\baselineskip}
\noindent\begin{tabularx}{\linewidth}{*{12}{>{\centering\arraybackslash}X}}
 \textcolor{gregoriocolor}{A} & \textcolor{gregoriocolor}{B} & \textcolor{gregoriocolor}{C} & \textcolor{gregoriocolor}{D} & \textcolor{gregoriocolor}{E} & F & \textcolor{gregoriocolor}{F} & \textcolor{gregoriocolor}{G} & \textcolor{gregoriocolor}{H} & \textcolor{gregoriocolor}{M} & \textcolor{gregoriocolor}{N} & \textcolor{gregoriocolor}{P} \\
 23 & 24 & 25 & 26 & 27 & 28 & 28 & 29 & 30 & 1 & 2 & 3 \\
\end{tabularx}

\begin{paracol}{2}
\selectlanguage{latin}
\lettrine[lines=2]{S}{ancti} Joánnis a Cruce 
 Presbyteri, Confessóris et Ecclesiásticæ Doctóris, sanctæ Terésiæ in 
 Carmelitárum reformatióne sócii, cujus dies natális décimo nono Kaléndas 
 Januárii recensétur.
\switchcolumn
\selectlanguage{english}
\lettrine[lines=2]{S}{t.} John of the Cross, priest and 
 confessor, and doctor of the Church, companion of St. Teresa in the reform 
 of Carmel, and whose birthday is the 14th of December.
\switchcolumn*
\selectlanguage{latin}
Eódem die natális 
 sancti Chrysógoni Mártyris, qui, post longa víncula et cárceres pro 
 constantíssima Christi confessióne tolerátos, Aquiléjam, jubénte Diocletiáno, 
 perdúctus, tandem, cæsus cápite et in mare projéctus, martyrium consummávit.
\switchcolumn
\selectlanguage{english}
Also, the birthday of St. 
 Chrysogonus, martyr. After a long imprisonment in chains for the 
 constant confession of Christ, he was ordered by Diocletian to be taken to 
 Aquileia, where he completed his martyrdom by being beheaded and thrown into 
 the sea.
\switchcolumn*
\selectlanguage{latin}
Romæ sancti 
 Crescentiáni Mártyris, qui in passióne beáti Marcélli Papæ memorátur.
\switchcolumn
\selectlanguage{english}
At Rome, St. Crescentian, martyr, 
 whose name is mentioned in the Acts of blessed Pope Marcellus.
\switchcolumn*
\selectlanguage{latin}
Apud Corínthum sancti 
 Alexándri Mártyris, qui sub Juliáno Apóstata et Sallústio Præside, pro 
 Christi fide certávit usque ad mortem.
\switchcolumn
\selectlanguage{english}
At Corinth, St. Alexander, martyr, 
 who fought unto death for the faith of Christ, under Julian the Apostate and 
 the governor Sallust.
\switchcolumn*
\selectlanguage{latin}
Perúsiæ sancti 
 Felicíssimi Mártyris.
\switchcolumn
\selectlanguage{english}
At Perugia, St. Felicissimus, 
 martyr.
\switchcolumn*
\selectlanguage{latin}
Amériæ, in Umbria, 
 sanctæ Firmínæ, Vírginis et Mártyris; quæ, in persecutióne Diocletiáni 
 Imperatóris, várie cruciáta est, ac demum, suspénsa et lampádibus ardéntibus 
 adústa, immaculátum spíritum Deo réddidit.
\switchcolumn
\selectlanguage{english}
At Amelia in Umbria, during the 
 persecution of Diocletian, St. Firmina, virgin and martyr. After being 
 subjected to various torments, to hanging, and to burning with flaming 
 torches, she yielded up her spirit.
\switchcolumn*
\selectlanguage{latin}
Córdubæ, in Hispánia, 
 sanctárum Vírginum et Mártyrum Floræ et Maríæ; quæ, post diutúrnos cárceres, 
 in persecutióne Arábica, gládio interémptæ sunt.
\switchcolumn
\selectlanguage{english}
At Cordova in Spain, the holy 
 virgins and martyrs Flora and Mary, who after a long imprisonment were slain 
 with the sword in the Arab persecution.
\switchcolumn*
\selectlanguage{latin}
Medioláni sancti 
 Protásii Epíscopi, qui apud Constántem Imperatórem in Concílio Sardicénsi 
 causam Athanásii deféndit, ac demum, pro Ecclésia sibi commíssa et pro 
 religióne multis perfúnctus labóribus, migrávit ad Dóminum.
\switchcolumn
\selectlanguage{english}
At Milan, St. Protase, bishop, who 
 defended the cause of Athanasius before Emperor Constans in the Council of 
 Sardica. Having sustained many labours for the church entrusted to him 
 and for religion, he departed this life to go to the Lord.
\switchcolumn*
\selectlanguage{latin}
In território 
 Arvernénsi sancti Portiáni Abbátis, qui, sub Theodoríco Rege, miráculis 
 cláruit; cujus étiam nomen índitum mansit tam monastério cui Sanctus ipse 
 præfuit, quam óppido quod in eódem loco póstea constrúctum fuit.
\switchcolumn
\selectlanguage{english}
In the territory of Auvergne, St. 
 Portian, an abbot who was renowned for miracles in the time of King 
 Theodoric. His name was given to the monastery that he had governed 
 and also the town which was later built there.
\switchcolumn*
\selectlanguage{latin}
In castro Blávio, in 
 Gállia, sancti Románi Presbyteri, cujus sanctitátis præcónium glória 
 miraculórum declárat.
\switchcolumn
\selectlanguage{english}
In the town of Blaye in France, St. 
 Romanus, priest, whose holiness is proclaimed by glorious miracles.
\switchcolumn*
\selectlanguage{latin}
\end{paracol}


% ---- martyrology/mart11/mart1125.htm
\needspace{10\baselineskip}
\begin{paracol}{2}
\selectlanguage{latin}
\begin{center}{\color{gregoriocolor} Séptimo Kaléndas Decémbris. 
 Luna\dots\ }\end{center}
\switchcolumn
\selectlanguage{english}
\begin{center}{\color{gregoriocolor} The 
 Twenty-Fifth Day of 
 November. The\dots\ Day of the Moon.}\end{center}
\end{paracol}

\noindent\begin{tabularx}{\linewidth}{*{19}{>{\centering\arraybackslash}X}}
 \textcolor{gregoriocolor}{a} & \textcolor{gregoriocolor}{b} & \textcolor{gregoriocolor}{c} & \textcolor{gregoriocolor}{d} & \textcolor{gregoriocolor}{e} & \textcolor{gregoriocolor}{f} & \textcolor{gregoriocolor}{g} & \textcolor{gregoriocolor}{h} & \textcolor{gregoriocolor}{i} & \textcolor{gregoriocolor}{k} & \textcolor{gregoriocolor}{l} & \textcolor{gregoriocolor}{m} & \textcolor{gregoriocolor}{n} & \textcolor{gregoriocolor}{p} & \textcolor{gregoriocolor}{q} & \textcolor{gregoriocolor}{r} & \textcolor{gregoriocolor}{s} & \textcolor{gregoriocolor}{t} & \textcolor{gregoriocolor}{u} \\
 5 & 6 & 7 & 8 & 9 & 10 & 11 & 12 & 13 & 14 & 15 & 16 & 17 & 18 & 19 & 20 & 21 & 22 & 23 \\
\end{tabularx}
\vspace{0.5\baselineskip}
\noindent\begin{tabularx}{\linewidth}{*{12}{>{\centering\arraybackslash}X}}
 \textcolor{gregoriocolor}{A} & \textcolor{gregoriocolor}{B} & \textcolor{gregoriocolor}{C} & \textcolor{gregoriocolor}{D} & \textcolor{gregoriocolor}{E} & F & \textcolor{gregoriocolor}{F} & \textcolor{gregoriocolor}{G} & \textcolor{gregoriocolor}{H} & \textcolor{gregoriocolor}{M} & \textcolor{gregoriocolor}{N} & \textcolor{gregoriocolor}{P} \\
 24 & 25 & 26 & 27 & 28 & 29 & 29 & 30 & 1 & 2 & 3 & 4 \\
\end{tabularx}

\begin{paracol}{2}
\selectlanguage{latin}
\lettrine[lines=2]{A}{lexandríæ} sanctæ 
 Catharínæ, Vírginis et Mártyris, quæ, ob fídei Christiánæ confessiónem, sub 
 Maximíno Imperatóre, in cárcerem trusa, et póstmodum scorpiónibus diutíssime 
 cæsa, tandem cápitis obtruncatióne martyrium complévit. Ipsíus corpus, 
 in montem Sínai mirabíliter ab Angelis delátum, ibídem, frequénti 
 Christianórum concúrsu, pia veneratióne cólitur.
\switchcolumn
\selectlanguage{english}
\lettrine[lines=2]{A}{t} Alexandria, St. Catherine, virgin 
 and martyr, in the time of Emperor Maximinus. For the confession of 
 the Christian faith she was cast into prison, endured a long scourging with 
 whips set with metal, and finally ended her martyrdom by having her head cut 
 off. Her body was miraculously carried by angels to Mount Sinai, where 
 pious veneration is paid to it by great gatherings of Christians.
\switchcolumn*
\selectlanguage{latin}
Romæ sancti Móysis, 
 Presbyteris et Mártyris; quem, cum áliis deténtum in cárcere, sanctus 
 Cypriánus per lítteras sæpe est consolátus. Ipse autem Móyses, cum non 
 tantum advérsus Gentíles, sed étiam advérsus schismáticos et hæréticos 
 Novatiános infrácto ánimo stetísset, demum (ut sanctus Cornélius Papa 
 testátur), in persecutióne Décii, exímio et admirábili martyrio decorátus 
 est.
\switchcolumn
\selectlanguage{english}
At Rome, St. Moses, priest and 
 martyr, who, along with others detained in prison, was often consoled by the 
 letters of St. Cyprian. He withstood with unbending courage not only 
 the heathen, but also the Novatian schismatics and heretics, and according 
 to the words of Pope St. Cornelius, he was finally crowned with a martyrdom 
 which fills the mind with admiration in the persecution of Decius.
\switchcolumn*
\selectlanguage{latin}
Antiochíæ sancti Erásmi 
 Mártyris.
\switchcolumn
\selectlanguage{english}
At Antioch, St. Erasmus, martyr.
\switchcolumn*
\selectlanguage{latin}
Cæsaréæ, in Cappadócia, pássio sancti Mercúrii mílitis, qui custodiéntis se Angeli patrocínio et 
 bárbaros vicit, et Décii sævítiam superávit; multísque auctus tormentórum 
 trophæis, martyrio coronátus migrávit in cælum.
\switchcolumn
\selectlanguage{english}
At Caesarea in Cappadocia, St. 
 Mercury, a soldier, who vanquished the barbarians and triumphed over the 
 cruelty of Decius through the protection of his guardian angel. 
 Finally, having acquired great glory from his sufferings, he was crowned 
 with martyrdom and went to reign forever in heaven.
\switchcolumn*
\selectlanguage{latin}
In Æmília, Itáliæ 
 província, sanctæ Jucúndæ Vírginis.
\switchcolumn
\selectlanguage{english}
In Emilia, a province of Italy, St. 
 Jucunda, virgin.
\switchcolumn*
\selectlanguage{latin}
\end{paracol}


% ---- martyrology/mart11/mart1126.htm
\needspace{10\baselineskip}
\begin{paracol}{2}
\selectlanguage{latin}
\begin{center}{\color{gregoriocolor} Sexto Kaléndas Decémbris. 
 Luna\dots\ }\end{center}
\switchcolumn
\selectlanguage{english}
\begin{center}{\color{gregoriocolor} The 
 Twenty-Sixth Day of 
 November. The\dots\ Day of the Moon.}\end{center}
\end{paracol}

\noindent\begin{tabularx}{\linewidth}{*{19}{>{\centering\arraybackslash}X}}
 \textcolor{gregoriocolor}{a} & \textcolor{gregoriocolor}{b} & \textcolor{gregoriocolor}{c} & \textcolor{gregoriocolor}{d} & \textcolor{gregoriocolor}{e} & \textcolor{gregoriocolor}{f} & \textcolor{gregoriocolor}{g} & \textcolor{gregoriocolor}{h} & \textcolor{gregoriocolor}{i} & \textcolor{gregoriocolor}{k} & \textcolor{gregoriocolor}{l} & \textcolor{gregoriocolor}{m} & \textcolor{gregoriocolor}{n} & \textcolor{gregoriocolor}{p} & \textcolor{gregoriocolor}{q} & \textcolor{gregoriocolor}{r} & \textcolor{gregoriocolor}{s} & \textcolor{gregoriocolor}{t} & \textcolor{gregoriocolor}{u} \\
 6 & 7 & 8 & 9 & 10 & 11 & 12 & 13 & 14 & 15 & 16 & 17 & 18 & 19 & 20 & 21 & 22 & 23 & 24 \\
\end{tabularx}
\vspace{0.5\baselineskip}
\noindent\begin{tabularx}{\linewidth}{*{12}{>{\centering\arraybackslash}X}}
 \textcolor{gregoriocolor}{A} & \textcolor{gregoriocolor}{B} & \textcolor{gregoriocolor}{C} & \textcolor{gregoriocolor}{D} & \textcolor{gregoriocolor}{E} & F & \textcolor{gregoriocolor}{F} & \textcolor{gregoriocolor}{G} & \textcolor{gregoriocolor}{H} & \textcolor{gregoriocolor}{M} & \textcolor{gregoriocolor}{N} & \textcolor{gregoriocolor}{P} \\
 25 & 26 & 27 & 28 & 29 & 1 & 30 & 1 & 2 & 3 & 4 & 5 \\
\end{tabularx}

\begin{paracol}{2}
\selectlanguage{latin}
\lettrine[lines=2]{A}{pud} Fabriánum, in 
 Picéno, beáti Silvéstri Abbátis, Institutóris Congregatiónis Monachórum 
 Silvestrinórum.
\switchcolumn
\selectlanguage{english}
\lettrine[lines=2]{A}{t} Fabriano in Piceno, St. 
 Sylvester, abbot, founder of the Congregation of Sylvestrine monks.
\switchcolumn*
\selectlanguage{latin}
Alexandríæ natális 
 sancti Petri, ejúsdem urbis Epíscopi et Mártyris; qui, cum esset ómnibus 
 virtútibus exornátus, ibídem, Galérii Maximiáni præcépto, cápite obtruncátus 
 est.
\switchcolumn
\selectlanguage{english}
At Alexandria, the birthday of St. 
 Peter, bishop of that city, graced with every virtue, who was beheaded by 
 command of Galerius Maximian.
\switchcolumn*
\selectlanguage{latin}
Passi sunt étiam 
 Alexandríæ, in eádem persecutióne, sancti Mártyres Faustus Présbyter, Dídius 
 et Ammónius, itémque Epíscopi quátuor Ægyptii, idest Philéas, Hesychius, 
 Pachómius et Theodórus, cum áliis sexcéntis sexagínta, quos persecutiónis 
 gládius evéxit ad cælos.
\switchcolumn
\selectlanguage{english}
There suffered also at Alexandria in 
 the same persecution the holy martyrs Faustus, a priest, Didius, and 
 Ammonius; likewise four bishops of Egypt, Phileas, Hesychius, Pachomius, and 
 Theodore, with others numbering six hundred and sixty, whom the sword of 
 persecution sent to heaven.
\switchcolumn*
\selectlanguage{latin}
Apud villam cui nomen 
 Fracta, in território Rhodigiénsi, sancti Bellíni, Epíscopi Patavíni et 
 Mártyris; qui a sicáriis, cum esset Ecclésiæ júrium defénsor exímius, 
 crudéliter impetítus ac multis illátis vulnéribus occísus est.
\switchcolumn
\selectlanguage{english}
In the village of Fracta, St. 
 Bellinus, bishop of Padua and martyr. The noble defender of the rights 
 of the Church was cruelly attacked by assassins, inflicting many wounds upon 
 him, and then slaying him.
\switchcolumn*
\selectlanguage{latin}
Nicomedíæ sancti 
 Marcélli Presbyteri, qui, Constántii témpore, ab Ariánis e rupe præcipitátus, 
 Martyr occúbuit.
\switchcolumn
\selectlanguage{english}
At Nicomedia, in the time of 
 Constantius, St. Marcellus, a priest, who died a martyr by being hurled from 
 a rock by the Arians.
\switchcolumn*
\selectlanguage{latin}
Romæ sancti Sirícii, 
 Papæ et Confessóris, doctrína, pietáte et religiónis zelo præclári, qui 
 vários damnávit hæréticos, et disciplínam ecclesiásticam salubérrimis 
 decrétis instaurávit.
\switchcolumn
\selectlanguage{english}
At Rome, St. Siricius, pope and 
 confessor, celebrated for his learning, piety, and zeal for religion, who 
 condemned various heretics and published salutary laws concerning 
 ecclesiastical discipline.
\switchcolumn*
\selectlanguage{latin}
Augustodúni sancti 
 Amatóris Epíscopi.
\switchcolumn
\selectlanguage{english}
At Autun, St. Amator, bishop.
\switchcolumn*
\selectlanguage{latin}
Constántiæ, in Germánia, sancti Conrádi Epíscopi.
\switchcolumn
\selectlanguage{english}
At Constance in Germany, St. Conrad, 
 bishop.
\switchcolumn*
\selectlanguage{latin}
Romæ sancti Leonárdi, a 
 Portu Maurítio, Sacerdótis ex Ordine Minórum et Confessóris, zelo animárum 
 et sacris per Itáliam expeditiónibus conspícui; quem Pius Nonus, Póntifex 
 Máximus, in Sanctórum cánonem rétulit, ac Pius Papa Undécimus cæléstem 
 Patrónum Sacerdótum qui ad sacras populáres Missiónes in regiónibus 
 cathólicis ubíque terrárum incúmbunt, elégit et constítuit.
\switchcolumn
\selectlanguage{english}
At Rome, St. Leonard of Port 
 Maurice, priest and confessor of the Order of Friars Minor. He was 
 remarkable for his zeal for souls and his holy expeditions throughout Italy. 
 He was canonized by Pope Pius IX, and Pope Pius XI chose and appointed him 
 the heavenly patron of priests to the preaching of missions to the people.
\switchcolumn*
\selectlanguage{latin}
In território Rheménsi 
 natális sancti Básoli Confessóris.
\switchcolumn
\selectlanguage{english}
In the district of Rheims, the 
 birthday of St. Basolus, confessor.
\switchcolumn*
\selectlanguage{latin}
Hadrianópoli, in 
 Paphlagónia, sancti Styliáni Anachorétæ, miráculis clari.
\switchcolumn
\selectlanguage{english}
At Adrianople in Paphlagonia, St. 
 Stylian, anchoret, renowned for miracles.
\switchcolumn*
\selectlanguage{latin}
In Arménia sancti 
 Nicónis Mónachi.
\switchcolumn
\selectlanguage{english}
In Armenia, St. Nicon, monk.
\switchcolumn*
\selectlanguage{latin}
\end{paracol}


% ---- martyrology/mart11/mart1127.htm
\needspace{10\baselineskip}
\begin{paracol}{2}
\selectlanguage{latin}
\begin{center}{\color{gregoriocolor} Quinto Kaléndas Decémbris. 
 Luna\dots\ }\end{center}
\switchcolumn
\selectlanguage{english}
\begin{center}{\color{gregoriocolor} The 
 Twenty-Seventh Day of 
 November. The\dots\ Day of the Moon.}\end{center}
\end{paracol}

\noindent\begin{tabularx}{\linewidth}{*{19}{>{\centering\arraybackslash}X}}
 \textcolor{gregoriocolor}{a} & \textcolor{gregoriocolor}{b} & \textcolor{gregoriocolor}{c} & \textcolor{gregoriocolor}{d} & \textcolor{gregoriocolor}{e} & \textcolor{gregoriocolor}{f} & \textcolor{gregoriocolor}{g} & \textcolor{gregoriocolor}{h} & \textcolor{gregoriocolor}{i} & \textcolor{gregoriocolor}{k} & \textcolor{gregoriocolor}{l} & \textcolor{gregoriocolor}{m} & \textcolor{gregoriocolor}{n} & \textcolor{gregoriocolor}{p} & \textcolor{gregoriocolor}{q} & \textcolor{gregoriocolor}{r} & \textcolor{gregoriocolor}{s} & \textcolor{gregoriocolor}{t} & \textcolor{gregoriocolor}{u} \\
 7 & 8 & 9 & 10 & 11 & 12 & 13 & 14 & 15 & 16 & 17 & 18 & 19 & 20 & 21 & 22 & 23 & 24 & 25 \\
\end{tabularx}
\vspace{0.5\baselineskip}
\noindent\begin{tabularx}{\linewidth}{*{12}{>{\centering\arraybackslash}X}}
 \textcolor{gregoriocolor}{A} & \textcolor{gregoriocolor}{B} & \textcolor{gregoriocolor}{C} & \textcolor{gregoriocolor}{D} & \textcolor{gregoriocolor}{E} & F & \textcolor{gregoriocolor}{F} & \textcolor{gregoriocolor}{G} & \textcolor{gregoriocolor}{H} & \textcolor{gregoriocolor}{M} & \textcolor{gregoriocolor}{N} & \textcolor{gregoriocolor}{P} \\
 26 & 27 & 28 & 29 & 1 & 2 & 1 & 2 & 3 & 4 & 5 & 6 \\
\end{tabularx}

\begin{paracol}{2}
\selectlanguage{latin}
\lettrine[lines=2]{A}{ntiochíæ} sanctórum 
 Mártyrum Basiléi Epíscopi, Auxílii et Saturníni.
\switchcolumn
\selectlanguage{english}
\lettrine[lines=2]{A}{t} Antioch, the holy martyrs 
 Basileus, bishop, Auxilius, and Saturninus.
\switchcolumn*
\selectlanguage{latin}
Sebáste, in Arménia, sanctórum Mártyrum Hirenárchi, Acácii Presbyteri, ac septem mulíerum. Harum 
 porro constántia Hirenárchus commótus, ad Christum convérsus, sub 
 Diocletiáno Imperatóre et Máximo Præside, una cum Acácio, secúri percútitur.
\switchcolumn
\selectlanguage{english}
At Sebaste in Armenia, in the reign 
 of Emperor Diocletian and under the governor Maximus, the holy martyrs 
 Hirenarchus, the priest Acacius, and seven women. Struck with the 
 constancy of these women, Hirenarchus was converted to Christ, and with 
 Acacius died under the axe.
\switchcolumn*
\selectlanguage{latin}
Apud Cæam flúvium, in 
 Gallæcia, sanctórum Facúndi et Primitívi, qui sub Attico Præside passi sunt.
\switchcolumn
\selectlanguage{english}
In Galicia, on the River Cea, the 
 Saints Facundus and Primitivus, who suffered under the governor Atticus.
\switchcolumn*
\selectlanguage{latin}
In Pérside sancti 
 Jacóbi intercísi, Mártyris conspícui, qui, témpore Theodósii junióris, cum 
 in Isdegérdis Regis grátiam Christum negásset, et proptérea mater ejus et 
 uxor ab ipsíus se consuetúdine subtraxíssent, hinc, in se revérsus, 
 intrépide coram Vararáne, Isdegérdis fílio ac successóre, se Christiánum 
 esse conféssus est; ideóque ab iráto Rege, lata in eum mortis senténtia, 
 membrátim jussus est concídi et cápite obtruncári. Quo étiam témpore 
 innúmeri álii Mártyres ibídem passi sunt.
\switchcolumn
\selectlanguage{english}
In Persia, St. James Intercisus, a 
 distinguished martyr. In the time of Theodosius the Younger he denied 
 Christ in order to please King Isdegerd, but his mother and his wife for 
 this reason withdrew from his company. Coming to himself, he returned 
 to the king, now Vararanus, the son and successor of Isdegerd, to declare his faith in our Lord, whereupon the angry monarch 
 condemned him to be cut in pieces and beheaded. Countless other 
 martyrs suffered at this time in the same country.
\switchcolumn*
\selectlanguage{latin}
Aquiléjæ sancti 
 Valeriáni Epíscopi.
\switchcolumn
\selectlanguage{english}
At Aquileia, St. Valerian, bishop.
\switchcolumn*
\selectlanguage{latin}
Apud Régium, in Gállia, 
 sancti Máximi, Epíscopi et Confessóris; qui, usque a primævæ ætátis annis 
 omni virtútum grátia præditus, primum Lirinénsis cœnóbii Pater, deínde 
 Regiénsis Ecclésiæ Epíscopus, signis et prodígiis ínclytus éxstitit.
\switchcolumn
\selectlanguage{english}
At Riez in France, St. Maximus, 
 bishop and confessor, who, from his tender years, was endowed with every 
 grace and virtue. Being first superior of the monastery of Lerins, and 
 afterwards bishop of the Church of Riez, he was celebrated for the working 
 of miracles and prodigies.
\switchcolumn*
\selectlanguage{latin}
Salisbúrgi, in Nórico, 
 sancti Virgílii, Epíscopi et Carinthiórum Apóstoli, qui a Gregório Nono, 
 Pontífice Máximo, in Sanctórum númerum adscríptus est.
\switchcolumn
\selectlanguage{english}
At Salzburg in Austria, St. Virgil, 
 bishop and apostle of Carinthia, who was placed among the number of saints 
 by Pope Gregory IX.
\switchcolumn*
\selectlanguage{latin}
Apud Indos, Persis 
 finítimos, sanctórum Bárlaam et Jósaphat, quorum actus mirándos sanctus 
 Joánnes Damascénus conscrípsit.
\switchcolumn
\selectlanguage{english}
In India, near the Persian boundary, 
 the Saints Barlaam and Josaphat, whose wonderful deeds were written by St. 
 John of Damascus.
\switchcolumn*
\selectlanguage{latin}
Lutétiæ Parisiórum 
 deposítio sancti Severíni, Mónachi et Solitárii.
\switchcolumn
\selectlanguage{english}
At Paris, the death of St. Severin, 
 monk and solitary.
\switchcolumn*
\selectlanguage{latin}
\end{paracol}


% ---- martyrology/mart11/mart1128.htm
\needspace{10\baselineskip}
\begin{paracol}{2}
\selectlanguage{latin}
\begin{center}{\color{gregoriocolor} Quarto Kaléndas Decémbris. 
 Luna\dots\ }\end{center}
\switchcolumn
\selectlanguage{english}
\begin{center}{\color{gregoriocolor} The 
 Twenty-Eighth Day of 
 November. The\dots\ Day of the Moon.}\end{center}
\end{paracol}

\noindent\begin{tabularx}{\linewidth}{*{19}{>{\centering\arraybackslash}X}}
 \textcolor{gregoriocolor}{a} & \textcolor{gregoriocolor}{b} & \textcolor{gregoriocolor}{c} & \textcolor{gregoriocolor}{d} & \textcolor{gregoriocolor}{e} & \textcolor{gregoriocolor}{f} & \textcolor{gregoriocolor}{g} & \textcolor{gregoriocolor}{h} & \textcolor{gregoriocolor}{i} & \textcolor{gregoriocolor}{k} & \textcolor{gregoriocolor}{l} & \textcolor{gregoriocolor}{m} & \textcolor{gregoriocolor}{n} & \textcolor{gregoriocolor}{p} & \textcolor{gregoriocolor}{q} & \textcolor{gregoriocolor}{r} & \textcolor{gregoriocolor}{s} & \textcolor{gregoriocolor}{t} & \textcolor{gregoriocolor}{u} \\
 8 & 9 & 10 & 11 & 12 & 13 & 14 & 15 & 16 & 17 & 18 & 19 & 20 & 21 & 22 & 23 & 24 & 25 & 26 \\
\end{tabularx}
\vspace{0.5\baselineskip}
\noindent\begin{tabularx}{\linewidth}{*{12}{>{\centering\arraybackslash}X}}
 \textcolor{gregoriocolor}{A} & \textcolor{gregoriocolor}{B} & \textcolor{gregoriocolor}{C} & \textcolor{gregoriocolor}{D} & \textcolor{gregoriocolor}{E} & F & \textcolor{gregoriocolor}{F} & \textcolor{gregoriocolor}{G} & \textcolor{gregoriocolor}{H} & \textcolor{gregoriocolor}{M} & \textcolor{gregoriocolor}{N} & \textcolor{gregoriocolor}{P} \\
 27 & 28 & 29 & 1 & 2 & 3 & 2 & 3 & 4 & 5 & 6 & 7 \\
\end{tabularx}

\begin{paracol}{2}
\selectlanguage{latin}
\lettrine[lines=2]{A}{pud} Corínthum natális 
 sancti Sósthenis, ex beáti Pauli Apóstoli discípulis; cujus mentiónem facit 
 idem Apóstolus Corínthiis scribens. Ipse autem Sósthenes, ex príncipe 
 Synagógæ convérsus ad Christum, fídei suæ primórdia, ante Galliónem 
 Procónsulem ácriter verberátus, præcláro inítio consecrávit.
\switchcolumn
\selectlanguage{english}
\lettrine[lines=2]{A}{t} Corinth, the birthday of St. 
 Sosthenes, disciple of the blessed apostle Paul, who is mentioned in his 
 Epistle to the Corinthians. He was chief of the synagogue when 
 converted to Christ, and as a glorious beginning, consecrated the first 
 fruits of his faith by being scourged before the proconsul Gallio.
\switchcolumn*
\selectlanguage{latin}
Romæ sancti Rufi, quem, 
 cum omni família sua, Christi Mártyrem Diocletiánus fecit.
\switchcolumn
\selectlanguage{english}
At Rome, St. Rufus, who was martyred 
 with all his family by Diocletian.
\switchcolumn*
\selectlanguage{latin}
In Africa sanctórum 
 Mártyrum Papiniáni et Mansuéti Episcopórum, qui, témpore Wandálicæ 
 persecutiónis, sub Rege Ariáno Genseríco, pro fídei cathólicæ defensióne, 
 candéntibus ferri láminis toto córpore adústi, gloriósum agónem consummárunt. 
 Quo étiam témpore álii novem sancti Epíscopi, scílicet Valeriánus, Urbánus, 
 Crescens, Eustáchius, Crescónius, Crescentiánus, Felix, Hortulánus et 
 Florentiánus, damnáti exsílio, cursum vitæ suæ implevérunt.
\switchcolumn
\selectlanguage{english}
In Africa, under the Arian king 
 Genseric, in the persecution of the Vandals, the holy martyrs Papinian and 
 Mansuetus, bishops, who, for the Catholic faith, were burned in every part 
 of their bodies with hot plates of iron, which ended their glorious trial. 
 At this time also, other holy bishops, Valerian, Urban, Crescens, Eustachius, 
 Cresconius, Crescentian, Felix, Hortulanus, and Florentian ended the course 
 of their lives in exile.
\switchcolumn*
\selectlanguage{latin}
Constantinópoli 
 sanctórum Mártyrum Stéphani junióris, Basilíi, Petri, Andréæ, et Sociórum 
 trecentórum et trigínta novem Monachórum; qui, sub Constantíno Coprónymo, 
 pro sanctárum Imáginum cultu váriis excruciáti supplíciis, veritátem 
 cathólicam effúso sánguine confirmárunt.
\switchcolumn
\selectlanguage{english}
At Constantinople, in the time of 
 Constantine Copronymus, the holy martyrs Stephen the Younger, Basil, Peter, 
 Andrew, and their companions, numbering three hundred and thirty-nine monks, 
 who were subjected to diverse torments for the veneration of holy images, 
 and confirmed the Catholic truth with the shedding of their blood.
\switchcolumn*
\selectlanguage{latin}
Neápoli, in Campánia, deposítio sancti Jacóbi Picéni, Sacerdótis ex Ordine Minórum et Confessóris, 
 vitæ asperitáte, apostólica prædicatióne ac plúribus pro re Christiána óbitis legatiónibus præclári; quem Benedíctus Décimus tértius, Póntifex 
 Máximus, Sanctórum fastis adjúnxit.
\switchcolumn
\selectlanguage{english}
At Naples in Campania, the death of 
 St. James della Marca, priest and confessor of the Order of Friars Minor, 
 celebrated for the austerity of his life, his apostolic preaching, and his 
 many diplomatic missions undertaken for the success of the affairs of 
 Christianity. His name was added to the calendar of the saints by the 
 Sovereign Pontiff, Benedict XIII.
\switchcolumn*
\selectlanguage{latin}
\end{paracol}


% ---- martyrology/mart11/mart1129.htm
\needspace{10\baselineskip}
\begin{paracol}{2}
\selectlanguage{latin}
\begin{center}{\color{gregoriocolor} Tértio Kaléndas Decémbris. 
 Luna\dots\ }\end{center}
\switchcolumn
\selectlanguage{english}
\begin{center}{\color{gregoriocolor} The 
 Twenty-Ninth Day of 
 November. The\dots\ Day of the Moon.}\end{center}
\end{paracol}

\noindent\begin{tabularx}{\linewidth}{*{19}{>{\centering\arraybackslash}X}}
 \textcolor{gregoriocolor}{a} & \textcolor{gregoriocolor}{b} & \textcolor{gregoriocolor}{c} & \textcolor{gregoriocolor}{d} & \textcolor{gregoriocolor}{e} & \textcolor{gregoriocolor}{f} & \textcolor{gregoriocolor}{g} & \textcolor{gregoriocolor}{h} & \textcolor{gregoriocolor}{i} & \textcolor{gregoriocolor}{k} & \textcolor{gregoriocolor}{l} & \textcolor{gregoriocolor}{m} & \textcolor{gregoriocolor}{n} & \textcolor{gregoriocolor}{p} & \textcolor{gregoriocolor}{q} & \textcolor{gregoriocolor}{r} & \textcolor{gregoriocolor}{s} & \textcolor{gregoriocolor}{t} & \textcolor{gregoriocolor}{u} \\
 9 & 10 & 11 & 12 & 13 & 14 & 15 & 16 & 17 & 18 & 19 & 20 & 21 & 22 & 23 & 24 & 25 & 26 & 27 \\
\end{tabularx}
\vspace{0.5\baselineskip}
\noindent\begin{tabularx}{\linewidth}{*{12}{>{\centering\arraybackslash}X}}
 \textcolor{gregoriocolor}{A} & \textcolor{gregoriocolor}{B} & \textcolor{gregoriocolor}{C} & \textcolor{gregoriocolor}{D} & \textcolor{gregoriocolor}{E} & F & \textcolor{gregoriocolor}{F} & \textcolor{gregoriocolor}{G} & \textcolor{gregoriocolor}{H} & \textcolor{gregoriocolor}{M} & \textcolor{gregoriocolor}{N} & \textcolor{gregoriocolor}{P} \\
 28 & 29 & 1 & 2 & 3 & 4 & 3 & 4 & 5 & 6 & 7 & 8 \\
\end{tabularx}

\begin{paracol}{2}
\selectlanguage{latin}
\lettrine[lines=1]{V}{igília} sancti Andréæ 
 Apóstoli.
\switchcolumn
\selectlanguage{english}
\lettrine[lines=1]{T}{he} Vigil of St. Andrew, apostle.
\switchcolumn*
\selectlanguage{latin}
Romæ, via Salária, 
 natális sanctórum Mártyrum Saturníni senis, et Sisínii Diáconi, sub 
 Maximiáno Príncipe; quos, diu in cárcere macerátos, jussit Urbis Præféctus 
 in equúleum levári et áttrahi nervis, fústibus ac scorpiónibus cædi, deínde 
 eis flammas appóni, et, depósitos de equúleo, cápite truncári.
\switchcolumn
\selectlanguage{english}
At Rome, on the Salarian Way, the 
 birthday of the holy martyr, Saturninus, an aged man, and the deacon 
 Sisinius, in the time of Emperor Maximian. After a long imprisonment, 
 by order of the prefect of the city they were placed on the rack, stretched 
 with ropes, scourged with rods and whips garnished with metal, then exposed 
 to the flames, taken down from the rack and beheaded.
\switchcolumn*
\selectlanguage{latin}
Tolósæ sancti Saturníni 
 Epíscopi, qui, tempóribus Décii, in Capitólio ejúsdem urbis a Pagánis tentus 
 atque de summa Capitólii arce per omnes gradus præcipitátus est, atque ita, 
 cápite collíso, excussóque cérebro, et toto córpore dilaniáto, dignam 
 Christo ánimam réddidit.
\switchcolumn
\selectlanguage{english}
At Toulouse, in the time of Decius, 
 the holy bishop Saturninus, who was taken to the capitol of that city by the 
 heathen and thrown down the steps from the highest part of the building. 
 The fall having crushed his head, dashed out his brain and mangled his whole 
 body, he rendered his worthy soul to our Lord.
\switchcolumn*
\selectlanguage{latin}
Item pássio sanctórum 
 Parámonis et Sociórum trecentórum septuagínta quinque, sub Décio Imperatóre 
 et Aquilíno Præside.
\switchcolumn
\selectlanguage{english}
Also, the martyrdom of the Saints 
 Paramon and his companions, to the number of three hundred and seventy-five 
 under Emperor Decius and the governor Aquilinus.
\switchcolumn*
\selectlanguage{latin}
Ancyræ, in Galátia, sancti Philómeni Mártyris, qui, in persecutióne Aureliáni Imperatóris, sub 
 Felíce Præside, igne probátus, mánibus pedibúsque ac demum cápite clavis 
 confíxo, martyrium consummávit.
\switchcolumn
\selectlanguage{english}
At Ancyra in Galatia, St. Philomenus, 
 martyr. During the persecution of Emperor Aurelian, under the governor 
 Felix, he was first exposed to the flames, then having his hands, feet, and 
 head pierced with nails, he fulfilled his martyrdom.
\switchcolumn*
\selectlanguage{latin}
Vérulis, in Hérnicis, 
 sanctórum Mártyrum Blásii et Demétrii.
\switchcolumn
\selectlanguage{english}
At Veroli, the holy martyrs Blaise 
 and Demetrius.
\switchcolumn*
\selectlanguage{latin}
Tudérti, in Umbria, 
 sanctæ Illuminátæ Vírginis.
\switchcolumn
\selectlanguage{english}
At Todi in Umbria, St. Illuminata, 
 virgin.
\switchcolumn*
\selectlanguage{latin}
\end{paracol}


% ---- martyrology/mart11/mart1130.htm
\needspace{10\baselineskip}
\begin{paracol}{2}
\selectlanguage{latin}
\begin{center}{\color{gregoriocolor} Prídie Kaléndas Decémbris. 
 Luna\dots\ }\end{center}
\switchcolumn
\selectlanguage{english}
\begin{center}{\color{gregoriocolor} The 
 Thirtieth Day of 
 November. The\dots\ Day of the Moon.}\end{center}
\end{paracol}

\noindent\begin{tabularx}{\linewidth}{*{19}{>{\centering\arraybackslash}X}}
 \textcolor{gregoriocolor}{a} & \textcolor{gregoriocolor}{b} & \textcolor{gregoriocolor}{c} & \textcolor{gregoriocolor}{d} & \textcolor{gregoriocolor}{e} & \textcolor{gregoriocolor}{f} & \textcolor{gregoriocolor}{g} & \textcolor{gregoriocolor}{h} & \textcolor{gregoriocolor}{i} & \textcolor{gregoriocolor}{k} & \textcolor{gregoriocolor}{l} & \textcolor{gregoriocolor}{m} & \textcolor{gregoriocolor}{n} & \textcolor{gregoriocolor}{p} & \textcolor{gregoriocolor}{q} & \textcolor{gregoriocolor}{r} & \textcolor{gregoriocolor}{s} & \textcolor{gregoriocolor}{t} & \textcolor{gregoriocolor}{u} \\
 10 & 11 & 12 & 13 & 14 & 15 & 16 & 17 & 18 & 19 & 20 & 21 & 22 & 23 & 24 & 25 & 26 & 27 & 28 \\
\end{tabularx}
\vspace{0.5\baselineskip}
\noindent\begin{tabularx}{\linewidth}{*{12}{>{\centering\arraybackslash}X}}
 \textcolor{gregoriocolor}{A} & \textcolor{gregoriocolor}{B} & \textcolor{gregoriocolor}{C} & \textcolor{gregoriocolor}{D} & \textcolor{gregoriocolor}{E} & F & \textcolor{gregoriocolor}{F} & \textcolor{gregoriocolor}{G} & \textcolor{gregoriocolor}{H} & \textcolor{gregoriocolor}{M} & \textcolor{gregoriocolor}{N} & \textcolor{gregoriocolor}{P} \\
 29 & 1 & 2 & 3 & 4 & 5 & 4 & 5 & 6 & 7 & 8 & 9 \\
\end{tabularx}

\begin{paracol}{2}
\selectlanguage{latin}
\lettrine[lines=2]{A}{pud} Patras, in Achája, 
 natális sancti Andréæ Apóstoli, qui in Thrácia et Scythia sacrum Christi 
 Evangélium prædicávit. Is, ab Ægéa Proncónsule comprehénsus, primum in cárcere clausus est, deínde gravíssime cæsus, ad 
 últimum suspénsus in cruce, 
 in ea pópulum docens bíduo supervíxit; et, rogáto Dómino ne eum síneret de 
 cruce depóni, circúmdatus est magno splendóre de cælo, et, abscedénte 
 póstmodum lúmine, emísit spíritum.
\switchcolumn
\selectlanguage{english}
\lettrine[lines=2]{A}{t} Patras in Achaia, the birthday of 
 the apostle St. Andrew, who preached the gospel of Christ in Thrace and 
 Sythia. He was apprehended by the proconsul Aegeas, imprisoned, and 
 severely scourged, and finally, being hung on a cross, he lived two days on 
 it, teaching the people. Having besought our Lord not to permit him to 
 be taken down from the cross, he was surrounded with a great brightness from 
 heaven, and when the light disappeared he breathed his last.
\switchcolumn*
\selectlanguage{latin}
Romæ pássio sanctórum 
 Cástuli et Euprépitis.
\switchcolumn
\selectlanguage{english}
At Rome, the martyrdom of the Saints 
 Castulus and Euprepis.
\switchcolumn*
\selectlanguage{latin}
Constantinópoli sanctæ 
 Mauræ, Vírginis et Mártyris.
\switchcolumn
\selectlanguage{english}
At Constantinople, St. Maura, virgin 
 and martyr.
\switchcolumn*
\selectlanguage{latin}
Item sanctæ Justínæ, 
 Vírginis et Mártyris.
\switchcolumn
\selectlanguage{english}
Also, St. Justina, virgin and 
 martyr.
\switchcolumn*
\selectlanguage{latin}
Romæ sancti Constántii 
 Confessóris, qui, fórtiter Pelagiánis resístens, ab eórum factióne pértulit 
 multa, quæ illum sanctis Confessóribus sociárunt.
\switchcolumn
\selectlanguage{english}
At Rome, St. Constantius, confessor, 
 who bravely opposed the Pelagians, and by enduring many injuries from them, 
 gained a place among the holy confessors.
\switchcolumn*
\selectlanguage{latin}
Apud Sántonas, in 
 Gállia, sancti Trojáni Epíscopi, magnæ sanctitátis viri, qui, sepúltus in 
 terris, se in cælis vívere multis virtútibus maniféstat.
\switchcolumn
\selectlanguage{english}
At Saintes in France, St. Trojan, 
 bishop and confessor, a man of great sanctity, who shews by many miracles 
 that he lives in heaven, though his body is buried on earth.
\switchcolumn*
\selectlanguage{latin}
In Palæstína beáti 
 Zósimi Confessóris, qui, sub Justíno Imperatóre, sanctitáte et miráculis 
 fuit insígnis.
\switchcolumn
\selectlanguage{english}
In Palestine, blessed Zosimus, 
 confessor, who was distinguished for his sanctity and miracles in the time 
 of Emperor Justin.
\switchcolumn*
\selectlanguage{latin}
\end{paracol}

\setrunningtitles{December}{December}

% ---- martyrology/mart12/mart1201.htm
\needspace{10\baselineskip}
\begin{paracol}{2}
\selectlanguage{latin}
\begin{center}{\color{gregoriocolor} Kaléndis Decémbris. 
 Luna\dots\ }\end{center}
\switchcolumn
\selectlanguage{english}
\begin{center}{\color{gregoriocolor} The 
 First Day of 
 December. The\dots\ Day of the Moon.}\end{center}
\end{paracol}

\noindent\begin{tabularx}{\linewidth}{*{19}{>{\centering\arraybackslash}X}}
 \textcolor{gregoriocolor}{a} & \textcolor{gregoriocolor}{b} & \textcolor{gregoriocolor}{c} & \textcolor{gregoriocolor}{d} & \textcolor{gregoriocolor}{e} & \textcolor{gregoriocolor}{f} & \textcolor{gregoriocolor}{g} & \textcolor{gregoriocolor}{h} & \textcolor{gregoriocolor}{i} & \textcolor{gregoriocolor}{k} & \textcolor{gregoriocolor}{l} & \textcolor{gregoriocolor}{m} & \textcolor{gregoriocolor}{n} & \textcolor{gregoriocolor}{p} & \textcolor{gregoriocolor}{q} & \textcolor{gregoriocolor}{r} & \textcolor{gregoriocolor}{s} & \textcolor{gregoriocolor}{t} & \textcolor{gregoriocolor}{u} \\
 11 & 12 & 13 & 14 & 15 & 16 & 17 & 18 & 19 & 20 & 21 & 22 & 23 & 24 & 25 & 26 & 27 & 28 & 29 \\
\end{tabularx}
\vspace{0.5\baselineskip}
\noindent\begin{tabularx}{\linewidth}{*{12}{>{\centering\arraybackslash}X}}
 \textcolor{gregoriocolor}{A} & \textcolor{gregoriocolor}{B} & \textcolor{gregoriocolor}{C} & \textcolor{gregoriocolor}{D} & \textcolor{gregoriocolor}{E} & F & \textcolor{gregoriocolor}{F} & \textcolor{gregoriocolor}{G} & \textcolor{gregoriocolor}{H} & \textcolor{gregoriocolor}{M} & \textcolor{gregoriocolor}{N} & \textcolor{gregoriocolor}{P} \\
 1 & 2 & 3 & 4 & 5 & 6 & 5 & 6 & 7 & 8 & 9 & 10 \\
\end{tabularx}

\begin{paracol}{2}
\selectlanguage{latin}
\lettrine[lines=1]{S}{ancti} Nahum Prophétæ, in Bégabar quiescéntis.
\switchcolumn
\selectlanguage{english}
\lettrine[lines=1]{T}{he} prophet Nahum, who was buried in Bagabar.
\switchcolumn*
\selectlanguage{latin}
Romæ sanctórum Mártyrum 
 Diodóri Presbyteri, et Mariáni Diáconi, cum áliis plúribus, qui, sub 
 Numeriáno Príncipe, cum in Arenário natalítia Mártyrum ágerent, illic, 
 obstrúcta a persecutóribus jánua cryptæ ac díruta désuper mole, martyrii 
 glóriam meruérunt.
\switchcolumn
\selectlanguage{english}
At Rome, the holy martyrs Diodorus, 
 a priest, and Marian, a deacon, with many others, while they were observing 
 the birthdays of the martyrs in the catacombs. They were made 
 partakers in the glory of martyrdom when the persecutors, by order of 
 Emperor Numerian, walled up the door of the oratory and piled up a great 
 mass of stones against it.
\switchcolumn*
\selectlanguage{latin}
Item Romæ pássio 
 sanctórum Lúcii, Rogáti, Cassiáni et Cándidæ.
\switchcolumn
\selectlanguage{english}
Also in Rome, the martyrdom of the 
 Saints Lucius, Rogatus, Cassian, and Candida.
\switchcolumn*
\selectlanguage{latin}
Nárniæ sancti Próculi, 
 Epíscopi et Mártyris; qui, post multa egrégia ópera, a Rege Gothórum Tótila 
 jussus est decollári.
\switchcolumn
\selectlanguage{english}
At Narni, St. Proculus, bishop and 
 martyr, who, after performing many good works, was beheaded by order of 
 Totila, king of the Goths.
\switchcolumn*
\selectlanguage{latin}
In civitáte Casalénsi 
 sancti Evásii, Epíscopi et Mártyris.
\switchcolumn
\selectlanguage{english}
At Casale, St. Evasius, bishop and 
 martyr.
\switchcolumn*
\selectlanguage{latin}
Eódem die sancti Ansáni 
 Mártyris, qui, sub Diocletiáno Imperatóre, Romæ conféssus Christum et in cárcerem trusus, deínde Senas, in Túscia, perdúctus, ibídem cápitis 
 obtruncatióne cursum martyrii perfécit.
\switchcolumn
\selectlanguage{english}
The same day, St. Ansanus, martyr, 
 who confessed Christ at Rome, and was cast into prison in the time of 
 Emperor Diocletian. Afterwards he was taken to Siena in Tuscany, where 
 he ended the course of his martyrdom by beheading.
\switchcolumn*
\selectlanguage{latin}
Amériæ, in Umbria, 
 sancti Olympíadis, viri Consuláris, qui a beáta Firmína ad fidem est 
 convérsus, et sub Diocletiáno, in equúleo tortus, martyrium consummávit.
\switchcolumn
\selectlanguage{english}
At Amelia in Umbria, St. Olympias, 
 ex-consul, who was converted to the faith by blessed Firmina, was tortured 
 on the rack, and under Diocletian achieved martyrdom.
\switchcolumn*
\selectlanguage{latin}
Arbéle, in Pérside, 
 sancti Anániæ Mártyris.
\switchcolumn
\selectlanguage{english}
At Arbela in Persia, St. Ananias, 
 martyr.
\switchcolumn*
\selectlanguage{latin}
Medioláni sancti 
 Castritiáni Epíscopi, qui, in máxima Ecclésiæ perturbatióne, virtútum 
 méritis ac rerum pie religioséque gestárum laude enítuit.
\switchcolumn
\selectlanguage{english}
At Milan, St. Castritian, bishop, 
 who was eminent for virtues and the practice of pious and religious deeds 
 during the greatest troubles of the Church.
\switchcolumn*
\selectlanguage{latin}
Bríxiæ sancti Ursicíni 
 Epíscopi.
\switchcolumn
\selectlanguage{english}
At Brescia, St. Ursicinus, bishop.
\switchcolumn*
\selectlanguage{latin}
Noviómi, in Bélgio, 
 sancti Elígii Epíscopi, cujus vitam admirándam múltiplex signórum númerus 
 comméndat.
\switchcolumn
\selectlanguage{english}
At Noyon in Belgium, St. Eligius, 
 bishop, whose life is rendered illustrious by a considerable number of 
 miracles.
\switchcolumn*
\selectlanguage{latin}
Apud Virodúnum, in 
 Gállia, sancti Ageríci Epíscopi.
\switchcolumn
\selectlanguage{english}
At Verdun in France, St. Agericus, 
 bishop.
\switchcolumn*
\selectlanguage{latin}
Eódem die sanctæ 
 Natalítiæ, uxóris beáti Hadriáni Mártyris, quæ, sub Diocletiáno Imperatóre, 
 sanctis Martyribus, Nicomedíæ in cárcere deténtis, multo témpore ministrávit; 
 impletóque eórum certámine, Constantinópolim est profécta, et ibídem in pace 
 quiévit.
\switchcolumn
\selectlanguage{english}
The same day, St. Natalia, wife of 
 the blessed martyr Adrian, in the time of Emperor Diocletian. She long 
 served the holy martyrs imprisoned at Nicomedia, and when their trials were 
 over, went to Constantinople where she peacefully went to her rest in the 
 Lord.
\switchcolumn*
\selectlanguage{latin}
\end{paracol}


% ---- martyrology/mart12/mart1202.htm
\needspace{10\baselineskip}
\begin{paracol}{2}
\selectlanguage{latin}
\begin{center}{\color{gregoriocolor} Quarto Nonas Decémbris. 
 Luna\dots\ }\end{center}
\switchcolumn
\selectlanguage{english}
\begin{center}{\color{gregoriocolor} The 
 Second Day of 
 December. The\dots\ Day of the Moon.}\end{center}
\end{paracol}

\noindent\begin{tabularx}{\linewidth}{*{19}{>{\centering\arraybackslash}X}}
 \textcolor{gregoriocolor}{a} & \textcolor{gregoriocolor}{b} & \textcolor{gregoriocolor}{c} & \textcolor{gregoriocolor}{d} & \textcolor{gregoriocolor}{e} & \textcolor{gregoriocolor}{f} & \textcolor{gregoriocolor}{g} & \textcolor{gregoriocolor}{h} & \textcolor{gregoriocolor}{i} & \textcolor{gregoriocolor}{k} & \textcolor{gregoriocolor}{l} & \textcolor{gregoriocolor}{m} & \textcolor{gregoriocolor}{n} & \textcolor{gregoriocolor}{p} & \textcolor{gregoriocolor}{q} & \textcolor{gregoriocolor}{r} & \textcolor{gregoriocolor}{s} & \textcolor{gregoriocolor}{t} & \textcolor{gregoriocolor}{u} \\
 12 & 13 & 14 & 15 & 16 & 17 & 18 & 19 & 20 & 21 & 22 & 23 & 24 & 25 & 26 & 27 & 28 & 29 & 1 \\
\end{tabularx}
\vspace{0.5\baselineskip}
\noindent\begin{tabularx}{\linewidth}{*{12}{>{\centering\arraybackslash}X}}
 \textcolor{gregoriocolor}{A} & \textcolor{gregoriocolor}{B} & \textcolor{gregoriocolor}{C} & \textcolor{gregoriocolor}{D} & \textcolor{gregoriocolor}{E} & F & \textcolor{gregoriocolor}{F} & \textcolor{gregoriocolor}{G} & \textcolor{gregoriocolor}{H} & \textcolor{gregoriocolor}{M} & \textcolor{gregoriocolor}{N} & \textcolor{gregoriocolor}{P} \\
 2 & 3 & 4 & 5 & 6 & 7 & 6 & 7 & 8 & 9 & 10 & 11 \\
\end{tabularx}

\begin{paracol}{2}
\selectlanguage{latin}
\lettrine[lines=2]{R}{omæ} pássio sanctæ 
 Bibiánæ, Vírginis et Mártyris, quæ, sub Juliáno Imperatóre sacrílego, ob 
 Christum támdiu plumbátis cæsa est, donec rédderet spíritum.
\switchcolumn
\selectlanguage{english}
\lettrine[lines=2]{A}{t} Rome, the martyrdom of the 
 saintly virgin Bibiana, under the sacrilegious Emperor Julian. For the 
 sake of our Lord she was scourged with leaded whips until she expired.
\switchcolumn*
\selectlanguage{latin}
Apud Forum Cornélii, in 
 Æmília, natális sancti Petri, Epíscopi Ravennátis, Confessóris et Ecclésiæ 
 Doctóris, cognoménto Chrysólogi, doctrína et sanctitáte célebris. 
 Ipsíus tamen festum prídie Nonas hujus mensis recólitur.
\switchcolumn
\selectlanguage{english}
At Imola, St. Peter Chrysologus, 
 bishop of Ravenna, confessor and doctor of the Church, celebrated for his 
 learning and sanctity. His feast is celebrated on the 4th of this 
 month.
\switchcolumn*
\selectlanguage{latin}
In Sanciáno, Sinárum ínsula, item natális sancti Francísci Xavérii, Sacerdótis e Societáte Jesu 
 et Confessóris, Indiárum Apóstoli, géntium conversióne, donis et miráculis 
 clari; qui plenus méritis et labóribus obdormívit in Dómino. Ipsum 
 beátum virum Pius Décimus, Póntifex Máximus, cæléstem sodalitáti et óperi 
 Propagándæ Fídei Protectórem elégit atque constítuit; Pius vero Papa 
 Undécimus peculiárem ómnibus Missiónibus Patrónum dedit et confirmávit. 
 Ejus autem festívitas, jussu Alexándri Papæ Séptimi, sequénti die celebrátur.
\switchcolumn
\selectlanguage{english}
In Sanchan, an island of China, the 
 birthday of St. Francis Xavier, priest of the Society of Jesus, confessor 
 and Apostle of the Indies. He was renowned for his conversion of the 
 heathen, his gifts and miracles, and he was filled with merits and good 
 works when he fell asleep in the Lord. Pope Pius X chose and appointed 
 him the heavenly protector of the Society for the Propagation of the Faith 
 and of the work for the same object. Pope Pius XI confirmed this and 
 appointed him the special patron of all the Foreign Missions. His 
 feast, by decree of Pope Alexander VII, is kept on the following day.
\switchcolumn*
\selectlanguage{latin}
Romæ sanctórum Mártyrum 
 Eusébii Presbyteri, Marcélli Diáconi, Hippólyti, Máximi, Adriæ, Paulínæ, 
 Neónis, Maríæ, Martánæ et Auréliæ; qui omnes in persecutióne Valeriáni, sub 
 Secundiáno Júdice, martyrium complevérunt.
\switchcolumn
\selectlanguage{english}
At Rome, the holy martyrs Eusebius, 
 a priest, Marcellus, a deacon, Hippolytus, Maximus, Adria, Paulina, Neon, 
 Mary, Martana, and Aurelia, who fulfilled their martyrdoms under the judge 
 Secundian in the persecution of Valerian.
\switchcolumn*
\selectlanguage{latin}
Item Romæ sancti 
 Pontiáni Mártyris, cum áliis quátuor.
\switchcolumn
\selectlanguage{english}
Also at Rome, St. Pontian, martyr, 
 with four others.
\switchcolumn*
\selectlanguage{latin}
In Africa natális 
 sanctórum Mártyrum Sevéri, Secúri, Januárii et Victoríni; qui ibídem 
 martyrio coronáti sunt.
\switchcolumn
\selectlanguage{english}
In Africa, the birthday of the holy 
 martyrs Severus, Securus, Januarius, and Victorinus, who were there crowned 
 with martyrdom.
\switchcolumn*
\selectlanguage{latin}
Aquiléjæ sancti 
 Chromátii, Epíscopi et Confessóris.
\switchcolumn
\selectlanguage{english}
At Aquileia, St. Chromatius, bishop 
 and confessor.
\switchcolumn*
\selectlanguage{latin}
Verónæ sancti Lupi, 
 Epíscopi et Confessóris.
\switchcolumn
\selectlanguage{english}
At Verona, St. Lupus, bishop and 
 confessor.
\switchcolumn*
\selectlanguage{latin}
Edéssæ, in Syria, 
 sancti Nonni Epíscopi, cujus précibus Pelágia pænitens ad Christum convérsa 
 est.
\switchcolumn
\selectlanguage{english}
At Edessa in Syria, St. Nonnus, 
 bishop, by whose prayers Pelagia the penitent was converted to Christ.
\switchcolumn*
\selectlanguage{latin}
Tróade, in Phrygia, 
 sancti Silváni Epíscopi, miráculis clari.
\switchcolumn
\selectlanguage{english}
At Troas in Phrygia, St. Silvanus, 
 bishop, renowned for miracles.
\switchcolumn*
\selectlanguage{latin}
Bríxiæ sancti Evásii 
 Epíscopi.
\switchcolumn
\selectlanguage{english}
At Brescia, St. Evasius, bishop.
\switchcolumn*
\selectlanguage{latin}
\end{paracol}


% ---- martyrology/mart12/mart1203.htm
\needspace{10\baselineskip}
\begin{paracol}{2}
\selectlanguage{latin}
\begin{center}{\color{gregoriocolor} Tértio Nonas Decémbris. 
 Luna\dots\ }\end{center}
\switchcolumn
\selectlanguage{english}
\begin{center}{\color{gregoriocolor} The 
 Third Day of 
 December. The\dots\ Day of the Moon.}\end{center}
\end{paracol}

\noindent\begin{tabularx}{\linewidth}{*{19}{>{\centering\arraybackslash}X}}
 \textcolor{gregoriocolor}{a} & \textcolor{gregoriocolor}{b} & \textcolor{gregoriocolor}{c} & \textcolor{gregoriocolor}{d} & \textcolor{gregoriocolor}{e} & \textcolor{gregoriocolor}{f} & \textcolor{gregoriocolor}{g} & \textcolor{gregoriocolor}{h} & \textcolor{gregoriocolor}{i} & \textcolor{gregoriocolor}{k} & \textcolor{gregoriocolor}{l} & \textcolor{gregoriocolor}{m} & \textcolor{gregoriocolor}{n} & \textcolor{gregoriocolor}{p} & \textcolor{gregoriocolor}{q} & \textcolor{gregoriocolor}{r} & \textcolor{gregoriocolor}{s} & \textcolor{gregoriocolor}{t} & \textcolor{gregoriocolor}{u} \\
 13 & 14 & 15 & 16 & 17 & 18 & 19 & 20 & 21 & 22 & 23 & 24 & 25 & 26 & 27 & 28 & 29 & 1 & 2 \\
\end{tabularx}
\vspace{0.5\baselineskip}
\noindent\begin{tabularx}{\linewidth}{*{12}{>{\centering\arraybackslash}X}}
 \textcolor{gregoriocolor}{A} & \textcolor{gregoriocolor}{B} & \textcolor{gregoriocolor}{C} & \textcolor{gregoriocolor}{D} & \textcolor{gregoriocolor}{E} & F & \textcolor{gregoriocolor}{F} & \textcolor{gregoriocolor}{G} & \textcolor{gregoriocolor}{H} & \textcolor{gregoriocolor}{M} & \textcolor{gregoriocolor}{N} & \textcolor{gregoriocolor}{P} \\
 3 & 4 & 5 & 6 & 7 & 8 & 7 & 8 & 9 & 10 & 11 & 12 \\
\end{tabularx}

\begin{paracol}{2}
\selectlanguage{latin}
\lettrine[lines=2]{S}{ancti} Francísci 
 Xavérii, Sacerdótis e Societáte Jesu et Confessóris, Indiárum Apóstoli, 
 sodalitátis et óperis Propagándæ Fídei atque Missiónum ómnium Patróni 
 cæléstis; qui prídie hujus diéi quiévit in pace.
\switchcolumn
\selectlanguage{english}
\lettrine[lines=2]{S}{t.} Francis Xavier, priest of the 
 Society of Jesus, confessor, Apostle of the Indies, and heavenly patron of 
 the Society for the Propagation of the Faith, and also of all the Missions, 
 who died on the day previous.
\switchcolumn*
\selectlanguage{latin}
In Judæa sancti 
 Sophóniæ Prophétæ.
\switchcolumn
\selectlanguage{english}
In Judea, the holy prophet Zephaniah.
\switchcolumn*
\selectlanguage{latin}
Romæ sanctórum Mártyrum 
 Cláudii Tribúni, et uxóris Hiláriæ, ac filiórum Jásonis et Mauri, cum 
 septuagínta milítibus. Ex eis Cláudium jussit Numeriánus Imperátor, 
 ingénti saxo alligátum, in flumen præcípitem dari; mílites
 vero et ipsíus 
 Cláudii fílios capitáli senténtia puníri. Beáta autem Hilária, cum 
 filiórum córpora sepelísset, paulo post, orans ad eórum sepúlcrum, tenta est 
 a Pagánis, et, in cárcerem trusa, migrávit ad Dóminum.
\switchcolumn
\selectlanguage{english}
At Rome, the holy martyrs Claudius, 
 a tribune, and Hilaria, his wife, with Jason and Maur, their sons, and 
 seventy soldiers. By the command of Emperor Numerian, Claudius was 
 fastened to a large stone and thrown into the river, the soldiers and the 
 sons of Claudius were condemned to capital punishment. But blessed 
 Hilaria, after having buried the bodies of her sons, and while praying at 
 their tomb, was arrested by the pagans, and shortly after departed for 
 heaven.
\switchcolumn*
\selectlanguage{latin}
Tingi, in Mauritánia, pássio sancti Cassiáni Mártyris, qui, cum exceptóris diu gessísset offícium, 
 tandem, admirátus intrépida beáti Marcélli Centuriónis respónsa et immóbilem 
 in Christi fide constántiam, atque cælitus inspirátus, exsecrábile duxit 
 Christianórum neci deservíri; ideóque, cum renuntiásset eídem offício, et 
 ipse, sub Christiána professióne, cápite abscíssus, triúmphum méruit 
 obtinére martyrii.
\switchcolumn
\selectlanguage{english}
At Tangier in Morocco, St. Cassian, 
 martyr. After having been a recorder for a long time, at length, by an 
 inspiration from heaven, he deemed it a hateful thing to contribute to the 
 massacre of the Christians, and therefore abandoned his office, and making a 
 profession of Christianity, he deserved to obtain the triumph of martyrdom.
\switchcolumn*
\selectlanguage{latin}
Item, in Africa 
 sanctórum Mártyrum Cláudii, Crispíni, Magínæ, Joánnis et Stéphani.
\switchcolumn
\selectlanguage{english}
Also in Africa, the holy martyrs 
 Claudius, Crispin, Magina, John, and Stephen.
\switchcolumn*
\selectlanguage{latin}
In Pannónia sancti 
 Agrícolæ Mártyris.
\switchcolumn
\selectlanguage{english}
In Hungary, St. Agricola, martyr.
\switchcolumn*
\selectlanguage{latin}
Nicomedíæ pássio 
 sanctórum Ambici, Victóris et Júlii.
\switchcolumn
\selectlanguage{english}
At Nicomedia, the martyrdom of the 
 Saints Ambicus, Victor, and Julius.
\switchcolumn*
\selectlanguage{latin}
Medioláni sancti 
 Miroclétis, Epíscopi et Confessóris; cujus aliquándo sanctus Ambrósius 
 méminit.
\switchcolumn
\selectlanguage{english}
At Milan, St. Mirocles, bishop and 
 confessor, sometimes mentioned by St. Ambrose.
\switchcolumn*
\selectlanguage{latin}
Dorcéstriæ, in Anglia, 
 sancti Biríni, qui fuit primus ejúsdem civitátis Epíscopus.
\switchcolumn
\selectlanguage{english}
At Dorchester in England, St. 
 Birinus, who was the first bishop of that city.
\switchcolumn*
\selectlanguage{latin}
Cúriæ, in Germánia, sancti Lúcii, Britannórum Regis, qui primus ex iis Régibus fidem Christi 
 suscépit, témpore sancti Eleuthérii Papæ.
\switchcolumn
\selectlanguage{english}
At Chur in Germany, St. Lucius, king 
 of the Britons, who in the time of Pope Eleutherius, was the first of their 
 kings to receive the faith of Christ.
\switchcolumn*
\selectlanguage{latin}
Senis, in Túscia, 
 sancti Galgáni Eremítæ.
\switchcolumn
\selectlanguage{english}
At Siena in Tuscany, St. Galganus, 
 hermit.
\switchcolumn*
\selectlanguage{latin}
\end{paracol}


% ---- martyrology/mart12/mart1204.htm
\needspace{10\baselineskip}
\begin{paracol}{2}
\selectlanguage{latin}
\begin{center}{\color{gregoriocolor} Prídie Nonas Decémbris. 
 Luna\dots\ }\end{center}
\switchcolumn
\selectlanguage{english}
\begin{center}{\color{gregoriocolor} The 
 Fourth Day of 
 December. The\dots\ Day of the Moon.}\end{center}
\end{paracol}

\noindent\begin{tabularx}{\linewidth}{*{19}{>{\centering\arraybackslash}X}}
 \textcolor{gregoriocolor}{a} & \textcolor{gregoriocolor}{b} & \textcolor{gregoriocolor}{c} & \textcolor{gregoriocolor}{d} & \textcolor{gregoriocolor}{e} & \textcolor{gregoriocolor}{f} & \textcolor{gregoriocolor}{g} & \textcolor{gregoriocolor}{h} & \textcolor{gregoriocolor}{i} & \textcolor{gregoriocolor}{k} & \textcolor{gregoriocolor}{l} & \textcolor{gregoriocolor}{m} & \textcolor{gregoriocolor}{n} & \textcolor{gregoriocolor}{p} & \textcolor{gregoriocolor}{q} & \textcolor{gregoriocolor}{r} & \textcolor{gregoriocolor}{s} & \textcolor{gregoriocolor}{t} & \textcolor{gregoriocolor}{u} \\
 14 & 15 & 16 & 17 & 18 & 19 & 20 & 21 & 22 & 23 & 24 & 25 & 26 & 27 & 28 & 29 & 1 & 2 & 3 \\
\end{tabularx}
\vspace{0.5\baselineskip}
\noindent\begin{tabularx}{\linewidth}{*{12}{>{\centering\arraybackslash}X}}
 \textcolor{gregoriocolor}{A} & \textcolor{gregoriocolor}{B} & \textcolor{gregoriocolor}{C} & \textcolor{gregoriocolor}{D} & \textcolor{gregoriocolor}{E} & F & \textcolor{gregoriocolor}{F} & \textcolor{gregoriocolor}{G} & \textcolor{gregoriocolor}{H} & \textcolor{gregoriocolor}{M} & \textcolor{gregoriocolor}{N} & \textcolor{gregoriocolor}{P} \\
 4 & 5 & 6 & 7 & 8 & 9 & 8 & 9 & 10 & 11 & 12 & 13 \\
\end{tabularx}

\begin{paracol}{2}
\selectlanguage{latin}
\lettrine[lines=2]{S}{ancti} Petri Chrysólogi, 
 Epíscopi Ravennátis, Confessóris et Ecclésiæ Doctóris, cujus memória quarto 
 Nonas hujus mensis recensétur.
\switchcolumn
\selectlanguage{english}
\lettrine[lines=2]{S}{t.} Peter Chrysologus, bishop of 
 Ravenna, confessor, and doctor of the Church, whose birthday is kept on the 
 2nd of December.
\switchcolumn*
\selectlanguage{latin}
Nicomedíæ pássio sanctæ 
 Bárbaræ, Vírginis et Mártyris; quæ, in persecutióne Maximíni, post diram cárceris maceratiónem, lampadárum adustiónem, mamillárum præcisiónem atque 
 ália torménta, gládio martyrium consummávit.
\switchcolumn
\selectlanguage{english}
At Nicomedia, the passion of St. 
 Barbara, virgin and martyr, in the persecution of Maximinus. After a 
 series of sufferings, a long imprisonment, the burning with torches, and the 
 cutting away of her breasts, her martyrdom was fulfilled by the sword.
\switchcolumn*
\selectlanguage{latin}
Constantinópoli 
 sanctórum Theóphanis et Sociórum.
\switchcolumn
\selectlanguage{english}
At Constantinople, St. Theophanes 
 and his companions.
\switchcolumn*
\selectlanguage{latin}
In Ponto beáti Melétii, 
 Epíscopi et Confessóris; qui, cum esset ob eruditiónis prærogatívam 
 præcípuus, ob virtútem tamen ánimi et vitæ sinceritátem longe magnificéntior 
 éxstitit.
\switchcolumn
\selectlanguage{english}
In Pontus, blessed Meletius, bishop 
 and confessor, who joined to an eminent gift of knowledge the more 
 distinguished glory of fortitude and integrity of life.
\switchcolumn*
\selectlanguage{latin}
Bonóniæ sancti Felícis 
 Epíscopi et Confessóris; qui, ántea Mediolanénsis Ecclésiæ, sub sancto 
 Ambrósio, Diáconus fúerat.
\switchcolumn
\selectlanguage{english}
At Bologna, St. Felix, bishop was 
 one time deacon of the Milanese Church under St. Ambrose.
\switchcolumn*
\selectlanguage{latin}
In Anglia, sancti 
 Osmúndi, Epíscopi et Confessóris.
\switchcolumn
\selectlanguage{english}
In England, St. Osmund, bishop and 
 confessor.
\switchcolumn*
\selectlanguage{latin}
Colóniæ Agrippínæ 
 sancti Annónis Epíscopi.
\switchcolumn
\selectlanguage{english}
At Cologne, St. Anno, bishop.
\switchcolumn*
\selectlanguage{latin}
In Mesopotámia sancti 
 Marúthæ Epíscopi, qui Dei Ecclésias, ob persecutiónem Isdegérdis Regis 
 collápsas, in Pérside reparávit, multísque miráculis clarus, apud hostes 
 étiam méruit honorári.
\switchcolumn
\selectlanguage{english}
In Mesopotamia, St. Maruthas, 
 bishop, who restored the churches of God that had been ruined in Persia by 
 the persecution of King Isdegerd. Being renowned for many miracles, he 
 merited to be honoured even by his enemies.
\switchcolumn*
\selectlanguage{latin}
Parmæ sancti Bernárdi, 
 Cardinális et ejúsdem civitátis Epíscopi, ex Ordine Vallis Umbrósæ.
\switchcolumn
\selectlanguage{english}
At Parma, St. Bernard, cardinal and 
 bishop of that city, of the Congregation of Vallombrosa of the Order of St. 
 Benedict.
\switchcolumn*
\selectlanguage{latin}
\end{paracol}


% ---- martyrology/mart12/mart1205.htm
\needspace{10\baselineskip}
\begin{paracol}{2}
\selectlanguage{latin}
\begin{center}{\color{gregoriocolor} Nonis Decémbris. 
 Luna\dots\ }\end{center}
\switchcolumn
\selectlanguage{english}
\begin{center}{\color{gregoriocolor} The 
 Fifth Day of 
 December. The\dots\ Day of the Moon.}\end{center}
\end{paracol}

\noindent\begin{tabularx}{\linewidth}{*{19}{>{\centering\arraybackslash}X}}
 \textcolor{gregoriocolor}{a} & \textcolor{gregoriocolor}{b} & \textcolor{gregoriocolor}{c} & \textcolor{gregoriocolor}{d} & \textcolor{gregoriocolor}{e} & \textcolor{gregoriocolor}{f} & \textcolor{gregoriocolor}{g} & \textcolor{gregoriocolor}{h} & \textcolor{gregoriocolor}{i} & \textcolor{gregoriocolor}{k} & \textcolor{gregoriocolor}{l} & \textcolor{gregoriocolor}{m} & \textcolor{gregoriocolor}{n} & \textcolor{gregoriocolor}{p} & \textcolor{gregoriocolor}{q} & \textcolor{gregoriocolor}{r} & \textcolor{gregoriocolor}{s} & \textcolor{gregoriocolor}{t} & \textcolor{gregoriocolor}{u} \\
 15 & 16 & 17 & 18 & 19 & 20 & 21 & 22 & 23 & 24 & 25 & 26 & 27 & 28 & 29 & 1 & 2 & 3 & 4 \\
\end{tabularx}
\vspace{0.5\baselineskip}
\noindent\begin{tabularx}{\linewidth}{*{12}{>{\centering\arraybackslash}X}}
 \textcolor{gregoriocolor}{A} & \textcolor{gregoriocolor}{B} & \textcolor{gregoriocolor}{C} & \textcolor{gregoriocolor}{D} & \textcolor{gregoriocolor}{E} & F & \textcolor{gregoriocolor}{F} & \textcolor{gregoriocolor}{G} & \textcolor{gregoriocolor}{H} & \textcolor{gregoriocolor}{M} & \textcolor{gregoriocolor}{N} & \textcolor{gregoriocolor}{P} \\
 5 & 6 & 7 & 8 & 9 & 10 & 9 & 10 & 11 & 12 & 13 & 14 \\
\end{tabularx}

\begin{paracol}{2}
\selectlanguage{latin}
\lettrine[lines=2]{I}{n} Judæa sancti Sabbæ 
 Abbátis, in óppido Cappadóciæ Mútala orti, qui miro sanctitátis exémplo 
 refúlsit, et pro fide cathólica, advérsus impugnántes sanctam Synodum 
 Chalcedonénsem, strénue laborávit, ac tandem in ea diœcésis Hierosolymitánæ 
 laura, quæ ipsíus sancti Sabbæ nómine póstmodum est insigníta, requiévit in 
 pace.
\switchcolumn
\selectlanguage{english}
\lettrine[lines=2]{I}{n} Judea, St. Sabbas, abbot, who was 
 born in the town of Mutala in Cappadocia. He gave a wondrous example 
 of holiness and laboured most zealously for the Catholic faith against those 
 who attacked the holy Council of Chalcedon. He rested in peace in the 
 monastery later named for him in the diocese of Jerusalem.
\switchcolumn*
\selectlanguage{latin}
Níciæ, apud Varum 
 flúvium, sancti Bassi Epíscopi, qui, in persecutióne Décii et Valeriáni, a 
 Perénnio Præside, ob Christi fidem, equúleo tortus, láminis candéntibus 
 ustus, fústibus et scorpiónibus cæsus, in ignem missus, et, cum inde 
 evasísset illæsus, duóbus clavis confíxus, illústre martyrium consummávit.
\switchcolumn
\selectlanguage{english}
At Nice, near the river Var, St. 
 Bassus, bishop. In the persecution of Decius and Valerian, he was 
 tortured by the governor Perennius for the faith of Christ, burned with hot 
 plates of metal, beaten with rods and whips garnished with pieces of iron, 
 and thrown into the fire. When he came out of it unhurt, he was 
 pierced with two spikes, and thus completed an illustrious martyrdom.
\switchcolumn*
\selectlanguage{latin}
Papíæ sancti Dalmátii, 
 Epíscopi et Mártyris; qui in persecutióne Maximiáni passus est.
\switchcolumn
\selectlanguage{english}
At Pavia, St. Dalmatius, bishop and 
 martyr, who suffered in the persecution of Maximian.
\switchcolumn*
\selectlanguage{latin}
Corfínii, in Pelígnis, 
 sancti Pelíni, Epíscopi Brundusíni, qui, cum ob ejus oratiónem, sub Juliáno Apóstata, templum Martis corruísset, a templórum Pontifícibus diríssime 
 cæsus est, atque, octogínta et quinque vulnéribus confóssus, martyrii 
 corónam proméruit.
\switchcolumn
\selectlanguage{english}
At Corfinio in Peligno, St. Pelinus, 
 bishop of Brindisi, at the time of Julian the Apostate. When the 
 temple of Mars fell to the ground at his prayer, he was severely scourged by 
 the priests of the temple, and being pierced with eighty-five wounds, he 
 merited the crown of martyrdom.
\switchcolumn*
\selectlanguage{latin}
Item sancti Anastásii 
 Mártyris, qui, præ ardóre martyrii, sponte se persecutóribus óbtulit.
\switchcolumn
\selectlanguage{english}
Also, St. Anastasius, martyr, who in 
 his ardent desire for martyrdom gave himself up voluntarily to the 
 persecutors.
\switchcolumn*
\selectlanguage{latin}
Thagúræ, in Africa, 
 sanctórum Mártyrum Júlii, Potámiæ, Crispíni, Felícis, Grati et aliórum 
 septem.
\switchcolumn
\selectlanguage{english}
At Thagura in Africa, the holy 
 martyrs Julius, Potamias, Crispin, Felix, Gratus, and seven others.
\switchcolumn*
\selectlanguage{latin}
Thebéste, in Numídia, sanctæ Crispínæ, nobilíssimæ féminæ, quæ, tempóribus Diocletiáni et 
 Maximiáni, cum sacrificáre nollet, jussu Anolíni Procónsulis decolláta est; 
 quam sanctus Augustínus sæpe láudibus célebrat.
\switchcolumn
\selectlanguage{english}
At Thebaste in Africa, St. Crispina, 
 a woman of the highest nobility who refused to sacrifice to idols during the 
 reign of Diocletian and Maximian, and was beheaded by order of the proconsul 
 Anolinus. Her praises are often celebrated by St. Augustine.
\switchcolumn*
\selectlanguage{latin}
Tréviris sancti Nicétii 
 Epíscopi, miræ sanctitátis viri.
\switchcolumn
\selectlanguage{english}
At Treves, St. Nicetius, bishop, a 
 man of great sanctity.
\switchcolumn*
\selectlanguage{latin}
Polyboti, in Asia, 
 sancti Joánnis Epíscopi, cognoménto Thaumatúrgi.
\switchcolumn
\selectlanguage{english}
At Polybotum in Asia, St. John, 
 bishop, surnamed the Wonderworker.
\switchcolumn*
\selectlanguage{latin}
\end{paracol}


% ---- martyrology/mart12/mart1206.htm
\needspace{10\baselineskip}
\begin{paracol}{2}
\selectlanguage{latin}
\begin{center}{\color{gregoriocolor} Octávo Idus Decémbris. 
 Luna\dots\ }\end{center}
\switchcolumn
\selectlanguage{english}
\begin{center}{\color{gregoriocolor} The 
 Sixth Day of 
 December. The\dots\ Day of the Moon.}\end{center}
\end{paracol}

\noindent\begin{tabularx}{\linewidth}{*{19}{>{\centering\arraybackslash}X}}
 \textcolor{gregoriocolor}{a} & \textcolor{gregoriocolor}{b} & \textcolor{gregoriocolor}{c} & \textcolor{gregoriocolor}{d} & \textcolor{gregoriocolor}{e} & \textcolor{gregoriocolor}{f} & \textcolor{gregoriocolor}{g} & \textcolor{gregoriocolor}{h} & \textcolor{gregoriocolor}{i} & \textcolor{gregoriocolor}{k} & \textcolor{gregoriocolor}{l} & \textcolor{gregoriocolor}{m} & \textcolor{gregoriocolor}{n} & \textcolor{gregoriocolor}{p} & \textcolor{gregoriocolor}{q} & \textcolor{gregoriocolor}{r} & \textcolor{gregoriocolor}{s} & \textcolor{gregoriocolor}{t} & \textcolor{gregoriocolor}{u} \\
 16 & 17 & 18 & 19 & 20 & 21 & 22 & 23 & 24 & 25 & 26 & 27 & 28 & 29 & 1 & 2 & 3 & 4 & 5 \\
\end{tabularx}
\vspace{0.5\baselineskip}
\noindent\begin{tabularx}{\linewidth}{*{12}{>{\centering\arraybackslash}X}}
 \textcolor{gregoriocolor}{A} & \textcolor{gregoriocolor}{B} & \textcolor{gregoriocolor}{C} & \textcolor{gregoriocolor}{D} & \textcolor{gregoriocolor}{E} & F & \textcolor{gregoriocolor}{F} & \textcolor{gregoriocolor}{G} & \textcolor{gregoriocolor}{H} & \textcolor{gregoriocolor}{M} & \textcolor{gregoriocolor}{N} & \textcolor{gregoriocolor}{P} \\
 6 & 7 & 8 & 9 & 10 & 11 & 10 & 11 & 12 & 13 & 14 & 15 \\
\end{tabularx}

\begin{paracol}{2}
\selectlanguage{latin}
\lettrine[lines=2]{M}{yræ,} quæ est 
 metrópolis Lyciæ, natális sancti Nicolái, Epíscopi et Confessóris, de quo, 
 inter plura miraculórum insígnia, illud memorábile fertur, quod Imperatórem 
 Constantínum ab intéritu quorúmdam se invocántium, longe constitútus, ad 
 misericórdiam per visum mónitis defléxit et minis.
\switchcolumn
\selectlanguage{english}
\lettrine[lines=2]{A}{t} Myra, which is the metropolis of 
 Lycia, the birthday of St. Nicholas, bishop and confessor, of whom it is 
 related, among other miracles, that, while at a great distance from Emperor 
 Constantine, he appeared to him in a vision and moved him to mercy so as to 
 deter him from putting to death some persons who had implored his 
 assistance.
\switchcolumn*
\selectlanguage{latin}
Eódem die sancti 
 Polychrónii Presbyteri, qui, témpore Constántii Imperatóris, cum ad altáre 
 Missas ágeret, invásus est ab Ariánis et jugulátus.
\switchcolumn
\selectlanguage{english}
On the same day, St. Polychronius, 
 priest, who was surprised while offering Mass at the altar and slain by the 
 Arians, in the reign of Emperor Constantius.
\switchcolumn*
\selectlanguage{latin}
In Africa sancti 
 Majórici, fílii sanctæ Dionysiæ, qui, cum esset adolescéntulus ac torménta 
 pavésceret, matris obtútibus verbísque corroborátus est, et, céteris fórtior 
 factus, in torméntis ánimam réddidit; quem amplexáta mater domi sepelívit, 
 et ad ejus sepúlcrum assídue oráre consuévit.
\switchcolumn
\selectlanguage{english}
In Africa, St. Majoricus, son of St. 
 Dionysia, who, being quite young and dreading the torments, was strengthened 
 by the looks and words of his mother, and becoming stronger than the rest, 
 expired in torments. His mother took him in her arms, and having 
 buried him in her own home, was wont to pray diligently at his tomb.
\switchcolumn*
\selectlanguage{latin}
Ibídem sanctárum 
 mulíerum Dionysiæ, quæ sancti Majórici Mártyris éxstitit mater, Datívæ ac 
 Leóntiæ; itémque religiósi viri, nómine Tértii, Æmiliáni médici, et 
 Bonifátii, cum áliis tribus. Hi omnes, in persecutióne Wandálica, sub 
 Ariáno Rege Hunneríco, gravíssimis et innúmeris supplíciis pro cathólicæ 
 fídei defensióne cruciáti, sanctórum Christi Confessórum número sociári 
 meruérunt.
\switchcolumn
\selectlanguage{english}
In the same place, the holy women 
 Dionysia, who was the mother of St. Majoricus the martyr, Dativa, and Leontia; 
 also a pious man named Tertius, Emilian a physician, Boniface, and three 
 others. In the persecution of the Vandals, under the Arian king 
 Hunneric, they were subjected to numberless most painful tortures for the 
 Catholic faith, and thus merited to rank among the confessors of Christ.
\switchcolumn*
\selectlanguage{latin}
Romæ sanctæ Aséllæ 
 Vírginis, quæ (ut beátus Hierónymus scribit), ex útero matris benedícta, 
 vitam in jejúniis et oratiónibus usque ad senéctam prodúxit.
\switchcolumn
\selectlanguage{english}
At Rome, St. Asella, virgin, who 
 according to the words of St. Jerome, being blessed from her mother's womb, 
 lived to old age in fasting and prayer.
\switchcolumn*
\selectlanguage{latin}
Granátæ, in Hispánia, 
 pássio beáti Petri Paschásii, Epíscopi Giennénsis et Mártyris, ex Ordine 
 beátæ Maríæ de Mercéde redemptiónis captivórum.
\switchcolumn
\selectlanguage{english}
At Granada in Spain, the passion of 
 blessed Peter Paschasius, bishop of Jaen and martyr, a member of the Order 
 of our Lady of Ransom for the Redemption of Captives.
\switchcolumn*
\selectlanguage{latin}
\end{paracol}


% ---- martyrology/mart12/mart1207.htm
\needspace{10\baselineskip}
\begin{paracol}{2}
\selectlanguage{latin}
\begin{center}{\color{gregoriocolor} Séptimo Idus Decémbris. 
 Luna\dots\ }\end{center}
\switchcolumn
\selectlanguage{english}
\begin{center}{\color{gregoriocolor} The 
 Seventh Day of 
 December. The\dots\ Day of the Moon.}\end{center}
\end{paracol}

\noindent\begin{tabularx}{\linewidth}{*{19}{>{\centering\arraybackslash}X}}
 \textcolor{gregoriocolor}{a} & \textcolor{gregoriocolor}{b} & \textcolor{gregoriocolor}{c} & \textcolor{gregoriocolor}{d} & \textcolor{gregoriocolor}{e} & \textcolor{gregoriocolor}{f} & \textcolor{gregoriocolor}{g} & \textcolor{gregoriocolor}{h} & \textcolor{gregoriocolor}{i} & \textcolor{gregoriocolor}{k} & \textcolor{gregoriocolor}{l} & \textcolor{gregoriocolor}{m} & \textcolor{gregoriocolor}{n} & \textcolor{gregoriocolor}{p} & \textcolor{gregoriocolor}{q} & \textcolor{gregoriocolor}{r} & \textcolor{gregoriocolor}{s} & \textcolor{gregoriocolor}{t} & \textcolor{gregoriocolor}{u} \\
 17 & 18 & 19 & 20 & 21 & 22 & 23 & 24 & 25 & 26 & 27 & 28 & 29 & 1 & 2 & 3 & 4 & 5 & 6 \\
\end{tabularx}
\vspace{0.5\baselineskip}
\noindent\begin{tabularx}{\linewidth}{*{12}{>{\centering\arraybackslash}X}}
 \textcolor{gregoriocolor}{A} & \textcolor{gregoriocolor}{B} & \textcolor{gregoriocolor}{C} & \textcolor{gregoriocolor}{D} & \textcolor{gregoriocolor}{E} & F & \textcolor{gregoriocolor}{F} & \textcolor{gregoriocolor}{G} & \textcolor{gregoriocolor}{H} & \textcolor{gregoriocolor}{M} & \textcolor{gregoriocolor}{N} & \textcolor{gregoriocolor}{P} \\
 7 & 8 & 9 & 10 & 11 & 12 & 11 & 12 & 13 & 14 & 15 & 16 \\
\end{tabularx}

\begin{paracol}{2}
\selectlanguage{latin}
\lettrine[lines=2]{V}{igília} Conceptiónis 
 Immaculátæ beátæ Maríæ Vírginis.
\switchcolumn
\selectlanguage{english}
\lettrine[lines=2]{T}{he} Vigil of the Immaculate 
 Conception of the Blessed Virgin Mary.
\switchcolumn*
\selectlanguage{latin}
Sancti Ambrósii 
 Epíscopi, Confessóris et Ecclésiæ Doctóris, qui prídie Nonas Aprílis 
 obdormívit in Dómino, sed hac die potíssimum cólitur, qua Mediolanénsem 
 Ecclésiam gubernándam suscépit.
\switchcolumn
\selectlanguage{english}
St. Ambrose, bishop and doctor of 
 the Church, who fell asleep in the Lord on the 4th of April; his feast is 
 kept on this day, the day on which he assumed the government of the Church 
 of Milan.
\switchcolumn*
\selectlanguage{latin}
Romæ beáti Eutychiáni 
 Papæ, qui per divérsa loca trecéntos quadragínta duos Mártyres manu sua 
 sepelívit; quibus et ipse deínde sociátus, sub Numeriáno Imperatóre, 
 martyrio coronátus est, et in cœmetério Callísti sepúltus.
\switchcolumn
\selectlanguage{english}
At Rome, blessed Eutychian, pope, 
 who with his own hand buried three hundred and forty-two martyrs in various 
 places. He himself was joined with them, crowned with martyrdom under 
 Emperor Numerian, and was buried in the cemetery of Callistus.
\switchcolumn*
\selectlanguage{latin}
Alexandríæ natális 
 beáti Agathónis militáris, qui, in persecutióne Décii, cum prohibéret 
 quosdam voléntes illúdere cadavéribus Mártyrum, clamor repénte totíus vulgi 
 advérsus eum extóllitur; oblátus autem Júdici, et in Christi confessióne 
 persístens, cápite pro pietáte damnátus est.
\switchcolumn
\selectlanguage{english}
At Alexandria, the birthday of 
 blessed Agatho, soldier. In the persecution of Decius, because he 
 prevented some people from mocking the bodies of the martyrs, a sudden 
 clamour was raised against him by the crowd. Being brought before the 
 judge, and persisting in his confession of Christ, he was sentenced to death 
 for his reverence.
\switchcolumn*
\selectlanguage{latin}
Antiochíæ sanctórum 
 Mártyrum Polycárpi et Theodóri.
\switchcolumn
\selectlanguage{english}
At Antioch, the holy martyrs 
 Polycarp and Theodore.
\switchcolumn*
\selectlanguage{latin}
Tubúrbi, in Africa, 
 sancti Servi Mártyris, qui, in persecutióne Wandálica, sub Ariáno Rege 
 Hunneríco, fústibus diutíssime cæsus, tróchleis frequénter in sublíme elevátus atque ictu céleri super sílices póndere córporis dimíssus, et 
 lapídibus acutíssimis perfricátus, martyrii palmam adéptus est.
\switchcolumn
\selectlanguage{english}
At Tuburbum in Africa, during the 
 persecution of the Vandals, under the Arian king Hunneric, St. Servus, 
 martyr, who, being for a very long time beaten with rods, lifted up on high 
 with pulleys, and suddenly dropped on flint-stones with his whole weight, 
 and rubbed over with sharp stones, obtained the palm of martyrdom.
\switchcolumn*
\selectlanguage{latin}
Theáni, in Campánia, sancti Urbáni, Epíscopi et Confessóris.
\switchcolumn
\selectlanguage{english}
At Teano in Campania, St. Urban, 
 bishop and confessor.
\switchcolumn*
\selectlanguage{latin}
Apud Sántonas, in 
 Gállia, sancti Martíni Abbátis, ad cujus túmulum crebérrima divínitus fiunt 
 mirácula.
\switchcolumn
\selectlanguage{english}
At Saintes in France, St. Martin, 
 abbot, at whose tomb frequent miracles have been worked through the power of 
 God.
\switchcolumn*
\selectlanguage{latin}
Eboríaci, in território 
 Meldénsi, commemorátio sanctæ Faræ, étiam Burgundofáræ nómine appellátæ, 
 Abbatíssæ et Vírginis, cujus dies natális tértio Nonas Aprilis recensétur.
\switchcolumn
\selectlanguage{english}
At Faremoutiers, in the diocese of 
 Meaux, the commemoration of St. Fara, who is also called Burgundofara, 
 abbess and virgin. Her birthday is on the 3rd of April.
\switchcolumn*
\selectlanguage{latin}
\end{paracol}


% ---- martyrology/mart12/mart1208.htm
\needspace{10\baselineskip}
\begin{paracol}{2}
\selectlanguage{latin}
\begin{center}{\color{gregoriocolor} Sexto Idus Decémbris. 
 Luna\dots\ }\end{center}
\switchcolumn
\selectlanguage{english}
\begin{center}{\color{gregoriocolor} The 
 Eighth Day of 
 December. The\dots\ Day of the Moon.}\end{center}
\end{paracol}

\noindent\begin{tabularx}{\linewidth}{*{19}{>{\centering\arraybackslash}X}}
 \textcolor{gregoriocolor}{a} & \textcolor{gregoriocolor}{b} & \textcolor{gregoriocolor}{c} & \textcolor{gregoriocolor}{d} & \textcolor{gregoriocolor}{e} & \textcolor{gregoriocolor}{f} & \textcolor{gregoriocolor}{g} & \textcolor{gregoriocolor}{h} & \textcolor{gregoriocolor}{i} & \textcolor{gregoriocolor}{k} & \textcolor{gregoriocolor}{l} & \textcolor{gregoriocolor}{m} & \textcolor{gregoriocolor}{n} & \textcolor{gregoriocolor}{p} & \textcolor{gregoriocolor}{q} & \textcolor{gregoriocolor}{r} & \textcolor{gregoriocolor}{s} & \textcolor{gregoriocolor}{t} & \textcolor{gregoriocolor}{u} \\
 18 & 19 & 20 & 21 & 22 & 23 & 24 & 25 & 26 & 27 & 28 & 29 & 1 & 2 & 3 & 4 & 5 & 6 & 7 \\
\end{tabularx}
\vspace{0.5\baselineskip}
\noindent\begin{tabularx}{\linewidth}{*{12}{>{\centering\arraybackslash}X}}
 \textcolor{gregoriocolor}{A} & \textcolor{gregoriocolor}{B} & \textcolor{gregoriocolor}{C} & \textcolor{gregoriocolor}{D} & \textcolor{gregoriocolor}{E} & F & \textcolor{gregoriocolor}{F} & \textcolor{gregoriocolor}{G} & \textcolor{gregoriocolor}{H} & \textcolor{gregoriocolor}{M} & \textcolor{gregoriocolor}{N} & \textcolor{gregoriocolor}{P} \\
 8 & 9 & 10 & 11 & 12 & 13 & 12 & 13 & 14 & 15 & 16 & 17 \\
\end{tabularx}

\begin{paracol}{2}
\selectlanguage{latin}
\lettrine[lines=2]{C}{oncéptio} Immaculáta gloriósæ semper Vírginis Genitrícis Dei Maríæ, quam fuísse 
 præservátam, singulári Dei privilégio, ab omni originális culpæ labe immúnem, 
 Pius Nonus, Póntifex Máximus, hac ipsa recurrénte die, solémniter definívit.
\switchcolumn
\selectlanguage{english}
\lettrine[lines=2]{T}{he} Immaculate 
 Conception of the glorious and ever Virgin Mary, Mother of God. On 
 this day, Pius IX solemnly declared her to have been by a singular privilege 
 of God preserved from all stain of original sin.
\switchcolumn*
\selectlanguage{latin}
Tréviris sancti 
 Euchárii, qui fuit discípulus beáti Petri Apóstoli et primus ejúsdem 
 civitátis Epíscopus.
\switchcolumn
\selectlanguage{english}
At Treves, St. Eucharius, a disciple 
 of blessed Peter the Apostle, first bishop of that city.
\switchcolumn*
\selectlanguage{latin}
Alexandríæ sancti 
 Macárii Mártyris, qui, témpore Décii, cum a Júdice multis verbis ad negándum 
 Christum suaderétur, et eo majóri constántia suam profiterétur fidem, vivus 
 ad últimum exúri jubétur.
\switchcolumn
\selectlanguage{english}
At Alexandria, St. Macarius, martyr, 
 whose constancy in professing the faith increased with the efforts made by 
 the judge to persuade him to deny Christ. He was finally condemned to 
 be burned alive.
\switchcolumn*
\selectlanguage{latin}
In Cypro sancti 
 Sophrónii Epíscopi, qui pupíllórum, orphanórum ac viduárum defénsor 
 miríficus, et páuperum atque oppressórum ómnium adjútor fuit.
\switchcolumn
\selectlanguage{english}
In Cyprus, the holy bishop 
 Sophronius, who was a devoted protector of orphans and widows, and a helper 
 of the poor and oppressed.
\switchcolumn*
\selectlanguage{latin}
In monastério 
 Luxoviénsi, in Gállia, sancti Romárici Abbátis, qui, cum in aula Theodobérti 
 Regis primus esset, renuntiávit sæculo, et monásticæ étiam observántiæ laude 
 céteris antecélluit.
\switchcolumn
\selectlanguage{english}
In the monastery of Luxeuil in 
 France, St. Romaricus, abbot, who left the highest station at the court of 
 King Theodobert, renounced the world, and surpassed others in the observance 
 of monastic discipline.
\switchcolumn*
\selectlanguage{latin}
Constantinópoli sancti 
 Patápii Solitárii, virtútibus et miráculis clari.
\switchcolumn
\selectlanguage{english}
At Constantinople, St. Patapius, 
 solitary, renowned for virtues and miracles.
\switchcolumn*
\selectlanguage{latin}
Romæ Invéntio sanctórum 
 Mártyrum Nemésii Diáconi, ejúsque fíliæ Lucíllæ Vírginis, Symphrónii, 
 Olympii Tribúni, hujúsque uxóris Exsupériæ et Theodúli fílii; quorum memória 
 octávo Kaléndas Septémbris recensétur.
\switchcolumn
\selectlanguage{english}
At Rome, the finding of the holy 
 martyrs Nemesis, a deacon, his daughter Lucina, a virgin, Symphronius, 
 Olympius the tribune and his wife Exuperia and his son Theodulus, whose 
 commemoration is made on the 25th of August.
\switchcolumn*
\selectlanguage{latin}
Verónæ Ordinátio sancti 
 Zenónis Epíscopi.
\switchcolumn
\selectlanguage{english}
At Verona, the ordination of St. 
 Zeno, bishop.
\switchcolumn*
\selectlanguage{latin}
\end{paracol}


% ---- martyrology/mart12/mart1209.htm
\needspace{10\baselineskip}
\begin{paracol}{2}
\selectlanguage{latin}
\begin{center}{\color{gregoriocolor} Quinto Idus Decémbris. 
 Luna\dots\ }\end{center}
\switchcolumn
\selectlanguage{english}
\begin{center}{\color{gregoriocolor} The 
 Ninth Day of 
 December. The\dots\ Day of the Moon.}\end{center}
\end{paracol}

\noindent\begin{tabularx}{\linewidth}{*{19}{>{\centering\arraybackslash}X}}
 \textcolor{gregoriocolor}{a} & \textcolor{gregoriocolor}{b} & \textcolor{gregoriocolor}{c} & \textcolor{gregoriocolor}{d} & \textcolor{gregoriocolor}{e} & \textcolor{gregoriocolor}{f} & \textcolor{gregoriocolor}{g} & \textcolor{gregoriocolor}{h} & \textcolor{gregoriocolor}{i} & \textcolor{gregoriocolor}{k} & \textcolor{gregoriocolor}{l} & \textcolor{gregoriocolor}{m} & \textcolor{gregoriocolor}{n} & \textcolor{gregoriocolor}{p} & \textcolor{gregoriocolor}{q} & \textcolor{gregoriocolor}{r} & \textcolor{gregoriocolor}{s} & \textcolor{gregoriocolor}{t} & \textcolor{gregoriocolor}{u} \\
 19 & 20 & 21 & 22 & 23 & 24 & 25 & 26 & 27 & 28 & 29 & 1 & 2 & 3 & 4 & 5 & 6 & 7 & 8 \\
\end{tabularx}
\vspace{0.5\baselineskip}
\noindent\begin{tabularx}{\linewidth}{*{12}{>{\centering\arraybackslash}X}}
 \textcolor{gregoriocolor}{A} & \textcolor{gregoriocolor}{B} & \textcolor{gregoriocolor}{C} & \textcolor{gregoriocolor}{D} & \textcolor{gregoriocolor}{E} & F & \textcolor{gregoriocolor}{F} & \textcolor{gregoriocolor}{G} & \textcolor{gregoriocolor}{H} & \textcolor{gregoriocolor}{M} & \textcolor{gregoriocolor}{N} & \textcolor{gregoriocolor}{P} \\
 9 & 10 & 11 & 12 & 13 & 14 & 13 & 14 & 15 & 16 & 17 & 18 \\
\end{tabularx}

\begin{paracol}{2}
\selectlanguage{latin}
\lettrine[lines=2]{C}{arthágine} sancti 
 Restitúti, Epíscopi et Mártyris, in cujus solemnitáte sanctus Augustínus de 
 ipso ad pópulum sermónem hábuit.
\switchcolumn
\selectlanguage{english}
\lettrine[lines=2]{A}{t} Carthage, St. Restitutus, bishop 
 and martyr, on whose feast St. Augustine delivered a discourse to the people 
 in which he set forth his praises.
\switchcolumn*
\selectlanguage{latin}
Item in Africa 
 sanctórum Mártyrum Petri, Succéssi, Bassiáni, Primitívi et aliórum vigínti.
\switchcolumn
\selectlanguage{english}
Also in Africa, the holy martyrs 
 Peter, Successus, Bassian, Primitivus, and twenty others.
\switchcolumn*
\selectlanguage{latin}
Toléti, in Hispánia, 
 natális sanctæ Leocádiæ, Vírginis et Mártyris; quæ, in persecutióne 
 Diocletiáni Imperatóris, a Præfécto Hispaniárum Daciáno inclúsa cárcere ac 
 dire maceráta, in eo tandem, cum gravíssimos beátæ Euláliæ et reliquórum 
 Mártyrum cruciátus audísset, impollútum spíritum, génibus in oratióne 
 pósitis, Christo réddidit.
\switchcolumn
\selectlanguage{english}
At Toledo in Spain, the birthday of 
 the holy virgin Leocadia, a martyr in the persecution of Emperor Diocletian. 
 She was condemned to a cruel imprisonment by Dacian, prefect of Spain, and 
 was pining away when, hearing of the barbarous tortures of blessed Eulalia 
 and the other martyrs, she knelt down to pray and yielded up her undefiled 
 spirit to Christ.
\switchcolumn*
\selectlanguage{latin}
Lemóvicis, in Aquitánia, sanctæ Valériæ, Vírginis et Mártyris.
\switchcolumn
\selectlanguage{english}
At Limoges in Aquitaine, St. 
 Valeria, virgin and martyr.
\switchcolumn*
\selectlanguage{latin}
Verónæ sancti Próculi 
 Epíscopi, qui, in persecutióne Diocletiáni, cólaphis ac fústibus cæsus, e 
 civitáte pulsus est, ac tandem in Ecclésiam suam restitútus, quiévit in 
 pace.
\switchcolumn
\selectlanguage{english}
At Verona, during the persecution of 
 Diocletian, St. Proculus, bishop, who was buffeted, scourged with rods, and 
 driven out of the city. Being at length restored to his church, he 
 died in peace.
\switchcolumn*
\selectlanguage{latin}
Papíæ sancti Syri, qui 
 fuit primus ejúsdem civitátis Epíscopus, atque apostólicis signis et 
 virtútibus cláruit.
\switchcolumn
\selectlanguage{english}
At Pavia, St. Syrus, first bishop of 
 that city, who was renowned for apostolic signs and virtues.
\switchcolumn*
\selectlanguage{latin}
Apaméæ, in Syria, beáti 
 Juliáni Epíscopi, qui, témpore Sevéri, sanctitáte refúlsit.
\switchcolumn
\selectlanguage{english}
At Apamea in Syria, blessed Julian, 
 bishop, who flourished in holiness in the time of Severus.
\switchcolumn*
\selectlanguage{latin}
Graji, in Burgúndia, 
 sancti Petri Fourier qui Canónicus Reguláris fuit Salvatóris Nostri, et 
 Canonissárum Regulárium Dóminæ Nostræ edocéndis puéllis Institútor; atque, 
 virtútibus ac miráculis clarus, a Leóne Décimo tértio, Pontífice Máximo, 
 Sanctórum catálogo adjúnctus est.
\switchcolumn
\selectlanguage{english}
At Gray in Burgundy, St. Peter 
 Fourier, Canon Regular of Our Saviour and the founder of the Canonesses 
 Regular of Our Lady for the education of children. Because of his 
 brilliant virtues and miracles, Leo XIII placed him the catalogue of the 
 Saints.
\switchcolumn*
\selectlanguage{latin}
Petragóricis, in 
 Gállia, sancti Cypriáni Abbátis, magnæ sanctitátis viri.
\switchcolumn
\selectlanguage{english}
At Perigueux in France, St. Cyprian, 
 abbot, a man of great sanctity.
\switchcolumn*
\selectlanguage{latin}
Naziánzi, in Cappadócia, 
 sanctæ Gorgóniæ, quæ fuit beátæ Nonnæ fília, atque beatórum Gregórii 
 Theólogi et Cæsárii soror, cujus ipse Gregórius virtútes et mirácula 
 conscrípsit.
\switchcolumn
\selectlanguage{english}
At Nazianzum in Cappadocia, St. 
 Gorgonia, of whose virtues and miracles St. Gregory has written. She 
 was the daughter of blessed Nonna and the sister of St. Gregory the 
 Theologian and St. Caesarius.
\switchcolumn*
\selectlanguage{latin}
\end{paracol}


% ---- martyrology/mart12/mart1210.htm
\needspace{10\baselineskip}
\begin{paracol}{2}
\selectlanguage{latin}
\begin{center}{\color{gregoriocolor} Quarto Idus Decémbris. 
 Luna\dots\ }\end{center}
\switchcolumn
\selectlanguage{english}
\begin{center}{\color{gregoriocolor} The 
 Tenth Day of 
 December. The\dots\ Day of the Moon.}\end{center}
\end{paracol}

\noindent\begin{tabularx}{\linewidth}{*{19}{>{\centering\arraybackslash}X}}
 \textcolor{gregoriocolor}{a} & \textcolor{gregoriocolor}{b} & \textcolor{gregoriocolor}{c} & \textcolor{gregoriocolor}{d} & \textcolor{gregoriocolor}{e} & \textcolor{gregoriocolor}{f} & \textcolor{gregoriocolor}{g} & \textcolor{gregoriocolor}{h} & \textcolor{gregoriocolor}{i} & \textcolor{gregoriocolor}{k} & \textcolor{gregoriocolor}{l} & \textcolor{gregoriocolor}{m} & \textcolor{gregoriocolor}{n} & \textcolor{gregoriocolor}{p} & \textcolor{gregoriocolor}{q} & \textcolor{gregoriocolor}{r} & \textcolor{gregoriocolor}{s} & \textcolor{gregoriocolor}{t} & \textcolor{gregoriocolor}{u} \\
 20 & 21 & 22 & 23 & 24 & 25 & 26 & 27 & 28 & 29 & 1 & 2 & 3 & 4 & 5 & 6 & 7 & 8 & 9 \\
\end{tabularx}
\vspace{0.5\baselineskip}
\noindent\begin{tabularx}{\linewidth}{*{12}{>{\centering\arraybackslash}X}}
 \textcolor{gregoriocolor}{A} & \textcolor{gregoriocolor}{B} & \textcolor{gregoriocolor}{C} & \textcolor{gregoriocolor}{D} & \textcolor{gregoriocolor}{E} & F & \textcolor{gregoriocolor}{F} & \textcolor{gregoriocolor}{G} & \textcolor{gregoriocolor}{H} & \textcolor{gregoriocolor}{M} & \textcolor{gregoriocolor}{N} & \textcolor{gregoriocolor}{P} \\
 10 & 11 & 12 & 13 & 14 & 15 & 14 & 15 & 16 & 17 & 18 & 19 \\
\end{tabularx}

\begin{paracol}{2}
\selectlanguage{latin}
\lettrine[lines=2]{S}{ancti} Melchíadis, Papæ 
 et Mártyris, cujus dies natális recensétur tértio Idus Januárii.
\switchcolumn
\selectlanguage{english}
\lettrine[lines=2]{S}{t.} Melchiades, pope and martyr, 
 whose birthday is mentioned on the 11th of January.
\switchcolumn*
\selectlanguage{latin}
Romæ, via Ostiénsi, 
 Dedicátio Basílicæ sancti Pauli Apóstoli; quæ, simul cum Dedicatióne 
 Basílicæ sancti Petri, Apostolórum Príncipis, ánnua celebritáte recólitur 
 quartodécimo Kaléndas Decémbris.
\switchcolumn
\selectlanguage{english}
At Rome, on the Ostian Way, the 
 Dedication of the Basilica of St. Paul the Apostle. The yearly 
 commemoration of this Dedication, together with that of St. Peter, prince of 
 the apostles, is observed on the 18th of November.
\switchcolumn*
\selectlanguage{latin}
Eódem die sanctórum 
 Mártyrum Carpóphori Presbyteri, et Abúndii Diáconi; qui, in Diocletiáni 
 persecutióne, primo fústibus crudelíssime cæsi, deínde in cárcerem, negátio 
 cibo et potu, retrúsi, et rursum in equúleo torti, et post hæc diu in cárcere maceráti, novíssime gládio percússi sunt.
\switchcolumn
\selectlanguage{english}
Also, the holy martyrs Carpophorus, 
 a priest, and Abundius, a deacon, in the persecution of Diocletian. 
 They were first cruelly beaten with rods, then imprisoned and denied food 
 and drink; being placed on the rack a second time and again thrown into 
 prison, they were finally beheaded.
\switchcolumn*
\selectlanguage{latin}
Alexandríæ sanctórum 
 Mártyrum Mennæ, Hermógenis et Eugraphi; qui sub Galério Maximiáno passi sunt.
\switchcolumn
\selectlanguage{english}
At Alexandria, the holy martyrs 
 Mennas, Hermogenes, and Eugraphus, who suffered under Galerius Maximian.
\switchcolumn*
\selectlanguage{latin}
Apud Leontínos, in 
 Sicília, sanctórum Mártyrum Mercúrii et Sociórum mílitum; qui, sub Tertyllo 
 Præside, témpore Licínii Imperatóris, gládio cæsi sunt.
\switchcolumn
\selectlanguage{english}
At Lentini in Sicily, the holy 
 martyrs Mercurius and his soldier companions, who were slain by the sword 
 under the governor Tertyllus, in the reign of Emperor Licinius.
\switchcolumn*
\selectlanguage{latin}
Ancyræ, in Galátia, sancti Gemélli Mártyris, qui, post dira torménta, sub Juliáno Apóstata, 
 crucis supplício martyrium consummávit.
\switchcolumn
\selectlanguage{english}
At Ancyra in Galatia, St. Gemellus, 
 martyr, who, after severe torments, fulfilled his martyrdom by being 
 crucified in the time of Julian the Apostate.
\switchcolumn*
\selectlanguage{latin}
Eméritæ, in Hispánia, 
 pássio sanctæ Euláliæ Vírginis, quæ, sub Maximiáno Imperatóre, cum esset 
 annórum duódecim, ibi, jussu Daciáni Præsidis, pro confessióne Christi, 
 plúrima torménta est perpéssa; novíssime, in equúleo suspénsa et exunguláta, 
 fáculis ardéntibus ex utróque látere appósitis, hausto igne, spíritum 
 réddidit.
\switchcolumn
\selectlanguage{english}
At Merida in Spain, in the time of 
 Maximian, the martyrdom of the holy virgin Eulalia, who at twelve years of 
 age suffered many torments for the confession of Christ by order of the 
 governor Dacian. She was stretched on the rack, torn with iron claws, 
 had her sides burned with flaming torches, and swallowing the fire she 
 expired.
\switchcolumn*
\selectlanguage{latin}
Item ibídem sanctæ 
 Júliæ, Vírginis et Mártyris; quæ beátæ Euláliæ sócia fuit, et illi ad 
 passiónem properánti indivídua comes adhæsit.
\switchcolumn
\selectlanguage{english}
Also, in the same city, St. Julia, 
 virgin and martyr, the companion of the blessed Eulalia, who would not be 
 separated from her when the latter went to suffer.
\switchcolumn*
\selectlanguage{latin}
Romæ beáti Gregórii 
 Papæ Tértii, qui sanctitáte meritísque præclárus migrávit in cælum.
\switchcolumn
\selectlanguage{english}
At Rome, Pope St. Gregory III, who 
 departed for heaven renowned for his sanctity and good works.
\switchcolumn*
\selectlanguage{latin}
Viénnæ, in Gállia, 
 sancti Sindúlphi, Epíscopi et Confessóris.
\switchcolumn
\selectlanguage{english}
At Vienne in France, St. Sindulph, 
 bishop and confessor.
\switchcolumn*
\selectlanguage{latin}
Bríxiæ sancti Deúsdedit 
 Epíscopi.
\switchcolumn
\selectlanguage{english}
At Brescia, St. Deusdedit, bishop.
\switchcolumn*
\selectlanguage{latin}
Lauréti, in Picéno, Translátio sacræ Domus Genitrícis Dei Maríæ, qua in domo Verbum caro factum 
 est. Ipsam vero beatíssimam Vírginem, Lauretánæ título nuncupátam, 
 Benedíctus Papa Décimus quintus ómnibus aereonáutis præcípuam apud Deum 
 Patrónam attríbuit.
\switchcolumn
\selectlanguage{english}
At Loreto in Piceno, the 
 Translation of the Holy House of Mary the Mother of God, wherein the Word 
 was made flesh. Pope Benedict XV declared the same Blessed Virgin 
 Mary, under the title of Loreto, to be the chief Patroness before God of 
 all airmen.
\switchcolumn*
\selectlanguage{latin}
\end{paracol}


% ---- martyrology/mart12/mart1211.htm
\needspace{10\baselineskip}
\begin{paracol}{2}
\selectlanguage{latin}
\begin{center}{\color{gregoriocolor} Tértio Idus Decémbris. 
 Luna\dots\ }\end{center}
\switchcolumn
\selectlanguage{english}
\begin{center}{\color{gregoriocolor} The 
 Eleventh Day of 
 December. The\dots\ Day of the Moon.}\end{center}
\end{paracol}

\noindent\begin{tabularx}{\linewidth}{*{19}{>{\centering\arraybackslash}X}}
 \textcolor{gregoriocolor}{a} & \textcolor{gregoriocolor}{b} & \textcolor{gregoriocolor}{c} & \textcolor{gregoriocolor}{d} & \textcolor{gregoriocolor}{e} & \textcolor{gregoriocolor}{f} & \textcolor{gregoriocolor}{g} & \textcolor{gregoriocolor}{h} & \textcolor{gregoriocolor}{i} & \textcolor{gregoriocolor}{k} & \textcolor{gregoriocolor}{l} & \textcolor{gregoriocolor}{m} & \textcolor{gregoriocolor}{n} & \textcolor{gregoriocolor}{p} & \textcolor{gregoriocolor}{q} & \textcolor{gregoriocolor}{r} & \textcolor{gregoriocolor}{s} & \textcolor{gregoriocolor}{t} & \textcolor{gregoriocolor}{u} \\
 21 & 22 & 23 & 24 & 25 & 26 & 27 & 28 & 29 & 1 & 2 & 3 & 4 & 5 & 6 & 7 & 8 & 9 & 10 \\
\end{tabularx}
\vspace{0.5\baselineskip}
\noindent\begin{tabularx}{\linewidth}{*{12}{>{\centering\arraybackslash}X}}
 \textcolor{gregoriocolor}{A} & \textcolor{gregoriocolor}{B} & \textcolor{gregoriocolor}{C} & \textcolor{gregoriocolor}{D} & \textcolor{gregoriocolor}{E} & F & \textcolor{gregoriocolor}{F} & \textcolor{gregoriocolor}{G} & \textcolor{gregoriocolor}{H} & \textcolor{gregoriocolor}{M} & \textcolor{gregoriocolor}{N} & \textcolor{gregoriocolor}{P} \\
 11 & 12 & 13 & 14 & 15 & 16 & 15 & 16 & 17 & 18 & 19 & 20 \\
\end{tabularx}

\begin{paracol}{2}
\selectlanguage{latin}
\lettrine[lines=2]{R}{omæ} sancti Dámasi 
 Primi, Papæ et Confessóris; qui Apollinárem hæresiárcham damnávit, et Petrum, 
 Episcopum Alexandrínum, fugátum restítuit; multa étiam sanctórum Mártyrum 
 córpora invénit, eorúmque memórias vérsibus exornávit.
\switchcolumn
\selectlanguage{english}
\lettrine[lines=2]{A}{t} Rome, St. Damasus, pope and 
 confessor, who condemned the heresiarch Apollinaris, and restored to his See 
 Peter, bishop of Alexandria, who had been driven from it. He also 
 discovered the bodies of many holy martyrs and composed verses in their 
 honour.
\switchcolumn*
\selectlanguage{latin}
Item Romæ pássio sancti 
 Trasónis, qui, cum Christiános laborántes in thermis, aliísque opéribus 
 públicis fatigátos, et in cárcere pósitos, de suis facultátibus áleret, 
 jussu Maximiáni tentus est, et cum áliis duóbus, id est Pontiáno et 
 Prætextáto, martyrio coronátus.
\switchcolumn
\selectlanguage{english}
Also at Rome, St. Thrason. He 
 was arrested by order of Maximian for supporting with his goods the 
 Christians who laboured in the baths and at other public works, and those 
 confined in jail. He was crowned with martyrdom with two others, Pontian and 
 Prætextatus.
\switchcolumn*
\selectlanguage{latin}
Ambiáni, in Gállia, 
 sanctórum Mártyrum Victórici et Fusciáni, sub eódem Imperatóre, in quorum 
 náribus et áuribus jussit Rictiovárus Præses immítti tarínchas, et clavis 
 ardéntibus témpora transfígi, deínde óculos evélli, ac póstmodum eórum 
 córpora jaculári; sicque, una cum sancto Gentiáno, eórum hóspite, capítibus 
 amputátis, migravérunt ad Dóminum.
\switchcolumn
\selectlanguage{english}
At Amiens in France, the holy 
 martyrs Victoricus and Fuscian, under the same emperor. By order of 
 Governor Rictiovarus, they had iron pins driven into their ears and 
 nostrils, heated nails into their temples, and arrows into their bodies and 
 their eyes torn out. They were beheaded with St. Gentian, their guest, 
 and they passed to the Lord.
\switchcolumn*
\selectlanguage{latin}
In Pérside sancti 
 Bársabæ Mártyris.
\switchcolumn
\selectlanguage{english}
In Persia, St. Barbabas, martyr.
\switchcolumn*
\selectlanguage{latin}
In Hispánia sancti 
 Eutychii Mártyris.
\switchcolumn
\selectlanguage{english}
In Spain, St. Eutychius, martyr.
\switchcolumn*
\selectlanguage{latin}
Placéntiæ sancti Sabíni 
 Epíscopi, miráculis clari.
\switchcolumn
\selectlanguage{english}
At Piacenza, St. Sabinus, bishop, 
 renowned for miracles.
\switchcolumn*
\selectlanguage{latin}
Constantinópoli sancti 
 Daniélis Stylítæ.
\switchcolumn
\selectlanguage{english}
At Constantinople, St. Daniel 
 Stylites.
\switchcolumn*
\selectlanguage{latin}
\end{paracol}


% ---- martyrology/mart12/mart1212.htm
\needspace{10\baselineskip}
\begin{paracol}{2}
\selectlanguage{latin}
\begin{center}{\color{gregoriocolor} Prídie Idus Decémbris. 
 Luna\dots\ }\end{center}
\switchcolumn
\selectlanguage{english}
\begin{center}{\color{gregoriocolor} The 
 Twelfth Day of 
 December. The\dots\ Day of the Moon.}\end{center}
\end{paracol}

\noindent\begin{tabularx}{\linewidth}{*{19}{>{\centering\arraybackslash}X}}
 \textcolor{gregoriocolor}{a} & \textcolor{gregoriocolor}{b} & \textcolor{gregoriocolor}{c} & \textcolor{gregoriocolor}{d} & \textcolor{gregoriocolor}{e} & \textcolor{gregoriocolor}{f} & \textcolor{gregoriocolor}{g} & \textcolor{gregoriocolor}{h} & \textcolor{gregoriocolor}{i} & \textcolor{gregoriocolor}{k} & \textcolor{gregoriocolor}{l} & \textcolor{gregoriocolor}{m} & \textcolor{gregoriocolor}{n} & \textcolor{gregoriocolor}{p} & \textcolor{gregoriocolor}{q} & \textcolor{gregoriocolor}{r} & \textcolor{gregoriocolor}{s} & \textcolor{gregoriocolor}{t} & \textcolor{gregoriocolor}{u} \\
 22 & 23 & 24 & 25 & 26 & 27 & 28 & 29 & 1 & 2 & 3 & 4 & 5 & 6 & 7 & 8 & 9 & 10 & 11 \\
\end{tabularx}
\vspace{0.5\baselineskip}
\noindent\begin{tabularx}{\linewidth}{*{12}{>{\centering\arraybackslash}X}}
 \textcolor{gregoriocolor}{A} & \textcolor{gregoriocolor}{B} & \textcolor{gregoriocolor}{C} & \textcolor{gregoriocolor}{D} & \textcolor{gregoriocolor}{E} & F & \textcolor{gregoriocolor}{F} & \textcolor{gregoriocolor}{G} & \textcolor{gregoriocolor}{H} & \textcolor{gregoriocolor}{M} & \textcolor{gregoriocolor}{N} & \textcolor{gregoriocolor}{P} \\
 12 & 13 & 14 & 15 & 16 & 17 & 16 & 17 & 18 & 19 & 20 & 21 \\
\end{tabularx}

\begin{paracol}{2}
\selectlanguage{latin}
\lettrine[lines=2]{A}{lexandríæ} sanctórum 
 Epímachi et Alexándri, qui, sub Décio Imperatóre, cum fuíssent multo témpore 
 in vínculis, atque, divérsis supplíciis affécti, perdurássent in fide, 
 ígnibus tandem consúmpti sunt. Sanctus vero Epímachus, simul cum sancti 
 Gordiáno Mártyre, festíva celebritáte cólitur sexto Idus Maji.
\switchcolumn
\selectlanguage{english}
\lettrine[lines=2]{A}{t} Alexandria, in the time of Decius, 
 the holy martyrs Epimachus and Alexander, who were kept in chains a long 
 time and subjected to various torments, but as they persevered in the faith, 
 they were finally consumed by fire. The feast of St. Epimachus 
 together with that of St. Gordian the martyr is observed on the 10th of May.
\switchcolumn*
\selectlanguage{latin}
Romæ sancti Synésii 
 Mártyris, qui, beáti Xysti Papæ Secúndi témpore ordinátus Lector, et, cum 
 multos convertísset ad Christum, apud Aureliánum Imperatórem accusátus, 
 martyrii corónam gládio percússus accépit.
\switchcolumn
\selectlanguage{english}
At Rome, the holy martyr Synesius, 
 who was ordained lector in the time of blessed Pope Sixtus. Having 
 converted many to Christ, he was accused before Emperor Aurelian, and being 
 put to the sword, received the crown of martyrdom.
\switchcolumn*
\selectlanguage{latin}
Eódem die sanctórum 
 Mártyrum Hermógenis, Donáti et aliórum vigínti duórum.
\switchcolumn
\selectlanguage{english}
On the same day, the holy martyrs 
 Hermogenes, Donatus, and twenty-two others.
\switchcolumn*
\selectlanguage{latin}
Tréviris sanctórum 
 Mártyrum Maxéntii, Constántii, Crescéntii, Justíni et Sociórum; qui in 
 persecutióne Diocletiáni, sub Rictiováro Præside, passi sunt.
\switchcolumn
\selectlanguage{english}
At Treves, the holy martyrs 
 Maxentius, Constantius, Crescentius, Justinus, and their companions, who 
 suffered in the persecution of Diocletian, under the governor Rictiovarus.
\switchcolumn*
\selectlanguage{latin}
Alexandríæ sanctárum 
 Ammonáriæ Vírginis, Mercúriæ, Dionysiæ et altérius Ammonáriæ. Harum 
 prima, in persecutióne Décii, inaudítis tormentórum genéribus superátis, 
 beátum vitæ finem, ferro cædénte, percépit; tres vero áliæ, cum Judex a 
 féminis superári erubésceret, ac dubitáret, ne, eádem in illas exércens 
 cruciaménta, viríli eárum étiam constántia vincerétur, statim jussæ sunt 
 decollári.
\switchcolumn
\selectlanguage{english}
At Alexandria, the holy women 
 Ammonaria, virgin, Mercuria, Dionysia, and another Ammonaria. The 
 first named, after having triumphed over unheard-of kinds of torments, in 
 the persecution of Decius, ended her blessed life by beheading. As to 
 the three others, the judge, being ashamed to be overcome by women, and 
 fearing that by resorting to tortures he would be vanquished by their 
 constancy, ordered them to be beheaded immediately.
\switchcolumn*
\selectlanguage{latin}
\end{paracol}


% ---- martyrology/mart12/mart1213.htm
\needspace{10\baselineskip}
\begin{paracol}{2}
\selectlanguage{latin}
\begin{center}{\color{gregoriocolor} Idibus Decémbris. 
 Luna\dots\ }\end{center}
\switchcolumn
\selectlanguage{english}
\begin{center}{\color{gregoriocolor} The 
 Thirteenth Day of 
 December. The\dots\ Day of the Moon.}\end{center}
\end{paracol}

\noindent\begin{tabularx}{\linewidth}{*{19}{>{\centering\arraybackslash}X}}
 \textcolor{gregoriocolor}{a} & \textcolor{gregoriocolor}{b} & \textcolor{gregoriocolor}{c} & \textcolor{gregoriocolor}{d} & \textcolor{gregoriocolor}{e} & \textcolor{gregoriocolor}{f} & \textcolor{gregoriocolor}{g} & \textcolor{gregoriocolor}{h} & \textcolor{gregoriocolor}{i} & \textcolor{gregoriocolor}{k} & \textcolor{gregoriocolor}{l} & \textcolor{gregoriocolor}{m} & \textcolor{gregoriocolor}{n} & \textcolor{gregoriocolor}{p} & \textcolor{gregoriocolor}{q} & \textcolor{gregoriocolor}{r} & \textcolor{gregoriocolor}{s} & \textcolor{gregoriocolor}{t} & \textcolor{gregoriocolor}{u} \\
 23 & 24 & 25 & 26 & 27 & 28 & 29 & 1 & 2 & 3 & 4 & 5 & 6 & 7 & 8 & 9 & 10 & 11 & 12 \\
\end{tabularx}
\vspace{0.5\baselineskip}
\noindent\begin{tabularx}{\linewidth}{*{12}{>{\centering\arraybackslash}X}}
 \textcolor{gregoriocolor}{A} & \textcolor{gregoriocolor}{B} & \textcolor{gregoriocolor}{C} & \textcolor{gregoriocolor}{D} & \textcolor{gregoriocolor}{E} & F & \textcolor{gregoriocolor}{F} & \textcolor{gregoriocolor}{G} & \textcolor{gregoriocolor}{H} & \textcolor{gregoriocolor}{M} & \textcolor{gregoriocolor}{N} & \textcolor{gregoriocolor}{P} \\
 13 & 14 & 15 & 16 & 17 & 18 & 17 & 18 & 19 & 20 & 21 & 22 \\
\end{tabularx}

\begin{paracol}{2}
\selectlanguage{latin}
\lettrine[lines=2]{S}{yracúsis,} in Sicília, 
 natális sanctæ Lúciæ, Vírginis et Mártyris, in persecutióne Diocletiáni. 
 Hæc nóbilis Virgo, cum eam lenónes, quibus, jubénte, Paschásio Consulári, 
 trádita erat ut a pópulo castitáti ejus illuderétur, dúcere vellent, 
 nullátenus per illos movéri pótuit, nec fúnibus ádditis, nec boum jugis 
 plúrimis; deínde vero, picem, resínam ac fervens óleum nil læsa súperans, 
 tandem, gládio in gútture percússa, martyrium consummávit.
\switchcolumn
\selectlanguage{english}
\lettrine[lines=2]{A}{t} Syracuse in Sicily, the birthday 
 of St. Lucy, virgin and martyr, in the persecution of Diocletian. By 
 order of the proconsul Paschasius, she was delivered to profligates, that 
 her chastity might be insulted by the people; but when they attempted to 
 lead her away they were not able to move her, either with ropes or by means 
 of many yoke of oxen. Then having hot pitch, resin, and burning oil 
 applied to her body without being injured, she finally had a sword driven 
 through her throat, and thus completed her martyrdom.
\switchcolumn*
\selectlanguage{latin}
Molíni, in Gállia, item 
 natális sanctæ Joánnæ-Francíscæ Frémiot de Chantal Víduæ, quæ Ordinis 
 Sanctimoniálium Visitatiónis sanctæ Maríæ fuit Institútrix; ac, nobilitáte 
 géneris, vitæ sanctimónia, quam in quadruplíci statu constánter duxit, et 
 miraculórum dono illústris, a Cleménte Décimo tértio, Pontífice Máximo, in 
 Sanctárum númerum reláta est. Sacrum ejus corpus Annésium, in Sabáudia, 
 translátum fuit, et solémni pompa in prima sui Ordinis Ecclésia tumulátum. 
 Ipsíus autem festum duodécimo Kaléndas Septémbris ab univérsa Ecclésia 
 Clemens Papa Décimus quartus celebrári mandávit.
\switchcolumn
\selectlanguage{english}
At Moulins in France, the birthday 
 of St. Jane Frances Fremiot de Chantal, widow, foundress of the Nuns of the 
 Visitation of St. Mary, distinguished by the nobility of her birth, by the 
 holiness she constantly displayed in four different states of life, and by 
 the gift of miracles. She was placed among the saints by Clement XIII. 
 Her holy body was taken to Annecy in Savoy and buried with great pomp in the 
 first church of her order. by order of Clement XIV, her feast is kept 
 by the whole Church on the 21st of August.
\switchcolumn*
\selectlanguage{latin}
In Arménia pássio 
 sanctórum Mártyrum Eustrátii, Auxéntii, Eugénii, Mardárii et Oréstis, in 
 persecutióne Diocletiáni. Ex ipsis Eustrátius, primum sub Lysia, 
 deínde Sebáste, sub Agricoláo Præside, una cum Oréste exquisítis torméntis 
 cruciátus, in fornácem missus réddidit spíritum; Oréstes autem, in stratum 
 férreum ignítum pósitus, migrávit ad Dóminum; céteri apud Arábracos, 
 sævíssimis agitáti supplíciis, sub Lysia Præside, diversímode martyrium 
 consummárunt. Eórum córpora, póstea Romam transláta, in Ecclésia 
 sancti Apollináris honorífice collocáta sunt.
\switchcolumn
\selectlanguage{english}
In Armenia, the martyrdom of the 
 holy martyrs Eustratius, Auxentius, Eugene, Mardarius, and Orestes, in the 
 persecution of Diocletian. Eustratius was the first subjected alone to 
 barbarous torments under Lysias. Then he was conducted to Sebaste, 
 where he was tortured together with Orestes under the governor Agricolaus, 
 and being cast into a furnace, yielded up his soul; but Orestes being laid 
 on a bed of heated iron, rendered his soul unto God. The others were 
 made to endure most grievous torments among the Arabraci, under the governor 
 Lysias, and fulfilled their martyrdom in different ways. Their relics 
 were afterwards carried to Rome and placed with due honours in the church of 
 St. Apollinaris.
\switchcolumn*
\selectlanguage{latin}
In Sulcitána ínsula, 
 apud Sardíniam, pássio sancti Antíochi, sub Hadriáno Imperatóre.
\switchcolumn
\selectlanguage{english}
At Sardinia, in the island of Sulci, 
 the martyrdom of St. Antiochus, under Emperor Hadrian.
\switchcolumn*
\selectlanguage{latin}
Cameráci, in Gállia, 
 sancti Authbérti, Epíscopi et Confessóris.
\switchcolumn
\selectlanguage{english}
At Cambrai in France, St. Aubert, 
 bishop and confessor.
\switchcolumn*
\selectlanguage{latin}
In pago Pontívo, in 
 Gállia, sancti Judóci, Presbyteri et Confessóris.
\switchcolumn
\selectlanguage{english}
In the parts of Ponthieu in France, 
 St. Judoc, priest and confessor.
\switchcolumn*
\selectlanguage{latin}
In território 
 Argentoraténsi sanctæ Othíliæ Vírginis.
\switchcolumn
\selectlanguage{english}
In the territory of Strasbourg, St. 
 Otilie, virgin.
\switchcolumn*
\selectlanguage{latin}
\end{paracol}


% ---- martyrology/mart12/mart1214.htm
\needspace{10\baselineskip}
\begin{paracol}{2}
\selectlanguage{latin}
\begin{center}{\color{gregoriocolor} Décimo nono Kaléndas Januárii. 
 Luna\dots\ }\end{center}
\switchcolumn
\selectlanguage{english}
\begin{center}{\color{gregoriocolor} The 
 Fourteenth Day of 
 December. The\dots\ Day of the Moon.}\end{center}
\end{paracol}

\noindent\begin{tabularx}{\linewidth}{*{19}{>{\centering\arraybackslash}X}}
 \textcolor{gregoriocolor}{a} & \textcolor{gregoriocolor}{b} & \textcolor{gregoriocolor}{c} & \textcolor{gregoriocolor}{d} & \textcolor{gregoriocolor}{e} & \textcolor{gregoriocolor}{f} & \textcolor{gregoriocolor}{g} & \textcolor{gregoriocolor}{h} & \textcolor{gregoriocolor}{i} & \textcolor{gregoriocolor}{k} & \textcolor{gregoriocolor}{l} & \textcolor{gregoriocolor}{m} & \textcolor{gregoriocolor}{n} & \textcolor{gregoriocolor}{p} & \textcolor{gregoriocolor}{q} & \textcolor{gregoriocolor}{r} & \textcolor{gregoriocolor}{s} & \textcolor{gregoriocolor}{t} & \textcolor{gregoriocolor}{u} \\
 24 & 25 & 26 & 27 & 28 & 29 & 1 & 2 & 3 & 4 & 5 & 6 & 7 & 8 & 9 & 10 & 11 & 12 & 13 \\
\end{tabularx}
\vspace{0.5\baselineskip}
\noindent\begin{tabularx}{\linewidth}{*{12}{>{\centering\arraybackslash}X}}
 \textcolor{gregoriocolor}{A} & \textcolor{gregoriocolor}{B} & \textcolor{gregoriocolor}{C} & \textcolor{gregoriocolor}{D} & \textcolor{gregoriocolor}{E} & F & \textcolor{gregoriocolor}{F} & \textcolor{gregoriocolor}{G} & \textcolor{gregoriocolor}{H} & \textcolor{gregoriocolor}{M} & \textcolor{gregoriocolor}{N} & \textcolor{gregoriocolor}{P} \\
 14 & 15 & 16 & 17 & 18 & 19 & 18 & 19 & 20 & 21 & 22 & 23 \\
\end{tabularx}

\begin{paracol}{2}
\selectlanguage{latin}
\lettrine[lines=2]{U}{bédæ,} in Hispánia, 
 natális sancti Joánnis a Cruce, Presbyteri et Confessóris, sanctæ Terésiæ in 
 Carmelitárum reformatióne sócii; quem, a Summo Pontífice Benedícto Décimo 
 tértio Sanctis adscríptum, Pius Papa Undécimus Doctórem universális Ecclésiæ 
 declarávit. Ipsíus tamen festívitas ágitur octávo Kaléndas Decémbris.
\switchcolumn
\selectlanguage{english}
\lettrine[lines=2]{A}{t} Ubeda in Spain, the birthday of 
 St. John of the Cross, priest and confessor, and the companion of St. Teresa 
 in the reform of the Carmelites. Pope Benedict XIII placed him on the 
 list of the saints, and Pope Pius XI declared him a doctor of the universal 
 Church. His feast, however, is observed on the 24th of November.
\switchcolumn*
\selectlanguage{latin}
Rhemis, in Gállia, 
 pássio sanctórum Nicásii Epíscopi, ac soróris Eutrópiæ Vírginis, et Sociórum 
 Mártyrum; qui a bárbaris Ecclésiæ hóstibus cæsi sunt.
\switchcolumn
\selectlanguage{english}
At Rheims in France, holy Bishop 
 Nicasius, his sister, the virgin Eutropia, and their companions, martyrs, 
 who were put to death by barbarians hostile to the Church.
\switchcolumn*
\selectlanguage{latin}
Alexandríæ sanctórum 
 Mártyrum Herónis, Arsénii, Isidóri, et Dióscori púeri. Horum tres 
 primos Judex, in persecutióne Deciána, cum eos, váriis torméntis dilánians, 
 pari armátos constántia vidéret, tradi ígnibus jubet; Dióscorus vero, 
 multiplíciter flagellátus, divíno nutu ad consolatiónem fidélium dimíssus 
 est.
\switchcolumn
\selectlanguage{english}
At Alexandria, the holy martyrs 
 Heron, Arsenius, Isidore, and the boy Dioscorus. In the persecution of 
 Decius, the first three were subjected to all the refinements of cruelty by 
 the judge, who, seeing them displaying the same constancy, ordered that they 
 should be cast into the fire. But Dioscorus, after repeated scourgings, 
 was set free by the intervention of Providence to the great consolation of 
 the faithful.
\switchcolumn*
\selectlanguage{latin}
Antiochíæ natális 
 sanctórum Mártyrum Drusi, Zósimi et Theodóri.
\switchcolumn
\selectlanguage{english}
At Antioch, the birthday of the holy 
 martyrs Drusus, Zosimus, and Theodore.
\switchcolumn*
\selectlanguage{latin}
Eódem die pássio 
 sanctórum Justi et Abúndii, qui, sub Numeriáno Imperatóre et Olybrio Præside, 
 conjécti in ignem, et, cum inde evasíssent illæsi, gládio percússi sunt.
\switchcolumn
\selectlanguage{english}
On the same day, the martyrdom of 
 Saints Justus and Abundius, who were cast into the flames in the time of 
 Emperor Numerian and the governor Olybrius, but escaping all injury, they 
 were smitten with the sword.
\switchcolumn*
\selectlanguage{latin}
In Cypro natális beáti 
 Spiridiónis Epíscopi, qui unus fuit ex illis Confessóribus, quos Galérius 
 Maximiánus, dextro óculo effósso et sinístro póplite succíso, ad metálla damnáverat. Hic prophetíæ dono et signórum glória 
 ínclitus fuit, et in 
 Nicæno Concílio philósophum éthnicum, Christiánæ religióni insultántem, 
 devícit et ad fidem perdúxit.
\switchcolumn
\selectlanguage{english}
In the island of Cyprus, the 
 birthday of blessed Spiridion, bishop. He was one of those confessors 
 who were condemned by Galerius Maximian to labour in the mines, after 
 suffering the loss of his right eye and cutting of the sinews of his left 
 knee. This prelate was renowned for the gift of prophecy and glorious 
 miracles, and in the Council of Nicea he confounded a heathen philosopher, 
 who had insulted the Christian religion, and brought him to the faith.
\switchcolumn*
\selectlanguage{latin}
Bérgomi sancti Viatóris, 
 Epíscopi et Confessóris.
\switchcolumn
\selectlanguage{english}
At Bergamo, St. Viator, bishop and 
 confessor.
\switchcolumn*
\selectlanguage{latin}
Papíæ sancti Pompéji 
 Epíscopi.
\switchcolumn
\selectlanguage{english}
At Pavia, St. Pompey, bishop.
\switchcolumn*
\selectlanguage{latin}
Neápoli, in Campánia, 
 sancti Agnélli Abbátis, virtúte miraculórum illústris, qui obséssam urbem 
 sæpe visus est Crucis vexíllo ab hóstibus liberáre.
\switchcolumn
\selectlanguage{english}
At Naples in Campania, St. Agnellus, 
 abbot. Illustrious for the gift of miracles, he was often seen with 
 the standard of the Cross, delivering the city besieged by enemies.
\switchcolumn*
\selectlanguage{latin}
Medioláni sancti 
 Matroniáni Eremítæ.
\switchcolumn
\selectlanguage{english}
At Milan, St. Matronian, hermit.
\switchcolumn*
\selectlanguage{latin}
\end{paracol}


% ---- martyrology/mart12/mart1215.htm
\needspace{10\baselineskip}
\begin{paracol}{2}
\selectlanguage{latin}
\begin{center}{\color{gregoriocolor} Décimo octávo Kaléndas Januárii. 
 Luna\dots\ }\end{center}
\switchcolumn
\selectlanguage{english}
\begin{center}{\color{gregoriocolor} The 
 Fifteenth Day of 
 December. The\dots\ Day of the Moon.}\end{center}
\end{paracol}

\noindent\begin{tabularx}{\linewidth}{*{19}{>{\centering\arraybackslash}X}}
 \textcolor{gregoriocolor}{a} & \textcolor{gregoriocolor}{b} & \textcolor{gregoriocolor}{c} & \textcolor{gregoriocolor}{d} & \textcolor{gregoriocolor}{e} & \textcolor{gregoriocolor}{f} & \textcolor{gregoriocolor}{g} & \textcolor{gregoriocolor}{h} & \textcolor{gregoriocolor}{i} & \textcolor{gregoriocolor}{k} & \textcolor{gregoriocolor}{l} & \textcolor{gregoriocolor}{m} & \textcolor{gregoriocolor}{n} & \textcolor{gregoriocolor}{p} & \textcolor{gregoriocolor}{q} & \textcolor{gregoriocolor}{r} & \textcolor{gregoriocolor}{s} & \textcolor{gregoriocolor}{t} & \textcolor{gregoriocolor}{u} \\
 25 & 26 & 27 & 28 & 29 & 1 & 2 & 3 & 4 & 5 & 6 & 7 & 8 & 9 & 10 & 11 & 12 & 13 & 14 \\
\end{tabularx}
\vspace{0.5\baselineskip}
\noindent\begin{tabularx}{\linewidth}{*{12}{>{\centering\arraybackslash}X}}
 \textcolor{gregoriocolor}{A} & \textcolor{gregoriocolor}{B} & \textcolor{gregoriocolor}{C} & \textcolor{gregoriocolor}{D} & \textcolor{gregoriocolor}{E} & F & \textcolor{gregoriocolor}{F} & \textcolor{gregoriocolor}{G} & \textcolor{gregoriocolor}{H} & \textcolor{gregoriocolor}{M} & \textcolor{gregoriocolor}{N} & \textcolor{gregoriocolor}{P} \\
 15 & 16 & 17 & 18 & 19 & 20 & 19 & 20 & 21 & 22 & 23 & 24 \\
\end{tabularx}

\begin{paracol}{2}
\selectlanguage{latin}
\lettrine[lines=2]{O}{ctáva} Conceptiónis 
 Immaculátæ beátæ Maríæ Vírginis.
\switchcolumn
\selectlanguage{english}
\lettrine[lines=2]{T}{he} Octave of the Immaculate 
 Conception of the Blessed Virgin Mary.
\switchcolumn*
\selectlanguage{latin}
Romæ sanctórum Mártyrum 
 Irenæi, Antónii, Theodóri, Saturníni, Victóris, et aliórum decem et septem, 
 qui, in persecutióne Valeriáni, pro Christo passi sunt.
\switchcolumn
\selectlanguage{english}
At Rome, the holy martyrs Irenaeus, 
 Anthony, Theodore, Saturninus, Victor, and seventeen others who suffered for 
 Christ in the persecution of Valerian.
\switchcolumn*
\selectlanguage{latin}
In Africa, pássio 
 sanctórum Faustíni, Lúcii, Cándidi, Cæliáni, Marci, Januárii et Fortunáti.
\switchcolumn
\selectlanguage{english}
In Africa, the martyrdom of Saints 
 Faustinus, Lucius, Candidus, Cælian, Mark, Januarius, and Fortunatus.
\switchcolumn*
\selectlanguage{latin}
Ibídem sancti Valeriáni 
 Epíscopi, qui, cum esset annórum plus octogínta, in persecutióne Wandálica, 
 sub Rege Ariáno Genseríco, convéntus ab eo ut tráderet Ecclésiæ utensília, 
 idque constánter renuísset, extra civitátem singuláris jussus est pelli; cumque præcéptum esset ut nullus eum neque in domo neque in agro dimítteret 
 habitáre, multo témpore in strata pública nudo sub áere jácuit, et, in 
 confessióne et defensióne cathólicæ veritátis, cursum beátæ vitæ complévit.
\switchcolumn
\selectlanguage{english}
In the same country, the holy bishop 
 Valerian, who, being upwards of eighty years of age, in the persecution of 
 the Vandals, under the Arian king Genseric, was asked to deliver the vessels 
 of the Church, and as he constantly refused, an order was issued to drive 
 him all alone out of the city, and all persons were forbidden to allow him 
 to stay in their houses or on their land. For a long time he remained 
 lying on the public road, in the open air, and thus in the confession and 
 defence of Catholic truth he ended his blessed life.
\switchcolumn*
\selectlanguage{latin}
In territorio 
 Aurelianénsi sancti Maximíni Confessóris.
\switchcolumn
\selectlanguage{english}
In the territory of Orleans, St. 
 Maximin, confessor.
\switchcolumn*
\selectlanguage{latin}
Apud Ibéros, trans 
 Pontum Euxínum, sanctæ Christiánæ ancíllæ, quæ miraculórum virtúte gentem 
 illam, témpore Constantíni, ad Christi fidem perdúxit.
\switchcolumn
\selectlanguage{english}
Among the Iberians across the Euxine 
 Sea, St. Christiana, a maidservant, who by virtue of her miracles led that 
 people to the faith of Christ, in the time of Constantine.
\switchcolumn*
\selectlanguage{latin}
Vercéllis Ordinátio 
 sancti Eusébii, Epíscopi et Mártyris.
\switchcolumn
\selectlanguage{english}
At Vercelli, the ordination of St. 
 Eusebius, bishop and martyr.
\switchcolumn*
\selectlanguage{latin}
\end{paracol}


% ---- martyrology/mart12/mart1216.htm
\needspace{10\baselineskip}
\begin{paracol}{2}
\selectlanguage{latin}
\begin{center}{\color{gregoriocolor} Décimo séptimo Kaléndas Januárii. 
 Luna\dots\ }\end{center}
\switchcolumn
\selectlanguage{english}
\begin{center}{\color{gregoriocolor} The 
 Sixteenth Day of 
 December. The\dots\ Day of the Moon.}\end{center}
\end{paracol}

\noindent\begin{tabularx}{\linewidth}{*{19}{>{\centering\arraybackslash}X}}
 \textcolor{gregoriocolor}{a} & \textcolor{gregoriocolor}{b} & \textcolor{gregoriocolor}{c} & \textcolor{gregoriocolor}{d} & \textcolor{gregoriocolor}{e} & \textcolor{gregoriocolor}{f} & \textcolor{gregoriocolor}{g} & \textcolor{gregoriocolor}{h} & \textcolor{gregoriocolor}{i} & \textcolor{gregoriocolor}{k} & \textcolor{gregoriocolor}{l} & \textcolor{gregoriocolor}{m} & \textcolor{gregoriocolor}{n} & \textcolor{gregoriocolor}{p} & \textcolor{gregoriocolor}{q} & \textcolor{gregoriocolor}{r} & \textcolor{gregoriocolor}{s} & \textcolor{gregoriocolor}{t} & \textcolor{gregoriocolor}{u} \\
 26 & 27 & 28 & 29 & 1 & 2 & 3 & 4 & 5 & 6 & 7 & 8 & 9 & 10 & 11 & 12 & 13 & 14 & 15 \\
\end{tabularx}
\vspace{0.5\baselineskip}
\noindent\begin{tabularx}{\linewidth}{*{12}{>{\centering\arraybackslash}X}}
 \textcolor{gregoriocolor}{A} & \textcolor{gregoriocolor}{B} & \textcolor{gregoriocolor}{C} & \textcolor{gregoriocolor}{D} & \textcolor{gregoriocolor}{E} & F & \textcolor{gregoriocolor}{F} & \textcolor{gregoriocolor}{G} & \textcolor{gregoriocolor}{H} & \textcolor{gregoriocolor}{M} & \textcolor{gregoriocolor}{N} & \textcolor{gregoriocolor}{P} \\
 16 & 17 & 18 & 19 & 20 & 21 & 20 & 21 & 22 & 23 & 24 & 25 \\
\end{tabularx}

\begin{paracol}{2}
\selectlanguage{latin}
\lettrine[lines=2]{S}{ancti} Eusébii, Epíscopi Vercellénsis et Mártyris; cujus dies natális 
 Kaléndis Augústi, et Ordinátio décimo octávo Kaléndas Januárii refértur.
\switchcolumn
\selectlanguage{english}
\lettrine[lines=2]{S}{t.} Eusebius, bishop of Vercelli and martyr. His birthday is 
 commemorated on the 1st of August and his ordination on the 15th of 
 December.
\switchcolumn*
\selectlanguage{latin}
Sanctórum Trium Puerórum, id est Ananíæ, Azaríæ 
 et Misaélis; quorum córpora apud Babylóniam, sub quodam specu, sunt pósita.
\switchcolumn
\selectlanguage{english}
The three young men, Ananias, Azarias, and Misael, whose bodies are buried 
 in a cave near Babylon.
\switchcolumn*
\selectlanguage{latin}
Ravénnæ sanctórum Mártyrum Valentíni, magístri 
 mílitum, ejúsque fílii Concórdii, atque Navális et Agrícolæ; qui, in 
 persecutióne Maximiáni, pro Christo passi sunt.
\switchcolumn
\selectlanguage{english}
At 
 Ravenna, the holy martyrs Valentine, an officer of the army, Concordius, his 
 son, Navalis, and Agricola, who suffered for Christ in the persecution of 
 Maximian.
\switchcolumn*
\selectlanguage{latin}
Fórmiis, in Campánia, sanctæ Albínæ, Vírginis 
 et Mártyris, sub Décio Imperatóre.
\switchcolumn
\selectlanguage{english}
At 
 Mola di Gaeta in Campania, St. Albina, virgin and martyr, under Emperor 
 Decius.
\switchcolumn*
\selectlanguage{latin}
In 
 Africa pássio plurimárum sanctárum Vírginum, quæ, 
 in persecutióne Wandálica, sub Ariáno Rege Hunneríco, suspéndia, póndera 
 laminásque ignítas perpéssæ, martyrii agónem felíciter consummárunt.
\switchcolumn
\selectlanguage{english}
In 
 Africa, many holy virgins who reached a happy end of their martyrdom in the 
 persecution of the Vandals under the Arian king Hunneric by having heavy 
 weights tied to them and burning plates of metal applied to their bodies.
\switchcolumn*
\selectlanguage{latin}
Viénnæ, in Gállia, beáti Adónis, Epíscopi et 
 Confessóris.
\switchcolumn
\selectlanguage{english}
At 
 Vienne in France, blessed Ado, bishop and confessor.
\switchcolumn*
\selectlanguage{latin}
In 
 Hibérnia sancti Beáni Epíscopi.
\switchcolumn
\selectlanguage{english}
In 
 Ireland, St. Bean, bishop.
\switchcolumn*
\selectlanguage{latin}
Gazæ, in Palæstína, sancti Ireniónis Epíscopi.
\switchcolumn
\selectlanguage{english}
At Gaza in Palestine, St. Irenion, bishop.
\switchcolumn*
\selectlanguage{latin}
\end{paracol}


% ---- martyrology/mart12/mart1217.htm
\needspace{10\baselineskip}
\begin{paracol}{2}
\selectlanguage{latin}
\begin{center}{\color{gregoriocolor} Sextodécimo Kaléndas Januárii. 
 Luna\dots\ }\end{center}
\switchcolumn
\selectlanguage{english}
\begin{center}{\color{gregoriocolor} The 
 Seventeenth Day of 
 December. The\dots\ Day of the Moon.}\end{center}
\end{paracol}

\noindent\begin{tabularx}{\linewidth}{*{19}{>{\centering\arraybackslash}X}}
 \textcolor{gregoriocolor}{a} & \textcolor{gregoriocolor}{b} & \textcolor{gregoriocolor}{c} & \textcolor{gregoriocolor}{d} & \textcolor{gregoriocolor}{e} & \textcolor{gregoriocolor}{f} & \textcolor{gregoriocolor}{g} & \textcolor{gregoriocolor}{h} & \textcolor{gregoriocolor}{i} & \textcolor{gregoriocolor}{k} & \textcolor{gregoriocolor}{l} & \textcolor{gregoriocolor}{m} & \textcolor{gregoriocolor}{n} & \textcolor{gregoriocolor}{p} & \textcolor{gregoriocolor}{q} & \textcolor{gregoriocolor}{r} & \textcolor{gregoriocolor}{s} & \textcolor{gregoriocolor}{t} & \textcolor{gregoriocolor}{u} \\
 27 & 28 & 29 & 1 & 2 & 3 & 4 & 5 & 6 & 7 & 8 & 9 & 10 & 11 & 12 & 13 & 14 & 15 & 16 \\
\end{tabularx}
\vspace{0.5\baselineskip}
\noindent\begin{tabularx}{\linewidth}{*{12}{>{\centering\arraybackslash}X}}
 \textcolor{gregoriocolor}{A} & \textcolor{gregoriocolor}{B} & \textcolor{gregoriocolor}{C} & \textcolor{gregoriocolor}{D} & \textcolor{gregoriocolor}{E} & F & \textcolor{gregoriocolor}{F} & \textcolor{gregoriocolor}{G} & \textcolor{gregoriocolor}{H} & \textcolor{gregoriocolor}{M} & \textcolor{gregoriocolor}{N} & \textcolor{gregoriocolor}{P} \\
 17 & 18 & 19 & 20 & 21 & 22 & 21 & 22 & 23 & 24 & 25 & 26 \\
\end{tabularx}

\begin{paracol}{2}
\selectlanguage{latin}
\lettrine[lines=2]{R}{omæ} natális sancti Joánnis de Matha, 
 Presbyteri et Confessóris, qui Ordinis sanctíssimæ Trinitátis redemptiónis 
 captivórum Fundátor éxstitit. Ipsíus tamen festívitas, ex dispositióne 
 Innocéntii Papæ Undécimi, ágitur sexto Idus Februárii.
\switchcolumn
\selectlanguage{english}
\lettrine[lines=2]{A}{t} Rome, the birthday of St. John of Matha, priest and confessor, founder of 
 the Order of the Most Holy Trinity for the Redemption of Captives, whose 
 feast, by decree of Pope Innocent XI, is observed on the 8th of 
 February.
\switchcolumn*
\selectlanguage{latin}
Massíliæ, in Gállia, beáti Lázari Epíscopi, 
 sanctárum Maríæ Magdalénæ ac Marthæ fratris, quem Dóminus in Evangélio 
 appellásse amícum et a mórtuis excitásse légitur.
\switchcolumn
\selectlanguage{english}
At Marseilles in France, blessed Lazarus, brother of the Saints Mary 
 Magdalene and Martha, of whom we read in the Gospel that our Lord called him 
 his friend and raised him from the dead.
\switchcolumn*
\selectlanguage{latin}
Eleutherópoli, in Palæstína, sanctórum Mártyrum 
 Floriáni, Calaníci, et Sociórum quinquagínta et octo; qui, témpore Heraclíi 
 Imperatóris, a Saracénis ob Christi fidem occísi sunt.
\switchcolumn
\selectlanguage{english}
At 
 Eleutheropolis, the holy martyrs Florian, Calanicus, and their fifty-eight 
 companions, who were slain by the Saracens because of the faith of Christ, 
 during the reign of Emperor Haraclius.
\switchcolumn*
\selectlanguage{latin}
In monastério Fuldénsi sancti Stúrmii, Abbátis et Saxóniæ 
 Apóstoli; quem Innocéntius Papa Secúndus, in Concílio secúndo Lateranénsi, 
 in Sanctórum númerum rétulit.
\switchcolumn
\selectlanguage{english}
In 
 the monastery of Fulda, the holy abbot Sturmius, apostle of Saxony, who was 
 ranked among the saints by Innocent II, in the second Lateran Council.
\switchcolumn*
\selectlanguage{latin}
Bigárdis, prope Bruxéllas, in Brabántia, sanctæ 
 Wivínæ Vírginis, cujus egrégiam sanctitátem mirácula crebra testántur.
\switchcolumn
\selectlanguage{english}
At 
 Bigarden, near Brussels, St. Wivina, virgin, whose eminent sanctity is 
 attested to by frequent miracles.
\switchcolumn*
\selectlanguage{latin}
Constantinópoli sanctæ Olympíadis Víduæ.
\switchcolumn
\selectlanguage{english}
At 
 Constantinople, St. Olympias, widow.
\switchcolumn*
\selectlanguage{latin}
Andániæ, apud Septem Ecclésias, in Bélgio, 
 beátæ Beggæ Víduæ, quæ fuit soror sanctæ Gertrúdis.
\switchcolumn
\selectlanguage{english}
At 
 Andenne, at the Seven Churches, blessed Begga, widow, the sister of St. 
 Gertrude.
\switchcolumn*
\selectlanguage{latin}
Eódem die Translátio sancti Ignátii, Epíscopi et Mártyris; qui, tértius post 
 beátum Petrum Apóstolum, Antiochénam rexit Ecclésiam. Ejus corpus ab 
 urbe Roma, ubi ipse, sub Trajáno, glorióse martyrium tertiodécimo Kaléndas 
 Januárii consummáverat, Antiochíam delátum, ibídem, in cœmetério 
 Ecclésiæ, extra portam Daphníticam, pósitum fuit; in qua celebritáte sanctus 
 Joánnes Chrysóstomus conciónem ad pópulum hábuit. Póstmodum vero ejus 
 relíquiæ rursus Romam translátæ sunt, et in Ecclésia sancti Cleméntis, una 
 cum córpore ejúsdem beatíssimi Papæ et Mártyris, summa veneratióne recónditæ.
\switchcolumn
\selectlanguage{english}
Also, the translation of St. Ignatius, bishop and martyr, who, the third 
 after the blessed Apostle Peter, governed the Church of Antioch. His 
 body was taken from Rome, where he had suffered martyrdom under Trajan on 
 the 20th of December, and deposited in the church cemetery near the Gate of 
 Daphne at Antioch. St. John Chrysostom, on that solemn occasion, 
 preached the sermon to the people. Afterwards his relics were carried 
 back to Rome and placed with the highest reverence in the church of St. 
 Clement, together with the body of that blessed pope and martyr.
\switchcolumn*
\selectlanguage{latin}
\end{paracol}


% ---- martyrology/mart12/mart1218.htm
\needspace{10\baselineskip}
\begin{paracol}{2}
\selectlanguage{latin}
\begin{center}{\color{gregoriocolor} Quintodécimo Kaléndas Januárii. 
 Luna\dots\ }\end{center}
\switchcolumn
\selectlanguage{english}
\begin{center}{\color{gregoriocolor} The 
 Eighteenth Day of 
 December. The\dots\ Day of the Moon.}\end{center}
\end{paracol}

\noindent\begin{tabularx}{\linewidth}{*{19}{>{\centering\arraybackslash}X}}
 \textcolor{gregoriocolor}{a} & \textcolor{gregoriocolor}{b} & \textcolor{gregoriocolor}{c} & \textcolor{gregoriocolor}{d} & \textcolor{gregoriocolor}{e} & \textcolor{gregoriocolor}{f} & \textcolor{gregoriocolor}{g} & \textcolor{gregoriocolor}{h} & \textcolor{gregoriocolor}{i} & \textcolor{gregoriocolor}{k} & \textcolor{gregoriocolor}{l} & \textcolor{gregoriocolor}{m} & \textcolor{gregoriocolor}{n} & \textcolor{gregoriocolor}{p} & \textcolor{gregoriocolor}{q} & \textcolor{gregoriocolor}{r} & \textcolor{gregoriocolor}{s} & \textcolor{gregoriocolor}{t} & \textcolor{gregoriocolor}{u} \\
 28 & 29 & 1 & 2 & 3 & 4 & 5 & 6 & 7 & 8 & 9 & 10 & 11 & 12 & 13 & 14 & 15 & 16 & 17 \\
\end{tabularx}
\vspace{0.5\baselineskip}
\noindent\begin{tabularx}{\linewidth}{*{12}{>{\centering\arraybackslash}X}}
 \textcolor{gregoriocolor}{A} & \textcolor{gregoriocolor}{B} & \textcolor{gregoriocolor}{C} & \textcolor{gregoriocolor}{D} & \textcolor{gregoriocolor}{E} & F & \textcolor{gregoriocolor}{F} & \textcolor{gregoriocolor}{G} & \textcolor{gregoriocolor}{H} & \textcolor{gregoriocolor}{M} & \textcolor{gregoriocolor}{N} & \textcolor{gregoriocolor}{P} \\
 18 & 19 & 20 & 21 & 22 & 23 & 22 & 23 & 24 & 25 & 26 & 27 \\
\end{tabularx}

\begin{paracol}{2}
\selectlanguage{latin}
\lettrine[lines=2]{P}{hilíppis,} in Macedónia, natális sanctórum Mártyrum Rufi et Zósimi, 
 qui ex illórum número discipulórum fuérunt, per quos primitíva Ecclésia in 
 Judæis et Græcis fundáta est; de quorum étiam felíci agóne scribit sanctus 
 Polycárpus in epístola ad Philippénses.
\switchcolumn
\selectlanguage{english}
\lettrine[lines=2]{A}{t} Philippi in Macedonia, the birthday of the holy martyrs Rufus and Zosimus, 
 who were of the number of disciples by whom the primitive church was founded 
 among the Jews and the Greeks. Their happy martyrdom is mentioned by 
 St. Polycarp in his Epistle to the Philippians.
\switchcolumn*
\selectlanguage{latin}
Laodicéæ, in Syria, pássio sanctórum Theotími 
 et Basiliáni.
\switchcolumn
\selectlanguage{english}
At Laodicea in Syria, the martyrdom of the Saints Theotimus and Basilian.
\switchcolumn*
\selectlanguage{latin}
In Africa sanctórum Mártyrum Quincti, Simplícii et aliórum; qui sub Décii et 
 Valeriáni persecutióne passi sunt.
\switchcolumn
\selectlanguage{english}
In 
 Africa, the holy martyrs Quinctus, Simplicius, and others who suffered in 
 the persecution of Decius and Valerian.
\switchcolumn*
\selectlanguage{latin}
Ibídem sancti Moysétis Mártyris.
\switchcolumn
\selectlanguage{english}
In 
 the same country, St. Moses, martyr.
\switchcolumn*
\selectlanguage{latin}
Item in Africa sanctórum Mártyrum Victúri, Victóris, Victoríni, Adjutóris, Quarti et 
 aliórum trigínta.
\switchcolumn
\selectlanguage{english}
Also in Africa, the holy martyrs Victurus, Victor, Victorinus, Adjutor, 
 Quartus, and thirty others.
\switchcolumn*
\selectlanguage{latin}
Mopsuéstiæ, in Cilícia, sancti Auxéntii 
 Epíscopi, qui, olim sub Licínio miles, pótius elégit cíngulum exúere quam 
 uvas Baccho offérre; factúsque Epíscopus, præclárus méritis quiévit in pace.
\switchcolumn
\selectlanguage{english}
At 
 Mopsuestia in Cilicia, St. Auxentius, bishop, who, being at first a soldier 
 under Licinius, preferred to surrender his military insignia rather than 
 offer grapes to Bacchus. Having been made a bishop, he was renowned 
 for his merit, and died in peace.
\switchcolumn*
\selectlanguage{latin}
Turónis, in Gállia, sancti Gratiáni Epíscopi, qui, a sancto Fabiáno Papa 
 primus ejúsdem civitátis Epíscopus ordinátus est, et multis clarus miráculis 
 obdormívit in Dómino.
\switchcolumn
\selectlanguage{english}
At 
 Tours in France, St. Gratian, appointed first bishop of that city by Pope 
 St. Fabian. Celebrated for many miracles, he calmly went to his repose 
 in the Lord.
\switchcolumn*
\selectlanguage{latin}
\end{paracol}


% ---- martyrology/mart12/mart1219.htm
\needspace{10\baselineskip}
\begin{paracol}{2}
\selectlanguage{latin}
\begin{center}{\color{gregoriocolor} Quartodécimo Kaléndas Januárii. 
 Luna\dots\ }\end{center}
\switchcolumn
\selectlanguage{english}
\begin{center}{\color{gregoriocolor} The 
 Ninteenth Day of 
 December. The\dots\ Day of the Moon.}\end{center}
\end{paracol}

\noindent\begin{tabularx}{\linewidth}{*{19}{>{\centering\arraybackslash}X}}
 \textcolor{gregoriocolor}{a} & \textcolor{gregoriocolor}{b} & \textcolor{gregoriocolor}{c} & \textcolor{gregoriocolor}{d} & \textcolor{gregoriocolor}{e} & \textcolor{gregoriocolor}{f} & \textcolor{gregoriocolor}{g} & \textcolor{gregoriocolor}{h} & \textcolor{gregoriocolor}{i} & \textcolor{gregoriocolor}{k} & \textcolor{gregoriocolor}{l} & \textcolor{gregoriocolor}{m} & \textcolor{gregoriocolor}{n} & \textcolor{gregoriocolor}{p} & \textcolor{gregoriocolor}{q} & \textcolor{gregoriocolor}{r} & \textcolor{gregoriocolor}{s} & \textcolor{gregoriocolor}{t} & \textcolor{gregoriocolor}{u} \\
 29 & 1 & 2 & 3 & 4 & 5 & 6 & 7 & 8 & 9 & 10 & 11 & 12 & 13 & 14 & 15 & 16 & 17 & 18 \\
\end{tabularx}
\vspace{0.5\baselineskip}
\noindent\begin{tabularx}{\linewidth}{*{12}{>{\centering\arraybackslash}X}}
 \textcolor{gregoriocolor}{A} & \textcolor{gregoriocolor}{B} & \textcolor{gregoriocolor}{C} & \textcolor{gregoriocolor}{D} & \textcolor{gregoriocolor}{E} & F & \textcolor{gregoriocolor}{F} & \textcolor{gregoriocolor}{G} & \textcolor{gregoriocolor}{H} & \textcolor{gregoriocolor}{M} & \textcolor{gregoriocolor}{N} & \textcolor{gregoriocolor}{P} \\
 19 & 20 & 21 & 22 & 23 & 24 & 23 & 24 & 25 & 26 & 27 & 28 \\
\end{tabularx}

\begin{paracol}{2}
\selectlanguage{latin}
\lettrine[lines=2]{I}{n} Mauritánia sancti Timóthei Diáconi, qui, ob Christi fidem, post diros cárceres, in ignem conjéctus, martyrium consummávit.
\switchcolumn
\selectlanguage{english}
\lettrine[lines=2]{I}{n} Morocco, St. Timothy, deacon, who after severe imprisonment for the sake 
 of Christ was cast into the fire and achieved martyrdom.
\switchcolumn*
\selectlanguage{latin}
Alexandríæ beáti Nemésii Mártyris, qui, primo 
 per calúmniam quasi latro Júdici delátus, eóque crímine absolútus, mox, in 
 persecutióne Décii, Christiánæ religiónis nómine accusátus, cum latrónibus 
 jussus est incéndi, Salvatóris déferens similitúdinem, qui una cum latrónibus 
 pértulit crucem.
\switchcolumn
\selectlanguage{english}
At Alexandria in Egypt, blessed Nemesius, martyr, who first was denounced 
 before the judge as a robber, and being freed from that charge, soon after, 
 in the persecution of Decius, was accused before the judge Emilian of being 
 a Christian. He was twice subjected to torture and condemned to be 
 burned alive with robbers, thus bearing a resemblance to our Saviour, who 
 was crucified with thieves.
\switchcolumn*
\selectlanguage{latin}
Nicææ, in Bithynia, sanctórum Mártyrum Daríi, 
 Zósimi, Pauli et Secúndi.
\switchcolumn
\selectlanguage{english}
At Nicaea, the Saints Darius, Zosimus, Paul, and Secundus, martyrs.
\switchcolumn*
\selectlanguage{latin}
Nicomedíæ sanctórum Mártyrum Cyríaci, Paulílli, 
 Secúndi, Anastásii, Syndímii et Sociórum.
\switchcolumn
\selectlanguage{english}
At Nicomedia, the holy martyrs Cyriac, Paulillus, Secundus, Anastasius, 
 Sindimius, and their companions.
\switchcolumn*
\selectlanguage{latin}
Gazæ, in Palæstína, pássio sanctárum Meuris et 
 Theæ.
\switchcolumn
\selectlanguage{english}
At Gaza in Palestine, the martyrdom of Saints Meuris and Thea.
\switchcolumn*
\selectlanguage{latin}
Romæ deposítio sancti Anastásii Papæ Primi, 
 viri ditíssimæ paupertátis et apostólicæ sollicitúdinis, quem (ut ait 
 sanctus Hierónymus) diu Roma habére non méruit, ne orbis caput sub tali 
 Epíscopo truncarétur; nam, haud multo post ejus óbitum, Roma a Gothis capta 
 et dirépta fuit.
\switchcolumn
\selectlanguage{english}
At Rome, the death of Pope St. Anastasius I, a man who was rich in his 
 poverty and filled with apostolic zeal. St. Jerome says that Rome did 
 not deserve to possess him long, lest the capital of the world should be 
 devastated under so fine a bishop, for shortly after his death Rome was 
 taken and sacked by the Goths.
\switchcolumn*
\selectlanguage{latin}
Antisiodóri sancti Gregórii, Epíscopi et Confessóris.
\switchcolumn
\selectlanguage{english}
At Auxerre, St. Gregory, bishop and confessor.
\switchcolumn*
\selectlanguage{latin}
Aureliánis, in Gállia, sancti Adjúti Abbátis, prophético spíritu illústris.
\switchcolumn
\selectlanguage{english}
At Orleans in France, St. Adjutus, abbot, famous for the spirit of prophecy.
\switchcolumn*
\selectlanguage{latin}
Romæ sanctæ Faustæ, quæ fuit mater sanctæ 
 Anastásiæ, ac nobilitáte et pietáte éxstitit insígnis.
\switchcolumn
\selectlanguage{english}
At Rome, St. Fausta, mother of St. Anastasia, renowned for her noble birth 
 and her holiness.
\switchcolumn*
\selectlanguage{latin}
Avenióne beáti Urbáni Papæ Quinti, qui, Sede 
 Apostólica Romæ restitúta, Græcórum cum Latínis conjunctióne perfécta, 
 infidélibus coércitis, de Ecclésia óptime méritus est. Ejus cultum 
 pervetústum Pius Nonus, Póntifex Máximus, ratum hábuit et confirmávit.
\switchcolumn
\selectlanguage{english}
At 
 Avignon, blessed Urban V, who deserved well of the Church by restoring the 
 Apostolic See to Rome, by bringing about a reunion of the Latins and the 
 Greeks, and by suppressing heretics. Pius IX approved and confirmed 
 the veneration which had long been paid to him.
\switchcolumn*
\selectlanguage{latin}
\end{paracol}


% ---- martyrology/mart12/mart1220.htm
\needspace{10\baselineskip}
\begin{paracol}{2}
\selectlanguage{latin}
\begin{center}{\color{gregoriocolor} Tertiodécimo Kaléndas Januárii. 
 Luna\dots\ }\end{center}
\switchcolumn
\selectlanguage{english}
\begin{center}{\color{gregoriocolor} The 
 Twentieth Day of 
 December. The\dots\ Day of the Moon.}\end{center}
\end{paracol}

\noindent\begin{tabularx}{\linewidth}{*{19}{>{\centering\arraybackslash}X}}
 \textcolor{gregoriocolor}{a} & \textcolor{gregoriocolor}{b} & \textcolor{gregoriocolor}{c} & \textcolor{gregoriocolor}{d} & \textcolor{gregoriocolor}{e} & \textcolor{gregoriocolor}{f} & \textcolor{gregoriocolor}{g} & \textcolor{gregoriocolor}{h} & \textcolor{gregoriocolor}{i} & \textcolor{gregoriocolor}{k} & \textcolor{gregoriocolor}{l} & \textcolor{gregoriocolor}{m} & \textcolor{gregoriocolor}{n} & \textcolor{gregoriocolor}{p} & \textcolor{gregoriocolor}{q} & \textcolor{gregoriocolor}{r} & \textcolor{gregoriocolor}{s} & \textcolor{gregoriocolor}{t} & \textcolor{gregoriocolor}{u} \\
 1 & 2 & 3 & 4 & 5 & 6 & 7 & 8 & 9 & 10 & 11 & 12 & 13 & 14 & 15 & 16 & 17 & 18 & 19 \\
\end{tabularx}
\vspace{0.5\baselineskip}
\noindent\begin{tabularx}{\linewidth}{*{12}{>{\centering\arraybackslash}X}}
 \textcolor{gregoriocolor}{A} & \textcolor{gregoriocolor}{B} & \textcolor{gregoriocolor}{C} & \textcolor{gregoriocolor}{D} & \textcolor{gregoriocolor}{E} & F & \textcolor{gregoriocolor}{F} & \textcolor{gregoriocolor}{G} & \textcolor{gregoriocolor}{H} & \textcolor{gregoriocolor}{M} & \textcolor{gregoriocolor}{N} & \textcolor{gregoriocolor}{P} \\
 20 & 21 & 22 & 23 & 24 & 25 & 24 & 25 & 26 & 27 & 28 & 29 \\
\end{tabularx}

\begin{paracol}{2}
\selectlanguage{latin}
\lettrine[lines=1]{V}{igília} sancti Thomæ Apóstoli.
\switchcolumn
\selectlanguage{english}
\lettrine[lines=1]{T}{he} Vigil of St. Thomas, Apostle.
\switchcolumn*
\selectlanguage{latin}
Romæ natális sancti Zephyríni, Papæ et Mártyris. 
 Ipsíus tamen festum recólitur séptimo Kaléndas Septémbris.
\switchcolumn
\selectlanguage{english}
At Rome, the birthday of St. Zephyrinus, pope and martyr. His feast is 
 celebrated on the 26th of August.
\switchcolumn*
\selectlanguage{latin}
Ibídem pássio sancti Ignátii, Epíscopi et Mártyris; qui, tértius post beátum 
 Petrum Apóstolum, Antiochénam rexit Ecclésiam. Hic, in persecutióne 
 Trajáni, damnátus ad béstias, Romam vinctus míttitur; ibíque, circumsedénte 
 Senátu, immaníssimis pœnárum supplíciis primo 
 est afféctus, dehinc objícitur leónibus, quorum déntibus præfocátus, hóstia 
 Christi effícitur. Ejus vero festívitas Kaléndis Februárii celebrátur.
\switchcolumn
\selectlanguage{english}
In the same city, the martyrdom of St. Ignatius, bishop and martyr. He 
 was the third after St. Peter the Apostle to rule the church of Antioch, and 
 in the persecution of Trajan was condemned to the beasts. By order of 
 Trajan he was sent to Rome in fetters, and there tortured and afflicted with 
 the most cruel torments in the midst of the assembled Senate. Finally 
 he was cast to the lions, and being ground by their teeth became a sacrifice 
 for Christ. His feast is observed on the 1st of February.
\switchcolumn*
\selectlanguage{latin}
Item Romæ sanctórum Mártyrum Liberáti et Bájuli.
\switchcolumn
\selectlanguage{english}
At Rome, the holy martyrs Liberatus and Bajulus.
\switchcolumn*
\selectlanguage{latin}
In Arábia sanctórum Mártyrum Eugénii et Macárii Presbyterórum, qui a Juliáno Apóstata, cum ipsíus impietátem arguíssent, sævíssimis 
 plagis affécti sunt, atque in vastíssimam erémum relegáti, et gládio cæsi.
\switchcolumn
\selectlanguage{english}
In Arabia, the holy martyrs Eugene and Macarius, priests. For 
 reproving Julian the Apostate for his impiety, they received severe stripes, 
 were banished to a vast desert, and finally were put to the sword.
\switchcolumn*
\selectlanguage{latin}
Alexandríæ sanctórum mílitum et Mártyrum 
 Ammónis, Zenónis, Ptolemæi, Ingenis et Theóphili; qui, tribunálibus astántes, 
 cum quidam Christiánus, in supplíciis pósitis, trepidáret et jam prope ad 
 negándum declináret, vultu, óculis ac nútibus illum conabántur erígere. 
 Cumque hac de causa clamor totíus pópuli in eos prosilíret, prorumpéntes in 
 médium se Christiános esse testáti sunt; per quorum victóriam Christus, qui 
 suis eam ánimi constántiam déderat, gloriosíssime triumphávit.
\switchcolumn
\selectlanguage{english}
At Alexandria, the holy martyrs Ammon, Zeno, Ptolemy, Ingen, and Theophilus, 
 soldiers. Standing near the tribunals, and seeing a Christian under 
 torture and almost ready to apostatize, they endeavoured to encourage him by 
 their looks and by signs. When on account of this the crowd raised an 
 outcry against them, they stepped forward and declared themselves 
 Christians. In their victory, Christ also who had given them fortitude 
 triumphed.
\switchcolumn*
\selectlanguage{latin}
Géldubæ, in 
 Germánia, sancti Júlii Mártyris.
\switchcolumn
\selectlanguage{english}
At Gelduba in Germany, St. Julius, martyr.
\switchcolumn*
\selectlanguage{latin}
Antiochíæ natális sancti Philogónii Epíscopi, 
 qui, Dei nutu ex causídico ad eam Ecclésiam regéndam accersítus, advérsus 
 Aríum, una cum sancto Alexándro Epíscopo et Sóciis, primum pro fide 
 cathólica certámen íniit, clarúsque méritis quiévit in Dómino; cujus ánnuam 
 festivitátem sanctus Joánnes Chrysóstomus præcláro encómio celebrávit.
\switchcolumn
\selectlanguage{english}
At Antioch, the birthday of St. Philogonius, bishop, who was called by the 
 will of God from the office of lawyer to the government of that church. 
 With the saintly bishop Alexander and his companions, he engaged in the 
 first contest for the Catholic faith against Arius. Renowned for 
 merits he rested in the Lord, and his feast was commemorated by St. John 
 Chrysostom with an excellent eulogy.
\switchcolumn*
\selectlanguage{latin}
Bríxiæ sancti Domínici, Epíscopi et Confessóris.
\switchcolumn
\selectlanguage{english}
At Brescia, St. Dominic, bishop and confessor.
\switchcolumn*
\selectlanguage{latin}
In Hispánia deposítio sancti Domínici de Silos Abbátis, e sancti Benedícti 
 Ordine, miráculis in captivórum liberatióne celebérrimi.
\switchcolumn
\selectlanguage{english}
In 
 Spain, the death of St. Dominic of Silos, abbot of the Order of St. 
 Benedict, renowned for the miracles which he had wrought for the liberation 
 of captives.
\switchcolumn*
\selectlanguage{latin}
\end{paracol}


% ---- martyrology/mart12/mart1221.htm
\needspace{10\baselineskip}
\begin{paracol}{2}
\selectlanguage{latin}
\begin{center}{\color{gregoriocolor} Duodécimo Kaléndas Januárii. 
 Luna\dots\ }\end{center}
\switchcolumn
\selectlanguage{english}
\begin{center}{\color{gregoriocolor} The 
 Twenty-First Day of 
 December. The\dots\ Day of the Moon.}\end{center}
\end{paracol}

\noindent\begin{tabularx}{\linewidth}{*{19}{>{\centering\arraybackslash}X}}
 \textcolor{gregoriocolor}{a} & \textcolor{gregoriocolor}{b} & \textcolor{gregoriocolor}{c} & \textcolor{gregoriocolor}{d} & \textcolor{gregoriocolor}{e} & \textcolor{gregoriocolor}{f} & \textcolor{gregoriocolor}{g} & \textcolor{gregoriocolor}{h} & \textcolor{gregoriocolor}{i} & \textcolor{gregoriocolor}{k} & \textcolor{gregoriocolor}{l} & \textcolor{gregoriocolor}{m} & \textcolor{gregoriocolor}{n} & \textcolor{gregoriocolor}{p} & \textcolor{gregoriocolor}{q} & \textcolor{gregoriocolor}{r} & \textcolor{gregoriocolor}{s} & \textcolor{gregoriocolor}{t} & \textcolor{gregoriocolor}{u} \\
 2 & 3 & 4 & 5 & 6 & 7 & 8 & 9 & 10 & 11 & 12 & 13 & 14 & 15 & 16 & 17 & 18 & 19 & 20 \\
\end{tabularx}
\vspace{0.5\baselineskip}
\noindent\begin{tabularx}{\linewidth}{*{12}{>{\centering\arraybackslash}X}}
 \textcolor{gregoriocolor}{A} & \textcolor{gregoriocolor}{B} & \textcolor{gregoriocolor}{C} & \textcolor{gregoriocolor}{D} & \textcolor{gregoriocolor}{E} & F & \textcolor{gregoriocolor}{F} & \textcolor{gregoriocolor}{G} & \textcolor{gregoriocolor}{H} & \textcolor{gregoriocolor}{M} & \textcolor{gregoriocolor}{N} & \textcolor{gregoriocolor}{P} \\
 21 & 22 & 23 & 24 & 25 & 26 & 25 & 26 & 27 & 28 & 29 & 1 \\
\end{tabularx}

\begin{paracol}{2}
\selectlanguage{latin}
\lettrine[lines=2]{C}{alamínæ} natális beáti Thomæ Apóstoli, qui 
 Parthis, Medis, Persis et Hyrcánis Evangélium prædicávit; ac demum in Indiam 
 pervénit, ibíque, cum eos pópulos in Christiána religióne instituísset, 
 Regis jussu lánceis transfíxus occúbuit. Ipsíus relíquiæ primo ad 
 urbem Edéssam, in Mesopotámia, deínde Ortónam, apud Frentános, translátæ 
 sunt.
\switchcolumn
\selectlanguage{english}
\lettrine[lines=2]{A}{t} Mylapore, the birthday of the blessed Apostle Thomas, who preached the 
 Gospel to the Parthians, Medes, Persians, and Hyrcanians. Having 
 finally penetrated into India, and instructed those nations in the Christian 
 religion, he died pierced with lances at the order of the king. His 
 remains were first taken to the city of Edessa in Mesopotamia, and then to 
 Ortona.
\switchcolumn*
\selectlanguage{latin}
Fribúrgi Helvetiórum item natális sancti Petri Canísii, Sacerdótis e 
 Societáte Jesu et Confessóris, doctrína et sanctitáte præclári; 
 qui, difficíllimis Germániæ tempóribus, fidem cathólicam strénue deféndit ac 
 propagávit. Eum vero Pius Undécimus, Póntifex Máximus, Sanctórum 
 catálogo adscrípsit, simúlque Doctórem universális Ecclésiæ declarávit, et 
 ipsíus festum quinto Kaléndas Maji agéndum esse decrévit.
\switchcolumn
\selectlanguage{english}
At Fribourg in Switzerland, the birthday also of St. Peter Canisius, priest 
 of the Society of Jesus, a confessor famed for his sanctity and learning. 
 He defended and spread the Catholic faith with the utmost zeal in Germany 
 during its most difficult times. Pope Pius XI added him to the list of 
 the saints, and at the same time declared him to be a doctor of the 
 universal Church, appointing his feast to be observed on the 27th of April.
\switchcolumn*
\selectlanguage{latin}
Antiochíæ sancti Anastásii, Epíscopi et 
 Mártyris; qui, Phocæ Imperatóris témpore, a Judæis, in seditióne ab ipsis 
 contra Christiános facta, sævíssime necátus est.
\switchcolumn
\selectlanguage{english}
At Antioch, St. Anastasius, bishop and martyr. During the reign of 
 Emperor Phocas he was cruelly murdered by Jews in a riot which they had 
 instigated against the Christians.
\switchcolumn*
\selectlanguage{latin}
Nicomedíæ sancti Glycérii Presbyteri, qui, in 
 persecutióne Diocletiáni, multis torméntis vexátus, demum, in ignem 
 conjéctus, martyrium consummávit.
\switchcolumn
\selectlanguage{english}
At Nicomedia, St. Glycerius, priest. During the persecution of 
 Diocletian he was subjected to many torments, and finally fulfilled his 
 martyrdom by being cast into the flames.
\switchcolumn*
\selectlanguage{latin}
In Túscia sanctórum Mártyrum Joánnis et Festi.
\switchcolumn
\selectlanguage{english}
In Tuscany, the holy martyrs John and Festus.
\switchcolumn*
\selectlanguage{latin}
In Lycia sancti Themístoclis Mártyris, qui, sub 
 Décio Imperatóre, pro sancto Dióscoro, qui quærebátur ad necem, se óbtulit, 
 et, equúleo tortus, raptátus ac fústibus cæsus, martyrii corónam adéptus 
 est.
\switchcolumn
\selectlanguage{english}
In Lycia, St. Themistocles, martyr. In the reign of Emperor Decius, he 
 offered himself to take the place of Dioscorus, whom they were seeking to 
 slay. He was tortured on the rack, dragged over rough ways and 
 scourged, and thus obtained the crown of martyrdom.
\switchcolumn*
\selectlanguage{latin}
Tréviris sancti Severíni, Epíscopi et 
 Confessóris.
\switchcolumn
\selectlanguage{english}
At Treves, St. Severinus, bishop and confessor.
\switchcolumn*
\selectlanguage{latin}
\end{paracol}


% ---- martyrology/mart12/mart1222.htm
\needspace{10\baselineskip}
\begin{paracol}{2}
\selectlanguage{latin}
\begin{center}{\color{gregoriocolor} Undécimo Kaléndas Januárii. 
 Luna\dots\ }\end{center}
\switchcolumn
\selectlanguage{english}
\begin{center}{\color{gregoriocolor} The 
 Twenty-Second Day of 
 December. The\dots\ Day of the Moon.}\end{center}
\end{paracol}

\noindent\begin{tabularx}{\linewidth}{*{19}{>{\centering\arraybackslash}X}}
 \textcolor{gregoriocolor}{a} & \textcolor{gregoriocolor}{b} & \textcolor{gregoriocolor}{c} & \textcolor{gregoriocolor}{d} & \textcolor{gregoriocolor}{e} & \textcolor{gregoriocolor}{f} & \textcolor{gregoriocolor}{g} & \textcolor{gregoriocolor}{h} & \textcolor{gregoriocolor}{i} & \textcolor{gregoriocolor}{k} & \textcolor{gregoriocolor}{l} & \textcolor{gregoriocolor}{m} & \textcolor{gregoriocolor}{n} & \textcolor{gregoriocolor}{p} & \textcolor{gregoriocolor}{q} & \textcolor{gregoriocolor}{r} & \textcolor{gregoriocolor}{s} & \textcolor{gregoriocolor}{t} & \textcolor{gregoriocolor}{u} \\
 3 & 4 & 5 & 6 & 7 & 8 & 9 & 10 & 11 & 12 & 13 & 14 & 15 & 16 & 17 & 18 & 19 & 20 & 21 \\
\end{tabularx}
\vspace{0.5\baselineskip}
\noindent\begin{tabularx}{\linewidth}{*{12}{>{\centering\arraybackslash}X}}
 \textcolor{gregoriocolor}{A} & \textcolor{gregoriocolor}{B} & \textcolor{gregoriocolor}{C} & \textcolor{gregoriocolor}{D} & \textcolor{gregoriocolor}{E} & F & \textcolor{gregoriocolor}{F} & \textcolor{gregoriocolor}{G} & \textcolor{gregoriocolor}{H} & \textcolor{gregoriocolor}{M} & \textcolor{gregoriocolor}{N} & \textcolor{gregoriocolor}{P} \\
 22 & 23 & 24 & 25 & 26 & 27 & 26 & 27 & 28 & 29 & 1 & 2 \\
\end{tabularx}

\begin{paracol}{2}
\selectlanguage{latin}
\lettrine[lines=2]{R}{omæ,} via Lavicána, inter duas Lauros, natális 
 sanctórum trigínta Mártyrum, qui omnes una die, in persecutióne Diocletiáni, 
 martyrio coronáti sunt.
\switchcolumn
\selectlanguage{english}
\lettrine[lines=2]{A}{t} Rome, on the Lavican Way, between the two laurels, the birthday of thirty 
 holy martyrs who were all crowned with martyrdom on the one day in the 
 persecution of Diocletian.
\switchcolumn*
\selectlanguage{latin}
Item Romæ sancti Flaviáni Expræfécti, viri 
 beátæ Mártyris Dafrósæ atque patris beatárum Vírginum et Mártyrum Bibiánæ ac 
 Demétriæ; qui, sub Juliáno Apóstata, pro Christo inscriptióne damnátus, et 
 ad Aquas Taurínas, in Etrúria, in exsílium missus, illic in oratióne 
 spíritum Deo réddidit.
\switchcolumn
\selectlanguage{english}
In the same city, St. Flavian, an ex-prefect, the husband of the blessed 
 martyr Dafrosa, and the father of the holy virgin martyrs, Bibiana and 
 Demetria. He was condemned under Julian the Apostate to be branded for 
 Christ, and was exiled to Aquæ Taurinæ, where he gave up his soul to God 
 in prayer.
\switchcolumn*
\selectlanguage{latin}
In Ægypto sanctórum Chærémonis, Epíscopi 
 Nilópolis, et aliórum plurimórum Mártyrum. Horum álii, sæviénte Décii 
 persecutióne, fuga dispérsi, in solitúdinis errántes, a béstiis interémpti 
 sunt; álii fame, frígore ac languóre consúmpti; álii a bárbaris et 
 latrónibus necáti; atque ita omnes, divérso mortis génere, eádem martyrii 
 glória coronáti sunt.
\switchcolumn
\selectlanguage{english}
In Egypt, St. Chaeremon, bishop of Nilopolis, and many other martyrs. 
 While the persecution of Decius was raging, some of them were dispersed in 
 flight, and wandering through deserts were killed by wild beasts; others 
 perished by famine, cold, and sickness; others again were murdered by 
 barbarians and robbers, and thus all were crowned with a glorious martyrdom.
\switchcolumn*
\selectlanguage{latin}
Apud Ostia Tiberína sanctórum Mártyrum Demétrii, Honoráti et Flori.
\switchcolumn
\selectlanguage{english}
At Ostia, the holy martyrs Demetrius, Honoratus, and Florus.
\switchcolumn*
\selectlanguage{latin}
Alexandríæ sancti Ischyriónis Mártyris, qui, 
 cum ad sacrificándum convíciis et injúriis cogerétur atque contémneret, ídeo, 
 præacúta sude per média víscera transverberátus, neci tráditur.
\switchcolumn
\selectlanguage{english}
At Alexandria, St. Ischyrion, martyr. Because he despised all the 
 injuries he was made to suffer in attempts to force him to sacrifice to 
 idols, his bowels were pierced with a sharp stake, bringing his death.
\switchcolumn*
\selectlanguage{latin}
Nicomedíæ sancti Zenónis mílitis, qui, cum 
 Diocletiánum Céreri immolántem derisísset, proptérea, maxíllis confráctis 
 dentibúsque excússis, cápite truncátus est.
\switchcolumn
\selectlanguage{english}
At Nicomedia, St. Zeno, a soldier who mocked Diocletian for sacrificing to 
 Ceres, wherefore his jawbones were broken, his teeth knocked out, and his 
 head struck off.
\switchcolumn*
\selectlanguage{latin}
Chicágiæ 
 sanctæ Francíscæ Xavériæ Cabríni, Vírginis, Institúti Missionariárum a 
 Sacratíssimo Corde Jesu Fundatrícis, exímia caritáte, invícta ánimi 
 fortitúdine et humilitáte insígnis, quam Pius Papa Duodécimus, Sanctárum 
 catálogo adscrípsit, et ómnium emigrántium cæléstem apud Deum Patrónam 
 constítuit.
\switchcolumn
\selectlanguage{english}
At Chicago, St. Frances Xavier Cabrini, virgin, foundress of the 
 Congregation of Missionaries of the Sacred Heart of Jesus, distinguished for 
 charity, humility, and invincible fortitude. Pope Pius XII added her to the 
 catalogue of saints, and named her as the heavenly patroness of all 
 emigrants.
\switchcolumn*
\selectlanguage{latin}
\end{paracol}


% ---- martyrology/mart12/mart1223.htm
\needspace{10\baselineskip}
\begin{paracol}{2}
\selectlanguage{latin}
\begin{center}{\color{gregoriocolor} Décimo Kaléndas Januárii. 
 Luna\dots\ }\end{center}
\switchcolumn
\selectlanguage{english}
\begin{center}{\color{gregoriocolor} The 
 Twenty-Third Day of 
 December. The\dots\ Day of the Moon.}\end{center}
\end{paracol}

\noindent\begin{tabularx}{\linewidth}{*{19}{>{\centering\arraybackslash}X}}
 \textcolor{gregoriocolor}{a} & \textcolor{gregoriocolor}{b} & \textcolor{gregoriocolor}{c} & \textcolor{gregoriocolor}{d} & \textcolor{gregoriocolor}{e} & \textcolor{gregoriocolor}{f} & \textcolor{gregoriocolor}{g} & \textcolor{gregoriocolor}{h} & \textcolor{gregoriocolor}{i} & \textcolor{gregoriocolor}{k} & \textcolor{gregoriocolor}{l} & \textcolor{gregoriocolor}{m} & \textcolor{gregoriocolor}{n} & \textcolor{gregoriocolor}{p} & \textcolor{gregoriocolor}{q} & \textcolor{gregoriocolor}{r} & \textcolor{gregoriocolor}{s} & \textcolor{gregoriocolor}{t} & \textcolor{gregoriocolor}{u} \\
 4 & 5 & 6 & 7 & 8 & 9 & 10 & 11 & 12 & 13 & 14 & 15 & 16 & 17 & 18 & 19 & 20 & 21 & 22 \\
\end{tabularx}
\vspace{0.5\baselineskip}
\noindent\begin{tabularx}{\linewidth}{*{12}{>{\centering\arraybackslash}X}}
 \textcolor{gregoriocolor}{A} & \textcolor{gregoriocolor}{B} & \textcolor{gregoriocolor}{C} & \textcolor{gregoriocolor}{D} & \textcolor{gregoriocolor}{E} & F & \textcolor{gregoriocolor}{F} & \textcolor{gregoriocolor}{G} & \textcolor{gregoriocolor}{H} & \textcolor{gregoriocolor}{M} & \textcolor{gregoriocolor}{N} & \textcolor{gregoriocolor}{P} \\
 23 & 24 & 25 & 26 & 27 & 28 & 27 & 28 & 29 & 1 & 2 & 3 \\
\end{tabularx}

\begin{paracol}{2}
\selectlanguage{latin}
\lettrine[lines=2]{R}{omæ} sanctæ Victóriæ, Vírginis et Mártyris, quæ, 
 in persecutióne Décii Imperatóris, cum esset desponsáta Eugénio pagáno et 
 nec núbere vellet neque sacrificáre, ídeo, post multa facta mirácula, quibus 
 plúrimas Deo Vírgines aggregáverat, a carnífice percússa est gládio in corde, 
 rogátu sui sponsi.
\switchcolumn
\selectlanguage{english}
\lettrine[lines=2]{A}{t} Rome, St. Victoria, virgin and martyr, during the persecution of Emperor 
 Decius. She had been promised in marriage to a pagan named Eugene, but 
 because she had refused to marry him and to offer sacrifice to idols, and 
 because by working many miracles she had brought many virgins to the service 
 of God, the executioner thrust a sword into her heart at the request of her 
 spouse.
\switchcolumn*
\selectlanguage{latin}
Nicomedíæ pássio sanctórum Migdónii et Mardónii, 
 quorum alter, in Diocletiáni persecutióne, igne cremátus, alter in fossam 
 projéctus occúbuit. Tunc étiam Diáconus sancti Anthimi, Epíscopi 
 Nicomediénsis, passus est; qui, cum lítteras perférret ad Mártyres, a 
 Gentílibus tentus est, atque, lapídibus óbrutus, migrávit ad Dóminum.
\switchcolumn
\selectlanguage{english}
At Nicomedia, the passion of Saints Migdonius and Mardonius, one of whom was 
 burned alive in the same persecution of Diocletian, and the other died in a 
 pit where he had been thrown. A deacon of St. Anthimus, bishop of 
 Nicomedia, suffered at the same time. He had been arrested by the 
 heathen when he was carrying letters to the martyrs, and being overwhelmed 
 with stones, went to our Lord.
\switchcolumn*
\selectlanguage{latin}
Ibídem natális sanctórum vigínti Mártyrum, quos 
 ípsamet Diocletiána persecútio, gravíssimis torméntis cruciátos, 
 Mártyres Christi fecit.
\switchcolumn
\selectlanguage{english}
Likewise, the birthday of twenty holy martyrs, whom the persecution of 
 Diocletian made martyrs for the faith of Christ, after subjecting them to 
 the most painful torments.
\switchcolumn*
\selectlanguage{latin}
In Creta sanctórum Mártyrum Theodúli, Saturníni, 
 Eupori, Gelásii, Euniciáni, Zétici, Leóminis, Agathópodis, Basílidis et 
 Evarísti; qui, in Décii persecutióne, crudélia passi sunt et cápite cæsi.
\switchcolumn
\selectlanguage{english}
In Crete, the holy martyrs Theodulus, Saturninus, Euporus, Gelasius, 
 Eunicianus, Zeticus, Leomines, Agathopodes, Basilides, and Everistus, who 
 were beheaded after suffering cruel torments in the persecution of Decius.
\switchcolumn*
\selectlanguage{latin}
Romæ beáti Sérvuli, qui (ut sanctus Gregórius 
 Papa scribit), a primæva sui ætáte usque ad finem vitæ, paralyticus jácuit 
 in pórticu prope Ecclésiam sancti Cleméntis, et demum, Angelórum cántibus 
 invitátus, ad paradísi glóriam transívit; ad cujus túmulum Deus mirácula 
 crebérrime osténdit.
\switchcolumn
\selectlanguage{english}
At Rome, blessed Servulus of whom St. Gregory writes that from his early 
 years to the end of his life he was a paralytic and had remained lying in a 
 porch near St. Clement's Church, and being invited by the chant of angels, 
 he went to enjoy the glory of Paradise. At his tomb frequent miracles 
 are wrought by God.
\switchcolumn*
\selectlanguage{latin}
\end{paracol}


% ---- martyrology/mart12/mart1224.htm
\needspace{10\baselineskip}
\begin{paracol}{2}
\selectlanguage{latin}
\begin{center}{\color{gregoriocolor} Nono Kaléndas Januárii. 
 Luna\dots\ }\end{center}
\switchcolumn
\selectlanguage{english}
\begin{center}{\color{gregoriocolor} The 
 Twenty-Fourth Day of 
 December. The\dots\ Day of the Moon.}\end{center}
\end{paracol}

\noindent\begin{tabularx}{\linewidth}{*{19}{>{\centering\arraybackslash}X}}
 \textcolor{gregoriocolor}{a} & \textcolor{gregoriocolor}{b} & \textcolor{gregoriocolor}{c} & \textcolor{gregoriocolor}{d} & \textcolor{gregoriocolor}{e} & \textcolor{gregoriocolor}{f} & \textcolor{gregoriocolor}{g} & \textcolor{gregoriocolor}{h} & \textcolor{gregoriocolor}{i} & \textcolor{gregoriocolor}{k} & \textcolor{gregoriocolor}{l} & \textcolor{gregoriocolor}{m} & \textcolor{gregoriocolor}{n} & \textcolor{gregoriocolor}{p} & \textcolor{gregoriocolor}{q} & \textcolor{gregoriocolor}{r} & \textcolor{gregoriocolor}{s} & \textcolor{gregoriocolor}{t} & \textcolor{gregoriocolor}{u} \\
 5 & 6 & 7 & 8 & 9 & 10 & 11 & 12 & 13 & 14 & 15 & 16 & 17 & 18 & 19 & 20 & 21 & 22 & 23 \\
\end{tabularx}
\vspace{0.5\baselineskip}
\noindent\begin{tabularx}{\linewidth}{*{12}{>{\centering\arraybackslash}X}}
 \textcolor{gregoriocolor}{A} & \textcolor{gregoriocolor}{B} & \textcolor{gregoriocolor}{C} & \textcolor{gregoriocolor}{D} & \textcolor{gregoriocolor}{E} & F & \textcolor{gregoriocolor}{F} & \textcolor{gregoriocolor}{G} & \textcolor{gregoriocolor}{H} & \textcolor{gregoriocolor}{M} & \textcolor{gregoriocolor}{N} & \textcolor{gregoriocolor}{P} \\
 24 & 25 & 26 & 27 & 28 & 29 & 28 & 29 & 1 & 2 & 3 & 4 \\
\end{tabularx}

\begin{paracol}{2}
\selectlanguage{latin}
\lettrine[lines=1]{V}{igília} Nativitátis Dómini nostri Jesu Christi.
\switchcolumn
\selectlanguage{english}
\lettrine[lines=1]{T}{he} Vigil of the Nativity of our Lord Jesus Christ.
\switchcolumn*
\selectlanguage{latin}
Cracóviæ, in Polónia, natális sancti Joánnis 
 Cántii, Presbyteri et Confessóris, quem, doctrína, propagándæ fídei zelo, 
 virtútibus ac miráculis clarum, Clemens Décimus tértius, Póntifex Máximus, 
 Sanctórum número adscrípsit. Ejus autem festívitas tertiodécimo 
 Kaléndas Novémbris celebrátur.
\switchcolumn
\selectlanguage{english}
At Cracow in Poland, the birthday of St. John Cantius, priest and confessor, 
 celebrated for his learning, for his zeal in propagating the faith, and for 
 his virtues and miracles, for which Pope Clement XIII added him to the 
 number of the saints. His feast is observed on the 20th of October.
\switchcolumn*
\selectlanguage{latin}
Apud Spolétum sancti Gregórii, Presbyteri et Mártyris; qui, tempóribus 
 Diocletiáni et Maximiáni Imperatórum, primo nodósis fústibus cæsus, 
 ac deínde, post cratículam et cárcerem, cárduis férreis in génibus percússus, 
 sed et ardéntibus lampádibus per látera incénsus, tandem est decollátus.
\switchcolumn
\selectlanguage{english}
At Spoleto, St. Gregory, priest and martyr. In the time of Emperors 
 Diocletian and Maximian, he was first beaten with rough clubs, exposed on 
 the gridiron and imprisoned, struck on the knees with iron carding 
 instruments, burned on the sides with firebrands, and finally beheaded.
\switchcolumn*
\selectlanguage{latin}
Trípoli, in Phœnícia, sanctórum Mártyrum 
 Luciáni, Metróbii, Pauli, Zenóbii, Theotími et Drusi.
\switchcolumn
\selectlanguage{english}
At Tripoli in Phoenicia, the holy martyrs Leucian, Metrobius, Paul, Zenobius, 
 Theotimus, and Drusus.
\switchcolumn*
\selectlanguage{latin}
Nicomedíæ sancti Euthymii Mártyris, qui, in 
 persecutióne Diocletiáni, cum multos ad martyrium præmisísset, et ipse, ense 
 tranfíxus, eos secútus est ad corónam.
\switchcolumn
\selectlanguage{english}
At Nicomedia, during the persecution of Diocletian, St. Euthymius, martyr, 
 who sent many before him to martyrdom, and being pierced with a sword, 
 followed them to share their crown.
\switchcolumn*
\selectlanguage{latin}
Antiochíæ natális sanctárum Vírginum 
 quadragínta, quæ, in Deciána persecutióne, per divérsa torménta martyrium 
 consummárunt.
\switchcolumn
\selectlanguage{english}
At Antioch, the birthday of forty holy virgins who suffered martyrdom by 
 divers torments in the Decian persecution.
\switchcolumn*
\selectlanguage{latin}
Burdígalæ sancti Delphíni Epíscopi, qui, 
 Theodósii témpore, cláruit sanctitáte.
\switchcolumn
\selectlanguage{english}
At Bordeaux, St. Delphinus, bishop, who was renowned for holiness in the 
 time of Theodosius.
\switchcolumn*
\selectlanguage{latin}
Romæ natális sanctæ Tharsíllæ Vírginis, ámitæ 
 sancti Gregórii Papæ, de qua ipse testátur quod in hora éxitus sui Jesum ad 
 se veniéntem víderit.
\switchcolumn
\selectlanguage{english}
At Rome, the birthday of the holy virgin Tharsilla, aunt of Pope St. 
 Gregory, who writes of her that at the hour of her death she saw Jesus 
 coming to her.
\switchcolumn*
\selectlanguage{latin}
Tréviris sanctæ Irmínæ Vírginis, fíliæ 
 Dagobérti Regis.
\switchcolumn
\selectlanguage{english}
At Treves, St. Irmina, virgin, daughter of King Dagobert.
\switchcolumn*
\selectlanguage{latin}
\end{paracol}


% ---- martyrology/mart12/mart1225.htm
\needspace{10\baselineskip}
\begin{paracol}{2}
\selectlanguage{latin}
\begin{center}{\color{gregoriocolor} Octávo Kaléndas Januárii. 
 Luna\dots\ }\end{center}
\switchcolumn
\selectlanguage{english}
\begin{center}{\color{gregoriocolor} The 
 Twenty-Fifth Day of 
 December. The\dots\ Day of the Moon.}\end{center}
\end{paracol}

\noindent\begin{tabularx}{\linewidth}{*{19}{>{\centering\arraybackslash}X}}
 \textcolor{gregoriocolor}{a} & \textcolor{gregoriocolor}{b} & \textcolor{gregoriocolor}{c} & \textcolor{gregoriocolor}{d} & \textcolor{gregoriocolor}{e} & \textcolor{gregoriocolor}{f} & \textcolor{gregoriocolor}{g} & \textcolor{gregoriocolor}{h} & \textcolor{gregoriocolor}{i} & \textcolor{gregoriocolor}{k} & \textcolor{gregoriocolor}{l} & \textcolor{gregoriocolor}{m} & \textcolor{gregoriocolor}{n} & \textcolor{gregoriocolor}{p} & \textcolor{gregoriocolor}{q} & \textcolor{gregoriocolor}{r} & \textcolor{gregoriocolor}{s} & \textcolor{gregoriocolor}{t} & \textcolor{gregoriocolor}{u} \\
 6 & 7 & 8 & 9 & 10 & 11 & 12 & 13 & 14 & 15 & 16 & 17 & 18 & 19 & 20 & 21 & 22 & 23 & 24 \\
\end{tabularx}
\vspace{0.5\baselineskip}
\noindent\begin{tabularx}{\linewidth}{*{12}{>{\centering\arraybackslash}X}}
 \textcolor{gregoriocolor}{A} & \textcolor{gregoriocolor}{B} & \textcolor{gregoriocolor}{C} & \textcolor{gregoriocolor}{D} & \textcolor{gregoriocolor}{E} & F & \textcolor{gregoriocolor}{F} & \textcolor{gregoriocolor}{G} & \textcolor{gregoriocolor}{H} & \textcolor{gregoriocolor}{M} & \textcolor{gregoriocolor}{N} & \textcolor{gregoriocolor}{P} \\
 25 & 26 & 27 & 28 & 29 & 30 & 29 & 1 & 2 & 3 & 4 & 5 \\
\end{tabularx}

\begin{paracol}{2}
\selectlanguage{latin}
\lettrine[lines=2]{A}{} nno a creatióne mundi, quando 
 in princípio Deus creávit cœlum et terram, quínquies millésimo centésimo 
 nonagésimo nono: A dilúvio autem, anno bis millésimo nongentésimo 
 quinquagésimo séptimo: A nativitáte Abrahæ, anno bis millésimo quintodécimo: 
 A Moyse et egréssu pópuli Israël de Ægypto, anno millésimo quingentésimo 
 décimo: Ab unctióne David in Regem, anno millésimo trigésimo secúndo; 
 Hebdómada sexagésima quinta, juxta Daniélis prophetíam: Olympíade centésima 
 nonagésima quarta: Ab urbe Roma cóndita, anno septingentésimo quinquagésimo 
 secúndo: Anno Impérii Octaviáni Augústi quadragésimo secúndo, toto Orbe in 
 pace compósito, sexta mundi ætáte, Jesus Christus ætérnus Deus, æterníque 
 Patris Fílius, mundum volens advéntu suo piíssimo consecráre, de Spíritu 
 Sancto concéptus, novémque post conceptiónem decúrsis ménsibus
	\rubric{(Hic vox elevatur, et omnes genua flectunt)}, in Béthlehem Judæ náscitur ex María Vírgine factus Homo.
\switchcolumn
\selectlanguage{english}
\lettrine[lines=2]{I}{} n the 5199th year 
 of the creation of the world, from the time when in the beginning God 
 created heaven and earth; from the flood, the 2957th year; from 
 the birth of Abraham, the 2015th year; from Moses and the 
 going-out of the people of Israel from Egypt, the 1510th year; from 
 the anointing of David as king, the 1032nd year; in the 65th 
 week according to the prophecy of Daniel; in the 194th Olympiad; 
 from the founding of the city of Rome, the 752nd year; in the 42nd 
 year of the rule of Octavian Augustus, when the whole world was at peace, in 
 the sixth age of the world: Jesus Christ, the eternal God and Son of the 
 eternal Father, desiring to sanctify the world by His most merciful coming, 
 having been conceived by the Holy Ghost, and nine months having passed since 
	His conception \rubric{(A higher tone of voice is now used, and all kneel)} was born in 
 Bethlehem of Juda of the Virgin Mary, having become man.
\switchcolumn*
\selectlanguage{latin}
	\rubric{Hic autem in priori voce dicitur, et in tono passionis:} Natívitas Dómini nostri Jesu Christi secúndum carnem. \rubric{Quod sequitur, legitur in tono Lectionis consueto; et surgunt omnes.}
\switchcolumn
\selectlanguage{english}
	\rubric{In the same higher tone of voice and in the tone of the Passion:} THE NATIVITY of our Lord Jesus Christ according to the flesh. \rubric{That which follows is said in the customary tone of the Martyrology, and all arise.}
\switchcolumn*
\selectlanguage{latin}
Eódem die natális sanctæ Anastásiæ, quæ, 
 témpore Diocletiáni, primo diram et immítem custódiam a viro suo Públio 
 perpéssa est, in qua tamen a Confessóre Christi Chrysógono multum consoláta 
 et confortáta fuit; deínde a Floro, Præfécto Illyrici, per diútinam 
 custódiam maceráta, ad últimum, mánibus et pédibus exténsis, ligáta est ad 
 palos, et circa eam ignis accénsus, in quo martyrium consummávit in ínsula 
 Palmária, ad quam una cum ducéntis viris et septuagínta féminis deportáta 
 fúerat, qui váriis interfectiónibus martyrium celebrárunt.
\switchcolumn
\selectlanguage{english}
The same day, the birthday of St. Anastasia, who, in the time of Diocletian, 
 first suffered a severe and harsh imprisonment on the part of her husband 
 Publius, in which, however, she was much consoled and encouraged by the 
 confessor of Christ, Chrysogonus. Afterwards she was thrown into 
 prison again by order of Florus, prefect of Illyria; and finally, having her 
 hands and feet stretched, she was tied to stakes with a fire kindled about 
 her, in the midst of which she ended her martyrdom on the island of Palmaria, 
 whither she had been brought with two hundred men and seventy women, who 
 have made martyrdom a glorious thing by the various kinds of death they so 
 valiantly endured.
\switchcolumn*
\selectlanguage{latin}
Barcinóne, in Hispánia, item natális sancti Petri Nolásci Confessóris, qui 
 Fundátor éxstitit Ordinis beátæ Maríæ de 
 Mercéde redemptiónis captivórum, ac virtúte et miráculis cláruit. 
 Ipsíus autem festum cólitur quinto Kaléndas Februárii.
\switchcolumn
\selectlanguage{english}
At Barcelona in Spain, St. Peter Nolasco, confessor and founder of the Order 
 of our Lady of Ransom for the Redemption of Captives, renowned for virtue 
 and miracles. His feast is celebrated on the 28th of January.
\switchcolumn*
\selectlanguage{latin}
Romæ, in cœmetério Aproniáni, sanctæ Eugéniæ 
 Vírginis, beáti Mártyris Philíppi fíliæ, quæ, témpore Galliéni Imperatóris, 
 post plúrima virtútum insígnia, post sacros Vírginum choros Christo 
 aggregátos, sub Præfécto Urbis Nicétio diu agonizávit, ac novíssime gládio 
 juguláta est.
\switchcolumn
\selectlanguage{english}
At Rome, in the cemetery of Apronian, St. Eugenia, virgin, the daughter of 
 blessed Philip, martyr. In the time of Emperor Gallienus, after 
 displaying many signs and virtues, gathering to Christ holy choirs of 
 virgins, and after long trials under Nicetius, prefect of the city, she was 
 finally put to the sword.
\switchcolumn*
\selectlanguage{latin}
Nicomediæ pássio multórum míllium Mártyrum, qui 
 cum in Christi Natáli ad Domínicum conveníssent, Diocletiánus Imperátor 
 jánuas Ecclésiæ claudi jussit, et ignem circumcírca parári, tripodémque cum 
 thure præ fóribus poni, ac præcónem magna voce clamáre ut qui incéndium 
 vellent effúgere, foras exírent et Jovi thus adolérent; cumque omnes una 
 voce respondíssent se pro Christo libéntius mori, incénso igne consúmpti 
 sunt, atque ita eo die nasci meruérunt in cælis, quo Christus in terris pro 
 salúte mundi olim nasci dignátus est.
\switchcolumn
\selectlanguage{english}
At Nicomedia, many thousand martyrs, who had assembled for divine service on 
 our Lord's Nativity. When Emperor Diocletian ordered the doors of the 
 church to be closed, fire to kindled here and there, a vessel with incense 
 to be put before the entrance, and a man to cry out that those who wished to 
 escape from the fire should come out and burn incense to Jupiter, all with 
 one voice answered that they preferred to die for Christ. They were 
 consumed in the fire, and thus merited to be born in heaven on the day on 
 which Christ vouchsafed to be born on earth for the salvation of the world.
\switchcolumn*
\selectlanguage{latin}
\end{paracol}


% ---- martyrology/mart12/mart1226.htm
\needspace{10\baselineskip}
\begin{paracol}{2}
\selectlanguage{latin}
\begin{center}{\color{gregoriocolor} Séptimo Kaléndas Januárii. 
 Luna\dots\ }\end{center}
\switchcolumn
\selectlanguage{english}
\begin{center}{\color{gregoriocolor} The 
 Twenty-Sixth Day of 
 December. The\dots\ Day of the Moon.}\end{center}
\end{paracol}

\noindent\begin{tabularx}{\linewidth}{*{19}{>{\centering\arraybackslash}X}}
 \textcolor{gregoriocolor}{a} & \textcolor{gregoriocolor}{b} & \textcolor{gregoriocolor}{c} & \textcolor{gregoriocolor}{d} & \textcolor{gregoriocolor}{e} & \textcolor{gregoriocolor}{f} & \textcolor{gregoriocolor}{g} & \textcolor{gregoriocolor}{h} & \textcolor{gregoriocolor}{i} & \textcolor{gregoriocolor}{k} & \textcolor{gregoriocolor}{l} & \textcolor{gregoriocolor}{m} & \textcolor{gregoriocolor}{n} & \textcolor{gregoriocolor}{p} & \textcolor{gregoriocolor}{q} & \textcolor{gregoriocolor}{r} & \textcolor{gregoriocolor}{s} & \textcolor{gregoriocolor}{t} & \textcolor{gregoriocolor}{u} \\
 7 & 8 & 9 & 10 & 11 & 12 & 13 & 14 & 15 & 16 & 17 & 18 & 19 & 20 & 21 & 22 & 23 & 24 & 25 \\
\end{tabularx}
\vspace{0.5\baselineskip}
\noindent\begin{tabularx}{\linewidth}{*{12}{>{\centering\arraybackslash}X}}
 \textcolor{gregoriocolor}{A} & \textcolor{gregoriocolor}{B} & \textcolor{gregoriocolor}{C} & \textcolor{gregoriocolor}{D} & \textcolor{gregoriocolor}{E} & F & \textcolor{gregoriocolor}{F} & \textcolor{gregoriocolor}{G} & \textcolor{gregoriocolor}{H} & \textcolor{gregoriocolor}{M} & \textcolor{gregoriocolor}{N} & \textcolor{gregoriocolor}{P} \\
 26 & 27 & 28 & 29 & 30 & 1 & 1 & 2 & 3 & 4 & 5 & 6 \\
\end{tabularx}

\begin{paracol}{2}
\selectlanguage{latin}
\lettrine[lines=2]{H}{ierosólymis} natális sancti Stéphani Protomártyris, qui a Judæis, 
 non longe post Ascensiónem Dómini, lapidátus est.
\switchcolumn
\selectlanguage{english}
\lettrine[lines=2]{A}{t} Jerusalem, the birthday of St. Stephen, the first martyr, who was stoned 
 to death by the Jews shortly after the Ascension of our Lord.
\switchcolumn*
\selectlanguage{latin}
Romæ sancti Maríni, ex órdine Senatório viri, 
 qui, sub Numeriáno Imperatóre et Marciáno Præfécto, Christiánæ religiónis 
 causa comprehénsus, equúleo et úngulis servíli more punítus, in sartáginem 
 deínde conjéctus, sed, igne in rorem convérso, liberátus, objéctus quoque 
 feris et ab illis nullátenus læsus; tandem, ad aram íterum ductus, et, cum 
 idóla oratióne ejus corruíssent, percússus gládio, martyrii triúmphum 
 adéptus est.
\switchcolumn
\selectlanguage{english}
At Rome, St. Marinus, a man of senatorial rank. In the time of Emperor 
 Numerian and the prefect Marcian, he was arrested for the Christian 
 religion, racked and torn with iron claws like a slave, then thrown into a 
 boiling cauldron; but being delivered because the fire became like a dew, he 
 was exposed to the beasts without being injured by them, and finally being 
 led to the altar, the idols of which toppled over at his prayer, he was 
 struck with the sword, and thus obtained the triumph of martyrs.
\switchcolumn*
\selectlanguage{latin}
Ibídem, via Appia, deposítio sancti Dionysii Papæ, 
 qui, multis pro Ecclésia impénsis labóribus, fídei documéntis clarus 
 effúlsit.
\switchcolumn
\selectlanguage{english}
Likewise at Rome, on the Appian Way, the death of Pope St. Dionysius, who 
 sustained many labours for the Church, and was renowned for his doctrinal 
 writings.
\switchcolumn*
\selectlanguage{latin}
Item Romæ sancti Zósimi, Papæ et Confessóris.
\switchcolumn
\selectlanguage{english}
In the same city, St. Zosimus, pope and confessor.
\switchcolumn*
\selectlanguage{latin}
In Mesopotámia sancti Archelái Epíscopi, doctrína et sanctitáte célebris.
\switchcolumn
\selectlanguage{english}
In Mesopotamia, St. Archelaus, bishop, famous for learning and holiness.
\switchcolumn*
\selectlanguage{latin}
Majúmæ, in Palæstína, sancti Zenónis Epíscopi.
\switchcolumn
\selectlanguage{english}
At Majuma, in Palestine, St. Zeno, bishop.
\switchcolumn*
\selectlanguage{latin}
Romæ sancti Theodóri, qui Mansionárius Ecclésiæ 
 sancti Petri fuit, cujus et méminit beátus Gregórius Papa.
\switchcolumn
\selectlanguage{english}
At Rome, St. Theodore, sacristan of the church of St. Peter, who is 
 mentioned by blessed Pope Gregory.
\switchcolumn*
\selectlanguage{latin}
\end{paracol}


% ---- martyrology/mart12/mart1227.htm
\needspace{10\baselineskip}
\begin{paracol}{2}
\selectlanguage{latin}
\begin{center}{\color{gregoriocolor} Sexto Kaléndas Januárii. 
 Luna\dots\ }\end{center}
\switchcolumn
\selectlanguage{english}
\begin{center}{\color{gregoriocolor} The 
 Twenty-Seventh Day of 
 December. The\dots\ Day of the Moon.}\end{center}
\end{paracol}

\noindent\begin{tabularx}{\linewidth}{*{19}{>{\centering\arraybackslash}X}}
 \textcolor{gregoriocolor}{a} & \textcolor{gregoriocolor}{b} & \textcolor{gregoriocolor}{c} & \textcolor{gregoriocolor}{d} & \textcolor{gregoriocolor}{e} & \textcolor{gregoriocolor}{f} & \textcolor{gregoriocolor}{g} & \textcolor{gregoriocolor}{h} & \textcolor{gregoriocolor}{i} & \textcolor{gregoriocolor}{k} & \textcolor{gregoriocolor}{l} & \textcolor{gregoriocolor}{m} & \textcolor{gregoriocolor}{n} & \textcolor{gregoriocolor}{p} & \textcolor{gregoriocolor}{q} & \textcolor{gregoriocolor}{r} & \textcolor{gregoriocolor}{s} & \textcolor{gregoriocolor}{t} & \textcolor{gregoriocolor}{u} \\
 8 & 9 & 10 & 11 & 12 & 13 & 14 & 15 & 16 & 17 & 18 & 19 & 20 & 21 & 22 & 23 & 24 & 25 & 26 \\
\end{tabularx}
\vspace{0.5\baselineskip}
\noindent\begin{tabularx}{\linewidth}{*{12}{>{\centering\arraybackslash}X}}
 \textcolor{gregoriocolor}{A} & \textcolor{gregoriocolor}{B} & \textcolor{gregoriocolor}{C} & \textcolor{gregoriocolor}{D} & \textcolor{gregoriocolor}{E} & F & \textcolor{gregoriocolor}{F} & \textcolor{gregoriocolor}{G} & \textcolor{gregoriocolor}{H} & \textcolor{gregoriocolor}{M} & \textcolor{gregoriocolor}{N} & \textcolor{gregoriocolor}{P} \\
 27 & 28 & 29 & 30 & 1 & 2 & 2 & 3 & 4 & 5 & 6 & 7 \\
\end{tabularx}

\begin{paracol}{2}
\selectlanguage{latin}
\lettrine[lines=2]{A}{pud} Ephesum natális sancti Joánnis, Apóstoli et Evangelístæ, 
 qui, post Evangélii scriptiónem, post exsílii relegatiónem et Apocalypsim 
 divínam, usque ad Trajáni Príncipis témpora persevérans, totíus Asiæ 
 fundávit rexítque Ecclésias, ac tandem, conféctus sénio, sexagésimo octávo 
 post passiónem Dómini anno mórtuus est, et juxta eándem urbem sepúltus.
\switchcolumn
\selectlanguage{english}
\lettrine[lines=2]{A}{t} Ephesus, the birthday of St. John, apostle and evangelist. After 
 writing his gospel, and after enduring exile and writing the divine 
 Apocalypse, he lived until the time of Emperor Trajan and founded and 
 governed the churches of all Asia. Worn out with age, he died in the 
 sixty-eighth year after the passion of our Lord and was buried near Ephesus.
\switchcolumn*
\selectlanguage{latin}
Constantinópoli sanctórum Confessórum Theodóri et Theóphanis fratrum, qui, a 
 puerítia in Palæstinénsi sancti Sabbæ 
 monastério nutríti, cum póstea pro sanctárum Imáginum cultu advérsus Leónem 
 Arménum strénue decertárent, ejus jussu verbéribus affécti sunt et exsílio 
 relegáti. Sed, eódem Leóne mórtuo, rursus Theóphilo Imperatóri, qui 
 eádem impietáte detinebátur, constánter resisténtes, verbéribus íterum cæsi 
 et in exsílium pulsi sunt, ubi Theodórus in cárcere exspirávit. 
 Theóphanes vero, pace demum Ecclésiæ réddita, factus est Nicænæ civitátis 
 Epíscopus, et confessiónis glória præclárus quiévit in Dómino.
\switchcolumn
\selectlanguage{english}
At Constantinople, the holy confessors Theodore and Theophanes, brothers, 
 who were brought up from their childhood in the monastery of St. Sabas. 
 Afterwards, they strove zealously for the veneration of holy images against 
 Leo the Armenian, and at his command they were scourged and banished. 
 After his death they again firmly opposed Emperor Theophilus, who was imbued 
 with the same impiety, and were scourged a second time and driven into exile, where Theodore died in prison. Theophanes, after peace had at 
 length been restored to the Church, was made bishop of Nicaea, and there, 
 famous for his glorious witness of the faith, rested in the Lord.
\switchcolumn*
\selectlanguage{latin}
Alexandríæ sancti Máximi Epíscopi, qui satis 
 clarus et insígnis título confessiónis efféctus est.
\switchcolumn
\selectlanguage{english}
At Alexandria, St. Maximus, bishop, well known and renowned by reason of his 
 confession.
\switchcolumn*
\selectlanguage{latin}
Constantinópoli sanctæ Nicáretes Vírginis, quæ, 
 sub Arcádio Imperatóre, cláruit sanctitáte.
\switchcolumn
\selectlanguage{english}
At Constantinople, St. Niceras, virgin, who was renowned for sanctity in the 
 time of Emperor Arcadius.
\switchcolumn*
\selectlanguage{latin}
\end{paracol}


% ---- martyrology/mart12/mart1228.htm
\needspace{10\baselineskip}
\begin{paracol}{2}
\selectlanguage{latin}
\begin{center}{\color{gregoriocolor} Quinto Kaléndas Januárii. 
 Luna\dots\ }\end{center}
\switchcolumn
\selectlanguage{english}
\begin{center}{\color{gregoriocolor} The 
 Twenty-Eighth Day of 
 December. The\dots\ Day of the Moon.}\end{center}
\end{paracol}

\noindent\begin{tabularx}{\linewidth}{*{19}{>{\centering\arraybackslash}X}}
 \textcolor{gregoriocolor}{a} & \textcolor{gregoriocolor}{b} & \textcolor{gregoriocolor}{c} & \textcolor{gregoriocolor}{d} & \textcolor{gregoriocolor}{e} & \textcolor{gregoriocolor}{f} & \textcolor{gregoriocolor}{g} & \textcolor{gregoriocolor}{h} & \textcolor{gregoriocolor}{i} & \textcolor{gregoriocolor}{k} & \textcolor{gregoriocolor}{l} & \textcolor{gregoriocolor}{m} & \textcolor{gregoriocolor}{n} & \textcolor{gregoriocolor}{p} & \textcolor{gregoriocolor}{q} & \textcolor{gregoriocolor}{r} & \textcolor{gregoriocolor}{s} & \textcolor{gregoriocolor}{t} & \textcolor{gregoriocolor}{u} \\
 9 & 10 & 11 & 12 & 13 & 14 & 15 & 16 & 17 & 18 & 19 & 20 & 21 & 22 & 23 & 24 & 25 & 26 & 27 \\
\end{tabularx}
\vspace{0.5\baselineskip}
\noindent\begin{tabularx}{\linewidth}{*{12}{>{\centering\arraybackslash}X}}
 \textcolor{gregoriocolor}{A} & \textcolor{gregoriocolor}{B} & \textcolor{gregoriocolor}{C} & \textcolor{gregoriocolor}{D} & \textcolor{gregoriocolor}{E} & F & \textcolor{gregoriocolor}{F} & \textcolor{gregoriocolor}{G} & \textcolor{gregoriocolor}{H} & \textcolor{gregoriocolor}{M} & \textcolor{gregoriocolor}{N} & \textcolor{gregoriocolor}{P} \\
 28 & 29 & 30 & 1 & 2 & 3 & 3 & 4 & 5 & 6 & 7 & 8 \\
\end{tabularx}

\begin{paracol}{2}
\selectlanguage{latin}
\lettrine[lines=2]{I}{n} Béthlehem Judæ natális sanctórum Innocéntium 
 Mártyrum, qui pro Christo ab Heróde Rege interfécti sunt.
\switchcolumn
\selectlanguage{english}
\lettrine[lines=2]{I}{n} Bethlehem of Juda, the birthday of the Holy Innocents, who were slain for 
 Christ by Herod the king.
\switchcolumn*
\selectlanguage{latin}
Lugdúni, in Gállia, item natális sancti Francísci Salésii, Epíscopi 
 Gebennénsis et Confessóris; quem, doctrína et flagrantíssimo in converténdis 
 hæréticis zelo præclárum, Alexánder Papa Séptimus in Sanctórum númerum rétulit, et ipsíus festivitátem quarto 
 Kaléndas Februárii, quo die sacrum illíus corpus Lugdúno Annésium, in 
 Sabáudia, fuit translátum, agéndam esse constítuit. Eum Pius Nonus, 
 Póntifex Máximus, Doctórem universális Ecclésiæ declarávit; et Pius Papa 
 Undécimus ómnibus Scriptóribus cathólicis, qui diáriis aliísve scriptis in 
 vulgus edéndis Christiánam sapiéntiam illústrant ac próvehunt et tuéntur, 
 cæléstem Patrónum dedit seu confirmávit.
\switchcolumn
\selectlanguage{english}
At Lyons in France, the birthday also of St. Francis de Sales, bishop of 
 Geneva and confessor. Because of his burning zeal for the conversion 
 of heretics and his learning, Pope Alexander VII placed him among the number 
 of the saints, and his feast is observed on the 29th of January, on which 
 day his holy body was translated from Lyons to Annecy in Savoy. Pope 
 Pius IX decreed him a doctor of the universal Church, and Pope Pius XI 
 constituted him the heavenly patron of all Catholic writers who explain, 
 promote, or defend Christian doctrine by publishing journals or other 
 writings in the vernacular.
\switchcolumn*
\selectlanguage{latin}
Ancyræ, in Galátia, sanctórum Mártyrum Eutychii 
 Presbyteri, et Domitiáni Diáconi.
\switchcolumn
\selectlanguage{english}
At Ancyra in Galatia, the holy martyrs Eutychius, priest, and Domitian, 
 deacon.
\switchcolumn*
\selectlanguage{latin}
In Africa natális sanctórum Mártyrum Cástoris, 
 Victóris et Rogatiáni.
\switchcolumn
\selectlanguage{english}
In Africa, the birthday of the holy martyrs Castor, Victor, and Rogatian.
\switchcolumn*
\selectlanguage{latin}
Nicomedíæ sanctórum Mártyrum Indis eunúchi, 
 Domnæ et Agapis ac Theóphilæ Vírginum, et Sociórum; qui, in persecutióne 
 Diocletiáni, post longa certámina, divérso mortis génere corónam martyrii 
 sunt assecúti.
\switchcolumn
\selectlanguage{english}
At Nicomedia, the holy martyrs Indes, a eunuch, Domna, Agapes, and Theophila, 
 virgins, and their companions, who, after long trials, attained to the crown 
 of martyrdom by various kinds of death, during the persecution of 
 Diocletian.
\switchcolumn*
\selectlanguage{latin}
Neocæsaréæ, in Ponto, sancti Troádii Mártyris, 
 in persecutióne Décii; cui quidem Troádio agonizánti sanctus Gregórius 
 Thaumatúrgus in spíritu ádfuit, eúmque ad subeúndum martyrium 
 roborávit.
\switchcolumn
\selectlanguage{english}
At Neocaesarea in Pontus, St. Troadius, martyr, in the persecution of Decius. 
 During his trial St. Gregory Thaumaturgus appeared to him in spirit and 
 encouraged him to undergo martyrdom.
\switchcolumn*
\selectlanguage{latin}
Arabíssi, in Arménia inferióre, sancti Cæsárii 
 Mártyris, qui sub Galério Maximiáno passus est.
\switchcolumn
\selectlanguage{english}
At Arabissus in Lower Armenia, St. Caesarius, martyr, who suffered under 
 Galerius Maximian.
\switchcolumn*
\selectlanguage{latin}
Romæ sancti Domniónis Presbyteri.
\switchcolumn
\selectlanguage{english}
At Rome, St. Domnio, priest.
\switchcolumn*
\selectlanguage{latin}
In monastério Lirinénsi, in Gállia, sancti Antónii Mónachi, miráculis clari.
\switchcolumn
\selectlanguage{english}
In the monastery of Lerins in France, St. Anthony, a monk famed for his 
 miracles.
\switchcolumn*
\selectlanguage{latin}
\end{paracol}


% ---- martyrology/mart12/mart1229.htm
\needspace{10\baselineskip}
\begin{paracol}{2}
\selectlanguage{latin}
\begin{center}{\color{gregoriocolor} Quarto Kaléndas Januárii. 
 Luna\dots\ }\end{center}
\switchcolumn
\selectlanguage{english}
\begin{center}{\color{gregoriocolor} The 
 Twenty-Ninth Day of 
 December. The\dots\ Day of the Moon.}\end{center}
\end{paracol}

\noindent\begin{tabularx}{\linewidth}{*{19}{>{\centering\arraybackslash}X}}
 \textcolor{gregoriocolor}{a} & \textcolor{gregoriocolor}{b} & \textcolor{gregoriocolor}{c} & \textcolor{gregoriocolor}{d} & \textcolor{gregoriocolor}{e} & \textcolor{gregoriocolor}{f} & \textcolor{gregoriocolor}{g} & \textcolor{gregoriocolor}{h} & \textcolor{gregoriocolor}{i} & \textcolor{gregoriocolor}{k} & \textcolor{gregoriocolor}{l} & \textcolor{gregoriocolor}{m} & \textcolor{gregoriocolor}{n} & \textcolor{gregoriocolor}{p} & \textcolor{gregoriocolor}{q} & \textcolor{gregoriocolor}{r} & \textcolor{gregoriocolor}{s} & \textcolor{gregoriocolor}{t} & \textcolor{gregoriocolor}{u} \\
 10 & 11 & 12 & 13 & 14 & 15 & 16 & 17 & 18 & 19 & 20 & 21 & 22 & 23 & 24 & 25 & 26 & 27 & 28 \\
\end{tabularx}
\vspace{0.5\baselineskip}
\noindent\begin{tabularx}{\linewidth}{*{12}{>{\centering\arraybackslash}X}}
 \textcolor{gregoriocolor}{A} & \textcolor{gregoriocolor}{B} & \textcolor{gregoriocolor}{C} & \textcolor{gregoriocolor}{D} & \textcolor{gregoriocolor}{E} & F & \textcolor{gregoriocolor}{F} & \textcolor{gregoriocolor}{G} & \textcolor{gregoriocolor}{H} & \textcolor{gregoriocolor}{M} & \textcolor{gregoriocolor}{N} & \textcolor{gregoriocolor}{P} \\
 29 & 30 & 1 & 2 & 3 & 4 & 4 & 5 & 6 & 7 & 8 & 9 \\
\end{tabularx}

\begin{paracol}{2}
\selectlanguage{latin}
\lettrine[lines=2]{C}{antuáriæ,} in Anglia, natális sancti Thomæ, 
 Epíscopi et Mártyris, qui, ob defensiónem justítiæ et ecclesiásticæ 
 immunitátis, in Basílica sua, ab impiórum hóminum factióne percússus gládio, 
 Martyr migrávit ad Christum.
\switchcolumn
\selectlanguage{english}
\lettrine[lines=2]{A}{t} Canterbury in England, the birthday of St. Thomas, bishop and martyr, 
 who, for the defence of justice and ecclesiastical immunity, was struck with 
 the sword in his own basilica by a faction of wicked men, and thus went to 
 Christ as martyr.
\switchcolumn*
\selectlanguage{latin}
Hierosólymis sancti David, Regis et Prophétæ.
\switchcolumn
\selectlanguage{english}
At Jerusalem, holy David, king and prophet.
\switchcolumn*
\selectlanguage{latin}
Areláte, in Gállia, natális sancti Tróphimi, 
 cujus méminit sanctus Paulus ad Timótheum scribens. Ipse autem 
 Tróphimus, ab eódem Apóstolo Epíscopus ordinátus, præfátæ urbi primus ad 
 Christi Evangélium prædicándum diréctus est; ex cujus prædicatiónis fonte (ut 
 sanctus Zósimus Papa scribit) tota Gállia rívulos fídei recépit.
\switchcolumn
\selectlanguage{english}
At Arles in France, the birthday of St. Trophimus, mentioned by St. Paul in 
 his Epistle to Timothy. Being ordained bishop by that apostle, he was 
 the first sent to preach the gospel of Christ in that city. From his 
 preaching, as from a fountain, according to the expression of Pope St. 
 Zosimus, all France received the waters of salvation.
\switchcolumn*
\selectlanguage{latin}
Romæ sanctórum Mártyrum Callísti, Felícis et 
 Bonifátii.
\switchcolumn
\selectlanguage{english}
At Rome, the holy martyrs Callistus, Felix, and Boniface.
\switchcolumn*
\selectlanguage{latin}
In Africa pássio sanctórum Mártyrum Domínici, Victóris, Primiáni, Lybósi, 
 Saturníni, Crescéntii, Secúndi et Honoráti.
\switchcolumn
\selectlanguage{english}
In Africa, the passion of the holy martyrs Dominic, Victor, Primian, Lybosus, 
 Saturninus, Crescentius, Secundus, and Honoratus.
\switchcolumn*
\selectlanguage{latin}
Constantinópoli sancti Marcélli Abbátis.
\switchcolumn
\selectlanguage{english}
At Constantinople, St. Marcellus, abbot.
\switchcolumn*
\selectlanguage{latin}
In pago Oxyménsi, in Gállia, sancti Ebrúlphi, Abbátis et Confessóris, 
 témpore Childebérti Regis.
\switchcolumn
\selectlanguage{english}
In the country of Hiesmes in France, St. Ebrulf, abbot and confessor, in the 
 time of King Childebert.
\switchcolumn*
\selectlanguage{latin}
Viénnæ, in Gállia, Commemorátio sancti 
 Crescéntis, Epíscopi et Mártyris, qui fuit discípulus beáti Pauli Apóstoli 
 ac primus ejúsdem civitátis Epíscopus, et cujus dies natális quinto Kaléndas 
 Júlii celebrátur.
\switchcolumn
\selectlanguage{english}
At Vienne in France, the commemoration of St. Crescens, bishop and martyr. 
 He was a disciple of St. Paul the Apostle and was the first bishop of that 
 city. His birthday is mentioned on the 27th of June.
\switchcolumn*
\selectlanguage{latin}
\end{paracol}


% ---- martyrology/mart12/mart1230.htm
\needspace{10\baselineskip}
\begin{paracol}{2}
\selectlanguage{latin}
\begin{center}{\color{gregoriocolor} Tértio Kaléndas Januárii. 
 Luna\dots\ }\end{center}
\switchcolumn
\selectlanguage{english}
\begin{center}{\color{gregoriocolor} The 
 Thirtieth Day of 
 December. The\dots\ Day of the Moon.}\end{center}
\end{paracol}

\noindent\begin{tabularx}{\linewidth}{*{19}{>{\centering\arraybackslash}X}}
 \textcolor{gregoriocolor}{a} & \textcolor{gregoriocolor}{b} & \textcolor{gregoriocolor}{c} & \textcolor{gregoriocolor}{d} & \textcolor{gregoriocolor}{e} & \textcolor{gregoriocolor}{f} & \textcolor{gregoriocolor}{g} & \textcolor{gregoriocolor}{h} & \textcolor{gregoriocolor}{i} & \textcolor{gregoriocolor}{k} & \textcolor{gregoriocolor}{l} & \textcolor{gregoriocolor}{m} & \textcolor{gregoriocolor}{n} & \textcolor{gregoriocolor}{p} & \textcolor{gregoriocolor}{q} & \textcolor{gregoriocolor}{r} & \textcolor{gregoriocolor}{s} & \textcolor{gregoriocolor}{t} & \textcolor{gregoriocolor}{u} \\
 11 & 12 & 13 & 14 & 15 & 16 & 17 & 18 & 19 & 20 & 21 & 22 & 23 & 24 & 25 & 26 & 27 & 28 & 29 \\
\end{tabularx}
\vspace{0.5\baselineskip}
\noindent\begin{tabularx}{\linewidth}{*{12}{>{\centering\arraybackslash}X}}
 \textcolor{gregoriocolor}{A} & \textcolor{gregoriocolor}{B} & \textcolor{gregoriocolor}{C} & \textcolor{gregoriocolor}{D} & \textcolor{gregoriocolor}{E} & F & \textcolor{gregoriocolor}{F} & \textcolor{gregoriocolor}{G} & \textcolor{gregoriocolor}{H} & \textcolor{gregoriocolor}{M} & \textcolor{gregoriocolor}{N} & \textcolor{gregoriocolor}{P} \\
 30 & 1 & 2 & 3 & 4 & 5 & 5 & 6 & 7 & 8 & 9 & 10 \\
\end{tabularx}

\begin{paracol}{2}
\selectlanguage{latin}
\lettrine[lines=2]{R}{omæ} natális sancti Felícis Primi, Papæ et 
 Mártyris, qui sub Aureliáno Príncipe Ecclésiam rexit. Ipsíus tamen 
 festum tértio Kaléndas Júnii celebrátur.
\switchcolumn
\selectlanguage{english}
\lettrine[lines=2]{A}{t} Rome, the birthday of St. Felix I, pope and martyr, who governed the 
 Church during the reign of Emperor Aurelian. His feast day is 
 celebrated on the 30th of May.
\switchcolumn*
\selectlanguage{latin}
Spoléti item natális sanctórum Mártyrum Sabíni, Assisiénsis Epíscopi, atque 
 Exsuperántii et Marcélli Diaconórum, ac Venustiáni Præsidis 
 cum uxóre et fíliis, sub Maximiáno Imperatóre. Ex ipsis Marcéllus et Exsuperántius, primum equúleo suspénsi, deínde fústibus gráviter mactáti, 
 postrémum, abrási úngulis et láterum exustióne assáti, martyrium 
 complevérunt; Venustiánus autem non multo post, simul cum uxóre et fíliis, 
 est gládio necátus; sanctus vero Sabínus, post detruncatiónem mánuum et 
 diútinam cárceris maceratiónem, ad mortem usque cæsus est. Horum 
 martyrium, licet divérso exstíterit témpore, una tamen die recólitur.
\switchcolumn
\selectlanguage{english}
At Spoleto, the birthday also of the holy martyrs Sabinus, bishop, 
 Exuperantius and Marcellus, deacons, and also Venustian, governor, along 
 with his wife and sons, under Emperor Maximian. Marcellus and 
 Exuperantius were first racked, then severely beaten with rods; afterwards 
 being torn with iron hooks, and burned in the sides, they fulfilled their 
 martyrdom. Not long after, Venustian was put to the sword with his 
 wife and sons. St. Sabinus, after having his hands cut off, and being 
 a long time confined in prison, was scourged to death. The martyrdom 
 of these saints is commemorated on the same day, although it occurred at 
 different times.
\switchcolumn*
\selectlanguage{latin}
Alexandríæ sanctórum Mansuéti, Sevéri, Appiáni, 
 Donáti, Honórii et Sociórum Mártyrum.
\switchcolumn
\selectlanguage{english}
At Alexandria, the Saints Mansuetus, Severus, Appian, Donatus, Honorius, and 
 their martyr companions.
\switchcolumn*
\selectlanguage{latin}
Thessalonícæ sanctæ Anysiæ Mártyris.
\switchcolumn
\selectlanguage{english}
At Thessalonica, St. Anysia, martyr.
\switchcolumn*
\selectlanguage{latin}
Ibídem sancti Anysii, ejúsdem civitátis Epíscopi.
\switchcolumn
\selectlanguage{english}
Likewise, St. Anysius, bishop of the same city.
\switchcolumn*
\selectlanguage{latin}
Medioláni sancti Eugénii, Epíscopi et Confessóris.
\switchcolumn
\selectlanguage{english}
At Milan, St. Eugene, bishop and confessor.
\switchcolumn*
\selectlanguage{latin}
Ravénnæ sancti Libérii Epíscopi.
\switchcolumn
\selectlanguage{english}
At Ravenna, St. Liberius, bishop.
\switchcolumn*
\selectlanguage{latin}
Aquilæ, in Vestínis, sancti Rainérii Epíscopi.
\switchcolumn
\selectlanguage{english}
At Aquila, in Abruzzi, St. Rainer, bishop.
\switchcolumn*
\selectlanguage{latin}
\end{paracol}


% ---- martyrology/mart12/mart1231.htm
\needspace{10\baselineskip}
\begin{paracol}{2}
\selectlanguage{latin}
\begin{center}{\color{gregoriocolor} Prídie Kaléndas Januárii. 
 Luna\dots\ }\end{center}
\switchcolumn
\selectlanguage{english}
\begin{center}{\color{gregoriocolor} The 
 Thirty-First Day of 
 December. The\dots\ Day of the Moon.}\end{center}
\end{paracol}

\noindent\begin{tabularx}{\linewidth}{*{19}{>{\centering\arraybackslash}X}}
 \textcolor{gregoriocolor}{a} & \textcolor{gregoriocolor}{b} & \textcolor{gregoriocolor}{c} & \textcolor{gregoriocolor}{d} & \textcolor{gregoriocolor}{e} & \textcolor{gregoriocolor}{f} & \textcolor{gregoriocolor}{g} & \textcolor{gregoriocolor}{h} & \textcolor{gregoriocolor}{i} & \textcolor{gregoriocolor}{k} & \textcolor{gregoriocolor}{l} & \textcolor{gregoriocolor}{m} & \textcolor{gregoriocolor}{n} & \textcolor{gregoriocolor}{p} & \textcolor{gregoriocolor}{q} & \textcolor{gregoriocolor}{r} & \textcolor{gregoriocolor}{s} & \textcolor{gregoriocolor}{t} & \textcolor{gregoriocolor}{u} \\
 12 & 13 & 14 & 15 & 16 & 17 & 18 & 19 & 20 & 21 & 22 & 23 & 24 & 25 & 26 & 27 & 28 & 29 & 30 \\
\end{tabularx}
\vspace{0.5\baselineskip}
\noindent\begin{tabularx}{\linewidth}{*{12}{>{\centering\arraybackslash}X}}
 \textcolor{gregoriocolor}{A} & \textcolor{gregoriocolor}{B} & \textcolor{gregoriocolor}{C} & \textcolor{gregoriocolor}{D} & \textcolor{gregoriocolor}{E} & F & \textcolor{gregoriocolor}{F} & \textcolor{gregoriocolor}{G} & \textcolor{gregoriocolor}{H} & \textcolor{gregoriocolor}{M} & \textcolor{gregoriocolor}{N} & \textcolor{gregoriocolor}{P} \\
 1 & 2 & 3 & 4 & 5 & 6 & 6 & 7 & 8 & 9 & 10 & 11 \\
\end{tabularx}

\begin{paracol}{2}
\selectlanguage{latin}
\lettrine[lines=2]{R}{omæ} natális sancti Silvéstri Primi, Papæ et 
 Confessóris; qui Magnum Constantínum Imperatórem baptizávit, et Nicænam 
 confirmávit Synodum, ac, multis áliis rebus sanctíssime gestis, quiévit in 
 pace.
\switchcolumn
\selectlanguage{english}
\lettrine[lines=2]{A}{t} Rome, the birthday of Pope St. Sylvester I, confessor, who baptized 
 Emperor Constantine the Great, and confirmed the council of Nicaea. 
 After performing many other holy deeds, he rested in peace.
\switchcolumn*
\selectlanguage{latin}
Item Romæ, via Salária, in cœmetério Priscíllæ, 
 sanctárum Mártyrum Donátæ, Paulínæ, Rústicæ, Nominándæ, Serótinæ, Hiláriæ et 
 Sociárum.
\switchcolumn
\selectlanguage{english}
At Rome, on the Salarian Way, in the cemetery of Priscilla, the holy martyrs 
 Donata, Paulina, Rustica, Nominanda, Serotina, Hilaria, and their 
 companions.
\switchcolumn*
\selectlanguage{latin}
Apud Sénonas beatórum Sabiniáni Epíscopi, et 
 Potentiáni; qui, a Pontífice Románo illuc ad prædicándum dirécti, 
 confessiónis suæ martyrio eándem metrópolim illustrárunt.
\switchcolumn
\selectlanguage{english}
At Sens, the blessed Sabinian, bishop, and Potentian. They had been 
 sent there to preach by the Roman Pontiff, and that metropolitan church was 
 illustrated by their confession and martyrdom.
\switchcolumn*
\selectlanguage{latin}
Cátanæ, in Sicília, pássio sanctórum Stéphani, 
 Pontiáni, Attali, Fabiáni, Cornélii, Sexti, Floris, Quinctiáni, Minervíni et 
 Simpliciáni.
\switchcolumn
\selectlanguage{english}
At Catania in Sicily, the passion of the Saints Stephen, Pontian, Attalus, 
 Fabian, Cornelius, Sextus, Flos, Quinctian, Minervinus, and Simplician.
\switchcolumn*
\selectlanguage{latin}
Apud Sénonas sanctæ Colúmbæ, Vírginis et 
 Mártyris; quæ, igne superáto, in persecutióne Aureliáni Imperatóris, gládio 
 cæsa est.
\switchcolumn
\selectlanguage{english}
At Sens, St. Columba, virgin and martyr, who, after having triumphed over 
 fire, was beheaded during the persecution of Emperor Aurelian.
\switchcolumn*
\selectlanguage{latin}
Eódem die sancti Zótici, Presbyteri Románi; 
 qui, Constantinópolim proféctus, illic alendórum orphanórum curam suscépit.
\switchcolumn
\selectlanguage{english}
On the same day, St. Zoticus, a Roman priest who went to Constantinople and 
 undertook the work of caring for orphans.
\switchcolumn*
\selectlanguage{latin}
Ravénnæ sancti Barbatiáni Presbyteri et 
 Confessóris.
\switchcolumn
\selectlanguage{english}
At Ravenna, St. Barbatian, priest and confessor.
\switchcolumn*
\selectlanguage{latin}
In pago Lalovésci, diœcésis Viennénsis, in 
 Delphinátu, deposítio sancti Joánnis-Francísci Regis, Sacerdótis e Societáte 
 Jesu et Confessóris, exímiæ in salúte animárum procuránda.
\switchcolumn
\selectlanguage{english}
At La Louvesc, in the diocese of Vienne in Dauphine, the death of St. John 
 Francis Regis, priest of the Society of Jesus and confessor. He was a 
 man of great love and patience in securing the salvation of souls.
\switchcolumn*
\selectlanguage{latin}
\end{paracol}


\end{document}
